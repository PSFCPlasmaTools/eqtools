%% Generated by Sphinx.
\def\sphinxdocclass{report}
\documentclass[letterpaper,10pt,english]{sphinxmanual}
\ifdefined\pdfpxdimen
   \let\sphinxpxdimen\pdfpxdimen\else\newdimen\sphinxpxdimen
\fi \sphinxpxdimen=.75bp\relax

\PassOptionsToPackage{warn}{textcomp}
\usepackage[utf8]{inputenc}
\ifdefined\DeclareUnicodeCharacter
% support both utf8 and utf8x syntaxes
  \ifdefined\DeclareUnicodeCharacterAsOptional
    \def\sphinxDUC#1{\DeclareUnicodeCharacter{"#1}}
  \else
    \let\sphinxDUC\DeclareUnicodeCharacter
  \fi
  \sphinxDUC{00A0}{\nobreakspace}
  \sphinxDUC{2500}{\sphinxunichar{2500}}
  \sphinxDUC{2502}{\sphinxunichar{2502}}
  \sphinxDUC{2514}{\sphinxunichar{2514}}
  \sphinxDUC{251C}{\sphinxunichar{251C}}
  \sphinxDUC{2572}{\textbackslash}
\fi
\usepackage{cmap}
\usepackage[T1]{fontenc}
\usepackage{amsmath,amssymb,amstext}
\usepackage{babel}



\usepackage{times}
\expandafter\ifx\csname T@LGR\endcsname\relax
\else
% LGR was declared as font encoding
  \substitutefont{LGR}{\rmdefault}{cmr}
  \substitutefont{LGR}{\sfdefault}{cmss}
  \substitutefont{LGR}{\ttdefault}{cmtt}
\fi
\expandafter\ifx\csname T@X2\endcsname\relax
  \expandafter\ifx\csname T@T2A\endcsname\relax
  \else
  % T2A was declared as font encoding
    \substitutefont{T2A}{\rmdefault}{cmr}
    \substitutefont{T2A}{\sfdefault}{cmss}
    \substitutefont{T2A}{\ttdefault}{cmtt}
  \fi
\else
% X2 was declared as font encoding
  \substitutefont{X2}{\rmdefault}{cmr}
  \substitutefont{X2}{\sfdefault}{cmss}
  \substitutefont{X2}{\ttdefault}{cmtt}
\fi


\usepackage[Bjarne]{fncychap}
\usepackage{sphinx}

\fvset{fontsize=\small}
\usepackage{geometry}

% Include hyperref last.
\usepackage{hyperref}
% Fix anchor placement for figures with captions.
\usepackage{hypcap}% it must be loaded after hyperref.
% Set up styles of URL: it should be placed after hyperref.
\urlstyle{same}

\usepackage{sphinxmessages}
\setcounter{tocdepth}{3}
\setcounter{secnumdepth}{3}
\hypersetup{bookmarksdepth=3}

\title{eqtools Documentation}
\date{Sep 01, 2020}
\release{1.3.2}
\author{Mark Chilenski, Ian Faust and John Walk}
\newcommand{\sphinxlogo}{\vbox{}}
\renewcommand{\releasename}{Release}
\makeindex
\begin{document}

\pagestyle{empty}
\sphinxmaketitle
\pagestyle{plain}
\sphinxtableofcontents
\pagestyle{normal}
\phantomsection\label{\detokenize{index::doc}}


Homepage: \sphinxurl{https://github.com/PSFCPlasmaTools/eqtools}


\chapter{Overview}
\label{\detokenize{index:overview}}
{\hyperref[\detokenize{eqtools:module-eqtools}]{\sphinxcrossref{\sphinxcode{\sphinxupquote{eqtools}}}}} is a Python package for working with magnetic equilibrium reconstructions from magnetic plasma confinement devices. At present, interfaces exist for data from the Alcator C-Mod and NSTX MDSplus trees as well as eqdsk a- and g-files. {\hyperref[\detokenize{eqtools:module-eqtools}]{\sphinxcrossref{\sphinxcode{\sphinxupquote{eqtools}}}}} is designed to be flexible and extensible such that it can become a uniform interface to perform mapping operations and accessing equilibrium data for any magnetic confinement device, regardless of how the data are accessed.

The main class of {\hyperref[\detokenize{eqtools:module-eqtools}]{\sphinxcrossref{\sphinxcode{\sphinxupquote{eqtools}}}}} is the {\hyperref[\detokenize{eqtools:eqtools.core.Equilibrium}]{\sphinxcrossref{\sphinxcode{\sphinxupquote{Equilibrium}}}}}, which contains all of the coordinate mapping functions as well as templates for methods to fetch data (primarily dictated to the quantities computed by EFIT). Subclasses such as {\hyperref[\detokenize{eqtools:eqtools.EFIT.EFITTree}]{\sphinxcrossref{\sphinxcode{\sphinxupquote{EFITTree}}}}}, {\hyperref[\detokenize{eqtools:eqtools.CModEFIT.CModEFITTree}]{\sphinxcrossref{\sphinxcode{\sphinxupquote{CModEFITTree}}}}}, {\hyperref[\detokenize{eqtools:eqtools.NSTXEFIT.NSTXEFITTree}]{\sphinxcrossref{\sphinxcode{\sphinxupquote{NSTXEFITTree}}}}} and {\hyperref[\detokenize{eqtools:eqtools.eqdskreader.EqdskReader}]{\sphinxcrossref{\sphinxcode{\sphinxupquote{EqdskReader}}}}} implement specific methods to access the data and convert it to the form needed for the routines in {\hyperref[\detokenize{eqtools:eqtools.core.Equilibrium}]{\sphinxcrossref{\sphinxcode{\sphinxupquote{Equilibrium}}}}}. These classes are smart about caching intermediate results, so you will get a performance boost by using the same instance throughout your analysis of a given shot.


\chapter{Installation}
\label{\detokenize{index:installation}}
The easiest way to install the latest release version is with \sphinxtitleref{pip}:

\begin{sphinxVerbatim}[commandchars=\\\{\}]
\PYG{n}{pip} \PYG{n}{install} \PYG{n}{eqtools}
\end{sphinxVerbatim}

To install from source, uncompress the source files and, from the directory containing \sphinxtitleref{setup.py}, run the following command:

\begin{sphinxVerbatim}[commandchars=\\\{\}]
\PYG{n}{python} \PYG{n}{setup}\PYG{o}{.}\PYG{n}{py} \PYG{n}{install}
\end{sphinxVerbatim}

Or, to build in place, run:

\begin{sphinxVerbatim}[commandchars=\\\{\}]
\PYG{n}{python} \PYG{n}{setup}\PYG{o}{.}\PYG{n}{py} \PYG{n}{build\PYGZus{}ext} \PYG{o}{\PYGZhy{}}\PYG{o}{\PYGZhy{}}\PYG{n}{inplace}
\end{sphinxVerbatim}


\chapter{Tutorial: Performing Coordinate Transforms on Alcator C-Mod Data}
\label{\detokenize{index:tutorial-performing-coordinate-transforms-on-alcator-c-mod-data}}
The basic class for manipulating EFIT results stored in the Alcator C-Mod MDSplus tree is {\hyperref[\detokenize{eqtools:eqtools.CModEFIT.CModEFITTree}]{\sphinxcrossref{\sphinxcode{\sphinxupquote{CModEFITTree}}}}}. To load the data from a specific shot, simply create the {\hyperref[\detokenize{eqtools:eqtools.CModEFIT.CModEFITTree}]{\sphinxcrossref{\sphinxcode{\sphinxupquote{CModEFITTree}}}}} object with the shot number as the argument:

\begin{sphinxVerbatim}[commandchars=\\\{\}]
\PYG{n}{e} \PYG{o}{=} \PYG{n}{eqtools}\PYG{o}{.}\PYG{n}{CModEFITTree}\PYG{p}{(}\PYG{l+m+mi}{1140729030}\PYG{p}{)}
\end{sphinxVerbatim}

The default EFIT to use is “ANALYSIS.” If you want to use a different tree, such as “EFIT20,” then you simply set this with the \sphinxtitleref{tree} keyword:

\begin{sphinxVerbatim}[commandchars=\\\{\}]
\PYG{n}{e} \PYG{o}{=} \PYG{n}{eqtools}\PYG{o}{.}\PYG{n}{CModEFITTree}\PYG{p}{(}\PYG{l+m+mi}{1140729030}\PYG{p}{,} \PYG{n}{tree}\PYG{o}{=}\PYG{l+s+s1}{\PYGZsq{}}\PYG{l+s+s1}{EFIT20}\PYG{l+s+s1}{\PYGZsq{}}\PYG{p}{)}
\end{sphinxVerbatim}

{\hyperref[\detokenize{eqtools:module-eqtools}]{\sphinxcrossref{\sphinxcode{\sphinxupquote{eqtools}}}}} understands units. The default is to convert all lengths to meters (whereas quantities in the tree are inconsistent \textendash{} some are meters, some centimeters). If you want to specify a different default unit, use the \sphinxtitleref{length\_unit} keyword:

\begin{sphinxVerbatim}[commandchars=\\\{\}]
\PYG{n}{e} \PYG{o}{=} \PYG{n}{eqtools}\PYG{o}{.}\PYG{n}{CModEFITTree}\PYG{p}{(}\PYG{l+m+mi}{1140729030}\PYG{p}{,} \PYG{n}{length\PYGZus{}unit}\PYG{o}{=}\PYG{l+s+s1}{\PYGZsq{}}\PYG{l+s+s1}{cm}\PYG{l+s+s1}{\PYGZsq{}}\PYG{p}{)}
\end{sphinxVerbatim}

Once this is loaded, you can access the data you would normally have to pull from specific nodes in the tree using convenient getter methods. For instance, to get the elongation as a function of time, you can run:

\begin{sphinxVerbatim}[commandchars=\\\{\}]
\PYG{n}{kappa} \PYG{o}{=} \PYG{n}{e}\PYG{o}{.}\PYG{n}{getElongation}\PYG{p}{(}\PYG{p}{)}
\end{sphinxVerbatim}

The timebase used for quantities like this is accessed with:

\begin{sphinxVerbatim}[commandchars=\\\{\}]
\PYG{n}{t} \PYG{o}{=} \PYG{n}{e}\PYG{o}{.}\PYG{n}{getTimeBase}\PYG{p}{(}\PYG{p}{)}
\end{sphinxVerbatim}

For length/area/volume quantities, {\hyperref[\detokenize{eqtools:module-eqtools}]{\sphinxcrossref{\sphinxcode{\sphinxupquote{eqtools}}}}} understands units. The default is to return in whatever units you specified when creating the {\hyperref[\detokenize{eqtools:eqtools.CModEFIT.CModEFITTree}]{\sphinxcrossref{\sphinxcode{\sphinxupquote{CModEFITTree}}}}}, but you can override this with the \sphinxtitleref{length\_unit} keyword. For instance, to get the vertical position of the magnetic axis in mm, you can run:

\begin{sphinxVerbatim}[commandchars=\\\{\}]
\PYG{n}{Z\PYGZus{}mag} \PYG{o}{=} \PYG{n}{e}\PYG{o}{.}\PYG{n}{getMagZ}\PYG{p}{(}\PYG{n}{length\PYGZus{}unit}\PYG{o}{=}\PYG{l+s+s1}{\PYGZsq{}}\PYG{l+s+s1}{mm}\PYG{l+s+s1}{\PYGZsq{}}\PYG{p}{)}
\end{sphinxVerbatim}

{\hyperref[\detokenize{eqtools:module-eqtools}]{\sphinxcrossref{\sphinxcode{\sphinxupquote{eqtools}}}}} can map from almost any coordinate to any common flux surface label. For instance, say you want to know what the square root of normalized toroidal flux corresponding to a normalized flux surface volume of 0.5 is at t=1.0s. You can simply call:

\begin{sphinxVerbatim}[commandchars=\\\{\}]
\PYG{n}{rho} \PYG{o}{=} \PYG{n}{e}\PYG{o}{.}\PYG{n}{volnorm2phinorm}\PYG{p}{(}\PYG{l+m+mf}{0.5}\PYG{p}{,} \PYG{l+m+mf}{1.0}\PYG{p}{,} \PYG{n}{sqrt}\PYG{o}{=}\PYG{k+kc}{True}\PYG{p}{)}
\end{sphinxVerbatim}

If a list of times is provided, the default behavior is to evaluate all of the points to be converted at each of the times. So, to follow the mapping of normalized poloidal flux values {[}0.1, 0.5, 1.0{]} to outboard midplane major radius at time points {[}1.0, 1.25, 1.5, 1.75{]}, you could call:

\begin{sphinxVerbatim}[commandchars=\\\{\}]
\PYG{n}{psinorm} \PYG{o}{=} \PYG{n}{e}\PYG{o}{.}\PYG{n}{psinorm2rmid}\PYG{p}{(}\PYG{p}{[}\PYG{l+m+mf}{0.1}\PYG{p}{,} \PYG{l+m+mf}{0.5}\PYG{p}{,} \PYG{l+m+mf}{1.0}\PYG{p}{]}\PYG{p}{,} \PYG{p}{[}\PYG{l+m+mf}{1.0}\PYG{p}{,} \PYG{l+m+mf}{1.25}\PYG{p}{,} \PYG{l+m+mf}{1.5}\PYG{p}{,} \PYG{l+m+mf}{1.75}\PYG{p}{]}\PYG{p}{)}
\end{sphinxVerbatim}

This will return a 4-by-3 array: one row for each time, one column for each location. If you want to override this behavior and instead consider a sequence of (psi, t) points, set the \sphinxtitleref{each\_t} keyword to False:

\begin{sphinxVerbatim}[commandchars=\\\{\}]
\PYG{n}{psinorm} \PYG{o}{=} \PYG{n}{e}\PYG{o}{.}\PYG{n}{psinorm2rmid}\PYG{p}{(}\PYG{p}{[}\PYG{l+m+mf}{0.3}\PYG{p}{,} \PYG{l+m+mf}{0.35}\PYG{p}{]}\PYG{p}{,} \PYG{p}{[}\PYG{l+m+mf}{1.0}\PYG{p}{,} \PYG{l+m+mf}{1.1}\PYG{p}{]}\PYG{p}{,} \PYG{n}{each\PYGZus{}t}\PYG{o}{=}\PYG{k+kc}{False}\PYG{p}{)}
\end{sphinxVerbatim}

This will return a two-element array with the Rmid values for (psinorm=0.3, t=1.0) and (psinorm=0.35, t=1.1).

For programmatically mapping between coordinates, the {\hyperref[\detokenize{eqtools:eqtools.core.Equilibrium.rho2rho}]{\sphinxcrossref{\sphinxcode{\sphinxupquote{rho2rho()}}}}} method is quite useful. To map from outboard midplane major radius to normalized flux surface volume, you can simply call:

\begin{sphinxVerbatim}[commandchars=\\\{\}]
\PYG{n}{e}\PYG{o}{.}\PYG{n}{rho2rho}\PYG{p}{(}\PYG{l+s+s1}{\PYGZsq{}}\PYG{l+s+s1}{Rmid}\PYG{l+s+s1}{\PYGZsq{}}\PYG{p}{,} \PYG{l+s+s1}{\PYGZsq{}}\PYG{l+s+s1}{volnorm}\PYG{l+s+s1}{\PYGZsq{}}\PYG{p}{,} \PYG{l+m+mf}{0.75}\PYG{p}{,} \PYG{l+m+mf}{1.0}\PYG{p}{)}
\end{sphinxVerbatim}

Finally, to get a look at the flux surfaces, simply run:

\begin{sphinxVerbatim}[commandchars=\\\{\}]
\PYG{n}{e}\PYG{o}{.}\PYG{n}{plotFlux}\PYG{p}{(}\PYG{p}{)}
\end{sphinxVerbatim}


\chapter{Package Reference}
\label{\detokenize{index:package-reference}}

\section{eqtools package}
\label{\detokenize{eqtools:eqtools-package}}\label{\detokenize{eqtools::doc}}

\subsection{Submodules}
\label{\detokenize{eqtools:submodules}}

\subsection{eqtools.AUGData module}
\label{\detokenize{eqtools:module-eqtools.AUGData}}\label{\detokenize{eqtools:eqtools-augdata-module}}\index{eqtools.AUGData (module)@\spxentry{eqtools.AUGData}\spxextra{module}}
This module provides classes inheriting \sphinxcode{\sphinxupquote{eqtools.Equilibrium}} for
working with ASDEX Upgrade experimental data.
\index{AUGDDData (class in eqtools.AUGData)@\spxentry{AUGDDData}\spxextra{class in eqtools.AUGData}}

\begin{fulllineitems}
\phantomsection\label{\detokenize{eqtools:eqtools.AUGData.AUGDDData}}\pysiglinewithargsret{\sphinxbfcode{\sphinxupquote{class }}\sphinxcode{\sphinxupquote{eqtools.AUGData.}}\sphinxbfcode{\sphinxupquote{AUGDDData}}}{\emph{shot}, \emph{shotfile='EQH'}, \emph{edition=0}, \emph{shotfile2=None}, \emph{length\_unit='m'}, \emph{tspline=False}, \emph{monotonic=True}, \emph{experiment='AUGD'}}{}
Bases: {\hyperref[\detokenize{eqtools:eqtools.core.Equilibrium}]{\sphinxcrossref{\sphinxcode{\sphinxupquote{eqtools.core.Equilibrium}}}}}

Inherits \sphinxcode{\sphinxupquote{eqtools.Equilibrium}} class. Machine-specific data
handling class for ASDEX Upgrade. Pulls AFS data from selected location
and shotfile, stores as object attributes. Each data variable or set of
variables is recovered with a corresponding getter method. Essential data
for mapping are pulled on initialization (e.g. psirz grid). Additional
data are pulled at the first request and stored for subsequent usage.

Intializes ASDEX Upgrade version of the Equilibrium object.  Pulls data to
storage in instance attributes.  Core attributes are populated from the AFS
data on initialization.  Additional attributes are initialized as None,
filled on the first request to the object.
\begin{quote}\begin{description}
\item[{Parameters}] \leavevmode
\sphinxstyleliteralstrong{\sphinxupquote{shot}} (\sphinxstyleliteralemphasis{\sphinxupquote{integer}}) \textendash{} ASDEX Upgrade shot index.

\item[{Keyword Arguments}] \leavevmode\begin{itemize}
\item {} 
\sphinxstyleliteralstrong{\sphinxupquote{shotfile}} (\sphinxstyleliteralemphasis{\sphinxupquote{string}}) \textendash{} Optional input for alternate shotfile, defaults to ‘EQH’
(i.e., CLISTE results are in EQH,EQI with other reconstructions
Available (FPP, EQE, ect.).

\item {} 
\sphinxstyleliteralstrong{\sphinxupquote{edition}} (\sphinxstyleliteralemphasis{\sphinxupquote{integer}}) \textendash{} Describes the edition of the shotfile to be used

\item {} 
\sphinxstyleliteralstrong{\sphinxupquote{shotfile2}} (\sphinxstyleliteralemphasis{\sphinxupquote{string}}) \textendash{} Describes companion 0D equilibrium data, will automatically
reference based off of shotfile, but can be manually specified for
unique reconstructions, etc.

\item {} 
\sphinxstyleliteralstrong{\sphinxupquote{length\_unit}} (\sphinxstyleliteralemphasis{\sphinxupquote{string}}) \textendash{} 
Sets the base unit used for any quantity whose
dimensions are length to any power. Valid options are:
\begin{quote}


\begin{savenotes}\sphinxattablestart
\centering
\begin{tabulary}{\linewidth}[t]{|T|T|}
\hline

’m’
&
meters
\\
\hline
’cm’
&
centimeters
\\
\hline
’mm’
&
millimeters
\\
\hline
’in’
&
inches
\\
\hline
’ft’
&
feet
\\
\hline
’yd’
&
yards
\\
\hline
’smoot’
&
smoots
\\
\hline
’cubit’
&
cubits
\\
\hline
’hand’
&
hands
\\
\hline
’default’
&
whatever the default in the tree is (no conversion is performed, units may be inconsistent)
\\
\hline
\end{tabulary}
\par
\sphinxattableend\end{savenotes}
\end{quote}

Default is ‘m’ (all units taken and returned in meters).


\item {} 
\sphinxstyleliteralstrong{\sphinxupquote{tspline}} (\sphinxstyleliteralemphasis{\sphinxupquote{Boolean}}) \textendash{} Sets whether or not interpolation in time is
performed using a tricubic spline or nearest-neighbor
interpolation. Tricubic spline interpolation requires at least
four complete equilibria at different times. It is also assumed
that they are functionally correlated, and that parameters do
not vary out of their boundaries (derivative = 0 boundary
condition). Default is False (use nearest neighbor interpolation).

\item {} 
\sphinxstyleliteralstrong{\sphinxupquote{monotonic}} (\sphinxstyleliteralemphasis{\sphinxupquote{Boolean}}) \textendash{} Sets whether or not the “monotonic” form of time
window finding is used. If True, the timebase must be monotonically
increasing. Default is False (use slower, safer method).

\item {} 
\sphinxstyleliteralstrong{\sphinxupquote{experiment}} \textendash{} Used to describe the work space that the shotfile is located
It defaults to ‘AUGD’ but can be set to other values

\end{itemize}

\end{description}\end{quote}
\index{getInfo() (eqtools.AUGData.AUGDDData method)@\spxentry{getInfo()}\spxextra{eqtools.AUGData.AUGDDData method}}

\begin{fulllineitems}
\phantomsection\label{\detokenize{eqtools:eqtools.AUGData.AUGDDData.getInfo}}\pysiglinewithargsret{\sphinxbfcode{\sphinxupquote{getInfo}}}{}{}
returns namedtuple of shot information
\begin{quote}\begin{description}
\item[{Returns}] \leavevmode

namedtuple containing
\begin{quote}


\begin{savenotes}\sphinxattablestart
\centering
\begin{tabulary}{\linewidth}[t]{|T|T|}
\hline

shot
&
ASDEX Upgrage shot index (long)
\\
\hline
tree
&
shotfile (string)
\\
\hline
nr
&
size of R-axis for spatial grid
\\
\hline
nz
&
size of Z-axis for spatial grid
\\
\hline
nt
&
size of timebase for flux grid
\\
\hline
\end{tabulary}
\par
\sphinxattableend\end{savenotes}
\end{quote}


\end{description}\end{quote}

\end{fulllineitems}

\index{getTimeBase() (eqtools.AUGData.AUGDDData method)@\spxentry{getTimeBase()}\spxextra{eqtools.AUGData.AUGDDData method}}

\begin{fulllineitems}
\phantomsection\label{\detokenize{eqtools:eqtools.AUGData.AUGDDData.getTimeBase}}\pysiglinewithargsret{\sphinxbfcode{\sphinxupquote{getTimeBase}}}{}{}
returns time base vector.
\begin{quote}\begin{description}
\item[{Returns}] \leavevmode
{[}nt{]} array of time points.

\item[{Return type}] \leavevmode
time (array)

\item[{Raises}] \leavevmode
\sphinxstyleliteralstrong{\sphinxupquote{ValueError}} \textendash{} if module cannot retrieve data from the AUG AFS system.

\end{description}\end{quote}

\end{fulllineitems}

\index{getFluxGrid() (eqtools.AUGData.AUGDDData method)@\spxentry{getFluxGrid()}\spxextra{eqtools.AUGData.AUGDDData method}}

\begin{fulllineitems}
\phantomsection\label{\detokenize{eqtools:eqtools.AUGData.AUGDDData.getFluxGrid}}\pysiglinewithargsret{\sphinxbfcode{\sphinxupquote{getFluxGrid}}}{}{}
returns flux grid.

Note that this method preserves whatever sign convention is used in AFS.
\begin{quote}\begin{description}
\item[{Returns}] \leavevmode
{[}nt,nz,nr{]} array of (non-normalized) flux on grid.

\item[{Return type}] \leavevmode
psiRZ (Array)

\item[{Raises}] \leavevmode
\sphinxstyleliteralstrong{\sphinxupquote{ValueError}} \textendash{} if module cannot retrieve data from the AUG AFS system.

\end{description}\end{quote}

\end{fulllineitems}

\index{getRGrid() (eqtools.AUGData.AUGDDData method)@\spxentry{getRGrid()}\spxextra{eqtools.AUGData.AUGDDData method}}

\begin{fulllineitems}
\phantomsection\label{\detokenize{eqtools:eqtools.AUGData.AUGDDData.getRGrid}}\pysiglinewithargsret{\sphinxbfcode{\sphinxupquote{getRGrid}}}{\emph{length\_unit=1}}{}
returns R-axis.
\begin{quote}\begin{description}
\item[{Returns}] \leavevmode
{[}nr{]} array of R-axis of flux grid.

\item[{Return type}] \leavevmode
rGrid (Array)

\item[{Raises}] \leavevmode
\sphinxstyleliteralstrong{\sphinxupquote{ValueError}} \textendash{} if module cannot retrieve data from the AUG AFS system.

\end{description}\end{quote}

\end{fulllineitems}

\index{getZGrid() (eqtools.AUGData.AUGDDData method)@\spxentry{getZGrid()}\spxextra{eqtools.AUGData.AUGDDData method}}

\begin{fulllineitems}
\phantomsection\label{\detokenize{eqtools:eqtools.AUGData.AUGDDData.getZGrid}}\pysiglinewithargsret{\sphinxbfcode{\sphinxupquote{getZGrid}}}{\emph{length\_unit=1}}{}
returns Z-axis.
\begin{quote}\begin{description}
\item[{Returns}] \leavevmode
{[}nz{]} array of Z-axis of flux grid.

\item[{Return type}] \leavevmode
zGrid (Array)

\item[{Raises}] \leavevmode
\sphinxstyleliteralstrong{\sphinxupquote{ValueError}} \textendash{} if module cannot retrieve data from the AUG AFS system.

\end{description}\end{quote}

\end{fulllineitems}

\index{getFluxAxis() (eqtools.AUGData.AUGDDData method)@\spxentry{getFluxAxis()}\spxextra{eqtools.AUGData.AUGDDData method}}

\begin{fulllineitems}
\phantomsection\label{\detokenize{eqtools:eqtools.AUGData.AUGDDData.getFluxAxis}}\pysiglinewithargsret{\sphinxbfcode{\sphinxupquote{getFluxAxis}}}{}{}
returns psi on magnetic axis.
\begin{quote}\begin{description}
\item[{Returns}] \leavevmode
{[}nt{]} array of psi on magnetic axis.

\item[{Return type}] \leavevmode
psiAxis (Array)

\item[{Raises}] \leavevmode
\sphinxstyleliteralstrong{\sphinxupquote{ValueError}} \textendash{} if module cannot retrieve data from the AUG AFS system.

\end{description}\end{quote}

\end{fulllineitems}

\index{getFluxLCFS() (eqtools.AUGData.AUGDDData method)@\spxentry{getFluxLCFS()}\spxextra{eqtools.AUGData.AUGDDData method}}

\begin{fulllineitems}
\phantomsection\label{\detokenize{eqtools:eqtools.AUGData.AUGDDData.getFluxLCFS}}\pysiglinewithargsret{\sphinxbfcode{\sphinxupquote{getFluxLCFS}}}{}{}
returns psi at separatrix.
\begin{quote}\begin{description}
\item[{Returns}] \leavevmode
{[}nt{]} array of psi at LCFS.

\item[{Return type}] \leavevmode
psiLCFS (Array)

\item[{Raises}] \leavevmode
\sphinxstyleliteralstrong{\sphinxupquote{ValueError}} \textendash{} if module cannot retrieve data from the AUG AFS system.

\end{description}\end{quote}

\end{fulllineitems}

\index{getFluxVol() (eqtools.AUGData.AUGDDData method)@\spxentry{getFluxVol()}\spxextra{eqtools.AUGData.AUGDDData method}}

\begin{fulllineitems}
\phantomsection\label{\detokenize{eqtools:eqtools.AUGData.AUGDDData.getFluxVol}}\pysiglinewithargsret{\sphinxbfcode{\sphinxupquote{getFluxVol}}}{\emph{length\_unit=3}}{}
returns volume within flux surface.
\begin{quote}\begin{description}
\item[{Keyword Arguments}] \leavevmode
\sphinxstyleliteralstrong{\sphinxupquote{length\_unit}} (\sphinxstyleliteralemphasis{\sphinxupquote{String}}\sphinxstyleliteralemphasis{\sphinxupquote{ or }}\sphinxstyleliteralemphasis{\sphinxupquote{3}}) \textendash{} unit for plasma volume.  Defaults to 3,
indicating default volumetric unit (typically m\textasciicircum{}3).

\item[{Returns}] \leavevmode
{[}nt,npsi{]} array of volume within flux surface.

\item[{Return type}] \leavevmode
fluxVol (Array)

\item[{Raises}] \leavevmode
\sphinxstyleliteralstrong{\sphinxupquote{ValueError}} \textendash{} if module cannot retrieve data from the AUG AFS system.

\end{description}\end{quote}

\end{fulllineitems}

\index{getVolLCFS() (eqtools.AUGData.AUGDDData method)@\spxentry{getVolLCFS()}\spxextra{eqtools.AUGData.AUGDDData method}}

\begin{fulllineitems}
\phantomsection\label{\detokenize{eqtools:eqtools.AUGData.AUGDDData.getVolLCFS}}\pysiglinewithargsret{\sphinxbfcode{\sphinxupquote{getVolLCFS}}}{\emph{length\_unit=3}}{}
returns volume within LCFS.
\begin{quote}\begin{description}
\item[{Keyword Arguments}] \leavevmode
\sphinxstyleliteralstrong{\sphinxupquote{length\_unit}} (\sphinxstyleliteralemphasis{\sphinxupquote{String}}\sphinxstyleliteralemphasis{\sphinxupquote{ or }}\sphinxstyleliteralemphasis{\sphinxupquote{3}}) \textendash{} unit for LCFS volume.  Defaults to 3,
denoting default volumetric unit (typically m\textasciicircum{}3).

\item[{Returns}] \leavevmode
{[}nt{]} array of volume within LCFS.

\item[{Return type}] \leavevmode
volLCFS (Array)

\item[{Raises}] \leavevmode
\sphinxstyleliteralstrong{\sphinxupquote{ValueError}} \textendash{} if module cannot retrieve data from the AUG AFS system.

\end{description}\end{quote}

\end{fulllineitems}

\index{getRmidPsi() (eqtools.AUGData.AUGDDData method)@\spxentry{getRmidPsi()}\spxextra{eqtools.AUGData.AUGDDData method}}

\begin{fulllineitems}
\phantomsection\label{\detokenize{eqtools:eqtools.AUGData.AUGDDData.getRmidPsi}}\pysiglinewithargsret{\sphinxbfcode{\sphinxupquote{getRmidPsi}}}{\emph{length\_unit=1}}{}
returns maximum major radius of each flux surface.
\begin{quote}\begin{description}
\item[{Keyword Arguments}] \leavevmode
\sphinxstyleliteralstrong{\sphinxupquote{length\_unit}} (\sphinxstyleliteralemphasis{\sphinxupquote{String}}\sphinxstyleliteralemphasis{\sphinxupquote{ or }}\sphinxstyleliteralemphasis{\sphinxupquote{1}}) \textendash{} unit of Rmid.  Defaults to 1, indicating
the default parameter unit (typically m).

\item[{Returns}] \leavevmode
{[}nt,npsi{]} array of maximum (outboard) major radius of
flux surface psi.

\item[{Return type}] \leavevmode
Rmid (Array)

\item[{Raises}] \leavevmode
\sphinxstyleliteralstrong{\sphinxupquote{NotImplementedError}} \textendash{} Not implemented on ASDEX-Upgrade reconstructions.

\end{description}\end{quote}

\end{fulllineitems}

\index{getRLCFS() (eqtools.AUGData.AUGDDData method)@\spxentry{getRLCFS()}\spxextra{eqtools.AUGData.AUGDDData method}}

\begin{fulllineitems}
\phantomsection\label{\detokenize{eqtools:eqtools.AUGData.AUGDDData.getRLCFS}}\pysiglinewithargsret{\sphinxbfcode{\sphinxupquote{getRLCFS}}}{\emph{length\_unit=1}}{}
returns R-values of LCFS position.
\begin{quote}\begin{description}
\item[{Returns}] \leavevmode
{[}nt,n{]} array of R of LCFS points.

\item[{Return type}] \leavevmode
RLCFS (Array)

\item[{Raises}] \leavevmode
\sphinxstyleliteralstrong{\sphinxupquote{ValueError}} \textendash{} if module cannot retrieve data from the AUG AFS system.

\end{description}\end{quote}

\end{fulllineitems}

\index{getZLCFS() (eqtools.AUGData.AUGDDData method)@\spxentry{getZLCFS()}\spxextra{eqtools.AUGData.AUGDDData method}}

\begin{fulllineitems}
\phantomsection\label{\detokenize{eqtools:eqtools.AUGData.AUGDDData.getZLCFS}}\pysiglinewithargsret{\sphinxbfcode{\sphinxupquote{getZLCFS}}}{\emph{length\_unit=1}}{}
returns Z-values of LCFS position.
\begin{quote}\begin{description}
\item[{Returns}] \leavevmode
{[}nt,n{]} array of Z of LCFS points.

\item[{Return type}] \leavevmode
ZLCFS (Array)

\item[{Raises}] \leavevmode
\sphinxstyleliteralstrong{\sphinxupquote{ValueError}} \textendash{} if module cannot retrieve data from the AUG AFS system.

\end{description}\end{quote}

\end{fulllineitems}

\index{remapLCFS() (eqtools.AUGData.AUGDDData method)@\spxentry{remapLCFS()}\spxextra{eqtools.AUGData.AUGDDData method}}

\begin{fulllineitems}
\phantomsection\label{\detokenize{eqtools:eqtools.AUGData.AUGDDData.remapLCFS}}\pysiglinewithargsret{\sphinxbfcode{\sphinxupquote{remapLCFS}}}{\emph{mask=False}}{}
Overwrites RLCFS, ZLCFS values pulled with explicitly-calculated
contour of psinorm=1 surface.  This is then masked down by the limiter
array using core.inPolygon, restricting the contour to the closed
plasma surface and the divertor legs.
\begin{quote}\begin{description}
\item[{Keyword Arguments}] \leavevmode
\sphinxstyleliteralstrong{\sphinxupquote{mask}} (\sphinxstyleliteralemphasis{\sphinxupquote{Boolean}}) \textendash{} Default False.  Set True to mask LCFS path to
limiter outline (using inPolygon).  Set False to draw full
contour of psi = psiLCFS.

\item[{Raises}] \leavevmode\begin{itemize}
\item {} 
\sphinxstyleliteralstrong{\sphinxupquote{NotImplementedError}} \textendash{} if \sphinxcode{\sphinxupquote{matplotlib.pyplot}} is not loaded.

\item {} 
\sphinxstyleliteralstrong{\sphinxupquote{ValueError}} \textendash{} if limiter outline is not available.

\end{itemize}

\end{description}\end{quote}

\end{fulllineitems}

\index{getF() (eqtools.AUGData.AUGDDData method)@\spxentry{getF()}\spxextra{eqtools.AUGData.AUGDDData method}}

\begin{fulllineitems}
\phantomsection\label{\detokenize{eqtools:eqtools.AUGData.AUGDDData.getF}}\pysiglinewithargsret{\sphinxbfcode{\sphinxupquote{getF}}}{}{}
returns F=RB\_\{Phi\}(Psi), often calculated for grad-shafranov
solutions.
\begin{quote}\begin{description}
\item[{Returns}] \leavevmode
{[}nt,npsi{]} array of F=RB\_\{Phi\}(Psi)

\item[{Return type}] \leavevmode
F (Array)

\item[{Raises}] \leavevmode
\sphinxstyleliteralstrong{\sphinxupquote{ValueError}} \textendash{} if module cannot retrieve data from the AUG AFS system.

\end{description}\end{quote}

\end{fulllineitems}

\index{getFluxPres() (eqtools.AUGData.AUGDDData method)@\spxentry{getFluxPres()}\spxextra{eqtools.AUGData.AUGDDData method}}

\begin{fulllineitems}
\phantomsection\label{\detokenize{eqtools:eqtools.AUGData.AUGDDData.getFluxPres}}\pysiglinewithargsret{\sphinxbfcode{\sphinxupquote{getFluxPres}}}{}{}
returns pressure at flux surface.
\begin{quote}\begin{description}
\item[{Returns}] \leavevmode
{[}nt,npsi{]} array of pressure on flux surface psi.

\item[{Return type}] \leavevmode
p (Array)

\item[{Raises}] \leavevmode
\sphinxstyleliteralstrong{\sphinxupquote{ValueError}} \textendash{} if module cannot retrieve data from AUG AFS system.

\end{description}\end{quote}

\end{fulllineitems}

\index{getFPrime() (eqtools.AUGData.AUGDDData method)@\spxentry{getFPrime()}\spxextra{eqtools.AUGData.AUGDDData method}}

\begin{fulllineitems}
\phantomsection\label{\detokenize{eqtools:eqtools.AUGData.AUGDDData.getFPrime}}\pysiglinewithargsret{\sphinxbfcode{\sphinxupquote{getFPrime}}}{}{}
returns F’, often calculated for grad-shafranov
solutions.
\begin{quote}\begin{description}
\item[{Returns}] \leavevmode
{[}nt,npsi{]} array of F=RB\_\{Phi\}(Psi)

\item[{Return type}] \leavevmode
F (Array)

\item[{Raises}] \leavevmode
\sphinxstyleliteralstrong{\sphinxupquote{ValueError}} \textendash{} if module cannot retrieve data from the AUG AFS system.

\end{description}\end{quote}

\end{fulllineitems}

\index{getFFPrime() (eqtools.AUGData.AUGDDData method)@\spxentry{getFFPrime()}\spxextra{eqtools.AUGData.AUGDDData method}}

\begin{fulllineitems}
\phantomsection\label{\detokenize{eqtools:eqtools.AUGData.AUGDDData.getFFPrime}}\pysiglinewithargsret{\sphinxbfcode{\sphinxupquote{getFFPrime}}}{}{}
returns FF’ function used for grad-shafranov solutions.
\begin{quote}\begin{description}
\item[{Returns}] \leavevmode
{[}nt,npsi{]} array of FF’ fromgrad-shafranov solution.

\item[{Return type}] \leavevmode
FFprime (Array)

\item[{Raises}] \leavevmode
\sphinxstyleliteralstrong{\sphinxupquote{ValueError}} \textendash{} if module cannot retrieve data from the AUG AFS system.

\end{description}\end{quote}

\end{fulllineitems}

\index{getPPrime() (eqtools.AUGData.AUGDDData method)@\spxentry{getPPrime()}\spxextra{eqtools.AUGData.AUGDDData method}}

\begin{fulllineitems}
\phantomsection\label{\detokenize{eqtools:eqtools.AUGData.AUGDDData.getPPrime}}\pysiglinewithargsret{\sphinxbfcode{\sphinxupquote{getPPrime}}}{}{}
returns plasma pressure gradient as a function of psi.
\begin{quote}\begin{description}
\item[{Returns}] \leavevmode
{[}nt,npsi{]} array of pressure gradient on flux surface
psi from grad-shafranov solution.

\item[{Return type}] \leavevmode
pprime (Array)

\item[{Raises}] \leavevmode
\sphinxstyleliteralstrong{\sphinxupquote{ValueError}} \textendash{} if module cannot retrieve data from the AUG AFS system.

\end{description}\end{quote}

\end{fulllineitems}

\index{getElongation() (eqtools.AUGData.AUGDDData method)@\spxentry{getElongation()}\spxextra{eqtools.AUGData.AUGDDData method}}

\begin{fulllineitems}
\phantomsection\label{\detokenize{eqtools:eqtools.AUGData.AUGDDData.getElongation}}\pysiglinewithargsret{\sphinxbfcode{\sphinxupquote{getElongation}}}{}{}
returns LCFS elongation.
\begin{quote}\begin{description}
\item[{Returns}] \leavevmode
{[}nt{]} array of LCFS elongation.

\item[{Return type}] \leavevmode
kappa (Array)

\item[{Raises}] \leavevmode
\sphinxstyleliteralstrong{\sphinxupquote{ValueError}} \textendash{} if module cannot retrieve data from AFS.

\end{description}\end{quote}

\end{fulllineitems}

\index{getUpperTriangularity() (eqtools.AUGData.AUGDDData method)@\spxentry{getUpperTriangularity()}\spxextra{eqtools.AUGData.AUGDDData method}}

\begin{fulllineitems}
\phantomsection\label{\detokenize{eqtools:eqtools.AUGData.AUGDDData.getUpperTriangularity}}\pysiglinewithargsret{\sphinxbfcode{\sphinxupquote{getUpperTriangularity}}}{}{}
returns LCFS upper triangularity.
\begin{quote}\begin{description}
\item[{Returns}] \leavevmode
{[}nt{]} array of LCFS upper triangularity.

\item[{Return type}] \leavevmode
deltau (Array)

\item[{Raises}] \leavevmode
\sphinxstyleliteralstrong{\sphinxupquote{ValueError}} \textendash{} if module cannot retrieve data from the AUG AFS system.

\end{description}\end{quote}

\end{fulllineitems}

\index{getLowerTriangularity() (eqtools.AUGData.AUGDDData method)@\spxentry{getLowerTriangularity()}\spxextra{eqtools.AUGData.AUGDDData method}}

\begin{fulllineitems}
\phantomsection\label{\detokenize{eqtools:eqtools.AUGData.AUGDDData.getLowerTriangularity}}\pysiglinewithargsret{\sphinxbfcode{\sphinxupquote{getLowerTriangularity}}}{}{}
returns LCFS lower triangularity.
\begin{quote}\begin{description}
\item[{Returns}] \leavevmode
{[}nt{]} array of LCFS lower triangularity.

\item[{Return type}] \leavevmode
deltal (Array)

\item[{Raises}] \leavevmode
\sphinxstyleliteralstrong{\sphinxupquote{ValueError}} \textendash{} if module cannot retrieve data from the AUG AFS system.

\end{description}\end{quote}

\end{fulllineitems}

\index{getShaping() (eqtools.AUGData.AUGDDData method)@\spxentry{getShaping()}\spxextra{eqtools.AUGData.AUGDDData method}}

\begin{fulllineitems}
\phantomsection\label{\detokenize{eqtools:eqtools.AUGData.AUGDDData.getShaping}}\pysiglinewithargsret{\sphinxbfcode{\sphinxupquote{getShaping}}}{}{}
pulls LCFS elongation and upper/lower triangularity.
\begin{quote}\begin{description}
\item[{Returns}] \leavevmode
namedtuple containing (kappa, delta\_u, delta\_l)

\item[{Raises}] \leavevmode
\sphinxstyleliteralstrong{\sphinxupquote{ValueError}} \textendash{} if module cannot retrieve data from the AUG AFS system.

\end{description}\end{quote}

\end{fulllineitems}

\index{getMagR() (eqtools.AUGData.AUGDDData method)@\spxentry{getMagR()}\spxextra{eqtools.AUGData.AUGDDData method}}

\begin{fulllineitems}
\phantomsection\label{\detokenize{eqtools:eqtools.AUGData.AUGDDData.getMagR}}\pysiglinewithargsret{\sphinxbfcode{\sphinxupquote{getMagR}}}{\emph{length\_unit=1}}{}
returns magnetic-axis major radius.
\begin{quote}\begin{description}
\item[{Returns}] \leavevmode
{[}nt{]} array of major radius of magnetic axis.

\item[{Return type}] \leavevmode
magR (Array)

\item[{Raises}] \leavevmode
\sphinxstyleliteralstrong{\sphinxupquote{ValueError}} \textendash{} if module cannot retrieve data from the AUG AFS system.

\end{description}\end{quote}

\end{fulllineitems}

\index{getMagZ() (eqtools.AUGData.AUGDDData method)@\spxentry{getMagZ()}\spxextra{eqtools.AUGData.AUGDDData method}}

\begin{fulllineitems}
\phantomsection\label{\detokenize{eqtools:eqtools.AUGData.AUGDDData.getMagZ}}\pysiglinewithargsret{\sphinxbfcode{\sphinxupquote{getMagZ}}}{\emph{length\_unit=1}}{}
returns magnetic-axis Z.
\begin{quote}\begin{description}
\item[{Returns}] \leavevmode
{[}nt{]} array of Z of magnetic axis.

\item[{Return type}] \leavevmode
magZ (Array)

\item[{Raises}] \leavevmode
\sphinxstyleliteralstrong{\sphinxupquote{ValueError}} \textendash{} if module cannot retrieve data from the AUG AFS system.

\end{description}\end{quote}

\end{fulllineitems}

\index{getAreaLCFS() (eqtools.AUGData.AUGDDData method)@\spxentry{getAreaLCFS()}\spxextra{eqtools.AUGData.AUGDDData method}}

\begin{fulllineitems}
\phantomsection\label{\detokenize{eqtools:eqtools.AUGData.AUGDDData.getAreaLCFS}}\pysiglinewithargsret{\sphinxbfcode{\sphinxupquote{getAreaLCFS}}}{\emph{length\_unit=2}}{}
returns LCFS cross-sectional area.
\begin{quote}\begin{description}
\item[{Keyword Arguments}] \leavevmode
\sphinxstyleliteralstrong{\sphinxupquote{length\_unit}} (\sphinxstyleliteralemphasis{\sphinxupquote{String}}\sphinxstyleliteralemphasis{\sphinxupquote{ or }}\sphinxstyleliteralemphasis{\sphinxupquote{2}}) \textendash{} unit for LCFS area.  Defaults to 2,
denoting default areal unit (typically m\textasciicircum{}2).

\item[{Returns}] \leavevmode
{[}nt{]} array of LCFS area.

\item[{Return type}] \leavevmode
areaLCFS (Array)

\item[{Raises}] \leavevmode
\sphinxstyleliteralstrong{\sphinxupquote{ValueError}} \textendash{} if module cannot retrieve data from the AUG AFS system.

\end{description}\end{quote}

\end{fulllineitems}

\index{getAOut() (eqtools.AUGData.AUGDDData method)@\spxentry{getAOut()}\spxextra{eqtools.AUGData.AUGDDData method}}

\begin{fulllineitems}
\phantomsection\label{\detokenize{eqtools:eqtools.AUGData.AUGDDData.getAOut}}\pysiglinewithargsret{\sphinxbfcode{\sphinxupquote{getAOut}}}{\emph{length\_unit=1}}{}
returns outboard-midplane minor radius at LCFS.
\begin{quote}\begin{description}
\item[{Keyword Arguments}] \leavevmode
\sphinxstyleliteralstrong{\sphinxupquote{length\_unit}} (\sphinxstyleliteralemphasis{\sphinxupquote{String}}\sphinxstyleliteralemphasis{\sphinxupquote{ or }}\sphinxstyleliteralemphasis{\sphinxupquote{1}}) \textendash{} unit for minor radius.  Defaults to 1,
denoting default length unit (typically m).

\item[{Returns}] \leavevmode
{[}nt{]} array of LCFS outboard-midplane minor radius.

\item[{Return type}] \leavevmode
aOut (Array)

\item[{Raises}] \leavevmode
\sphinxstyleliteralstrong{\sphinxupquote{ValueError}} \textendash{} if module cannot retrieve data from the AUG AFS system.

\end{description}\end{quote}

\end{fulllineitems}

\index{getRmidOut() (eqtools.AUGData.AUGDDData method)@\spxentry{getRmidOut()}\spxextra{eqtools.AUGData.AUGDDData method}}

\begin{fulllineitems}
\phantomsection\label{\detokenize{eqtools:eqtools.AUGData.AUGDDData.getRmidOut}}\pysiglinewithargsret{\sphinxbfcode{\sphinxupquote{getRmidOut}}}{\emph{length\_unit=1}}{}
returns outboard-midplane major radius.
\begin{quote}\begin{description}
\item[{Keyword Arguments}] \leavevmode
\sphinxstyleliteralstrong{\sphinxupquote{length\_unit}} (\sphinxstyleliteralemphasis{\sphinxupquote{String}}\sphinxstyleliteralemphasis{\sphinxupquote{ or }}\sphinxstyleliteralemphasis{\sphinxupquote{1}}) \textendash{} unit for major radius.  Defaults to 1,
denoting default length unit (typically m).

\item[{Raises}] \leavevmode
\sphinxstyleliteralstrong{\sphinxupquote{NotImplementedError}} \textendash{} Not implemented on ASDEX-Upgrade reconstructions.

\end{description}\end{quote}

\end{fulllineitems}

\index{getGeometry() (eqtools.AUGData.AUGDDData method)@\spxentry{getGeometry()}\spxextra{eqtools.AUGData.AUGDDData method}}

\begin{fulllineitems}
\phantomsection\label{\detokenize{eqtools:eqtools.AUGData.AUGDDData.getGeometry}}\pysiglinewithargsret{\sphinxbfcode{\sphinxupquote{getGeometry}}}{\emph{length\_unit=None}}{}
pulls dimensional geometry parameters.
\begin{quote}\begin{description}
\item[{Returns}] \leavevmode
namedtuple containing (magR,magZ,areaLCFS,aOut,RmidOut)

\item[{Raises}] \leavevmode
\sphinxstyleliteralstrong{\sphinxupquote{ValueError}} \textendash{} if module cannot retrieve data from the AUG AFS system.

\end{description}\end{quote}

\end{fulllineitems}

\index{getQProfile() (eqtools.AUGData.AUGDDData method)@\spxentry{getQProfile()}\spxextra{eqtools.AUGData.AUGDDData method}}

\begin{fulllineitems}
\phantomsection\label{\detokenize{eqtools:eqtools.AUGData.AUGDDData.getQProfile}}\pysiglinewithargsret{\sphinxbfcode{\sphinxupquote{getQProfile}}}{}{}
returns profile of safety factor q.
\begin{quote}\begin{description}
\item[{Returns}] \leavevmode
{[}nt,npsi{]} array of q on flux surface psi.

\item[{Return type}] \leavevmode
qpsi (Array)

\item[{Raises}] \leavevmode
\sphinxstyleliteralstrong{\sphinxupquote{ValueError}} \textendash{} if module cannot retrieve data from the AUG AFS system.

\end{description}\end{quote}

\end{fulllineitems}

\index{getQ0() (eqtools.AUGData.AUGDDData method)@\spxentry{getQ0()}\spxextra{eqtools.AUGData.AUGDDData method}}

\begin{fulllineitems}
\phantomsection\label{\detokenize{eqtools:eqtools.AUGData.AUGDDData.getQ0}}\pysiglinewithargsret{\sphinxbfcode{\sphinxupquote{getQ0}}}{}{}
returns q on magnetic axis,q0.
\begin{quote}\begin{description}
\item[{Returns}] \leavevmode
{[}nt{]} array of q(psi=0).

\item[{Return type}] \leavevmode
q0 (Array)

\item[{Raises}] \leavevmode
\sphinxstyleliteralstrong{\sphinxupquote{ValueError}} \textendash{} if module cannot retrieve data from the AUG AFS system.

\end{description}\end{quote}

\end{fulllineitems}

\index{getQ95() (eqtools.AUGData.AUGDDData method)@\spxentry{getQ95()}\spxextra{eqtools.AUGData.AUGDDData method}}

\begin{fulllineitems}
\phantomsection\label{\detokenize{eqtools:eqtools.AUGData.AUGDDData.getQ95}}\pysiglinewithargsret{\sphinxbfcode{\sphinxupquote{getQ95}}}{}{}
returns q at 95\% flux surface.
\begin{quote}\begin{description}
\item[{Returns}] \leavevmode
{[}nt{]} array of q(psi=0.95).

\item[{Return type}] \leavevmode
q95 (Array)

\item[{Raises}] \leavevmode
\sphinxstyleliteralstrong{\sphinxupquote{ValueError}} \textendash{} if module cannot retrieve data from the AUG AFS system.

\end{description}\end{quote}

\end{fulllineitems}

\index{getQLCFS() (eqtools.AUGData.AUGDDData method)@\spxentry{getQLCFS()}\spxextra{eqtools.AUGData.AUGDDData method}}

\begin{fulllineitems}
\phantomsection\label{\detokenize{eqtools:eqtools.AUGData.AUGDDData.getQLCFS}}\pysiglinewithargsret{\sphinxbfcode{\sphinxupquote{getQLCFS}}}{}{}
returns q on LCFS (interpolated).
\begin{quote}\begin{description}
\item[{Raises}] \leavevmode
\sphinxstyleliteralstrong{\sphinxupquote{NotImplementedError}} \textendash{} Not implemented on ASDEX-Upgrade reconstructions.

\end{description}\end{quote}

\end{fulllineitems}

\index{getQ1Surf() (eqtools.AUGData.AUGDDData method)@\spxentry{getQ1Surf()}\spxextra{eqtools.AUGData.AUGDDData method}}

\begin{fulllineitems}
\phantomsection\label{\detokenize{eqtools:eqtools.AUGData.AUGDDData.getQ1Surf}}\pysiglinewithargsret{\sphinxbfcode{\sphinxupquote{getQ1Surf}}}{\emph{length\_unit=1}}{}
returns outboard-midplane minor radius of q=1 surface.
\begin{quote}\begin{description}
\item[{Raises}] \leavevmode
\sphinxstyleliteralstrong{\sphinxupquote{NotImplementedError}} \textendash{} Not implemented on ASDEX-Upgrade reconstructions.

\end{description}\end{quote}

\end{fulllineitems}

\index{getQ2Surf() (eqtools.AUGData.AUGDDData method)@\spxentry{getQ2Surf()}\spxextra{eqtools.AUGData.AUGDDData method}}

\begin{fulllineitems}
\phantomsection\label{\detokenize{eqtools:eqtools.AUGData.AUGDDData.getQ2Surf}}\pysiglinewithargsret{\sphinxbfcode{\sphinxupquote{getQ2Surf}}}{\emph{length\_unit=1}}{}
returns outboard-midplane minor radius of q=2 surface.
\begin{quote}\begin{description}
\item[{Raises}] \leavevmode
\sphinxstyleliteralstrong{\sphinxupquote{NotImplementedError}} \textendash{} Not implemented on ASDEX-Upgrade reconstructions.

\end{description}\end{quote}

\end{fulllineitems}

\index{getQ3Surf() (eqtools.AUGData.AUGDDData method)@\spxentry{getQ3Surf()}\spxextra{eqtools.AUGData.AUGDDData method}}

\begin{fulllineitems}
\phantomsection\label{\detokenize{eqtools:eqtools.AUGData.AUGDDData.getQ3Surf}}\pysiglinewithargsret{\sphinxbfcode{\sphinxupquote{getQ3Surf}}}{\emph{length\_unit=1}}{}
returns outboard-midplane minor radius of q=3 surface.
\begin{quote}\begin{description}
\item[{Raises}] \leavevmode
\sphinxstyleliteralstrong{\sphinxupquote{NotImplementedError}} \textendash{} Not implemented on ASDEX-Upgrade reconstructions.

\end{description}\end{quote}

\end{fulllineitems}

\index{getQs() (eqtools.AUGData.AUGDDData method)@\spxentry{getQs()}\spxextra{eqtools.AUGData.AUGDDData method}}

\begin{fulllineitems}
\phantomsection\label{\detokenize{eqtools:eqtools.AUGData.AUGDDData.getQs}}\pysiglinewithargsret{\sphinxbfcode{\sphinxupquote{getQs}}}{\emph{length\_unit=1}}{}
pulls q values.
\begin{quote}\begin{description}
\item[{Raises}] \leavevmode
\sphinxstyleliteralstrong{\sphinxupquote{NotImplementedError}} \textendash{} Not implemented on ASDEX-Upgrade reconstructions.

\end{description}\end{quote}

\end{fulllineitems}

\index{getBtVac() (eqtools.AUGData.AUGDDData method)@\spxentry{getBtVac()}\spxextra{eqtools.AUGData.AUGDDData method}}

\begin{fulllineitems}
\phantomsection\label{\detokenize{eqtools:eqtools.AUGData.AUGDDData.getBtVac}}\pysiglinewithargsret{\sphinxbfcode{\sphinxupquote{getBtVac}}}{}{}
Returns vacuum toroidal field on-axis. THIS MAY BE INCORRECT
\begin{quote}\begin{description}
\item[{Returns}] \leavevmode
{[}nt{]} array of vacuum toroidal field.

\item[{Return type}] \leavevmode
BtVac (Array)

\item[{Raises}] \leavevmode
\sphinxstyleliteralstrong{\sphinxupquote{ValueError}} \textendash{} if module cannot retrieve data from the AUG AFS system.

\end{description}\end{quote}

\end{fulllineitems}

\index{getBtPla() (eqtools.AUGData.AUGDDData method)@\spxentry{getBtPla()}\spxextra{eqtools.AUGData.AUGDDData method}}

\begin{fulllineitems}
\phantomsection\label{\detokenize{eqtools:eqtools.AUGData.AUGDDData.getBtPla}}\pysiglinewithargsret{\sphinxbfcode{\sphinxupquote{getBtPla}}}{}{}
returns on-axis plasma toroidal field.
\begin{quote}\begin{description}
\item[{Raises}] \leavevmode
\sphinxstyleliteralstrong{\sphinxupquote{NotImplementedError}} \textendash{} Not implemented on ASDEX-Upgrade reconstructions.

\end{description}\end{quote}

\end{fulllineitems}

\index{getBpAvg() (eqtools.AUGData.AUGDDData method)@\spxentry{getBpAvg()}\spxextra{eqtools.AUGData.AUGDDData method}}

\begin{fulllineitems}
\phantomsection\label{\detokenize{eqtools:eqtools.AUGData.AUGDDData.getBpAvg}}\pysiglinewithargsret{\sphinxbfcode{\sphinxupquote{getBpAvg}}}{}{}
returns average poloidal field.
\begin{quote}\begin{description}
\item[{Raises}] \leavevmode
\sphinxstyleliteralstrong{\sphinxupquote{NotImplementedError}} \textendash{} Not implemented on ASDEX-Upgrade reconstructions.

\end{description}\end{quote}

\end{fulllineitems}

\index{getFields() (eqtools.AUGData.AUGDDData method)@\spxentry{getFields()}\spxextra{eqtools.AUGData.AUGDDData method}}

\begin{fulllineitems}
\phantomsection\label{\detokenize{eqtools:eqtools.AUGData.AUGDDData.getFields}}\pysiglinewithargsret{\sphinxbfcode{\sphinxupquote{getFields}}}{}{}
pulls vacuum and plasma toroidal field, avg poloidal field.
\begin{quote}\begin{description}
\item[{Raises}] \leavevmode
\sphinxstyleliteralstrong{\sphinxupquote{NotImplementedError}} \textendash{} Not implemented on ASDEX-Upgrade reconstructions.

\end{description}\end{quote}

\end{fulllineitems}

\index{getIpCalc() (eqtools.AUGData.AUGDDData method)@\spxentry{getIpCalc()}\spxextra{eqtools.AUGData.AUGDDData method}}

\begin{fulllineitems}
\phantomsection\label{\detokenize{eqtools:eqtools.AUGData.AUGDDData.getIpCalc}}\pysiglinewithargsret{\sphinxbfcode{\sphinxupquote{getIpCalc}}}{}{}
returns Plasma Current, is the same as getIpMeas.
\begin{quote}\begin{description}
\item[{Returns}] \leavevmode
{[}nt{]} array of the reconstructed plasma current.

\item[{Return type}] \leavevmode
IpCalc (Array)

\item[{Raises}] \leavevmode
\sphinxstyleliteralstrong{\sphinxupquote{ValueError}} \textendash{} if module cannot retrieve data from the AUG AFS system.

\end{description}\end{quote}

\end{fulllineitems}

\index{getIpMeas() (eqtools.AUGData.AUGDDData method)@\spxentry{getIpMeas()}\spxextra{eqtools.AUGData.AUGDDData method}}

\begin{fulllineitems}
\phantomsection\label{\detokenize{eqtools:eqtools.AUGData.AUGDDData.getIpMeas}}\pysiglinewithargsret{\sphinxbfcode{\sphinxupquote{getIpMeas}}}{}{}
returns magnetics-measured plasma current.
\begin{quote}\begin{description}
\item[{Returns}] \leavevmode
{[}nt{]} array of measured plasma current.

\item[{Return type}] \leavevmode
IpMeas (Array)

\item[{Raises}] \leavevmode
\sphinxstyleliteralstrong{\sphinxupquote{ValueError}} \textendash{} if module cannot retrieve data from the AUG AFS system.

\end{description}\end{quote}

\end{fulllineitems}

\index{getJp() (eqtools.AUGData.AUGDDData method)@\spxentry{getJp()}\spxextra{eqtools.AUGData.AUGDDData method}}

\begin{fulllineitems}
\phantomsection\label{\detokenize{eqtools:eqtools.AUGData.AUGDDData.getJp}}\pysiglinewithargsret{\sphinxbfcode{\sphinxupquote{getJp}}}{}{}
returns the calculated plasma current density Jp on flux grid.
\begin{quote}\begin{description}
\item[{Returns}] \leavevmode
{[}nt,nz,nr{]} array of current density.

\item[{Return type}] \leavevmode
Jp (Array)

\item[{Raises}] \leavevmode
\sphinxstyleliteralstrong{\sphinxupquote{ValueError}} \textendash{} if module cannot retrieve data from the AUG AFS system.

\end{description}\end{quote}

\end{fulllineitems}

\index{getBetaT() (eqtools.AUGData.AUGDDData method)@\spxentry{getBetaT()}\spxextra{eqtools.AUGData.AUGDDData method}}

\begin{fulllineitems}
\phantomsection\label{\detokenize{eqtools:eqtools.AUGData.AUGDDData.getBetaT}}\pysiglinewithargsret{\sphinxbfcode{\sphinxupquote{getBetaT}}}{}{}
returns the calculated toroidal beta.
\begin{quote}\begin{description}
\item[{Raises}] \leavevmode
\sphinxstyleliteralstrong{\sphinxupquote{NotImplementedError}} \textendash{} Not implemented on ASDEX-Upgrade reconstructions.

\end{description}\end{quote}

\end{fulllineitems}

\index{getBetaP() (eqtools.AUGData.AUGDDData method)@\spxentry{getBetaP()}\spxextra{eqtools.AUGData.AUGDDData method}}

\begin{fulllineitems}
\phantomsection\label{\detokenize{eqtools:eqtools.AUGData.AUGDDData.getBetaP}}\pysiglinewithargsret{\sphinxbfcode{\sphinxupquote{getBetaP}}}{}{}
returns the calculated poloidal beta.
\begin{quote}\begin{description}
\item[{Returns}] \leavevmode
{[}nt{]} array of the calculated average poloidal beta.

\item[{Return type}] \leavevmode
BetaP (Array)

\item[{Raises}] \leavevmode
\sphinxstyleliteralstrong{\sphinxupquote{ValueError}} \textendash{} if module cannot retrieve data from the AUG AFS system.

\end{description}\end{quote}

\end{fulllineitems}

\index{getLi() (eqtools.AUGData.AUGDDData method)@\spxentry{getLi()}\spxextra{eqtools.AUGData.AUGDDData method}}

\begin{fulllineitems}
\phantomsection\label{\detokenize{eqtools:eqtools.AUGData.AUGDDData.getLi}}\pysiglinewithargsret{\sphinxbfcode{\sphinxupquote{getLi}}}{}{}
returns the calculated internal inductance.
\begin{quote}\begin{description}
\item[{Returns}] \leavevmode
{[}nt{]} array of the calculated internal inductance.

\item[{Return type}] \leavevmode
Li (Array)

\item[{Raises}] \leavevmode
\sphinxstyleliteralstrong{\sphinxupquote{ValueError}} \textendash{} if module cannot retrieve data from the AUG afs system.

\end{description}\end{quote}

\end{fulllineitems}

\index{getBetas() (eqtools.AUGData.AUGDDData method)@\spxentry{getBetas()}\spxextra{eqtools.AUGData.AUGDDData method}}

\begin{fulllineitems}
\phantomsection\label{\detokenize{eqtools:eqtools.AUGData.AUGDDData.getBetas}}\pysiglinewithargsret{\sphinxbfcode{\sphinxupquote{getBetas}}}{}{}
pulls calculated betap, betat, internal inductance.
\begin{quote}\begin{description}
\item[{Raises}] \leavevmode
\sphinxstyleliteralstrong{\sphinxupquote{NotImplementedError}} \textendash{} Not implemented on ASDEX-Upgrade reconstructions.

\end{description}\end{quote}

\end{fulllineitems}

\index{getDiamagFlux() (eqtools.AUGData.AUGDDData method)@\spxentry{getDiamagFlux()}\spxextra{eqtools.AUGData.AUGDDData method}}

\begin{fulllineitems}
\phantomsection\label{\detokenize{eqtools:eqtools.AUGData.AUGDDData.getDiamagFlux}}\pysiglinewithargsret{\sphinxbfcode{\sphinxupquote{getDiamagFlux}}}{}{}
returns the measured diamagnetic-loop flux.
\begin{quote}\begin{description}
\item[{Raises}] \leavevmode
\sphinxstyleliteralstrong{\sphinxupquote{NotImplementedError}} \textendash{} Not implemented on ASDEX-Upgrade reconstructions.

\end{description}\end{quote}

\end{fulllineitems}

\index{getDiamagBetaT() (eqtools.AUGData.AUGDDData method)@\spxentry{getDiamagBetaT()}\spxextra{eqtools.AUGData.AUGDDData method}}

\begin{fulllineitems}
\phantomsection\label{\detokenize{eqtools:eqtools.AUGData.AUGDDData.getDiamagBetaT}}\pysiglinewithargsret{\sphinxbfcode{\sphinxupquote{getDiamagBetaT}}}{}{}
returns diamagnetic-loop toroidal beta.
\begin{quote}\begin{description}
\item[{Raises}] \leavevmode
\sphinxstyleliteralstrong{\sphinxupquote{NotImplementedError}} \textendash{} Not implemented on ASDEX-Upgrade reconstructions.

\end{description}\end{quote}

\end{fulllineitems}

\index{getDiamagBetaP() (eqtools.AUGData.AUGDDData method)@\spxentry{getDiamagBetaP()}\spxextra{eqtools.AUGData.AUGDDData method}}

\begin{fulllineitems}
\phantomsection\label{\detokenize{eqtools:eqtools.AUGData.AUGDDData.getDiamagBetaP}}\pysiglinewithargsret{\sphinxbfcode{\sphinxupquote{getDiamagBetaP}}}{}{}
returns diamagnetic-loop avg poloidal beta.
\begin{quote}\begin{description}
\item[{Raises}] \leavevmode
\sphinxstyleliteralstrong{\sphinxupquote{NotImplementedError}} \textendash{} Not implemented on ASDEX-Upgrade reconstructions.

\end{description}\end{quote}

\end{fulllineitems}

\index{getDiamagTauE() (eqtools.AUGData.AUGDDData method)@\spxentry{getDiamagTauE()}\spxextra{eqtools.AUGData.AUGDDData method}}

\begin{fulllineitems}
\phantomsection\label{\detokenize{eqtools:eqtools.AUGData.AUGDDData.getDiamagTauE}}\pysiglinewithargsret{\sphinxbfcode{\sphinxupquote{getDiamagTauE}}}{}{}
returns diamagnetic-loop energy confinement time.
\begin{quote}\begin{description}
\item[{Raises}] \leavevmode
\sphinxstyleliteralstrong{\sphinxupquote{NotImplementedError}} \textendash{} Not implemented on ASDEX-Upgrade reconstructions.

\end{description}\end{quote}

\end{fulllineitems}

\index{getDiamagWp() (eqtools.AUGData.AUGDDData method)@\spxentry{getDiamagWp()}\spxextra{eqtools.AUGData.AUGDDData method}}

\begin{fulllineitems}
\phantomsection\label{\detokenize{eqtools:eqtools.AUGData.AUGDDData.getDiamagWp}}\pysiglinewithargsret{\sphinxbfcode{\sphinxupquote{getDiamagWp}}}{}{}
returns diamagnetic-loop plasma stored energy.
\begin{quote}\begin{description}
\item[{Raises}] \leavevmode
\sphinxstyleliteralstrong{\sphinxupquote{NotImplementedError}} \textendash{} Not implemented on ASDEX-Upgrade reconstructions.

\end{description}\end{quote}

\end{fulllineitems}

\index{getDiamag() (eqtools.AUGData.AUGDDData method)@\spxentry{getDiamag()}\spxextra{eqtools.AUGData.AUGDDData method}}

\begin{fulllineitems}
\phantomsection\label{\detokenize{eqtools:eqtools.AUGData.AUGDDData.getDiamag}}\pysiglinewithargsret{\sphinxbfcode{\sphinxupquote{getDiamag}}}{}{}
pulls diamagnetic flux measurements, toroidal and poloidal beta,
energy confinement time and stored energy.
\begin{quote}\begin{description}
\item[{Raises}] \leavevmode
\sphinxstyleliteralstrong{\sphinxupquote{NotImplementedError}} \textendash{} Not implemented on ASDEX-Upgrade reconstructions.

\end{description}\end{quote}

\end{fulllineitems}

\index{getWMHD() (eqtools.AUGData.AUGDDData method)@\spxentry{getWMHD()}\spxextra{eqtools.AUGData.AUGDDData method}}

\begin{fulllineitems}
\phantomsection\label{\detokenize{eqtools:eqtools.AUGData.AUGDDData.getWMHD}}\pysiglinewithargsret{\sphinxbfcode{\sphinxupquote{getWMHD}}}{}{}
returns calculated MHD stored energy.
\begin{quote}\begin{description}
\item[{Returns}] \leavevmode
{[}nt{]} array of the calculated stored energy.

\item[{Return type}] \leavevmode
WMHD (Array)

\item[{Raises}] \leavevmode
\sphinxstyleliteralstrong{\sphinxupquote{ValueError}} \textendash{} if module cannot retrieve data from the AUG afs system.

\end{description}\end{quote}

\end{fulllineitems}

\index{getTauMHD() (eqtools.AUGData.AUGDDData method)@\spxentry{getTauMHD()}\spxextra{eqtools.AUGData.AUGDDData method}}

\begin{fulllineitems}
\phantomsection\label{\detokenize{eqtools:eqtools.AUGData.AUGDDData.getTauMHD}}\pysiglinewithargsret{\sphinxbfcode{\sphinxupquote{getTauMHD}}}{}{}
returns the calculated MHD energy confinement time.
\begin{quote}\begin{description}
\item[{Raises}] \leavevmode
\sphinxstyleliteralstrong{\sphinxupquote{NotImplementedError}} \textendash{} Not implemented on ASDEX-Upgrade reconstructions.

\end{description}\end{quote}

\end{fulllineitems}

\index{getPinj() (eqtools.AUGData.AUGDDData method)@\spxentry{getPinj()}\spxextra{eqtools.AUGData.AUGDDData method}}

\begin{fulllineitems}
\phantomsection\label{\detokenize{eqtools:eqtools.AUGData.AUGDDData.getPinj}}\pysiglinewithargsret{\sphinxbfcode{\sphinxupquote{getPinj}}}{}{}
returns the injected power.
\begin{quote}\begin{description}
\item[{Raises}] \leavevmode\begin{itemize}
\item {} 
\sphinxstyleliteralstrong{\sphinxupquote{NotImplementedError}} \textendash{} Not implemented on ASDEX-Upgrade reconstructions.

\item {} 
\sphinxstyleliteralstrong{\sphinxupquote{}} \textendash{} 

\end{itemize}

\end{description}\end{quote}

\end{fulllineitems}

\index{getWbdot() (eqtools.AUGData.AUGDDData method)@\spxentry{getWbdot()}\spxextra{eqtools.AUGData.AUGDDData method}}

\begin{fulllineitems}
\phantomsection\label{\detokenize{eqtools:eqtools.AUGData.AUGDDData.getWbdot}}\pysiglinewithargsret{\sphinxbfcode{\sphinxupquote{getWbdot}}}{}{}
returns the calculated d/dt of magnetic stored energy.
\begin{quote}\begin{description}
\item[{Raises}] \leavevmode
\sphinxstyleliteralstrong{\sphinxupquote{NotImplementedError}} \textendash{} Not implemented on ASDEX-Upgrade reconstructions.

\end{description}\end{quote}

\end{fulllineitems}

\index{getWpdot() (eqtools.AUGData.AUGDDData method)@\spxentry{getWpdot()}\spxextra{eqtools.AUGData.AUGDDData method}}

\begin{fulllineitems}
\phantomsection\label{\detokenize{eqtools:eqtools.AUGData.AUGDDData.getWpdot}}\pysiglinewithargsret{\sphinxbfcode{\sphinxupquote{getWpdot}}}{}{}
returns the calculated d/dt of plasma stored energy.
\begin{quote}\begin{description}
\item[{Raises}] \leavevmode
\sphinxstyleliteralstrong{\sphinxupquote{NotImplementedError}} \textendash{} Not implemented on ASDEX-Upgrade reconstructions.

\end{description}\end{quote}

\end{fulllineitems}

\index{getBCentr() (eqtools.AUGData.AUGDDData method)@\spxentry{getBCentr()}\spxextra{eqtools.AUGData.AUGDDData method}}

\begin{fulllineitems}
\phantomsection\label{\detokenize{eqtools:eqtools.AUGData.AUGDDData.getBCentr}}\pysiglinewithargsret{\sphinxbfcode{\sphinxupquote{getBCentr}}}{}{}
returns Vacuum toroidal magnetic field at center of plasma
\begin{quote}\begin{description}
\item[{Returns}] \leavevmode
{[}nt{]} array of B\_t at center {[}T{]}

\item[{Return type}] \leavevmode
B\_cent (Array)

\item[{Raises}] \leavevmode
\sphinxstyleliteralstrong{\sphinxupquote{ValueError}} \textendash{} if module cannot retrieve data from the AUG afs system.

\end{description}\end{quote}

\end{fulllineitems}

\index{getRCentr() (eqtools.AUGData.AUGDDData method)@\spxentry{getRCentr()}\spxextra{eqtools.AUGData.AUGDDData method}}

\begin{fulllineitems}
\phantomsection\label{\detokenize{eqtools:eqtools.AUGData.AUGDDData.getRCentr}}\pysiglinewithargsret{\sphinxbfcode{\sphinxupquote{getRCentr}}}{\emph{length\_unit=1}}{}
Returns Radius of BCenter measurement
\begin{quote}\begin{description}
\item[{Returns}] \leavevmode
Radial position where Bcent calculated {[}m{]}

\item[{Return type}] \leavevmode
R

\end{description}\end{quote}

\end{fulllineitems}

\index{getEnergy() (eqtools.AUGData.AUGDDData method)@\spxentry{getEnergy()}\spxextra{eqtools.AUGData.AUGDDData method}}

\begin{fulllineitems}
\phantomsection\label{\detokenize{eqtools:eqtools.AUGData.AUGDDData.getEnergy}}\pysiglinewithargsret{\sphinxbfcode{\sphinxupquote{getEnergy}}}{}{}
pulls the calculated energy parameters - stored energy, tau\_E,
injected power, d/dt of magnetic and plasma stored energy.
\begin{quote}\begin{description}
\item[{Raises}] \leavevmode
\sphinxstyleliteralstrong{\sphinxupquote{NotImplementedError}} \textendash{} Not implemented on ASDEX-Upgrade reconstructions.

\end{description}\end{quote}

\end{fulllineitems}

\index{getMachineCrossSection() (eqtools.AUGData.AUGDDData method)@\spxentry{getMachineCrossSection()}\spxextra{eqtools.AUGData.AUGDDData method}}

\begin{fulllineitems}
\phantomsection\label{\detokenize{eqtools:eqtools.AUGData.AUGDDData.getMachineCrossSection}}\pysiglinewithargsret{\sphinxbfcode{\sphinxupquote{getMachineCrossSection}}}{}{}
Returns R,Z coordinates of vacuum-vessel wall for masking, plotting
routines.
\begin{quote}\begin{description}
\item[{Returns}] \leavevmode

(\sphinxtitleref{R\_limiter}, \sphinxtitleref{Z\_limiter})
\begin{itemize}
\item {} 
\sphinxstylestrong{R\_limiter} (\sphinxtitleref{Array}) - {[}n{]} array of x-values for machine cross-section.

\item {} 
\sphinxstylestrong{Z\_limiter} (\sphinxtitleref{Array}) - {[}n{]} array of y-values for machine cross-section.

\end{itemize}


\end{description}\end{quote}

\end{fulllineitems}

\index{getMachineCrossSectionFull() (eqtools.AUGData.AUGDDData method)@\spxentry{getMachineCrossSectionFull()}\spxextra{eqtools.AUGData.AUGDDData method}}

\begin{fulllineitems}
\phantomsection\label{\detokenize{eqtools:eqtools.AUGData.AUGDDData.getMachineCrossSectionFull}}\pysiglinewithargsret{\sphinxbfcode{\sphinxupquote{getMachineCrossSectionFull}}}{}{}
Returns R,Z coordinates of vacuum-vessel wall for plotting routines.

Absent additional vector-graphic data on machine cross-section, returns
{\hyperref[\detokenize{eqtools:eqtools.AUGData.AUGDDData.getMachineCrossSection}]{\sphinxcrossref{\sphinxcode{\sphinxupquote{getMachineCrossSection()}}}}}.
\begin{quote}\begin{description}
\item[{Returns}] \leavevmode
result from getMachineCrossSection().

\end{description}\end{quote}

\end{fulllineitems}

\index{getCurrentSign() (eqtools.AUGData.AUGDDData method)@\spxentry{getCurrentSign()}\spxextra{eqtools.AUGData.AUGDDData method}}

\begin{fulllineitems}
\phantomsection\label{\detokenize{eqtools:eqtools.AUGData.AUGDDData.getCurrentSign}}\pysiglinewithargsret{\sphinxbfcode{\sphinxupquote{getCurrentSign}}}{}{}
Returns the sign of the current, based on the check in Steve Wolfe’s
IDL implementation efit\_rz2psi.pro.
\begin{quote}\begin{description}
\item[{Returns}] \leavevmode
1 for positive-direction current, -1 for negative.

\item[{Return type}] \leavevmode
currentSign (Integer)

\end{description}\end{quote}

\end{fulllineitems}

\index{getParam() (eqtools.AUGData.AUGDDData method)@\spxentry{getParam()}\spxextra{eqtools.AUGData.AUGDDData method}}

\begin{fulllineitems}
\phantomsection\label{\detokenize{eqtools:eqtools.AUGData.AUGDDData.getParam}}\pysiglinewithargsret{\sphinxbfcode{\sphinxupquote{getParam}}}{\emph{path}}{}
Backup function, applying a direct path input for tree-like data
storage access for parameters not typically found in
\sphinxcode{\sphinxupquote{Equilbrium}} object.
Directly calls attributes read from g/a-files in copy-safe manner.
\begin{quote}\begin{description}
\item[{Parameters}] \leavevmode
\sphinxstyleliteralstrong{\sphinxupquote{name}} (\sphinxstyleliteralemphasis{\sphinxupquote{String}}) \textendash{} Parameter name for value stored in EqdskReader
instance.

\item[{Raises}] \leavevmode
\sphinxstyleliteralstrong{\sphinxupquote{NotImplementedError}} \textendash{} Not implemented on ASDEX-Upgrade reconstructions.

\end{description}\end{quote}

\end{fulllineitems}

\index{getSSQ() (eqtools.AUGData.AUGDDData method)@\spxentry{getSSQ()}\spxextra{eqtools.AUGData.AUGDDData method}}

\begin{fulllineitems}
\phantomsection\label{\detokenize{eqtools:eqtools.AUGData.AUGDDData.getSSQ}}\pysiglinewithargsret{\sphinxbfcode{\sphinxupquote{getSSQ}}}{\emph{inp}, \emph{**kwargs}}{}
returns single value quantities in the case SV file doesn’t exist
and coniditions the data in a way that is expected from a dd SV
shotfile. This seamlessly hides the lack of an SV file.
\begin{quote}\begin{description}
\item[{Returns}] \leavevmode
corresponding data

\item[{Return type}] \leavevmode
signal (dd.signal Object)

\item[{Raises}] \leavevmode
\sphinxstyleliteralstrong{\sphinxupquote{ValueError}} \textendash{} if module cannot retrieve data from the AUG AFS system.

\end{description}\end{quote}

\end{fulllineitems}

\index{rz2BR() (eqtools.AUGData.AUGDDData method)@\spxentry{rz2BR()}\spxextra{eqtools.AUGData.AUGDDData method}}

\begin{fulllineitems}
\phantomsection\label{\detokenize{eqtools:eqtools.AUGData.AUGDDData.rz2BR}}\pysiglinewithargsret{\sphinxbfcode{\sphinxupquote{rz2BR}}}{\emph{R}, \emph{Z}, \emph{t}, \emph{return\_t=False}, \emph{make\_grid=False}, \emph{each\_t=True}, \emph{length\_unit=1}}{}
Calculates the major radial component of the magnetic field at the given (R, Z, t) coordinates.

Uses
\begin{equation*}
\begin{split}B_R = -\frac{1}{2 \pi R}\frac{\partial \psi}{\partial Z}\end{split}
\end{equation*}\begin{quote}\begin{description}
\item[{Parameters}] \leavevmode\begin{itemize}
\item {} 
\sphinxstyleliteralstrong{\sphinxupquote{R}} (\sphinxstyleliteralemphasis{\sphinxupquote{Array-like}}\sphinxstyleliteralemphasis{\sphinxupquote{ or }}\sphinxstyleliteralemphasis{\sphinxupquote{scalar float}}) \textendash{} Values of the radial coordinate to
map to radial field. If \sphinxtitleref{R} and \sphinxtitleref{Z} are both scalar values,
they are used as the coordinate pair for all of the values in
\sphinxtitleref{t}. Must have the same shape as \sphinxtitleref{Z} unless the \sphinxtitleref{make\_grid}
keyword is set. If the \sphinxtitleref{make\_grid} keyword is True, \sphinxtitleref{R} must
have exactly one dimension.

\item {} 
\sphinxstyleliteralstrong{\sphinxupquote{Z}} (\sphinxstyleliteralemphasis{\sphinxupquote{Array-like}}\sphinxstyleliteralemphasis{\sphinxupquote{ or }}\sphinxstyleliteralemphasis{\sphinxupquote{scalar float}}) \textendash{} Values of the vertical coordinate to
map to radial field. If \sphinxtitleref{R} and \sphinxtitleref{Z} are both scalar values,
they are used as the coordinate pair for all of the values in
\sphinxtitleref{t}. Must have the same shape as \sphinxtitleref{R} unless the \sphinxtitleref{make\_grid}
keyword is set. If the \sphinxtitleref{make\_grid} keyword is True, \sphinxtitleref{Z} must
have exactly one dimension.

\item {} 
\sphinxstyleliteralstrong{\sphinxupquote{t}} (\sphinxstyleliteralemphasis{\sphinxupquote{Array-like}}\sphinxstyleliteralemphasis{\sphinxupquote{ or }}\sphinxstyleliteralemphasis{\sphinxupquote{scalar float}}) \textendash{} Times to perform the conversion at.
If \sphinxtitleref{t} is a single value, it is used for all of the elements of
\sphinxtitleref{R}, \sphinxtitleref{Z}. If the \sphinxtitleref{each\_t} keyword is True, then \sphinxtitleref{t} must be
scalar or have exactly one dimension. If the \sphinxtitleref{each\_t} keyword is
False, \sphinxtitleref{t} must have the same shape as \sphinxtitleref{R} and \sphinxtitleref{Z} (or their
meshgrid if \sphinxtitleref{make\_grid} is True).

\end{itemize}

\item[{Keyword Arguments}] \leavevmode\begin{itemize}
\item {} 
\sphinxstyleliteralstrong{\sphinxupquote{each\_t}} (\sphinxstyleliteralemphasis{\sphinxupquote{Boolean}}) \textendash{} When True, the elements in \sphinxtitleref{R}, \sphinxtitleref{Z} are evaluated
at each value in \sphinxtitleref{t}. If True, \sphinxtitleref{t} must have only one dimension
(or be a scalar). If False, \sphinxtitleref{t} must match the shape of \sphinxtitleref{R} and
\sphinxtitleref{Z} or be a scalar. Default is True (evaluate ALL \sphinxtitleref{R}, \sphinxtitleref{Z} at
EACH element in \sphinxtitleref{t}).

\item {} 
\sphinxstyleliteralstrong{\sphinxupquote{make\_grid}} (\sphinxstyleliteralemphasis{\sphinxupquote{Boolean}}) \textendash{} Set to True to pass \sphinxtitleref{R} and \sphinxtitleref{Z} through
\sphinxcode{\sphinxupquote{scipy.meshgrid()}} before evaluating. If this is set to
True, \sphinxtitleref{R} and \sphinxtitleref{Z} must each only have a single dimension, but
can have different lengths. Default is False (do not form
meshgrid).

\item {} 
\sphinxstyleliteralstrong{\sphinxupquote{length\_unit}} (\sphinxstyleliteralemphasis{\sphinxupquote{String}}\sphinxstyleliteralemphasis{\sphinxupquote{ or }}\sphinxstyleliteralemphasis{\sphinxupquote{1}}) \textendash{} 
Length unit that \sphinxtitleref{R}, \sphinxtitleref{Z} are given in.
If a string is given, it must be a valid unit specifier:
\begin{quote}


\begin{savenotes}\sphinxattablestart
\centering
\begin{tabulary}{\linewidth}[t]{|T|T|}
\hline

’m’
&
meters
\\
\hline
’cm’
&
centimeters
\\
\hline
’mm’
&
millimeters
\\
\hline
’in’
&
inches
\\
\hline
’ft’
&
feet
\\
\hline
’yd’
&
yards
\\
\hline
’smoot’
&
smoots
\\
\hline
’cubit’
&
cubits
\\
\hline
’hand’
&
hands
\\
\hline
’default’
&
meters
\\
\hline
\end{tabulary}
\par
\sphinxattableend\end{savenotes}
\end{quote}

If length\_unit is 1 or None, meters are assumed. The default
value is 1 (use meters).


\item {} 
\sphinxstyleliteralstrong{\sphinxupquote{return\_t}} (\sphinxstyleliteralemphasis{\sphinxupquote{Boolean}}) \textendash{} Set to True to return a tuple of (\sphinxtitleref{BR},
\sphinxtitleref{time\_idxs}), where \sphinxtitleref{time\_idxs} is the array of time indices
actually used in evaluating \sphinxtitleref{BR} with nearest-neighbor
interpolation. (This is mostly present as an internal helper.)
Default is False (only return \sphinxtitleref{BR}).

\end{itemize}

\item[{Returns}] \leavevmode

\sphinxtitleref{BR} or (\sphinxtitleref{BR}, \sphinxtitleref{time\_idxs})
\begin{itemize}
\item {} 
\sphinxstylestrong{BR} (\sphinxtitleref{Array or scalar float}) - The major radial component of
the magnetic field. If all of the input arguments are scalar, then
a scalar is returned. Otherwise, a scipy Array is returned. If \sphinxtitleref{R}
and \sphinxtitleref{Z} both have the same shape then \sphinxtitleref{BR} has this shape as well,
unless the \sphinxtitleref{make\_grid} keyword was True, in which case \sphinxtitleref{BR} has
shape (len(\sphinxtitleref{Z}), len(\sphinxtitleref{R})).

\item {} 
\sphinxstylestrong{time\_idxs} (Array with same shape as \sphinxtitleref{BR}) - The indices
(in \sphinxcode{\sphinxupquote{self.getTimeBase()}}) that were used for
nearest-neighbor interpolation. Only returned if \sphinxtitleref{return\_t} is
True.

\end{itemize}


\end{description}\end{quote}
\subsubsection*{Examples}

All assume that \sphinxtitleref{Eq\_instance} is a valid instance of the appropriate
extension of the \sphinxcode{\sphinxupquote{Equilibrium}} abstract class.

Find single BR value at R=0.6m, Z=0.0m, t=0.26s:

\begin{sphinxVerbatim}[commandchars=\\\{\}]
\PYG{n}{BR\PYGZus{}val} \PYG{o}{=} \PYG{n}{Eq\PYGZus{}instance}\PYG{o}{.}\PYG{n}{rz2BR}\PYG{p}{(}\PYG{l+m+mf}{0.6}\PYG{p}{,} \PYG{l+m+mi}{0}\PYG{p}{,} \PYG{l+m+mf}{0.26}\PYG{p}{)}
\end{sphinxVerbatim}

Find BR values at (R, Z) points (0.6m, 0m) and (0.8m, 0m) at the
single time t=0.26s. Note that the \sphinxtitleref{Z} vector must be fully
specified, even if the values are all the same:

\begin{sphinxVerbatim}[commandchars=\\\{\}]
\PYG{n}{BR\PYGZus{}arr} \PYG{o}{=} \PYG{n}{Eq\PYGZus{}instance}\PYG{o}{.}\PYG{n}{rz2BR}\PYG{p}{(}\PYG{p}{[}\PYG{l+m+mf}{0.6}\PYG{p}{,} \PYG{l+m+mf}{0.8}\PYG{p}{]}\PYG{p}{,} \PYG{p}{[}\PYG{l+m+mi}{0}\PYG{p}{,} \PYG{l+m+mi}{0}\PYG{p}{]}\PYG{p}{,} \PYG{l+m+mf}{0.26}\PYG{p}{)}
\end{sphinxVerbatim}

Find BR values at (R, Z) points (0.6m, 0m) at times t={[}0.2s, 0.3s{]}:

\begin{sphinxVerbatim}[commandchars=\\\{\}]
\PYG{n}{BR\PYGZus{}arr} \PYG{o}{=} \PYG{n}{Eq\PYGZus{}instance}\PYG{o}{.}\PYG{n}{rz2BR}\PYG{p}{(}\PYG{l+m+mf}{0.6}\PYG{p}{,} \PYG{l+m+mi}{0}\PYG{p}{,} \PYG{p}{[}\PYG{l+m+mf}{0.2}\PYG{p}{,} \PYG{l+m+mf}{0.3}\PYG{p}{]}\PYG{p}{)}
\end{sphinxVerbatim}

Find BR values at (R, Z, t) points (0.6m, 0m, 0.2s) and
(0.5m, 0.2m, 0.3s):

\begin{sphinxVerbatim}[commandchars=\\\{\}]
\PYG{n}{BR\PYGZus{}arr} \PYG{o}{=} \PYG{n}{Eq\PYGZus{}instance}\PYG{o}{.}\PYG{n}{rz2BR}\PYG{p}{(}\PYG{p}{[}\PYG{l+m+mf}{0.6}\PYG{p}{,} \PYG{l+m+mf}{0.5}\PYG{p}{]}\PYG{p}{,} \PYG{p}{[}\PYG{l+m+mi}{0}\PYG{p}{,} \PYG{l+m+mf}{0.2}\PYG{p}{]}\PYG{p}{,} \PYG{p}{[}\PYG{l+m+mf}{0.2}\PYG{p}{,} \PYG{l+m+mf}{0.3}\PYG{p}{]}\PYG{p}{,} \PYG{n}{each\PYGZus{}t}\PYG{o}{=}\PYG{k+kc}{False}\PYG{p}{)}
\end{sphinxVerbatim}

Find BR values on grid defined by 1D vector of radial positions \sphinxtitleref{R}
and 1D vector of vertical positions \sphinxtitleref{Z} at time t=0.2s:

\begin{sphinxVerbatim}[commandchars=\\\{\}]
\PYG{n}{BR\PYGZus{}mat} \PYG{o}{=} \PYG{n}{Eq\PYGZus{}instance}\PYG{o}{.}\PYG{n}{rz2BR}\PYG{p}{(}\PYG{n}{R}\PYG{p}{,} \PYG{n}{Z}\PYG{p}{,} \PYG{l+m+mf}{0.2}\PYG{p}{,} \PYG{n}{make\PYGZus{}grid}\PYG{o}{=}\PYG{k+kc}{True}\PYG{p}{)}
\end{sphinxVerbatim}

\end{fulllineitems}

\index{rz2BZ() (eqtools.AUGData.AUGDDData method)@\spxentry{rz2BZ()}\spxextra{eqtools.AUGData.AUGDDData method}}

\begin{fulllineitems}
\phantomsection\label{\detokenize{eqtools:eqtools.AUGData.AUGDDData.rz2BZ}}\pysiglinewithargsret{\sphinxbfcode{\sphinxupquote{rz2BZ}}}{\emph{R}, \emph{Z}, \emph{t}, \emph{return\_t=False}, \emph{make\_grid=False}, \emph{each\_t=True}, \emph{length\_unit=1}}{}
Calculates the vertical component of the magnetic field at the given (R, Z, t) coordinates.

Uses
\begin{equation*}
\begin{split}B_Z = \frac{1}{2 \pi R}\frac{\partial \psi}{\partial R}\end{split}
\end{equation*}\begin{quote}\begin{description}
\item[{Parameters}] \leavevmode\begin{itemize}
\item {} 
\sphinxstyleliteralstrong{\sphinxupquote{R}} (\sphinxstyleliteralemphasis{\sphinxupquote{Array-like}}\sphinxstyleliteralemphasis{\sphinxupquote{ or }}\sphinxstyleliteralemphasis{\sphinxupquote{scalar float}}) \textendash{} Values of the radial coordinate to
map to vertical field. If \sphinxtitleref{R} and \sphinxtitleref{Z} are both scalar values,
they are used as the coordinate pair for all of the values in
\sphinxtitleref{t}. Must have the same shape as \sphinxtitleref{Z} unless the \sphinxtitleref{make\_grid}
keyword is set. If the \sphinxtitleref{make\_grid} keyword is True, \sphinxtitleref{R} must
have exactly one dimension.

\item {} 
\sphinxstyleliteralstrong{\sphinxupquote{Z}} (\sphinxstyleliteralemphasis{\sphinxupquote{Array-like}}\sphinxstyleliteralemphasis{\sphinxupquote{ or }}\sphinxstyleliteralemphasis{\sphinxupquote{scalar float}}) \textendash{} Values of the vertical coordinate to
map to vertical field. If \sphinxtitleref{R} and \sphinxtitleref{Z} are both scalar values,
they are used as the coordinate pair for all of the values in
\sphinxtitleref{t}. Must have the same shape as \sphinxtitleref{R} unless the \sphinxtitleref{make\_grid}
keyword is set. If the \sphinxtitleref{make\_grid} keyword is True, \sphinxtitleref{Z} must
have exactly one dimension.

\item {} 
\sphinxstyleliteralstrong{\sphinxupquote{t}} (\sphinxstyleliteralemphasis{\sphinxupquote{Array-like}}\sphinxstyleliteralemphasis{\sphinxupquote{ or }}\sphinxstyleliteralemphasis{\sphinxupquote{scalar float}}) \textendash{} Times to perform the conversion at.
If \sphinxtitleref{t} is a single value, it is used for all of the elements of
\sphinxtitleref{R}, \sphinxtitleref{Z}. If the \sphinxtitleref{each\_t} keyword is True, then \sphinxtitleref{t} must be
scalar or have exactly one dimension. If the \sphinxtitleref{each\_t} keyword is
False, \sphinxtitleref{t} must have the same shape as \sphinxtitleref{R} and \sphinxtitleref{Z} (or their
meshgrid if \sphinxtitleref{make\_grid} is True).

\end{itemize}

\item[{Keyword Arguments}] \leavevmode\begin{itemize}
\item {} 
\sphinxstyleliteralstrong{\sphinxupquote{each\_t}} (\sphinxstyleliteralemphasis{\sphinxupquote{Boolean}}) \textendash{} When True, the elements in \sphinxtitleref{R}, \sphinxtitleref{Z} are evaluated
at each value in \sphinxtitleref{t}. If True, \sphinxtitleref{t} must have only one dimension
(or be a scalar). If False, \sphinxtitleref{t} must match the shape of \sphinxtitleref{R} and
\sphinxtitleref{Z} or be a scalar. Default is True (evaluate ALL \sphinxtitleref{R}, \sphinxtitleref{Z} at
EACH element in \sphinxtitleref{t}).

\item {} 
\sphinxstyleliteralstrong{\sphinxupquote{make\_grid}} (\sphinxstyleliteralemphasis{\sphinxupquote{Boolean}}) \textendash{} Set to True to pass \sphinxtitleref{R} and \sphinxtitleref{Z} through
\sphinxcode{\sphinxupquote{scipy.meshgrid()}} before evaluating. If this is set to
True, \sphinxtitleref{R} and \sphinxtitleref{Z} must each only have a single dimension, but
can have different lengths. Default is False (do not form
meshgrid).

\item {} 
\sphinxstyleliteralstrong{\sphinxupquote{length\_unit}} (\sphinxstyleliteralemphasis{\sphinxupquote{String}}\sphinxstyleliteralemphasis{\sphinxupquote{ or }}\sphinxstyleliteralemphasis{\sphinxupquote{1}}) \textendash{} 
Length unit that \sphinxtitleref{R}, \sphinxtitleref{Z} are given in.
If a string is given, it must be a valid unit specifier:
\begin{quote}


\begin{savenotes}\sphinxattablestart
\centering
\begin{tabulary}{\linewidth}[t]{|T|T|}
\hline

’m’
&
meters
\\
\hline
’cm’
&
centimeters
\\
\hline
’mm’
&
millimeters
\\
\hline
’in’
&
inches
\\
\hline
’ft’
&
feet
\\
\hline
’yd’
&
yards
\\
\hline
’smoot’
&
smoots
\\
\hline
’cubit’
&
cubits
\\
\hline
’hand’
&
hands
\\
\hline
’default’
&
meters
\\
\hline
\end{tabulary}
\par
\sphinxattableend\end{savenotes}
\end{quote}

If length\_unit is 1 or None, meters are assumed. The default
value is 1 (use meters).


\item {} 
\sphinxstyleliteralstrong{\sphinxupquote{return\_t}} (\sphinxstyleliteralemphasis{\sphinxupquote{Boolean}}) \textendash{} Set to True to return a tuple of (\sphinxtitleref{BZ},
\sphinxtitleref{time\_idxs}), where \sphinxtitleref{time\_idxs} is the array of time indices
actually used in evaluating \sphinxtitleref{BZ} with nearest-neighbor
interpolation. (This is mostly present as an internal helper.)
Default is False (only return \sphinxtitleref{BZ}).

\end{itemize}

\item[{Returns}] \leavevmode

\sphinxtitleref{BZ} or (\sphinxtitleref{BZ}, \sphinxtitleref{time\_idxs})
\begin{itemize}
\item {} 
\sphinxstylestrong{BZ} (\sphinxtitleref{Array or scalar float}) - The vertical component of the
magnetic field. If all of the input arguments are scalar, then a
scalar is returned. Otherwise, a scipy Array is returned. If \sphinxtitleref{R}
and \sphinxtitleref{Z} both have the same shape then \sphinxtitleref{BZ} has this shape as well,
unless the \sphinxtitleref{make\_grid} keyword was True, in which case \sphinxtitleref{BZ} has
shape (len(\sphinxtitleref{Z}), len(\sphinxtitleref{R})).

\item {} 
\sphinxstylestrong{time\_idxs} (Array with same shape as \sphinxtitleref{BZ}) - The indices
(in \sphinxcode{\sphinxupquote{self.getTimeBase()}}) that were used for
nearest-neighbor interpolation. Only returned if \sphinxtitleref{return\_t} is
True.

\end{itemize}


\end{description}\end{quote}
\subsubsection*{Examples}

All assume that \sphinxtitleref{Eq\_instance} is a valid instance of the appropriate
extension of the \sphinxcode{\sphinxupquote{Equilibrium}} abstract class.

Find single BZ value at R=0.6m, Z=0.0m, t=0.26s:

\begin{sphinxVerbatim}[commandchars=\\\{\}]
\PYG{n}{BZ\PYGZus{}val} \PYG{o}{=} \PYG{n}{Eq\PYGZus{}instance}\PYG{o}{.}\PYG{n}{rz2BZ}\PYG{p}{(}\PYG{l+m+mf}{0.6}\PYG{p}{,} \PYG{l+m+mi}{0}\PYG{p}{,} \PYG{l+m+mf}{0.26}\PYG{p}{)}
\end{sphinxVerbatim}

Find BZ values at (R, Z) points (0.6m, 0m) and (0.8m, 0m) at the
single time t=0.26s. Note that the \sphinxtitleref{Z} vector must be fully
specified, even if the values are all the same:

\begin{sphinxVerbatim}[commandchars=\\\{\}]
\PYG{n}{BZ\PYGZus{}arr} \PYG{o}{=} \PYG{n}{Eq\PYGZus{}instance}\PYG{o}{.}\PYG{n}{rz2BZ}\PYG{p}{(}\PYG{p}{[}\PYG{l+m+mf}{0.6}\PYG{p}{,} \PYG{l+m+mf}{0.8}\PYG{p}{]}\PYG{p}{,} \PYG{p}{[}\PYG{l+m+mi}{0}\PYG{p}{,} \PYG{l+m+mi}{0}\PYG{p}{]}\PYG{p}{,} \PYG{l+m+mf}{0.26}\PYG{p}{)}
\end{sphinxVerbatim}

Find BZ values at (R, Z) points (0.6m, 0m) at times t={[}0.2s, 0.3s{]}:

\begin{sphinxVerbatim}[commandchars=\\\{\}]
\PYG{n}{BZ\PYGZus{}arr} \PYG{o}{=} \PYG{n}{Eq\PYGZus{}instance}\PYG{o}{.}\PYG{n}{rz2BZ}\PYG{p}{(}\PYG{l+m+mf}{0.6}\PYG{p}{,} \PYG{l+m+mi}{0}\PYG{p}{,} \PYG{p}{[}\PYG{l+m+mf}{0.2}\PYG{p}{,} \PYG{l+m+mf}{0.3}\PYG{p}{]}\PYG{p}{)}
\end{sphinxVerbatim}

Find BZ values at (R, Z, t) points (0.6m, 0m, 0.2s) and
(0.5m, 0.2m, 0.3s):

\begin{sphinxVerbatim}[commandchars=\\\{\}]
\PYG{n}{BZ\PYGZus{}arr} \PYG{o}{=} \PYG{n}{Eq\PYGZus{}instance}\PYG{o}{.}\PYG{n}{rz2BZ}\PYG{p}{(}\PYG{p}{[}\PYG{l+m+mf}{0.6}\PYG{p}{,} \PYG{l+m+mf}{0.5}\PYG{p}{]}\PYG{p}{,} \PYG{p}{[}\PYG{l+m+mi}{0}\PYG{p}{,} \PYG{l+m+mf}{0.2}\PYG{p}{]}\PYG{p}{,} \PYG{p}{[}\PYG{l+m+mf}{0.2}\PYG{p}{,} \PYG{l+m+mf}{0.3}\PYG{p}{]}\PYG{p}{,} \PYG{n}{each\PYGZus{}t}\PYG{o}{=}\PYG{k+kc}{False}\PYG{p}{)}
\end{sphinxVerbatim}

Find BZ values on grid defined by 1D vector of radial positions \sphinxtitleref{R}
and 1D vector of vertical positions \sphinxtitleref{Z} at time t=0.2s:

\begin{sphinxVerbatim}[commandchars=\\\{\}]
\PYG{n}{BZ\PYGZus{}mat} \PYG{o}{=} \PYG{n}{Eq\PYGZus{}instance}\PYG{o}{.}\PYG{n}{rz2BZ}\PYG{p}{(}\PYG{n}{R}\PYG{p}{,} \PYG{n}{Z}\PYG{p}{,} \PYG{l+m+mf}{0.2}\PYG{p}{,} \PYG{n}{make\PYGZus{}grid}\PYG{o}{=}\PYG{k+kc}{True}\PYG{p}{)}
\end{sphinxVerbatim}

\end{fulllineitems}


\end{fulllineitems}

\index{YGCAUGInterface (class in eqtools.AUGData)@\spxentry{YGCAUGInterface}\spxextra{class in eqtools.AUGData}}

\begin{fulllineitems}
\phantomsection\label{\detokenize{eqtools:eqtools.AUGData.YGCAUGInterface}}\pysigline{\sphinxbfcode{\sphinxupquote{class }}\sphinxcode{\sphinxupquote{eqtools.AUGData.}}\sphinxbfcode{\sphinxupquote{YGCAUGInterface}}}
Bases: \sphinxcode{\sphinxupquote{object}}
\index{getMachineCrossSection() (eqtools.AUGData.YGCAUGInterface method)@\spxentry{getMachineCrossSection()}\spxextra{eqtools.AUGData.YGCAUGInterface method}}

\begin{fulllineitems}
\phantomsection\label{\detokenize{eqtools:eqtools.AUGData.YGCAUGInterface.getMachineCrossSection}}\pysiglinewithargsret{\sphinxbfcode{\sphinxupquote{getMachineCrossSection}}}{\emph{shot}}{}
Returns R,Z coordinates of vacuum-vessel wall for masking, plotting
routines.
\begin{quote}\begin{description}
\item[{Returns}] \leavevmode

(\sphinxtitleref{R\_limiter}, \sphinxtitleref{Z\_limiter})
\begin{itemize}
\item {} 
\sphinxstylestrong{R\_limiter} (\sphinxtitleref{Array}) - {[}n{]} array of x-values for machine cross-section.

\item {} 
\sphinxstylestrong{Z\_limiter} (\sphinxtitleref{Array}) - {[}n{]} array of y-values for machine cross-section.

\end{itemize}


\end{description}\end{quote}

\end{fulllineitems}

\index{getMachineCrossSectionFull() (eqtools.AUGData.YGCAUGInterface method)@\spxentry{getMachineCrossSectionFull()}\spxextra{eqtools.AUGData.YGCAUGInterface method}}

\begin{fulllineitems}
\phantomsection\label{\detokenize{eqtools:eqtools.AUGData.YGCAUGInterface.getMachineCrossSectionFull}}\pysiglinewithargsret{\sphinxbfcode{\sphinxupquote{getMachineCrossSectionFull}}}{\emph{shot}}{}
Returns R,Z coordinates of vacuum-vessel wall for plotting routines.

Absent additional vector-graphic data on machine cross-section, returns
{\hyperref[\detokenize{eqtools:eqtools.AUGData.YGCAUGInterface.getMachineCrossSection}]{\sphinxcrossref{\sphinxcode{\sphinxupquote{getMachineCrossSection()}}}}}.
\begin{quote}\begin{description}
\item[{Returns}] \leavevmode
result from getMachineCrossSection().

\end{description}\end{quote}

\end{fulllineitems}


\end{fulllineitems}

\index{AUGDDDataProp (class in eqtools.AUGData)@\spxentry{AUGDDDataProp}\spxextra{class in eqtools.AUGData}}

\begin{fulllineitems}
\phantomsection\label{\detokenize{eqtools:eqtools.AUGData.AUGDDDataProp}}\pysiglinewithargsret{\sphinxbfcode{\sphinxupquote{class }}\sphinxcode{\sphinxupquote{eqtools.AUGData.}}\sphinxbfcode{\sphinxupquote{AUGDDDataProp}}}{\emph{shot}, \emph{shotfile='EQH'}, \emph{edition=0}, \emph{shotfile2=None}, \emph{length\_unit='m'}, \emph{tspline=False}, \emph{monotonic=True}, \emph{experiment='AUGD'}}{}
Bases: {\hyperref[\detokenize{eqtools:eqtools.AUGData.AUGDDData}]{\sphinxcrossref{\sphinxcode{\sphinxupquote{eqtools.AUGData.AUGDDData}}}}}, {\hyperref[\detokenize{eqtools:eqtools.core.PropertyAccessMixin}]{\sphinxcrossref{\sphinxcode{\sphinxupquote{eqtools.core.PropertyAccessMixin}}}}}

AUGDDData with the PropertyAccessMixin added to enable property-style
access. This is good for interactive use, but may drag the performance down.

\end{fulllineitems}



\subsection{eqtools.CModEFIT module}
\label{\detokenize{eqtools:module-eqtools.CModEFIT}}\label{\detokenize{eqtools:eqtools-cmodefit-module}}\index{eqtools.CModEFIT (module)@\spxentry{eqtools.CModEFIT}\spxextra{module}}
This module provides classes inheriting {\hyperref[\detokenize{eqtools:eqtools.EFIT.EFITTree}]{\sphinxcrossref{\sphinxcode{\sphinxupquote{eqtools.EFIT.EFITTree}}}}} for
working with C-Mod EFIT data.
\index{CModEFITTree (class in eqtools.CModEFIT)@\spxentry{CModEFITTree}\spxextra{class in eqtools.CModEFIT}}

\begin{fulllineitems}
\phantomsection\label{\detokenize{eqtools:eqtools.CModEFIT.CModEFITTree}}\pysiglinewithargsret{\sphinxbfcode{\sphinxupquote{class }}\sphinxcode{\sphinxupquote{eqtools.CModEFIT.}}\sphinxbfcode{\sphinxupquote{CModEFITTree}}}{\emph{shot}, \emph{tree='ANALYSIS'}, \emph{length\_unit='m'}, \emph{gfile='g\_eqdsk'}, \emph{afile='a\_eqdsk'}, \emph{tspline=False}, \emph{monotonic=True}}{}
Bases: {\hyperref[\detokenize{eqtools:eqtools.EFIT.EFITTree}]{\sphinxcrossref{\sphinxcode{\sphinxupquote{eqtools.EFIT.EFITTree}}}}}

Inherits {\hyperref[\detokenize{eqtools:eqtools.EFIT.EFITTree}]{\sphinxcrossref{\sphinxcode{\sphinxupquote{eqtools.EFIT.EFITTree}}}}} class. Machine-specific data
handling class for Alcator C-Mod. Pulls EFIT data from selected MDS tree
and shot, stores as object attributes. Each EFIT variable or set of
variables is recovered with a corresponding getter method. Essential data
for EFIT mapping are pulled on initialization (e.g. psirz grid). Additional
data are pulled at the first request and stored for subsequent usage.

Intializes C-Mod version of EFITTree object.  Pulls data from MDS tree for
storage in instance attributes.  Core attributes are populated from the MDS
tree on initialization.  Additional attributes are initialized as None,
filled on the first request to the object.
\begin{quote}\begin{description}
\item[{Parameters}] \leavevmode
\sphinxstyleliteralstrong{\sphinxupquote{shot}} (\sphinxstyleliteralemphasis{\sphinxupquote{integer}}) \textendash{} C-Mod shot index.

\item[{Keyword Arguments}] \leavevmode\begin{itemize}
\item {} 
\sphinxstyleliteralstrong{\sphinxupquote{tree}} (\sphinxstyleliteralemphasis{\sphinxupquote{string}}) \textendash{} Optional input for EFIT tree, defaults to ‘ANALYSIS’
(i.e., EFIT data are under analysis::top.efit.results).
For any string TREE (such as ‘EFIT20’) other than ‘ANALYSIS’,
data are taken from TREE::top.results.

\item {} 
\sphinxstyleliteralstrong{\sphinxupquote{length\_unit}} (\sphinxstyleliteralemphasis{\sphinxupquote{string}}) \textendash{} 
Sets the base unit used for any quantity whose
dimensions are length to any power. Valid options are:
\begin{quote}


\begin{savenotes}\sphinxattablestart
\centering
\begin{tabulary}{\linewidth}[t]{|T|T|}
\hline

’m’
&
meters
\\
\hline
’cm’
&
centimeters
\\
\hline
’mm’
&
millimeters
\\
\hline
’in’
&
inches
\\
\hline
’ft’
&
feet
\\
\hline
’yd’
&
yards
\\
\hline
’smoot’
&
smoots
\\
\hline
’cubit’
&
cubits
\\
\hline
’hand’
&
hands
\\
\hline
’default’
&
whatever the default in the tree is (no conversion is performed, units may be inconsistent)
\\
\hline
\end{tabulary}
\par
\sphinxattableend\end{savenotes}
\end{quote}

Default is ‘m’ (all units taken and returned in meters).


\item {} 
\sphinxstyleliteralstrong{\sphinxupquote{gfile}} (\sphinxstyleliteralemphasis{\sphinxupquote{string}}) \textendash{} Optional input for EFIT geqdsk location name,
defaults to ‘g\_eqdsk’ (i.e., EFIT data are under
tree::top.results.G\_EQDSK)

\item {} 
\sphinxstyleliteralstrong{\sphinxupquote{afile}} (\sphinxstyleliteralemphasis{\sphinxupquote{string}}) \textendash{} Optional input for EFIT aeqdsk location name,
defaults to ‘a\_eqdsk’ (i.e., EFIT data are under
tree::top.results.A\_EQDSK)

\item {} 
\sphinxstyleliteralstrong{\sphinxupquote{tspline}} (\sphinxstyleliteralemphasis{\sphinxupquote{Boolean}}) \textendash{} Sets whether or not interpolation in time is
performed using a tricubic spline or nearest-neighbor
interpolation. Tricubic spline interpolation requires at least
four complete equilibria at different times. It is also assumed
that they are functionally correlated, and that parameters do
not vary out of their boundaries (derivative = 0 boundary
condition). Default is False (use nearest neighbor interpolation).

\item {} 
\sphinxstyleliteralstrong{\sphinxupquote{monotonic}} (\sphinxstyleliteralemphasis{\sphinxupquote{Boolean}}) \textendash{} Sets whether or not the “monotonic” form of time
window finding is used. If True, the timebase must be monotonically
increasing. Default is False (use slower, safer method).

\end{itemize}

\end{description}\end{quote}
\index{getFluxVol() (eqtools.CModEFIT.CModEFITTree method)@\spxentry{getFluxVol()}\spxextra{eqtools.CModEFIT.CModEFITTree method}}

\begin{fulllineitems}
\phantomsection\label{\detokenize{eqtools:eqtools.CModEFIT.CModEFITTree.getFluxVol}}\pysiglinewithargsret{\sphinxbfcode{\sphinxupquote{getFluxVol}}}{\emph{length\_unit=3}}{}
returns volume within flux surface.
\begin{quote}\begin{description}
\item[{Keyword Arguments}] \leavevmode
\sphinxstyleliteralstrong{\sphinxupquote{length\_unit}} (\sphinxstyleliteralemphasis{\sphinxupquote{String}}\sphinxstyleliteralemphasis{\sphinxupquote{ or }}\sphinxstyleliteralemphasis{\sphinxupquote{3}}) \textendash{} unit for plasma volume.  Defaults to 3,
indicating default volumetric unit (typically m\textasciicircum{}3).

\item[{Returns}] \leavevmode
{[}nt,npsi{]} array of volume within flux surface.

\item[{Return type}] \leavevmode
fluxVol (Array)

\item[{Raises}] \leavevmode
\sphinxstyleliteralstrong{\sphinxupquote{ValueError}} \textendash{} if module cannot retrieve data from MDS tree.

\end{description}\end{quote}

\end{fulllineitems}

\index{getRmidPsi() (eqtools.CModEFIT.CModEFITTree method)@\spxentry{getRmidPsi()}\spxextra{eqtools.CModEFIT.CModEFITTree method}}

\begin{fulllineitems}
\phantomsection\label{\detokenize{eqtools:eqtools.CModEFIT.CModEFITTree.getRmidPsi}}\pysiglinewithargsret{\sphinxbfcode{\sphinxupquote{getRmidPsi}}}{\emph{length\_unit=1}}{}
returns maximum major radius of each flux surface.
\begin{quote}\begin{description}
\item[{Keyword Arguments}] \leavevmode
\sphinxstyleliteralstrong{\sphinxupquote{length\_unit}} (\sphinxstyleliteralemphasis{\sphinxupquote{String}}\sphinxstyleliteralemphasis{\sphinxupquote{ or }}\sphinxstyleliteralemphasis{\sphinxupquote{1}}) \textendash{} unit of Rmid.  Defaults to 1, indicating
the default parameter unit (typically m).

\item[{Returns}] \leavevmode
{[}nt,npsi{]} array of maximum (outboard) major radius of
flux surface psi.

\item[{Return type}] \leavevmode
Rmid (Array)

\item[{Raises}] \leavevmode
\sphinxstyleliteralstrong{\sphinxupquote{Value Error}} \textendash{} if module cannot retrieve data from MDS tree.

\end{description}\end{quote}

\end{fulllineitems}

\index{getF() (eqtools.CModEFIT.CModEFITTree method)@\spxentry{getF()}\spxextra{eqtools.CModEFIT.CModEFITTree method}}

\begin{fulllineitems}
\phantomsection\label{\detokenize{eqtools:eqtools.CModEFIT.CModEFITTree.getF}}\pysiglinewithargsret{\sphinxbfcode{\sphinxupquote{getF}}}{}{}
returns F=RB\_\{Phi\}(Psi), often calculated for grad-shafranov
solutions.
\begin{quote}\begin{description}
\item[{Returns}] \leavevmode
{[}nt,npsi{]} array of F=RB\_\{Phi\}(Psi)

\item[{Return type}] \leavevmode
F (Array)

\item[{Raises}] \leavevmode
\sphinxstyleliteralstrong{\sphinxupquote{ValueError}} \textendash{} if module cannot retrieve data from MDS tree.

\end{description}\end{quote}

\end{fulllineitems}

\index{getFluxPres() (eqtools.CModEFIT.CModEFITTree method)@\spxentry{getFluxPres()}\spxextra{eqtools.CModEFIT.CModEFITTree method}}

\begin{fulllineitems}
\phantomsection\label{\detokenize{eqtools:eqtools.CModEFIT.CModEFITTree.getFluxPres}}\pysiglinewithargsret{\sphinxbfcode{\sphinxupquote{getFluxPres}}}{}{}
returns pressure at flux surface.
\begin{quote}\begin{description}
\item[{Returns}] \leavevmode
{[}nt,npsi{]} array of pressure on flux surface psi.

\item[{Return type}] \leavevmode
p (Array)

\item[{Raises}] \leavevmode
\sphinxstyleliteralstrong{\sphinxupquote{ValueError}} \textendash{} if module cannot retrieve data from MDS tree.

\end{description}\end{quote}

\end{fulllineitems}

\index{getFFPrime() (eqtools.CModEFIT.CModEFITTree method)@\spxentry{getFFPrime()}\spxextra{eqtools.CModEFIT.CModEFITTree method}}

\begin{fulllineitems}
\phantomsection\label{\detokenize{eqtools:eqtools.CModEFIT.CModEFITTree.getFFPrime}}\pysiglinewithargsret{\sphinxbfcode{\sphinxupquote{getFFPrime}}}{}{}
returns FF’ function used for grad-shafranov solutions.
\begin{quote}\begin{description}
\item[{Returns}] \leavevmode
{[}nt,npsi{]} array of FF’ fromgrad-shafranov solution.

\item[{Return type}] \leavevmode
FFprime (Array)

\item[{Raises}] \leavevmode
\sphinxstyleliteralstrong{\sphinxupquote{ValueError}} \textendash{} if module cannot retrieve data from MDS tree.

\end{description}\end{quote}

\end{fulllineitems}

\index{getPPrime() (eqtools.CModEFIT.CModEFITTree method)@\spxentry{getPPrime()}\spxextra{eqtools.CModEFIT.CModEFITTree method}}

\begin{fulllineitems}
\phantomsection\label{\detokenize{eqtools:eqtools.CModEFIT.CModEFITTree.getPPrime}}\pysiglinewithargsret{\sphinxbfcode{\sphinxupquote{getPPrime}}}{}{}
returns plasma pressure gradient as a function of psi.
\begin{quote}\begin{description}
\item[{Returns}] \leavevmode
{[}nt,npsi{]} array of pressure gradient on flux surface
psi from grad-shafranov solution.

\item[{Return type}] \leavevmode
pprime (Array)

\item[{Raises}] \leavevmode
\sphinxstyleliteralstrong{\sphinxupquote{ValueError}} \textendash{} if module cannot retrieve data from MDS tree.

\end{description}\end{quote}

\end{fulllineitems}

\index{getQProfile() (eqtools.CModEFIT.CModEFITTree method)@\spxentry{getQProfile()}\spxextra{eqtools.CModEFIT.CModEFITTree method}}

\begin{fulllineitems}
\phantomsection\label{\detokenize{eqtools:eqtools.CModEFIT.CModEFITTree.getQProfile}}\pysiglinewithargsret{\sphinxbfcode{\sphinxupquote{getQProfile}}}{}{}
returns profile of safety factor q.
\begin{quote}\begin{description}
\item[{Returns}] \leavevmode
{[}nt,npsi{]} array of q on flux surface psi.

\item[{Return type}] \leavevmode
qpsi (Array)

\item[{Raises}] \leavevmode
\sphinxstyleliteralstrong{\sphinxupquote{ValueError}} \textendash{} if module cannot retrieve data from MDS tree.

\end{description}\end{quote}

\end{fulllineitems}

\index{getRLCFS() (eqtools.CModEFIT.CModEFITTree method)@\spxentry{getRLCFS()}\spxextra{eqtools.CModEFIT.CModEFITTree method}}

\begin{fulllineitems}
\phantomsection\label{\detokenize{eqtools:eqtools.CModEFIT.CModEFITTree.getRLCFS}}\pysiglinewithargsret{\sphinxbfcode{\sphinxupquote{getRLCFS}}}{\emph{length\_unit=1}}{}
returns R-values of LCFS position.
\begin{quote}\begin{description}
\item[{Returns}] \leavevmode
{[}nt,n{]} array of R of LCFS points.

\item[{Return type}] \leavevmode
RLCFS (Array)

\item[{Raises}] \leavevmode
\sphinxstyleliteralstrong{\sphinxupquote{ValueError}} \textendash{} if module cannot retrieve data from MDS tree.

\end{description}\end{quote}

\end{fulllineitems}

\index{getZLCFS() (eqtools.CModEFIT.CModEFITTree method)@\spxentry{getZLCFS()}\spxextra{eqtools.CModEFIT.CModEFITTree method}}

\begin{fulllineitems}
\phantomsection\label{\detokenize{eqtools:eqtools.CModEFIT.CModEFITTree.getZLCFS}}\pysiglinewithargsret{\sphinxbfcode{\sphinxupquote{getZLCFS}}}{\emph{length\_unit=1}}{}
returns Z-values of LCFS position.
\begin{quote}\begin{description}
\item[{Returns}] \leavevmode
{[}nt,n{]} array of Z of LCFS points.

\item[{Return type}] \leavevmode
ZLCFS (Array)

\item[{Raises}] \leavevmode
\sphinxstyleliteralstrong{\sphinxupquote{ValueError}} \textendash{} if module cannot retrieve data from MDS tree.

\end{description}\end{quote}

\end{fulllineitems}

\index{getMachineCrossSectionFull() (eqtools.CModEFIT.CModEFITTree method)@\spxentry{getMachineCrossSectionFull()}\spxextra{eqtools.CModEFIT.CModEFITTree method}}

\begin{fulllineitems}
\phantomsection\label{\detokenize{eqtools:eqtools.CModEFIT.CModEFITTree.getMachineCrossSectionFull}}\pysiglinewithargsret{\sphinxbfcode{\sphinxupquote{getMachineCrossSectionFull}}}{}{}
Pulls C-Mod cross-section data from tree, converts to plottable
vector format for use in other plotting routines
\begin{quote}\begin{description}
\item[{Returns}] \leavevmode

(\sphinxtitleref{x}, \sphinxtitleref{y})
\begin{itemize}
\item {} 
\sphinxstylestrong{x} (\sphinxtitleref{Array}) - {[}n{]} array of x-values for machine cross-section.

\item {} 
\sphinxstylestrong{y} (\sphinxtitleref{Array}) - {[}n{]} array of y-values for machine cross-section.

\end{itemize}


\item[{Raises}] \leavevmode
\sphinxstyleliteralstrong{\sphinxupquote{ValueError}} \textendash{} if module cannot retrieve data from MDS tree.

\end{description}\end{quote}

\end{fulllineitems}

\index{getRCentr() (eqtools.CModEFIT.CModEFITTree method)@\spxentry{getRCentr()}\spxextra{eqtools.CModEFIT.CModEFITTree method}}

\begin{fulllineitems}
\phantomsection\label{\detokenize{eqtools:eqtools.CModEFIT.CModEFITTree.getRCentr}}\pysiglinewithargsret{\sphinxbfcode{\sphinxupquote{getRCentr}}}{\emph{length\_unit=1}}{}
returns EFIT radius where Bcentr evaluated
\begin{quote}\begin{description}
\item[{Returns}] \leavevmode
Radial position where Bcent calculated {[}m{]}

\item[{Return type}] \leavevmode
R

\item[{Raises}] \leavevmode
\sphinxstyleliteralstrong{\sphinxupquote{ValueError}} \textendash{} if module cannot retrieve data from MDS tree.

\end{description}\end{quote}

\end{fulllineitems}


\end{fulllineitems}

\index{CModEFITTreeProp (class in eqtools.CModEFIT)@\spxentry{CModEFITTreeProp}\spxextra{class in eqtools.CModEFIT}}

\begin{fulllineitems}
\phantomsection\label{\detokenize{eqtools:eqtools.CModEFIT.CModEFITTreeProp}}\pysiglinewithargsret{\sphinxbfcode{\sphinxupquote{class }}\sphinxcode{\sphinxupquote{eqtools.CModEFIT.}}\sphinxbfcode{\sphinxupquote{CModEFITTreeProp}}}{\emph{shot}, \emph{tree='ANALYSIS'}, \emph{length\_unit='m'}, \emph{gfile='g\_eqdsk'}, \emph{afile='a\_eqdsk'}, \emph{tspline=False}, \emph{monotonic=True}}{}
Bases: {\hyperref[\detokenize{eqtools:eqtools.CModEFIT.CModEFITTree}]{\sphinxcrossref{\sphinxcode{\sphinxupquote{eqtools.CModEFIT.CModEFITTree}}}}}, {\hyperref[\detokenize{eqtools:eqtools.core.PropertyAccessMixin}]{\sphinxcrossref{\sphinxcode{\sphinxupquote{eqtools.core.PropertyAccessMixin}}}}}

CModEFITTree with the PropertyAccessMixin added to enable property-style
access. This is good for interactive use, but may drag the performance down.

\end{fulllineitems}



\subsection{eqtools.D3DEFIT module}
\label{\detokenize{eqtools:module-eqtools.D3DEFIT}}\label{\detokenize{eqtools:eqtools-d3defit-module}}\index{eqtools.D3DEFIT (module)@\spxentry{eqtools.D3DEFIT}\spxextra{module}}
This module provides classes inheriting {\hyperref[\detokenize{eqtools:eqtools.EFIT.EFITTree}]{\sphinxcrossref{\sphinxcode{\sphinxupquote{eqtools.EFIT.EFITTree}}}}} for
working with DIII-D EFIT data.
\index{D3DEFITTree (class in eqtools.D3DEFIT)@\spxentry{D3DEFITTree}\spxextra{class in eqtools.D3DEFIT}}

\begin{fulllineitems}
\phantomsection\label{\detokenize{eqtools:eqtools.D3DEFIT.D3DEFITTree}}\pysiglinewithargsret{\sphinxbfcode{\sphinxupquote{class }}\sphinxcode{\sphinxupquote{eqtools.D3DEFIT.}}\sphinxbfcode{\sphinxupquote{D3DEFITTree}}}{\emph{shot}, \emph{tree='EFIT01'}, \emph{length\_unit='m'}, \emph{gfile='geqdsk'}, \emph{afile='aeqdsk'}, \emph{tspline=False}, \emph{monotonic=True}}{}
Bases: {\hyperref[\detokenize{eqtools:eqtools.EFIT.EFITTree}]{\sphinxcrossref{\sphinxcode{\sphinxupquote{eqtools.EFIT.EFITTree}}}}}

Inherits {\hyperref[\detokenize{eqtools:eqtools.EFIT.EFITTree}]{\sphinxcrossref{\sphinxcode{\sphinxupquote{eqtools.EFIT.EFITTree}}}}} class. Machine-specific data
handling class for DIII-D. Pulls EFIT data from selected MDS tree
and shot, stores as object attributes. Each EFIT variable or set of
variables is recovered with a corresponding getter method. Essential data
for EFIT mapping are pulled on initialization (e.g. psirz grid). Additional
data are pulled at the first request and stored for subsequent usage.

Intializes DIII-D version of EFITTree object.  Pulls data from MDS tree for
storage in instance attributes.  Core attributes are populated from the MDS
tree on initialization.  Additional attributes are initialized as None,
filled on the first request to the object.
\begin{quote}\begin{description}
\item[{Parameters}] \leavevmode
\sphinxstyleliteralstrong{\sphinxupquote{shot}} (\sphinxstyleliteralemphasis{\sphinxupquote{integer}}) \textendash{} DIII-D shot index.

\item[{Keyword Arguments}] \leavevmode\begin{itemize}
\item {} 
\sphinxstyleliteralstrong{\sphinxupquote{tree}} (\sphinxstyleliteralemphasis{\sphinxupquote{string}}) \textendash{} Optional input for EFIT tree, defaults to ‘EFIT01’
(i.e., EFIT data are under EFIT01::top.results).

\item {} 
\sphinxstyleliteralstrong{\sphinxupquote{length\_unit}} (\sphinxstyleliteralemphasis{\sphinxupquote{string}}) \textendash{} 
Sets the base unit used for any quantity whose
dimensions are length to any power. Valid options are:
\begin{quote}


\begin{savenotes}\sphinxattablestart
\centering
\begin{tabulary}{\linewidth}[t]{|T|T|}
\hline

’m’
&
meters
\\
\hline
’cm’
&
centimeters
\\
\hline
’mm’
&
millimeters
\\
\hline
’in’
&
inches
\\
\hline
’ft’
&
feet
\\
\hline
’yd’
&
yards
\\
\hline
’smoot’
&
smoots
\\
\hline
’cubit’
&
cubits
\\
\hline
’hand’
&
hands
\\
\hline
’default’
&
whatever the default in the tree is (no conversion is performed, units may be inconsistent)
\\
\hline
\end{tabulary}
\par
\sphinxattableend\end{savenotes}
\end{quote}

Default is ‘m’ (all units taken and returned in meters).


\item {} 
\sphinxstyleliteralstrong{\sphinxupquote{gfile}} (\sphinxstyleliteralemphasis{\sphinxupquote{string}}) \textendash{} Optional input for EFIT geqdsk location name,
defaults to ‘geqdsk’ (i.e., EFIT data are under
tree::top.results.GEQDSK)

\item {} 
\sphinxstyleliteralstrong{\sphinxupquote{afile}} (\sphinxstyleliteralemphasis{\sphinxupquote{string}}) \textendash{} Optional input for EFIT aeqdsk location name,
defaults to ‘aeqdsk’ (i.e., EFIT data are under
tree::top.results.AEQDSK)

\item {} 
\sphinxstyleliteralstrong{\sphinxupquote{tspline}} (\sphinxstyleliteralemphasis{\sphinxupquote{Boolean}}) \textendash{} Sets whether or not interpolation in time is
performed using a tricubic spline or nearest-neighbor
interpolation. Tricubic spline interpolation requires at least
four complete equilibria at different times. It is also assumed
that they are functionally correlated, and that parameters do
not vary out of their boundaries (derivative = 0 boundary
condition). Default is False (use nearest neighbor interpolation).

\item {} 
\sphinxstyleliteralstrong{\sphinxupquote{monotonic}} (\sphinxstyleliteralemphasis{\sphinxupquote{Boolean}}) \textendash{} Sets whether or not the “monotonic” form of time
window finding is used. If True, the timebase must be monotonically
increasing. Default is False (use slower, safer method).

\end{itemize}

\end{description}\end{quote}
\index{getFluxVol() (eqtools.D3DEFIT.D3DEFITTree method)@\spxentry{getFluxVol()}\spxextra{eqtools.D3DEFIT.D3DEFITTree method}}

\begin{fulllineitems}
\phantomsection\label{\detokenize{eqtools:eqtools.D3DEFIT.D3DEFITTree.getFluxVol}}\pysiglinewithargsret{\sphinxbfcode{\sphinxupquote{getFluxVol}}}{}{}
Not implemented in D3DEFIT tree.
\begin{quote}\begin{description}
\item[{Returns}] \leavevmode
volume within flux surface {[}psi,t{]}

\end{description}\end{quote}

\end{fulllineitems}

\index{getRmidPsi() (eqtools.D3DEFIT.D3DEFITTree method)@\spxentry{getRmidPsi()}\spxextra{eqtools.D3DEFIT.D3DEFITTree method}}

\begin{fulllineitems}
\phantomsection\label{\detokenize{eqtools:eqtools.D3DEFIT.D3DEFITTree.getRmidPsi}}\pysiglinewithargsret{\sphinxbfcode{\sphinxupquote{getRmidPsi}}}{\emph{length\_unit=1}}{}
returns maximum major radius of each flux surface.
\begin{quote}\begin{description}
\item[{Keyword Arguments}] \leavevmode
\sphinxstyleliteralstrong{\sphinxupquote{length\_unit}} (\sphinxstyleliteralemphasis{\sphinxupquote{String}}\sphinxstyleliteralemphasis{\sphinxupquote{ or }}\sphinxstyleliteralemphasis{\sphinxupquote{1}}) \textendash{} unit of Rmid.  Defaults to 1, indicating
the default parameter unit (typically m).

\item[{Returns}] \leavevmode
{[}nt,npsi{]} array of maximum (outboard) major radius of
flux surface psi.

\item[{Return type}] \leavevmode
Rmid (Array)

\item[{Raises}] \leavevmode
\sphinxstyleliteralstrong{\sphinxupquote{Value Error}} \textendash{} if module cannot retrieve data from MDS tree.

\end{description}\end{quote}

\end{fulllineitems}


\end{fulllineitems}

\index{D3DEFITTreeProp (class in eqtools.D3DEFIT)@\spxentry{D3DEFITTreeProp}\spxextra{class in eqtools.D3DEFIT}}

\begin{fulllineitems}
\phantomsection\label{\detokenize{eqtools:eqtools.D3DEFIT.D3DEFITTreeProp}}\pysiglinewithargsret{\sphinxbfcode{\sphinxupquote{class }}\sphinxcode{\sphinxupquote{eqtools.D3DEFIT.}}\sphinxbfcode{\sphinxupquote{D3DEFITTreeProp}}}{\emph{shot}, \emph{tree='EFIT01'}, \emph{length\_unit='m'}, \emph{gfile='geqdsk'}, \emph{afile='aeqdsk'}, \emph{tspline=False}, \emph{monotonic=True}}{}
Bases: {\hyperref[\detokenize{eqtools:eqtools.D3DEFIT.D3DEFITTree}]{\sphinxcrossref{\sphinxcode{\sphinxupquote{eqtools.D3DEFIT.D3DEFITTree}}}}}, {\hyperref[\detokenize{eqtools:eqtools.core.PropertyAccessMixin}]{\sphinxcrossref{\sphinxcode{\sphinxupquote{eqtools.core.PropertyAccessMixin}}}}}

D3DEFITTree with the PropertyAccessMixin added to enable property-style
access. This is good for interactive use, but may drag the performance down.

\end{fulllineitems}



\subsection{eqtools.EFIT module}
\label{\detokenize{eqtools:module-eqtools.EFIT}}\label{\detokenize{eqtools:eqtools-efit-module}}\index{eqtools.EFIT (module)@\spxentry{eqtools.EFIT}\spxextra{module}}
Provides class inheriting {\hyperref[\detokenize{eqtools:eqtools.core.Equilibrium}]{\sphinxcrossref{\sphinxcode{\sphinxupquote{eqtools.core.Equilibrium}}}}} for working
with EFIT data.
\index{EFITTree (class in eqtools.EFIT)@\spxentry{EFITTree}\spxextra{class in eqtools.EFIT}}

\begin{fulllineitems}
\phantomsection\label{\detokenize{eqtools:eqtools.EFIT.EFITTree}}\pysiglinewithargsret{\sphinxbfcode{\sphinxupquote{class }}\sphinxcode{\sphinxupquote{eqtools.EFIT.}}\sphinxbfcode{\sphinxupquote{EFITTree}}}{\emph{shot}, \emph{tree}, \emph{root}, \emph{length\_unit='m'}, \emph{gfile='g\_eqdsk'}, \emph{afile='a\_eqdsk'}, \emph{tspline=False}, \emph{monotonic=True}}{}
Bases: {\hyperref[\detokenize{eqtools:eqtools.core.Equilibrium}]{\sphinxcrossref{\sphinxcode{\sphinxupquote{eqtools.core.Equilibrium}}}}}

Inherits {\hyperref[\detokenize{eqtools:eqtools.core.Equilibrium}]{\sphinxcrossref{\sphinxcode{\sphinxupquote{Equilibrium}}}}} class.
EFIT-specific data handling class for machines using standard EFIT tag
names/tree structure with MDSplus.  Constructor and/or data loading may
need overriding in a machine-specific implementation.  Pulls EFIT data
from selected MDS tree and shot, stores as object attributes.  Each EFIT
variable or set of variables is recovered with a corresponding getter
method.  Essential data for EFIT mapping are pulled on initialization
(e.g. psirz grid).  Additional data are pulled at the first request and
stored for subsequent usage.

Intializes {\hyperref[\detokenize{eqtools:eqtools.EFIT.EFITTree}]{\sphinxcrossref{\sphinxcode{\sphinxupquote{EFITTree}}}}} object. Pulls data from MDS tree for
storage in instance attributes. Core attributes are populated from the MDS
tree on initialization. Additional attributes are initialized as None,
filled on the first request to the object.
\begin{quote}\begin{description}
\item[{Parameters}] \leavevmode\begin{itemize}
\item {} 
\sphinxstyleliteralstrong{\sphinxupquote{shot}} (\sphinxstyleliteralemphasis{\sphinxupquote{integer}}) \textendash{} Shot number

\item {} 
\sphinxstyleliteralstrong{\sphinxupquote{tree}} (\sphinxstyleliteralemphasis{\sphinxupquote{string}}) \textendash{} MDSplus tree to open to fetch EFIT data.

\item {} 
\sphinxstyleliteralstrong{\sphinxupquote{root}} (\sphinxstyleliteralemphasis{\sphinxupquote{string}}) \textendash{} Root path for EFIT data in MDSplus tree.

\end{itemize}

\item[{Keyword Arguments}] \leavevmode\begin{itemize}
\item {} 
\sphinxstyleliteralstrong{\sphinxupquote{length\_unit}} (\sphinxstyleliteralemphasis{\sphinxupquote{string}}) \textendash{} 
Sets the base unit used for any
quantity whose dimensions are length to any power.
Valid options are:
\begin{quote}


\begin{savenotes}\sphinxattablestart
\centering
\begin{tabulary}{\linewidth}[t]{|T|T|}
\hline

’m’
&
meters
\\
\hline
’cm’
&
centimeters
\\
\hline
’mm’
&
millimeters
\\
\hline
’in’
&
inches
\\
\hline
’ft’
&
feet
\\
\hline
’yd’
&
yards
\\
\hline
’smoot’
&
smoots
\\
\hline
’cubit’
&
cubits
\\
\hline
’hand’
&
hands
\\
\hline
’default’
&
whatever the default in the tree is (no conversion is performed, units may be inconsistent)
\\
\hline
\end{tabulary}
\par
\sphinxattableend\end{savenotes}
\end{quote}

Default is ‘m’ (all units taken and returned in meters).


\item {} 
\sphinxstyleliteralstrong{\sphinxupquote{tspline}} (\sphinxstyleliteralemphasis{\sphinxupquote{boolean}}) \textendash{} Sets whether or not interpolation in time is
performed using a tricubic spline or nearest-neighbor
interpolation. Tricubic spline interpolation requires at least
four complete equilibria at different times. It is also assumed
that they are functionally correlated, and that parameters do
not vary out of their boundaries (derivative = 0 boundary
condition). Default is False (use nearest neighbor interpolation).

\item {} 
\sphinxstyleliteralstrong{\sphinxupquote{monotonic}} (\sphinxstyleliteralemphasis{\sphinxupquote{boolean}}) \textendash{} Sets whether or not the “monotonic” form of time
window finding is used. If True, the timebase must be
monotonically increasing. Default is False (use slower,
safer method).

\end{itemize}

\end{description}\end{quote}
\index{getInfo() (eqtools.EFIT.EFITTree method)@\spxentry{getInfo()}\spxextra{eqtools.EFIT.EFITTree method}}

\begin{fulllineitems}
\phantomsection\label{\detokenize{eqtools:eqtools.EFIT.EFITTree.getInfo}}\pysiglinewithargsret{\sphinxbfcode{\sphinxupquote{getInfo}}}{}{}
returns namedtuple of shot information
\begin{quote}\begin{description}
\item[{Returns}] \leavevmode

namedtuple containing
\begin{quote}


\begin{savenotes}\sphinxattablestart
\centering
\begin{tabulary}{\linewidth}[t]{|T|T|}
\hline

shot
&
C-Mod shot index (long)
\\
\hline
tree
&
EFIT tree (string)
\\
\hline
nr
&
size of R-axis for spatial grid
\\
\hline
nz
&
size of Z-axis for spatial grid
\\
\hline
nt
&
size of timebase for flux grid
\\
\hline
\end{tabulary}
\par
\sphinxattableend\end{savenotes}
\end{quote}


\end{description}\end{quote}

\end{fulllineitems}

\index{getTimeBase() (eqtools.EFIT.EFITTree method)@\spxentry{getTimeBase()}\spxextra{eqtools.EFIT.EFITTree method}}

\begin{fulllineitems}
\phantomsection\label{\detokenize{eqtools:eqtools.EFIT.EFITTree.getTimeBase}}\pysiglinewithargsret{\sphinxbfcode{\sphinxupquote{getTimeBase}}}{}{}
returns EFIT time base vector.
\begin{quote}\begin{description}
\item[{Returns}] \leavevmode
{[}nt{]} array of time points.

\item[{Return type}] \leavevmode
time (array)

\item[{Raises}] \leavevmode
\sphinxstyleliteralstrong{\sphinxupquote{ValueError}} \textendash{} if module cannot retrieve data from MDS tree.

\end{description}\end{quote}

\end{fulllineitems}

\index{getFluxGrid() (eqtools.EFIT.EFITTree method)@\spxentry{getFluxGrid()}\spxextra{eqtools.EFIT.EFITTree method}}

\begin{fulllineitems}
\phantomsection\label{\detokenize{eqtools:eqtools.EFIT.EFITTree.getFluxGrid}}\pysiglinewithargsret{\sphinxbfcode{\sphinxupquote{getFluxGrid}}}{}{}
returns EFIT flux grid.

Note that this method preserves whatever sign convention is used in the
tree. For C-Mod, this means that the result should be multiplied by
-1 * {\hyperref[\detokenize{eqtools:eqtools.EFIT.EFITTree.getCurrentSign}]{\sphinxcrossref{\sphinxcode{\sphinxupquote{getCurrentSign()}}}}} in most cases.
\begin{quote}\begin{description}
\item[{Returns}] \leavevmode
{[}nt,nz,nr{]} array of (non-normalized) flux on grid.

\item[{Return type}] \leavevmode
psiRZ (Array)

\item[{Raises}] \leavevmode
\sphinxstyleliteralstrong{\sphinxupquote{ValueError}} \textendash{} if module cannot retrieve data from MDS tree.

\end{description}\end{quote}

\end{fulllineitems}

\index{getRGrid() (eqtools.EFIT.EFITTree method)@\spxentry{getRGrid()}\spxextra{eqtools.EFIT.EFITTree method}}

\begin{fulllineitems}
\phantomsection\label{\detokenize{eqtools:eqtools.EFIT.EFITTree.getRGrid}}\pysiglinewithargsret{\sphinxbfcode{\sphinxupquote{getRGrid}}}{\emph{length\_unit=1}}{}
returns EFIT R-axis.
\begin{quote}\begin{description}
\item[{Returns}] \leavevmode
{[}nr{]} array of R-axis of flux grid.

\item[{Return type}] \leavevmode
rGrid (Array)

\item[{Raises}] \leavevmode
\sphinxstyleliteralstrong{\sphinxupquote{ValueError}} \textendash{} if module cannot retrieve data from MDS tree.

\end{description}\end{quote}

\end{fulllineitems}

\index{getZGrid() (eqtools.EFIT.EFITTree method)@\spxentry{getZGrid()}\spxextra{eqtools.EFIT.EFITTree method}}

\begin{fulllineitems}
\phantomsection\label{\detokenize{eqtools:eqtools.EFIT.EFITTree.getZGrid}}\pysiglinewithargsret{\sphinxbfcode{\sphinxupquote{getZGrid}}}{\emph{length\_unit=1}}{}
returns EFIT Z-axis.
\begin{quote}\begin{description}
\item[{Returns}] \leavevmode
{[}nz{]} array of Z-axis of flux grid.

\item[{Return type}] \leavevmode
zGrid (Array)

\item[{Raises}] \leavevmode
\sphinxstyleliteralstrong{\sphinxupquote{ValueError}} \textendash{} if module cannot retrieve data from MDS tree.

\end{description}\end{quote}

\end{fulllineitems}

\index{getFluxAxis() (eqtools.EFIT.EFITTree method)@\spxentry{getFluxAxis()}\spxextra{eqtools.EFIT.EFITTree method}}

\begin{fulllineitems}
\phantomsection\label{\detokenize{eqtools:eqtools.EFIT.EFITTree.getFluxAxis}}\pysiglinewithargsret{\sphinxbfcode{\sphinxupquote{getFluxAxis}}}{}{}
returns psi on magnetic axis.
\begin{quote}\begin{description}
\item[{Returns}] \leavevmode
{[}nt{]} array of psi on magnetic axis.

\item[{Return type}] \leavevmode
psiAxis (Array)

\item[{Raises}] \leavevmode
\sphinxstyleliteralstrong{\sphinxupquote{ValueError}} \textendash{} if module cannot retrieve data from MDS tree.

\end{description}\end{quote}

\end{fulllineitems}

\index{getFluxLCFS() (eqtools.EFIT.EFITTree method)@\spxentry{getFluxLCFS()}\spxextra{eqtools.EFIT.EFITTree method}}

\begin{fulllineitems}
\phantomsection\label{\detokenize{eqtools:eqtools.EFIT.EFITTree.getFluxLCFS}}\pysiglinewithargsret{\sphinxbfcode{\sphinxupquote{getFluxLCFS}}}{}{}
returns psi at separatrix.
\begin{quote}\begin{description}
\item[{Returns}] \leavevmode
{[}nt{]} array of psi at LCFS.

\item[{Return type}] \leavevmode
psiLCFS (Array)

\item[{Raises}] \leavevmode
\sphinxstyleliteralstrong{\sphinxupquote{ValueError}} \textendash{} if module cannot retrieve data from MDS tree.

\end{description}\end{quote}

\end{fulllineitems}

\index{getFluxVol() (eqtools.EFIT.EFITTree method)@\spxentry{getFluxVol()}\spxextra{eqtools.EFIT.EFITTree method}}

\begin{fulllineitems}
\phantomsection\label{\detokenize{eqtools:eqtools.EFIT.EFITTree.getFluxVol}}\pysiglinewithargsret{\sphinxbfcode{\sphinxupquote{getFluxVol}}}{\emph{length\_unit=3}}{}
returns volume within flux surface.
\begin{quote}\begin{description}
\item[{Keyword Arguments}] \leavevmode
\sphinxstyleliteralstrong{\sphinxupquote{length\_unit}} (\sphinxstyleliteralemphasis{\sphinxupquote{String}}\sphinxstyleliteralemphasis{\sphinxupquote{ or }}\sphinxstyleliteralemphasis{\sphinxupquote{3}}) \textendash{} unit for plasma volume.  Defaults to 3,
indicating default volumetric unit (typically m\textasciicircum{}3).

\item[{Returns}] \leavevmode
{[}nt,npsi{]} array of volume within flux surface.

\item[{Return type}] \leavevmode
fluxVol (Array)

\item[{Raises}] \leavevmode
\sphinxstyleliteralstrong{\sphinxupquote{ValueError}} \textendash{} if module cannot retrieve data from MDS tree.

\end{description}\end{quote}

\end{fulllineitems}

\index{getVolLCFS() (eqtools.EFIT.EFITTree method)@\spxentry{getVolLCFS()}\spxextra{eqtools.EFIT.EFITTree method}}

\begin{fulllineitems}
\phantomsection\label{\detokenize{eqtools:eqtools.EFIT.EFITTree.getVolLCFS}}\pysiglinewithargsret{\sphinxbfcode{\sphinxupquote{getVolLCFS}}}{\emph{length\_unit=3}}{}
returns volume within LCFS.
\begin{quote}\begin{description}
\item[{Keyword Arguments}] \leavevmode
\sphinxstyleliteralstrong{\sphinxupquote{length\_unit}} (\sphinxstyleliteralemphasis{\sphinxupquote{String}}\sphinxstyleliteralemphasis{\sphinxupquote{ or }}\sphinxstyleliteralemphasis{\sphinxupquote{3}}) \textendash{} unit for LCFS volume.  Defaults to 3,
denoting default volumetric unit (typically m\textasciicircum{}3).

\item[{Returns}] \leavevmode
{[}nt{]} array of volume within LCFS.

\item[{Return type}] \leavevmode
volLCFS (Array)

\item[{Raises}] \leavevmode
\sphinxstyleliteralstrong{\sphinxupquote{ValueError}} \textendash{} if module cannot retrieve data from MDS tree.

\end{description}\end{quote}

\end{fulllineitems}

\index{getRmidPsi() (eqtools.EFIT.EFITTree method)@\spxentry{getRmidPsi()}\spxextra{eqtools.EFIT.EFITTree method}}

\begin{fulllineitems}
\phantomsection\label{\detokenize{eqtools:eqtools.EFIT.EFITTree.getRmidPsi}}\pysiglinewithargsret{\sphinxbfcode{\sphinxupquote{getRmidPsi}}}{\emph{length\_unit=1}}{}
returns maximum major radius of each flux surface.
\begin{quote}\begin{description}
\item[{Keyword Arguments}] \leavevmode
\sphinxstyleliteralstrong{\sphinxupquote{length\_unit}} (\sphinxstyleliteralemphasis{\sphinxupquote{String}}\sphinxstyleliteralemphasis{\sphinxupquote{ or }}\sphinxstyleliteralemphasis{\sphinxupquote{1}}) \textendash{} unit of Rmid.  Defaults to 1, indicating
the default parameter unit (typically m).

\item[{Returns}] \leavevmode
{[}nt,npsi{]} array of maximum (outboard) major radius of
flux surface psi.

\item[{Return type}] \leavevmode
Rmid (Array)

\item[{Raises}] \leavevmode
\sphinxstyleliteralstrong{\sphinxupquote{Value Error}} \textendash{} if module cannot retrieve data from MDS tree.

\end{description}\end{quote}

\end{fulllineitems}

\index{getRLCFS() (eqtools.EFIT.EFITTree method)@\spxentry{getRLCFS()}\spxextra{eqtools.EFIT.EFITTree method}}

\begin{fulllineitems}
\phantomsection\label{\detokenize{eqtools:eqtools.EFIT.EFITTree.getRLCFS}}\pysiglinewithargsret{\sphinxbfcode{\sphinxupquote{getRLCFS}}}{\emph{length\_unit=1}}{}
returns R-values of LCFS position.
\begin{quote}\begin{description}
\item[{Returns}] \leavevmode
{[}nt,n{]} array of R of LCFS points.

\item[{Return type}] \leavevmode
RLCFS (Array)

\item[{Raises}] \leavevmode
\sphinxstyleliteralstrong{\sphinxupquote{ValueError}} \textendash{} if module cannot retrieve data from MDS tree.

\end{description}\end{quote}

\end{fulllineitems}

\index{getZLCFS() (eqtools.EFIT.EFITTree method)@\spxentry{getZLCFS()}\spxextra{eqtools.EFIT.EFITTree method}}

\begin{fulllineitems}
\phantomsection\label{\detokenize{eqtools:eqtools.EFIT.EFITTree.getZLCFS}}\pysiglinewithargsret{\sphinxbfcode{\sphinxupquote{getZLCFS}}}{\emph{length\_unit=1}}{}
returns Z-values of LCFS position.
\begin{quote}\begin{description}
\item[{Returns}] \leavevmode
{[}nt,n{]} array of Z of LCFS points.

\item[{Return type}] \leavevmode
ZLCFS (Array)

\item[{Raises}] \leavevmode
\sphinxstyleliteralstrong{\sphinxupquote{ValueError}} \textendash{} if module cannot retrieve data from MDS tree.

\end{description}\end{quote}

\end{fulllineitems}

\index{remapLCFS() (eqtools.EFIT.EFITTree method)@\spxentry{remapLCFS()}\spxextra{eqtools.EFIT.EFITTree method}}

\begin{fulllineitems}
\phantomsection\label{\detokenize{eqtools:eqtools.EFIT.EFITTree.remapLCFS}}\pysiglinewithargsret{\sphinxbfcode{\sphinxupquote{remapLCFS}}}{\emph{mask=False}}{}
Overwrites RLCFS, ZLCFS values pulled from EFIT with
explicitly-calculated contour of psinorm=1 surface.  This is then masked
down by the limiter array using core.inPolygon, restricting the contour
to the closed plasma surface and the divertor legs.
\begin{quote}\begin{description}
\item[{Keyword Arguments}] \leavevmode
\sphinxstyleliteralstrong{\sphinxupquote{mask}} (\sphinxstyleliteralemphasis{\sphinxupquote{Boolean}}) \textendash{} Default False.  Set True to mask LCFS path to
limiter outline (using inPolygon).  Set False to draw full
contour of psi = psiLCFS.

\item[{Raises}] \leavevmode\begin{itemize}
\item {} 
\sphinxstyleliteralstrong{\sphinxupquote{NotImplementedError}} \textendash{} if \sphinxcode{\sphinxupquote{matplotlib.pyplot}} is not loaded.

\item {} 
\sphinxstyleliteralstrong{\sphinxupquote{ValueError}} \textendash{} if limiter outline is not available.

\end{itemize}

\end{description}\end{quote}

\end{fulllineitems}

\index{getF() (eqtools.EFIT.EFITTree method)@\spxentry{getF()}\spxextra{eqtools.EFIT.EFITTree method}}

\begin{fulllineitems}
\phantomsection\label{\detokenize{eqtools:eqtools.EFIT.EFITTree.getF}}\pysiglinewithargsret{\sphinxbfcode{\sphinxupquote{getF}}}{}{}
returns F=RB\_\{Phi\}(Psi), often calculated for grad-shafranov
solutions.

Note that this method preserves whatever sign convention is used in the
tree. For C-Mod, this means that the result should be multiplied by
-1 * {\hyperref[\detokenize{eqtools:eqtools.EFIT.EFITTree.getCurrentSign}]{\sphinxcrossref{\sphinxcode{\sphinxupquote{getCurrentSign()}}}}} in most cases.
\begin{quote}\begin{description}
\item[{Returns}] \leavevmode
{[}nt,npsi{]} array of F=RB\_\{Phi\}(Psi)

\item[{Return type}] \leavevmode
F (Array)

\item[{Raises}] \leavevmode
\sphinxstyleliteralstrong{\sphinxupquote{ValueError}} \textendash{} if module cannot retrieve data from MDS tree.

\end{description}\end{quote}

\end{fulllineitems}

\index{getFluxPres() (eqtools.EFIT.EFITTree method)@\spxentry{getFluxPres()}\spxextra{eqtools.EFIT.EFITTree method}}

\begin{fulllineitems}
\phantomsection\label{\detokenize{eqtools:eqtools.EFIT.EFITTree.getFluxPres}}\pysiglinewithargsret{\sphinxbfcode{\sphinxupquote{getFluxPres}}}{}{}
returns pressure at flux surface.
\begin{quote}\begin{description}
\item[{Returns}] \leavevmode
{[}nt,npsi{]} array of pressure on flux surface psi.

\item[{Return type}] \leavevmode
p (Array)

\item[{Raises}] \leavevmode
\sphinxstyleliteralstrong{\sphinxupquote{ValueError}} \textendash{} if module cannot retrieve data from MDS tree.

\end{description}\end{quote}

\end{fulllineitems}

\index{getFFPrime() (eqtools.EFIT.EFITTree method)@\spxentry{getFFPrime()}\spxextra{eqtools.EFIT.EFITTree method}}

\begin{fulllineitems}
\phantomsection\label{\detokenize{eqtools:eqtools.EFIT.EFITTree.getFFPrime}}\pysiglinewithargsret{\sphinxbfcode{\sphinxupquote{getFFPrime}}}{}{}
returns FF’ function used for grad-shafranov solutions.
\begin{quote}\begin{description}
\item[{Returns}] \leavevmode
{[}nt,npsi{]} array of FF’ fromgrad-shafranov solution.

\item[{Return type}] \leavevmode
FFprime (Array)

\item[{Raises}] \leavevmode
\sphinxstyleliteralstrong{\sphinxupquote{ValueError}} \textendash{} if module cannot retrieve data from MDS tree.

\end{description}\end{quote}

\end{fulllineitems}

\index{getPPrime() (eqtools.EFIT.EFITTree method)@\spxentry{getPPrime()}\spxextra{eqtools.EFIT.EFITTree method}}

\begin{fulllineitems}
\phantomsection\label{\detokenize{eqtools:eqtools.EFIT.EFITTree.getPPrime}}\pysiglinewithargsret{\sphinxbfcode{\sphinxupquote{getPPrime}}}{}{}
returns plasma pressure gradient as a function of psi.
\begin{quote}\begin{description}
\item[{Returns}] \leavevmode
{[}nt,npsi{]} array of pressure gradient on flux surface
psi from grad-shafranov solution.

\item[{Return type}] \leavevmode
pprime (Array)

\item[{Raises}] \leavevmode
\sphinxstyleliteralstrong{\sphinxupquote{ValueError}} \textendash{} if module cannot retrieve data from MDS tree.

\end{description}\end{quote}

\end{fulllineitems}

\index{getElongation() (eqtools.EFIT.EFITTree method)@\spxentry{getElongation()}\spxextra{eqtools.EFIT.EFITTree method}}

\begin{fulllineitems}
\phantomsection\label{\detokenize{eqtools:eqtools.EFIT.EFITTree.getElongation}}\pysiglinewithargsret{\sphinxbfcode{\sphinxupquote{getElongation}}}{}{}
returns LCFS elongation.
\begin{quote}\begin{description}
\item[{Returns}] \leavevmode
{[}nt{]} array of LCFS elongation.

\item[{Return type}] \leavevmode
kappa (Array)

\item[{Raises}] \leavevmode
\sphinxstyleliteralstrong{\sphinxupquote{ValueError}} \textendash{} if module cannot retrieve data from MDS tree.

\end{description}\end{quote}

\end{fulllineitems}

\index{getUpperTriangularity() (eqtools.EFIT.EFITTree method)@\spxentry{getUpperTriangularity()}\spxextra{eqtools.EFIT.EFITTree method}}

\begin{fulllineitems}
\phantomsection\label{\detokenize{eqtools:eqtools.EFIT.EFITTree.getUpperTriangularity}}\pysiglinewithargsret{\sphinxbfcode{\sphinxupquote{getUpperTriangularity}}}{}{}
returns LCFS upper triangularity.
\begin{quote}\begin{description}
\item[{Returns}] \leavevmode
{[}nt{]} array of LCFS upper triangularity.

\item[{Return type}] \leavevmode
deltau (Array)

\item[{Raises}] \leavevmode
\sphinxstyleliteralstrong{\sphinxupquote{ValueError}} \textendash{} if module cannot retrieve data from MDS tree.

\end{description}\end{quote}

\end{fulllineitems}

\index{getLowerTriangularity() (eqtools.EFIT.EFITTree method)@\spxentry{getLowerTriangularity()}\spxextra{eqtools.EFIT.EFITTree method}}

\begin{fulllineitems}
\phantomsection\label{\detokenize{eqtools:eqtools.EFIT.EFITTree.getLowerTriangularity}}\pysiglinewithargsret{\sphinxbfcode{\sphinxupquote{getLowerTriangularity}}}{}{}
returns LCFS lower triangularity.
\begin{quote}\begin{description}
\item[{Returns}] \leavevmode
{[}nt{]} array of LCFS lower triangularity.

\item[{Return type}] \leavevmode
deltal (Array)

\item[{Raises}] \leavevmode
\sphinxstyleliteralstrong{\sphinxupquote{ValueError}} \textendash{} if module cannot retrieve data from MDS tree.

\end{description}\end{quote}

\end{fulllineitems}

\index{getShaping() (eqtools.EFIT.EFITTree method)@\spxentry{getShaping()}\spxextra{eqtools.EFIT.EFITTree method}}

\begin{fulllineitems}
\phantomsection\label{\detokenize{eqtools:eqtools.EFIT.EFITTree.getShaping}}\pysiglinewithargsret{\sphinxbfcode{\sphinxupquote{getShaping}}}{}{}
pulls LCFS elongation and upper/lower triangularity.
\begin{quote}\begin{description}
\item[{Returns}] \leavevmode
namedtuple containing (kappa, delta\_u, delta\_l)

\item[{Raises}] \leavevmode
\sphinxstyleliteralstrong{\sphinxupquote{ValueError}} \textendash{} if module cannot retrieve data from MDS tree.

\end{description}\end{quote}

\end{fulllineitems}

\index{getMagR() (eqtools.EFIT.EFITTree method)@\spxentry{getMagR()}\spxextra{eqtools.EFIT.EFITTree method}}

\begin{fulllineitems}
\phantomsection\label{\detokenize{eqtools:eqtools.EFIT.EFITTree.getMagR}}\pysiglinewithargsret{\sphinxbfcode{\sphinxupquote{getMagR}}}{\emph{length\_unit=1}}{}
returns magnetic-axis major radius.
\begin{quote}\begin{description}
\item[{Returns}] \leavevmode
{[}nt{]} array of major radius of magnetic axis.

\item[{Return type}] \leavevmode
magR (Array)

\item[{Raises}] \leavevmode
\sphinxstyleliteralstrong{\sphinxupquote{ValueError}} \textendash{} if module cannot retrieve data from MDS tree.

\end{description}\end{quote}

\end{fulllineitems}

\index{getMagZ() (eqtools.EFIT.EFITTree method)@\spxentry{getMagZ()}\spxextra{eqtools.EFIT.EFITTree method}}

\begin{fulllineitems}
\phantomsection\label{\detokenize{eqtools:eqtools.EFIT.EFITTree.getMagZ}}\pysiglinewithargsret{\sphinxbfcode{\sphinxupquote{getMagZ}}}{\emph{length\_unit=1}}{}
returns magnetic-axis Z.
\begin{quote}\begin{description}
\item[{Returns}] \leavevmode
{[}nt{]} array of Z of magnetic axis.

\item[{Return type}] \leavevmode
magZ (Array)

\item[{Raises}] \leavevmode
\sphinxstyleliteralstrong{\sphinxupquote{ValueError}} \textendash{} if module cannot retrieve data from MDS tree.

\end{description}\end{quote}

\end{fulllineitems}

\index{getAreaLCFS() (eqtools.EFIT.EFITTree method)@\spxentry{getAreaLCFS()}\spxextra{eqtools.EFIT.EFITTree method}}

\begin{fulllineitems}
\phantomsection\label{\detokenize{eqtools:eqtools.EFIT.EFITTree.getAreaLCFS}}\pysiglinewithargsret{\sphinxbfcode{\sphinxupquote{getAreaLCFS}}}{\emph{length\_unit=2}}{}
returns LCFS cross-sectional area.
\begin{quote}\begin{description}
\item[{Keyword Arguments}] \leavevmode
\sphinxstyleliteralstrong{\sphinxupquote{length\_unit}} (\sphinxstyleliteralemphasis{\sphinxupquote{String}}\sphinxstyleliteralemphasis{\sphinxupquote{ or }}\sphinxstyleliteralemphasis{\sphinxupquote{2}}) \textendash{} unit for LCFS area.  Defaults to 2,
denoting default areal unit (typically m\textasciicircum{}2).

\item[{Returns}] \leavevmode
{[}nt{]} array of LCFS area.

\item[{Return type}] \leavevmode
areaLCFS (Array)

\item[{Raises}] \leavevmode
\sphinxstyleliteralstrong{\sphinxupquote{ValueError}} \textendash{} if module cannot retrieve data from MDS tree.

\end{description}\end{quote}

\end{fulllineitems}

\index{getAOut() (eqtools.EFIT.EFITTree method)@\spxentry{getAOut()}\spxextra{eqtools.EFIT.EFITTree method}}

\begin{fulllineitems}
\phantomsection\label{\detokenize{eqtools:eqtools.EFIT.EFITTree.getAOut}}\pysiglinewithargsret{\sphinxbfcode{\sphinxupquote{getAOut}}}{\emph{length\_unit=1}}{}
returns outboard-midplane minor radius at LCFS.
\begin{quote}\begin{description}
\item[{Keyword Arguments}] \leavevmode
\sphinxstyleliteralstrong{\sphinxupquote{length\_unit}} (\sphinxstyleliteralemphasis{\sphinxupquote{String}}\sphinxstyleliteralemphasis{\sphinxupquote{ or }}\sphinxstyleliteralemphasis{\sphinxupquote{1}}) \textendash{} unit for minor radius.  Defaults to 1,
denoting default length unit (typically m).

\item[{Returns}] \leavevmode
{[}nt{]} array of LCFS outboard-midplane minor radius.

\item[{Return type}] \leavevmode
aOut (Array)

\item[{Raises}] \leavevmode
\sphinxstyleliteralstrong{\sphinxupquote{ValueError}} \textendash{} if module cannot retrieve data from MDS tree.

\end{description}\end{quote}

\end{fulllineitems}

\index{getRmidOut() (eqtools.EFIT.EFITTree method)@\spxentry{getRmidOut()}\spxextra{eqtools.EFIT.EFITTree method}}

\begin{fulllineitems}
\phantomsection\label{\detokenize{eqtools:eqtools.EFIT.EFITTree.getRmidOut}}\pysiglinewithargsret{\sphinxbfcode{\sphinxupquote{getRmidOut}}}{\emph{length\_unit=1}}{}
returns outboard-midplane major radius.
\begin{quote}\begin{description}
\item[{Keyword Arguments}] \leavevmode
\sphinxstyleliteralstrong{\sphinxupquote{length\_unit}} (\sphinxstyleliteralemphasis{\sphinxupquote{String}}\sphinxstyleliteralemphasis{\sphinxupquote{ or }}\sphinxstyleliteralemphasis{\sphinxupquote{1}}) \textendash{} unit for major radius.  Defaults to 1,
denoting default length unit (typically m).

\item[{Returns}] \leavevmode
{[}nt{]} array of major radius of LCFS.

\item[{Return type}] \leavevmode
RmidOut (Array)

\item[{Raises}] \leavevmode
\sphinxstyleliteralstrong{\sphinxupquote{ValueError}} \textendash{} if module cannot retrieve data from MDS tree.

\end{description}\end{quote}

\end{fulllineitems}

\index{getGeometry() (eqtools.EFIT.EFITTree method)@\spxentry{getGeometry()}\spxextra{eqtools.EFIT.EFITTree method}}

\begin{fulllineitems}
\phantomsection\label{\detokenize{eqtools:eqtools.EFIT.EFITTree.getGeometry}}\pysiglinewithargsret{\sphinxbfcode{\sphinxupquote{getGeometry}}}{\emph{length\_unit=None}}{}
pulls dimensional geometry parameters.
\begin{quote}\begin{description}
\item[{Returns}] \leavevmode
namedtuple containing (magR,magZ,areaLCFS,aOut,RmidOut)

\item[{Raises}] \leavevmode
\sphinxstyleliteralstrong{\sphinxupquote{ValueError}} \textendash{} if module cannot retrieve data from MDS tree.

\end{description}\end{quote}

\end{fulllineitems}

\index{getQProfile() (eqtools.EFIT.EFITTree method)@\spxentry{getQProfile()}\spxextra{eqtools.EFIT.EFITTree method}}

\begin{fulllineitems}
\phantomsection\label{\detokenize{eqtools:eqtools.EFIT.EFITTree.getQProfile}}\pysiglinewithargsret{\sphinxbfcode{\sphinxupquote{getQProfile}}}{}{}
returns profile of safety factor q.
\begin{quote}\begin{description}
\item[{Returns}] \leavevmode
{[}nt,npsi{]} array of q on flux surface psi.

\item[{Return type}] \leavevmode
qpsi (Array)

\item[{Raises}] \leavevmode
\sphinxstyleliteralstrong{\sphinxupquote{ValueError}} \textendash{} if module cannot retrieve data from MDS tree.

\end{description}\end{quote}

\end{fulllineitems}

\index{getQ0() (eqtools.EFIT.EFITTree method)@\spxentry{getQ0()}\spxextra{eqtools.EFIT.EFITTree method}}

\begin{fulllineitems}
\phantomsection\label{\detokenize{eqtools:eqtools.EFIT.EFITTree.getQ0}}\pysiglinewithargsret{\sphinxbfcode{\sphinxupquote{getQ0}}}{}{}
returns q on magnetic axis,q0.
\begin{quote}\begin{description}
\item[{Returns}] \leavevmode
{[}nt{]} array of q(psi=0).

\item[{Return type}] \leavevmode
q0 (Array)

\item[{Raises}] \leavevmode
\sphinxstyleliteralstrong{\sphinxupquote{ValueError}} \textendash{} if module cannot retrieve data from MDS tree.

\end{description}\end{quote}

\end{fulllineitems}

\index{getQ95() (eqtools.EFIT.EFITTree method)@\spxentry{getQ95()}\spxextra{eqtools.EFIT.EFITTree method}}

\begin{fulllineitems}
\phantomsection\label{\detokenize{eqtools:eqtools.EFIT.EFITTree.getQ95}}\pysiglinewithargsret{\sphinxbfcode{\sphinxupquote{getQ95}}}{}{}
returns q at 95\% flux surface.
\begin{quote}\begin{description}
\item[{Returns}] \leavevmode
{[}nt{]} array of q(psi=0.95).

\item[{Return type}] \leavevmode
q95 (Array)

\item[{Raises}] \leavevmode
\sphinxstyleliteralstrong{\sphinxupquote{ValueError}} \textendash{} if module cannot retrieve data from MDS tree.

\end{description}\end{quote}

\end{fulllineitems}

\index{getQLCFS() (eqtools.EFIT.EFITTree method)@\spxentry{getQLCFS()}\spxextra{eqtools.EFIT.EFITTree method}}

\begin{fulllineitems}
\phantomsection\label{\detokenize{eqtools:eqtools.EFIT.EFITTree.getQLCFS}}\pysiglinewithargsret{\sphinxbfcode{\sphinxupquote{getQLCFS}}}{}{}
returns q on LCFS (interpolated).
\begin{quote}\begin{description}
\item[{Returns}] \leavevmode
{[}nt{]} array of q* (interpolated).

\item[{Return type}] \leavevmode
qLCFS (Array)

\item[{Raises}] \leavevmode
\sphinxstyleliteralstrong{\sphinxupquote{ValueError}} \textendash{} if module cannot retrieve data from MDS tree.

\end{description}\end{quote}

\end{fulllineitems}

\index{getQ1Surf() (eqtools.EFIT.EFITTree method)@\spxentry{getQ1Surf()}\spxextra{eqtools.EFIT.EFITTree method}}

\begin{fulllineitems}
\phantomsection\label{\detokenize{eqtools:eqtools.EFIT.EFITTree.getQ1Surf}}\pysiglinewithargsret{\sphinxbfcode{\sphinxupquote{getQ1Surf}}}{\emph{length\_unit=1}}{}
returns outboard-midplane minor radius of q=1 surface.
\begin{quote}\begin{description}
\item[{Keyword Arguments}] \leavevmode
\sphinxstyleliteralstrong{\sphinxupquote{length\_unit}} (\sphinxstyleliteralemphasis{\sphinxupquote{String}}\sphinxstyleliteralemphasis{\sphinxupquote{ or }}\sphinxstyleliteralemphasis{\sphinxupquote{1}}) \textendash{} unit for minor radius.  Defaults to 1,
denoting default length unit (typically m).

\item[{Returns}] \leavevmode
{[}nt{]} array of minor radius of q=1 surface.

\item[{Return type}] \leavevmode
qr1 (Array)

\item[{Raises}] \leavevmode
\sphinxstyleliteralstrong{\sphinxupquote{ValueError}} \textendash{} if module cannot retrieve data from MDS tree.

\end{description}\end{quote}

\end{fulllineitems}

\index{getQ2Surf() (eqtools.EFIT.EFITTree method)@\spxentry{getQ2Surf()}\spxextra{eqtools.EFIT.EFITTree method}}

\begin{fulllineitems}
\phantomsection\label{\detokenize{eqtools:eqtools.EFIT.EFITTree.getQ2Surf}}\pysiglinewithargsret{\sphinxbfcode{\sphinxupquote{getQ2Surf}}}{\emph{length\_unit=1}}{}
returns outboard-midplane minor radius of q=2 surface.
\begin{quote}\begin{description}
\item[{Keyword Arguments}] \leavevmode
\sphinxstyleliteralstrong{\sphinxupquote{length\_unit}} (\sphinxstyleliteralemphasis{\sphinxupquote{String}}\sphinxstyleliteralemphasis{\sphinxupquote{ or }}\sphinxstyleliteralemphasis{\sphinxupquote{1}}) \textendash{} unit for minor radius.  Defaults to 1,
denoting default length unit (typically m).

\item[{Returns}] \leavevmode
{[}nt{]} array of minor radius of q=2 surface.

\item[{Return type}] \leavevmode
qr2 (Array)

\item[{Raises}] \leavevmode
\sphinxstyleliteralstrong{\sphinxupquote{ValueError}} \textendash{} if module cannot retrieve data from MDS tree.

\end{description}\end{quote}

\end{fulllineitems}

\index{getQ3Surf() (eqtools.EFIT.EFITTree method)@\spxentry{getQ3Surf()}\spxextra{eqtools.EFIT.EFITTree method}}

\begin{fulllineitems}
\phantomsection\label{\detokenize{eqtools:eqtools.EFIT.EFITTree.getQ3Surf}}\pysiglinewithargsret{\sphinxbfcode{\sphinxupquote{getQ3Surf}}}{\emph{length\_unit=1}}{}
returns outboard-midplane minor radius of q=3 surface.
\begin{quote}\begin{description}
\item[{Keyword Arguments}] \leavevmode
\sphinxstyleliteralstrong{\sphinxupquote{length\_unit}} (\sphinxstyleliteralemphasis{\sphinxupquote{String}}\sphinxstyleliteralemphasis{\sphinxupquote{ or }}\sphinxstyleliteralemphasis{\sphinxupquote{1}}) \textendash{} unit for minor radius.  Defaults to 1,
denoting default length unit (typically m).

\item[{Returns}] \leavevmode
{[}nt{]} array of minor radius of q=3 surface.

\item[{Return type}] \leavevmode
qr3 (Array)

\item[{Raises}] \leavevmode
\sphinxstyleliteralstrong{\sphinxupquote{ValueError}} \textendash{} if module cannot retrieve data from MDS tree.

\end{description}\end{quote}

\end{fulllineitems}

\index{getQs() (eqtools.EFIT.EFITTree method)@\spxentry{getQs()}\spxextra{eqtools.EFIT.EFITTree method}}

\begin{fulllineitems}
\phantomsection\label{\detokenize{eqtools:eqtools.EFIT.EFITTree.getQs}}\pysiglinewithargsret{\sphinxbfcode{\sphinxupquote{getQs}}}{\emph{length\_unit=1}}{}
pulls q values.
\begin{quote}\begin{description}
\item[{Returns}] \leavevmode
namedtuple containing (q0,q95,qLCFS,rq1,rq2,rq3).

\item[{Raises}] \leavevmode
\sphinxstyleliteralstrong{\sphinxupquote{ValueError}} \textendash{} if module cannot retrieve data from MDS tree.

\end{description}\end{quote}

\end{fulllineitems}

\index{getBtVac() (eqtools.EFIT.EFITTree method)@\spxentry{getBtVac()}\spxextra{eqtools.EFIT.EFITTree method}}

\begin{fulllineitems}
\phantomsection\label{\detokenize{eqtools:eqtools.EFIT.EFITTree.getBtVac}}\pysiglinewithargsret{\sphinxbfcode{\sphinxupquote{getBtVac}}}{}{}
Returns vacuum toroidal field on-axis.
\begin{quote}\begin{description}
\item[{Returns}] \leavevmode
{[}nt{]} array of vacuum toroidal field.

\item[{Return type}] \leavevmode
BtVac (Array)

\item[{Raises}] \leavevmode
\sphinxstyleliteralstrong{\sphinxupquote{ValueError}} \textendash{} if module cannot retrieve data from MDS tree.

\end{description}\end{quote}

\end{fulllineitems}

\index{getBtPla() (eqtools.EFIT.EFITTree method)@\spxentry{getBtPla()}\spxextra{eqtools.EFIT.EFITTree method}}

\begin{fulllineitems}
\phantomsection\label{\detokenize{eqtools:eqtools.EFIT.EFITTree.getBtPla}}\pysiglinewithargsret{\sphinxbfcode{\sphinxupquote{getBtPla}}}{}{}
returns on-axis plasma toroidal field.
\begin{quote}\begin{description}
\item[{Returns}] \leavevmode
{[}nt{]} array of toroidal field including plasma effects.

\item[{Return type}] \leavevmode
BtPla (Array)

\item[{Raises}] \leavevmode
\sphinxstyleliteralstrong{\sphinxupquote{ValueError}} \textendash{} if module cannot retrieve data from MDS tree.

\end{description}\end{quote}

\end{fulllineitems}

\index{getBpAvg() (eqtools.EFIT.EFITTree method)@\spxentry{getBpAvg()}\spxextra{eqtools.EFIT.EFITTree method}}

\begin{fulllineitems}
\phantomsection\label{\detokenize{eqtools:eqtools.EFIT.EFITTree.getBpAvg}}\pysiglinewithargsret{\sphinxbfcode{\sphinxupquote{getBpAvg}}}{}{}
returns average poloidal field.
\begin{quote}\begin{description}
\item[{Returns}] \leavevmode
{[}nt{]} array of average poloidal field.

\item[{Return type}] \leavevmode
BpAvg (Array)

\item[{Raises}] \leavevmode
\sphinxstyleliteralstrong{\sphinxupquote{ValueError}} \textendash{} if module cannot retrieve data from MDS tree.

\end{description}\end{quote}

\end{fulllineitems}

\index{getFields() (eqtools.EFIT.EFITTree method)@\spxentry{getFields()}\spxextra{eqtools.EFIT.EFITTree method}}

\begin{fulllineitems}
\phantomsection\label{\detokenize{eqtools:eqtools.EFIT.EFITTree.getFields}}\pysiglinewithargsret{\sphinxbfcode{\sphinxupquote{getFields}}}{}{}
pulls vacuum and plasma toroidal field, avg poloidal field.
\begin{quote}\begin{description}
\item[{Returns}] \leavevmode
namedtuple containing (btaxv,btaxp,bpolav).

\item[{Raises}] \leavevmode
\sphinxstyleliteralstrong{\sphinxupquote{ValueError}} \textendash{} if module cannot retrieve data from MDS tree.

\end{description}\end{quote}

\end{fulllineitems}

\index{getIpCalc() (eqtools.EFIT.EFITTree method)@\spxentry{getIpCalc()}\spxextra{eqtools.EFIT.EFITTree method}}

\begin{fulllineitems}
\phantomsection\label{\detokenize{eqtools:eqtools.EFIT.EFITTree.getIpCalc}}\pysiglinewithargsret{\sphinxbfcode{\sphinxupquote{getIpCalc}}}{}{}
returns EFIT-calculated plasma current.
\begin{quote}\begin{description}
\item[{Returns}] \leavevmode
{[}nt{]} array of EFIT-reconstructed plasma current.

\item[{Return type}] \leavevmode
IpCalc (Array)

\item[{Raises}] \leavevmode
\sphinxstyleliteralstrong{\sphinxupquote{ValueError}} \textendash{} if module cannot retrieve data from MDS tree.

\end{description}\end{quote}

\end{fulllineitems}

\index{getIpMeas() (eqtools.EFIT.EFITTree method)@\spxentry{getIpMeas()}\spxextra{eqtools.EFIT.EFITTree method}}

\begin{fulllineitems}
\phantomsection\label{\detokenize{eqtools:eqtools.EFIT.EFITTree.getIpMeas}}\pysiglinewithargsret{\sphinxbfcode{\sphinxupquote{getIpMeas}}}{}{}
returns magnetics-measured plasma current.
\begin{quote}\begin{description}
\item[{Returns}] \leavevmode
{[}nt{]} array of measured plasma current.

\item[{Return type}] \leavevmode
IpMeas (Array)

\item[{Raises}] \leavevmode
\sphinxstyleliteralstrong{\sphinxupquote{ValueError}} \textendash{} if module cannot retrieve data from MDS tree.

\end{description}\end{quote}

\end{fulllineitems}

\index{getJp() (eqtools.EFIT.EFITTree method)@\spxentry{getJp()}\spxextra{eqtools.EFIT.EFITTree method}}

\begin{fulllineitems}
\phantomsection\label{\detokenize{eqtools:eqtools.EFIT.EFITTree.getJp}}\pysiglinewithargsret{\sphinxbfcode{\sphinxupquote{getJp}}}{}{}
returns EFIT-calculated plasma current density Jp on flux grid.
\begin{quote}\begin{description}
\item[{Returns}] \leavevmode
{[}nt,nz,nr{]} array of current density.

\item[{Return type}] \leavevmode
Jp (Array)

\item[{Raises}] \leavevmode
\sphinxstyleliteralstrong{\sphinxupquote{ValueError}} \textendash{} if module cannot retrieve data from MDS tree.

\end{description}\end{quote}

\end{fulllineitems}

\index{getBetaT() (eqtools.EFIT.EFITTree method)@\spxentry{getBetaT()}\spxextra{eqtools.EFIT.EFITTree method}}

\begin{fulllineitems}
\phantomsection\label{\detokenize{eqtools:eqtools.EFIT.EFITTree.getBetaT}}\pysiglinewithargsret{\sphinxbfcode{\sphinxupquote{getBetaT}}}{}{}
returns EFIT-calculated toroidal beta.
\begin{quote}\begin{description}
\item[{Returns}] \leavevmode
{[}nt{]} array of EFIT-calculated average toroidal beta.

\item[{Return type}] \leavevmode
BetaT (Array)

\item[{Raises}] \leavevmode
\sphinxstyleliteralstrong{\sphinxupquote{ValueError}} \textendash{} if module cannot retrieve data from MDS tree.

\end{description}\end{quote}

\end{fulllineitems}

\index{getBetaP() (eqtools.EFIT.EFITTree method)@\spxentry{getBetaP()}\spxextra{eqtools.EFIT.EFITTree method}}

\begin{fulllineitems}
\phantomsection\label{\detokenize{eqtools:eqtools.EFIT.EFITTree.getBetaP}}\pysiglinewithargsret{\sphinxbfcode{\sphinxupquote{getBetaP}}}{}{}
returns EFIT-calculated poloidal beta.
\begin{quote}\begin{description}
\item[{Returns}] \leavevmode
{[}nt{]} array of EFIT-calculated average poloidal beta.

\item[{Return type}] \leavevmode
BetaP (Array)

\item[{Raises}] \leavevmode
\sphinxstyleliteralstrong{\sphinxupquote{ValueError}} \textendash{} if module cannot retrieve data from MDS tree.

\end{description}\end{quote}

\end{fulllineitems}

\index{getLi() (eqtools.EFIT.EFITTree method)@\spxentry{getLi()}\spxextra{eqtools.EFIT.EFITTree method}}

\begin{fulllineitems}
\phantomsection\label{\detokenize{eqtools:eqtools.EFIT.EFITTree.getLi}}\pysiglinewithargsret{\sphinxbfcode{\sphinxupquote{getLi}}}{}{}
returns EFIT-calculated internal inductance.
\begin{quote}\begin{description}
\item[{Returns}] \leavevmode
{[}nt{]} array of EFIT-calculated internal inductance.

\item[{Return type}] \leavevmode
Li (Array)

\item[{Raises}] \leavevmode
\sphinxstyleliteralstrong{\sphinxupquote{ValueError}} \textendash{} if module cannot retrieve data from MDS tree.

\end{description}\end{quote}

\end{fulllineitems}

\index{getBetas() (eqtools.EFIT.EFITTree method)@\spxentry{getBetas()}\spxextra{eqtools.EFIT.EFITTree method}}

\begin{fulllineitems}
\phantomsection\label{\detokenize{eqtools:eqtools.EFIT.EFITTree.getBetas}}\pysiglinewithargsret{\sphinxbfcode{\sphinxupquote{getBetas}}}{}{}
pulls calculated betap, betat, internal inductance
\begin{quote}\begin{description}
\item[{Returns}] \leavevmode
namedtuple containing (betat,betap,Li)

\item[{Raises}] \leavevmode
\sphinxstyleliteralstrong{\sphinxupquote{ValueError}} \textendash{} if module cannot retrieve data from MDS tree.

\end{description}\end{quote}

\end{fulllineitems}

\index{getDiamagFlux() (eqtools.EFIT.EFITTree method)@\spxentry{getDiamagFlux()}\spxextra{eqtools.EFIT.EFITTree method}}

\begin{fulllineitems}
\phantomsection\label{\detokenize{eqtools:eqtools.EFIT.EFITTree.getDiamagFlux}}\pysiglinewithargsret{\sphinxbfcode{\sphinxupquote{getDiamagFlux}}}{}{}
returns measured diamagnetic-loop flux.
\begin{quote}\begin{description}
\item[{Returns}] \leavevmode
{[}nt{]} array of diamagnetic-loop flux.

\item[{Return type}] \leavevmode
Flux (Array)

\item[{Raises}] \leavevmode
\sphinxstyleliteralstrong{\sphinxupquote{ValueError}} \textendash{} if module cannot retrieve data from MDS tree.

\end{description}\end{quote}

\end{fulllineitems}

\index{getDiamagBetaT() (eqtools.EFIT.EFITTree method)@\spxentry{getDiamagBetaT()}\spxextra{eqtools.EFIT.EFITTree method}}

\begin{fulllineitems}
\phantomsection\label{\detokenize{eqtools:eqtools.EFIT.EFITTree.getDiamagBetaT}}\pysiglinewithargsret{\sphinxbfcode{\sphinxupquote{getDiamagBetaT}}}{}{}
returns diamagnetic-loop toroidal beta.
\begin{quote}\begin{description}
\item[{Returns}] \leavevmode
{[}nt{]} array of measured toroidal beta.

\item[{Return type}] \leavevmode
BetaT (Array)

\item[{Raises}] \leavevmode
\sphinxstyleliteralstrong{\sphinxupquote{ValueError}} \textendash{} if module cannot retrieve data from MDS tree.

\end{description}\end{quote}

\end{fulllineitems}

\index{getDiamagBetaP() (eqtools.EFIT.EFITTree method)@\spxentry{getDiamagBetaP()}\spxextra{eqtools.EFIT.EFITTree method}}

\begin{fulllineitems}
\phantomsection\label{\detokenize{eqtools:eqtools.EFIT.EFITTree.getDiamagBetaP}}\pysiglinewithargsret{\sphinxbfcode{\sphinxupquote{getDiamagBetaP}}}{}{}
returns diamagnetic-loop avg poloidal beta.
\begin{quote}\begin{description}
\item[{Returns}] \leavevmode
{[}nt{]} array of measured poloidal beta.

\item[{Return type}] \leavevmode
BetaP (Array)

\item[{Raises}] \leavevmode
\sphinxstyleliteralstrong{\sphinxupquote{ValueError}} \textendash{} if module cannot retrieve data from MDS tree.

\end{description}\end{quote}

\end{fulllineitems}

\index{getDiamagTauE() (eqtools.EFIT.EFITTree method)@\spxentry{getDiamagTauE()}\spxextra{eqtools.EFIT.EFITTree method}}

\begin{fulllineitems}
\phantomsection\label{\detokenize{eqtools:eqtools.EFIT.EFITTree.getDiamagTauE}}\pysiglinewithargsret{\sphinxbfcode{\sphinxupquote{getDiamagTauE}}}{}{}
returns diamagnetic-loop energy confinement time.
\begin{quote}\begin{description}
\item[{Returns}] \leavevmode
{[}nt{]} array of measured energy confinement time.

\item[{Return type}] \leavevmode
tauE (Array)

\item[{Raises}] \leavevmode
\sphinxstyleliteralstrong{\sphinxupquote{ValueError}} \textendash{} if module cannot retrieve data from MDS tree.

\end{description}\end{quote}

\end{fulllineitems}

\index{getDiamagWp() (eqtools.EFIT.EFITTree method)@\spxentry{getDiamagWp()}\spxextra{eqtools.EFIT.EFITTree method}}

\begin{fulllineitems}
\phantomsection\label{\detokenize{eqtools:eqtools.EFIT.EFITTree.getDiamagWp}}\pysiglinewithargsret{\sphinxbfcode{\sphinxupquote{getDiamagWp}}}{}{}
returns diamagnetic-loop plasma stored energy.
\begin{quote}\begin{description}
\item[{Returns}] \leavevmode
{[}nt{]} array of measured plasma stored energy.

\item[{Return type}] \leavevmode
Wp (Array)

\item[{Raises}] \leavevmode
\sphinxstyleliteralstrong{\sphinxupquote{ValueError}} \textendash{} if module cannot retrieve data from MDS tree.

\end{description}\end{quote}

\end{fulllineitems}

\index{getDiamag() (eqtools.EFIT.EFITTree method)@\spxentry{getDiamag()}\spxextra{eqtools.EFIT.EFITTree method}}

\begin{fulllineitems}
\phantomsection\label{\detokenize{eqtools:eqtools.EFIT.EFITTree.getDiamag}}\pysiglinewithargsret{\sphinxbfcode{\sphinxupquote{getDiamag}}}{}{}
pulls diamagnetic flux measurements, toroidal and poloidal beta,
energy confinement time and stored energy.
\begin{quote}\begin{description}
\item[{Returns}] \leavevmode
namedtuple containing (diamag. flux, betatd, betapd, tauDiamag, WDiamag)

\item[{Raises}] \leavevmode
\sphinxstyleliteralstrong{\sphinxupquote{ValueError}} \textendash{} if module cannot retrieve data from MDS tree.

\end{description}\end{quote}

\end{fulllineitems}

\index{getWMHD() (eqtools.EFIT.EFITTree method)@\spxentry{getWMHD()}\spxextra{eqtools.EFIT.EFITTree method}}

\begin{fulllineitems}
\phantomsection\label{\detokenize{eqtools:eqtools.EFIT.EFITTree.getWMHD}}\pysiglinewithargsret{\sphinxbfcode{\sphinxupquote{getWMHD}}}{}{}
returns EFIT-calculated MHD stored energy.
\begin{quote}\begin{description}
\item[{Returns}] \leavevmode
{[}nt{]} array of EFIT-calculated stored energy.

\item[{Return type}] \leavevmode
WMHD (Array)

\item[{Raises}] \leavevmode
\sphinxstyleliteralstrong{\sphinxupquote{ValueError}} \textendash{} if module cannot retrieve data from MDS tree.

\end{description}\end{quote}

\end{fulllineitems}

\index{getTauMHD() (eqtools.EFIT.EFITTree method)@\spxentry{getTauMHD()}\spxextra{eqtools.EFIT.EFITTree method}}

\begin{fulllineitems}
\phantomsection\label{\detokenize{eqtools:eqtools.EFIT.EFITTree.getTauMHD}}\pysiglinewithargsret{\sphinxbfcode{\sphinxupquote{getTauMHD}}}{}{}
returns EFIT-calculated MHD energy confinement time.
\begin{quote}\begin{description}
\item[{Returns}] \leavevmode
{[}nt{]} array of EFIT-calculated energy confinement time.

\item[{Return type}] \leavevmode
tauMHD (Array)

\item[{Raises}] \leavevmode
\sphinxstyleliteralstrong{\sphinxupquote{ValueError}} \textendash{} if module cannot retrieve data from MDS tree.

\end{description}\end{quote}

\end{fulllineitems}

\index{getPinj() (eqtools.EFIT.EFITTree method)@\spxentry{getPinj()}\spxextra{eqtools.EFIT.EFITTree method}}

\begin{fulllineitems}
\phantomsection\label{\detokenize{eqtools:eqtools.EFIT.EFITTree.getPinj}}\pysiglinewithargsret{\sphinxbfcode{\sphinxupquote{getPinj}}}{}{}
returns EFIT-calculated injected power.
\begin{quote}\begin{description}
\item[{Returns}] \leavevmode
{[}nt{]} array of EFIT-reconstructed injected power.

\item[{Return type}] \leavevmode
Pinj (Array)

\item[{Raises}] \leavevmode
\sphinxstyleliteralstrong{\sphinxupquote{ValueError}} \textendash{} if module cannot retrieve data from MDS tree.

\end{description}\end{quote}

\end{fulllineitems}

\index{getWbdot() (eqtools.EFIT.EFITTree method)@\spxentry{getWbdot()}\spxextra{eqtools.EFIT.EFITTree method}}

\begin{fulllineitems}
\phantomsection\label{\detokenize{eqtools:eqtools.EFIT.EFITTree.getWbdot}}\pysiglinewithargsret{\sphinxbfcode{\sphinxupquote{getWbdot}}}{}{}
returns EFIT-calculated d/dt of magnetic stored energy.
\begin{quote}\begin{description}
\item[{Returns}] \leavevmode
{[}nt{]} array of d(Wb)/dt

\item[{Return type}] \leavevmode
dWdt (Array)

\item[{Raises}] \leavevmode
\sphinxstyleliteralstrong{\sphinxupquote{ValueError}} \textendash{} if module cannot retrieve data from MDS tree.

\end{description}\end{quote}

\end{fulllineitems}

\index{getWpdot() (eqtools.EFIT.EFITTree method)@\spxentry{getWpdot()}\spxextra{eqtools.EFIT.EFITTree method}}

\begin{fulllineitems}
\phantomsection\label{\detokenize{eqtools:eqtools.EFIT.EFITTree.getWpdot}}\pysiglinewithargsret{\sphinxbfcode{\sphinxupquote{getWpdot}}}{}{}
returns EFIT-calculated d/dt of plasma stored energy.
\begin{quote}\begin{description}
\item[{Returns}] \leavevmode
{[}nt{]} array of d(Wp)/dt

\item[{Return type}] \leavevmode
dWdt (Array)

\item[{Raises}] \leavevmode
\sphinxstyleliteralstrong{\sphinxupquote{ValueError}} \textendash{} if module cannot retrieve data from MDS tree.

\end{description}\end{quote}

\end{fulllineitems}

\index{getBCentr() (eqtools.EFIT.EFITTree method)@\spxentry{getBCentr()}\spxextra{eqtools.EFIT.EFITTree method}}

\begin{fulllineitems}
\phantomsection\label{\detokenize{eqtools:eqtools.EFIT.EFITTree.getBCentr}}\pysiglinewithargsret{\sphinxbfcode{\sphinxupquote{getBCentr}}}{}{}
returns EFIT-Vacuum toroidal magnetic field in Tesla at Rcentr
\begin{quote}\begin{description}
\item[{Returns}] \leavevmode
{[}nt{]} array of B\_t at center {[}T{]}

\item[{Return type}] \leavevmode
B\_cent (Array)

\item[{Raises}] \leavevmode
\sphinxstyleliteralstrong{\sphinxupquote{ValueError}} \textendash{} if module cannot retrieve data from MDS tree.

\end{description}\end{quote}

\end{fulllineitems}

\index{getRCentr() (eqtools.EFIT.EFITTree method)@\spxentry{getRCentr()}\spxextra{eqtools.EFIT.EFITTree method}}

\begin{fulllineitems}
\phantomsection\label{\detokenize{eqtools:eqtools.EFIT.EFITTree.getRCentr}}\pysiglinewithargsret{\sphinxbfcode{\sphinxupquote{getRCentr}}}{\emph{length\_unit=1}}{}
returns EFIT radius where Bcentr evaluated
\begin{quote}\begin{description}
\item[{Returns}] \leavevmode
Radial position where Bcent calculated {[}m{]}

\item[{Return type}] \leavevmode
R

\item[{Raises}] \leavevmode
\sphinxstyleliteralstrong{\sphinxupquote{ValueError}} \textendash{} if module cannot retrieve data from MDS tree.

\end{description}\end{quote}

\end{fulllineitems}

\index{getEnergy() (eqtools.EFIT.EFITTree method)@\spxentry{getEnergy()}\spxextra{eqtools.EFIT.EFITTree method}}

\begin{fulllineitems}
\phantomsection\label{\detokenize{eqtools:eqtools.EFIT.EFITTree.getEnergy}}\pysiglinewithargsret{\sphinxbfcode{\sphinxupquote{getEnergy}}}{}{}
pulls EFIT-calculated energy parameters - stored energy, tau\_E,
injected power, d/dt of magnetic and plasma stored energy.
\begin{quote}\begin{description}
\item[{Returns}] \leavevmode
namedtuple containing (WMHD,tauMHD,Pinj,Wbdot,Wpdot)

\item[{Raises}] \leavevmode
\sphinxstyleliteralstrong{\sphinxupquote{ValueError}} \textendash{} if module cannot retrieve data from MDS tree.

\end{description}\end{quote}

\end{fulllineitems}

\index{getMachineCrossSection() (eqtools.EFIT.EFITTree method)@\spxentry{getMachineCrossSection()}\spxextra{eqtools.EFIT.EFITTree method}}

\begin{fulllineitems}
\phantomsection\label{\detokenize{eqtools:eqtools.EFIT.EFITTree.getMachineCrossSection}}\pysiglinewithargsret{\sphinxbfcode{\sphinxupquote{getMachineCrossSection}}}{}{}
Returns R,Z coordinates of vacuum-vessel wall for masking, plotting
routines.
\begin{quote}\begin{description}
\item[{Returns}] \leavevmode

(\sphinxtitleref{R\_limiter}, \sphinxtitleref{Z\_limiter})
\begin{itemize}
\item {} 
\sphinxstylestrong{R\_limiter} (\sphinxtitleref{Array}) - {[}n{]} array of x-values for machine cross-section.

\item {} 
\sphinxstylestrong{Z\_limiter} (\sphinxtitleref{Array}) - {[}n{]} array of y-values for machine cross-section.

\end{itemize}


\end{description}\end{quote}

\end{fulllineitems}

\index{getMachineCrossSectionFull() (eqtools.EFIT.EFITTree method)@\spxentry{getMachineCrossSectionFull()}\spxextra{eqtools.EFIT.EFITTree method}}

\begin{fulllineitems}
\phantomsection\label{\detokenize{eqtools:eqtools.EFIT.EFITTree.getMachineCrossSectionFull}}\pysiglinewithargsret{\sphinxbfcode{\sphinxupquote{getMachineCrossSectionFull}}}{}{}
Returns R,Z coordinates of vacuum-vessel wall for plotting routines.

Absent additional vector-graphic data on machine cross-section, returns
{\hyperref[\detokenize{eqtools:eqtools.EFIT.EFITTree.getMachineCrossSection}]{\sphinxcrossref{\sphinxcode{\sphinxupquote{getMachineCrossSection()}}}}}.
\begin{quote}\begin{description}
\item[{Returns}] \leavevmode
result from getMachineCrossSection().

\end{description}\end{quote}

\end{fulllineitems}

\index{getCurrentSign() (eqtools.EFIT.EFITTree method)@\spxentry{getCurrentSign()}\spxextra{eqtools.EFIT.EFITTree method}}

\begin{fulllineitems}
\phantomsection\label{\detokenize{eqtools:eqtools.EFIT.EFITTree.getCurrentSign}}\pysiglinewithargsret{\sphinxbfcode{\sphinxupquote{getCurrentSign}}}{}{}
Returns the sign of the current, based on the check in Steve Wolfe’s
IDL implementation efit\_rz2psi.pro.
\begin{quote}\begin{description}
\item[{Returns}] \leavevmode
1 for positive-direction current, -1 for negative.

\item[{Return type}] \leavevmode
currentSign (Integer)

\end{description}\end{quote}

\end{fulllineitems}

\index{getParam() (eqtools.EFIT.EFITTree method)@\spxentry{getParam()}\spxextra{eqtools.EFIT.EFITTree method}}

\begin{fulllineitems}
\phantomsection\label{\detokenize{eqtools:eqtools.EFIT.EFITTree.getParam}}\pysiglinewithargsret{\sphinxbfcode{\sphinxupquote{getParam}}}{\emph{path}}{}
Backup function, applying a direct path input for tree-like data
storage access for parameters not typically found in
\sphinxcode{\sphinxupquote{Equilbrium}} object.
Directly calls attributes read from g/a-files in copy-safe manner.
\begin{quote}\begin{description}
\item[{Parameters}] \leavevmode
\sphinxstyleliteralstrong{\sphinxupquote{name}} (\sphinxstyleliteralemphasis{\sphinxupquote{String}}) \textendash{} Parameter name for value stored in EqdskReader
instance.

\item[{Raises}] \leavevmode
\sphinxstyleliteralstrong{\sphinxupquote{AttributeError}} \textendash{} raised if no attribute is found.

\end{description}\end{quote}

\end{fulllineitems}


\end{fulllineitems}



\subsection{eqtools.FromArrays module}
\label{\detokenize{eqtools:module-eqtools.FromArrays}}\label{\detokenize{eqtools:eqtools-fromarrays-module}}\index{eqtools.FromArrays (module)@\spxentry{eqtools.FromArrays}\spxextra{module}}\index{ArrayEquilibrium (class in eqtools.FromArrays)@\spxentry{ArrayEquilibrium}\spxextra{class in eqtools.FromArrays}}

\begin{fulllineitems}
\phantomsection\label{\detokenize{eqtools:eqtools.FromArrays.ArrayEquilibrium}}\pysiglinewithargsret{\sphinxbfcode{\sphinxupquote{class }}\sphinxcode{\sphinxupquote{eqtools.FromArrays.}}\sphinxbfcode{\sphinxupquote{ArrayEquilibrium}}}{\emph{psiRZ}, \emph{rGrid}, \emph{zGrid}, \emph{time}, \emph{q}, \emph{fluxVol}, \emph{psiLCFS}, \emph{psiAxis}, \emph{rmag}, \emph{zmag}, \emph{Rout}, \emph{**kwargs}}{}
Bases: {\hyperref[\detokenize{eqtools:eqtools.core.Equilibrium}]{\sphinxcrossref{\sphinxcode{\sphinxupquote{eqtools.core.Equilibrium}}}}}

Class to represent an equilibrium specified as arrays of data.

Create ArrayEquilibrium instance from arrays of data.

Has very little checking on the shape/type of the arrays at this point.
\begin{quote}\begin{description}
\item[{Parameters}] \leavevmode\begin{itemize}
\item {} 
\sphinxstyleliteralstrong{\sphinxupquote{psiRZ}} \textendash{} Array-like, (M, N, P).
Flux values at M times, N Z locations and P R locations.

\item {} 
\sphinxstyleliteralstrong{\sphinxupquote{rGrid}} \textendash{} Array-like, (P,).
R coordinates that psiRZ is given at.

\item {} 
\sphinxstyleliteralstrong{\sphinxupquote{zGrid}} \textendash{} Array-like, (N,).
Z coordinates that psiRZ is given at.

\item {} 
\sphinxstyleliteralstrong{\sphinxupquote{time}} \textendash{} Array-like, (M,).
Times that psiRZ is given at.

\item {} 
\sphinxstyleliteralstrong{\sphinxupquote{q}} \textendash{} Array-like, (S, M).
q profile evaluated at S values of psinorm from 0 to 1, given at M
times.

\item {} 
\sphinxstyleliteralstrong{\sphinxupquote{fluxVol}} \textendash{} Array-like, (S, M).
Flux surface volumes evaluated at S values of psinorm from 0 to 1,
given at M times.

\item {} 
\sphinxstyleliteralstrong{\sphinxupquote{psiLCFS}} \textendash{} Array-like, (M,).
Flux at the last closed flux surface, given at M times.

\item {} 
\sphinxstyleliteralstrong{\sphinxupquote{psiAxis}} \textendash{} Array-like, (M,).
Flux at the magnetic axis, given at M times.

\item {} 
\sphinxstyleliteralstrong{\sphinxupquote{rmag}} \textendash{} Array-like, (M,).
Radial coordinate of the magnetic axis, given at M times.

\item {} 
\sphinxstyleliteralstrong{\sphinxupquote{zmag}} \textendash{} Array-like, (M,).
Vertical coordinate of the magnetic axis, given at M times.

\item {} 
\sphinxstyleliteralstrong{\sphinxupquote{Rout}} \textendash{} Outboard midplane radius of the last closed flux surface.

\end{itemize}

\item[{Keyword Arguments}] \leavevmode\begin{itemize}
\item {} 
\sphinxstyleliteralstrong{\sphinxupquote{length\_unit}} \textendash{} 
String.
Sets the base unit used for any quantity whose
dimensions are length to any power. Valid options are:
\begin{quote}


\begin{savenotes}\sphinxattablestart
\centering
\begin{tabulary}{\linewidth}[t]{|T|T|}
\hline

’m’
&
meters
\\
\hline
’cm’
&
centimeters
\\
\hline
’mm’
&
millimeters
\\
\hline
’in’
&
inches
\\
\hline
’ft’
&
feet
\\
\hline
’yd’
&
yards
\\
\hline
’smoot’
&
smoots
\\
\hline
’cubit’
&
cubits
\\
\hline
’hand’
&
hands
\\
\hline
’default’
&
whatever the default in the tree is (no conversion is performed, units may be inconsistent)
\\
\hline
\end{tabulary}
\par
\sphinxattableend\end{savenotes}
\end{quote}

Default is ‘m’ (all units taken and returned in meters).


\item {} 
\sphinxstyleliteralstrong{\sphinxupquote{tspline}} \textendash{} Boolean.
Sets whether or not interpolation in time is
performed using a tricubic spline or nearest-neighbor
interpolation. Tricubic spline interpolation requires at least
four complete equilibria at different times. It is also assumed
that they are functionally correlated, and that parameters do
not vary out of their boundaries (derivative = 0 boundary
condition). Default is False (use nearest neighbor interpolation).

\item {} 
\sphinxstyleliteralstrong{\sphinxupquote{monotonic}} \textendash{} Boolean.
Sets whether or not the “monotonic” form of time window
finding is used. If True, the timebase must be monotonically
increasing. Default is False (use slower, safer method).

\item {} 
\sphinxstyleliteralstrong{\sphinxupquote{verbose}} \textendash{} Boolean.
Allows or blocks console readout during operation.  Defaults to True,
displaying useful information for the user.  Set to False for quiet
usage or to avoid console clutter for multiple instances.

\end{itemize}

\end{description}\end{quote}
\index{getTimeBase() (eqtools.FromArrays.ArrayEquilibrium method)@\spxentry{getTimeBase()}\spxextra{eqtools.FromArrays.ArrayEquilibrium method}}

\begin{fulllineitems}
\phantomsection\label{\detokenize{eqtools:eqtools.FromArrays.ArrayEquilibrium.getTimeBase}}\pysiglinewithargsret{\sphinxbfcode{\sphinxupquote{getTimeBase}}}{}{}
Returns a copy of the time base vector, array dimensions are (M,).

\end{fulllineitems}

\index{getFluxGrid() (eqtools.FromArrays.ArrayEquilibrium method)@\spxentry{getFluxGrid()}\spxextra{eqtools.FromArrays.ArrayEquilibrium method}}

\begin{fulllineitems}
\phantomsection\label{\detokenize{eqtools:eqtools.FromArrays.ArrayEquilibrium.getFluxGrid}}\pysiglinewithargsret{\sphinxbfcode{\sphinxupquote{getFluxGrid}}}{}{}
Returns a copy of the flux array, dimensions are (M, N, P), corresponding to (time, Z, R).

\end{fulllineitems}

\index{getRGrid() (eqtools.FromArrays.ArrayEquilibrium method)@\spxentry{getRGrid()}\spxextra{eqtools.FromArrays.ArrayEquilibrium method}}

\begin{fulllineitems}
\phantomsection\label{\detokenize{eqtools:eqtools.FromArrays.ArrayEquilibrium.getRGrid}}\pysiglinewithargsret{\sphinxbfcode{\sphinxupquote{getRGrid}}}{\emph{length\_unit=1}}{}
Returns a copy of the radial grid, dimensions are (P,).

\end{fulllineitems}

\index{getZGrid() (eqtools.FromArrays.ArrayEquilibrium method)@\spxentry{getZGrid()}\spxextra{eqtools.FromArrays.ArrayEquilibrium method}}

\begin{fulllineitems}
\phantomsection\label{\detokenize{eqtools:eqtools.FromArrays.ArrayEquilibrium.getZGrid}}\pysiglinewithargsret{\sphinxbfcode{\sphinxupquote{getZGrid}}}{\emph{length\_unit=1}}{}
Returns a copy of the vertical grid, dimensions are (N,).

\end{fulllineitems}

\index{getQProfile() (eqtools.FromArrays.ArrayEquilibrium method)@\spxentry{getQProfile()}\spxextra{eqtools.FromArrays.ArrayEquilibrium method}}

\begin{fulllineitems}
\phantomsection\label{\detokenize{eqtools:eqtools.FromArrays.ArrayEquilibrium.getQProfile}}\pysiglinewithargsret{\sphinxbfcode{\sphinxupquote{getQProfile}}}{}{}
Returns safety factor q profile (over Q values of psinorm from 0 to 1), dimensions are (Q, M)

\end{fulllineitems}

\index{getFluxVol() (eqtools.FromArrays.ArrayEquilibrium method)@\spxentry{getFluxVol()}\spxextra{eqtools.FromArrays.ArrayEquilibrium method}}

\begin{fulllineitems}
\phantomsection\label{\detokenize{eqtools:eqtools.FromArrays.ArrayEquilibrium.getFluxVol}}\pysiglinewithargsret{\sphinxbfcode{\sphinxupquote{getFluxVol}}}{\emph{length\_unit=3}}{}
returns volume within flux surface {[}psi,t{]}

\end{fulllineitems}

\index{getFluxLCFS() (eqtools.FromArrays.ArrayEquilibrium method)@\spxentry{getFluxLCFS()}\spxextra{eqtools.FromArrays.ArrayEquilibrium method}}

\begin{fulllineitems}
\phantomsection\label{\detokenize{eqtools:eqtools.FromArrays.ArrayEquilibrium.getFluxLCFS}}\pysiglinewithargsret{\sphinxbfcode{\sphinxupquote{getFluxLCFS}}}{}{}
returns psi at separatrix {[}t{]}

\end{fulllineitems}

\index{getFluxAxis() (eqtools.FromArrays.ArrayEquilibrium method)@\spxentry{getFluxAxis()}\spxextra{eqtools.FromArrays.ArrayEquilibrium method}}

\begin{fulllineitems}
\phantomsection\label{\detokenize{eqtools:eqtools.FromArrays.ArrayEquilibrium.getFluxAxis}}\pysiglinewithargsret{\sphinxbfcode{\sphinxupquote{getFluxAxis}}}{}{}
returns psi on magnetic axis {[}t{]}

\end{fulllineitems}

\index{getMagR() (eqtools.FromArrays.ArrayEquilibrium method)@\spxentry{getMagR()}\spxextra{eqtools.FromArrays.ArrayEquilibrium method}}

\begin{fulllineitems}
\phantomsection\label{\detokenize{eqtools:eqtools.FromArrays.ArrayEquilibrium.getMagR}}\pysiglinewithargsret{\sphinxbfcode{\sphinxupquote{getMagR}}}{\emph{length\_unit=1}}{}
returns magnetic-axis major radius {[}t{]}

\end{fulllineitems}

\index{getMagZ() (eqtools.FromArrays.ArrayEquilibrium method)@\spxentry{getMagZ()}\spxextra{eqtools.FromArrays.ArrayEquilibrium method}}

\begin{fulllineitems}
\phantomsection\label{\detokenize{eqtools:eqtools.FromArrays.ArrayEquilibrium.getMagZ}}\pysiglinewithargsret{\sphinxbfcode{\sphinxupquote{getMagZ}}}{\emph{length\_unit=1}}{}
returns magnetic-axis Z {[}t{]}

\end{fulllineitems}

\index{getRmidOut() (eqtools.FromArrays.ArrayEquilibrium method)@\spxentry{getRmidOut()}\spxextra{eqtools.FromArrays.ArrayEquilibrium method}}

\begin{fulllineitems}
\phantomsection\label{\detokenize{eqtools:eqtools.FromArrays.ArrayEquilibrium.getRmidOut}}\pysiglinewithargsret{\sphinxbfcode{\sphinxupquote{getRmidOut}}}{\emph{length\_unit=1}}{}
returns outboard-midplane major radius {[}t{]}

\end{fulllineitems}

\index{getRLCFS() (eqtools.FromArrays.ArrayEquilibrium method)@\spxentry{getRLCFS()}\spxextra{eqtools.FromArrays.ArrayEquilibrium method}}

\begin{fulllineitems}
\phantomsection\label{\detokenize{eqtools:eqtools.FromArrays.ArrayEquilibrium.getRLCFS}}\pysiglinewithargsret{\sphinxbfcode{\sphinxupquote{getRLCFS}}}{\emph{length\_unit=1}}{}
Abstract method.  See child classes for implementation.

Returns R-positions (n points) mapping LCFS {[}t,n{]}

\end{fulllineitems}

\index{getZLCFS() (eqtools.FromArrays.ArrayEquilibrium method)@\spxentry{getZLCFS()}\spxextra{eqtools.FromArrays.ArrayEquilibrium method}}

\begin{fulllineitems}
\phantomsection\label{\detokenize{eqtools:eqtools.FromArrays.ArrayEquilibrium.getZLCFS}}\pysiglinewithargsret{\sphinxbfcode{\sphinxupquote{getZLCFS}}}{\emph{length\_unit=1}}{}
Abstract method.  See child classes for implementation.

Returns Z-positions (n points) mapping LCFS {[}t,n{]}

\end{fulllineitems}

\index{getCurrentSign() (eqtools.FromArrays.ArrayEquilibrium method)@\spxentry{getCurrentSign()}\spxextra{eqtools.FromArrays.ArrayEquilibrium method}}

\begin{fulllineitems}
\phantomsection\label{\detokenize{eqtools:eqtools.FromArrays.ArrayEquilibrium.getCurrentSign}}\pysiglinewithargsret{\sphinxbfcode{\sphinxupquote{getCurrentSign}}}{}{}
Abstract method.  See child classes for implementation.

Returns calculated current direction, where CCW = +

\end{fulllineitems}


\end{fulllineitems}



\subsection{eqtools.NSTXEFIT module}
\label{\detokenize{eqtools:module-eqtools.NSTXEFIT}}\label{\detokenize{eqtools:eqtools-nstxefit-module}}\index{eqtools.NSTXEFIT (module)@\spxentry{eqtools.NSTXEFIT}\spxextra{module}}
This module provides classes inheriting {\hyperref[\detokenize{eqtools:eqtools.EFIT.EFITTree}]{\sphinxcrossref{\sphinxcode{\sphinxupquote{eqtools.EFIT.EFITTree}}}}} for
working with NSTX EFIT data.
\index{NSTXEFITTree (class in eqtools.NSTXEFIT)@\spxentry{NSTXEFITTree}\spxextra{class in eqtools.NSTXEFIT}}

\begin{fulllineitems}
\phantomsection\label{\detokenize{eqtools:eqtools.NSTXEFIT.NSTXEFITTree}}\pysiglinewithargsret{\sphinxbfcode{\sphinxupquote{class }}\sphinxcode{\sphinxupquote{eqtools.NSTXEFIT.}}\sphinxbfcode{\sphinxupquote{NSTXEFITTree}}}{\emph{shot}, \emph{tree='EFIT01'}, \emph{length\_unit='m'}, \emph{gfile='geqdsk'}, \emph{afile='aeqdsk'}, \emph{tspline=False}, \emph{monotonic=True}}{}
Bases: {\hyperref[\detokenize{eqtools:eqtools.EFIT.EFITTree}]{\sphinxcrossref{\sphinxcode{\sphinxupquote{eqtools.EFIT.EFITTree}}}}}

Inherits \sphinxcode{\sphinxupquote{EFITTree}} class. Machine-specific data
handling class for the National Spherical Torus Experiment (NSTX). Pulls EFIT
data from selected MDS tree and shot, stores as object attributes. Each EFIT
variable or set of variables is recovered with a corresponding getter method.
Essential data for EFIT mapping are pulled on initialization (e.g. psirz grid).
Additional data are pulled at the first request and stored for subsequent usage.

Intializes NSTX version of EFITTree object.  Pulls data from MDS tree for storage
in instance attributes.  Core attributes are populated from the MDS tree on initialization.
Additional attributes are initialized as None, filled on the first request to the object.
\begin{quote}\begin{description}
\item[{Parameters}] \leavevmode
\sphinxstyleliteralstrong{\sphinxupquote{shot}} (\sphinxstyleliteralemphasis{\sphinxupquote{integer}}) \textendash{} NSTX shot index (long)

\item[{Keyword Arguments}] \leavevmode\begin{itemize}
\item {} 
\sphinxstyleliteralstrong{\sphinxupquote{tree}} (\sphinxstyleliteralemphasis{\sphinxupquote{string}}) \textendash{} Optional input for EFIT tree, defaults to ‘EFIT01’
(i.e., EFIT data are under EFIT01::top.results).

\item {} 
\sphinxstyleliteralstrong{\sphinxupquote{length\_unit}} (\sphinxstyleliteralemphasis{\sphinxupquote{string}}) \textendash{} 
Sets the base unit used for any quantity whose
dimensions are length to any power. Valid options are:
\begin{quote}


\begin{savenotes}\sphinxattablestart
\centering
\begin{tabulary}{\linewidth}[t]{|T|T|}
\hline

’m’
&
meters
\\
\hline
’cm’
&
centimeters
\\
\hline
’mm’
&
millimeters
\\
\hline
’in’
&
inches
\\
\hline
’ft’
&
feet
\\
\hline
’yd’
&
yards
\\
\hline
’smoot’
&
smoots
\\
\hline
’cubit’
&
cubits
\\
\hline
’hand’
&
hands
\\
\hline
’default’
&
whatever the default in the tree is (no conversion is performed, units may be inconsistent)
\\
\hline
\end{tabulary}
\par
\sphinxattableend\end{savenotes}
\end{quote}

Default is ‘m’ (all units taken and returned in meters).


\item {} 
\sphinxstyleliteralstrong{\sphinxupquote{gfile}} (\sphinxstyleliteralemphasis{\sphinxupquote{string}}) \textendash{} Optional input for EFIT geqdsk location name,
defaults to ‘geqdsk’ (i.e., EFIT data are under
tree::top.results.GEQDSK)

\item {} 
\sphinxstyleliteralstrong{\sphinxupquote{afile}} (\sphinxstyleliteralemphasis{\sphinxupquote{string}}) \textendash{} Optional input for EFIT aeqdsk location name,
defaults to ‘aeqdsk’ (i.e., EFIT data are under
tree::top.results.AEQDSK)

\item {} 
\sphinxstyleliteralstrong{\sphinxupquote{tspline}} (\sphinxstyleliteralemphasis{\sphinxupquote{Boolean}}) \textendash{} Sets whether or not interpolation in time is
performed using a tricubic spline or nearest-neighbor
interpolation. Tricubic spline interpolation requires at least
four complete equilibria at different times. It is also assumed
that they are functionally correlated, and that parameters do
not vary out of their boundaries (derivative = 0 boundary
condition). Default is False (use nearest neighbor interpolation).

\item {} 
\sphinxstyleliteralstrong{\sphinxupquote{monotonic}} (\sphinxstyleliteralemphasis{\sphinxupquote{Boolean}}) \textendash{} Sets whether or not the “monotonic” form of time
window finding is used. If True, the timebase must be monotonically
increasing. Default is False (use slower, safer method).

\end{itemize}

\end{description}\end{quote}
\index{getFluxGrid() (eqtools.NSTXEFIT.NSTXEFITTree method)@\spxentry{getFluxGrid()}\spxextra{eqtools.NSTXEFIT.NSTXEFITTree method}}

\begin{fulllineitems}
\phantomsection\label{\detokenize{eqtools:eqtools.NSTXEFIT.NSTXEFITTree.getFluxGrid}}\pysiglinewithargsret{\sphinxbfcode{\sphinxupquote{getFluxGrid}}}{}{}
returns EFIT flux grid.
\begin{quote}\begin{description}
\item[{Returns}] \leavevmode
{[}nt,nz,nr{]} array of (non-normalized) flux on grid.

\item[{Return type}] \leavevmode
psiRZ (Array)

\item[{Raises}] \leavevmode
\sphinxstyleliteralstrong{\sphinxupquote{ValueError}} \textendash{} if module cannot retrieve data from MDS tree.

\end{description}\end{quote}

\end{fulllineitems}

\index{getMachineCrossSection() (eqtools.NSTXEFIT.NSTXEFITTree method)@\spxentry{getMachineCrossSection()}\spxextra{eqtools.NSTXEFIT.NSTXEFITTree method}}

\begin{fulllineitems}
\phantomsection\label{\detokenize{eqtools:eqtools.NSTXEFIT.NSTXEFITTree.getMachineCrossSection}}\pysiglinewithargsret{\sphinxbfcode{\sphinxupquote{getMachineCrossSection}}}{}{}
Returns R,Z coordinates of vacuum-vessel wall for masking, plotting routines.
\begin{quote}\begin{description}
\item[{Returns}] \leavevmode
The requested data.

\end{description}\end{quote}

\end{fulllineitems}

\index{getFluxVol() (eqtools.NSTXEFIT.NSTXEFITTree method)@\spxentry{getFluxVol()}\spxextra{eqtools.NSTXEFIT.NSTXEFITTree method}}

\begin{fulllineitems}
\phantomsection\label{\detokenize{eqtools:eqtools.NSTXEFIT.NSTXEFITTree.getFluxVol}}\pysiglinewithargsret{\sphinxbfcode{\sphinxupquote{getFluxVol}}}{}{}
Not implemented in NSTXEFIT tree.
\begin{quote}\begin{description}
\item[{Returns}] \leavevmode
volume within flux surface {[}psi,t{]}

\end{description}\end{quote}

\end{fulllineitems}

\index{getRmidPsi() (eqtools.NSTXEFIT.NSTXEFITTree method)@\spxentry{getRmidPsi()}\spxextra{eqtools.NSTXEFIT.NSTXEFITTree method}}

\begin{fulllineitems}
\phantomsection\label{\detokenize{eqtools:eqtools.NSTXEFIT.NSTXEFITTree.getRmidPsi}}\pysiglinewithargsret{\sphinxbfcode{\sphinxupquote{getRmidPsi}}}{\emph{length\_unit=1}}{}
returns maximum major radius of each flux surface.
\begin{quote}\begin{description}
\item[{Keyword Arguments}] \leavevmode
\sphinxstyleliteralstrong{\sphinxupquote{length\_unit}} (\sphinxstyleliteralemphasis{\sphinxupquote{String}}\sphinxstyleliteralemphasis{\sphinxupquote{ or }}\sphinxstyleliteralemphasis{\sphinxupquote{1}}) \textendash{} unit of Rmid.  Defaults to 1, indicating
the default parameter unit (typically m).

\item[{Returns}] \leavevmode
{[}nt,npsi{]} array of maximum (outboard) major radius of
flux surface psi.

\item[{Return type}] \leavevmode
Rmid (Array)

\item[{Raises}] \leavevmode
\sphinxstyleliteralstrong{\sphinxupquote{Value Error}} \textendash{} if module cannot retrieve data from MDS tree.

\end{description}\end{quote}

\end{fulllineitems}

\index{getIpCalc() (eqtools.NSTXEFIT.NSTXEFITTree method)@\spxentry{getIpCalc()}\spxextra{eqtools.NSTXEFIT.NSTXEFITTree method}}

\begin{fulllineitems}
\phantomsection\label{\detokenize{eqtools:eqtools.NSTXEFIT.NSTXEFITTree.getIpCalc}}\pysiglinewithargsret{\sphinxbfcode{\sphinxupquote{getIpCalc}}}{}{}
returns EFIT-calculated plasma current.
\begin{quote}\begin{description}
\item[{Returns}] \leavevmode
{[}nt{]} array of EFIT-reconstructed plasma current.

\item[{Return type}] \leavevmode
IpCalc (Array)

\item[{Raises}] \leavevmode
\sphinxstyleliteralstrong{\sphinxupquote{ValueError}} \textendash{} if module cannot retrieve data from MDS tree.

\end{description}\end{quote}

\end{fulllineitems}

\index{getVolLCFS() (eqtools.NSTXEFIT.NSTXEFITTree method)@\spxentry{getVolLCFS()}\spxextra{eqtools.NSTXEFIT.NSTXEFITTree method}}

\begin{fulllineitems}
\phantomsection\label{\detokenize{eqtools:eqtools.NSTXEFIT.NSTXEFITTree.getVolLCFS}}\pysiglinewithargsret{\sphinxbfcode{\sphinxupquote{getVolLCFS}}}{\emph{length\_unit=3}}{}
returns volume within LCFS.
\begin{quote}\begin{description}
\item[{Keyword Arguments}] \leavevmode
\sphinxstyleliteralstrong{\sphinxupquote{length\_unit}} (\sphinxstyleliteralemphasis{\sphinxupquote{String}}\sphinxstyleliteralemphasis{\sphinxupquote{ or }}\sphinxstyleliteralemphasis{\sphinxupquote{3}}) \textendash{} unit for LCFS volume.  Defaults to 3,
denoting default volumetric unit (typically m\textasciicircum{}3).

\item[{Returns}] \leavevmode
{[}nt{]} array of volume within LCFS.

\item[{Return type}] \leavevmode
volLCFS (Array)

\item[{Raises}] \leavevmode
\sphinxstyleliteralstrong{\sphinxupquote{ValueError}} \textendash{} if module cannot retrieve data from MDS tree.

\end{description}\end{quote}

\end{fulllineitems}

\index{getJp() (eqtools.NSTXEFIT.NSTXEFITTree method)@\spxentry{getJp()}\spxextra{eqtools.NSTXEFIT.NSTXEFITTree method}}

\begin{fulllineitems}
\phantomsection\label{\detokenize{eqtools:eqtools.NSTXEFIT.NSTXEFITTree.getJp}}\pysiglinewithargsret{\sphinxbfcode{\sphinxupquote{getJp}}}{}{}
Not implemented in NSTXEFIT tree.
\begin{quote}\begin{description}
\item[{Returns}] \leavevmode
EFIT-calculated plasma current density Jp on flux grid {[}t,r,z{]}

\end{description}\end{quote}

\end{fulllineitems}

\index{rz2volnorm() (eqtools.NSTXEFIT.NSTXEFITTree method)@\spxentry{rz2volnorm()}\spxextra{eqtools.NSTXEFIT.NSTXEFITTree method}}

\begin{fulllineitems}
\phantomsection\label{\detokenize{eqtools:eqtools.NSTXEFIT.NSTXEFITTree.rz2volnorm}}\pysiglinewithargsret{\sphinxbfcode{\sphinxupquote{rz2volnorm}}}{\emph{*args}, \emph{**kwargs}}{}
Calculated normalized volume of flux surfaces not stored in NSTX EFIT.
\begin{quote}\begin{description}
\item[{Returns}] \leavevmode
All mapping with Volnorm not implemented

\end{description}\end{quote}

\end{fulllineitems}

\index{psinorm2volnorm() (eqtools.NSTXEFIT.NSTXEFITTree method)@\spxentry{psinorm2volnorm()}\spxextra{eqtools.NSTXEFIT.NSTXEFITTree method}}

\begin{fulllineitems}
\phantomsection\label{\detokenize{eqtools:eqtools.NSTXEFIT.NSTXEFITTree.psinorm2volnorm}}\pysiglinewithargsret{\sphinxbfcode{\sphinxupquote{psinorm2volnorm}}}{\emph{*args}, \emph{**kwargs}}{}
Calculated normalized volume of flux surfaces not stored in NSTX EFIT.
\begin{quote}\begin{description}
\item[{Returns}] \leavevmode
All maping with Volnorm not implemented

\end{description}\end{quote}

\end{fulllineitems}


\end{fulllineitems}

\index{NSTXEFITTreeProp (class in eqtools.NSTXEFIT)@\spxentry{NSTXEFITTreeProp}\spxextra{class in eqtools.NSTXEFIT}}

\begin{fulllineitems}
\phantomsection\label{\detokenize{eqtools:eqtools.NSTXEFIT.NSTXEFITTreeProp}}\pysiglinewithargsret{\sphinxbfcode{\sphinxupquote{class }}\sphinxcode{\sphinxupquote{eqtools.NSTXEFIT.}}\sphinxbfcode{\sphinxupquote{NSTXEFITTreeProp}}}{\emph{shot}, \emph{tree='EFIT01'}, \emph{length\_unit='m'}, \emph{gfile='geqdsk'}, \emph{afile='aeqdsk'}, \emph{tspline=False}, \emph{monotonic=True}}{}
Bases: {\hyperref[\detokenize{eqtools:eqtools.NSTXEFIT.NSTXEFITTree}]{\sphinxcrossref{\sphinxcode{\sphinxupquote{eqtools.NSTXEFIT.NSTXEFITTree}}}}}, {\hyperref[\detokenize{eqtools:eqtools.core.PropertyAccessMixin}]{\sphinxcrossref{\sphinxcode{\sphinxupquote{eqtools.core.PropertyAccessMixin}}}}}

NSTXEFITTree with the PropertyAccessMixin added to enable property-style
access. This is good for interactive use, but may drag the performance down.

\end{fulllineitems}



\subsection{eqtools.TCVLIUQE module}
\label{\detokenize{eqtools:module-eqtools.TCVLIUQE}}\label{\detokenize{eqtools:eqtools-tcvliuqe-module}}\index{eqtools.TCVLIUQE (module)@\spxentry{eqtools.TCVLIUQE}\spxextra{module}}
This module provides classes inheriting {\hyperref[\detokenize{eqtools:eqtools.EFIT.EFITTree}]{\sphinxcrossref{\sphinxcode{\sphinxupquote{eqtools.EFIT.EFITTree}}}}} for
working with TCV LIUQE Equilibrium.
\index{greenArea() (in module eqtools.TCVLIUQE)@\spxentry{greenArea()}\spxextra{in module eqtools.TCVLIUQE}}

\begin{fulllineitems}
\phantomsection\label{\detokenize{eqtools:eqtools.TCVLIUQE.greenArea}}\pysiglinewithargsret{\sphinxcode{\sphinxupquote{eqtools.TCVLIUQE.}}\sphinxbfcode{\sphinxupquote{greenArea}}}{\emph{vs}}{}
\end{fulllineitems}

\index{TCVLIUQETree (class in eqtools.TCVLIUQE)@\spxentry{TCVLIUQETree}\spxextra{class in eqtools.TCVLIUQE}}

\begin{fulllineitems}
\phantomsection\label{\detokenize{eqtools:eqtools.TCVLIUQE.TCVLIUQETree}}\pysiglinewithargsret{\sphinxbfcode{\sphinxupquote{class }}\sphinxcode{\sphinxupquote{eqtools.TCVLIUQE.}}\sphinxbfcode{\sphinxupquote{TCVLIUQETree}}}{\emph{shot}, \emph{tree='tcv\_shot'}, \emph{length\_unit='m'}, \emph{gfile='g\_eqdsk'}, \emph{afile='a\_eqdsk'}, \emph{tspline=False}, \emph{monotonic=True}}{}
Bases: {\hyperref[\detokenize{eqtools:eqtools.EFIT.EFITTree}]{\sphinxcrossref{\sphinxcode{\sphinxupquote{eqtools.EFIT.EFITTree}}}}}

Inherits {\hyperref[\detokenize{eqtools:eqtools.EFIT.EFITTree}]{\sphinxcrossref{\sphinxcode{\sphinxupquote{eqtools.EFIT.EFITTree}}}}} class. Machine-specific data
handling class for TCV Machine. Pulls LIUQUE data from selected MDS tree
and shot, stores as object attributes eventually transforming it in the
equivalent quantity for EFIT. Each  variable or set of
variables is recovered with a corresponding getter method. Essential data
for LIUQUE mapping are pulled on initialization (e.g. psirz grid). Additional
data are pulled at the first request and stored for subsequent usage.

Intializes TCV version of EFITTree object.  Pulls data from MDS tree for
storage in instance attributes.  Core attributes are populated from the MDS
tree on initialization.  Additional attributes are initialized as None,
filled on the first request to the object.
\begin{quote}\begin{description}
\item[{Parameters}] \leavevmode
\sphinxstyleliteralstrong{\sphinxupquote{shot}} (\sphinxstyleliteralemphasis{\sphinxupquote{integer}}) \textendash{} TCV shot index.

\item[{Keyword Arguments}] \leavevmode\begin{itemize}
\item {} 
\sphinxstyleliteralstrong{\sphinxupquote{tree}} (\sphinxstyleliteralemphasis{\sphinxupquote{string}}) \textendash{} Optional input for LIUQE tree, defaults to ‘RESULTS’
(i.e., LIUQE data are under results::).

\item {} 
\sphinxstyleliteralstrong{\sphinxupquote{length\_unit}} (\sphinxstyleliteralemphasis{\sphinxupquote{string}}) \textendash{} 
Sets the base unit used for any quantity whose
dimensions are length to any power. Valid options are:
\begin{quote}


\begin{savenotes}\sphinxattablestart
\centering
\begin{tabulary}{\linewidth}[t]{|T|T|}
\hline

’m’
&
meters
\\
\hline
’cm’
&
centimeters
\\
\hline
’mm’
&
millimeters
\\
\hline
’in’
&
inches
\\
\hline
’ft’
&
feet
\\
\hline
’yd’
&
yards
\\
\hline
’smoot’
&
smoots
\\
\hline
’cubit’
&
cubits
\\
\hline
’hand’
&
hands
\\
\hline
’default’
&
whatever the default in the tree is (no conversion is performed, units may be inconsistent)
\\
\hline
\end{tabulary}
\par
\sphinxattableend\end{savenotes}
\end{quote}

Default is ‘m’ (all units taken and returned in meters).


\item {} 
\sphinxstyleliteralstrong{\sphinxupquote{gfile}} (\sphinxstyleliteralemphasis{\sphinxupquote{string}}) \textendash{} Optional input for EFIT geqdsk location name,
defaults to ‘g\_eqdsk’ (i.e., EFIT data are under
tree::top.results.G\_EQDSK)

\item {} 
\sphinxstyleliteralstrong{\sphinxupquote{afile}} (\sphinxstyleliteralemphasis{\sphinxupquote{string}}) \textendash{} Optional input for EFIT aeqdsk location name,
defaults to ‘a\_eqdsk’ (i.e., EFIT data are under
tree::top.results.A\_EQDSK)

\item {} 
\sphinxstyleliteralstrong{\sphinxupquote{tspline}} (\sphinxstyleliteralemphasis{\sphinxupquote{Boolean}}) \textendash{} Sets whether or not interpolation in time is
performed using a tricubic spline or nearest-neighbor
interpolation. Tricubic spline interpolation requires at least
four complete equilibria at different times. It is also assumed
that they are functionally correlated, and that parameters do
not vary out of their boundaries (derivative = 0 boundary
condition). Default is False (use nearest neighbor interpolation).

\item {} 
\sphinxstyleliteralstrong{\sphinxupquote{monotonic}} (\sphinxstyleliteralemphasis{\sphinxupquote{Boolean}}) \textendash{} Sets whether or not the “monotonic” form of time
window finding is used. If True, the timebase must be monotonically
increasing. Default is False (use slower, safer method).

\end{itemize}

\end{description}\end{quote}
\index{getInfo() (eqtools.TCVLIUQE.TCVLIUQETree method)@\spxentry{getInfo()}\spxextra{eqtools.TCVLIUQE.TCVLIUQETree method}}

\begin{fulllineitems}
\phantomsection\label{\detokenize{eqtools:eqtools.TCVLIUQE.TCVLIUQETree.getInfo}}\pysiglinewithargsret{\sphinxbfcode{\sphinxupquote{getInfo}}}{}{}
returns namedtuple of shot information
\begin{quote}\begin{description}
\item[{Returns}] \leavevmode

namedtuple containing
\begin{quote}


\begin{savenotes}\sphinxattablestart
\centering
\begin{tabulary}{\linewidth}[t]{|T|T|}
\hline

shot
&
TCV shot index (long)
\\
\hline
tree
&
LIUQE tree (string)
\\
\hline
nr
&
size of R-axis for spatial grid
\\
\hline
nz
&
size of Z-axis for spatial grid
\\
\hline
nt
&
size of timebase for flux grid
\\
\hline
\end{tabulary}
\par
\sphinxattableend\end{savenotes}
\end{quote}


\end{description}\end{quote}

\end{fulllineitems}

\index{getTimeBase() (eqtools.TCVLIUQE.TCVLIUQETree method)@\spxentry{getTimeBase()}\spxextra{eqtools.TCVLIUQE.TCVLIUQETree method}}

\begin{fulllineitems}
\phantomsection\label{\detokenize{eqtools:eqtools.TCVLIUQE.TCVLIUQETree.getTimeBase}}\pysiglinewithargsret{\sphinxbfcode{\sphinxupquote{getTimeBase}}}{}{}
returns LIUQE time base vector.
\begin{quote}\begin{description}
\item[{Returns}] \leavevmode
{[}nt{]} array of time points.

\item[{Return type}] \leavevmode
time (array)

\item[{Raises}] \leavevmode
\sphinxstyleliteralstrong{\sphinxupquote{ValueError}} \textendash{} if module cannot retrieve data from MDS tree.

\end{description}\end{quote}

\end{fulllineitems}

\index{getFluxGrid() (eqtools.TCVLIUQE.TCVLIUQETree method)@\spxentry{getFluxGrid()}\spxextra{eqtools.TCVLIUQE.TCVLIUQETree method}}

\begin{fulllineitems}
\phantomsection\label{\detokenize{eqtools:eqtools.TCVLIUQE.TCVLIUQETree.getFluxGrid}}\pysiglinewithargsret{\sphinxbfcode{\sphinxupquote{getFluxGrid}}}{}{}
returns LIUQE flux grid.
\begin{quote}\begin{description}
\item[{Returns}] \leavevmode
{[}nt,nz,nr{]} array of (non-normalized) flux on grid.

\item[{Return type}] \leavevmode
psiRZ (Array)

\item[{Raises}] \leavevmode
\sphinxstyleliteralstrong{\sphinxupquote{ValueError}} \textendash{} if module cannot retrieve data from MDS tree.

\end{description}\end{quote}

\end{fulllineitems}

\index{getRGrid() (eqtools.TCVLIUQE.TCVLIUQETree method)@\spxentry{getRGrid()}\spxextra{eqtools.TCVLIUQE.TCVLIUQETree method}}

\begin{fulllineitems}
\phantomsection\label{\detokenize{eqtools:eqtools.TCVLIUQE.TCVLIUQETree.getRGrid}}\pysiglinewithargsret{\sphinxbfcode{\sphinxupquote{getRGrid}}}{\emph{length\_unit=1}}{}
returns LIUQE R-axis.
\begin{quote}\begin{description}
\item[{Returns}] \leavevmode
{[}nr{]} array of R-axis of flux grid.

\item[{Return type}] \leavevmode
rGrid (Array)

\item[{Raises}] \leavevmode
\sphinxstyleliteralstrong{\sphinxupquote{ValueError}} \textendash{} if module cannot retrieve data from MDS tree.

\end{description}\end{quote}

\end{fulllineitems}

\index{getZGrid() (eqtools.TCVLIUQE.TCVLIUQETree method)@\spxentry{getZGrid()}\spxextra{eqtools.TCVLIUQE.TCVLIUQETree method}}

\begin{fulllineitems}
\phantomsection\label{\detokenize{eqtools:eqtools.TCVLIUQE.TCVLIUQETree.getZGrid}}\pysiglinewithargsret{\sphinxbfcode{\sphinxupquote{getZGrid}}}{\emph{length\_unit=1}}{}
returns LIUQE Z-axis.
\begin{quote}\begin{description}
\item[{Returns}] \leavevmode
{[}nz{]} array of Z-axis of flux grid.

\item[{Return type}] \leavevmode
zGrid (Array)

\item[{Raises}] \leavevmode
\sphinxstyleliteralstrong{\sphinxupquote{ValueError}} \textendash{} if module cannot retrieve data from MDS tree.

\end{description}\end{quote}

\end{fulllineitems}

\index{getFluxAxis() (eqtools.TCVLIUQE.TCVLIUQETree method)@\spxentry{getFluxAxis()}\spxextra{eqtools.TCVLIUQE.TCVLIUQETree method}}

\begin{fulllineitems}
\phantomsection\label{\detokenize{eqtools:eqtools.TCVLIUQE.TCVLIUQETree.getFluxAxis}}\pysiglinewithargsret{\sphinxbfcode{\sphinxupquote{getFluxAxis}}}{}{}
returns psi on magnetic axis.
\begin{quote}\begin{description}
\item[{Returns}] \leavevmode
{[}nt{]} array of psi on magnetic axis.

\item[{Return type}] \leavevmode
psiAxis (Array)

\item[{Raises}] \leavevmode
\sphinxstyleliteralstrong{\sphinxupquote{ValueError}} \textendash{} if module cannot retrieve data from MDS tree.

\end{description}\end{quote}

\end{fulllineitems}

\index{getFluxLCFS() (eqtools.TCVLIUQE.TCVLIUQETree method)@\spxentry{getFluxLCFS()}\spxextra{eqtools.TCVLIUQE.TCVLIUQETree method}}

\begin{fulllineitems}
\phantomsection\label{\detokenize{eqtools:eqtools.TCVLIUQE.TCVLIUQETree.getFluxLCFS}}\pysiglinewithargsret{\sphinxbfcode{\sphinxupquote{getFluxLCFS}}}{}{}
returns psi at separatrix.
\begin{quote}\begin{description}
\item[{Returns}] \leavevmode
{[}nt{]} array of psi at LCFS.

\item[{Return type}] \leavevmode
psiLCFS (Array)

\item[{Raises}] \leavevmode
\sphinxstyleliteralstrong{\sphinxupquote{ValueError}} \textendash{} if module cannot retrieve data from MDS tree.

\end{description}\end{quote}

\end{fulllineitems}

\index{getFluxVol() (eqtools.TCVLIUQE.TCVLIUQETree method)@\spxentry{getFluxVol()}\spxextra{eqtools.TCVLIUQE.TCVLIUQETree method}}

\begin{fulllineitems}
\phantomsection\label{\detokenize{eqtools:eqtools.TCVLIUQE.TCVLIUQETree.getFluxVol}}\pysiglinewithargsret{\sphinxbfcode{\sphinxupquote{getFluxVol}}}{\emph{length\_unit=3}}{}
returns volume within flux surface. This is not implemented in LIUQE
as default output. So we use contour and GREEN theorem to get the area
within a default grid of the PSI. Then we compute the volume by multipling
for 2pi * VolLCFS / AreaLCFS.
\begin{quote}\begin{description}
\item[{Keyword Arguments}] \leavevmode
\sphinxstyleliteralstrong{\sphinxupquote{length\_unit}} (\sphinxstyleliteralemphasis{\sphinxupquote{String}}\sphinxstyleliteralemphasis{\sphinxupquote{ or }}\sphinxstyleliteralemphasis{\sphinxupquote{3}}) \textendash{} unit for plasma volume.  Defaults to 3,
indicating default volumetric unit (typically m\textasciicircum{}3).

\item[{Returns}] \leavevmode
{[}nt,npsi{]} array of volume within flux surface.

\item[{Return type}] \leavevmode
fluxVol (Array)

\item[{Raises}] \leavevmode
\sphinxstyleliteralstrong{\sphinxupquote{ValueError}} \textendash{} if module cannot retrieve data from MDS tree.

\end{description}\end{quote}

\end{fulllineitems}

\index{getVolLCFS() (eqtools.TCVLIUQE.TCVLIUQETree method)@\spxentry{getVolLCFS()}\spxextra{eqtools.TCVLIUQE.TCVLIUQETree method}}

\begin{fulllineitems}
\phantomsection\label{\detokenize{eqtools:eqtools.TCVLIUQE.TCVLIUQETree.getVolLCFS}}\pysiglinewithargsret{\sphinxbfcode{\sphinxupquote{getVolLCFS}}}{\emph{length\_unit=3}}{}
returns volume within LCFS.
\begin{quote}\begin{description}
\item[{Keyword Arguments}] \leavevmode
\sphinxstyleliteralstrong{\sphinxupquote{length\_unit}} (\sphinxstyleliteralemphasis{\sphinxupquote{String}}\sphinxstyleliteralemphasis{\sphinxupquote{ or }}\sphinxstyleliteralemphasis{\sphinxupquote{3}}) \textendash{} unit for LCFS volume.  Defaults to 3,
denoting default volumetric unit (typically m\textasciicircum{}3).

\item[{Returns}] \leavevmode
{[}nt{]} array of volume within LCFS.

\item[{Return type}] \leavevmode
volLCFS (Array)

\item[{Raises}] \leavevmode
\sphinxstyleliteralstrong{\sphinxupquote{ValueError}} \textendash{} if module cannot retrieve data from MDS tree.

\end{description}\end{quote}

\end{fulllineitems}

\index{getRmidPsi() (eqtools.TCVLIUQE.TCVLIUQETree method)@\spxentry{getRmidPsi()}\spxextra{eqtools.TCVLIUQE.TCVLIUQETree method}}

\begin{fulllineitems}
\phantomsection\label{\detokenize{eqtools:eqtools.TCVLIUQE.TCVLIUQETree.getRmidPsi}}\pysiglinewithargsret{\sphinxbfcode{\sphinxupquote{getRmidPsi}}}{\emph{length\_unit=1}}{}
returns maximum major radius of each flux surface.
\begin{quote}\begin{description}
\item[{Keyword Arguments}] \leavevmode
\sphinxstyleliteralstrong{\sphinxupquote{length\_unit}} (\sphinxstyleliteralemphasis{\sphinxupquote{String}}\sphinxstyleliteralemphasis{\sphinxupquote{ or }}\sphinxstyleliteralemphasis{\sphinxupquote{1}}) \textendash{} unit of Rmid.  Defaults to 1, indicating
the default parameter unit (typically m).

\item[{Returns}] \leavevmode
{[}nt,npsi{]} array of maximum (outboard) major radius of
flux surface psi.

\item[{Return type}] \leavevmode
Rmid (Array)

\item[{Raises}] \leavevmode
\sphinxstyleliteralstrong{\sphinxupquote{Value Error}} \textendash{} if module cannot retrieve data from MDS tree.

\end{description}\end{quote}

\end{fulllineitems}

\index{getRLCFS() (eqtools.TCVLIUQE.TCVLIUQETree method)@\spxentry{getRLCFS()}\spxextra{eqtools.TCVLIUQE.TCVLIUQETree method}}

\begin{fulllineitems}
\phantomsection\label{\detokenize{eqtools:eqtools.TCVLIUQE.TCVLIUQETree.getRLCFS}}\pysiglinewithargsret{\sphinxbfcode{\sphinxupquote{getRLCFS}}}{\emph{length\_unit=1}}{}
returns R-values of LCFS position.
\begin{quote}\begin{description}
\item[{Returns}] \leavevmode
{[}nt,n{]} array of R of LCFS points.

\item[{Return type}] \leavevmode
RLCFS (Array)

\item[{Raises}] \leavevmode
\sphinxstyleliteralstrong{\sphinxupquote{ValueError}} \textendash{} if module cannot retrieve data from MDS tree.

\end{description}\end{quote}

\end{fulllineitems}

\index{getZLCFS() (eqtools.TCVLIUQE.TCVLIUQETree method)@\spxentry{getZLCFS()}\spxextra{eqtools.TCVLIUQE.TCVLIUQETree method}}

\begin{fulllineitems}
\phantomsection\label{\detokenize{eqtools:eqtools.TCVLIUQE.TCVLIUQETree.getZLCFS}}\pysiglinewithargsret{\sphinxbfcode{\sphinxupquote{getZLCFS}}}{\emph{length\_unit=1}}{}
returns Z-values of LCFS position.
\begin{quote}\begin{description}
\item[{Returns}] \leavevmode
{[}nt,n{]} array of Z of LCFS points.

\item[{Return type}] \leavevmode
ZLCFS (Array)

\item[{Raises}] \leavevmode
\sphinxstyleliteralstrong{\sphinxupquote{ValueError}} \textendash{} if module cannot retrieve data from MDS tree.

\end{description}\end{quote}

\end{fulllineitems}

\index{getF() (eqtools.TCVLIUQE.TCVLIUQETree method)@\spxentry{getF()}\spxextra{eqtools.TCVLIUQE.TCVLIUQETree method}}

\begin{fulllineitems}
\phantomsection\label{\detokenize{eqtools:eqtools.TCVLIUQE.TCVLIUQETree.getF}}\pysiglinewithargsret{\sphinxbfcode{\sphinxupquote{getF}}}{}{}
returns F=RB\_\{Phi\}(Psi), often calculated for grad-shafranov
solutions. Not implemented on LIUQE
\begin{quote}\begin{description}
\item[{Returns}] \leavevmode
{[}nt,npsi{]} array of F=RB\_\{Phi\}(Psi)
Not stored on LIUQE nodes

\item[{Return type}] \leavevmode
F (Array)

\item[{Raises}] \leavevmode
\sphinxstyleliteralstrong{\sphinxupquote{ValueError}} \textendash{} if module cannot retrieve data from MDS tree.

\end{description}\end{quote}

\end{fulllineitems}

\index{getFluxPres() (eqtools.TCVLIUQE.TCVLIUQETree method)@\spxentry{getFluxPres()}\spxextra{eqtools.TCVLIUQE.TCVLIUQETree method}}

\begin{fulllineitems}
\phantomsection\label{\detokenize{eqtools:eqtools.TCVLIUQE.TCVLIUQETree.getFluxPres}}\pysiglinewithargsret{\sphinxbfcode{\sphinxupquote{getFluxPres}}}{}{}~\begin{description}
\item[{returns pressure at flux surface. Not implemented. We have pressure}] \leavevmode
saved on the same grid of psi

\end{description}
\begin{quote}\begin{description}
\item[{Returns}] \leavevmode
{[}nt,npsi{]} array of pressure on flux surface psi.
Not implemented on LIUQE nodes. We have pressure on the grid use for psi

\item[{Return type}] \leavevmode
p (Array)

\item[{Raises}] \leavevmode
\sphinxstyleliteralstrong{\sphinxupquote{ValueError}} \textendash{} if module cannot retrieve data from MDS tree.

\end{description}\end{quote}

\end{fulllineitems}

\index{getFFPrime() (eqtools.TCVLIUQE.TCVLIUQETree method)@\spxentry{getFFPrime()}\spxextra{eqtools.TCVLIUQE.TCVLIUQETree method}}

\begin{fulllineitems}
\phantomsection\label{\detokenize{eqtools:eqtools.TCVLIUQE.TCVLIUQETree.getFFPrime}}\pysiglinewithargsret{\sphinxbfcode{\sphinxupquote{getFFPrime}}}{}{}
returns FF’ function used for grad-shafranov solutions.
\begin{quote}\begin{description}
\item[{Returns}] \leavevmode
{[}nt,npsi{]} array of FF’ fromgrad-shafranov solution.

\item[{Return type}] \leavevmode
FFprime (Array)

\item[{Raises}] \leavevmode
\sphinxstyleliteralstrong{\sphinxupquote{ValueError}} \textendash{} if module cannot retrieve data from MDS tree.

\end{description}\end{quote}

\end{fulllineitems}

\index{getPPrime() (eqtools.TCVLIUQE.TCVLIUQETree method)@\spxentry{getPPrime()}\spxextra{eqtools.TCVLIUQE.TCVLIUQETree method}}

\begin{fulllineitems}
\phantomsection\label{\detokenize{eqtools:eqtools.TCVLIUQE.TCVLIUQETree.getPPrime}}\pysiglinewithargsret{\sphinxbfcode{\sphinxupquote{getPPrime}}}{}{}
returns plasma pressure gradient as a function of psi.
\begin{quote}\begin{description}
\item[{Returns}] \leavevmode
{[}nt,npsi{]} array of pressure gradient on flux surface
psi from grad-shafranov solution.

\item[{Return type}] \leavevmode
pprime (Array)

\item[{Raises}] \leavevmode
\sphinxstyleliteralstrong{\sphinxupquote{ValueError}} \textendash{} if module cannot retrieve data from MDS tree.

\end{description}\end{quote}

\end{fulllineitems}

\index{getElongation() (eqtools.TCVLIUQE.TCVLIUQETree method)@\spxentry{getElongation()}\spxextra{eqtools.TCVLIUQE.TCVLIUQETree method}}

\begin{fulllineitems}
\phantomsection\label{\detokenize{eqtools:eqtools.TCVLIUQE.TCVLIUQETree.getElongation}}\pysiglinewithargsret{\sphinxbfcode{\sphinxupquote{getElongation}}}{}{}
returns LCFS elongation.
\begin{quote}\begin{description}
\item[{Returns}] \leavevmode
{[}nt{]} array of LCFS elongation.

\item[{Return type}] \leavevmode
kappa (Array)

\item[{Raises}] \leavevmode
\sphinxstyleliteralstrong{\sphinxupquote{ValueError}} \textendash{} if module cannot retrieve data from MDS tree.

\end{description}\end{quote}

\end{fulllineitems}

\index{getUpperTriangularity() (eqtools.TCVLIUQE.TCVLIUQETree method)@\spxentry{getUpperTriangularity()}\spxextra{eqtools.TCVLIUQE.TCVLIUQETree method}}

\begin{fulllineitems}
\phantomsection\label{\detokenize{eqtools:eqtools.TCVLIUQE.TCVLIUQETree.getUpperTriangularity}}\pysiglinewithargsret{\sphinxbfcode{\sphinxupquote{getUpperTriangularity}}}{}{}
returns LCFS upper triangularity.
\begin{quote}\begin{description}
\item[{Returns}] \leavevmode
{[}nt{]} array of LCFS upper triangularity.

\item[{Return type}] \leavevmode
deltau (Array)

\item[{Raises}] \leavevmode
\sphinxstyleliteralstrong{\sphinxupquote{ValueError}} \textendash{} if module cannot retrieve data from MDS tree.

\end{description}\end{quote}

\end{fulllineitems}

\index{getLowerTriangularity() (eqtools.TCVLIUQE.TCVLIUQETree method)@\spxentry{getLowerTriangularity()}\spxextra{eqtools.TCVLIUQE.TCVLIUQETree method}}

\begin{fulllineitems}
\phantomsection\label{\detokenize{eqtools:eqtools.TCVLIUQE.TCVLIUQETree.getLowerTriangularity}}\pysiglinewithargsret{\sphinxbfcode{\sphinxupquote{getLowerTriangularity}}}{}{}
returns LCFS lower triangularity.
\begin{quote}\begin{description}
\item[{Returns}] \leavevmode
{[}nt{]} array of LCFS lower triangularity.

\item[{Return type}] \leavevmode
deltal (Array)

\item[{Raises}] \leavevmode
\sphinxstyleliteralstrong{\sphinxupquote{ValueError}} \textendash{} if module cannot retrieve data from MDS tree.

\end{description}\end{quote}

\end{fulllineitems}

\index{getMagR() (eqtools.TCVLIUQE.TCVLIUQETree method)@\spxentry{getMagR()}\spxextra{eqtools.TCVLIUQE.TCVLIUQETree method}}

\begin{fulllineitems}
\phantomsection\label{\detokenize{eqtools:eqtools.TCVLIUQE.TCVLIUQETree.getMagR}}\pysiglinewithargsret{\sphinxbfcode{\sphinxupquote{getMagR}}}{\emph{length\_unit=1}}{}
returns magnetic-axis major radius.
\begin{quote}\begin{description}
\item[{Returns}] \leavevmode
{[}nt{]} array of major radius of magnetic axis.

\item[{Return type}] \leavevmode
magR (Array)

\item[{Raises}] \leavevmode
\sphinxstyleliteralstrong{\sphinxupquote{ValueError}} \textendash{} if module cannot retrieve data from MDS tree.

\end{description}\end{quote}

\end{fulllineitems}

\index{getMagZ() (eqtools.TCVLIUQE.TCVLIUQETree method)@\spxentry{getMagZ()}\spxextra{eqtools.TCVLIUQE.TCVLIUQETree method}}

\begin{fulllineitems}
\phantomsection\label{\detokenize{eqtools:eqtools.TCVLIUQE.TCVLIUQETree.getMagZ}}\pysiglinewithargsret{\sphinxbfcode{\sphinxupquote{getMagZ}}}{\emph{length\_unit=1}}{}
returns magnetic-axis Z.
\begin{quote}\begin{description}
\item[{Returns}] \leavevmode
{[}nt{]} array of Z of magnetic axis.

\item[{Return type}] \leavevmode
magZ (Array)

\item[{Raises}] \leavevmode
\sphinxstyleliteralstrong{\sphinxupquote{ValueError}} \textendash{} if module cannot retrieve data from MDS tree.

\end{description}\end{quote}

\end{fulllineitems}

\index{getAreaLCFS() (eqtools.TCVLIUQE.TCVLIUQETree method)@\spxentry{getAreaLCFS()}\spxextra{eqtools.TCVLIUQE.TCVLIUQETree method}}

\begin{fulllineitems}
\phantomsection\label{\detokenize{eqtools:eqtools.TCVLIUQE.TCVLIUQETree.getAreaLCFS}}\pysiglinewithargsret{\sphinxbfcode{\sphinxupquote{getAreaLCFS}}}{\emph{length\_unit=2}}{}
returns LCFS cross-sectional area.
\begin{quote}\begin{description}
\item[{Keyword Arguments}] \leavevmode
\sphinxstyleliteralstrong{\sphinxupquote{length\_unit}} (\sphinxstyleliteralemphasis{\sphinxupquote{String}}\sphinxstyleliteralemphasis{\sphinxupquote{ or }}\sphinxstyleliteralemphasis{\sphinxupquote{2}}) \textendash{} unit for LCFS area.  Defaults to 2,
denoting default areal unit (typically m\textasciicircum{}2).

\item[{Returns}] \leavevmode
{[}nt{]} array of LCFS area.

\item[{Return type}] \leavevmode
areaLCFS (Array)

\item[{Raises}] \leavevmode
\sphinxstyleliteralstrong{\sphinxupquote{ValueError}} \textendash{} if module cannot retrieve data from MDS tree.

\end{description}\end{quote}

\end{fulllineitems}

\index{getAOut() (eqtools.TCVLIUQE.TCVLIUQETree method)@\spxentry{getAOut()}\spxextra{eqtools.TCVLIUQE.TCVLIUQETree method}}

\begin{fulllineitems}
\phantomsection\label{\detokenize{eqtools:eqtools.TCVLIUQE.TCVLIUQETree.getAOut}}\pysiglinewithargsret{\sphinxbfcode{\sphinxupquote{getAOut}}}{\emph{length\_unit=1}}{}~\begin{description}
\item[{returns outboard-midplane minor radius at LCFS. In LIUQE it is the last value}] \leavevmode
of

\end{description}

esults::r\_max\_psi
\begin{quote}
\begin{description}
\item[{Keyword Args:}] \leavevmode\begin{description}
\item[{length\_unit (String or 1): unit for minor radius.  Defaults to 1,}] \leavevmode
denoting default length unit (typically m).

\end{description}

\item[{Returns:}] \leavevmode
aOut (Array): {[}nt{]} array of LCFS outboard-midplane minor radius.

\item[{Raises:}] \leavevmode
ValueError: if module cannot retrieve data from MDS tree.

\end{description}
\end{quote}

\end{fulllineitems}

\index{getRmidOut() (eqtools.TCVLIUQE.TCVLIUQETree method)@\spxentry{getRmidOut()}\spxextra{eqtools.TCVLIUQE.TCVLIUQETree method}}

\begin{fulllineitems}
\phantomsection\label{\detokenize{eqtools:eqtools.TCVLIUQE.TCVLIUQETree.getRmidOut}}\pysiglinewithargsret{\sphinxbfcode{\sphinxupquote{getRmidOut}}}{\emph{length\_unit=1}}{}
returns outboard-midplane major radius. It uses getA
\begin{quote}\begin{description}
\item[{Keyword Arguments}] \leavevmode
\sphinxstyleliteralstrong{\sphinxupquote{length\_unit}} (\sphinxstyleliteralemphasis{\sphinxupquote{String}}\sphinxstyleliteralemphasis{\sphinxupquote{ or }}\sphinxstyleliteralemphasis{\sphinxupquote{1}}) \textendash{} unit for major radius.  Defaults to 1,
denoting default length unit (typically m).

\item[{Returns}] \leavevmode
{[}nt{]} array of major radius of LCFS.

\item[{Return type}] \leavevmode
RmidOut (Array)

\item[{Raises}] \leavevmode
\sphinxstyleliteralstrong{\sphinxupquote{ValueError}} \textendash{} if module cannot retrieve data from MDS tree.

\end{description}\end{quote}

\end{fulllineitems}

\index{getQProfile() (eqtools.TCVLIUQE.TCVLIUQETree method)@\spxentry{getQProfile()}\spxextra{eqtools.TCVLIUQE.TCVLIUQETree method}}

\begin{fulllineitems}
\phantomsection\label{\detokenize{eqtools:eqtools.TCVLIUQE.TCVLIUQETree.getQProfile}}\pysiglinewithargsret{\sphinxbfcode{\sphinxupquote{getQProfile}}}{}{}
returns profile of safety factor q.
\begin{quote}\begin{description}
\item[{Returns}] \leavevmode
{[}nt,npsi{]} array of q on flux surface psi.

\item[{Return type}] \leavevmode
qpsi (Array)

\item[{Raises}] \leavevmode
\sphinxstyleliteralstrong{\sphinxupquote{ValueError}} \textendash{} if module cannot retrieve data from MDS tree.

\end{description}\end{quote}

\end{fulllineitems}

\index{getQ0() (eqtools.TCVLIUQE.TCVLIUQETree method)@\spxentry{getQ0()}\spxextra{eqtools.TCVLIUQE.TCVLIUQETree method}}

\begin{fulllineitems}
\phantomsection\label{\detokenize{eqtools:eqtools.TCVLIUQE.TCVLIUQETree.getQ0}}\pysiglinewithargsret{\sphinxbfcode{\sphinxupquote{getQ0}}}{}{}
returns q on magnetic axis,q0.
\begin{quote}\begin{description}
\item[{Returns}] \leavevmode
{[}nt{]} array of q(psi=0).

\item[{Return type}] \leavevmode
q0 (Array)

\item[{Raises}] \leavevmode
\sphinxstyleliteralstrong{\sphinxupquote{ValueError}} \textendash{} if module cannot retrieve data from MDS tree.

\end{description}\end{quote}

\end{fulllineitems}

\index{getQ95() (eqtools.TCVLIUQE.TCVLIUQETree method)@\spxentry{getQ95()}\spxextra{eqtools.TCVLIUQE.TCVLIUQETree method}}

\begin{fulllineitems}
\phantomsection\label{\detokenize{eqtools:eqtools.TCVLIUQE.TCVLIUQETree.getQ95}}\pysiglinewithargsret{\sphinxbfcode{\sphinxupquote{getQ95}}}{}{}
returns q at 95\% flux surface.
\begin{quote}\begin{description}
\item[{Returns}] \leavevmode
{[}nt{]} array of q(psi=0.95).

\item[{Return type}] \leavevmode
q95 (Array)

\item[{Raises}] \leavevmode
\sphinxstyleliteralstrong{\sphinxupquote{ValueError}} \textendash{} if module cannot retrieve data from MDS tree.

\end{description}\end{quote}

\end{fulllineitems}

\index{getQLCFS() (eqtools.TCVLIUQE.TCVLIUQETree method)@\spxentry{getQLCFS()}\spxextra{eqtools.TCVLIUQE.TCVLIUQETree method}}

\begin{fulllineitems}
\phantomsection\label{\detokenize{eqtools:eqtools.TCVLIUQE.TCVLIUQETree.getQLCFS}}\pysiglinewithargsret{\sphinxbfcode{\sphinxupquote{getQLCFS}}}{}{}
returns q on LCFS (interpolated).
\begin{quote}\begin{description}
\item[{Returns}] \leavevmode
{[}nt{]} array of q* (interpolated).

\item[{Return type}] \leavevmode
qLCFS (Array)

\item[{Raises}] \leavevmode
\sphinxstyleliteralstrong{\sphinxupquote{ValueError}} \textendash{} if module cannot retrieve data from MDS tree.

\end{description}\end{quote}

\end{fulllineitems}

\index{getBtVac() (eqtools.TCVLIUQE.TCVLIUQETree method)@\spxentry{getBtVac()}\spxextra{eqtools.TCVLIUQE.TCVLIUQETree method}}

\begin{fulllineitems}
\phantomsection\label{\detokenize{eqtools:eqtools.TCVLIUQE.TCVLIUQETree.getBtVac}}\pysiglinewithargsret{\sphinxbfcode{\sphinxupquote{getBtVac}}}{}{}
Returns vacuum toroidal field on-axis. We use MDSplus.Connection
for a proper use of the TDI function tcv\_eq()
\begin{quote}\begin{description}
\item[{Returns}] \leavevmode
{[}nt{]} array of vacuum toroidal field.

\item[{Return type}] \leavevmode
BtVac (Array)

\item[{Raises}] \leavevmode
\sphinxstyleliteralstrong{\sphinxupquote{ValueError}} \textendash{} if module cannot retrieve data from MDS tree.

\end{description}\end{quote}

\end{fulllineitems}

\index{getBtPla() (eqtools.TCVLIUQE.TCVLIUQETree method)@\spxentry{getBtPla()}\spxextra{eqtools.TCVLIUQE.TCVLIUQETree method}}

\begin{fulllineitems}
\phantomsection\label{\detokenize{eqtools:eqtools.TCVLIUQE.TCVLIUQETree.getBtPla}}\pysiglinewithargsret{\sphinxbfcode{\sphinxupquote{getBtPla}}}{}{}
returns on-axis plasma toroidal field.
\begin{quote}\begin{description}
\item[{Returns}] \leavevmode
{[}nt{]} array of toroidal field including plasma effects.

\item[{Return type}] \leavevmode
BtPla (Array)

\item[{Raises}] \leavevmode
\sphinxstyleliteralstrong{\sphinxupquote{ValueError}} \textendash{} if module cannot retrieve data from MDS tree.

\end{description}\end{quote}

\end{fulllineitems}

\index{getIpCalc() (eqtools.TCVLIUQE.TCVLIUQETree method)@\spxentry{getIpCalc()}\spxextra{eqtools.TCVLIUQE.TCVLIUQETree method}}

\begin{fulllineitems}
\phantomsection\label{\detokenize{eqtools:eqtools.TCVLIUQE.TCVLIUQETree.getIpCalc}}\pysiglinewithargsret{\sphinxbfcode{\sphinxupquote{getIpCalc}}}{}{}
returns EFIT-calculated plasma current.
\begin{quote}\begin{description}
\item[{Returns}] \leavevmode
{[}nt{]} array of EFIT-reconstructed plasma current.

\item[{Return type}] \leavevmode
IpCalc (Array)

\item[{Raises}] \leavevmode
\sphinxstyleliteralstrong{\sphinxupquote{ValueError}} \textendash{} if module cannot retrieve data from MDS tree.

\end{description}\end{quote}

\end{fulllineitems}

\index{getIpMeas() (eqtools.TCVLIUQE.TCVLIUQETree method)@\spxentry{getIpMeas()}\spxextra{eqtools.TCVLIUQE.TCVLIUQETree method}}

\begin{fulllineitems}
\phantomsection\label{\detokenize{eqtools:eqtools.TCVLIUQE.TCVLIUQETree.getIpMeas}}\pysiglinewithargsret{\sphinxbfcode{\sphinxupquote{getIpMeas}}}{}{}
returns magnetics-measured plasma current.
\begin{quote}\begin{description}
\item[{Returns}] \leavevmode
{[}nt{]} array of measured plasma current.

\item[{Return type}] \leavevmode
IpMeas (Array)

\item[{Raises}] \leavevmode
\sphinxstyleliteralstrong{\sphinxupquote{ValueError}} \textendash{} if module cannot retrieve data from MDS tree.

\end{description}\end{quote}

\end{fulllineitems}

\index{getBetaT() (eqtools.TCVLIUQE.TCVLIUQETree method)@\spxentry{getBetaT()}\spxextra{eqtools.TCVLIUQE.TCVLIUQETree method}}

\begin{fulllineitems}
\phantomsection\label{\detokenize{eqtools:eqtools.TCVLIUQE.TCVLIUQETree.getBetaT}}\pysiglinewithargsret{\sphinxbfcode{\sphinxupquote{getBetaT}}}{}{}
returns LIUQE-calculated toroidal beta.
\begin{quote}\begin{description}
\item[{Returns}] \leavevmode
{[}nt{]} array of LIUQE-calculated average toroidal beta.

\item[{Return type}] \leavevmode
BetaT (Array)

\item[{Raises}] \leavevmode
\sphinxstyleliteralstrong{\sphinxupquote{ValueError}} \textendash{} if module cannot retrieve data from MDS tree.

\end{description}\end{quote}

\end{fulllineitems}

\index{getBetaP() (eqtools.TCVLIUQE.TCVLIUQETree method)@\spxentry{getBetaP()}\spxextra{eqtools.TCVLIUQE.TCVLIUQETree method}}

\begin{fulllineitems}
\phantomsection\label{\detokenize{eqtools:eqtools.TCVLIUQE.TCVLIUQETree.getBetaP}}\pysiglinewithargsret{\sphinxbfcode{\sphinxupquote{getBetaP}}}{}{}
returns LIUQE-calculated poloidal beta.
\begin{quote}\begin{description}
\item[{Returns}] \leavevmode
{[}nt{]} array of LIUQE-calculated average poloidal beta.

\item[{Return type}] \leavevmode
BetaP (Array)

\item[{Raises}] \leavevmode
\sphinxstyleliteralstrong{\sphinxupquote{ValueError}} \textendash{} if module cannot retrieve data from MDS tree.

\end{description}\end{quote}

\end{fulllineitems}

\index{getLi() (eqtools.TCVLIUQE.TCVLIUQETree method)@\spxentry{getLi()}\spxextra{eqtools.TCVLIUQE.TCVLIUQETree method}}

\begin{fulllineitems}
\phantomsection\label{\detokenize{eqtools:eqtools.TCVLIUQE.TCVLIUQETree.getLi}}\pysiglinewithargsret{\sphinxbfcode{\sphinxupquote{getLi}}}{}{}
returns LIUQE-calculated internal inductance.
\begin{quote}\begin{description}
\item[{Returns}] \leavevmode
{[}nt{]} array of LIUQE-calculated internal inductance.

\item[{Return type}] \leavevmode
Li (Array)

\item[{Raises}] \leavevmode
\sphinxstyleliteralstrong{\sphinxupquote{ValueError}} \textendash{} if module cannot retrieve data from MDS tree.

\end{description}\end{quote}

\end{fulllineitems}

\index{getDiamagWp() (eqtools.TCVLIUQE.TCVLIUQETree method)@\spxentry{getDiamagWp()}\spxextra{eqtools.TCVLIUQE.TCVLIUQETree method}}

\begin{fulllineitems}
\phantomsection\label{\detokenize{eqtools:eqtools.TCVLIUQE.TCVLIUQETree.getDiamagWp}}\pysiglinewithargsret{\sphinxbfcode{\sphinxupquote{getDiamagWp}}}{}{}
returns diamagnetic-loop plasma stored energy.
\begin{quote}\begin{description}
\item[{Returns}] \leavevmode
{[}nt{]} array of measured plasma stored energy.

\item[{Return type}] \leavevmode
Wp (Array)

\item[{Raises}] \leavevmode
\sphinxstyleliteralstrong{\sphinxupquote{ValueError}} \textendash{} if module cannot retrieve data from MDS tree.

\end{description}\end{quote}

\end{fulllineitems}

\index{getTauMHD() (eqtools.TCVLIUQE.TCVLIUQETree method)@\spxentry{getTauMHD()}\spxextra{eqtools.TCVLIUQE.TCVLIUQETree method}}

\begin{fulllineitems}
\phantomsection\label{\detokenize{eqtools:eqtools.TCVLIUQE.TCVLIUQETree.getTauMHD}}\pysiglinewithargsret{\sphinxbfcode{\sphinxupquote{getTauMHD}}}{}{}
returns LIUQE-calculated MHD energy confinement time.
\begin{quote}\begin{description}
\item[{Returns}] \leavevmode
{[}nt{]} array of LIUQE-calculated energy confinement time.

\item[{Return type}] \leavevmode
tauMHD (Array)

\item[{Raises}] \leavevmode
\sphinxstyleliteralstrong{\sphinxupquote{ValueError}} \textendash{} if module cannot retrieve data from MDS tree.

\end{description}\end{quote}

\end{fulllineitems}

\index{getMachineCrossSection() (eqtools.TCVLIUQE.TCVLIUQETree method)@\spxentry{getMachineCrossSection()}\spxextra{eqtools.TCVLIUQE.TCVLIUQETree method}}

\begin{fulllineitems}
\phantomsection\label{\detokenize{eqtools:eqtools.TCVLIUQE.TCVLIUQETree.getMachineCrossSection}}\pysiglinewithargsret{\sphinxbfcode{\sphinxupquote{getMachineCrossSection}}}{}{}
Pulls TCV cross-section data from tree, converts to plottable
vector format for use in other plotting routines
\begin{quote}\begin{description}
\item[{Returns}] \leavevmode

(\sphinxtitleref{x}, \sphinxtitleref{y})
\begin{itemize}
\item {} 
\sphinxstylestrong{x} (\sphinxtitleref{Array}) - {[}n{]} array of x-values for machine cross-section.

\item {} 
\sphinxstylestrong{y} (\sphinxtitleref{Array}) - {[}n{]} array of y-values for machine cross-section.

\end{itemize}


\item[{Raises}] \leavevmode
\sphinxstyleliteralstrong{\sphinxupquote{ValueError}} \textendash{} if module cannot retrieve data from MDS tree.

\end{description}\end{quote}

\end{fulllineitems}

\index{getMachineCrossSectionPatch() (eqtools.TCVLIUQE.TCVLIUQETree method)@\spxentry{getMachineCrossSectionPatch()}\spxextra{eqtools.TCVLIUQE.TCVLIUQETree method}}

\begin{fulllineitems}
\phantomsection\label{\detokenize{eqtools:eqtools.TCVLIUQE.TCVLIUQETree.getMachineCrossSectionPatch}}\pysiglinewithargsret{\sphinxbfcode{\sphinxupquote{getMachineCrossSectionPatch}}}{}{}
Pulls TCV cross-section data from tree, converts it directly to
a matplotlib patch which can be simply added to the approriate axes
call in plotFlux()
\begin{quote}\begin{description}
\item[{Returns}] \leavevmode
tiles matplotlib Patch, vessel matplotlib Patch

\item[{Raises}] \leavevmode
\sphinxstyleliteralstrong{\sphinxupquote{ValueError}} \textendash{} if module cannot retrieve data from MDS tree.

\end{description}\end{quote}

\end{fulllineitems}

\index{plotFlux() (eqtools.TCVLIUQE.TCVLIUQETree method)@\spxentry{plotFlux()}\spxextra{eqtools.TCVLIUQE.TCVLIUQETree method}}

\begin{fulllineitems}
\phantomsection\label{\detokenize{eqtools:eqtools.TCVLIUQE.TCVLIUQETree.plotFlux}}\pysiglinewithargsret{\sphinxbfcode{\sphinxupquote{plotFlux}}}{\emph{fill=True}, \emph{mask=False}}{}
Plots LIQUE TCV flux contours directly from psi grid.

Returns the Figure instance created and the time slider widget (in case
you need to modify the callback). \sphinxtitleref{f.axes} contains the contour plot as
the first element and the time slice slider as the second element.
\begin{quote}\begin{description}
\item[{Keyword Arguments}] \leavevmode
\sphinxstyleliteralstrong{\sphinxupquote{fill}} (\sphinxstyleliteralemphasis{\sphinxupquote{Boolean}}) \textendash{} Set True to plot filled contours.  Set False (default) to plot white-background
color contours.

\end{description}\end{quote}

\end{fulllineitems}


\end{fulllineitems}

\index{TCVLIUQETreeProp (class in eqtools.TCVLIUQE)@\spxentry{TCVLIUQETreeProp}\spxextra{class in eqtools.TCVLIUQE}}

\begin{fulllineitems}
\phantomsection\label{\detokenize{eqtools:eqtools.TCVLIUQE.TCVLIUQETreeProp}}\pysiglinewithargsret{\sphinxbfcode{\sphinxupquote{class }}\sphinxcode{\sphinxupquote{eqtools.TCVLIUQE.}}\sphinxbfcode{\sphinxupquote{TCVLIUQETreeProp}}}{\emph{shot}, \emph{tree='tcv\_shot'}, \emph{length\_unit='m'}, \emph{gfile='g\_eqdsk'}, \emph{afile='a\_eqdsk'}, \emph{tspline=False}, \emph{monotonic=True}}{}
Bases: {\hyperref[\detokenize{eqtools:eqtools.TCVLIUQE.TCVLIUQETree}]{\sphinxcrossref{\sphinxcode{\sphinxupquote{eqtools.TCVLIUQE.TCVLIUQETree}}}}}, {\hyperref[\detokenize{eqtools:eqtools.core.PropertyAccessMixin}]{\sphinxcrossref{\sphinxcode{\sphinxupquote{eqtools.core.PropertyAccessMixin}}}}}

TCVLIUQETree with the PropertyAccessMixin added to enable property-style
access. This is good for interactive use, but may drag the performance down.

\end{fulllineitems}



\subsection{eqtools.afilereader module}
\label{\detokenize{eqtools:module-eqtools.afilereader}}\label{\detokenize{eqtools:eqtools-afilereader-module}}\index{eqtools.afilereader (module)@\spxentry{eqtools.afilereader}\spxextra{module}}
This module contains the AFileReader class, a lightweight data
handler for a-file (time-history) datasets.
\begin{description}
\item[{Classes:}] \leavevmode\begin{description}
\item[{AFileReader:}] \leavevmode
Data-storage class for a-file data.  Reads
data from ASCII a-file, storing as copy-safe object
attributes.

\end{description}

\end{description}
\index{AFileReader (class in eqtools.afilereader)@\spxentry{AFileReader}\spxextra{class in eqtools.afilereader}}

\begin{fulllineitems}
\phantomsection\label{\detokenize{eqtools:eqtools.afilereader.AFileReader}}\pysiglinewithargsret{\sphinxbfcode{\sphinxupquote{class }}\sphinxcode{\sphinxupquote{eqtools.afilereader.}}\sphinxbfcode{\sphinxupquote{AFileReader}}}{\emph{afile}}{}
Bases: \sphinxcode{\sphinxupquote{object}}

Class to read ASCII a-file (time-history data storage) into lightweight,
user-friendly data structure.

A-files store data blocks of scalar time-history data for EFIT
plasma equilibrium.  Each parameter is read into a pseudo-private object
attribute (marked by a leading underscore), followed by the standard
EFIT variable names.

initialize object, reading from file.
\begin{quote}\begin{description}
\item[{Parameters}] \leavevmode
\sphinxstyleliteralstrong{\sphinxupquote{afile}} (\sphinxstyleliteralemphasis{\sphinxupquote{String}}) \textendash{} file path to a-file

\end{description}\end{quote}
\subsubsection*{Examples}

Load a-file data located at \sphinxtitleref{file\_path}:

\begin{sphinxVerbatim}[commandchars=\\\{\}]
\PYG{n}{afr} \PYG{o}{=} \PYG{n}{eqtools}\PYG{o}{.}\PYG{n}{AFileReader}\PYG{p}{(}\PYG{n}{file\PYGZus{}path}\PYG{p}{)}
\end{sphinxVerbatim}

Recover a datapoint (for example, \sphinxtitleref{shot}, stored as \sphinxtitleref{afr.\_shot}),
using copy-protected \_\_getattribute\_\_ method:

\begin{sphinxVerbatim}[commandchars=\\\{\}]
\PYG{n}{shot} \PYG{o}{=} \PYG{n}{afr}\PYG{o}{.}\PYG{n}{shot}
\end{sphinxVerbatim}

Assign a new attribute to afr \textendash{} note that this will raise an
AttributeError if attempting to overwrite a previously-stored
attribute:

\begin{sphinxVerbatim}[commandchars=\\\{\}]
\PYG{n}{afr}\PYG{o}{.}\PYG{n}{attribute} \PYG{o}{=} \PYG{n}{val}
\end{sphinxVerbatim}

\end{fulllineitems}



\subsection{eqtools.core module}
\label{\detokenize{eqtools:module-eqtools.core}}\label{\detokenize{eqtools:eqtools-core-module}}\index{eqtools.core (module)@\spxentry{eqtools.core}\spxextra{module}}
This module provides the core classes for {\hyperref[\detokenize{eqtools:module-eqtools}]{\sphinxcrossref{\sphinxcode{\sphinxupquote{eqtools}}}}}, including the
base {\hyperref[\detokenize{eqtools:eqtools.core.Equilibrium}]{\sphinxcrossref{\sphinxcode{\sphinxupquote{Equilibrium}}}}} class.
\index{ModuleWarning@\spxentry{ModuleWarning}}

\begin{fulllineitems}
\phantomsection\label{\detokenize{eqtools:eqtools.core.ModuleWarning}}\pysigline{\sphinxbfcode{\sphinxupquote{exception }}\sphinxcode{\sphinxupquote{eqtools.core.}}\sphinxbfcode{\sphinxupquote{ModuleWarning}}}
Bases: \sphinxcode{\sphinxupquote{Warning}}

Warning class to notify the user of unavailable modules.

\end{fulllineitems}

\index{PropertyAccessMixin (class in eqtools.core)@\spxentry{PropertyAccessMixin}\spxextra{class in eqtools.core}}

\begin{fulllineitems}
\phantomsection\label{\detokenize{eqtools:eqtools.core.PropertyAccessMixin}}\pysigline{\sphinxbfcode{\sphinxupquote{class }}\sphinxcode{\sphinxupquote{eqtools.core.}}\sphinxbfcode{\sphinxupquote{PropertyAccessMixin}}}
Bases: \sphinxcode{\sphinxupquote{object}}

Mixin to implement access of getter methods through a property-type
interface without the need to apply a decorator to every property.

For any getter \sphinxtitleref{obj.getSomething()}, the call \sphinxtitleref{obj.Something} will do the
same thing.

This is accomplished by overriding \sphinxcode{\sphinxupquote{\_\_getattribute\_\_()}} such that if
an attribute \sphinxtitleref{ATTR} does not exist it then attempts to call \sphinxtitleref{self.getATTR()}.
If \sphinxtitleref{self.getATTR()} does not exist, an \sphinxcode{\sphinxupquote{AttributeError}} will be
raised as usual.

Also overrides \sphinxcode{\sphinxupquote{\_\_setattr\_\_()}} such that it will raise an
\sphinxcode{\sphinxupquote{AttributeError}} when attempting to write an attribute \sphinxtitleref{ATTR} for
which there is already a method \sphinxtitleref{getATTR}.

\end{fulllineitems}

\index{inPolygon() (in module eqtools.core)@\spxentry{inPolygon()}\spxextra{in module eqtools.core}}

\begin{fulllineitems}
\phantomsection\label{\detokenize{eqtools:eqtools.core.inPolygon}}\pysiglinewithargsret{\sphinxcode{\sphinxupquote{eqtools.core.}}\sphinxbfcode{\sphinxupquote{inPolygon}}}{\emph{polyx}, \emph{polyy}, \emph{pointx}, \emph{pointy}}{}
Function calculating whether a given point is within a 2D polygon.

Given an array of X,Y coordinates describing a 2D polygon, checks whether a
point given by x,y coordinates lies within the polygon. Operates via a
ray-casting approach - the function projects a semi-infinite ray parallel to
the positive horizontal axis, and counts how many edges of the polygon this
ray intersects. For a simply-connected polygon, this determines whether the
point is inside (even number of crossings) or outside (odd number of
crossings) the polygon, by the Jordan Curve Theorem.
\begin{quote}\begin{description}
\item[{Parameters}] \leavevmode\begin{itemize}
\item {} 
\sphinxstyleliteralstrong{\sphinxupquote{polyx}} (\sphinxstyleliteralemphasis{\sphinxupquote{Array-like}}) \textendash{} Array of x-coordinates of the vertices of the polygon.

\item {} 
\sphinxstyleliteralstrong{\sphinxupquote{polyy}} (\sphinxstyleliteralemphasis{\sphinxupquote{Array-like}}) \textendash{} Array of y-coordinates of the vertices of the polygon.

\item {} 
\sphinxstyleliteralstrong{\sphinxupquote{pointx}} (\sphinxstyleliteralemphasis{\sphinxupquote{Int}}\sphinxstyleliteralemphasis{\sphinxupquote{ or }}\sphinxstyleliteralemphasis{\sphinxupquote{float}}) \textendash{} x-coordinate of test point.

\item {} 
\sphinxstyleliteralstrong{\sphinxupquote{pointy}} (\sphinxstyleliteralemphasis{\sphinxupquote{Int}}\sphinxstyleliteralemphasis{\sphinxupquote{ or }}\sphinxstyleliteralemphasis{\sphinxupquote{float}}) \textendash{} y-coordinate of test point.

\end{itemize}

\item[{Returns}] \leavevmode
True/False result for whether the point is contained within the polygon.

\item[{Return type}] \leavevmode
result (Boolean)

\end{description}\end{quote}

\end{fulllineitems}

\index{Equilibrium (class in eqtools.core)@\spxentry{Equilibrium}\spxextra{class in eqtools.core}}

\begin{fulllineitems}
\phantomsection\label{\detokenize{eqtools:eqtools.core.Equilibrium}}\pysiglinewithargsret{\sphinxbfcode{\sphinxupquote{class }}\sphinxcode{\sphinxupquote{eqtools.core.}}\sphinxbfcode{\sphinxupquote{Equilibrium}}}{\emph{length\_unit='m'}, \emph{tspline=False}, \emph{monotonic=True}, \emph{verbose=True}}{}
Bases: \sphinxcode{\sphinxupquote{object}}

Abstract class of data handling object for magnetic reconstruction outputs.

Defines the mapping routines and method fingerprints necessary. Each
variable or set of variables is recovered with a corresponding getter method.
Essential data for mapping are pulled on initialization (psirz grid, for
example) to frontload overhead. Additional data are pulled at the first
request and stored for subsequent usage.

\begin{sphinxadmonition}{note}{Note:}
This abstract class should not be used directly. Device- and code-
specific subclasses are set up to account for inter-device/-code
differences in data storage.
\end{sphinxadmonition}
\begin{quote}\begin{description}
\item[{Keyword Arguments}] \leavevmode\begin{itemize}
\item {} 
\sphinxstyleliteralstrong{\sphinxupquote{length\_unit}} (\sphinxstyleliteralemphasis{\sphinxupquote{String}}) \textendash{} 
Sets the base unit used for any quantity whose
dimensions are length to any power. Valid options are:
\begin{quote}


\begin{savenotes}\sphinxattablestart
\centering
\begin{tabulary}{\linewidth}[t]{|T|T|}
\hline

’m’
&
meters
\\
\hline
’cm’
&
centimeters
\\
\hline
’mm’
&
millimeters
\\
\hline
’in’
&
inches
\\
\hline
’ft’
&
feet
\\
\hline
’yd’
&
yards
\\
\hline
’smoot’
&
smoots
\\
\hline
’cubit’
&
cubits
\\
\hline
’hand’
&
hands
\\
\hline
’default’
&
whatever the default in the tree is (no conversion is performed, units may be inconsistent)
\\
\hline
\end{tabulary}
\par
\sphinxattableend\end{savenotes}
\end{quote}

Default is ‘m’ (all units taken and returned in meters).


\item {} 
\sphinxstyleliteralstrong{\sphinxupquote{tspline}} (\sphinxstyleliteralemphasis{\sphinxupquote{Boolean}}) \textendash{} Sets whether or not interpolation in time is
performed using a tricubic spline or nearest-neighbor interpolation.
Tricubic spline interpolation requires at least four complete
equilibria at different times. It is also assumed that they are
functionally correlated, and that parameters do not vary out of
their boundaries (derivative = 0 boundary condition). Default is
False (use nearest-neighbor interpolation).

\item {} 
\sphinxstyleliteralstrong{\sphinxupquote{monotonic}} (\sphinxstyleliteralemphasis{\sphinxupquote{Boolean}}) \textendash{} Sets whether or not the “monotonic” form of time
window finding is used. If True, the timebase must be monotonically
increasing. Default is False (use slower, safer method).

\item {} 
\sphinxstyleliteralstrong{\sphinxupquote{verbose}} (\sphinxstyleliteralemphasis{\sphinxupquote{Boolean}}) \textendash{} Allows or blocks console readout during operation.
Defaults to True, displaying useful information for the user. Set to
False for quiet usage or to avoid console clutter for multiple
instances.

\end{itemize}

\item[{Raises}] \leavevmode\begin{itemize}
\item {} 
\sphinxstyleliteralstrong{\sphinxupquote{ValueError}} \textendash{} If \sphinxtitleref{length\_unit} is not a valid unit specifier.

\item {} 
\sphinxstyleliteralstrong{\sphinxupquote{ValueError}} \textendash{} If \sphinxtitleref{tspline} is True but module trispline did not load
    successfully.

\end{itemize}

\end{description}\end{quote}
\index{rho2rho() (eqtools.core.Equilibrium method)@\spxentry{rho2rho()}\spxextra{eqtools.core.Equilibrium method}}

\begin{fulllineitems}
\phantomsection\label{\detokenize{eqtools:eqtools.core.Equilibrium.rho2rho}}\pysiglinewithargsret{\sphinxbfcode{\sphinxupquote{rho2rho}}}{\emph{origin}, \emph{destination}, \emph{*args}, \emph{**kwargs}}{}
Convert from one coordinate to another.
\begin{quote}\begin{description}
\item[{Parameters}] \leavevmode\begin{itemize}
\item {} 
\sphinxstyleliteralstrong{\sphinxupquote{origin}} (\sphinxstyleliteralemphasis{\sphinxupquote{String}}) \textendash{} 
Indicates which coordinates the data are given in.
Valid options are:
\begin{quote}


\begin{savenotes}\sphinxattablestart
\centering
\begin{tabulary}{\linewidth}[t]{|T|T|}
\hline

RZ
&
R,Z coordinates
\\
\hline
psinorm
&
Normalized poloidal flux
\\
\hline
phinorm
&
Normalized toroidal flux
\\
\hline
volnorm
&
Normalized volume
\\
\hline
Rmid
&
Midplane major radius
\\
\hline
r/a
&
Normalized minor radius
\\
\hline
\end{tabulary}
\par
\sphinxattableend\end{savenotes}
\end{quote}

Additionally, each valid option may be prepended with ‘sqrt’
to specify the square root of the desired unit.


\item {} 
\sphinxstyleliteralstrong{\sphinxupquote{destination}} (\sphinxstyleliteralemphasis{\sphinxupquote{String}}) \textendash{} 
Indicates which coordinates to convert to.
Valid options are:
\begin{quote}


\begin{savenotes}\sphinxattablestart
\centering
\begin{tabulary}{\linewidth}[t]{|T|T|}
\hline

psinorm
&
Normalized poloidal flux
\\
\hline
phinorm
&
Normalized toroidal flux
\\
\hline
volnorm
&
Normalized volume
\\
\hline
Rmid
&
Midplane major radius
\\
\hline
r/a
&
Normalized minor radius
\\
\hline
q
&
Safety factor
\\
\hline
F
&
Flux function \(F=RB_{\phi}\)
\\
\hline
FFPrime
&
Flux function \(FF'\)
\\
\hline
p
&
Pressure
\\
\hline
pprime
&
Pressure gradient
\\
\hline
v
&
Flux surface volume
\\
\hline
\end{tabulary}
\par
\sphinxattableend\end{savenotes}
\end{quote}

Additionally, each valid option may be prepended with ‘sqrt’
to specify the square root of the desired unit.


\item {} 
\sphinxstyleliteralstrong{\sphinxupquote{rho}} (\sphinxstyleliteralemphasis{\sphinxupquote{Array-like}}\sphinxstyleliteralemphasis{\sphinxupquote{ or }}\sphinxstyleliteralemphasis{\sphinxupquote{scalar float}}) \textendash{} Values of the starting coordinate
to map to the new coordinate. Will be two arguments \sphinxtitleref{R}, \sphinxtitleref{Z} if
\sphinxtitleref{origin} is ‘RZ’.

\item {} 
\sphinxstyleliteralstrong{\sphinxupquote{t}} (\sphinxstyleliteralemphasis{\sphinxupquote{Array-like}}\sphinxstyleliteralemphasis{\sphinxupquote{ or }}\sphinxstyleliteralemphasis{\sphinxupquote{scalar float}}) \textendash{} Times to perform the conversion at.
If \sphinxtitleref{t} is a single value, it is used for all of the elements of
\sphinxtitleref{rho}. If the \sphinxtitleref{each\_t} keyword is True, then \sphinxtitleref{t} must be scalar
or have exactly one dimension. If the \sphinxtitleref{each\_t} keyword is False,
\sphinxtitleref{t} must have the same shape as \sphinxtitleref{rho} (or the meshgrid of \sphinxtitleref{R}
and \sphinxtitleref{Z} if \sphinxtitleref{make\_grid} is True).

\end{itemize}

\item[{Keyword Arguments}] \leavevmode\begin{itemize}
\item {} 
\sphinxstyleliteralstrong{\sphinxupquote{sqrt}} (\sphinxstyleliteralemphasis{\sphinxupquote{Boolean}}) \textendash{} Set to True to return the square root of \sphinxtitleref{rho}. Only
the square root of positive values is taken. Negative values are
replaced with zeros, consistent with Steve Wolfe’s IDL
implementation efit\_rz2rho.pro. Default is False.

\item {} 
\sphinxstyleliteralstrong{\sphinxupquote{each\_t}} (\sphinxstyleliteralemphasis{\sphinxupquote{Boolean}}) \textendash{} When True, the elements in \sphinxtitleref{rho} are evaluated at
each value in \sphinxtitleref{t}. If True, \sphinxtitleref{t} must have only one dimension (or
be a scalar). If False, \sphinxtitleref{t} must match the shape of \sphinxtitleref{rho} or be
a scalar. Default is True (evaluate ALL \sphinxtitleref{rho} at EACH element in
\sphinxtitleref{t}).

\item {} 
\sphinxstyleliteralstrong{\sphinxupquote{make\_grid}} (\sphinxstyleliteralemphasis{\sphinxupquote{Boolean}}) \textendash{} Only applicable if \sphinxtitleref{origin} is ‘RZ’. Set to
True to pass \sphinxtitleref{R} and \sphinxtitleref{Z} through \sphinxcode{\sphinxupquote{scipy.meshgrid()}}
before evaluating. If this is set to True, \sphinxtitleref{R} and \sphinxtitleref{Z} must each
only have a single dimension, but can have different lengths.
Default is False (do not form meshgrid).

\item {} 
\sphinxstyleliteralstrong{\sphinxupquote{rho}} (\sphinxstyleliteralemphasis{\sphinxupquote{Boolean}}) \textendash{} Set to True to return r/a (normalized minor radius)
instead of Rmid when \sphinxtitleref{destination} is Rmid. Default is False
(return major radius, Rmid).

\item {} 
\sphinxstyleliteralstrong{\sphinxupquote{length\_unit}} (\sphinxstyleliteralemphasis{\sphinxupquote{String}}\sphinxstyleliteralemphasis{\sphinxupquote{ or }}\sphinxstyleliteralemphasis{\sphinxupquote{1}}) \textendash{} 
Length unit that quantities are
given/returned in, as applicable. If a string is given, it must
be a valid unit specifier:
\begin{quote}


\begin{savenotes}\sphinxattablestart
\centering
\begin{tabulary}{\linewidth}[t]{|T|T|}
\hline

’m’
&
meters
\\
\hline
’cm’
&
centimeters
\\
\hline
’mm’
&
millimeters
\\
\hline
’in’
&
inches
\\
\hline
’ft’
&
feet
\\
\hline
’yd’
&
yards
\\
\hline
’smoot’
&
smoots
\\
\hline
’cubit’
&
cubits
\\
\hline
’hand’
&
hands
\\
\hline
’default’
&
meters
\\
\hline
\end{tabulary}
\par
\sphinxattableend\end{savenotes}
\end{quote}

If length\_unit is 1 or None, meters are assumed. The default
value is 1 (use meters).


\item {} 
\sphinxstyleliteralstrong{\sphinxupquote{k}} (\sphinxstyleliteralemphasis{\sphinxupquote{positive int}}) \textendash{} The degree of polynomial spline interpolation to
use in converting coordinates.

\item {} 
\sphinxstyleliteralstrong{\sphinxupquote{return\_t}} (\sphinxstyleliteralemphasis{\sphinxupquote{Boolean}}) \textendash{} Set to True to return a tuple of (\sphinxtitleref{rho},
\sphinxtitleref{time\_idxs}), where \sphinxtitleref{time\_idxs} is the array of time indices
actually used in evaluating \sphinxtitleref{rho} with nearest-neighbor
interpolation. (This is mostly present as an internal helper.)
Default is False (only return \sphinxtitleref{rho}).

\end{itemize}

\item[{Returns}] \leavevmode

\sphinxtitleref{rho} or (\sphinxtitleref{rho}, \sphinxtitleref{time\_idxs})
\begin{itemize}
\item {} 
\sphinxstylestrong{rho} (\sphinxtitleref{Array or scalar float}) - The converted coordinates. If
all of the input arguments are scalar, then a scalar is returned.
Otherwise, a scipy Array is returned.

\item {} 
\sphinxstylestrong{time\_idxs} (Array with same shape as \sphinxtitleref{rho}) - The indices
(in \sphinxcode{\sphinxupquote{self.getTimeBase()}}) that were used for
nearest-neighbor interpolation. Only returned if \sphinxtitleref{return\_t} is
True.

\end{itemize}


\item[{Raises}] \leavevmode
\sphinxstyleliteralstrong{\sphinxupquote{ValueError}} \textendash{} If \sphinxtitleref{origin} is not one of the supported values.

\end{description}\end{quote}
\subsubsection*{Examples}

All assume that \sphinxtitleref{Eq\_instance} is a valid instance of the appropriate
extension of the {\hyperref[\detokenize{eqtools:eqtools.core.Equilibrium}]{\sphinxcrossref{\sphinxcode{\sphinxupquote{Equilibrium}}}}} abstract class.

Find single psinorm value at r/a=0.6, t=0.26s:

\begin{sphinxVerbatim}[commandchars=\\\{\}]
\PYG{n}{psi\PYGZus{}val} \PYG{o}{=} \PYG{n}{Eq\PYGZus{}instance}\PYG{o}{.}\PYG{n}{rho2rho}\PYG{p}{(}\PYG{l+s+s1}{\PYGZsq{}}\PYG{l+s+s1}{r/a}\PYG{l+s+s1}{\PYGZsq{}}\PYG{p}{,} \PYG{l+s+s1}{\PYGZsq{}}\PYG{l+s+s1}{psinorm}\PYG{l+s+s1}{\PYGZsq{}}\PYG{p}{,} \PYG{l+m+mf}{0.6}\PYG{p}{,} \PYG{l+m+mf}{0.26}\PYG{p}{)}
\end{sphinxVerbatim}

Find psinorm values at r/a points 0.6 and 0.8 at the
single time t=0.26s:

\begin{sphinxVerbatim}[commandchars=\\\{\}]
\PYG{n}{psi\PYGZus{}arr} \PYG{o}{=} \PYG{n}{Eq\PYGZus{}instance}\PYG{o}{.}\PYG{n}{rho2rho}\PYG{p}{(}\PYG{l+s+s1}{\PYGZsq{}}\PYG{l+s+s1}{r/a}\PYG{l+s+s1}{\PYGZsq{}}\PYG{p}{,} \PYG{l+s+s1}{\PYGZsq{}}\PYG{l+s+s1}{psinorm}\PYG{l+s+s1}{\PYGZsq{}}\PYG{p}{,} \PYG{p}{[}\PYG{l+m+mf}{0.6}\PYG{p}{,} \PYG{l+m+mf}{0.8}\PYG{p}{]}\PYG{p}{,} \PYG{l+m+mf}{0.26}\PYG{p}{)}
\end{sphinxVerbatim}

Find psinorm values at r/a of 0.6 at times t={[}0.2s, 0.3s{]}:

\begin{sphinxVerbatim}[commandchars=\\\{\}]
\PYG{n}{psi\PYGZus{}arr} \PYG{o}{=} \PYG{n}{Eq\PYGZus{}instance}\PYG{o}{.}\PYG{n}{rho2rho}\PYG{p}{(}\PYG{l+s+s1}{\PYGZsq{}}\PYG{l+s+s1}{r/a}\PYG{l+s+s1}{\PYGZsq{}}\PYG{p}{,} \PYG{l+s+s1}{\PYGZsq{}}\PYG{l+s+s1}{psinorm}\PYG{l+s+s1}{\PYGZsq{}}\PYG{p}{,} \PYG{l+m+mf}{0.6}\PYG{p}{,} \PYG{p}{[}\PYG{l+m+mf}{0.2}\PYG{p}{,} \PYG{l+m+mf}{0.3}\PYG{p}{]}\PYG{p}{)}
\end{sphinxVerbatim}

Find psinorm values at (r/a, t) points (0.6, 0.2s) and (0.5, 0.3s):

\begin{sphinxVerbatim}[commandchars=\\\{\}]
\PYG{n}{psi\PYGZus{}arr} \PYG{o}{=} \PYG{n}{Eq\PYGZus{}instance}\PYG{o}{.}\PYG{n}{rho2rho}\PYG{p}{(}\PYG{l+s+s1}{\PYGZsq{}}\PYG{l+s+s1}{r/a}\PYG{l+s+s1}{\PYGZsq{}}\PYG{p}{,} \PYG{l+s+s1}{\PYGZsq{}}\PYG{l+s+s1}{psinorm}\PYG{l+s+s1}{\PYGZsq{}}\PYG{p}{,} \PYG{p}{[}\PYG{l+m+mf}{0.6}\PYG{p}{,} \PYG{l+m+mf}{0.5}\PYG{p}{]}\PYG{p}{,} \PYG{p}{[}\PYG{l+m+mf}{0.2}\PYG{p}{,} \PYG{l+m+mf}{0.3}\PYG{p}{]}\PYG{p}{,} \PYG{n}{each\PYGZus{}t}\PYG{o}{=}\PYG{k+kc}{False}\PYG{p}{)}
\end{sphinxVerbatim}

\end{fulllineitems}

\index{rz2psi() (eqtools.core.Equilibrium method)@\spxentry{rz2psi()}\spxextra{eqtools.core.Equilibrium method}}

\begin{fulllineitems}
\phantomsection\label{\detokenize{eqtools:eqtools.core.Equilibrium.rz2psi}}\pysiglinewithargsret{\sphinxbfcode{\sphinxupquote{rz2psi}}}{\emph{R}, \emph{Z}, \emph{t}, \emph{return\_t=False}, \emph{make\_grid=False}, \emph{each\_t=True}, \emph{length\_unit=1}}{}
Converts the passed R, Z, t arrays to psi (unnormalized poloidal flux) values.

What is usually returned by EFIT is the stream function,
\(\psi=\psi_p/(2\pi)\) which has units of Wb/rad.
\begin{quote}\begin{description}
\item[{Parameters}] \leavevmode\begin{itemize}
\item {} 
\sphinxstyleliteralstrong{\sphinxupquote{R}} (\sphinxstyleliteralemphasis{\sphinxupquote{Array-like}}\sphinxstyleliteralemphasis{\sphinxupquote{ or }}\sphinxstyleliteralemphasis{\sphinxupquote{scalar float}}) \textendash{} Values of the radial coordinate to
map to poloidal flux. If \sphinxtitleref{R} and \sphinxtitleref{Z} are both scalar values,
they are used as the coordinate pair for all of the values in
\sphinxtitleref{t}. Must have the same shape as \sphinxtitleref{Z} unless the \sphinxtitleref{make\_grid}
keyword is set. If the \sphinxtitleref{make\_grid} keyword is True, \sphinxtitleref{R} must
have exactly one dimension.

\item {} 
\sphinxstyleliteralstrong{\sphinxupquote{Z}} (\sphinxstyleliteralemphasis{\sphinxupquote{Array-like}}\sphinxstyleliteralemphasis{\sphinxupquote{ or }}\sphinxstyleliteralemphasis{\sphinxupquote{scalar float}}) \textendash{} Values of the vertical coordinate to
map to poloidal flux. If \sphinxtitleref{R} and \sphinxtitleref{Z} are both scalar values,
they are used as the coordinate pair for all of the values in
\sphinxtitleref{t}. Must have the same shape as \sphinxtitleref{R} unless the \sphinxtitleref{make\_grid}
keyword is set. If the \sphinxtitleref{make\_grid} keyword is True, \sphinxtitleref{Z} must
have exactly one dimension.

\item {} 
\sphinxstyleliteralstrong{\sphinxupquote{t}} (\sphinxstyleliteralemphasis{\sphinxupquote{Array-like}}\sphinxstyleliteralemphasis{\sphinxupquote{ or }}\sphinxstyleliteralemphasis{\sphinxupquote{scalar float}}) \textendash{} Times to perform the conversion at.
If \sphinxtitleref{t} is a single value, it is used for all of the elements of
\sphinxtitleref{R}, \sphinxtitleref{Z}. If the \sphinxtitleref{each\_t} keyword is True, then \sphinxtitleref{t} must be
scalar or have exactly one dimension. If the \sphinxtitleref{each\_t} keyword is
False, \sphinxtitleref{t} must have the same shape as \sphinxtitleref{R} and \sphinxtitleref{Z} (or their
meshgrid if \sphinxtitleref{make\_grid} is True).

\end{itemize}

\item[{Keyword Arguments}] \leavevmode\begin{itemize}
\item {} 
\sphinxstyleliteralstrong{\sphinxupquote{each\_t}} (\sphinxstyleliteralemphasis{\sphinxupquote{Boolean}}) \textendash{} When True, the elements in \sphinxtitleref{R}, \sphinxtitleref{Z} are evaluated
at each value in \sphinxtitleref{t}. If True, \sphinxtitleref{t} must have only one dimension
(or be a scalar). If False, \sphinxtitleref{t} must match the shape of \sphinxtitleref{R} and
\sphinxtitleref{Z} or be a scalar. Default is True (evaluate ALL \sphinxtitleref{R}, \sphinxtitleref{Z} at
EACH element in \sphinxtitleref{t}).

\item {} 
\sphinxstyleliteralstrong{\sphinxupquote{make\_grid}} (\sphinxstyleliteralemphasis{\sphinxupquote{Boolean}}) \textendash{} Set to True to pass \sphinxtitleref{R} and \sphinxtitleref{Z} through
\sphinxcode{\sphinxupquote{scipy.meshgrid()}} before evaluating. If this is set to
True, \sphinxtitleref{R} and \sphinxtitleref{Z} must each only have a single dimension, but
can have different lengths. Default is False (do not form
meshgrid).

\item {} 
\sphinxstyleliteralstrong{\sphinxupquote{length\_unit}} (\sphinxstyleliteralemphasis{\sphinxupquote{String}}\sphinxstyleliteralemphasis{\sphinxupquote{ or }}\sphinxstyleliteralemphasis{\sphinxupquote{1}}) \textendash{} 
Length unit that \sphinxtitleref{R}, \sphinxtitleref{Z} are given in.
If a string is given, it must be a valid unit specifier:
\begin{quote}


\begin{savenotes}\sphinxattablestart
\centering
\begin{tabulary}{\linewidth}[t]{|T|T|}
\hline

’m’
&
meters
\\
\hline
’cm’
&
centimeters
\\
\hline
’mm’
&
millimeters
\\
\hline
’in’
&
inches
\\
\hline
’ft’
&
feet
\\
\hline
’yd’
&
yards
\\
\hline
’smoot’
&
smoots
\\
\hline
’cubit’
&
cubits
\\
\hline
’hand’
&
hands
\\
\hline
’default’
&
meters
\\
\hline
\end{tabulary}
\par
\sphinxattableend\end{savenotes}
\end{quote}

If length\_unit is 1 or None, meters are assumed. The default
value is 1 (use meters).


\item {} 
\sphinxstyleliteralstrong{\sphinxupquote{return\_t}} (\sphinxstyleliteralemphasis{\sphinxupquote{Boolean}}) \textendash{} Set to True to return a tuple of (\sphinxtitleref{rho},
\sphinxtitleref{time\_idxs}), where \sphinxtitleref{time\_idxs} is the array of time indices
actually used in evaluating \sphinxtitleref{rho} with nearest-neighbor
interpolation. (This is mostly present as an internal helper.)
Default is False (only return \sphinxtitleref{rho}).

\end{itemize}

\item[{Returns}] \leavevmode

\sphinxtitleref{psi} or (\sphinxtitleref{psi}, \sphinxtitleref{time\_idxs})
\begin{itemize}
\item {} 
\sphinxstylestrong{psi} (\sphinxtitleref{Array or scalar float}) - The unnormalized poloidal
flux. If all of the input arguments are scalar, then a scalar is
returned. Otherwise, a scipy Array is returned. If \sphinxtitleref{R} and \sphinxtitleref{Z}
both have the same shape then \sphinxtitleref{psi} has this shape as well,
unless the \sphinxtitleref{make\_grid} keyword was True, in which case \sphinxtitleref{psi} has
shape (len(\sphinxtitleref{Z}), len(\sphinxtitleref{R})).

\item {} 
\sphinxstylestrong{time\_idxs} (Array with same shape as \sphinxtitleref{psi}) - The indices
(in \sphinxcode{\sphinxupquote{self.getTimeBase()}}) that were used for
nearest-neighbor interpolation. Only returned if \sphinxtitleref{return\_t} is
True.

\end{itemize}


\end{description}\end{quote}
\subsubsection*{Examples}

All assume that \sphinxtitleref{Eq\_instance} is a valid instance of the appropriate
extension of the {\hyperref[\detokenize{eqtools:eqtools.core.Equilibrium}]{\sphinxcrossref{\sphinxcode{\sphinxupquote{Equilibrium}}}}} abstract class.

Find single psi value at R=0.6m, Z=0.0m, t=0.26s:

\begin{sphinxVerbatim}[commandchars=\\\{\}]
\PYG{n}{psi\PYGZus{}val} \PYG{o}{=} \PYG{n}{Eq\PYGZus{}instance}\PYG{o}{.}\PYG{n}{rz2psi}\PYG{p}{(}\PYG{l+m+mf}{0.6}\PYG{p}{,} \PYG{l+m+mi}{0}\PYG{p}{,} \PYG{l+m+mf}{0.26}\PYG{p}{)}
\end{sphinxVerbatim}

Find psi values at (R, Z) points (0.6m, 0m) and (0.8m, 0m) at the
single time t=0.26s. Note that the \sphinxtitleref{Z} vector must be fully
specified, even if the values are all the same:

\begin{sphinxVerbatim}[commandchars=\\\{\}]
\PYG{n}{psi\PYGZus{}arr} \PYG{o}{=} \PYG{n}{Eq\PYGZus{}instance}\PYG{o}{.}\PYG{n}{rz2psi}\PYG{p}{(}\PYG{p}{[}\PYG{l+m+mf}{0.6}\PYG{p}{,} \PYG{l+m+mf}{0.8}\PYG{p}{]}\PYG{p}{,} \PYG{p}{[}\PYG{l+m+mi}{0}\PYG{p}{,} \PYG{l+m+mi}{0}\PYG{p}{]}\PYG{p}{,} \PYG{l+m+mf}{0.26}\PYG{p}{)}
\end{sphinxVerbatim}

Find psi values at (R, Z) points (0.6m, 0m) at times t={[}0.2s, 0.3s{]}:

\begin{sphinxVerbatim}[commandchars=\\\{\}]
\PYG{n}{psi\PYGZus{}arr} \PYG{o}{=} \PYG{n}{Eq\PYGZus{}instance}\PYG{o}{.}\PYG{n}{rz2psi}\PYG{p}{(}\PYG{l+m+mf}{0.6}\PYG{p}{,} \PYG{l+m+mi}{0}\PYG{p}{,} \PYG{p}{[}\PYG{l+m+mf}{0.2}\PYG{p}{,} \PYG{l+m+mf}{0.3}\PYG{p}{]}\PYG{p}{)}
\end{sphinxVerbatim}

Find psi values at (R, Z, t) points (0.6m, 0m, 0.2s) and
(0.5m, 0.2m, 0.3s):

\begin{sphinxVerbatim}[commandchars=\\\{\}]
\PYG{n}{psi\PYGZus{}arr} \PYG{o}{=} \PYG{n}{Eq\PYGZus{}instance}\PYG{o}{.}\PYG{n}{rz2psi}\PYG{p}{(}\PYG{p}{[}\PYG{l+m+mf}{0.6}\PYG{p}{,} \PYG{l+m+mf}{0.5}\PYG{p}{]}\PYG{p}{,} \PYG{p}{[}\PYG{l+m+mi}{0}\PYG{p}{,} \PYG{l+m+mf}{0.2}\PYG{p}{]}\PYG{p}{,} \PYG{p}{[}\PYG{l+m+mf}{0.2}\PYG{p}{,} \PYG{l+m+mf}{0.3}\PYG{p}{]}\PYG{p}{,} \PYG{n}{each\PYGZus{}t}\PYG{o}{=}\PYG{k+kc}{False}\PYG{p}{)}
\end{sphinxVerbatim}

Find psi values on grid defined by 1D vector of radial positions \sphinxtitleref{R}
and 1D vector of vertical positions \sphinxtitleref{Z} at time t=0.2s:

\begin{sphinxVerbatim}[commandchars=\\\{\}]
\PYG{n}{psi\PYGZus{}mat} \PYG{o}{=} \PYG{n}{Eq\PYGZus{}instance}\PYG{o}{.}\PYG{n}{rz2psi}\PYG{p}{(}\PYG{n}{R}\PYG{p}{,} \PYG{n}{Z}\PYG{p}{,} \PYG{l+m+mf}{0.2}\PYG{p}{,} \PYG{n}{make\PYGZus{}grid}\PYG{o}{=}\PYG{k+kc}{True}\PYG{p}{)}
\end{sphinxVerbatim}

\end{fulllineitems}

\index{rz2psinorm() (eqtools.core.Equilibrium method)@\spxentry{rz2psinorm()}\spxextra{eqtools.core.Equilibrium method}}

\begin{fulllineitems}
\phantomsection\label{\detokenize{eqtools:eqtools.core.Equilibrium.rz2psinorm}}\pysiglinewithargsret{\sphinxbfcode{\sphinxupquote{rz2psinorm}}}{\emph{R}, \emph{Z}, \emph{t}, \emph{return\_t=False}, \emph{sqrt=False}, \emph{make\_grid=False}, \emph{each\_t=True}, \emph{length\_unit=1}}{}
Calculates the normalized poloidal flux at the given (R, Z, t).

Uses the definition:
\begin{equation*}
\begin{split}\texttt{psi\_norm} = \frac{\psi - \psi(0)}{\psi(a) - \psi(0)}\end{split}
\end{equation*}\begin{quote}\begin{description}
\item[{Parameters}] \leavevmode\begin{itemize}
\item {} 
\sphinxstyleliteralstrong{\sphinxupquote{R}} (\sphinxstyleliteralemphasis{\sphinxupquote{Array-like}}\sphinxstyleliteralemphasis{\sphinxupquote{ or }}\sphinxstyleliteralemphasis{\sphinxupquote{scalar float}}) \textendash{} Values of the radial coordinate to
map to psinorm. If \sphinxtitleref{R} and \sphinxtitleref{Z} are both scalar values,
they are used as the coordinate pair for all of the values in
\sphinxtitleref{t}. Must have the same shape as \sphinxtitleref{Z} unless the \sphinxtitleref{make\_grid}
keyword is set. If the \sphinxtitleref{make\_grid} keyword is True, \sphinxtitleref{R} must
have exactly one dimension.

\item {} 
\sphinxstyleliteralstrong{\sphinxupquote{Z}} (\sphinxstyleliteralemphasis{\sphinxupquote{Array-like}}\sphinxstyleliteralemphasis{\sphinxupquote{ or }}\sphinxstyleliteralemphasis{\sphinxupquote{scalar float}}) \textendash{} Values of the vertical coordinate to
map to psinorm. If \sphinxtitleref{R} and \sphinxtitleref{Z} are both scalar values,
they are used as the coordinate pair for all of the values in
\sphinxtitleref{t}. Must have the same shape as \sphinxtitleref{R} unless the \sphinxtitleref{make\_grid}
keyword is set. If the \sphinxtitleref{make\_grid} keyword is True, \sphinxtitleref{Z} must
have exactly one dimension.

\item {} 
\sphinxstyleliteralstrong{\sphinxupquote{t}} (\sphinxstyleliteralemphasis{\sphinxupquote{Array-like}}\sphinxstyleliteralemphasis{\sphinxupquote{ or }}\sphinxstyleliteralemphasis{\sphinxupquote{scalar float}}) \textendash{} Times to perform the conversion at.
If \sphinxtitleref{t} is a single value, it is used for all of the elements of
\sphinxtitleref{R}, \sphinxtitleref{Z}. If the \sphinxtitleref{each\_t} keyword is True, then \sphinxtitleref{t} must be
scalar or have exactly one dimension. If the \sphinxtitleref{each\_t} keyword is
False, \sphinxtitleref{t} must have the same shape as \sphinxtitleref{R} and \sphinxtitleref{Z} (or their
meshgrid if \sphinxtitleref{make\_grid} is True).

\end{itemize}

\item[{Keyword Arguments}] \leavevmode\begin{itemize}
\item {} 
\sphinxstyleliteralstrong{\sphinxupquote{sqrt}} (\sphinxstyleliteralemphasis{\sphinxupquote{Boolean}}) \textendash{} Set to True to return the square root of psinorm.
Only the square root of positive values is taken. Negative
values are replaced with zeros, consistent with Steve Wolfe’s
IDL implementation efit\_rz2rho.pro. Default is False.

\item {} 
\sphinxstyleliteralstrong{\sphinxupquote{each\_t}} (\sphinxstyleliteralemphasis{\sphinxupquote{Boolean}}) \textendash{} When True, the elements in \sphinxtitleref{R}, \sphinxtitleref{Z} are evaluated
at each value in \sphinxtitleref{t}. If True, \sphinxtitleref{t} must have only one dimension
(or be a scalar). If False, \sphinxtitleref{t} must match the shape of \sphinxtitleref{R} and
\sphinxtitleref{Z} or be a scalar. Default is True (evaluate ALL \sphinxtitleref{R}, \sphinxtitleref{Z} at
EACH element in \sphinxtitleref{t}).

\item {} 
\sphinxstyleliteralstrong{\sphinxupquote{make\_grid}} (\sphinxstyleliteralemphasis{\sphinxupquote{Boolean}}) \textendash{} Set to True to pass \sphinxtitleref{R} and \sphinxtitleref{Z} through
\sphinxcode{\sphinxupquote{scipy.meshgrid()}} before evaluating. If this is set to
True, \sphinxtitleref{R} and \sphinxtitleref{Z} must each only have a single dimension, but
can have different lengths. Default is False (do not form
meshgrid).

\item {} 
\sphinxstyleliteralstrong{\sphinxupquote{length\_unit}} (\sphinxstyleliteralemphasis{\sphinxupquote{String}}\sphinxstyleliteralemphasis{\sphinxupquote{ or }}\sphinxstyleliteralemphasis{\sphinxupquote{1}}) \textendash{} 
Length unit that \sphinxtitleref{R}, \sphinxtitleref{Z} are given in.
If a string is given, it must be a valid unit specifier:
\begin{quote}


\begin{savenotes}\sphinxattablestart
\centering
\begin{tabulary}{\linewidth}[t]{|T|T|}
\hline

’m’
&
meters
\\
\hline
’cm’
&
centimeters
\\
\hline
’mm’
&
millimeters
\\
\hline
’in’
&
inches
\\
\hline
’ft’
&
feet
\\
\hline
’yd’
&
yards
\\
\hline
’smoot’
&
smoots
\\
\hline
’cubit’
&
cubits
\\
\hline
’hand’
&
hands
\\
\hline
’default’
&
meters
\\
\hline
\end{tabulary}
\par
\sphinxattableend\end{savenotes}
\end{quote}

If length\_unit is 1 or None, meters are assumed. The default
value is 1 (use meters).


\item {} 
\sphinxstyleliteralstrong{\sphinxupquote{return\_t}} (\sphinxstyleliteralemphasis{\sphinxupquote{Boolean}}) \textendash{} Set to True to return a tuple of (\sphinxtitleref{rho},
\sphinxtitleref{time\_idxs}), where \sphinxtitleref{time\_idxs} is the array of time indices
actually used in evaluating \sphinxtitleref{rho} with nearest-neighbor
interpolation. (This is mostly present as an internal helper.)
Default is False (only return \sphinxtitleref{rho}).

\end{itemize}

\item[{Returns}] \leavevmode

\sphinxtitleref{psinorm} or (\sphinxtitleref{psinorm}, \sphinxtitleref{time\_idxs})
\begin{itemize}
\item {} 
\sphinxstylestrong{psinorm} (\sphinxtitleref{Array or scalar float}) - The normalized poloidal
flux. If all of the input arguments are scalar, then a scalar is
returned. Otherwise, a scipy Array is returned. If \sphinxtitleref{R} and \sphinxtitleref{Z}
both have the same shape then \sphinxtitleref{psinorm} has this shape as well,
unless the \sphinxtitleref{make\_grid} keyword was True, in which case \sphinxtitleref{psinorm}
has shape (len(\sphinxtitleref{Z}), len(\sphinxtitleref{R})).

\item {} 
\sphinxstylestrong{time\_idxs} (Array with same shape as \sphinxtitleref{psinorm}) - The indices
(in \sphinxcode{\sphinxupquote{self.getTimeBase()}}) that were used for
nearest-neighbor interpolation. Only returned if \sphinxtitleref{return\_t} is
True.

\end{itemize}


\end{description}\end{quote}
\subsubsection*{Examples}

All assume that \sphinxtitleref{Eq\_instance} is a valid instance of the
appropriate extension of the {\hyperref[\detokenize{eqtools:eqtools.core.Equilibrium}]{\sphinxcrossref{\sphinxcode{\sphinxupquote{Equilibrium}}}}} abstract class.

Find single psinorm value at R=0.6m, Z=0.0m, t=0.26s:

\begin{sphinxVerbatim}[commandchars=\\\{\}]
\PYG{n}{psi\PYGZus{}val} \PYG{o}{=} \PYG{n}{Eq\PYGZus{}instance}\PYG{o}{.}\PYG{n}{rz2psinorm}\PYG{p}{(}\PYG{l+m+mf}{0.6}\PYG{p}{,} \PYG{l+m+mi}{0}\PYG{p}{,} \PYG{l+m+mf}{0.26}\PYG{p}{)}
\end{sphinxVerbatim}

Find psinorm values at (R, Z) points (0.6m, 0m) and (0.8m, 0m) at the
single time t=0.26s. Note that the \sphinxtitleref{Z} vector must be fully specified,
even if the values are all the same:

\begin{sphinxVerbatim}[commandchars=\\\{\}]
\PYG{n}{psi\PYGZus{}arr} \PYG{o}{=} \PYG{n}{Eq\PYGZus{}instance}\PYG{o}{.}\PYG{n}{rz2psinorm}\PYG{p}{(}\PYG{p}{[}\PYG{l+m+mf}{0.6}\PYG{p}{,} \PYG{l+m+mf}{0.8}\PYG{p}{]}\PYG{p}{,} \PYG{p}{[}\PYG{l+m+mi}{0}\PYG{p}{,} \PYG{l+m+mi}{0}\PYG{p}{]}\PYG{p}{,} \PYG{l+m+mf}{0.26}\PYG{p}{)}
\end{sphinxVerbatim}

Find psinorm values at (R, Z) points (0.6m, 0m) at times t={[}0.2s, 0.3s{]}:

\begin{sphinxVerbatim}[commandchars=\\\{\}]
\PYG{n}{psi\PYGZus{}arr} \PYG{o}{=} \PYG{n}{Eq\PYGZus{}instance}\PYG{o}{.}\PYG{n}{rz2psinorm}\PYG{p}{(}\PYG{l+m+mf}{0.6}\PYG{p}{,} \PYG{l+m+mi}{0}\PYG{p}{,} \PYG{p}{[}\PYG{l+m+mf}{0.2}\PYG{p}{,} \PYG{l+m+mf}{0.3}\PYG{p}{]}\PYG{p}{)}
\end{sphinxVerbatim}

Find psinorm values at (R, Z, t) points (0.6m, 0m, 0.2s) and (0.5m, 0.2m, 0.3s):

\begin{sphinxVerbatim}[commandchars=\\\{\}]
\PYG{n}{psi\PYGZus{}arr} \PYG{o}{=} \PYG{n}{Eq\PYGZus{}instance}\PYG{o}{.}\PYG{n}{rz2psinorm}\PYG{p}{(}\PYG{p}{[}\PYG{l+m+mf}{0.6}\PYG{p}{,} \PYG{l+m+mf}{0.5}\PYG{p}{]}\PYG{p}{,} \PYG{p}{[}\PYG{l+m+mi}{0}\PYG{p}{,} \PYG{l+m+mf}{0.2}\PYG{p}{]}\PYG{p}{,} \PYG{p}{[}\PYG{l+m+mf}{0.2}\PYG{p}{,} \PYG{l+m+mf}{0.3}\PYG{p}{]}\PYG{p}{,} \PYG{n}{each\PYGZus{}t}\PYG{o}{=}\PYG{k+kc}{False}\PYG{p}{)}
\end{sphinxVerbatim}

Find psinorm values on grid defined by 1D vector of radial positions \sphinxtitleref{R}
and 1D vector of vertical positions \sphinxtitleref{Z} at time t=0.2s:

\begin{sphinxVerbatim}[commandchars=\\\{\}]
\PYG{n}{psi\PYGZus{}mat} \PYG{o}{=} \PYG{n}{Eq\PYGZus{}instance}\PYG{o}{.}\PYG{n}{rz2psinorm}\PYG{p}{(}\PYG{n}{R}\PYG{p}{,} \PYG{n}{Z}\PYG{p}{,} \PYG{l+m+mf}{0.2}\PYG{p}{,} \PYG{n}{make\PYGZus{}grid}\PYG{o}{=}\PYG{k+kc}{True}\PYG{p}{)}
\end{sphinxVerbatim}

\end{fulllineitems}

\index{rz2phinorm() (eqtools.core.Equilibrium method)@\spxentry{rz2phinorm()}\spxextra{eqtools.core.Equilibrium method}}

\begin{fulllineitems}
\phantomsection\label{\detokenize{eqtools:eqtools.core.Equilibrium.rz2phinorm}}\pysiglinewithargsret{\sphinxbfcode{\sphinxupquote{rz2phinorm}}}{\emph{*args}, \emph{**kwargs}}{}
Calculates the normalized toroidal flux.

Uses the definitions:
\begin{equation*}
\begin{split}\texttt{phi} &= \int q(\psi)\,d\psi\\
\texttt{phi\_norm} &= \frac{\phi}{\phi(a)}\end{split}
\end{equation*}
This is based on the IDL version efit\_rz2rho.pro by Steve Wolfe.
\begin{quote}\begin{description}
\item[{Parameters}] \leavevmode\begin{itemize}
\item {} 
\sphinxstyleliteralstrong{\sphinxupquote{R}} (\sphinxstyleliteralemphasis{\sphinxupquote{Array-like}}\sphinxstyleliteralemphasis{\sphinxupquote{ or }}\sphinxstyleliteralemphasis{\sphinxupquote{scalar float}}) \textendash{} Values of the radial coordinate to
map to phinorm. If \sphinxtitleref{R} and \sphinxtitleref{Z} are both scalar values,
they are used as the coordinate pair for all of the values in
\sphinxtitleref{t}. Must have the same shape as \sphinxtitleref{Z} unless the \sphinxtitleref{make\_grid}
keyword is set. If the \sphinxtitleref{make\_grid} keyword is True, \sphinxtitleref{R} must
have exactly one dimension.

\item {} 
\sphinxstyleliteralstrong{\sphinxupquote{Z}} (\sphinxstyleliteralemphasis{\sphinxupquote{Array-like}}\sphinxstyleliteralemphasis{\sphinxupquote{ or }}\sphinxstyleliteralemphasis{\sphinxupquote{scalar float}}) \textendash{} Values of the vertical coordinate to
map to phinorm. If \sphinxtitleref{R} and \sphinxtitleref{Z} are both scalar values,
they are used as the coordinate pair for all of the values in
\sphinxtitleref{t}. Must have the same shape as \sphinxtitleref{R} unless the \sphinxtitleref{make\_grid}
keyword is set. If the \sphinxtitleref{make\_grid} keyword is True, \sphinxtitleref{Z} must
have exactly one dimension.

\item {} 
\sphinxstyleliteralstrong{\sphinxupquote{t}} (\sphinxstyleliteralemphasis{\sphinxupquote{Array-like}}\sphinxstyleliteralemphasis{\sphinxupquote{ or }}\sphinxstyleliteralemphasis{\sphinxupquote{scalar float}}) \textendash{} Times to perform the conversion at.
If \sphinxtitleref{t} is a single value, it is used for all of the elements of
\sphinxtitleref{R}, \sphinxtitleref{Z}. If the \sphinxtitleref{each\_t} keyword is True, then \sphinxtitleref{t} must be
scalar or have exactly one dimension. If the \sphinxtitleref{each\_t} keyword is
False, \sphinxtitleref{t} must have the same shape as \sphinxtitleref{R} and \sphinxtitleref{Z} (or their
meshgrid if \sphinxtitleref{make\_grid} is True).

\end{itemize}

\item[{Keyword Arguments}] \leavevmode\begin{itemize}
\item {} 
\sphinxstyleliteralstrong{\sphinxupquote{sqrt}} (\sphinxstyleliteralemphasis{\sphinxupquote{Boolean}}) \textendash{} Set to True to return the square root of phinorm.
Only the square root of positive values is taken. Negative
values are replaced with zeros, consistent with Steve Wolfe’s
IDL implementation efit\_rz2rho.pro. Default is False.

\item {} 
\sphinxstyleliteralstrong{\sphinxupquote{each\_t}} (\sphinxstyleliteralemphasis{\sphinxupquote{Boolean}}) \textendash{} When True, the elements in \sphinxtitleref{R}, \sphinxtitleref{Z} are evaluated
at each value in \sphinxtitleref{t}. If True, \sphinxtitleref{t} must have only one dimension
(or be a scalar). If False, \sphinxtitleref{t} must match the shape of \sphinxtitleref{R} and
\sphinxtitleref{Z} or be a scalar. Default is True (evaluate ALL \sphinxtitleref{R}, \sphinxtitleref{Z} at
EACH element in \sphinxtitleref{t}).

\item {} 
\sphinxstyleliteralstrong{\sphinxupquote{make\_grid}} (\sphinxstyleliteralemphasis{\sphinxupquote{Boolean}}) \textendash{} Set to True to pass \sphinxtitleref{R} and \sphinxtitleref{Z} through
\sphinxcode{\sphinxupquote{scipy.meshgrid()}} before evaluating. If this is set to
True, \sphinxtitleref{R} and \sphinxtitleref{Z} must each only have a single dimension, but
can have different lengths. Default is False (do not form
meshgrid).

\item {} 
\sphinxstyleliteralstrong{\sphinxupquote{length\_unit}} (\sphinxstyleliteralemphasis{\sphinxupquote{String}}\sphinxstyleliteralemphasis{\sphinxupquote{ or }}\sphinxstyleliteralemphasis{\sphinxupquote{1}}) \textendash{} 
Length unit that \sphinxtitleref{R}, \sphinxtitleref{Z} are given in.
If a string is given, it must be a valid unit specifier:
\begin{quote}


\begin{savenotes}\sphinxattablestart
\centering
\begin{tabulary}{\linewidth}[t]{|T|T|}
\hline

’m’
&
meters
\\
\hline
’cm’
&
centimeters
\\
\hline
’mm’
&
millimeters
\\
\hline
’in’
&
inches
\\
\hline
’ft’
&
feet
\\
\hline
’yd’
&
yards
\\
\hline
’smoot’
&
smoots
\\
\hline
’cubit’
&
cubits
\\
\hline
’hand’
&
hands
\\
\hline
’default’
&
meters
\\
\hline
\end{tabulary}
\par
\sphinxattableend\end{savenotes}
\end{quote}

If length\_unit is 1 or None, meters are assumed. The default
value is 1 (use meters).


\item {} 
\sphinxstyleliteralstrong{\sphinxupquote{k}} (\sphinxstyleliteralemphasis{\sphinxupquote{positive int}}) \textendash{} The degree of polynomial spline interpolation to
use in converting psinorm to phinorm.

\item {} 
\sphinxstyleliteralstrong{\sphinxupquote{return\_t}} (\sphinxstyleliteralemphasis{\sphinxupquote{Boolean}}) \textendash{} Set to True to return a tuple of (\sphinxtitleref{rho},
\sphinxtitleref{time\_idxs}), where \sphinxtitleref{time\_idxs} is the array of time indices
actually used in evaluating \sphinxtitleref{rho} with nearest-neighbor
interpolation. (This is mostly present as an internal helper.)
Default is False (only return \sphinxtitleref{rho}).

\end{itemize}

\item[{Returns}] \leavevmode

\sphinxtitleref{phinorm} or (\sphinxtitleref{phinorm}, \sphinxtitleref{time\_idxs})
\begin{itemize}
\item {} 
\sphinxstylestrong{phinorm} (\sphinxtitleref{Array or scalar float}) - The normalized toroidal
flux. If all of the input arguments are scalar, then a scalar is
returned. Otherwise, a scipy Array is returned. If \sphinxtitleref{R} and \sphinxtitleref{Z}
both have the same shape then \sphinxtitleref{phinorm} has this shape as well,
unless the \sphinxtitleref{make\_grid} keyword was True, in which case \sphinxtitleref{phinorm}
has shape (len(\sphinxtitleref{Z}), len(\sphinxtitleref{R})).

\item {} 
\sphinxstylestrong{time\_idxs} (Array with same shape as \sphinxtitleref{phinorm}) - The indices
(in \sphinxcode{\sphinxupquote{self.getTimeBase()}}) that were used for
nearest-neighbor interpolation. Only returned if \sphinxtitleref{return\_t} is
True.

\end{itemize}


\end{description}\end{quote}
\subsubsection*{Examples}

All assume that \sphinxtitleref{Eq\_instance} is a valid instance of the appropriate
extension of the {\hyperref[\detokenize{eqtools:eqtools.core.Equilibrium}]{\sphinxcrossref{\sphinxcode{\sphinxupquote{Equilibrium}}}}} abstract class.

Find single phinorm value at R=0.6m, Z=0.0m, t=0.26s:

\begin{sphinxVerbatim}[commandchars=\\\{\}]
\PYG{n}{phi\PYGZus{}val} \PYG{o}{=} \PYG{n}{Eq\PYGZus{}instance}\PYG{o}{.}\PYG{n}{rz2phinorm}\PYG{p}{(}\PYG{l+m+mf}{0.6}\PYG{p}{,} \PYG{l+m+mi}{0}\PYG{p}{,} \PYG{l+m+mf}{0.26}\PYG{p}{)}
\end{sphinxVerbatim}

Find phinorm values at (R, Z) points (0.6m, 0m) and (0.8m, 0m) at the
single time t=0.26s. Note that the \sphinxtitleref{Z} vector must be fully specified,
even if the values are all the same:

\begin{sphinxVerbatim}[commandchars=\\\{\}]
\PYG{n}{phi\PYGZus{}arr} \PYG{o}{=} \PYG{n}{Eq\PYGZus{}instance}\PYG{o}{.}\PYG{n}{rz2phinorm}\PYG{p}{(}\PYG{p}{[}\PYG{l+m+mf}{0.6}\PYG{p}{,} \PYG{l+m+mf}{0.8}\PYG{p}{]}\PYG{p}{,} \PYG{p}{[}\PYG{l+m+mi}{0}\PYG{p}{,} \PYG{l+m+mi}{0}\PYG{p}{]}\PYG{p}{,} \PYG{l+m+mf}{0.26}\PYG{p}{)}
\end{sphinxVerbatim}

Find phinorm values at (R, Z) points (0.6m, 0m) at times t={[}0.2s, 0.3s{]}:

\begin{sphinxVerbatim}[commandchars=\\\{\}]
\PYG{n}{phi\PYGZus{}arr} \PYG{o}{=} \PYG{n}{Eq\PYGZus{}instance}\PYG{o}{.}\PYG{n}{rz2phinorm}\PYG{p}{(}\PYG{l+m+mf}{0.6}\PYG{p}{,} \PYG{l+m+mi}{0}\PYG{p}{,} \PYG{p}{[}\PYG{l+m+mf}{0.2}\PYG{p}{,} \PYG{l+m+mf}{0.3}\PYG{p}{]}\PYG{p}{)}
\end{sphinxVerbatim}

Find phinorm values at (R, Z, t) points (0.6m, 0m, 0.2s) and (0.5m, 0.2m, 0.3s):

\begin{sphinxVerbatim}[commandchars=\\\{\}]
\PYG{n}{phi\PYGZus{}arr} \PYG{o}{=} \PYG{n}{Eq\PYGZus{}instance}\PYG{o}{.}\PYG{n}{rz2phinorm}\PYG{p}{(}\PYG{p}{[}\PYG{l+m+mf}{0.6}\PYG{p}{,} \PYG{l+m+mf}{0.5}\PYG{p}{]}\PYG{p}{,} \PYG{p}{[}\PYG{l+m+mi}{0}\PYG{p}{,} \PYG{l+m+mf}{0.2}\PYG{p}{]}\PYG{p}{,} \PYG{p}{[}\PYG{l+m+mf}{0.2}\PYG{p}{,} \PYG{l+m+mf}{0.3}\PYG{p}{]}\PYG{p}{,} \PYG{n}{each\PYGZus{}t}\PYG{o}{=}\PYG{k+kc}{False}\PYG{p}{)}
\end{sphinxVerbatim}

Find phinorm values on grid defined by 1D vector of radial positions \sphinxtitleref{R}
and 1D vector of vertical positions \sphinxtitleref{Z} at time t=0.2s:

\begin{sphinxVerbatim}[commandchars=\\\{\}]
\PYG{n}{phi\PYGZus{}mat} \PYG{o}{=} \PYG{n}{Eq\PYGZus{}instance}\PYG{o}{.}\PYG{n}{rz2phinorm}\PYG{p}{(}\PYG{n}{R}\PYG{p}{,} \PYG{n}{Z}\PYG{p}{,} \PYG{l+m+mf}{0.2}\PYG{p}{,} \PYG{n}{make\PYGZus{}grid}\PYG{o}{=}\PYG{k+kc}{True}\PYG{p}{)}
\end{sphinxVerbatim}

\end{fulllineitems}

\index{rz2volnorm() (eqtools.core.Equilibrium method)@\spxentry{rz2volnorm()}\spxextra{eqtools.core.Equilibrium method}}

\begin{fulllineitems}
\phantomsection\label{\detokenize{eqtools:eqtools.core.Equilibrium.rz2volnorm}}\pysiglinewithargsret{\sphinxbfcode{\sphinxupquote{rz2volnorm}}}{\emph{*args}, \emph{**kwargs}}{}
Calculates the normalized flux surface volume.

Based on the IDL version efit\_rz2rho.pro by Steve Wolfe.
\begin{quote}\begin{description}
\item[{Parameters}] \leavevmode\begin{itemize}
\item {} 
\sphinxstyleliteralstrong{\sphinxupquote{R}} (\sphinxstyleliteralemphasis{\sphinxupquote{Array-like}}\sphinxstyleliteralemphasis{\sphinxupquote{ or }}\sphinxstyleliteralemphasis{\sphinxupquote{scalar float}}) \textendash{} Values of the radial coordinate to
map to volnorm. If \sphinxtitleref{R} and \sphinxtitleref{Z} are both scalar values,
they are used as the coordinate pair for all of the values in
\sphinxtitleref{t}. Must have the same shape as \sphinxtitleref{Z} unless the \sphinxtitleref{make\_grid}
keyword is set. If the \sphinxtitleref{make\_grid} keyword is True, \sphinxtitleref{R} must
have exactly one dimension.

\item {} 
\sphinxstyleliteralstrong{\sphinxupquote{Z}} (\sphinxstyleliteralemphasis{\sphinxupquote{Array-like}}\sphinxstyleliteralemphasis{\sphinxupquote{ or }}\sphinxstyleliteralemphasis{\sphinxupquote{scalar float}}) \textendash{} Values of the vertical coordinate to
map to volnorm. If \sphinxtitleref{R} and \sphinxtitleref{Z} are both scalar values,
they are used as the coordinate pair for all of the values in
\sphinxtitleref{t}. Must have the same shape as \sphinxtitleref{R} unless the \sphinxtitleref{make\_grid}
keyword is set. If the \sphinxtitleref{make\_grid} keyword is True, \sphinxtitleref{Z} must
have exactly one dimension.

\item {} 
\sphinxstyleliteralstrong{\sphinxupquote{t}} (\sphinxstyleliteralemphasis{\sphinxupquote{Array-like}}\sphinxstyleliteralemphasis{\sphinxupquote{ or }}\sphinxstyleliteralemphasis{\sphinxupquote{scalar float}}) \textendash{} Times to perform the conversion at.
If \sphinxtitleref{t} is a single value, it is used for all of the elements of
\sphinxtitleref{R}, \sphinxtitleref{Z}. If the \sphinxtitleref{each\_t} keyword is True, then \sphinxtitleref{t} must be
scalar or have exactly one dimension. If the \sphinxtitleref{each\_t} keyword is
False, \sphinxtitleref{t} must have the same shape as \sphinxtitleref{R} and \sphinxtitleref{Z} (or their
meshgrid if \sphinxtitleref{make\_grid} is True).

\end{itemize}

\item[{Keyword Arguments}] \leavevmode\begin{itemize}
\item {} 
\sphinxstyleliteralstrong{\sphinxupquote{sqrt}} (\sphinxstyleliteralemphasis{\sphinxupquote{Boolean}}) \textendash{} Set to True to return the square root of volnorm.
Only the square root of positive values is taken. Negative
values are replaced with zeros, consistent with Steve Wolfe’s
IDL implementation efit\_rz2rho.pro. Default is False.

\item {} 
\sphinxstyleliteralstrong{\sphinxupquote{each\_t}} (\sphinxstyleliteralemphasis{\sphinxupquote{Boolean}}) \textendash{} When True, the elements in \sphinxtitleref{R}, \sphinxtitleref{Z} are evaluated
at each value in \sphinxtitleref{t}. If True, \sphinxtitleref{t} must have only one dimension
(or be a scalar). If False, \sphinxtitleref{t} must match the shape of \sphinxtitleref{R} and
\sphinxtitleref{Z} or be a scalar. Default is True (evaluate ALL \sphinxtitleref{R}, \sphinxtitleref{Z} at
EACH element in \sphinxtitleref{t}).

\item {} 
\sphinxstyleliteralstrong{\sphinxupquote{make\_grid}} (\sphinxstyleliteralemphasis{\sphinxupquote{Boolean}}) \textendash{} Set to True to pass \sphinxtitleref{R} and \sphinxtitleref{Z} through
\sphinxcode{\sphinxupquote{scipy.meshgrid()}} before evaluating. If this is set to
True, \sphinxtitleref{R} and \sphinxtitleref{Z} must each only have a single dimension, but
can have different lengths. Default is False (do not form
meshgrid).

\item {} 
\sphinxstyleliteralstrong{\sphinxupquote{length\_unit}} (\sphinxstyleliteralemphasis{\sphinxupquote{String}}\sphinxstyleliteralemphasis{\sphinxupquote{ or }}\sphinxstyleliteralemphasis{\sphinxupquote{1}}) \textendash{} 
Length unit that \sphinxtitleref{R}, \sphinxtitleref{Z} are given in.
If a string is given, it must be a valid unit specifier:
\begin{quote}


\begin{savenotes}\sphinxattablestart
\centering
\begin{tabulary}{\linewidth}[t]{|T|T|}
\hline

’m’
&
meters
\\
\hline
’cm’
&
centimeters
\\
\hline
’mm’
&
millimeters
\\
\hline
’in’
&
inches
\\
\hline
’ft’
&
feet
\\
\hline
’yd’
&
yards
\\
\hline
’smoot’
&
smoots
\\
\hline
’cubit’
&
cubits
\\
\hline
’hand’
&
hands
\\
\hline
’default’
&
meters
\\
\hline
\end{tabulary}
\par
\sphinxattableend\end{savenotes}
\end{quote}

If length\_unit is 1 or None, meters are assumed. The default
value is 1 (use meters).


\item {} 
\sphinxstyleliteralstrong{\sphinxupquote{k}} (\sphinxstyleliteralemphasis{\sphinxupquote{positive int}}) \textendash{} The degree of polynomial spline interpolation to
use in converting psinorm to volnorm.

\item {} 
\sphinxstyleliteralstrong{\sphinxupquote{return\_t}} (\sphinxstyleliteralemphasis{\sphinxupquote{Boolean}}) \textendash{} Set to True to return a tuple of (\sphinxtitleref{rho},
\sphinxtitleref{time\_idxs}), where \sphinxtitleref{time\_idxs} is the array of time indices
actually used in evaluating \sphinxtitleref{rho} with nearest-neighbor
interpolation. (This is mostly present as an internal helper.)
Default is False (only return \sphinxtitleref{rho}).

\end{itemize}

\item[{Returns}] \leavevmode

\sphinxtitleref{volnorm} or (\sphinxtitleref{volnorm}, \sphinxtitleref{time\_idxs})
\begin{itemize}
\item {} 
\sphinxstylestrong{volnorm} (\sphinxtitleref{Array or scalar float}) - The normalized volume.
If all of the input arguments are scalar, then a scalar is
returned. Otherwise, a scipy Array is returned. If \sphinxtitleref{R} and \sphinxtitleref{Z}
both have the same shape then \sphinxtitleref{volnorm} has this shape as well,
unless the \sphinxtitleref{make\_grid} keyword was True, in which case \sphinxtitleref{volnorm}
has shape (len(\sphinxtitleref{Z}), len(\sphinxtitleref{R})).

\item {} 
\sphinxstylestrong{time\_idxs} (Array with same shape as \sphinxtitleref{volnorm}) - The indices
(in \sphinxcode{\sphinxupquote{self.getTimeBase()}}) that were used for
nearest-neighbor interpolation. Only returned if \sphinxtitleref{return\_t} is
True.

\end{itemize}


\end{description}\end{quote}
\subsubsection*{Examples}

All assume that \sphinxtitleref{Eq\_instance} is a valid instance of the appropriate
extension of the {\hyperref[\detokenize{eqtools:eqtools.core.Equilibrium}]{\sphinxcrossref{\sphinxcode{\sphinxupquote{Equilibrium}}}}} abstract class.

Find single volnorm value at R=0.6m, Z=0.0m, t=0.26s:

\begin{sphinxVerbatim}[commandchars=\\\{\}]
\PYG{n}{psi\PYGZus{}val} \PYG{o}{=} \PYG{n}{Eq\PYGZus{}instance}\PYG{o}{.}\PYG{n}{rz2volnorm}\PYG{p}{(}\PYG{l+m+mf}{0.6}\PYG{p}{,} \PYG{l+m+mi}{0}\PYG{p}{,} \PYG{l+m+mf}{0.26}\PYG{p}{)}
\end{sphinxVerbatim}

Find volnorm values at (R, Z) points (0.6m, 0m) and (0.8m, 0m) at the
single time t=0.26s. Note that the \sphinxtitleref{Z} vector must be fully specified,
even if the values are all the same:

\begin{sphinxVerbatim}[commandchars=\\\{\}]
\PYG{n}{vol\PYGZus{}arr} \PYG{o}{=} \PYG{n}{Eq\PYGZus{}instance}\PYG{o}{.}\PYG{n}{rz2volnorm}\PYG{p}{(}\PYG{p}{[}\PYG{l+m+mf}{0.6}\PYG{p}{,} \PYG{l+m+mf}{0.8}\PYG{p}{]}\PYG{p}{,} \PYG{p}{[}\PYG{l+m+mi}{0}\PYG{p}{,} \PYG{l+m+mi}{0}\PYG{p}{]}\PYG{p}{,} \PYG{l+m+mf}{0.26}\PYG{p}{)}
\end{sphinxVerbatim}

Find volnorm values at (R, Z) points (0.6m, 0m) at times t={[}0.2s, 0.3s{]}:

\begin{sphinxVerbatim}[commandchars=\\\{\}]
\PYG{n}{vol\PYGZus{}arr} \PYG{o}{=} \PYG{n}{Eq\PYGZus{}instance}\PYG{o}{.}\PYG{n}{rz2volnorm}\PYG{p}{(}\PYG{l+m+mf}{0.6}\PYG{p}{,} \PYG{l+m+mi}{0}\PYG{p}{,} \PYG{p}{[}\PYG{l+m+mf}{0.2}\PYG{p}{,} \PYG{l+m+mf}{0.3}\PYG{p}{]}\PYG{p}{)}
\end{sphinxVerbatim}

Find volnorm values at (R, Z, t) points (0.6m, 0m, 0.2s) and (0.5m, 0.2m, 0.3s):

\begin{sphinxVerbatim}[commandchars=\\\{\}]
\PYG{n}{vol\PYGZus{}arr} \PYG{o}{=} \PYG{n}{Eq\PYGZus{}instance}\PYG{o}{.}\PYG{n}{rz2volnorm}\PYG{p}{(}\PYG{p}{[}\PYG{l+m+mf}{0.6}\PYG{p}{,} \PYG{l+m+mf}{0.5}\PYG{p}{]}\PYG{p}{,} \PYG{p}{[}\PYG{l+m+mi}{0}\PYG{p}{,} \PYG{l+m+mf}{0.2}\PYG{p}{]}\PYG{p}{,} \PYG{p}{[}\PYG{l+m+mf}{0.2}\PYG{p}{,} \PYG{l+m+mf}{0.3}\PYG{p}{]}\PYG{p}{,} \PYG{n}{each\PYGZus{}t}\PYG{o}{=}\PYG{k+kc}{False}\PYG{p}{)}
\end{sphinxVerbatim}

Find volnorm values on grid defined by 1D vector of radial positions \sphinxtitleref{R}
and 1D vector of vertical positions \sphinxtitleref{Z} at time t=0.2s:

\begin{sphinxVerbatim}[commandchars=\\\{\}]
\PYG{n}{vol\PYGZus{}mat} \PYG{o}{=} \PYG{n}{Eq\PYGZus{}instance}\PYG{o}{.}\PYG{n}{rz2volnorm}\PYG{p}{(}\PYG{n}{R}\PYG{p}{,} \PYG{n}{Z}\PYG{p}{,} \PYG{l+m+mf}{0.2}\PYG{p}{,} \PYG{n}{make\PYGZus{}grid}\PYG{o}{=}\PYG{k+kc}{True}\PYG{p}{)}
\end{sphinxVerbatim}

\end{fulllineitems}

\index{rz2rmid() (eqtools.core.Equilibrium method)@\spxentry{rz2rmid()}\spxextra{eqtools.core.Equilibrium method}}

\begin{fulllineitems}
\phantomsection\label{\detokenize{eqtools:eqtools.core.Equilibrium.rz2rmid}}\pysiglinewithargsret{\sphinxbfcode{\sphinxupquote{rz2rmid}}}{\emph{*args}, \emph{**kwargs}}{}
Maps the given points to the outboard midplane major radius, Rmid.

Based on the IDL version efit\_rz2rmid.pro by Steve Wolfe.
\begin{quote}\begin{description}
\item[{Parameters}] \leavevmode\begin{itemize}
\item {} 
\sphinxstyleliteralstrong{\sphinxupquote{R}} (\sphinxstyleliteralemphasis{\sphinxupquote{Array-like}}\sphinxstyleliteralemphasis{\sphinxupquote{ or }}\sphinxstyleliteralemphasis{\sphinxupquote{scalar float}}) \textendash{} Values of the radial coordinate to
map to Rmid. If \sphinxtitleref{R} and \sphinxtitleref{Z} are both scalar values,
they are used as the coordinate pair for all of the values in
\sphinxtitleref{t}. Must have the same shape as \sphinxtitleref{Z} unless the \sphinxtitleref{make\_grid}
keyword is set. If the \sphinxtitleref{make\_grid} keyword is True, \sphinxtitleref{R} must
have exactly one dimension.

\item {} 
\sphinxstyleliteralstrong{\sphinxupquote{Z}} (\sphinxstyleliteralemphasis{\sphinxupquote{Array-like}}\sphinxstyleliteralemphasis{\sphinxupquote{ or }}\sphinxstyleliteralemphasis{\sphinxupquote{scalar float}}) \textendash{} Values of the vertical coordinate to
map to Rmid. If \sphinxtitleref{R} and \sphinxtitleref{Z} are both scalar values,
they are used as the coordinate pair for all of the values in
\sphinxtitleref{t}. Must have the same shape as \sphinxtitleref{R} unless the \sphinxtitleref{make\_grid}
keyword is set. If the \sphinxtitleref{make\_grid} keyword is True, \sphinxtitleref{Z} must
have exactly one dimension.

\item {} 
\sphinxstyleliteralstrong{\sphinxupquote{t}} (\sphinxstyleliteralemphasis{\sphinxupquote{Array-like}}\sphinxstyleliteralemphasis{\sphinxupquote{ or }}\sphinxstyleliteralemphasis{\sphinxupquote{scalar float}}) \textendash{} Times to perform the conversion at.
If \sphinxtitleref{t} is a single value, it is used for all of the elements of
\sphinxtitleref{R}, \sphinxtitleref{Z}. If the \sphinxtitleref{each\_t} keyword is True, then \sphinxtitleref{t} must be
scalar or have exactly one dimension. If the \sphinxtitleref{each\_t} keyword is
False, \sphinxtitleref{t} must have the same shape as \sphinxtitleref{R} and \sphinxtitleref{Z} (or their
meshgrid if \sphinxtitleref{make\_grid} is True).

\end{itemize}

\item[{Keyword Arguments}] \leavevmode\begin{itemize}
\item {} 
\sphinxstyleliteralstrong{\sphinxupquote{sqrt}} (\sphinxstyleliteralemphasis{\sphinxupquote{Boolean}}) \textendash{} Set to True to return the square root of Rmid.
Only the square root of positive values is taken. Negative
values are replaced with zeros, consistent with Steve Wolfe’s
IDL implementation efit\_rz2rho.pro. Default is False.

\item {} 
\sphinxstyleliteralstrong{\sphinxupquote{each\_t}} (\sphinxstyleliteralemphasis{\sphinxupquote{Boolean}}) \textendash{} When True, the elements in \sphinxtitleref{R}, \sphinxtitleref{Z} are evaluated
at each value in \sphinxtitleref{t}. If True, \sphinxtitleref{t} must have only one dimension
(or be a scalar). If False, \sphinxtitleref{t} must match the shape of \sphinxtitleref{R} and
\sphinxtitleref{Z} or be a scalar. Default is True (evaluate ALL \sphinxtitleref{R}, \sphinxtitleref{Z} at
EACH element in \sphinxtitleref{t}).

\item {} 
\sphinxstyleliteralstrong{\sphinxupquote{make\_grid}} (\sphinxstyleliteralemphasis{\sphinxupquote{Boolean}}) \textendash{} Set to True to pass \sphinxtitleref{R} and \sphinxtitleref{Z} through
\sphinxcode{\sphinxupquote{scipy.meshgrid()}} before evaluating. If this is set to
True, \sphinxtitleref{R} and \sphinxtitleref{Z} must each only have a single dimension, but
can have different lengths. Default is False (do not form
meshgrid).

\item {} 
\sphinxstyleliteralstrong{\sphinxupquote{rho}} (\sphinxstyleliteralemphasis{\sphinxupquote{Boolean}}) \textendash{} Set to True to return r/a (normalized minor radius)
instead of Rmid. Default is False (return major radius, Rmid).

\item {} 
\sphinxstyleliteralstrong{\sphinxupquote{length\_unit}} (\sphinxstyleliteralemphasis{\sphinxupquote{String}}\sphinxstyleliteralemphasis{\sphinxupquote{ or }}\sphinxstyleliteralemphasis{\sphinxupquote{1}}) \textendash{} 
Length unit that \sphinxtitleref{R}, \sphinxtitleref{Z} are given in,
AND that \sphinxtitleref{Rmid} is returned in. If a string is given, it must
be a valid unit specifier:
\begin{quote}


\begin{savenotes}\sphinxattablestart
\centering
\begin{tabulary}{\linewidth}[t]{|T|T|}
\hline

’m’
&
meters
\\
\hline
’cm’
&
centimeters
\\
\hline
’mm’
&
millimeters
\\
\hline
’in’
&
inches
\\
\hline
’ft’
&
feet
\\
\hline
’yd’
&
yards
\\
\hline
’smoot’
&
smoots
\\
\hline
’cubit’
&
cubits
\\
\hline
’hand’
&
hands
\\
\hline
’default’
&
meters
\\
\hline
\end{tabulary}
\par
\sphinxattableend\end{savenotes}
\end{quote}

If length\_unit is 1 or None, meters are assumed. The default
value is 1 (use meters).


\item {} 
\sphinxstyleliteralstrong{\sphinxupquote{k}} (\sphinxstyleliteralemphasis{\sphinxupquote{positive int}}) \textendash{} The degree of polynomial spline interpolation to
use in converting psinorm to Rmid.

\item {} 
\sphinxstyleliteralstrong{\sphinxupquote{return\_t}} (\sphinxstyleliteralemphasis{\sphinxupquote{Boolean}}) \textendash{} Set to True to return a tuple of (\sphinxtitleref{rho},
\sphinxtitleref{time\_idxs}), where \sphinxtitleref{time\_idxs} is the array of time indices
actually used in evaluating \sphinxtitleref{rho} with nearest-neighbor
interpolation. (This is mostly present as an internal helper.)
Default is False (only return \sphinxtitleref{rho}).

\end{itemize}

\item[{Returns}] \leavevmode

\sphinxtitleref{Rmid} or (\sphinxtitleref{Rmid}, \sphinxtitleref{time\_idxs})
\begin{itemize}
\item {} 
\sphinxstylestrong{Rmid} (\sphinxtitleref{Array or scalar float}) - The outboard midplan major
radius. If all of the input arguments are scalar, then a scalar
is returned. Otherwise, a scipy Array is returned. If \sphinxtitleref{R} and \sphinxtitleref{Z}
both have the same shape then \sphinxtitleref{Rmid} has this shape as well,
unless the \sphinxtitleref{make\_grid} keyword was True, in which case \sphinxtitleref{Rmid}
has shape (len(\sphinxtitleref{Z}), len(\sphinxtitleref{R})).

\item {} 
\sphinxstylestrong{time\_idxs} (Array with same shape as \sphinxtitleref{Rmid}) - The indices
(in \sphinxcode{\sphinxupquote{self.getTimeBase()}}) that were used for
nearest-neighbor interpolation. Only returned if \sphinxtitleref{return\_t} is
True.

\end{itemize}


\end{description}\end{quote}
\subsubsection*{Examples}

All assume that \sphinxtitleref{Eq\_instance} is a valid instance of the appropriate
extension of the {\hyperref[\detokenize{eqtools:eqtools.core.Equilibrium}]{\sphinxcrossref{\sphinxcode{\sphinxupquote{Equilibrium}}}}} abstract class.

Find single Rmid value at R=0.6m, Z=0.0m, t=0.26s:

\begin{sphinxVerbatim}[commandchars=\\\{\}]
\PYG{n}{R\PYGZus{}mid\PYGZus{}val} \PYG{o}{=} \PYG{n}{Eq\PYGZus{}instance}\PYG{o}{.}\PYG{n}{rz2rmid}\PYG{p}{(}\PYG{l+m+mf}{0.6}\PYG{p}{,} \PYG{l+m+mi}{0}\PYG{p}{,} \PYG{l+m+mf}{0.26}\PYG{p}{)}
\end{sphinxVerbatim}

Find R\_mid values at (R, Z) points (0.6m, 0m) and (0.8m, 0m) at the
single time t=0.26s. Note that the \sphinxtitleref{Z} vector must be fully specified,
even if the values are all the same:

\begin{sphinxVerbatim}[commandchars=\\\{\}]
\PYG{n}{R\PYGZus{}mid\PYGZus{}arr} \PYG{o}{=} \PYG{n}{Eq\PYGZus{}instance}\PYG{o}{.}\PYG{n}{rz2rmid}\PYG{p}{(}\PYG{p}{[}\PYG{l+m+mf}{0.6}\PYG{p}{,} \PYG{l+m+mf}{0.8}\PYG{p}{]}\PYG{p}{,} \PYG{p}{[}\PYG{l+m+mi}{0}\PYG{p}{,} \PYG{l+m+mi}{0}\PYG{p}{]}\PYG{p}{,} \PYG{l+m+mf}{0.26}\PYG{p}{)}
\end{sphinxVerbatim}

Find Rmid values at (R, Z) points (0.6m, 0m) at times t={[}0.2s, 0.3s{]}:

\begin{sphinxVerbatim}[commandchars=\\\{\}]
\PYG{n}{R\PYGZus{}mid\PYGZus{}arr} \PYG{o}{=} \PYG{n}{Eq\PYGZus{}instance}\PYG{o}{.}\PYG{n}{rz2rmid}\PYG{p}{(}\PYG{l+m+mf}{0.6}\PYG{p}{,} \PYG{l+m+mi}{0}\PYG{p}{,} \PYG{p}{[}\PYG{l+m+mf}{0.2}\PYG{p}{,} \PYG{l+m+mf}{0.3}\PYG{p}{]}\PYG{p}{)}
\end{sphinxVerbatim}

Find Rmid values at (R, Z, t) points (0.6m, 0m, 0.2s) and (0.5m, 0.2m, 0.3s):

\begin{sphinxVerbatim}[commandchars=\\\{\}]
\PYG{n}{R\PYGZus{}mid\PYGZus{}arr} \PYG{o}{=} \PYG{n}{Eq\PYGZus{}instance}\PYG{o}{.}\PYG{n}{rz2rmid}\PYG{p}{(}\PYG{p}{[}\PYG{l+m+mf}{0.6}\PYG{p}{,} \PYG{l+m+mf}{0.5}\PYG{p}{]}\PYG{p}{,} \PYG{p}{[}\PYG{l+m+mi}{0}\PYG{p}{,} \PYG{l+m+mf}{0.2}\PYG{p}{]}\PYG{p}{,} \PYG{p}{[}\PYG{l+m+mf}{0.2}\PYG{p}{,} \PYG{l+m+mf}{0.3}\PYG{p}{]}\PYG{p}{,} \PYG{n}{each\PYGZus{}t}\PYG{o}{=}\PYG{k+kc}{False}\PYG{p}{)}
\end{sphinxVerbatim}

Find Rmid values on grid defined by 1D vector of radial positions \sphinxtitleref{R}
and 1D vector of vertical positions \sphinxtitleref{Z} at time t=0.2s:

\begin{sphinxVerbatim}[commandchars=\\\{\}]
\PYG{n}{R\PYGZus{}mid\PYGZus{}mat} \PYG{o}{=} \PYG{n}{Eq\PYGZus{}instance}\PYG{o}{.}\PYG{n}{rz2rmid}\PYG{p}{(}\PYG{n}{R}\PYG{p}{,} \PYG{n}{Z}\PYG{p}{,} \PYG{l+m+mf}{0.2}\PYG{p}{,} \PYG{n}{make\PYGZus{}grid}\PYG{o}{=}\PYG{k+kc}{True}\PYG{p}{)}
\end{sphinxVerbatim}

\end{fulllineitems}

\index{rz2roa() (eqtools.core.Equilibrium method)@\spxentry{rz2roa()}\spxextra{eqtools.core.Equilibrium method}}

\begin{fulllineitems}
\phantomsection\label{\detokenize{eqtools:eqtools.core.Equilibrium.rz2roa}}\pysiglinewithargsret{\sphinxbfcode{\sphinxupquote{rz2roa}}}{\emph{*args}, \emph{**kwargs}}{}
Maps the given points to the normalized minor radius, r/a.

Based on the IDL version efit\_rz2rmid.pro by Steve Wolfe.
\begin{quote}\begin{description}
\item[{Parameters}] \leavevmode\begin{itemize}
\item {} 
\sphinxstyleliteralstrong{\sphinxupquote{R}} (\sphinxstyleliteralemphasis{\sphinxupquote{Array-like}}\sphinxstyleliteralemphasis{\sphinxupquote{ or }}\sphinxstyleliteralemphasis{\sphinxupquote{scalar float}}) \textendash{} Values of the radial coordinate to
map to r/a. If \sphinxtitleref{R} and \sphinxtitleref{Z} are both scalar values,
they are used as the coordinate pair for all of the values in
\sphinxtitleref{t}. Must have the same shape as \sphinxtitleref{Z} unless the \sphinxtitleref{make\_grid}
keyword is set. If the \sphinxtitleref{make\_grid} keyword is True, \sphinxtitleref{R} must
have exactly one dimension.

\item {} 
\sphinxstyleliteralstrong{\sphinxupquote{Z}} (\sphinxstyleliteralemphasis{\sphinxupquote{Array-like}}\sphinxstyleliteralemphasis{\sphinxupquote{ or }}\sphinxstyleliteralemphasis{\sphinxupquote{scalar float}}) \textendash{} Values of the vertical coordinate to
map to r/a. If \sphinxtitleref{R} and \sphinxtitleref{Z} are both scalar values,
they are used as the coordinate pair for all of the values in
\sphinxtitleref{t}. Must have the same shape as \sphinxtitleref{R} unless the \sphinxtitleref{make\_grid}
keyword is set. If the \sphinxtitleref{make\_grid} keyword is True, \sphinxtitleref{Z} must
have exactly one dimension.

\item {} 
\sphinxstyleliteralstrong{\sphinxupquote{t}} (\sphinxstyleliteralemphasis{\sphinxupquote{Array-like}}\sphinxstyleliteralemphasis{\sphinxupquote{ or }}\sphinxstyleliteralemphasis{\sphinxupquote{scalar float}}) \textendash{} Times to perform the conversion at.
If \sphinxtitleref{t} is a single value, it is used for all of the elements of
\sphinxtitleref{R}, \sphinxtitleref{Z}. If the \sphinxtitleref{each\_t} keyword is True, then \sphinxtitleref{t} must be
scalar or have exactly one dimension. If the \sphinxtitleref{each\_t} keyword is
False, \sphinxtitleref{t} must have the same shape as \sphinxtitleref{R} and \sphinxtitleref{Z} (or their
meshgrid if \sphinxtitleref{make\_grid} is True).

\end{itemize}

\item[{Keyword Arguments}] \leavevmode\begin{itemize}
\item {} 
\sphinxstyleliteralstrong{\sphinxupquote{sqrt}} (\sphinxstyleliteralemphasis{\sphinxupquote{Boolean}}) \textendash{} Set to True to return the square root of r/a.
Only the square root of positive values is taken. Negative
values are replaced with zeros, consistent with Steve Wolfe’s
IDL implementation efit\_rz2rho.pro. Default is False.

\item {} 
\sphinxstyleliteralstrong{\sphinxupquote{each\_t}} (\sphinxstyleliteralemphasis{\sphinxupquote{Boolean}}) \textendash{} When True, the elements in \sphinxtitleref{R}, \sphinxtitleref{Z} are evaluated
at each value in \sphinxtitleref{t}. If True, \sphinxtitleref{t} must have only one dimension
(or be a scalar). If False, \sphinxtitleref{t} must match the shape of \sphinxtitleref{R} and
\sphinxtitleref{Z} or be a scalar. Default is True (evaluate ALL \sphinxtitleref{R}, \sphinxtitleref{Z} at
EACH element in \sphinxtitleref{t}).

\item {} 
\sphinxstyleliteralstrong{\sphinxupquote{make\_grid}} (\sphinxstyleliteralemphasis{\sphinxupquote{Boolean}}) \textendash{} Set to True to pass \sphinxtitleref{R} and \sphinxtitleref{Z} through
\sphinxcode{\sphinxupquote{scipy.meshgrid()}} before evaluating. If this is set to
True, \sphinxtitleref{R} and \sphinxtitleref{Z} must each only have a single dimension, but
can have different lengths. Default is False (do not form
meshgrid).

\item {} 
\sphinxstyleliteralstrong{\sphinxupquote{length\_unit}} (\sphinxstyleliteralemphasis{\sphinxupquote{String}}\sphinxstyleliteralemphasis{\sphinxupquote{ or }}\sphinxstyleliteralemphasis{\sphinxupquote{1}}) \textendash{} 
Length unit that \sphinxtitleref{R}, \sphinxtitleref{Z} are given in.
If a string is given, it must be a valid unit specifier:
\begin{quote}


\begin{savenotes}\sphinxattablestart
\centering
\begin{tabulary}{\linewidth}[t]{|T|T|}
\hline

’m’
&
meters
\\
\hline
’cm’
&
centimeters
\\
\hline
’mm’
&
millimeters
\\
\hline
’in’
&
inches
\\
\hline
’ft’
&
feet
\\
\hline
’yd’
&
yards
\\
\hline
’smoot’
&
smoots
\\
\hline
’cubit’
&
cubits
\\
\hline
’hand’
&
hands
\\
\hline
’default’
&
meters
\\
\hline
\end{tabulary}
\par
\sphinxattableend\end{savenotes}
\end{quote}

If length\_unit is 1 or None, meters are assumed. The default
value is 1 (use meters).


\item {} 
\sphinxstyleliteralstrong{\sphinxupquote{k}} (\sphinxstyleliteralemphasis{\sphinxupquote{positive int}}) \textendash{} The degree of polynomial spline interpolation to
use in converting psinorm to Rmid.

\item {} 
\sphinxstyleliteralstrong{\sphinxupquote{return\_t}} (\sphinxstyleliteralemphasis{\sphinxupquote{Boolean}}) \textendash{} Set to True to return a tuple of (\sphinxtitleref{rho},
\sphinxtitleref{time\_idxs}), where \sphinxtitleref{time\_idxs} is the array of time indices
actually used in evaluating \sphinxtitleref{rho} with nearest-neighbor
interpolation. (This is mostly present as an internal helper.)
Default is False (only return \sphinxtitleref{rho}).

\end{itemize}

\item[{Returns}] \leavevmode

\sphinxtitleref{roa} or (\sphinxtitleref{roa}, \sphinxtitleref{time\_idxs})
\begin{itemize}
\item {} 
\sphinxstylestrong{roa} (\sphinxtitleref{Array or scalar float}) - The normalized minor radius.
If all of the input arguments are scalar, then a scalar
is returned. Otherwise, a scipy Array is returned. If \sphinxtitleref{R} and \sphinxtitleref{Z}
both have the same shape then \sphinxtitleref{roa} has this shape as well,
unless the \sphinxtitleref{make\_grid} keyword was True, in which case \sphinxtitleref{roa}
has shape (len(\sphinxtitleref{Z}), len(\sphinxtitleref{R})).

\item {} 
\sphinxstylestrong{time\_idxs} (Array with same shape as \sphinxtitleref{roa}) - The indices
(in \sphinxcode{\sphinxupquote{self.getTimeBase()}}) that were used for
nearest-neighbor interpolation. Only returned if \sphinxtitleref{return\_t} is
True.

\end{itemize}


\end{description}\end{quote}
\subsubsection*{Examples}

All assume that \sphinxtitleref{Eq\_instance} is a valid instance of the appropriate
extension of the {\hyperref[\detokenize{eqtools:eqtools.core.Equilibrium}]{\sphinxcrossref{\sphinxcode{\sphinxupquote{Equilibrium}}}}} abstract class.

Find single r/a value at R=0.6m, Z=0.0m, t=0.26s:

\begin{sphinxVerbatim}[commandchars=\\\{\}]
\PYG{n}{roa\PYGZus{}val} \PYG{o}{=} \PYG{n}{Eq\PYGZus{}instance}\PYG{o}{.}\PYG{n}{rz2roa}\PYG{p}{(}\PYG{l+m+mf}{0.6}\PYG{p}{,} \PYG{l+m+mi}{0}\PYG{p}{,} \PYG{l+m+mf}{0.26}\PYG{p}{)}
\end{sphinxVerbatim}

Find r/a values at (R, Z) points (0.6m, 0m) and (0.8m, 0m) at the
single time t=0.26s. Note that the Z vector must be fully specified,
even if the values are all the same:

\begin{sphinxVerbatim}[commandchars=\\\{\}]
\PYG{n}{roa\PYGZus{}arr} \PYG{o}{=} \PYG{n}{Eq\PYGZus{}instance}\PYG{o}{.}\PYG{n}{rz2roa}\PYG{p}{(}\PYG{p}{[}\PYG{l+m+mf}{0.6}\PYG{p}{,} \PYG{l+m+mf}{0.8}\PYG{p}{]}\PYG{p}{,} \PYG{p}{[}\PYG{l+m+mi}{0}\PYG{p}{,} \PYG{l+m+mi}{0}\PYG{p}{]}\PYG{p}{,} \PYG{l+m+mf}{0.26}\PYG{p}{)}
\end{sphinxVerbatim}

Find r/a values at (R, Z) points (0.6m, 0m) at times t={[}0.2s, 0.3s{]}:

\begin{sphinxVerbatim}[commandchars=\\\{\}]
\PYG{n}{roa\PYGZus{}arr} \PYG{o}{=} \PYG{n}{Eq\PYGZus{}instance}\PYG{o}{.}\PYG{n}{rz2roa}\PYG{p}{(}\PYG{l+m+mf}{0.6}\PYG{p}{,} \PYG{l+m+mi}{0}\PYG{p}{,} \PYG{p}{[}\PYG{l+m+mf}{0.2}\PYG{p}{,} \PYG{l+m+mf}{0.3}\PYG{p}{]}\PYG{p}{)}
\end{sphinxVerbatim}

Find r/a values at (R, Z, t) points (0.6m, 0m, 0.2s) and (0.5m, 0.2m, 0.3s):

\begin{sphinxVerbatim}[commandchars=\\\{\}]
\PYG{n}{roa\PYGZus{}arr} \PYG{o}{=} \PYG{n}{Eq\PYGZus{}instance}\PYG{o}{.}\PYG{n}{rz2roa}\PYG{p}{(}\PYG{p}{[}\PYG{l+m+mf}{0.6}\PYG{p}{,} \PYG{l+m+mf}{0.5}\PYG{p}{]}\PYG{p}{,} \PYG{p}{[}\PYG{l+m+mi}{0}\PYG{p}{,} \PYG{l+m+mf}{0.2}\PYG{p}{]}\PYG{p}{,} \PYG{p}{[}\PYG{l+m+mf}{0.2}\PYG{p}{,} \PYG{l+m+mf}{0.3}\PYG{p}{]}\PYG{p}{,} \PYG{n}{each\PYGZus{}t}\PYG{o}{=}\PYG{k+kc}{False}\PYG{p}{)}
\end{sphinxVerbatim}

Find r/a values on grid defined by 1D vector of radial positions \sphinxtitleref{R}
and 1D vector of vertical positions \sphinxtitleref{Z} at time t=0.2s:

\begin{sphinxVerbatim}[commandchars=\\\{\}]
\PYG{n}{roa\PYGZus{}mat} \PYG{o}{=} \PYG{n}{Eq\PYGZus{}instance}\PYG{o}{.}\PYG{n}{rz2roa}\PYG{p}{(}\PYG{n}{R}\PYG{p}{,} \PYG{n}{Z}\PYG{p}{,} \PYG{l+m+mf}{0.2}\PYG{p}{,} \PYG{n}{make\PYGZus{}grid}\PYG{o}{=}\PYG{k+kc}{True}\PYG{p}{)}
\end{sphinxVerbatim}

\end{fulllineitems}

\index{rz2rho() (eqtools.core.Equilibrium method)@\spxentry{rz2rho()}\spxextra{eqtools.core.Equilibrium method}}

\begin{fulllineitems}
\phantomsection\label{\detokenize{eqtools:eqtools.core.Equilibrium.rz2rho}}\pysiglinewithargsret{\sphinxbfcode{\sphinxupquote{rz2rho}}}{\emph{method}, \emph{*args}, \emph{**kwargs}}{}
Convert the passed (R, Z, t) coordinates into one of several coordinates.
\begin{quote}\begin{description}
\item[{Parameters}] \leavevmode\begin{itemize}
\item {} 
\sphinxstyleliteralstrong{\sphinxupquote{method}} (\sphinxstyleliteralemphasis{\sphinxupquote{String}}) \textendash{} 
Indicates which coordinates to convert to. Valid
options are:
\begin{quote}


\begin{savenotes}\sphinxattablestart
\centering
\begin{tabulary}{\linewidth}[t]{|T|T|}
\hline

psinorm
&
Normalized poloidal flux
\\
\hline
phinorm
&
Normalized toroidal flux
\\
\hline
volnorm
&
Normalized volume
\\
\hline
Rmid
&
Midplane major radius
\\
\hline
r/a
&
Normalized minor radius
\\
\hline
q
&
Safety factor
\\
\hline
F
&
Flux function \(F=RB_{\phi}\)
\\
\hline
FFPrime
&
Flux function \(FF'\)
\\
\hline
p
&
Pressure
\\
\hline
pprime
&
Pressure gradient
\\
\hline
v
&
Flux surface volume
\\
\hline
\end{tabulary}
\par
\sphinxattableend\end{savenotes}
\end{quote}

Additionally, each valid option may be prepended with ‘sqrt’
to specify the square root of the desired unit.


\item {} 
\sphinxstyleliteralstrong{\sphinxupquote{R}} (\sphinxstyleliteralemphasis{\sphinxupquote{Array-like}}\sphinxstyleliteralemphasis{\sphinxupquote{ or }}\sphinxstyleliteralemphasis{\sphinxupquote{scalar float}}) \textendash{} Values of the radial coordinate to
map to \sphinxtitleref{rho}. If \sphinxtitleref{R} and \sphinxtitleref{Z} are both scalar values,
they are used as the coordinate pair for all of the values in
\sphinxtitleref{t}. Must have the same shape as \sphinxtitleref{Z} unless the \sphinxtitleref{make\_grid}
keyword is set. If the \sphinxtitleref{make\_grid} keyword is True, \sphinxtitleref{R} must
have exactly one dimension.

\item {} 
\sphinxstyleliteralstrong{\sphinxupquote{Z}} (\sphinxstyleliteralemphasis{\sphinxupquote{Array-like}}\sphinxstyleliteralemphasis{\sphinxupquote{ or }}\sphinxstyleliteralemphasis{\sphinxupquote{scalar float}}) \textendash{} Values of the vertical coordinate to
map to \sphinxtitleref{rho}. If \sphinxtitleref{R} and \sphinxtitleref{Z} are both scalar values,
they are used as the coordinate pair for all of the values in
\sphinxtitleref{t}. Must have the same shape as \sphinxtitleref{R} unless the \sphinxtitleref{make\_grid}
keyword is set. If the \sphinxtitleref{make\_grid} keyword is True, \sphinxtitleref{Z} must
have exactly one dimension.

\item {} 
\sphinxstyleliteralstrong{\sphinxupquote{t}} (\sphinxstyleliteralemphasis{\sphinxupquote{Array-like}}\sphinxstyleliteralemphasis{\sphinxupquote{ or }}\sphinxstyleliteralemphasis{\sphinxupquote{scalar float}}) \textendash{} Times to perform the conversion at.
If \sphinxtitleref{t} is a single value, it is used for all of the elements of
\sphinxtitleref{R}, \sphinxtitleref{Z}. If the \sphinxtitleref{each\_t} keyword is True, then \sphinxtitleref{t} must be
scalar or have exactly one dimension. If the \sphinxtitleref{each\_t} keyword is
False, \sphinxtitleref{t} must have the same shape as \sphinxtitleref{R} and \sphinxtitleref{Z} (or their
meshgrid if \sphinxtitleref{make\_grid} is True).

\end{itemize}

\item[{Keyword Arguments}] \leavevmode\begin{itemize}
\item {} 
\sphinxstyleliteralstrong{\sphinxupquote{sqrt}} (\sphinxstyleliteralemphasis{\sphinxupquote{Boolean}}) \textendash{} Set to True to return the square root of \sphinxtitleref{rho}.
Only the square root of positive values is taken. Negative
values are replaced with zeros, consistent with Steve Wolfe’s
IDL implementation efit\_rz2rho.pro. Default is False.

\item {} 
\sphinxstyleliteralstrong{\sphinxupquote{each\_t}} (\sphinxstyleliteralemphasis{\sphinxupquote{Boolean}}) \textendash{} When True, the elements in \sphinxtitleref{R}, \sphinxtitleref{Z} are evaluated
at each value in \sphinxtitleref{t}. If True, \sphinxtitleref{t} must have only one dimension
(or be a scalar). If False, \sphinxtitleref{t} must match the shape of \sphinxtitleref{R} and
\sphinxtitleref{Z} or be a scalar. Default is True (evaluate ALL \sphinxtitleref{R}, \sphinxtitleref{Z} at
EACH element in \sphinxtitleref{t}).

\item {} 
\sphinxstyleliteralstrong{\sphinxupquote{make\_grid}} (\sphinxstyleliteralemphasis{\sphinxupquote{Boolean}}) \textendash{} Set to True to pass \sphinxtitleref{R} and \sphinxtitleref{Z} through
\sphinxcode{\sphinxupquote{scipy.meshgrid()}} before evaluating. If this is set to
True, \sphinxtitleref{R} and \sphinxtitleref{Z} must each only have a single dimension, but
can have different lengths. Default is False (do not form
meshgrid).

\item {} 
\sphinxstyleliteralstrong{\sphinxupquote{rho}} (\sphinxstyleliteralemphasis{\sphinxupquote{Boolean}}) \textendash{} Set to True to return r/a (normalized minor radius)
instead of Rmid when \sphinxtitleref{destination} is Rmid. Default is False
(return major radius, Rmid).

\item {} 
\sphinxstyleliteralstrong{\sphinxupquote{length\_unit}} (\sphinxstyleliteralemphasis{\sphinxupquote{String}}\sphinxstyleliteralemphasis{\sphinxupquote{ or }}\sphinxstyleliteralemphasis{\sphinxupquote{1}}) \textendash{} 
Length unit that \sphinxtitleref{R}, \sphinxtitleref{Z} are given in,
AND that \sphinxtitleref{Rmid} is returned in. If a string is given, it must
be a valid unit specifier:
\begin{quote}


\begin{savenotes}\sphinxattablestart
\centering
\begin{tabulary}{\linewidth}[t]{|T|T|}
\hline

’m’
&
meters
\\
\hline
’cm’
&
centimeters
\\
\hline
’mm’
&
millimeters
\\
\hline
’in’
&
inches
\\
\hline
’ft’
&
feet
\\
\hline
’yd’
&
yards
\\
\hline
’smoot’
&
smoots
\\
\hline
’cubit’
&
cubits
\\
\hline
’hand’
&
hands
\\
\hline
’default’
&
meters
\\
\hline
\end{tabulary}
\par
\sphinxattableend\end{savenotes}
\end{quote}

If length\_unit is 1 or None, meters are assumed. The default
value is 1 (use meters).


\item {} 
\sphinxstyleliteralstrong{\sphinxupquote{k}} (\sphinxstyleliteralemphasis{\sphinxupquote{positive int}}) \textendash{} The degree of polynomial spline interpolation to
use in converting coordinates.

\item {} 
\sphinxstyleliteralstrong{\sphinxupquote{return\_t}} (\sphinxstyleliteralemphasis{\sphinxupquote{Boolean}}) \textendash{} Set to True to return a tuple of (\sphinxtitleref{rho},
\sphinxtitleref{time\_idxs}), where \sphinxtitleref{time\_idxs} is the array of time indices
actually used in evaluating \sphinxtitleref{rho} with nearest-neighbor
interpolation. (This is mostly present as an internal helper.)
Default is False (only return \sphinxtitleref{rho}).

\end{itemize}

\item[{Returns}] \leavevmode

\sphinxtitleref{rho} or (\sphinxtitleref{rho}, \sphinxtitleref{time\_idxs})
\begin{itemize}
\item {} 
\sphinxstylestrong{rho} (\sphinxtitleref{Array or scalar float}) - The converted coordinates. If
all of the input arguments are scalar, then a scalar is returned.
Otherwise, a scipy Array is returned.

\item {} 
\sphinxstylestrong{time\_idxs} (Array with same shape as \sphinxtitleref{rho}) - The indices
(in \sphinxcode{\sphinxupquote{self.getTimeBase()}}) that were used for
nearest-neighbor interpolation. Only returned if \sphinxtitleref{return\_t} is
True.

\end{itemize}


\item[{Raises}] \leavevmode
\sphinxstyleliteralstrong{\sphinxupquote{ValueError}} \textendash{} If \sphinxtitleref{method} is not one of the supported values.

\end{description}\end{quote}
\subsubsection*{Examples}

All assume that \sphinxtitleref{Eq\_instance} is a valid instance of the appropriate
extension of the {\hyperref[\detokenize{eqtools:eqtools.core.Equilibrium}]{\sphinxcrossref{\sphinxcode{\sphinxupquote{Equilibrium}}}}} abstract class.

Find single psinorm value at R=0.6m, Z=0.0m, t=0.26s:

\begin{sphinxVerbatim}[commandchars=\\\{\}]
\PYG{n}{psi\PYGZus{}val} \PYG{o}{=} \PYG{n}{Eq\PYGZus{}instance}\PYG{o}{.}\PYG{n}{rz2rho}\PYG{p}{(}\PYG{l+s+s1}{\PYGZsq{}}\PYG{l+s+s1}{psinorm}\PYG{l+s+s1}{\PYGZsq{}}\PYG{p}{,} \PYG{l+m+mf}{0.6}\PYG{p}{,} \PYG{l+m+mi}{0}\PYG{p}{,} \PYG{l+m+mf}{0.26}\PYG{p}{)}
\end{sphinxVerbatim}

Find psinorm values at (R, Z) points (0.6m, 0m) and (0.8m, 0m) at the
single time t=0.26s. Note that the \sphinxtitleref{Z} vector must be fully specified,
even if the values are all the same:

\begin{sphinxVerbatim}[commandchars=\\\{\}]
\PYG{n}{psi\PYGZus{}arr} \PYG{o}{=} \PYG{n}{Eq\PYGZus{}instance}\PYG{o}{.}\PYG{n}{rz2rho}\PYG{p}{(}\PYG{l+s+s1}{\PYGZsq{}}\PYG{l+s+s1}{psinorm}\PYG{l+s+s1}{\PYGZsq{}}\PYG{p}{,} \PYG{p}{[}\PYG{l+m+mf}{0.6}\PYG{p}{,} \PYG{l+m+mf}{0.8}\PYG{p}{]}\PYG{p}{,} \PYG{p}{[}\PYG{l+m+mi}{0}\PYG{p}{,} \PYG{l+m+mi}{0}\PYG{p}{]}\PYG{p}{,} \PYG{l+m+mf}{0.26}\PYG{p}{)}
\end{sphinxVerbatim}

Find psinorm values at (R, Z) points (0.6m, 0m) at times t={[}0.2s, 0.3s{]}:

\begin{sphinxVerbatim}[commandchars=\\\{\}]
\PYG{n}{psi\PYGZus{}arr} \PYG{o}{=} \PYG{n}{Eq\PYGZus{}instance}\PYG{o}{.}\PYG{n}{rz2rho}\PYG{p}{(}\PYG{l+s+s1}{\PYGZsq{}}\PYG{l+s+s1}{psinorm}\PYG{l+s+s1}{\PYGZsq{}}\PYG{p}{,} \PYG{l+m+mf}{0.6}\PYG{p}{,} \PYG{l+m+mi}{0}\PYG{p}{,} \PYG{p}{[}\PYG{l+m+mf}{0.2}\PYG{p}{,} \PYG{l+m+mf}{0.3}\PYG{p}{]}\PYG{p}{)}
\end{sphinxVerbatim}

Find psinorm values at (R, Z, t) points (0.6m, 0m, 0.2s) and (0.5m, 0.2m, 0.3s):

\begin{sphinxVerbatim}[commandchars=\\\{\}]
\PYG{n}{psi\PYGZus{}arr} \PYG{o}{=} \PYG{n}{Eq\PYGZus{}instance}\PYG{o}{.}\PYG{n}{rz2rho}\PYG{p}{(}\PYG{l+s+s1}{\PYGZsq{}}\PYG{l+s+s1}{psinorm}\PYG{l+s+s1}{\PYGZsq{}}\PYG{p}{,} \PYG{p}{[}\PYG{l+m+mf}{0.6}\PYG{p}{,} \PYG{l+m+mf}{0.5}\PYG{p}{]}\PYG{p}{,} \PYG{p}{[}\PYG{l+m+mi}{0}\PYG{p}{,} \PYG{l+m+mf}{0.2}\PYG{p}{]}\PYG{p}{,} \PYG{p}{[}\PYG{l+m+mf}{0.2}\PYG{p}{,} \PYG{l+m+mf}{0.3}\PYG{p}{]}\PYG{p}{,} \PYG{n}{each\PYGZus{}t}\PYG{o}{=}\PYG{k+kc}{False}\PYG{p}{)}
\end{sphinxVerbatim}

Find psinorm values on grid defined by 1D vector of radial positions \sphinxtitleref{R}
and 1D vector of vertical positions \sphinxtitleref{Z} at time t=0.2s:

\begin{sphinxVerbatim}[commandchars=\\\{\}]
\PYG{n}{psi\PYGZus{}mat} \PYG{o}{=} \PYG{n}{Eq\PYGZus{}instance}\PYG{o}{.}\PYG{n}{rz2rho}\PYG{p}{(}\PYG{l+s+s1}{\PYGZsq{}}\PYG{l+s+s1}{psinorm}\PYG{l+s+s1}{\PYGZsq{}}\PYG{p}{,} \PYG{n}{R}\PYG{p}{,} \PYG{n}{Z}\PYG{p}{,} \PYG{l+m+mf}{0.2}\PYG{p}{,} \PYG{n}{make\PYGZus{}grid}\PYG{o}{=}\PYG{k+kc}{True}\PYG{p}{)}
\end{sphinxVerbatim}

\end{fulllineitems}

\index{rmid2roa() (eqtools.core.Equilibrium method)@\spxentry{rmid2roa()}\spxextra{eqtools.core.Equilibrium method}}

\begin{fulllineitems}
\phantomsection\label{\detokenize{eqtools:eqtools.core.Equilibrium.rmid2roa}}\pysiglinewithargsret{\sphinxbfcode{\sphinxupquote{rmid2roa}}}{\emph{R\_mid}, \emph{t}, \emph{each\_t=True}, \emph{return\_t=False}, \emph{sqrt=False}, \emph{blob=None}, \emph{length\_unit=1}}{}
Convert the passed (R\_mid, t) coordinates into r/a.
\begin{quote}\begin{description}
\item[{Parameters}] \leavevmode\begin{itemize}
\item {} 
\sphinxstyleliteralstrong{\sphinxupquote{R\_mid}} (\sphinxstyleliteralemphasis{\sphinxupquote{Array-like}}\sphinxstyleliteralemphasis{\sphinxupquote{ or }}\sphinxstyleliteralemphasis{\sphinxupquote{scalar float}}) \textendash{} Values of the outboard midplane
major radius to map to r/a.

\item {} 
\sphinxstyleliteralstrong{\sphinxupquote{t}} (\sphinxstyleliteralemphasis{\sphinxupquote{Array-like}}\sphinxstyleliteralemphasis{\sphinxupquote{ or }}\sphinxstyleliteralemphasis{\sphinxupquote{scalar float}}) \textendash{} Times to perform the conversion at.
If \sphinxtitleref{t} is a single value, it is used for all of the elements of
\sphinxtitleref{R\_mid}. If the \sphinxtitleref{each\_t} keyword is True, then \sphinxtitleref{t} must be scalar
or have exactly one dimension. If the \sphinxtitleref{each\_t} keyword is False,
\sphinxtitleref{t} must have the same shape as \sphinxtitleref{R\_mid}.

\end{itemize}

\item[{Keyword Arguments}] \leavevmode\begin{itemize}
\item {} 
\sphinxstyleliteralstrong{\sphinxupquote{sqrt}} (\sphinxstyleliteralemphasis{\sphinxupquote{Boolean}}) \textendash{} Set to True to return the square root of r/a.
Only the square root of positive values is taken. Negative
values are replaced with zeros, consistent with Steve Wolfe’s
IDL implementation efit\_rz2rho.pro. Default is False.

\item {} 
\sphinxstyleliteralstrong{\sphinxupquote{each\_t}} (\sphinxstyleliteralemphasis{\sphinxupquote{Boolean}}) \textendash{} When True, the elements in \sphinxtitleref{R\_mid} are evaluated
at each value in \sphinxtitleref{t}. If True, \sphinxtitleref{t} must have only one dimension
(or be a scalar). If False, \sphinxtitleref{t} must match the shape of \sphinxtitleref{R\_mid}
or be a scalar. Default is True (evaluate ALL \sphinxtitleref{R\_mid} at EACH
element in \sphinxtitleref{t}).

\item {} 
\sphinxstyleliteralstrong{\sphinxupquote{length\_unit}} (\sphinxstyleliteralemphasis{\sphinxupquote{String}}\sphinxstyleliteralemphasis{\sphinxupquote{ or }}\sphinxstyleliteralemphasis{\sphinxupquote{1}}) \textendash{} 
Length unit that \sphinxtitleref{R\_mid} is given in.
If a string is given, it must be a valid unit specifier:
\begin{quote}


\begin{savenotes}\sphinxattablestart
\centering
\begin{tabulary}{\linewidth}[t]{|T|T|}
\hline

’m’
&
meters
\\
\hline
’cm’
&
centimeters
\\
\hline
’mm’
&
millimeters
\\
\hline
’in’
&
inches
\\
\hline
’ft’
&
feet
\\
\hline
’yd’
&
yards
\\
\hline
’smoot’
&
smoots
\\
\hline
’cubit’
&
cubits
\\
\hline
’hand’
&
hands
\\
\hline
’default’
&
meters
\\
\hline
\end{tabulary}
\par
\sphinxattableend\end{savenotes}
\end{quote}

If length\_unit is 1 or None, meters are assumed. The default
value is 1 (use meters).


\item {} 
\sphinxstyleliteralstrong{\sphinxupquote{return\_t}} (\sphinxstyleliteralemphasis{\sphinxupquote{Boolean}}) \textendash{} Set to True to return a tuple of (\sphinxtitleref{rho},
\sphinxtitleref{time\_idxs}), where \sphinxtitleref{time\_idxs} is the array of time indices
actually used in evaluating \sphinxtitleref{rho} with nearest-neighbor
interpolation. (This is mostly present as an internal helper.)
Default is False (only return \sphinxtitleref{rho}).

\end{itemize}

\item[{Returns}] \leavevmode

\sphinxtitleref{roa} or (\sphinxtitleref{roa}, \sphinxtitleref{time\_idxs})
\begin{itemize}
\item {} 
\sphinxstylestrong{roa} (\sphinxtitleref{Array or scalar float}) - Normalized midplane minor
radius. If all of the input arguments are scalar, then a scalar
is returned. Otherwise, a scipy Array is returned.

\item {} 
\sphinxstylestrong{time\_idxs} (Array with same shape as \sphinxtitleref{roa}) - The indices
(in \sphinxcode{\sphinxupquote{self.getTimeBase()}}) that were used for
nearest-neighbor interpolation. Only returned if \sphinxtitleref{return\_t} is
True.

\end{itemize}


\end{description}\end{quote}
\subsubsection*{Examples}

All assume that \sphinxtitleref{Eq\_instance} is a valid instance of the appropriate
extension of the {\hyperref[\detokenize{eqtools:eqtools.core.Equilibrium}]{\sphinxcrossref{\sphinxcode{\sphinxupquote{Equilibrium}}}}} abstract class.

Find single r/a value at R\_mid=0.6m, t=0.26s:

\begin{sphinxVerbatim}[commandchars=\\\{\}]
\PYG{n}{roa\PYGZus{}val} \PYG{o}{=} \PYG{n}{Eq\PYGZus{}instance}\PYG{o}{.}\PYG{n}{rmid2roa}\PYG{p}{(}\PYG{l+m+mf}{0.6}\PYG{p}{,} \PYG{l+m+mf}{0.26}\PYG{p}{)}
\end{sphinxVerbatim}

Find roa values at R\_mid points 0.6m and 0.8m at the
single time t=0.26s.:

\begin{sphinxVerbatim}[commandchars=\\\{\}]
\PYG{n}{roa\PYGZus{}arr} \PYG{o}{=} \PYG{n}{Eq\PYGZus{}instance}\PYG{o}{.}\PYG{n}{rmid2roa}\PYG{p}{(}\PYG{p}{[}\PYG{l+m+mf}{0.6}\PYG{p}{,} \PYG{l+m+mf}{0.8}\PYG{p}{]}\PYG{p}{,} \PYG{l+m+mf}{0.26}\PYG{p}{)}
\end{sphinxVerbatim}

Find roa values at R\_mid of 0.6m at times t={[}0.2s, 0.3s{]}:

\begin{sphinxVerbatim}[commandchars=\\\{\}]
\PYG{n}{roa\PYGZus{}arr} \PYG{o}{=} \PYG{n}{Eq\PYGZus{}instance}\PYG{o}{.}\PYG{n}{rmid2roa}\PYG{p}{(}\PYG{l+m+mf}{0.6}\PYG{p}{,} \PYG{p}{[}\PYG{l+m+mf}{0.2}\PYG{p}{,} \PYG{l+m+mf}{0.3}\PYG{p}{]}\PYG{p}{)}
\end{sphinxVerbatim}

Find r/a values at (R\_mid, t) points (0.6m, 0.2s) and (0.5m, 0.3s):

\begin{sphinxVerbatim}[commandchars=\\\{\}]
\PYG{n}{roa\PYGZus{}arr} \PYG{o}{=} \PYG{n}{Eq\PYGZus{}instance}\PYG{o}{.}\PYG{n}{rmid2roa}\PYG{p}{(}\PYG{p}{[}\PYG{l+m+mf}{0.6}\PYG{p}{,} \PYG{l+m+mf}{0.5}\PYG{p}{]}\PYG{p}{,} \PYG{p}{[}\PYG{l+m+mf}{0.2}\PYG{p}{,} \PYG{l+m+mf}{0.3}\PYG{p}{]}\PYG{p}{,} \PYG{n}{each\PYGZus{}t}\PYG{o}{=}\PYG{k+kc}{False}\PYG{p}{)}
\end{sphinxVerbatim}

\end{fulllineitems}

\index{rmid2psinorm() (eqtools.core.Equilibrium method)@\spxentry{rmid2psinorm()}\spxextra{eqtools.core.Equilibrium method}}

\begin{fulllineitems}
\phantomsection\label{\detokenize{eqtools:eqtools.core.Equilibrium.rmid2psinorm}}\pysiglinewithargsret{\sphinxbfcode{\sphinxupquote{rmid2psinorm}}}{\emph{R\_mid}, \emph{t}, \emph{**kwargs}}{}
Calculates the normalized poloidal flux corresponding to the passed R\_mid (mapped outboard midplane major radius) values.
\begin{quote}\begin{description}
\item[{Parameters}] \leavevmode\begin{itemize}
\item {} 
\sphinxstyleliteralstrong{\sphinxupquote{R\_mid}} (\sphinxstyleliteralemphasis{\sphinxupquote{Array-like}}\sphinxstyleliteralemphasis{\sphinxupquote{ or }}\sphinxstyleliteralemphasis{\sphinxupquote{scalar float}}) \textendash{} Values of the outboard midplane
major radius to map to psinorm.

\item {} 
\sphinxstyleliteralstrong{\sphinxupquote{t}} (\sphinxstyleliteralemphasis{\sphinxupquote{Array-like}}\sphinxstyleliteralemphasis{\sphinxupquote{ or }}\sphinxstyleliteralemphasis{\sphinxupquote{scalar float}}) \textendash{} Times to perform the conversion at.
If \sphinxtitleref{t} is a single value, it is used for all of the elements of
\sphinxtitleref{R\_mid}. If the \sphinxtitleref{each\_t} keyword is True, then \sphinxtitleref{t} must be scalar
or have exactly one dimension. If the \sphinxtitleref{each\_t} keyword is False,
\sphinxtitleref{t} must have the same shape as \sphinxtitleref{R\_mid}.

\end{itemize}

\item[{Keyword Arguments}] \leavevmode\begin{itemize}
\item {} 
\sphinxstyleliteralstrong{\sphinxupquote{sqrt}} (\sphinxstyleliteralemphasis{\sphinxupquote{Boolean}}) \textendash{} Set to True to return the square root of psinorm.
Only the square root of positive values is taken. Negative
values are replaced with zeros, consistent with Steve Wolfe’s
IDL implementation efit\_rz2rho.pro. Default is False.

\item {} 
\sphinxstyleliteralstrong{\sphinxupquote{each\_t}} (\sphinxstyleliteralemphasis{\sphinxupquote{Boolean}}) \textendash{} When True, the elements in \sphinxtitleref{R\_mid} are evaluated
at each value in \sphinxtitleref{t}. If True, \sphinxtitleref{t} must have only one dimension
(or be a scalar). If False, \sphinxtitleref{t} must match the shape of \sphinxtitleref{R\_mid}
or be a scalar. Default is True (evaluate ALL \sphinxtitleref{R\_mid} at EACH
element in \sphinxtitleref{t}).

\item {} 
\sphinxstyleliteralstrong{\sphinxupquote{length\_unit}} (\sphinxstyleliteralemphasis{\sphinxupquote{String}}\sphinxstyleliteralemphasis{\sphinxupquote{ or }}\sphinxstyleliteralemphasis{\sphinxupquote{1}}) \textendash{} 
Length unit that \sphinxtitleref{R\_mid} is given in.
If a string is given, it must be a valid unit specifier:
\begin{quote}


\begin{savenotes}\sphinxattablestart
\centering
\begin{tabulary}{\linewidth}[t]{|T|T|}
\hline

’m’
&
meters
\\
\hline
’cm’
&
centimeters
\\
\hline
’mm’
&
millimeters
\\
\hline
’in’
&
inches
\\
\hline
’ft’
&
feet
\\
\hline
’yd’
&
yards
\\
\hline
’smoot’
&
smoots
\\
\hline
’cubit’
&
cubits
\\
\hline
’hand’
&
hands
\\
\hline
’default’
&
meters
\\
\hline
\end{tabulary}
\par
\sphinxattableend\end{savenotes}
\end{quote}

If length\_unit is 1 or None, meters are assumed. The default
value is 1 (use meters).


\item {} 
\sphinxstyleliteralstrong{\sphinxupquote{k}} (\sphinxstyleliteralemphasis{\sphinxupquote{positive int}}) \textendash{} The degree of polynomial spline interpolation to
use in converting coordinates.

\item {} 
\sphinxstyleliteralstrong{\sphinxupquote{return\_t}} (\sphinxstyleliteralemphasis{\sphinxupquote{Boolean}}) \textendash{} Set to True to return a tuple of (\sphinxtitleref{rho},
\sphinxtitleref{time\_idxs}), where \sphinxtitleref{time\_idxs} is the array of time indices
actually used in evaluating \sphinxtitleref{rho} with nearest-neighbor
interpolation. (This is mostly present as an internal helper.)
Default is False (only return \sphinxtitleref{rho}).

\end{itemize}

\item[{Returns}] \leavevmode

\sphinxtitleref{psinorm} or (\sphinxtitleref{psinorm}, \sphinxtitleref{time\_idxs})
\begin{itemize}
\item {} 
\sphinxstylestrong{psinorm} (\sphinxtitleref{Array or scalar float}) - Normalized poloidal flux.
If all of the input arguments are scalar, then a scalar is
returned. Otherwise, a scipy Array is returned.

\item {} 
\sphinxstylestrong{time\_idxs} (Array with same shape as \sphinxtitleref{psinorm}) - The indices
(in \sphinxcode{\sphinxupquote{self.getTimeBase()}}) that were used for
nearest-neighbor interpolation. Only returned if \sphinxtitleref{return\_t} is
True.

\end{itemize}


\end{description}\end{quote}
\subsubsection*{Examples}

All assume that \sphinxtitleref{Eq\_instance} is a valid instance of the appropriate
extension of the {\hyperref[\detokenize{eqtools:eqtools.core.Equilibrium}]{\sphinxcrossref{\sphinxcode{\sphinxupquote{Equilibrium}}}}} abstract class.

Find single psinorm value for Rmid=0.7m, t=0.26s:

\begin{sphinxVerbatim}[commandchars=\\\{\}]
\PYG{n}{psinorm\PYGZus{}val} \PYG{o}{=} \PYG{n}{Eq\PYGZus{}instance}\PYG{o}{.}\PYG{n}{rmid2psinorm}\PYG{p}{(}\PYG{l+m+mf}{0.7}\PYG{p}{,} \PYG{l+m+mf}{0.26}\PYG{p}{)}
\end{sphinxVerbatim}

Find psinorm values at R\_mid values of 0.5m and 0.7m at the single time
t=0.26s:

\begin{sphinxVerbatim}[commandchars=\\\{\}]
\PYG{n}{psinorm\PYGZus{}arr} \PYG{o}{=} \PYG{n}{Eq\PYGZus{}instance}\PYG{o}{.}\PYG{n}{rmid2psinorm}\PYG{p}{(}\PYG{p}{[}\PYG{l+m+mf}{0.5}\PYG{p}{,} \PYG{l+m+mf}{0.7}\PYG{p}{]}\PYG{p}{,} \PYG{l+m+mf}{0.26}\PYG{p}{)}
\end{sphinxVerbatim}

Find psinorm values at R\_mid=0.5m at times t={[}0.2s, 0.3s{]}:

\begin{sphinxVerbatim}[commandchars=\\\{\}]
\PYG{n}{psinorm\PYGZus{}arr} \PYG{o}{=} \PYG{n}{Eq\PYGZus{}instance}\PYG{o}{.}\PYG{n}{rmid2psinorm}\PYG{p}{(}\PYG{l+m+mf}{0.5}\PYG{p}{,} \PYG{p}{[}\PYG{l+m+mf}{0.2}\PYG{p}{,} \PYG{l+m+mf}{0.3}\PYG{p}{]}\PYG{p}{)}
\end{sphinxVerbatim}

Find psinorm values at (R\_mid, t) points (0.6m, 0.2s) and (0.5m, 0.3s):

\begin{sphinxVerbatim}[commandchars=\\\{\}]
\PYG{n}{psinorm\PYGZus{}arr} \PYG{o}{=} \PYG{n}{Eq\PYGZus{}instance}\PYG{o}{.}\PYG{n}{rmid2psinorm}\PYG{p}{(}\PYG{p}{[}\PYG{l+m+mf}{0.6}\PYG{p}{,} \PYG{l+m+mf}{0.5}\PYG{p}{]}\PYG{p}{,} \PYG{p}{[}\PYG{l+m+mf}{0.2}\PYG{p}{,} \PYG{l+m+mf}{0.3}\PYG{p}{]}\PYG{p}{,} \PYG{n}{each\PYGZus{}t}\PYG{o}{=}\PYG{k+kc}{False}\PYG{p}{)}
\end{sphinxVerbatim}

\end{fulllineitems}

\index{rmid2phinorm() (eqtools.core.Equilibrium method)@\spxentry{rmid2phinorm()}\spxextra{eqtools.core.Equilibrium method}}

\begin{fulllineitems}
\phantomsection\label{\detokenize{eqtools:eqtools.core.Equilibrium.rmid2phinorm}}\pysiglinewithargsret{\sphinxbfcode{\sphinxupquote{rmid2phinorm}}}{\emph{*args}, \emph{**kwargs}}{}
Calculates the normalized toroidal flux.

Uses the definitions:
\begin{align*}\!\begin{aligned}
\texttt{phi} &= \int q(\psi)\,d\psi\\
\texttt{phi\_norm} &= \frac{\phi}{\phi(a)}\\
\end{aligned}\end{align*}
This is based on the IDL version efit\_rz2rho.pro by Steve Wolfe.
\begin{quote}\begin{description}
\item[{Parameters}] \leavevmode\begin{itemize}
\item {} 
\sphinxstyleliteralstrong{\sphinxupquote{R\_mid}} (\sphinxstyleliteralemphasis{\sphinxupquote{Array-like}}\sphinxstyleliteralemphasis{\sphinxupquote{ or }}\sphinxstyleliteralemphasis{\sphinxupquote{scalar float}}) \textendash{} Values of the outboard midplane
major radius to map to phinorm.

\item {} 
\sphinxstyleliteralstrong{\sphinxupquote{t}} (\sphinxstyleliteralemphasis{\sphinxupquote{Array-like}}\sphinxstyleliteralemphasis{\sphinxupquote{ or }}\sphinxstyleliteralemphasis{\sphinxupquote{scalar float}}) \textendash{} Times to perform the conversion at.
If \sphinxtitleref{t} is a single value, it is used for all of the elements of
\sphinxtitleref{R\_mid}. If the \sphinxtitleref{each\_t} keyword is True, then \sphinxtitleref{t} must be scalar
or have exactly one dimension. If the \sphinxtitleref{each\_t} keyword is False,
\sphinxtitleref{t} must have the same shape as \sphinxtitleref{R\_mid}.

\end{itemize}

\item[{Keyword Arguments}] \leavevmode\begin{itemize}
\item {} 
\sphinxstyleliteralstrong{\sphinxupquote{sqrt}} (\sphinxstyleliteralemphasis{\sphinxupquote{Boolean}}) \textendash{} Set to True to return the square root of phinorm.
Only the square root of positive values is taken. Negative
values are replaced with zeros, consistent with Steve Wolfe’s
IDL implementation efit\_rz2rho.pro. Default is False.

\item {} 
\sphinxstyleliteralstrong{\sphinxupquote{each\_t}} (\sphinxstyleliteralemphasis{\sphinxupquote{Boolean}}) \textendash{} When True, the elements in \sphinxtitleref{R\_mid} are evaluated
at each value in \sphinxtitleref{t}. If True, \sphinxtitleref{t} must have only one dimension
(or be a scalar). If False, \sphinxtitleref{t} must match the shape of \sphinxtitleref{R\_mid}
or be a scalar. Default is True (evaluate ALL \sphinxtitleref{R\_mid} at EACH
element in \sphinxtitleref{t}).

\item {} 
\sphinxstyleliteralstrong{\sphinxupquote{length\_unit}} (\sphinxstyleliteralemphasis{\sphinxupquote{String}}\sphinxstyleliteralemphasis{\sphinxupquote{ or }}\sphinxstyleliteralemphasis{\sphinxupquote{1}}) \textendash{} 
Length unit that \sphinxtitleref{R\_mid} is given in.
If a string is given, it must be a valid unit specifier:
\begin{quote}


\begin{savenotes}\sphinxattablestart
\centering
\begin{tabulary}{\linewidth}[t]{|T|T|}
\hline

’m’
&
meters
\\
\hline
’cm’
&
centimeters
\\
\hline
’mm’
&
millimeters
\\
\hline
’in’
&
inches
\\
\hline
’ft’
&
feet
\\
\hline
’yd’
&
yards
\\
\hline
’smoot’
&
smoots
\\
\hline
’cubit’
&
cubits
\\
\hline
’hand’
&
hands
\\
\hline
’default’
&
meters
\\
\hline
\end{tabulary}
\par
\sphinxattableend\end{savenotes}
\end{quote}

If length\_unit is 1 or None, meters are assumed. The default
value is 1 (use meters).


\item {} 
\sphinxstyleliteralstrong{\sphinxupquote{k}} (\sphinxstyleliteralemphasis{\sphinxupquote{positive int}}) \textendash{} The degree of polynomial spline interpolation to
use in converting coordinates.

\item {} 
\sphinxstyleliteralstrong{\sphinxupquote{return\_t}} (\sphinxstyleliteralemphasis{\sphinxupquote{Boolean}}) \textendash{} Set to True to return a tuple of (\sphinxtitleref{rho},
\sphinxtitleref{time\_idxs}), where \sphinxtitleref{time\_idxs} is the array of time indices
actually used in evaluating \sphinxtitleref{rho} with nearest-neighbor
interpolation. (This is mostly present as an internal helper.)
Default is False (only return \sphinxtitleref{rho}).

\end{itemize}

\item[{Returns}] \leavevmode

\sphinxtitleref{phinorm} or (\sphinxtitleref{phinorm}, \sphinxtitleref{time\_idxs})
\begin{itemize}
\item {} 
\sphinxstylestrong{phinorm} (\sphinxtitleref{Array or scalar float}) - Normalized toroidal flux.
If all of the input arguments are scalar, then a scalar is
returned. Otherwise, a scipy Array is returned.

\item {} 
\sphinxstylestrong{time\_idxs} (Array with same shape as \sphinxtitleref{phinorm}) - The indices
(in \sphinxcode{\sphinxupquote{self.getTimeBase()}}) that were used for
nearest-neighbor interpolation. Only returned if \sphinxtitleref{return\_t} is
True.

\end{itemize}


\end{description}\end{quote}
\subsubsection*{Examples}

All assume that \sphinxtitleref{Eq\_instance} is a valid instance of the appropriate
extension of the {\hyperref[\detokenize{eqtools:eqtools.core.Equilibrium}]{\sphinxcrossref{\sphinxcode{\sphinxupquote{Equilibrium}}}}} abstract class.

Find single phinorm value at R\_mid=0.6m, t=0.26s:

\begin{sphinxVerbatim}[commandchars=\\\{\}]
\PYG{n}{phi\PYGZus{}val} \PYG{o}{=} \PYG{n}{Eq\PYGZus{}instance}\PYG{o}{.}\PYG{n}{rmid2phinorm}\PYG{p}{(}\PYG{l+m+mf}{0.6}\PYG{p}{,} \PYG{l+m+mf}{0.26}\PYG{p}{)}
\end{sphinxVerbatim}

Find phinorm values at R\_mid points 0.6m and 0.8m at the single time
t=0.26s:

\begin{sphinxVerbatim}[commandchars=\\\{\}]
\PYG{n}{phi\PYGZus{}arr} \PYG{o}{=} \PYG{n}{Eq\PYGZus{}instance}\PYG{o}{.}\PYG{n}{rmid2phinorm}\PYG{p}{(}\PYG{p}{[}\PYG{l+m+mf}{0.6}\PYG{p}{,} \PYG{l+m+mf}{0.8}\PYG{p}{]}\PYG{p}{,} \PYG{l+m+mf}{0.26}\PYG{p}{)}
\end{sphinxVerbatim}

Find phinorm values at R\_mid point 0.6m at times t={[}0.2s, 0.3s{]}:

\begin{sphinxVerbatim}[commandchars=\\\{\}]
\PYG{n}{phi\PYGZus{}arr} \PYG{o}{=} \PYG{n}{Eq\PYGZus{}instance}\PYG{o}{.}\PYG{n}{rmid2phinorm}\PYG{p}{(}\PYG{l+m+mf}{0.6}\PYG{p}{,} \PYG{p}{[}\PYG{l+m+mf}{0.2}\PYG{p}{,} \PYG{l+m+mf}{0.3}\PYG{p}{]}\PYG{p}{)}
\end{sphinxVerbatim}

Find phinorm values at (R, t) points (0.6m, 0.2s) and (0.5m, 0.3s):

\begin{sphinxVerbatim}[commandchars=\\\{\}]
\PYG{n}{phi\PYGZus{}arr} \PYG{o}{=} \PYG{n}{Eq\PYGZus{}instance}\PYG{o}{.}\PYG{n}{rmid2phinorm}\PYG{p}{(}\PYG{p}{[}\PYG{l+m+mf}{0.6}\PYG{p}{,} \PYG{l+m+mf}{0.5}\PYG{p}{]}\PYG{p}{,} \PYG{p}{[}\PYG{l+m+mf}{0.2}\PYG{p}{,} \PYG{l+m+mf}{0.3}\PYG{p}{]}\PYG{p}{,} \PYG{n}{each\PYGZus{}t}\PYG{o}{=}\PYG{k+kc}{False}\PYG{p}{)}
\end{sphinxVerbatim}

\end{fulllineitems}

\index{rmid2volnorm() (eqtools.core.Equilibrium method)@\spxentry{rmid2volnorm()}\spxextra{eqtools.core.Equilibrium method}}

\begin{fulllineitems}
\phantomsection\label{\detokenize{eqtools:eqtools.core.Equilibrium.rmid2volnorm}}\pysiglinewithargsret{\sphinxbfcode{\sphinxupquote{rmid2volnorm}}}{\emph{*args}, \emph{**kwargs}}{}
Calculates the normalized flux surface volume.

Based on the IDL version efit\_rz2rho.pro by Steve Wolfe.
\begin{quote}\begin{description}
\item[{Parameters}] \leavevmode\begin{itemize}
\item {} 
\sphinxstyleliteralstrong{\sphinxupquote{R\_mid}} (\sphinxstyleliteralemphasis{\sphinxupquote{Array-like}}\sphinxstyleliteralemphasis{\sphinxupquote{ or }}\sphinxstyleliteralemphasis{\sphinxupquote{scalar float}}) \textendash{} Values of the outboard midplane
major radius to map to volnorm.

\item {} 
\sphinxstyleliteralstrong{\sphinxupquote{t}} (\sphinxstyleliteralemphasis{\sphinxupquote{Array-like}}\sphinxstyleliteralemphasis{\sphinxupquote{ or }}\sphinxstyleliteralemphasis{\sphinxupquote{scalar float}}) \textendash{} Times to perform the conversion at.
If \sphinxtitleref{t} is a single value, it is used for all of the elements of
\sphinxtitleref{R\_mid}. If the \sphinxtitleref{each\_t} keyword is True, then \sphinxtitleref{t} must be scalar
or have exactly one dimension. If the \sphinxtitleref{each\_t} keyword is False,
\sphinxtitleref{t} must have the same shape as \sphinxtitleref{R\_mid}.

\end{itemize}

\item[{Keyword Arguments}] \leavevmode\begin{itemize}
\item {} 
\sphinxstyleliteralstrong{\sphinxupquote{sqrt}} (\sphinxstyleliteralemphasis{\sphinxupquote{Boolean}}) \textendash{} Set to True to return the square root of volnorm.
Only the square root of positive values is taken. Negative
values are replaced with zeros, consistent with Steve Wolfe’s
IDL implementation efit\_rz2rho.pro. Default is False.

\item {} 
\sphinxstyleliteralstrong{\sphinxupquote{each\_t}} (\sphinxstyleliteralemphasis{\sphinxupquote{Boolean}}) \textendash{} When True, the elements in \sphinxtitleref{R\_mid} are evaluated
at each value in \sphinxtitleref{t}. If True, \sphinxtitleref{t} must have only one dimension
(or be a scalar). If False, \sphinxtitleref{t} must match the shape of \sphinxtitleref{R\_mid}
or be a scalar. Default is True (evaluate ALL \sphinxtitleref{R\_mid} at EACH
element in \sphinxtitleref{t}).

\item {} 
\sphinxstyleliteralstrong{\sphinxupquote{length\_unit}} (\sphinxstyleliteralemphasis{\sphinxupquote{String}}\sphinxstyleliteralemphasis{\sphinxupquote{ or }}\sphinxstyleliteralemphasis{\sphinxupquote{1}}) \textendash{} 
Length unit that \sphinxtitleref{R\_mid} is given in.
If a string is given, it must be a valid unit specifier:
\begin{quote}


\begin{savenotes}\sphinxattablestart
\centering
\begin{tabulary}{\linewidth}[t]{|T|T|}
\hline

’m’
&
meters
\\
\hline
’cm’
&
centimeters
\\
\hline
’mm’
&
millimeters
\\
\hline
’in’
&
inches
\\
\hline
’ft’
&
feet
\\
\hline
’yd’
&
yards
\\
\hline
’smoot’
&
smoots
\\
\hline
’cubit’
&
cubits
\\
\hline
’hand’
&
hands
\\
\hline
’default’
&
meters
\\
\hline
\end{tabulary}
\par
\sphinxattableend\end{savenotes}
\end{quote}

If length\_unit is 1 or None, meters are assumed. The default
value is 1 (use meters).


\item {} 
\sphinxstyleliteralstrong{\sphinxupquote{k}} (\sphinxstyleliteralemphasis{\sphinxupquote{positive int}}) \textendash{} The degree of polynomial spline interpolation to
use in converting coordinates.

\item {} 
\sphinxstyleliteralstrong{\sphinxupquote{return\_t}} (\sphinxstyleliteralemphasis{\sphinxupquote{Boolean}}) \textendash{} Set to True to return a tuple of (\sphinxtitleref{rho},
\sphinxtitleref{time\_idxs}), where \sphinxtitleref{time\_idxs} is the array of time indices
actually used in evaluating \sphinxtitleref{rho} with nearest-neighbor
interpolation. (This is mostly present as an internal helper.)
Default is False (only return \sphinxtitleref{rho}).

\end{itemize}

\item[{Returns}] \leavevmode

\sphinxtitleref{volnorm} or (\sphinxtitleref{volnorm}, \sphinxtitleref{time\_idxs})
\begin{itemize}
\item {} 
\sphinxstylestrong{volnorm} (\sphinxtitleref{Array or scalar float}) - Normalized volume.
If all of the input arguments are scalar, then a scalar is
returned. Otherwise, a scipy Array is returned.

\item {} 
\sphinxstylestrong{time\_idxs} (Array with same shape as \sphinxtitleref{volnorm}) - The indices
(in \sphinxcode{\sphinxupquote{self.getTimeBase()}}) that were used for
nearest-neighbor interpolation. Only returned if \sphinxtitleref{return\_t} is
True.

\end{itemize}


\end{description}\end{quote}
\subsubsection*{Examples}

All assume that \sphinxtitleref{Eq\_instance} is a valid instance of the appropriate
extension of the {\hyperref[\detokenize{eqtools:eqtools.core.Equilibrium}]{\sphinxcrossref{\sphinxcode{\sphinxupquote{Equilibrium}}}}} abstract class.

Find single volnorm value at R\_mid=0.6m, t=0.26s:

\begin{sphinxVerbatim}[commandchars=\\\{\}]
\PYG{n}{vol\PYGZus{}val} \PYG{o}{=} \PYG{n}{Eq\PYGZus{}instance}\PYG{o}{.}\PYG{n}{rmid2volnorm}\PYG{p}{(}\PYG{l+m+mf}{0.6}\PYG{p}{,} \PYG{l+m+mf}{0.26}\PYG{p}{)}
\end{sphinxVerbatim}

Find volnorm values at R\_mid points 0.6m and 0.8m at the single time
t=0.26s:

\begin{sphinxVerbatim}[commandchars=\\\{\}]
\PYG{n}{vol\PYGZus{}arr} \PYG{o}{=} \PYG{n}{Eq\PYGZus{}instance}\PYG{o}{.}\PYG{n}{rmid2volnorm}\PYG{p}{(}\PYG{p}{[}\PYG{l+m+mf}{0.6}\PYG{p}{,} \PYG{l+m+mf}{0.8}\PYG{p}{]}\PYG{p}{,} \PYG{l+m+mf}{0.26}\PYG{p}{)}
\end{sphinxVerbatim}

Find volnorm values at R\_mid points 0.6m at times t={[}0.2s, 0.3s{]}:

\begin{sphinxVerbatim}[commandchars=\\\{\}]
\PYG{n}{vol\PYGZus{}arr} \PYG{o}{=} \PYG{n}{Eq\PYGZus{}instance}\PYG{o}{.}\PYG{n}{rmid2volnorm}\PYG{p}{(}\PYG{l+m+mf}{0.6}\PYG{p}{,} \PYG{p}{[}\PYG{l+m+mf}{0.2}\PYG{p}{,} \PYG{l+m+mf}{0.3}\PYG{p}{]}\PYG{p}{)}
\end{sphinxVerbatim}

Find volnorm values at (R\_mid, t) points (0.6m, 0.2s) and (0.5m, 0.3s):

\begin{sphinxVerbatim}[commandchars=\\\{\}]
\PYG{n}{vol\PYGZus{}arr} \PYG{o}{=} \PYG{n}{Eq\PYGZus{}instance}\PYG{o}{.}\PYG{n}{rmid2volnorm}\PYG{p}{(}\PYG{p}{[}\PYG{l+m+mf}{0.6}\PYG{p}{,} \PYG{l+m+mf}{0.5}\PYG{p}{]}\PYG{p}{,} \PYG{p}{[}\PYG{l+m+mf}{0.2}\PYG{p}{,} \PYG{l+m+mf}{0.3}\PYG{p}{]}\PYG{p}{,} \PYG{n}{each\PYGZus{}t}\PYG{o}{=}\PYG{k+kc}{False}\PYG{p}{)}
\end{sphinxVerbatim}

\end{fulllineitems}

\index{rmid2rho() (eqtools.core.Equilibrium method)@\spxentry{rmid2rho()}\spxextra{eqtools.core.Equilibrium method}}

\begin{fulllineitems}
\phantomsection\label{\detokenize{eqtools:eqtools.core.Equilibrium.rmid2rho}}\pysiglinewithargsret{\sphinxbfcode{\sphinxupquote{rmid2rho}}}{\emph{method}, \emph{R\_mid}, \emph{t}, \emph{**kwargs}}{}
Convert the passed (R\_mid, t) coordinates into one of several coordinates.
\begin{quote}\begin{description}
\item[{Parameters}] \leavevmode\begin{itemize}
\item {} 
\sphinxstyleliteralstrong{\sphinxupquote{method}} (\sphinxstyleliteralemphasis{\sphinxupquote{String}}) \textendash{} 
Indicates which coordinates to convert to. Valid
options are:
\begin{quote}


\begin{savenotes}\sphinxattablestart
\centering
\begin{tabulary}{\linewidth}[t]{|T|T|}
\hline

psinorm
&
Normalized poloidal flux
\\
\hline
phinorm
&
Normalized toroidal flux
\\
\hline
volnorm
&
Normalized volume
\\
\hline
r/a
&
Normalized minor radius
\\
\hline
F
&
Flux function \(F=RB_{\phi}\)
\\
\hline
FFPrime
&
Flux function \(FF'\)
\\
\hline
p
&
Pressure
\\
\hline
pprime
&
Pressure gradient
\\
\hline
v
&
Flux surface volume
\\
\hline
\end{tabulary}
\par
\sphinxattableend\end{savenotes}
\end{quote}

Additionally, each valid option may be prepended with ‘sqrt’
to specify the square root of the desired unit.


\item {} 
\sphinxstyleliteralstrong{\sphinxupquote{R\_mid}} (\sphinxstyleliteralemphasis{\sphinxupquote{Array-like}}\sphinxstyleliteralemphasis{\sphinxupquote{ or }}\sphinxstyleliteralemphasis{\sphinxupquote{scalar float}}) \textendash{} Values of the outboard midplane
major radius to map to rho.

\item {} 
\sphinxstyleliteralstrong{\sphinxupquote{t}} (\sphinxstyleliteralemphasis{\sphinxupquote{Array-like}}\sphinxstyleliteralemphasis{\sphinxupquote{ or }}\sphinxstyleliteralemphasis{\sphinxupquote{scalar float}}) \textendash{} Times to perform the conversion at.
If \sphinxtitleref{t} is a single value, it is used for all of the elements of
\sphinxtitleref{R\_mid}. If the \sphinxtitleref{each\_t} keyword is True, then \sphinxtitleref{t} must be scalar
or have exactly one dimension. If the \sphinxtitleref{each\_t} keyword is False,
\sphinxtitleref{t} must have the same shape as \sphinxtitleref{R\_mid}.

\end{itemize}

\item[{Keyword Arguments}] \leavevmode\begin{itemize}
\item {} 
\sphinxstyleliteralstrong{\sphinxupquote{sqrt}} (\sphinxstyleliteralemphasis{\sphinxupquote{Boolean}}) \textendash{} Set to True to return the square root of rho.
Only the square root of positive values is taken. Negative
values are replaced with zeros, consistent with Steve Wolfe’s
IDL implementation efit\_rz2rho.pro. Default is False.

\item {} 
\sphinxstyleliteralstrong{\sphinxupquote{each\_t}} (\sphinxstyleliteralemphasis{\sphinxupquote{Boolean}}) \textendash{} When True, the elements in \sphinxtitleref{R\_mid} are evaluated
at each value in \sphinxtitleref{t}. If True, \sphinxtitleref{t} must have only one dimension
(or be a scalar). If False, \sphinxtitleref{t} must match the shape of \sphinxtitleref{R\_mid}
or be a scalar. Default is True (evaluate ALL \sphinxtitleref{R\_mid} at EACH
element in \sphinxtitleref{t}).

\item {} 
\sphinxstyleliteralstrong{\sphinxupquote{length\_unit}} (\sphinxstyleliteralemphasis{\sphinxupquote{String}}\sphinxstyleliteralemphasis{\sphinxupquote{ or }}\sphinxstyleliteralemphasis{\sphinxupquote{1}}) \textendash{} 
Length unit that \sphinxtitleref{R\_mid} is given in.
If a string is given, it must be a valid unit specifier:
\begin{quote}


\begin{savenotes}\sphinxattablestart
\centering
\begin{tabulary}{\linewidth}[t]{|T|T|}
\hline

’m’
&
meters
\\
\hline
’cm’
&
centimeters
\\
\hline
’mm’
&
millimeters
\\
\hline
’in’
&
inches
\\
\hline
’ft’
&
feet
\\
\hline
’yd’
&
yards
\\
\hline
’smoot’
&
smoots
\\
\hline
’cubit’
&
cubits
\\
\hline
’hand’
&
hands
\\
\hline
’default’
&
meters
\\
\hline
\end{tabulary}
\par
\sphinxattableend\end{savenotes}
\end{quote}

If length\_unit is 1 or None, meters are assumed. The default
value is 1 (use meters).


\item {} 
\sphinxstyleliteralstrong{\sphinxupquote{k}} (\sphinxstyleliteralemphasis{\sphinxupquote{positive int}}) \textendash{} The degree of polynomial spline interpolation to
use in converting coordinates.

\item {} 
\sphinxstyleliteralstrong{\sphinxupquote{return\_t}} (\sphinxstyleliteralemphasis{\sphinxupquote{Boolean}}) \textendash{} Set to True to return a tuple of (\sphinxtitleref{rho},
\sphinxtitleref{time\_idxs}), where \sphinxtitleref{time\_idxs} is the array of time indices
actually used in evaluating \sphinxtitleref{rho} with nearest-neighbor
interpolation. (This is mostly present as an internal helper.)
Default is False (only return \sphinxtitleref{rho}).

\end{itemize}

\item[{Returns}] \leavevmode

\sphinxtitleref{rho} or (\sphinxtitleref{rho}, \sphinxtitleref{time\_idxs})
\begin{itemize}
\item {} 
\sphinxstylestrong{rho} (\sphinxtitleref{Array or scalar float}) - The converted coordinates. If
all of the input arguments are scalar, then a scalar is returned.
Otherwise, a scipy Array is returned.

\item {} 
\sphinxstylestrong{time\_idxs} (Array with same shape as \sphinxtitleref{rho}) - The indices
(in \sphinxcode{\sphinxupquote{self.getTimeBase()}}) that were used for
nearest-neighbor interpolation. Only returned if \sphinxtitleref{return\_t} is
True.

\end{itemize}


\end{description}\end{quote}
\subsubsection*{Examples}

All assume that \sphinxtitleref{Eq\_instance} is a valid instance of the appropriate
extension of the {\hyperref[\detokenize{eqtools:eqtools.core.Equilibrium}]{\sphinxcrossref{\sphinxcode{\sphinxupquote{Equilibrium}}}}} abstract class.

Find single psinorm value at R\_mid=0.6m, t=0.26s:

\begin{sphinxVerbatim}[commandchars=\\\{\}]
\PYG{n}{psi\PYGZus{}val} \PYG{o}{=} \PYG{n}{Eq\PYGZus{}instance}\PYG{o}{.}\PYG{n}{rmid2rho}\PYG{p}{(}\PYG{l+s+s1}{\PYGZsq{}}\PYG{l+s+s1}{psinorm}\PYG{l+s+s1}{\PYGZsq{}}\PYG{p}{,} \PYG{l+m+mf}{0.6}\PYG{p}{,} \PYG{l+m+mf}{0.26}\PYG{p}{)}
\end{sphinxVerbatim}

Find psinorm values at R\_mid points 0.6m and 0.8m at the
single time t=0.26s.:

\begin{sphinxVerbatim}[commandchars=\\\{\}]
\PYG{n}{psi\PYGZus{}arr} \PYG{o}{=} \PYG{n}{Eq\PYGZus{}instance}\PYG{o}{.}\PYG{n}{rmid2rho}\PYG{p}{(}\PYG{l+s+s1}{\PYGZsq{}}\PYG{l+s+s1}{psinorm}\PYG{l+s+s1}{\PYGZsq{}}\PYG{p}{,} \PYG{p}{[}\PYG{l+m+mf}{0.6}\PYG{p}{,} \PYG{l+m+mf}{0.8}\PYG{p}{]}\PYG{p}{,} \PYG{l+m+mf}{0.26}\PYG{p}{)}
\end{sphinxVerbatim}

Find psinorm values at R\_mid of 0.6m at times t={[}0.2s, 0.3s{]}:

\begin{sphinxVerbatim}[commandchars=\\\{\}]
\PYG{n}{psi\PYGZus{}arr} \PYG{o}{=} \PYG{n}{Eq\PYGZus{}instance}\PYG{o}{.}\PYG{n}{rmid2rho}\PYG{p}{(}\PYG{l+s+s1}{\PYGZsq{}}\PYG{l+s+s1}{psinorm}\PYG{l+s+s1}{\PYGZsq{}}\PYG{p}{,} \PYG{l+m+mf}{0.6}\PYG{p}{,} \PYG{p}{[}\PYG{l+m+mf}{0.2}\PYG{p}{,} \PYG{l+m+mf}{0.3}\PYG{p}{]}\PYG{p}{)}
\end{sphinxVerbatim}

Find psinorm values at (R\_mid, t) points (0.6m, 0.2s) and (0.5m, 0.3s):

\begin{sphinxVerbatim}[commandchars=\\\{\}]
\PYG{n}{psi\PYGZus{}arr} \PYG{o}{=} \PYG{n}{Eq\PYGZus{}instance}\PYG{o}{.}\PYG{n}{rmid2rho}\PYG{p}{(}\PYG{l+s+s1}{\PYGZsq{}}\PYG{l+s+s1}{psinorm}\PYG{l+s+s1}{\PYGZsq{}}\PYG{p}{,} \PYG{p}{[}\PYG{l+m+mf}{0.6}\PYG{p}{,} \PYG{l+m+mf}{0.5}\PYG{p}{]}\PYG{p}{,} \PYG{p}{[}\PYG{l+m+mf}{0.2}\PYG{p}{,} \PYG{l+m+mf}{0.3}\PYG{p}{]}\PYG{p}{,} \PYG{n}{each\PYGZus{}t}\PYG{o}{=}\PYG{k+kc}{False}\PYG{p}{)}
\end{sphinxVerbatim}

\end{fulllineitems}

\index{roa2rmid() (eqtools.core.Equilibrium method)@\spxentry{roa2rmid()}\spxextra{eqtools.core.Equilibrium method}}

\begin{fulllineitems}
\phantomsection\label{\detokenize{eqtools:eqtools.core.Equilibrium.roa2rmid}}\pysiglinewithargsret{\sphinxbfcode{\sphinxupquote{roa2rmid}}}{\emph{roa}, \emph{t}, \emph{each\_t=True}, \emph{return\_t=False}, \emph{blob=None}, \emph{length\_unit=1}}{}
Convert the passed (r/a, t) coordinates into Rmid.
\begin{quote}\begin{description}
\item[{Parameters}] \leavevmode\begin{itemize}
\item {} 
\sphinxstyleliteralstrong{\sphinxupquote{roa}} (\sphinxstyleliteralemphasis{\sphinxupquote{Array-like}}\sphinxstyleliteralemphasis{\sphinxupquote{ or }}\sphinxstyleliteralemphasis{\sphinxupquote{scalar float}}) \textendash{} Values of the normalized minor
radius to map to Rmid.

\item {} 
\sphinxstyleliteralstrong{\sphinxupquote{t}} (\sphinxstyleliteralemphasis{\sphinxupquote{Array-like}}\sphinxstyleliteralemphasis{\sphinxupquote{ or }}\sphinxstyleliteralemphasis{\sphinxupquote{scalar float}}) \textendash{} Times to perform the conversion at.
If \sphinxtitleref{t} is a single value, it is used for all of the elements of
\sphinxtitleref{roa}. If the \sphinxtitleref{each\_t} keyword is True, then \sphinxtitleref{t} must be scalar
or have exactly one dimension. If the \sphinxtitleref{each\_t} keyword is False,
\sphinxtitleref{t} must have the same shape as \sphinxtitleref{roa}.

\end{itemize}

\item[{Keyword Arguments}] \leavevmode\begin{itemize}
\item {} 
\sphinxstyleliteralstrong{\sphinxupquote{each\_t}} (\sphinxstyleliteralemphasis{\sphinxupquote{Boolean}}) \textendash{} When True, the elements in \sphinxtitleref{roa} are evaluated
at each value in \sphinxtitleref{t}. If True, \sphinxtitleref{t} must have only one dimension
(or be a scalar). If False, \sphinxtitleref{t} must match the shape of \sphinxtitleref{roa}
or be a scalar. Default is True (evaluate ALL \sphinxtitleref{roa} at EACH
element in \sphinxtitleref{t}).

\item {} 
\sphinxstyleliteralstrong{\sphinxupquote{length\_unit}} (\sphinxstyleliteralemphasis{\sphinxupquote{String}}\sphinxstyleliteralemphasis{\sphinxupquote{ or }}\sphinxstyleliteralemphasis{\sphinxupquote{1}}) \textendash{} 
Length unit that \sphinxtitleref{Rmid} is returned in.
If a string is given, it must be a valid unit specifier:
\begin{quote}


\begin{savenotes}\sphinxattablestart
\centering
\begin{tabulary}{\linewidth}[t]{|T|T|}
\hline

’m’
&
meters
\\
\hline
’cm’
&
centimeters
\\
\hline
’mm’
&
millimeters
\\
\hline
’in’
&
inches
\\
\hline
’ft’
&
feet
\\
\hline
’yd’
&
yards
\\
\hline
’smoot’
&
smoots
\\
\hline
’cubit’
&
cubits
\\
\hline
’hand’
&
hands
\\
\hline
’default’
&
meters
\\
\hline
\end{tabulary}
\par
\sphinxattableend\end{savenotes}
\end{quote}

If length\_unit is 1 or None, meters are assumed. The default
value is 1 (use meters).


\item {} 
\sphinxstyleliteralstrong{\sphinxupquote{return\_t}} (\sphinxstyleliteralemphasis{\sphinxupquote{Boolean}}) \textendash{} Set to True to return a tuple of (\sphinxtitleref{rho},
\sphinxtitleref{time\_idxs}), where \sphinxtitleref{time\_idxs} is the array of time indices
actually used in evaluating \sphinxtitleref{rho} with nearest-neighbor
interpolation. (This is mostly present as an internal helper.)
Default is False (only return \sphinxtitleref{rho}).

\end{itemize}

\item[{Returns}] \leavevmode

\sphinxtitleref{Rmid} or (\sphinxtitleref{Rmid}, \sphinxtitleref{time\_idxs})
\begin{itemize}
\item {} 
\sphinxstylestrong{Rmid} (\sphinxtitleref{Array or scalar float}) - The converted coordinates. If
all of the input arguments are scalar, then a scalar is returned.
Otherwise, a scipy Array is returned.

\item {} 
\sphinxstylestrong{time\_idxs} (Array with same shape as \sphinxtitleref{Rmid}) - The indices
(in \sphinxcode{\sphinxupquote{self.getTimeBase()}}) that were used for
nearest-neighbor interpolation. Only returned if \sphinxtitleref{return\_t} is
True.

\end{itemize}


\end{description}\end{quote}
\subsubsection*{Examples}

All assume that \sphinxtitleref{Eq\_instance} is a valid instance of the appropriate
extension of the {\hyperref[\detokenize{eqtools:eqtools.core.Equilibrium}]{\sphinxcrossref{\sphinxcode{\sphinxupquote{Equilibrium}}}}} abstract class.

Find single R\_mid value at r/a=0.6, t=0.26s:

\begin{sphinxVerbatim}[commandchars=\\\{\}]
\PYG{n}{R\PYGZus{}mid\PYGZus{}val} \PYG{o}{=} \PYG{n}{Eq\PYGZus{}instance}\PYG{o}{.}\PYG{n}{roa2rmid}\PYG{p}{(}\PYG{l+m+mf}{0.6}\PYG{p}{,} \PYG{l+m+mf}{0.26}\PYG{p}{)}
\end{sphinxVerbatim}

Find R\_mid values at r/a points 0.6 and 0.8 at the
single time t=0.26s.:

\begin{sphinxVerbatim}[commandchars=\\\{\}]
\PYG{n}{R\PYGZus{}mid\PYGZus{}arr} \PYG{o}{=} \PYG{n}{Eq\PYGZus{}instance}\PYG{o}{.}\PYG{n}{roa2rmid}\PYG{p}{(}\PYG{p}{[}\PYG{l+m+mf}{0.6}\PYG{p}{,} \PYG{l+m+mf}{0.8}\PYG{p}{]}\PYG{p}{,} \PYG{l+m+mf}{0.26}\PYG{p}{)}
\end{sphinxVerbatim}

Find R\_mid values at r/a of 0.6 at times t={[}0.2s, 0.3s{]}:

\begin{sphinxVerbatim}[commandchars=\\\{\}]
\PYG{n}{R\PYGZus{}mid\PYGZus{}arr} \PYG{o}{=} \PYG{n}{Eq\PYGZus{}instance}\PYG{o}{.}\PYG{n}{roa2rmid}\PYG{p}{(}\PYG{l+m+mf}{0.6}\PYG{p}{,} \PYG{p}{[}\PYG{l+m+mf}{0.2}\PYG{p}{,} \PYG{l+m+mf}{0.3}\PYG{p}{]}\PYG{p}{)}
\end{sphinxVerbatim}

Find R\_mid values at (roa, t) points (0.6, 0.2s) and (0.5, 0.3s):

\begin{sphinxVerbatim}[commandchars=\\\{\}]
\PYG{n}{R\PYGZus{}mid\PYGZus{}arr} \PYG{o}{=} \PYG{n}{Eq\PYGZus{}instance}\PYG{o}{.}\PYG{n}{roa2rmid}\PYG{p}{(}\PYG{p}{[}\PYG{l+m+mf}{0.6}\PYG{p}{,} \PYG{l+m+mf}{0.5}\PYG{p}{]}\PYG{p}{,} \PYG{p}{[}\PYG{l+m+mf}{0.2}\PYG{p}{,} \PYG{l+m+mf}{0.3}\PYG{p}{]}\PYG{p}{,} \PYG{n}{each\PYGZus{}t}\PYG{o}{=}\PYG{k+kc}{False}\PYG{p}{)}
\end{sphinxVerbatim}

\end{fulllineitems}

\index{roa2psinorm() (eqtools.core.Equilibrium method)@\spxentry{roa2psinorm()}\spxextra{eqtools.core.Equilibrium method}}

\begin{fulllineitems}
\phantomsection\label{\detokenize{eqtools:eqtools.core.Equilibrium.roa2psinorm}}\pysiglinewithargsret{\sphinxbfcode{\sphinxupquote{roa2psinorm}}}{\emph{*args}, \emph{**kwargs}}{}
Convert the passed (r/a, t) coordinates into psinorm.
\begin{quote}\begin{description}
\item[{Parameters}] \leavevmode\begin{itemize}
\item {} 
\sphinxstyleliteralstrong{\sphinxupquote{roa}} (\sphinxstyleliteralemphasis{\sphinxupquote{Array-like}}\sphinxstyleliteralemphasis{\sphinxupquote{ or }}\sphinxstyleliteralemphasis{\sphinxupquote{scalar float}}) \textendash{} Values of the normalized minor
radius to map to psinorm.

\item {} 
\sphinxstyleliteralstrong{\sphinxupquote{t}} (\sphinxstyleliteralemphasis{\sphinxupquote{Array-like}}\sphinxstyleliteralemphasis{\sphinxupquote{ or }}\sphinxstyleliteralemphasis{\sphinxupquote{scalar float}}) \textendash{} Times to perform the conversion at.
If \sphinxtitleref{t} is a single value, it is used for all of the elements of
\sphinxtitleref{roa}. If the \sphinxtitleref{each\_t} keyword is True, then \sphinxtitleref{t} must be scalar
or have exactly one dimension. If the \sphinxtitleref{each\_t} keyword is False,
\sphinxtitleref{t} must have the same shape as \sphinxtitleref{roa}.

\end{itemize}

\item[{Keyword Arguments}] \leavevmode\begin{itemize}
\item {} 
\sphinxstyleliteralstrong{\sphinxupquote{sqrt}} (\sphinxstyleliteralemphasis{\sphinxupquote{Boolean}}) \textendash{} Set to True to return the square root of psinorm.
Only the square root of positive values is taken. Negative
values are replaced with zeros, consistent with Steve Wolfe’s
IDL implementation efit\_rz2rho.pro. Default is False.

\item {} 
\sphinxstyleliteralstrong{\sphinxupquote{each\_t}} (\sphinxstyleliteralemphasis{\sphinxupquote{Boolean}}) \textendash{} When True, the elements in \sphinxtitleref{roa} are evaluated
at each value in \sphinxtitleref{t}. If True, \sphinxtitleref{t} must have only one dimension
(or be a scalar). If False, \sphinxtitleref{t} must match the shape of \sphinxtitleref{roa}
or be a scalar. Default is True (evaluate ALL \sphinxtitleref{roa} at EACH
element in \sphinxtitleref{t}).

\item {} 
\sphinxstyleliteralstrong{\sphinxupquote{k}} (\sphinxstyleliteralemphasis{\sphinxupquote{positive int}}) \textendash{} The degree of polynomial spline interpolation to
use in converting coordinates.

\item {} 
\sphinxstyleliteralstrong{\sphinxupquote{return\_t}} (\sphinxstyleliteralemphasis{\sphinxupquote{Boolean}}) \textendash{} Set to True to return a tuple of (\sphinxtitleref{rho},
\sphinxtitleref{time\_idxs}), where \sphinxtitleref{time\_idxs} is the array of time indices
actually used in evaluating \sphinxtitleref{rho} with nearest-neighbor
interpolation. (This is mostly present as an internal helper.)
Default is False (only return \sphinxtitleref{rho}).

\end{itemize}

\item[{Returns}] \leavevmode

\sphinxtitleref{psinorm} or (\sphinxtitleref{psinorm}, \sphinxtitleref{time\_idxs})
\begin{itemize}
\item {} 
\sphinxstylestrong{psinorm} (\sphinxtitleref{Array or scalar float}) - The converted coordinates. If
all of the input arguments are scalar, then a scalar is returned.
Otherwise, a scipy Array is returned.

\item {} 
\sphinxstylestrong{time\_idxs} (Array with same shape as \sphinxtitleref{psinorm}) - The indices
(in \sphinxcode{\sphinxupquote{self.getTimeBase()}}) that were used for
nearest-neighbor interpolation. Only returned if \sphinxtitleref{return\_t} is
True.

\end{itemize}


\end{description}\end{quote}
\subsubsection*{Examples}

All assume that \sphinxtitleref{Eq\_instance} is a valid instance of the appropriate
extension of the {\hyperref[\detokenize{eqtools:eqtools.core.Equilibrium}]{\sphinxcrossref{\sphinxcode{\sphinxupquote{Equilibrium}}}}} abstract class.

Find single psinorm value at r/a=0.6, t=0.26s:

\begin{sphinxVerbatim}[commandchars=\\\{\}]
\PYG{n}{psinorm\PYGZus{}val} \PYG{o}{=} \PYG{n}{Eq\PYGZus{}instance}\PYG{o}{.}\PYG{n}{roa2psinorm}\PYG{p}{(}\PYG{l+m+mf}{0.6}\PYG{p}{,} \PYG{l+m+mf}{0.26}\PYG{p}{)}
\end{sphinxVerbatim}

Find psinorm values at r/a points 0.6 and 0.8 at the
single time t=0.26s.:

\begin{sphinxVerbatim}[commandchars=\\\{\}]
\PYG{n}{psinorm\PYGZus{}arr} \PYG{o}{=} \PYG{n}{Eq\PYGZus{}instance}\PYG{o}{.}\PYG{n}{roa2psinorm}\PYG{p}{(}\PYG{p}{[}\PYG{l+m+mf}{0.6}\PYG{p}{,} \PYG{l+m+mf}{0.8}\PYG{p}{]}\PYG{p}{,} \PYG{l+m+mf}{0.26}\PYG{p}{)}
\end{sphinxVerbatim}

Find psinorm values at r/a of 0.6 at times t={[}0.2s, 0.3s{]}:

\begin{sphinxVerbatim}[commandchars=\\\{\}]
\PYG{n}{psinorm\PYGZus{}arr} \PYG{o}{=} \PYG{n}{Eq\PYGZus{}instance}\PYG{o}{.}\PYG{n}{roa2psinorm}\PYG{p}{(}\PYG{l+m+mf}{0.6}\PYG{p}{,} \PYG{p}{[}\PYG{l+m+mf}{0.2}\PYG{p}{,} \PYG{l+m+mf}{0.3}\PYG{p}{]}\PYG{p}{)}
\end{sphinxVerbatim}

Find psinorm values at (roa, t) points (0.6, 0.2s) and (0.5, 0.3s):

\begin{sphinxVerbatim}[commandchars=\\\{\}]
\PYG{n}{psinorm\PYGZus{}arr} \PYG{o}{=} \PYG{n}{Eq\PYGZus{}instance}\PYG{o}{.}\PYG{n}{roa2psinorm}\PYG{p}{(}\PYG{p}{[}\PYG{l+m+mf}{0.6}\PYG{p}{,} \PYG{l+m+mf}{0.5}\PYG{p}{]}\PYG{p}{,} \PYG{p}{[}\PYG{l+m+mf}{0.2}\PYG{p}{,} \PYG{l+m+mf}{0.3}\PYG{p}{]}\PYG{p}{,} \PYG{n}{each\PYGZus{}t}\PYG{o}{=}\PYG{k+kc}{False}\PYG{p}{)}
\end{sphinxVerbatim}

\end{fulllineitems}

\index{roa2phinorm() (eqtools.core.Equilibrium method)@\spxentry{roa2phinorm()}\spxextra{eqtools.core.Equilibrium method}}

\begin{fulllineitems}
\phantomsection\label{\detokenize{eqtools:eqtools.core.Equilibrium.roa2phinorm}}\pysiglinewithargsret{\sphinxbfcode{\sphinxupquote{roa2phinorm}}}{\emph{*args}, \emph{**kwargs}}{}
Convert the passed (r/a, t) coordinates into phinorm.
\begin{quote}\begin{description}
\item[{Parameters}] \leavevmode\begin{itemize}
\item {} 
\sphinxstyleliteralstrong{\sphinxupquote{roa}} (\sphinxstyleliteralemphasis{\sphinxupquote{Array-like}}\sphinxstyleliteralemphasis{\sphinxupquote{ or }}\sphinxstyleliteralemphasis{\sphinxupquote{scalar float}}) \textendash{} Values of the normalized minor
radius to map to phinorm.

\item {} 
\sphinxstyleliteralstrong{\sphinxupquote{t}} (\sphinxstyleliteralemphasis{\sphinxupquote{Array-like}}\sphinxstyleliteralemphasis{\sphinxupquote{ or }}\sphinxstyleliteralemphasis{\sphinxupquote{scalar float}}) \textendash{} Times to perform the conversion at.
If \sphinxtitleref{t} is a single value, it is used for all of the elements of
\sphinxtitleref{roa}. If the \sphinxtitleref{each\_t} keyword is True, then \sphinxtitleref{t} must be scalar
or have exactly one dimension. If the \sphinxtitleref{each\_t} keyword is False,
\sphinxtitleref{t} must have the same shape as \sphinxtitleref{roa}.

\end{itemize}

\item[{Keyword Arguments}] \leavevmode\begin{itemize}
\item {} 
\sphinxstyleliteralstrong{\sphinxupquote{sqrt}} (\sphinxstyleliteralemphasis{\sphinxupquote{Boolean}}) \textendash{} Set to True to return the square root of phinorm.
Only the square root of positive values is taken. Negative
values are replaced with zeros, consistent with Steve Wolfe’s
IDL implementation efit\_rz2rho.pro. Default is False.

\item {} 
\sphinxstyleliteralstrong{\sphinxupquote{each\_t}} (\sphinxstyleliteralemphasis{\sphinxupquote{Boolean}}) \textendash{} When True, the elements in \sphinxtitleref{roa} are evaluated
at each value in \sphinxtitleref{t}. If True, \sphinxtitleref{t} must have only one dimension
(or be a scalar). If False, \sphinxtitleref{t} must match the shape of \sphinxtitleref{roa}
or be a scalar. Default is True (evaluate ALL \sphinxtitleref{roa} at EACH
element in \sphinxtitleref{t}).

\item {} 
\sphinxstyleliteralstrong{\sphinxupquote{k}} (\sphinxstyleliteralemphasis{\sphinxupquote{positive int}}) \textendash{} The degree of polynomial spline interpolation to
use in converting coordinates.

\item {} 
\sphinxstyleliteralstrong{\sphinxupquote{return\_t}} (\sphinxstyleliteralemphasis{\sphinxupquote{Boolean}}) \textendash{} Set to True to return a tuple of (\sphinxtitleref{rho},
\sphinxtitleref{time\_idxs}), where \sphinxtitleref{time\_idxs} is the array of time indices
actually used in evaluating \sphinxtitleref{rho} with nearest-neighbor
interpolation. (This is mostly present as an internal helper.)
Default is False (only return \sphinxtitleref{rho}).

\end{itemize}

\item[{Returns}] \leavevmode

\sphinxtitleref{phinorm} or (\sphinxtitleref{phinorm}, \sphinxtitleref{time\_idxs})
\begin{itemize}
\item {} 
\sphinxstylestrong{phinorm} (\sphinxtitleref{Array or scalar float}) - The converted coordinates. If
all of the input arguments are scalar, then a scalar is returned.
Otherwise, a scipy Array is returned.

\item {} 
\sphinxstylestrong{time\_idxs} (Array with same shape as \sphinxtitleref{phinorm}) - The indices
(in \sphinxcode{\sphinxupquote{self.getTimeBase()}}) that were used for
nearest-neighbor interpolation. Only returned if \sphinxtitleref{return\_t} is
True.

\end{itemize}


\end{description}\end{quote}
\subsubsection*{Examples}

All assume that \sphinxtitleref{Eq\_instance} is a valid instance of the appropriate
extension of the {\hyperref[\detokenize{eqtools:eqtools.core.Equilibrium}]{\sphinxcrossref{\sphinxcode{\sphinxupquote{Equilibrium}}}}} abstract class.

Find single phinorm value at r/a=0.6, t=0.26s:

\begin{sphinxVerbatim}[commandchars=\\\{\}]
\PYG{n}{phinorm\PYGZus{}val} \PYG{o}{=} \PYG{n}{Eq\PYGZus{}instance}\PYG{o}{.}\PYG{n}{roa2phinorm}\PYG{p}{(}\PYG{l+m+mf}{0.6}\PYG{p}{,} \PYG{l+m+mf}{0.26}\PYG{p}{)}
\end{sphinxVerbatim}

Find phinorm values at r/a points 0.6 and 0.8 at the
single time t=0.26s.:

\begin{sphinxVerbatim}[commandchars=\\\{\}]
\PYG{n}{phinorm\PYGZus{}arr} \PYG{o}{=} \PYG{n}{Eq\PYGZus{}instance}\PYG{o}{.}\PYG{n}{roa2phinorm}\PYG{p}{(}\PYG{p}{[}\PYG{l+m+mf}{0.6}\PYG{p}{,} \PYG{l+m+mf}{0.8}\PYG{p}{]}\PYG{p}{,} \PYG{l+m+mf}{0.26}\PYG{p}{)}
\end{sphinxVerbatim}

Find phinorm values at r/a of 0.6 at times t={[}0.2s, 0.3s{]}:

\begin{sphinxVerbatim}[commandchars=\\\{\}]
\PYG{n}{phinorm\PYGZus{}arr} \PYG{o}{=} \PYG{n}{Eq\PYGZus{}instance}\PYG{o}{.}\PYG{n}{roa2phinorm}\PYG{p}{(}\PYG{l+m+mf}{0.6}\PYG{p}{,} \PYG{p}{[}\PYG{l+m+mf}{0.2}\PYG{p}{,} \PYG{l+m+mf}{0.3}\PYG{p}{]}\PYG{p}{)}
\end{sphinxVerbatim}

Find phinorm values at (roa, t) points (0.6, 0.2s) and (0.5, 0.3s):

\begin{sphinxVerbatim}[commandchars=\\\{\}]
\PYG{n}{phinorm\PYGZus{}arr} \PYG{o}{=} \PYG{n}{Eq\PYGZus{}instance}\PYG{o}{.}\PYG{n}{roa2phinorm}\PYG{p}{(}\PYG{p}{[}\PYG{l+m+mf}{0.6}\PYG{p}{,} \PYG{l+m+mf}{0.5}\PYG{p}{]}\PYG{p}{,} \PYG{p}{[}\PYG{l+m+mf}{0.2}\PYG{p}{,} \PYG{l+m+mf}{0.3}\PYG{p}{]}\PYG{p}{,} \PYG{n}{each\PYGZus{}t}\PYG{o}{=}\PYG{k+kc}{False}\PYG{p}{)}
\end{sphinxVerbatim}

\end{fulllineitems}

\index{roa2volnorm() (eqtools.core.Equilibrium method)@\spxentry{roa2volnorm()}\spxextra{eqtools.core.Equilibrium method}}

\begin{fulllineitems}
\phantomsection\label{\detokenize{eqtools:eqtools.core.Equilibrium.roa2volnorm}}\pysiglinewithargsret{\sphinxbfcode{\sphinxupquote{roa2volnorm}}}{\emph{*args}, \emph{**kwargs}}{}
Convert the passed (r/a, t) coordinates into volnorm.
\begin{quote}\begin{description}
\item[{Parameters}] \leavevmode\begin{itemize}
\item {} 
\sphinxstyleliteralstrong{\sphinxupquote{roa}} (\sphinxstyleliteralemphasis{\sphinxupquote{Array-like}}\sphinxstyleliteralemphasis{\sphinxupquote{ or }}\sphinxstyleliteralemphasis{\sphinxupquote{scalar float}}) \textendash{} Values of the normalized minor
radius to map to volnorm.

\item {} 
\sphinxstyleliteralstrong{\sphinxupquote{t}} (\sphinxstyleliteralemphasis{\sphinxupquote{Array-like}}\sphinxstyleliteralemphasis{\sphinxupquote{ or }}\sphinxstyleliteralemphasis{\sphinxupquote{scalar float}}) \textendash{} Times to perform the conversion at.
If \sphinxtitleref{t} is a single value, it is used for all of the elements of
\sphinxtitleref{roa}. If the \sphinxtitleref{each\_t} keyword is True, then \sphinxtitleref{t} must be scalar
or have exactly one dimension. If the \sphinxtitleref{each\_t} keyword is False,
\sphinxtitleref{t} must have the same shape as \sphinxtitleref{roa}.

\end{itemize}

\item[{Keyword Arguments}] \leavevmode\begin{itemize}
\item {} 
\sphinxstyleliteralstrong{\sphinxupquote{sqrt}} (\sphinxstyleliteralemphasis{\sphinxupquote{Boolean}}) \textendash{} Set to True to return the square root of volnorm.
Only the square root of positive values is taken. Negative
values are replaced with zeros, consistent with Steve Wolfe’s
IDL implementation efit\_rz2rho.pro. Default is False.

\item {} 
\sphinxstyleliteralstrong{\sphinxupquote{each\_t}} (\sphinxstyleliteralemphasis{\sphinxupquote{Boolean}}) \textendash{} When True, the elements in \sphinxtitleref{roa} are evaluated
at each value in \sphinxtitleref{t}. If True, \sphinxtitleref{t} must have only one dimension
(or be a scalar). If False, \sphinxtitleref{t} must match the shape of \sphinxtitleref{roa}
or be a scalar. Default is True (evaluate ALL \sphinxtitleref{roa} at EACH
element in \sphinxtitleref{t}).

\item {} 
\sphinxstyleliteralstrong{\sphinxupquote{k}} (\sphinxstyleliteralemphasis{\sphinxupquote{positive int}}) \textendash{} The degree of polynomial spline interpolation to
use in converting coordinates.

\item {} 
\sphinxstyleliteralstrong{\sphinxupquote{return\_t}} (\sphinxstyleliteralemphasis{\sphinxupquote{Boolean}}) \textendash{} Set to True to return a tuple of (\sphinxtitleref{rho},
\sphinxtitleref{time\_idxs}), where \sphinxtitleref{time\_idxs} is the array of time indices
actually used in evaluating \sphinxtitleref{rho} with nearest-neighbor
interpolation. (This is mostly present as an internal helper.)
Default is False (only return \sphinxtitleref{rho}).

\end{itemize}

\item[{Returns}] \leavevmode

\sphinxtitleref{volnorm} or (\sphinxtitleref{volnorm}, \sphinxtitleref{time\_idxs})
\begin{itemize}
\item {} 
\sphinxstylestrong{volnorm} (\sphinxtitleref{Array or scalar float}) - The converted coordinates. If
all of the input arguments are scalar, then a scalar is returned.
Otherwise, a scipy Array is returned.

\item {} 
\sphinxstylestrong{time\_idxs} (Array with same shape as \sphinxtitleref{volnorm}) - The indices
(in \sphinxcode{\sphinxupquote{self.getTimeBase()}}) that were used for
nearest-neighbor interpolation. Only returned if \sphinxtitleref{return\_t} is
True.

\end{itemize}


\end{description}\end{quote}
\subsubsection*{Examples}

All assume that \sphinxtitleref{Eq\_instance} is a valid instance of the appropriate
extension of the {\hyperref[\detokenize{eqtools:eqtools.core.Equilibrium}]{\sphinxcrossref{\sphinxcode{\sphinxupquote{Equilibrium}}}}} abstract class.

Find single volnorm value at r/a=0.6, t=0.26s:

\begin{sphinxVerbatim}[commandchars=\\\{\}]
\PYG{n}{volnorm\PYGZus{}val} \PYG{o}{=} \PYG{n}{Eq\PYGZus{}instance}\PYG{o}{.}\PYG{n}{roa2volnorm}\PYG{p}{(}\PYG{l+m+mf}{0.6}\PYG{p}{,} \PYG{l+m+mf}{0.26}\PYG{p}{)}
\end{sphinxVerbatim}

Find volnorm values at r/a points 0.6 and 0.8 at the
single time t=0.26s.:

\begin{sphinxVerbatim}[commandchars=\\\{\}]
\PYG{n}{volnorm\PYGZus{}arr} \PYG{o}{=} \PYG{n}{Eq\PYGZus{}instance}\PYG{o}{.}\PYG{n}{roa2volnorm}\PYG{p}{(}\PYG{p}{[}\PYG{l+m+mf}{0.6}\PYG{p}{,} \PYG{l+m+mf}{0.8}\PYG{p}{]}\PYG{p}{,} \PYG{l+m+mf}{0.26}\PYG{p}{)}
\end{sphinxVerbatim}

Find volnorm values at r/a of 0.6 at times t={[}0.2s, 0.3s{]}:

\begin{sphinxVerbatim}[commandchars=\\\{\}]
\PYG{n}{volnorm\PYGZus{}arr} \PYG{o}{=} \PYG{n}{Eq\PYGZus{}instance}\PYG{o}{.}\PYG{n}{roa2volnorm}\PYG{p}{(}\PYG{l+m+mf}{0.6}\PYG{p}{,} \PYG{p}{[}\PYG{l+m+mf}{0.2}\PYG{p}{,} \PYG{l+m+mf}{0.3}\PYG{p}{]}\PYG{p}{)}
\end{sphinxVerbatim}

Find volnorm values at (roa, t) points (0.6, 0.2s) and (0.5, 0.3s):

\begin{sphinxVerbatim}[commandchars=\\\{\}]
\PYG{n}{volnorm\PYGZus{}arr} \PYG{o}{=} \PYG{n}{Eq\PYGZus{}instance}\PYG{o}{.}\PYG{n}{roa2volnorm}\PYG{p}{(}\PYG{p}{[}\PYG{l+m+mf}{0.6}\PYG{p}{,} \PYG{l+m+mf}{0.5}\PYG{p}{]}\PYG{p}{,} \PYG{p}{[}\PYG{l+m+mf}{0.2}\PYG{p}{,} \PYG{l+m+mf}{0.3}\PYG{p}{]}\PYG{p}{,} \PYG{n}{each\PYGZus{}t}\PYG{o}{=}\PYG{k+kc}{False}\PYG{p}{)}
\end{sphinxVerbatim}

\end{fulllineitems}

\index{roa2rho() (eqtools.core.Equilibrium method)@\spxentry{roa2rho()}\spxextra{eqtools.core.Equilibrium method}}

\begin{fulllineitems}
\phantomsection\label{\detokenize{eqtools:eqtools.core.Equilibrium.roa2rho}}\pysiglinewithargsret{\sphinxbfcode{\sphinxupquote{roa2rho}}}{\emph{method}, \emph{*args}, \emph{**kwargs}}{}
Convert the passed (r/a, t) coordinates into one of several coordinates.
\begin{quote}\begin{description}
\item[{Parameters}] \leavevmode\begin{itemize}
\item {} 
\sphinxstyleliteralstrong{\sphinxupquote{method}} (\sphinxstyleliteralemphasis{\sphinxupquote{String}}) \textendash{} 
Indicates which coordinates to convert to.
Valid options are:
\begin{quote}


\begin{savenotes}\sphinxattablestart
\centering
\begin{tabulary}{\linewidth}[t]{|T|T|}
\hline

psinorm
&
Normalized poloidal flux
\\
\hline
phinorm
&
Normalized toroidal flux
\\
\hline
volnorm
&
Normalized volume
\\
\hline
Rmid
&
Midplane major radius
\\
\hline
q
&
Safety factor
\\
\hline
F
&
Flux function \(F=RB_{\phi}\)
\\
\hline
FFPrime
&
Flux function \(FF'\)
\\
\hline
p
&
Pressure
\\
\hline
pprime
&
Pressure gradient
\\
\hline
v
&
Flux surface volume
\\
\hline
\end{tabulary}
\par
\sphinxattableend\end{savenotes}
\end{quote}

Additionally, each valid option may be prepended with ‘sqrt’
to specify the square root of the desired unit.


\item {} 
\sphinxstyleliteralstrong{\sphinxupquote{roa}} (\sphinxstyleliteralemphasis{\sphinxupquote{Array-like}}\sphinxstyleliteralemphasis{\sphinxupquote{ or }}\sphinxstyleliteralemphasis{\sphinxupquote{scalar float}}) \textendash{} Values of the normalized minor
radius to map to rho.

\item {} 
\sphinxstyleliteralstrong{\sphinxupquote{t}} (\sphinxstyleliteralemphasis{\sphinxupquote{Array-like}}\sphinxstyleliteralemphasis{\sphinxupquote{ or }}\sphinxstyleliteralemphasis{\sphinxupquote{scalar float}}) \textendash{} Times to perform the conversion at.
If \sphinxtitleref{t} is a single value, it is used for all of the elements of
\sphinxtitleref{roa}. If the \sphinxtitleref{each\_t} keyword is True, then \sphinxtitleref{t} must be scalar
or have exactly one dimension. If the \sphinxtitleref{each\_t} keyword is False,
\sphinxtitleref{t} must have the same shape as \sphinxtitleref{roa}.

\end{itemize}

\item[{Keyword Arguments}] \leavevmode\begin{itemize}
\item {} 
\sphinxstyleliteralstrong{\sphinxupquote{sqrt}} (\sphinxstyleliteralemphasis{\sphinxupquote{Boolean}}) \textendash{} Set to True to return the square root of rho.
Only the square root of positive values is taken. Negative
values are replaced with zeros, consistent with Steve Wolfe’s
IDL implementation efit\_rz2rho.pro. Default is False.

\item {} 
\sphinxstyleliteralstrong{\sphinxupquote{each\_t}} (\sphinxstyleliteralemphasis{\sphinxupquote{Boolean}}) \textendash{} When True, the elements in \sphinxtitleref{roa} are evaluated
at each value in \sphinxtitleref{t}. If True, \sphinxtitleref{t} must have only one dimension
(or be a scalar). If False, \sphinxtitleref{t} must match the shape of \sphinxtitleref{roa}
or be a scalar. Default is True (evaluate ALL \sphinxtitleref{roa} at EACH
element in \sphinxtitleref{t}).

\item {} 
\sphinxstyleliteralstrong{\sphinxupquote{length\_unit}} (\sphinxstyleliteralemphasis{\sphinxupquote{String}}\sphinxstyleliteralemphasis{\sphinxupquote{ or }}\sphinxstyleliteralemphasis{\sphinxupquote{1}}) \textendash{} 
Length unit that \sphinxtitleref{Rmid} is returned in.
If a string is given, it must be a valid unit specifier:
\begin{quote}


\begin{savenotes}\sphinxattablestart
\centering
\begin{tabulary}{\linewidth}[t]{|T|T|}
\hline

’m’
&
meters
\\
\hline
’cm’
&
centimeters
\\
\hline
’mm’
&
millimeters
\\
\hline
’in’
&
inches
\\
\hline
’ft’
&
feet
\\
\hline
’yd’
&
yards
\\
\hline
’smoot’
&
smoots
\\
\hline
’cubit’
&
cubits
\\
\hline
’hand’
&
hands
\\
\hline
’default’
&
meters
\\
\hline
\end{tabulary}
\par
\sphinxattableend\end{savenotes}
\end{quote}

If length\_unit is 1 or None, meters are assumed. The default
value is 1 (use meters).


\item {} 
\sphinxstyleliteralstrong{\sphinxupquote{k}} (\sphinxstyleliteralemphasis{\sphinxupquote{positive int}}) \textendash{} The degree of polynomial spline interpolation to
use in converting coordinates.

\item {} 
\sphinxstyleliteralstrong{\sphinxupquote{return\_t}} (\sphinxstyleliteralemphasis{\sphinxupquote{Boolean}}) \textendash{} Set to True to return a tuple of (\sphinxtitleref{rho},
\sphinxtitleref{time\_idxs}), where \sphinxtitleref{time\_idxs} is the array of time indices
actually used in evaluating \sphinxtitleref{rho} with nearest-neighbor
interpolation. (This is mostly present as an internal helper.)
Default is False (only return \sphinxtitleref{rho}).

\end{itemize}

\item[{Returns}] \leavevmode

\sphinxtitleref{rho} or (\sphinxtitleref{rho}, \sphinxtitleref{time\_idxs})
\begin{itemize}
\item {} 
\sphinxstylestrong{rho} (\sphinxtitleref{Array or scalar float}) - The converted coordinates. If
all of the input arguments are scalar, then a scalar is returned.
Otherwise, a scipy Array is returned.

\item {} 
\sphinxstylestrong{time\_idxs} (Array with same shape as \sphinxtitleref{rho}) - The indices
(in \sphinxcode{\sphinxupquote{self.getTimeBase()}}) that were used for
nearest-neighbor interpolation. Only returned if \sphinxtitleref{return\_t} is
True.

\end{itemize}


\end{description}\end{quote}
\subsubsection*{Examples}

All assume that \sphinxtitleref{Eq\_instance} is a valid instance of the appropriate
extension of the {\hyperref[\detokenize{eqtools:eqtools.core.Equilibrium}]{\sphinxcrossref{\sphinxcode{\sphinxupquote{Equilibrium}}}}} abstract class.

Find single psinorm value at r/a=0.6, t=0.26s:

\begin{sphinxVerbatim}[commandchars=\\\{\}]
\PYG{n}{psi\PYGZus{}val} \PYG{o}{=} \PYG{n}{Eq\PYGZus{}instance}\PYG{o}{.}\PYG{n}{roa2rho}\PYG{p}{(}\PYG{l+s+s1}{\PYGZsq{}}\PYG{l+s+s1}{psinorm}\PYG{l+s+s1}{\PYGZsq{}}\PYG{p}{,} \PYG{l+m+mf}{0.6}\PYG{p}{,} \PYG{l+m+mf}{0.26}\PYG{p}{)}
\end{sphinxVerbatim}

Find psinorm values at r/a points 0.6 and 0.8 at the
single time t=0.26s:

\begin{sphinxVerbatim}[commandchars=\\\{\}]
\PYG{n}{psi\PYGZus{}arr} \PYG{o}{=} \PYG{n}{Eq\PYGZus{}instance}\PYG{o}{.}\PYG{n}{roa2rho}\PYG{p}{(}\PYG{l+s+s1}{\PYGZsq{}}\PYG{l+s+s1}{psinorm}\PYG{l+s+s1}{\PYGZsq{}}\PYG{p}{,} \PYG{p}{[}\PYG{l+m+mf}{0.6}\PYG{p}{,} \PYG{l+m+mf}{0.8}\PYG{p}{]}\PYG{p}{,} \PYG{l+m+mf}{0.26}\PYG{p}{)}
\end{sphinxVerbatim}

Find psinorm values at r/a of 0.6 at times t={[}0.2s, 0.3s{]}:

\begin{sphinxVerbatim}[commandchars=\\\{\}]
\PYG{n}{psi\PYGZus{}arr} \PYG{o}{=} \PYG{n}{Eq\PYGZus{}instance}\PYG{o}{.}\PYG{n}{roa2rho}\PYG{p}{(}\PYG{l+s+s1}{\PYGZsq{}}\PYG{l+s+s1}{psinorm}\PYG{l+s+s1}{\PYGZsq{}}\PYG{p}{,} \PYG{l+m+mf}{0.6}\PYG{p}{,} \PYG{p}{[}\PYG{l+m+mf}{0.2}\PYG{p}{,} \PYG{l+m+mf}{0.3}\PYG{p}{]}\PYG{p}{)}
\end{sphinxVerbatim}

Find psinorm values at (r/a, t) points (0.6, 0.2s) and (0.5, 0.3s):

\begin{sphinxVerbatim}[commandchars=\\\{\}]
\PYG{n}{psi\PYGZus{}arr} \PYG{o}{=} \PYG{n}{Eq\PYGZus{}instance}\PYG{o}{.}\PYG{n}{roa2rho}\PYG{p}{(}\PYG{l+s+s1}{\PYGZsq{}}\PYG{l+s+s1}{psinorm}\PYG{l+s+s1}{\PYGZsq{}}\PYG{p}{,} \PYG{p}{[}\PYG{l+m+mf}{0.6}\PYG{p}{,} \PYG{l+m+mf}{0.5}\PYG{p}{]}\PYG{p}{,} \PYG{p}{[}\PYG{l+m+mf}{0.2}\PYG{p}{,} \PYG{l+m+mf}{0.3}\PYG{p}{]}\PYG{p}{,} \PYG{n}{each\PYGZus{}t}\PYG{o}{=}\PYG{k+kc}{False}\PYG{p}{)}
\end{sphinxVerbatim}

\end{fulllineitems}

\index{psinorm2rmid() (eqtools.core.Equilibrium method)@\spxentry{psinorm2rmid()}\spxextra{eqtools.core.Equilibrium method}}

\begin{fulllineitems}
\phantomsection\label{\detokenize{eqtools:eqtools.core.Equilibrium.psinorm2rmid}}\pysiglinewithargsret{\sphinxbfcode{\sphinxupquote{psinorm2rmid}}}{\emph{psi\_norm}, \emph{t}, \emph{**kwargs}}{}
Calculates the outboard R\_mid location corresponding to the passed psinorm (normalized poloidal flux) values.
\begin{quote}\begin{description}
\item[{Parameters}] \leavevmode\begin{itemize}
\item {} 
\sphinxstyleliteralstrong{\sphinxupquote{psi\_norm}} (\sphinxstyleliteralemphasis{\sphinxupquote{Array-like}}\sphinxstyleliteralemphasis{\sphinxupquote{ or }}\sphinxstyleliteralemphasis{\sphinxupquote{scalar float}}) \textendash{} Values of the normalized
poloidal flux to map to Rmid.

\item {} 
\sphinxstyleliteralstrong{\sphinxupquote{t}} (\sphinxstyleliteralemphasis{\sphinxupquote{Array-like}}\sphinxstyleliteralemphasis{\sphinxupquote{ or }}\sphinxstyleliteralemphasis{\sphinxupquote{scalar float}}) \textendash{} Times to perform the conversion at.
If \sphinxtitleref{t} is a single value, it is used for all of the elements of
\sphinxtitleref{psi\_norm}. If the \sphinxtitleref{each\_t} keyword is True, then \sphinxtitleref{t} must be scalar
or have exactly one dimension. If the \sphinxtitleref{each\_t} keyword is False,
\sphinxtitleref{t} must have the same shape as \sphinxtitleref{psi\_norm}.

\end{itemize}

\item[{Keyword Arguments}] \leavevmode\begin{itemize}
\item {} 
\sphinxstyleliteralstrong{\sphinxupquote{sqrt}} (\sphinxstyleliteralemphasis{\sphinxupquote{Boolean}}) \textendash{} Set to True to return the square root of Rmid. Only
the square root of positive values is taken. Negative values are
replaced with zeros, consistent with Steve Wolfe’s IDL
implementation efit\_rz2rho.pro. Default is False.

\item {} 
\sphinxstyleliteralstrong{\sphinxupquote{each\_t}} (\sphinxstyleliteralemphasis{\sphinxupquote{Boolean}}) \textendash{} When True, the elements in \sphinxtitleref{psi\_norm} are evaluated at
each value in \sphinxtitleref{t}. If True, \sphinxtitleref{t} must have only one dimension (or
be a scalar). If False, \sphinxtitleref{t} must match the shape of \sphinxtitleref{psi\_norm} or be
a scalar. Default is True (evaluate ALL \sphinxtitleref{psi\_norm} at EACH element in
\sphinxtitleref{t}).

\item {} 
\sphinxstyleliteralstrong{\sphinxupquote{rho}} (\sphinxstyleliteralemphasis{\sphinxupquote{Boolean}}) \textendash{} Set to True to return r/a (normalized minor radius)
instead of Rmid. Default is False (return major radius, Rmid).

\item {} 
\sphinxstyleliteralstrong{\sphinxupquote{length\_unit}} (\sphinxstyleliteralemphasis{\sphinxupquote{String}}\sphinxstyleliteralemphasis{\sphinxupquote{ or }}\sphinxstyleliteralemphasis{\sphinxupquote{1}}) \textendash{} 
Length unit that \sphinxtitleref{Rmid} is returned in.
If a string is given, it must be a valid unit specifier:
\begin{quote}


\begin{savenotes}\sphinxattablestart
\centering
\begin{tabulary}{\linewidth}[t]{|T|T|}
\hline

’m’
&
meters
\\
\hline
’cm’
&
centimeters
\\
\hline
’mm’
&
millimeters
\\
\hline
’in’
&
inches
\\
\hline
’ft’
&
feet
\\
\hline
’yd’
&
yards
\\
\hline
’smoot’
&
smoots
\\
\hline
’cubit’
&
cubits
\\
\hline
’hand’
&
hands
\\
\hline
’default’
&
meters
\\
\hline
\end{tabulary}
\par
\sphinxattableend\end{savenotes}
\end{quote}

If length\_unit is 1 or None, meters are assumed. The default
value is 1 (use meters).


\item {} 
\sphinxstyleliteralstrong{\sphinxupquote{k}} (\sphinxstyleliteralemphasis{\sphinxupquote{positive int}}) \textendash{} The degree of polynomial spline interpolation to
use in converting coordinates.

\item {} 
\sphinxstyleliteralstrong{\sphinxupquote{return\_t}} (\sphinxstyleliteralemphasis{\sphinxupquote{Boolean}}) \textendash{} Set to True to return a tuple of (\sphinxtitleref{rho},
\sphinxtitleref{time\_idxs}), where \sphinxtitleref{time\_idxs} is the array of time indices
actually used in evaluating \sphinxtitleref{rho} with nearest-neighbor
interpolation. (This is mostly present as an internal helper.)
Default is False (only return \sphinxtitleref{rho}).

\end{itemize}

\item[{Returns}] \leavevmode

\sphinxtitleref{Rmid} or (\sphinxtitleref{Rmid}, \sphinxtitleref{time\_idxs})
\begin{itemize}
\item {} 
\sphinxstylestrong{Rmid} (\sphinxtitleref{Array or scalar float}) - The converted coordinates. If
all of the input arguments are scalar, then a scalar is returned.
Otherwise, a scipy Array is returned.

\item {} 
\sphinxstylestrong{time\_idxs} (Array with same shape as \sphinxtitleref{Rmid}) - The indices
(in \sphinxcode{\sphinxupquote{self.getTimeBase()}}) that were used for
nearest-neighbor interpolation. Only returned if \sphinxtitleref{return\_t} is
True.

\end{itemize}


\end{description}\end{quote}
\subsubsection*{Examples}

All assume that \sphinxtitleref{Eq\_instance} is a valid instance of the appropriate
extension of the {\hyperref[\detokenize{eqtools:eqtools.core.Equilibrium}]{\sphinxcrossref{\sphinxcode{\sphinxupquote{Equilibrium}}}}} abstract class.

Find single R\_mid value for psinorm=0.7, t=0.26s:

\begin{sphinxVerbatim}[commandchars=\\\{\}]
\PYG{n}{R\PYGZus{}mid\PYGZus{}val} \PYG{o}{=} \PYG{n}{Eq\PYGZus{}instance}\PYG{o}{.}\PYG{n}{psinorm2rmid}\PYG{p}{(}\PYG{l+m+mf}{0.7}\PYG{p}{,} \PYG{l+m+mf}{0.26}\PYG{p}{)}
\end{sphinxVerbatim}

Find R\_mid values at psi\_norm values of 0.5 and 0.7 at the single time
t=0.26s:

\begin{sphinxVerbatim}[commandchars=\\\{\}]
\PYG{n}{R\PYGZus{}mid\PYGZus{}arr} \PYG{o}{=} \PYG{n}{Eq\PYGZus{}instance}\PYG{o}{.}\PYG{n}{psinorm2rmid}\PYG{p}{(}\PYG{p}{[}\PYG{l+m+mf}{0.5}\PYG{p}{,} \PYG{l+m+mf}{0.7}\PYG{p}{]}\PYG{p}{,} \PYG{l+m+mf}{0.26}\PYG{p}{)}
\end{sphinxVerbatim}

Find R\_mid values at psi\_norm=0.5 at times t={[}0.2s, 0.3s{]}:

\begin{sphinxVerbatim}[commandchars=\\\{\}]
\PYG{n}{R\PYGZus{}mid\PYGZus{}arr} \PYG{o}{=} \PYG{n}{Eq\PYGZus{}instance}\PYG{o}{.}\PYG{n}{psinorm2rmid}\PYG{p}{(}\PYG{l+m+mf}{0.5}\PYG{p}{,} \PYG{p}{[}\PYG{l+m+mf}{0.2}\PYG{p}{,} \PYG{l+m+mf}{0.3}\PYG{p}{]}\PYG{p}{)}
\end{sphinxVerbatim}

Find R\_mid values at (psinorm, t) points (0.6, 0.2s) and (0.5, 0.3s):

\begin{sphinxVerbatim}[commandchars=\\\{\}]
\PYG{n}{R\PYGZus{}mid\PYGZus{}arr} \PYG{o}{=} \PYG{n}{Eq\PYGZus{}instance}\PYG{o}{.}\PYG{n}{psinorm2rmid}\PYG{p}{(}\PYG{p}{[}\PYG{l+m+mf}{0.6}\PYG{p}{,} \PYG{l+m+mf}{0.5}\PYG{p}{]}\PYG{p}{,} \PYG{p}{[}\PYG{l+m+mf}{0.2}\PYG{p}{,} \PYG{l+m+mf}{0.3}\PYG{p}{]}\PYG{p}{,} \PYG{n}{each\PYGZus{}t}\PYG{o}{=}\PYG{k+kc}{False}\PYG{p}{)}
\end{sphinxVerbatim}

\end{fulllineitems}

\index{psinorm2roa() (eqtools.core.Equilibrium method)@\spxentry{psinorm2roa()}\spxextra{eqtools.core.Equilibrium method}}

\begin{fulllineitems}
\phantomsection\label{\detokenize{eqtools:eqtools.core.Equilibrium.psinorm2roa}}\pysiglinewithargsret{\sphinxbfcode{\sphinxupquote{psinorm2roa}}}{\emph{psi\_norm}, \emph{t}, \emph{**kwargs}}{}
Calculates the normalized minor radius location corresponding to the passed psi\_norm (normalized poloidal flux) values.
\begin{quote}\begin{description}
\item[{Parameters}] \leavevmode\begin{itemize}
\item {} 
\sphinxstyleliteralstrong{\sphinxupquote{psi\_norm}} (\sphinxstyleliteralemphasis{\sphinxupquote{Array-like}}\sphinxstyleliteralemphasis{\sphinxupquote{ or }}\sphinxstyleliteralemphasis{\sphinxupquote{scalar float}}) \textendash{} Values of the normalized
poloidal flux to map to r/a.

\item {} 
\sphinxstyleliteralstrong{\sphinxupquote{t}} (\sphinxstyleliteralemphasis{\sphinxupquote{Array-like}}\sphinxstyleliteralemphasis{\sphinxupquote{ or }}\sphinxstyleliteralemphasis{\sphinxupquote{scalar float}}) \textendash{} Times to perform the conversion at.
If \sphinxtitleref{t} is a single value, it is used for all of the elements of
\sphinxtitleref{psi\_norm}. If the \sphinxtitleref{each\_t} keyword is True, then \sphinxtitleref{t} must be scalar
or have exactly one dimension. If the \sphinxtitleref{each\_t} keyword is False,
\sphinxtitleref{t} must have the same shape as \sphinxtitleref{psi\_norm}.

\end{itemize}

\item[{Keyword Arguments}] \leavevmode\begin{itemize}
\item {} 
\sphinxstyleliteralstrong{\sphinxupquote{sqrt}} (\sphinxstyleliteralemphasis{\sphinxupquote{Boolean}}) \textendash{} Set to True to return the square root of r/a. Only
the square root of positive values is taken. Negative values are
replaced with zeros, consistent with Steve Wolfe’s IDL
implementation efit\_rz2rho.pro. Default is False.

\item {} 
\sphinxstyleliteralstrong{\sphinxupquote{each\_t}} (\sphinxstyleliteralemphasis{\sphinxupquote{Boolean}}) \textendash{} When True, the elements in \sphinxtitleref{psi\_norm} are evaluated at
each value in \sphinxtitleref{t}. If True, \sphinxtitleref{t} must have only one dimension (or
be a scalar). If False, \sphinxtitleref{t} must match the shape of \sphinxtitleref{psi\_norm} or be
a scalar. Default is True (evaluate ALL \sphinxtitleref{psi\_norm} at EACH element in
\sphinxtitleref{t}).

\item {} 
\sphinxstyleliteralstrong{\sphinxupquote{k}} (\sphinxstyleliteralemphasis{\sphinxupquote{positive int}}) \textendash{} The degree of polynomial spline interpolation to
use in converting coordinates.

\item {} 
\sphinxstyleliteralstrong{\sphinxupquote{return\_t}} (\sphinxstyleliteralemphasis{\sphinxupquote{Boolean}}) \textendash{} Set to True to return a tuple of (\sphinxtitleref{rho},
\sphinxtitleref{time\_idxs}), where \sphinxtitleref{time\_idxs} is the array of time indices
actually used in evaluating \sphinxtitleref{rho} with nearest-neighbor
interpolation. (This is mostly present as an internal helper.)
Default is False (only return \sphinxtitleref{rho}).

\end{itemize}

\item[{Returns}] \leavevmode

\sphinxtitleref{roa} or (\sphinxtitleref{roa}, \sphinxtitleref{time\_idxs})
\begin{itemize}
\item {} 
\sphinxstylestrong{roa} (\sphinxtitleref{Array or scalar float}) - Normalized midplane minor
radius. If all of the input arguments are scalar, then a scalar
is returned. Otherwise, a scipy Array is returned.

\item {} 
\sphinxstylestrong{time\_idxs} (Array with same shape as \sphinxtitleref{roa}) - The indices
(in \sphinxcode{\sphinxupquote{self.getTimeBase()}}) that were used for
nearest-neighbor interpolation. Only returned if \sphinxtitleref{return\_t} is
True.

\end{itemize}


\end{description}\end{quote}
\subsubsection*{Examples}

All assume that \sphinxtitleref{Eq\_instance} is a valid instance of the appropriate
extension of the {\hyperref[\detokenize{eqtools:eqtools.core.Equilibrium}]{\sphinxcrossref{\sphinxcode{\sphinxupquote{Equilibrium}}}}} abstract class.

Find single r/a value for psinorm=0.7, t=0.26s:

\begin{sphinxVerbatim}[commandchars=\\\{\}]
\PYG{n}{roa\PYGZus{}val} \PYG{o}{=} \PYG{n}{Eq\PYGZus{}instance}\PYG{o}{.}\PYG{n}{psinorm2roa}\PYG{p}{(}\PYG{l+m+mf}{0.7}\PYG{p}{,} \PYG{l+m+mf}{0.26}\PYG{p}{)}
\end{sphinxVerbatim}

Find r/a values at psi\_norm values of 0.5 and 0.7 at the single time
t=0.26s:

\begin{sphinxVerbatim}[commandchars=\\\{\}]
\PYG{n}{roa\PYGZus{}arr} \PYG{o}{=} \PYG{n}{Eq\PYGZus{}instance}\PYG{o}{.}\PYG{n}{psinorm2roa}\PYG{p}{(}\PYG{p}{[}\PYG{l+m+mf}{0.5}\PYG{p}{,} \PYG{l+m+mf}{0.7}\PYG{p}{]}\PYG{p}{,} \PYG{l+m+mf}{0.26}\PYG{p}{)}
\end{sphinxVerbatim}

Find r/a values at psi\_norm=0.5 at times t={[}0.2s, 0.3s{]}:

\begin{sphinxVerbatim}[commandchars=\\\{\}]
\PYG{n}{roa\PYGZus{}arr} \PYG{o}{=} \PYG{n}{Eq\PYGZus{}instance}\PYG{o}{.}\PYG{n}{psinorm2roa}\PYG{p}{(}\PYG{l+m+mf}{0.5}\PYG{p}{,} \PYG{p}{[}\PYG{l+m+mf}{0.2}\PYG{p}{,} \PYG{l+m+mf}{0.3}\PYG{p}{]}\PYG{p}{)}
\end{sphinxVerbatim}

Find r/a values at (psinorm, t) points (0.6, 0.2s) and (0.5, 0.3s):

\begin{sphinxVerbatim}[commandchars=\\\{\}]
\PYG{n}{roa\PYGZus{}arr} \PYG{o}{=} \PYG{n}{Eq\PYGZus{}instance}\PYG{o}{.}\PYG{n}{psinorm2roa}\PYG{p}{(}\PYG{p}{[}\PYG{l+m+mf}{0.6}\PYG{p}{,} \PYG{l+m+mf}{0.5}\PYG{p}{]}\PYG{p}{,} \PYG{p}{[}\PYG{l+m+mf}{0.2}\PYG{p}{,} \PYG{l+m+mf}{0.3}\PYG{p}{]}\PYG{p}{,} \PYG{n}{each\PYGZus{}t}\PYG{o}{=}\PYG{k+kc}{False}\PYG{p}{)}
\end{sphinxVerbatim}

\end{fulllineitems}

\index{psinorm2volnorm() (eqtools.core.Equilibrium method)@\spxentry{psinorm2volnorm()}\spxextra{eqtools.core.Equilibrium method}}

\begin{fulllineitems}
\phantomsection\label{\detokenize{eqtools:eqtools.core.Equilibrium.psinorm2volnorm}}\pysiglinewithargsret{\sphinxbfcode{\sphinxupquote{psinorm2volnorm}}}{\emph{psi\_norm}, \emph{t}, \emph{**kwargs}}{}
Calculates the normalized volume corresponding to the passed psi\_norm (normalized poloidal flux) values.
\begin{quote}\begin{description}
\item[{Parameters}] \leavevmode\begin{itemize}
\item {} 
\sphinxstyleliteralstrong{\sphinxupquote{psi\_norm}} (\sphinxstyleliteralemphasis{\sphinxupquote{Array-like}}\sphinxstyleliteralemphasis{\sphinxupquote{ or }}\sphinxstyleliteralemphasis{\sphinxupquote{scalar float}}) \textendash{} Values of the normalized
poloidal flux to map to volnorm.

\item {} 
\sphinxstyleliteralstrong{\sphinxupquote{t}} (\sphinxstyleliteralemphasis{\sphinxupquote{Array-like}}\sphinxstyleliteralemphasis{\sphinxupquote{ or }}\sphinxstyleliteralemphasis{\sphinxupquote{scalar float}}) \textendash{} Times to perform the conversion at.
If \sphinxtitleref{t} is a single value, it is used for all of the elements of
\sphinxtitleref{psi\_norm}. If the \sphinxtitleref{each\_t} keyword is True, then \sphinxtitleref{t} must be scalar
or have exactly one dimension. If the \sphinxtitleref{each\_t} keyword is False,
\sphinxtitleref{t} must have the same shape as \sphinxtitleref{psi\_norm}.

\end{itemize}

\item[{Keyword Arguments}] \leavevmode\begin{itemize}
\item {} 
\sphinxstyleliteralstrong{\sphinxupquote{sqrt}} (\sphinxstyleliteralemphasis{\sphinxupquote{Boolean}}) \textendash{} Set to True to return the square root of volnorm. Only
the square root of positive values is taken. Negative values are
replaced with zeros, consistent with Steve Wolfe’s IDL
implementation efit\_rz2rho.pro. Default is False.

\item {} 
\sphinxstyleliteralstrong{\sphinxupquote{each\_t}} (\sphinxstyleliteralemphasis{\sphinxupquote{Boolean}}) \textendash{} When True, the elements in \sphinxtitleref{psi\_norm} are evaluated at
each value in \sphinxtitleref{t}. If True, \sphinxtitleref{t} must have only one dimension (or
be a scalar). If False, \sphinxtitleref{t} must match the shape of \sphinxtitleref{psi\_norm} or be
a scalar. Default is True (evaluate ALL \sphinxtitleref{psi\_norm} at EACH element in
\sphinxtitleref{t}).

\item {} 
\sphinxstyleliteralstrong{\sphinxupquote{k}} (\sphinxstyleliteralemphasis{\sphinxupquote{positive int}}) \textendash{} The degree of polynomial spline interpolation to
use in converting coordinates.

\item {} 
\sphinxstyleliteralstrong{\sphinxupquote{return\_t}} (\sphinxstyleliteralemphasis{\sphinxupquote{Boolean}}) \textendash{} Set to True to return a tuple of (\sphinxtitleref{rho},
\sphinxtitleref{time\_idxs}), where \sphinxtitleref{time\_idxs} is the array of time indices
actually used in evaluating \sphinxtitleref{rho} with nearest-neighbor
interpolation. (This is mostly present as an internal helper.)
Default is False (only return \sphinxtitleref{rho}).

\end{itemize}

\item[{Returns}] \leavevmode

\sphinxtitleref{volnorm} or (\sphinxtitleref{volnorm}, \sphinxtitleref{time\_idxs})
\begin{itemize}
\item {} 
\sphinxstylestrong{volnorm} (\sphinxtitleref{Array or scalar float}) - The converted coordinates. If
all of the input arguments are scalar, then a scalar is returned.
Otherwise, a scipy Array is returned.

\item {} 
\sphinxstylestrong{time\_idxs} (Array with same shape as \sphinxtitleref{volnorm}) - The indices
(in \sphinxcode{\sphinxupquote{self.getTimeBase()}}) that were used for
nearest-neighbor interpolation. Only returned if \sphinxtitleref{return\_t} is
True.

\end{itemize}


\end{description}\end{quote}
\subsubsection*{Examples}

All assume that \sphinxtitleref{Eq\_instance} is a valid instance of the appropriate
extension of the {\hyperref[\detokenize{eqtools:eqtools.core.Equilibrium}]{\sphinxcrossref{\sphinxcode{\sphinxupquote{Equilibrium}}}}} abstract class.

Find single volnorm value for psinorm=0.7, t=0.26s:

\begin{sphinxVerbatim}[commandchars=\\\{\}]
\PYG{n}{volnorm\PYGZus{}val} \PYG{o}{=} \PYG{n}{Eq\PYGZus{}instance}\PYG{o}{.}\PYG{n}{psinorm2volnorm}\PYG{p}{(}\PYG{l+m+mf}{0.7}\PYG{p}{,} \PYG{l+m+mf}{0.26}\PYG{p}{)}
\end{sphinxVerbatim}

Find volnorm values at psi\_norm values of 0.5 and 0.7 at the single time
t=0.26s:

\begin{sphinxVerbatim}[commandchars=\\\{\}]
\PYG{n}{volnorm\PYGZus{}arr} \PYG{o}{=} \PYG{n}{Eq\PYGZus{}instance}\PYG{o}{.}\PYG{n}{psinorm2volnorm}\PYG{p}{(}\PYG{p}{[}\PYG{l+m+mf}{0.5}\PYG{p}{,} \PYG{l+m+mf}{0.7}\PYG{p}{]}\PYG{p}{,} \PYG{l+m+mf}{0.26}\PYG{p}{)}
\end{sphinxVerbatim}

Find volnorm values at psi\_norm=0.5 at times t={[}0.2s, 0.3s{]}:

\begin{sphinxVerbatim}[commandchars=\\\{\}]
\PYG{n}{volnorm\PYGZus{}arr} \PYG{o}{=} \PYG{n}{Eq\PYGZus{}instance}\PYG{o}{.}\PYG{n}{psinorm2volnorm}\PYG{p}{(}\PYG{l+m+mf}{0.5}\PYG{p}{,} \PYG{p}{[}\PYG{l+m+mf}{0.2}\PYG{p}{,} \PYG{l+m+mf}{0.3}\PYG{p}{]}\PYG{p}{)}
\end{sphinxVerbatim}

Find volnorm values at (psinorm, t) points (0.6, 0.2s) and (0.5, 0.3s):

\begin{sphinxVerbatim}[commandchars=\\\{\}]
\PYG{n}{volnorm\PYGZus{}arr} \PYG{o}{=} \PYG{n}{Eq\PYGZus{}instance}\PYG{o}{.}\PYG{n}{psinorm2volnorm}\PYG{p}{(}\PYG{p}{[}\PYG{l+m+mf}{0.6}\PYG{p}{,} \PYG{l+m+mf}{0.5}\PYG{p}{]}\PYG{p}{,} \PYG{p}{[}\PYG{l+m+mf}{0.2}\PYG{p}{,} \PYG{l+m+mf}{0.3}\PYG{p}{]}\PYG{p}{,} \PYG{n}{each\PYGZus{}t}\PYG{o}{=}\PYG{k+kc}{False}\PYG{p}{)}
\end{sphinxVerbatim}

\end{fulllineitems}

\index{psinorm2phinorm() (eqtools.core.Equilibrium method)@\spxentry{psinorm2phinorm()}\spxextra{eqtools.core.Equilibrium method}}

\begin{fulllineitems}
\phantomsection\label{\detokenize{eqtools:eqtools.core.Equilibrium.psinorm2phinorm}}\pysiglinewithargsret{\sphinxbfcode{\sphinxupquote{psinorm2phinorm}}}{\emph{psi\_norm}, \emph{t}, \emph{**kwargs}}{}
Calculates the normalized toroidal flux corresponding to the passed psi\_norm (normalized poloidal flux) values.
\begin{quote}\begin{description}
\item[{Parameters}] \leavevmode\begin{itemize}
\item {} 
\sphinxstyleliteralstrong{\sphinxupquote{psi\_norm}} (\sphinxstyleliteralemphasis{\sphinxupquote{Array-like}}\sphinxstyleliteralemphasis{\sphinxupquote{ or }}\sphinxstyleliteralemphasis{\sphinxupquote{scalar float}}) \textendash{} Values of the normalized
poloidal flux to map to phinorm.

\item {} 
\sphinxstyleliteralstrong{\sphinxupquote{t}} (\sphinxstyleliteralemphasis{\sphinxupquote{Array-like}}\sphinxstyleliteralemphasis{\sphinxupquote{ or }}\sphinxstyleliteralemphasis{\sphinxupquote{scalar float}}) \textendash{} Times to perform the conversion at.
If \sphinxtitleref{t} is a single value, it is used for all of the elements of
\sphinxtitleref{psi\_norm}. If the \sphinxtitleref{each\_t} keyword is True, then \sphinxtitleref{t} must be scalar
or have exactly one dimension. If the \sphinxtitleref{each\_t} keyword is False,
\sphinxtitleref{t} must have the same shape as \sphinxtitleref{psi\_norm}.

\end{itemize}

\item[{Keyword Arguments}] \leavevmode\begin{itemize}
\item {} 
\sphinxstyleliteralstrong{\sphinxupquote{sqrt}} (\sphinxstyleliteralemphasis{\sphinxupquote{Boolean}}) \textendash{} Set to True to return the square root of phinorm. Only
the square root of positive values is taken. Negative values are
replaced with zeros, consistent with Steve Wolfe’s IDL
implementation efit\_rz2rho.pro. Default is False.

\item {} 
\sphinxstyleliteralstrong{\sphinxupquote{each\_t}} (\sphinxstyleliteralemphasis{\sphinxupquote{Boolean}}) \textendash{} When True, the elements in \sphinxtitleref{psi\_norm} are evaluated at
each value in \sphinxtitleref{t}. If True, \sphinxtitleref{t} must have only one dimension (or
be a scalar). If False, \sphinxtitleref{t} must match the shape of \sphinxtitleref{psi\_norm} or be
a scalar. Default is True (evaluate ALL \sphinxtitleref{psi\_norm} at EACH element in
\sphinxtitleref{t}).

\item {} 
\sphinxstyleliteralstrong{\sphinxupquote{k}} (\sphinxstyleliteralemphasis{\sphinxupquote{positive int}}) \textendash{} The degree of polynomial spline interpolation to
use in converting coordinates.

\item {} 
\sphinxstyleliteralstrong{\sphinxupquote{return\_t}} (\sphinxstyleliteralemphasis{\sphinxupquote{Boolean}}) \textendash{} Set to True to return a tuple of (\sphinxtitleref{rho},
\sphinxtitleref{time\_idxs}), where \sphinxtitleref{time\_idxs} is the array of time indices
actually used in evaluating \sphinxtitleref{rho} with nearest-neighbor
interpolation. (This is mostly present as an internal helper.)
Default is False (only return \sphinxtitleref{rho}).

\end{itemize}

\item[{Returns}] \leavevmode

\sphinxtitleref{phinorm} or (\sphinxtitleref{phinorm}, \sphinxtitleref{time\_idxs})
\begin{itemize}
\item {} 
\sphinxstylestrong{phinorm} (\sphinxtitleref{Array or scalar float}) - The converted coordinates. If
all of the input arguments are scalar, then a scalar is returned.
Otherwise, a scipy Array is returned.

\item {} 
\sphinxstylestrong{time\_idxs} (Array with same shape as \sphinxtitleref{phinorm}) - The indices
(in \sphinxcode{\sphinxupquote{self.getTimeBase()}}) that were used for
nearest-neighbor interpolation. Only returned if \sphinxtitleref{return\_t} is
True.

\end{itemize}


\end{description}\end{quote}
\subsubsection*{Examples}

All assume that \sphinxtitleref{Eq\_instance} is a valid instance of the appropriate
extension of the {\hyperref[\detokenize{eqtools:eqtools.core.Equilibrium}]{\sphinxcrossref{\sphinxcode{\sphinxupquote{Equilibrium}}}}} abstract class.

Find single phinorm value for psinorm=0.7, t=0.26s:

\begin{sphinxVerbatim}[commandchars=\\\{\}]
\PYG{n}{phinorm\PYGZus{}val} \PYG{o}{=} \PYG{n}{Eq\PYGZus{}instance}\PYG{o}{.}\PYG{n}{psinorm2phinorm}\PYG{p}{(}\PYG{l+m+mf}{0.7}\PYG{p}{,} \PYG{l+m+mf}{0.26}\PYG{p}{)}
\end{sphinxVerbatim}

Find phinorm values at psi\_norm values of 0.5 and 0.7 at the single time
t=0.26s:

\begin{sphinxVerbatim}[commandchars=\\\{\}]
\PYG{n}{phinorm\PYGZus{}arr} \PYG{o}{=} \PYG{n}{Eq\PYGZus{}instance}\PYG{o}{.}\PYG{n}{psinorm2phinorm}\PYG{p}{(}\PYG{p}{[}\PYG{l+m+mf}{0.5}\PYG{p}{,} \PYG{l+m+mf}{0.7}\PYG{p}{]}\PYG{p}{,} \PYG{l+m+mf}{0.26}\PYG{p}{)}
\end{sphinxVerbatim}

Find phinorm values at psi\_norm=0.5 at times t={[}0.2s, 0.3s{]}:

\begin{sphinxVerbatim}[commandchars=\\\{\}]
\PYG{n}{phinorm\PYGZus{}arr} \PYG{o}{=} \PYG{n}{Eq\PYGZus{}instance}\PYG{o}{.}\PYG{n}{psinorm2phinorm}\PYG{p}{(}\PYG{l+m+mf}{0.5}\PYG{p}{,} \PYG{p}{[}\PYG{l+m+mf}{0.2}\PYG{p}{,} \PYG{l+m+mf}{0.3}\PYG{p}{]}\PYG{p}{)}
\end{sphinxVerbatim}

Find phinorm values at (psinorm, t) points (0.6, 0.2s) and (0.5, 0.3s):

\begin{sphinxVerbatim}[commandchars=\\\{\}]
\PYG{n}{phinorm\PYGZus{}arr} \PYG{o}{=} \PYG{n}{Eq\PYGZus{}instance}\PYG{o}{.}\PYG{n}{psinorm2phinorm}\PYG{p}{(}\PYG{p}{[}\PYG{l+m+mf}{0.6}\PYG{p}{,} \PYG{l+m+mf}{0.5}\PYG{p}{]}\PYG{p}{,} \PYG{p}{[}\PYG{l+m+mf}{0.2}\PYG{p}{,} \PYG{l+m+mf}{0.3}\PYG{p}{]}\PYG{p}{,} \PYG{n}{each\PYGZus{}t}\PYG{o}{=}\PYG{k+kc}{False}\PYG{p}{)}
\end{sphinxVerbatim}

\end{fulllineitems}

\index{psinorm2rho() (eqtools.core.Equilibrium method)@\spxentry{psinorm2rho()}\spxextra{eqtools.core.Equilibrium method}}

\begin{fulllineitems}
\phantomsection\label{\detokenize{eqtools:eqtools.core.Equilibrium.psinorm2rho}}\pysiglinewithargsret{\sphinxbfcode{\sphinxupquote{psinorm2rho}}}{\emph{method}, \emph{*args}, \emph{**kwargs}}{}
Convert the passed (psinorm, t) coordinates into one of several coordinates.
\begin{quote}\begin{description}
\item[{Parameters}] \leavevmode\begin{itemize}
\item {} 
\sphinxstyleliteralstrong{\sphinxupquote{method}} (\sphinxstyleliteralemphasis{\sphinxupquote{String}}) \textendash{} 
Indicates which coordinates to convert to.
Valid options are:
\begin{quote}


\begin{savenotes}\sphinxattablestart
\centering
\begin{tabulary}{\linewidth}[t]{|T|T|}
\hline

phinorm
&
Normalized toroidal flux
\\
\hline
volnorm
&
Normalized volume
\\
\hline
Rmid
&
Midplane major radius
\\
\hline
r/a
&
Normalized minor radius
\\
\hline
q
&
Safety factor
\\
\hline
F
&
Flux function \(F=RB_{\phi}\)
\\
\hline
FFPrime
&
Flux function \(FF'\)
\\
\hline
p
&
Pressure
\\
\hline
pprime
&
Pressure gradient
\\
\hline
v
&
Flux surface volume
\\
\hline
\end{tabulary}
\par
\sphinxattableend\end{savenotes}
\end{quote}

Additionally, each valid option may be prepended with ‘sqrt’
to specify the square root of the desired unit.


\item {} 
\sphinxstyleliteralstrong{\sphinxupquote{psi\_norm}} (\sphinxstyleliteralemphasis{\sphinxupquote{Array-like}}\sphinxstyleliteralemphasis{\sphinxupquote{ or }}\sphinxstyleliteralemphasis{\sphinxupquote{scalar float}}) \textendash{} Values of the normalized
poloidal flux to map to rho.

\item {} 
\sphinxstyleliteralstrong{\sphinxupquote{t}} (\sphinxstyleliteralemphasis{\sphinxupquote{Array-like}}\sphinxstyleliteralemphasis{\sphinxupquote{ or }}\sphinxstyleliteralemphasis{\sphinxupquote{scalar float}}) \textendash{} Times to perform the conversion at.
If \sphinxtitleref{t} is a single value, it is used for all of the elements of
\sphinxtitleref{psi\_norm}. If the \sphinxtitleref{each\_t} keyword is True, then \sphinxtitleref{t} must be scalar
or have exactly one dimension. If the \sphinxtitleref{each\_t} keyword is False,
\sphinxtitleref{t} must have the same shape as \sphinxtitleref{psi\_norm}.

\end{itemize}

\item[{Keyword Arguments}] \leavevmode\begin{itemize}
\item {} 
\sphinxstyleliteralstrong{\sphinxupquote{sqrt}} (\sphinxstyleliteralemphasis{\sphinxupquote{Boolean}}) \textendash{} Set to True to return the square root of rho. Only
the square root of positive values is taken. Negative values are
replaced with zeros, consistent with Steve Wolfe’s IDL
implementation efit\_rz2rho.pro. Default is False.

\item {} 
\sphinxstyleliteralstrong{\sphinxupquote{each\_t}} (\sphinxstyleliteralemphasis{\sphinxupquote{Boolean}}) \textendash{} When True, the elements in \sphinxtitleref{psi\_norm} are evaluated at
each value in \sphinxtitleref{t}. If True, \sphinxtitleref{t} must have only one dimension (or
be a scalar). If False, \sphinxtitleref{t} must match the shape of \sphinxtitleref{psi\_norm} or be
a scalar. Default is True (evaluate ALL \sphinxtitleref{psi\_norm} at EACH element in
\sphinxtitleref{t}).

\item {} 
\sphinxstyleliteralstrong{\sphinxupquote{rho}} (\sphinxstyleliteralemphasis{\sphinxupquote{Boolean}}) \textendash{} Set to True to return r/a (normalized minor radius)
instead of Rmid. Default is False (return major radius, Rmid).

\item {} 
\sphinxstyleliteralstrong{\sphinxupquote{length\_unit}} (\sphinxstyleliteralemphasis{\sphinxupquote{String}}\sphinxstyleliteralemphasis{\sphinxupquote{ or }}\sphinxstyleliteralemphasis{\sphinxupquote{1}}) \textendash{} 
Length unit that \sphinxtitleref{Rmid} is returned in.
If a string is given, it must be a valid unit specifier:
\begin{quote}


\begin{savenotes}\sphinxattablestart
\centering
\begin{tabulary}{\linewidth}[t]{|T|T|}
\hline

’m’
&
meters
\\
\hline
’cm’
&
centimeters
\\
\hline
’mm’
&
millimeters
\\
\hline
’in’
&
inches
\\
\hline
’ft’
&
feet
\\
\hline
’yd’
&
yards
\\
\hline
’smoot’
&
smoots
\\
\hline
’cubit’
&
cubits
\\
\hline
’hand’
&
hands
\\
\hline
’default’
&
meters
\\
\hline
\end{tabulary}
\par
\sphinxattableend\end{savenotes}
\end{quote}

If length\_unit is 1 or None, meters are assumed. The default
value is 1 (use meters).


\item {} 
\sphinxstyleliteralstrong{\sphinxupquote{k}} (\sphinxstyleliteralemphasis{\sphinxupquote{positive int}}) \textendash{} The degree of polynomial spline interpolation to
use in converting coordinates.

\item {} 
\sphinxstyleliteralstrong{\sphinxupquote{return\_t}} (\sphinxstyleliteralemphasis{\sphinxupquote{Boolean}}) \textendash{} Set to True to return a tuple of (\sphinxtitleref{rho},
\sphinxtitleref{time\_idxs}), where \sphinxtitleref{time\_idxs} is the array of time indices
actually used in evaluating \sphinxtitleref{rho} with nearest-neighbor
interpolation. (This is mostly present as an internal helper.)
Default is False (only return \sphinxtitleref{rho}).

\end{itemize}

\item[{Returns}] \leavevmode

\sphinxtitleref{rho} or (\sphinxtitleref{rho}, \sphinxtitleref{time\_idxs})
\begin{itemize}
\item {} 
\sphinxstylestrong{rho} (\sphinxtitleref{Array or scalar float}) - The converted coordinates. If
all of the input arguments are scalar, then a scalar is returned.
Otherwise, a scipy Array is returned.

\item {} 
\sphinxstylestrong{time\_idxs} (Array with same shape as \sphinxtitleref{rho}) - The indices
(in \sphinxcode{\sphinxupquote{self.getTimeBase()}}) that were used for
nearest-neighbor interpolation. Only returned if \sphinxtitleref{return\_t} is
True.

\end{itemize}


\item[{Raises}] \leavevmode
\sphinxstyleliteralstrong{\sphinxupquote{ValueError}} \textendash{} If \sphinxtitleref{method} is not one of the supported values.

\end{description}\end{quote}
\subsubsection*{Examples}

All assume that \sphinxtitleref{Eq\_instance} is a valid instance of the appropriate
extension of the {\hyperref[\detokenize{eqtools:eqtools.core.Equilibrium}]{\sphinxcrossref{\sphinxcode{\sphinxupquote{Equilibrium}}}}} abstract class.

Find single phinorm value at psinorm=0.6, t=0.26s:

\begin{sphinxVerbatim}[commandchars=\\\{\}]
\PYG{n}{phi\PYGZus{}val} \PYG{o}{=} \PYG{n}{Eq\PYGZus{}instance}\PYG{o}{.}\PYG{n}{psinorm2rho}\PYG{p}{(}\PYG{l+s+s1}{\PYGZsq{}}\PYG{l+s+s1}{phinorm}\PYG{l+s+s1}{\PYGZsq{}}\PYG{p}{,} \PYG{l+m+mf}{0.6}\PYG{p}{,} \PYG{l+m+mf}{0.26}\PYG{p}{)}
\end{sphinxVerbatim}

Find phinorm values at phinorm of 0.6 and 0.8 at the
single time t=0.26s:

\begin{sphinxVerbatim}[commandchars=\\\{\}]
\PYG{n}{phi\PYGZus{}arr} \PYG{o}{=} \PYG{n}{Eq\PYGZus{}instance}\PYG{o}{.}\PYG{n}{psinorm2rho}\PYG{p}{(}\PYG{l+s+s1}{\PYGZsq{}}\PYG{l+s+s1}{phinorm}\PYG{l+s+s1}{\PYGZsq{}}\PYG{p}{,} \PYG{p}{[}\PYG{l+m+mf}{0.6}\PYG{p}{,} \PYG{l+m+mf}{0.8}\PYG{p}{]}\PYG{p}{,} \PYG{l+m+mf}{0.26}\PYG{p}{)}
\end{sphinxVerbatim}

Find phinorm values at psinorm of 0.6 at times t={[}0.2s, 0.3s{]}:

\begin{sphinxVerbatim}[commandchars=\\\{\}]
\PYG{n}{phi\PYGZus{}arr} \PYG{o}{=} \PYG{n}{Eq\PYGZus{}instance}\PYG{o}{.}\PYG{n}{psinorm2rho}\PYG{p}{(}\PYG{l+s+s1}{\PYGZsq{}}\PYG{l+s+s1}{phinorm}\PYG{l+s+s1}{\PYGZsq{}}\PYG{p}{,} \PYG{l+m+mf}{0.6}\PYG{p}{,} \PYG{p}{[}\PYG{l+m+mf}{0.2}\PYG{p}{,} \PYG{l+m+mf}{0.3}\PYG{p}{]}\PYG{p}{)}
\end{sphinxVerbatim}

Find phinorm values at (psinorm, t) points (0.6, 0.2s) and (0.5m, 0.3s):

\begin{sphinxVerbatim}[commandchars=\\\{\}]
\PYG{n}{phi\PYGZus{}arr} \PYG{o}{=} \PYG{n}{Eq\PYGZus{}instance}\PYG{o}{.}\PYG{n}{psinorm2rho}\PYG{p}{(}\PYG{l+s+s1}{\PYGZsq{}}\PYG{l+s+s1}{phinorm}\PYG{l+s+s1}{\PYGZsq{}}\PYG{p}{,} \PYG{p}{[}\PYG{l+m+mf}{0.6}\PYG{p}{,} \PYG{l+m+mf}{0.5}\PYG{p}{]}\PYG{p}{,} \PYG{p}{[}\PYG{l+m+mf}{0.2}\PYG{p}{,} \PYG{l+m+mf}{0.3}\PYG{p}{]}\PYG{p}{,} \PYG{n}{each\PYGZus{}t}\PYG{o}{=}\PYG{k+kc}{False}\PYG{p}{)}
\end{sphinxVerbatim}

\end{fulllineitems}

\index{phinorm2psinorm() (eqtools.core.Equilibrium method)@\spxentry{phinorm2psinorm()}\spxextra{eqtools.core.Equilibrium method}}

\begin{fulllineitems}
\phantomsection\label{\detokenize{eqtools:eqtools.core.Equilibrium.phinorm2psinorm}}\pysiglinewithargsret{\sphinxbfcode{\sphinxupquote{phinorm2psinorm}}}{\emph{phinorm}, \emph{t}, \emph{**kwargs}}{}
Calculates the normalized poloidal flux corresponding to the passed phinorm (normalized toroidal flux) values.
\begin{quote}\begin{description}
\item[{Parameters}] \leavevmode\begin{itemize}
\item {} 
\sphinxstyleliteralstrong{\sphinxupquote{phinorm}} (\sphinxstyleliteralemphasis{\sphinxupquote{Array-like}}\sphinxstyleliteralemphasis{\sphinxupquote{ or }}\sphinxstyleliteralemphasis{\sphinxupquote{scalar float}}) \textendash{} Values of the normalized
toroidal flux to map to psinorm.

\item {} 
\sphinxstyleliteralstrong{\sphinxupquote{t}} (\sphinxstyleliteralemphasis{\sphinxupquote{Array-like}}\sphinxstyleliteralemphasis{\sphinxupquote{ or }}\sphinxstyleliteralemphasis{\sphinxupquote{scalar float}}) \textendash{} Times to perform the conversion at.
If \sphinxtitleref{t} is a single value, it is used for all of the elements of
\sphinxtitleref{phinorm}. If the \sphinxtitleref{each\_t} keyword is True, then \sphinxtitleref{t} must be scalar
or have exactly one dimension. If the \sphinxtitleref{each\_t} keyword is False,
\sphinxtitleref{t} must have the same shape as \sphinxtitleref{phinorm}.

\end{itemize}

\item[{Keyword Arguments}] \leavevmode\begin{itemize}
\item {} 
\sphinxstyleliteralstrong{\sphinxupquote{sqrt}} (\sphinxstyleliteralemphasis{\sphinxupquote{Boolean}}) \textendash{} Set to True to return the square root of psinorm.
Only the square root of positive values is taken. Negative
values are replaced with zeros, consistent with Steve Wolfe’s
IDL implementation efit\_rz2rho.pro. Default is False.

\item {} 
\sphinxstyleliteralstrong{\sphinxupquote{each\_t}} (\sphinxstyleliteralemphasis{\sphinxupquote{Boolean}}) \textendash{} When True, the elements in \sphinxtitleref{phinorm} are evaluated
at each value in \sphinxtitleref{t}. If True, \sphinxtitleref{t} must have only one dimension
(or be a scalar). If False, \sphinxtitleref{t} must match the shape of \sphinxtitleref{phinorm}
or be a scalar. Default is True (evaluate ALL \sphinxtitleref{phinorm} at EACH
element in \sphinxtitleref{t}).

\item {} 
\sphinxstyleliteralstrong{\sphinxupquote{k}} (\sphinxstyleliteralemphasis{\sphinxupquote{positive int}}) \textendash{} The degree of polynomial spline interpolation to
use in converting coordinates.

\item {} 
\sphinxstyleliteralstrong{\sphinxupquote{return\_t}} (\sphinxstyleliteralemphasis{\sphinxupquote{Boolean}}) \textendash{} Set to True to return a tuple of (\sphinxtitleref{rho},
\sphinxtitleref{time\_idxs}), where \sphinxtitleref{time\_idxs} is the array of time indices
actually used in evaluating \sphinxtitleref{rho} with nearest-neighbor
interpolation. (This is mostly present as an internal helper.)
Default is False (only return \sphinxtitleref{rho}).

\end{itemize}

\item[{Returns}] \leavevmode

\sphinxtitleref{psinorm} or (\sphinxtitleref{psinorm}, \sphinxtitleref{time\_idxs})
\begin{itemize}
\item {} 
\sphinxstylestrong{psinorm} (\sphinxtitleref{Array or scalar float}) - The converted coordinates. If
all of the input arguments are scalar, then a scalar is returned.
Otherwise, a scipy Array is returned.

\item {} 
\sphinxstylestrong{time\_idxs} (Array with same shape as \sphinxtitleref{psinorm}) - The indices
(in \sphinxcode{\sphinxupquote{self.getTimeBase()}}) that were used for
nearest-neighbor interpolation. Only returned if \sphinxtitleref{return\_t} is
True.

\end{itemize}


\end{description}\end{quote}
\subsubsection*{Examples}

All assume that \sphinxtitleref{Eq\_instance} is a valid instance of the appropriate
extension of the {\hyperref[\detokenize{eqtools:eqtools.core.Equilibrium}]{\sphinxcrossref{\sphinxcode{\sphinxupquote{Equilibrium}}}}} abstract class.

Find single psinorm value for phinorm=0.7, t=0.26s:

\begin{sphinxVerbatim}[commandchars=\\\{\}]
\PYG{n}{psinorm\PYGZus{}val} \PYG{o}{=} \PYG{n}{Eq\PYGZus{}instance}\PYG{o}{.}\PYG{n}{phinorm2psinorm}\PYG{p}{(}\PYG{l+m+mf}{0.7}\PYG{p}{,} \PYG{l+m+mf}{0.26}\PYG{p}{)}
\end{sphinxVerbatim}

Find psinorm values at phinorm values of 0.5 and 0.7 at the single time
t=0.26s:

\begin{sphinxVerbatim}[commandchars=\\\{\}]
\PYG{n}{psinorm\PYGZus{}arr} \PYG{o}{=} \PYG{n}{Eq\PYGZus{}instance}\PYG{o}{.}\PYG{n}{phinorm2psinorm}\PYG{p}{(}\PYG{p}{[}\PYG{l+m+mf}{0.5}\PYG{p}{,} \PYG{l+m+mf}{0.7}\PYG{p}{]}\PYG{p}{,} \PYG{l+m+mf}{0.26}\PYG{p}{)}
\end{sphinxVerbatim}

Find psinorm values at phinorm=0.5 at times t={[}0.2s, 0.3s{]}:

\begin{sphinxVerbatim}[commandchars=\\\{\}]
\PYG{n}{psinorm\PYGZus{}arr} \PYG{o}{=} \PYG{n}{Eq\PYGZus{}instance}\PYG{o}{.}\PYG{n}{phinorm2psinorm}\PYG{p}{(}\PYG{l+m+mf}{0.5}\PYG{p}{,} \PYG{p}{[}\PYG{l+m+mf}{0.2}\PYG{p}{,} \PYG{l+m+mf}{0.3}\PYG{p}{]}\PYG{p}{)}
\end{sphinxVerbatim}

Find psinorm values at (phinorm, t) points (0.6, 0.2s) and (0.5, 0.3s):

\begin{sphinxVerbatim}[commandchars=\\\{\}]
\PYG{n}{psinorm\PYGZus{}arr} \PYG{o}{=} \PYG{n}{Eq\PYGZus{}instance}\PYG{o}{.}\PYG{n}{phinorm2psinorm}\PYG{p}{(}\PYG{p}{[}\PYG{l+m+mf}{0.6}\PYG{p}{,} \PYG{l+m+mf}{0.5}\PYG{p}{]}\PYG{p}{,} \PYG{p}{[}\PYG{l+m+mf}{0.2}\PYG{p}{,} \PYG{l+m+mf}{0.3}\PYG{p}{]}\PYG{p}{,} \PYG{n}{each\PYGZus{}t}\PYG{o}{=}\PYG{k+kc}{False}\PYG{p}{)}
\end{sphinxVerbatim}

\end{fulllineitems}

\index{phinorm2volnorm() (eqtools.core.Equilibrium method)@\spxentry{phinorm2volnorm()}\spxextra{eqtools.core.Equilibrium method}}

\begin{fulllineitems}
\phantomsection\label{\detokenize{eqtools:eqtools.core.Equilibrium.phinorm2volnorm}}\pysiglinewithargsret{\sphinxbfcode{\sphinxupquote{phinorm2volnorm}}}{\emph{*args}, \emph{**kwargs}}{}
Calculates the normalized flux surface volume corresponding to the passed phinorm (normalized toroidal flux) values.
\begin{quote}\begin{description}
\item[{Parameters}] \leavevmode\begin{itemize}
\item {} 
\sphinxstyleliteralstrong{\sphinxupquote{phinorm}} (\sphinxstyleliteralemphasis{\sphinxupquote{Array-like}}\sphinxstyleliteralemphasis{\sphinxupquote{ or }}\sphinxstyleliteralemphasis{\sphinxupquote{scalar float}}) \textendash{} Values of the normalized
toroidal flux to map to volnorm.

\item {} 
\sphinxstyleliteralstrong{\sphinxupquote{t}} (\sphinxstyleliteralemphasis{\sphinxupquote{Array-like}}\sphinxstyleliteralemphasis{\sphinxupquote{ or }}\sphinxstyleliteralemphasis{\sphinxupquote{scalar float}}) \textendash{} Times to perform the conversion at.
If \sphinxtitleref{t} is a single value, it is used for all of the elements of
\sphinxtitleref{phinorm}. If the \sphinxtitleref{each\_t} keyword is True, then \sphinxtitleref{t} must be scalar
or have exactly one dimension. If the \sphinxtitleref{each\_t} keyword is False,
\sphinxtitleref{t} must have the same shape as \sphinxtitleref{phinorm}.

\end{itemize}

\item[{Keyword Arguments}] \leavevmode\begin{itemize}
\item {} 
\sphinxstyleliteralstrong{\sphinxupquote{sqrt}} (\sphinxstyleliteralemphasis{\sphinxupquote{Boolean}}) \textendash{} Set to True to return the square root of volnorm.
Only the square root of positive values is taken. Negative
values are replaced with zeros, consistent with Steve Wolfe’s
IDL implementation efit\_rz2rho.pro. Default is False.

\item {} 
\sphinxstyleliteralstrong{\sphinxupquote{each\_t}} (\sphinxstyleliteralemphasis{\sphinxupquote{Boolean}}) \textendash{} When True, the elements in \sphinxtitleref{phinorm} are evaluated
at each value in \sphinxtitleref{t}. If True, \sphinxtitleref{t} must have only one dimension
(or be a scalar). If False, \sphinxtitleref{t} must match the shape of \sphinxtitleref{phinorm}
or be a scalar. Default is True (evaluate ALL \sphinxtitleref{phinorm} at EACH
element in \sphinxtitleref{t}).

\item {} 
\sphinxstyleliteralstrong{\sphinxupquote{k}} (\sphinxstyleliteralemphasis{\sphinxupquote{positive int}}) \textendash{} The degree of polynomial spline interpolation to
use in converting coordinates.

\item {} 
\sphinxstyleliteralstrong{\sphinxupquote{return\_t}} (\sphinxstyleliteralemphasis{\sphinxupquote{Boolean}}) \textendash{} Set to True to return a tuple of (\sphinxtitleref{rho},
\sphinxtitleref{time\_idxs}), where \sphinxtitleref{time\_idxs} is the array of time indices
actually used in evaluating \sphinxtitleref{rho} with nearest-neighbor
interpolation. (This is mostly present as an internal helper.)
Default is False (only return \sphinxtitleref{rho}).

\end{itemize}

\item[{Returns}] \leavevmode

\sphinxtitleref{volnorm} or (\sphinxtitleref{volnorm}, \sphinxtitleref{time\_idxs})
\begin{itemize}
\item {} 
\sphinxstylestrong{volnorm} (\sphinxtitleref{Array or scalar float}) - The converted coordinates. If
all of the input arguments are scalar, then a scalar is returned.
Otherwise, a scipy Array is returned.

\item {} 
\sphinxstylestrong{time\_idxs} (Array with same shape as \sphinxtitleref{volnorm}) - The indices
(in \sphinxcode{\sphinxupquote{self.getTimeBase()}}) that were used for
nearest-neighbor interpolation. Only returned if \sphinxtitleref{return\_t} is
True.

\end{itemize}


\end{description}\end{quote}
\subsubsection*{Examples}

All assume that \sphinxtitleref{Eq\_instance} is a valid instance of the appropriate
extension of the {\hyperref[\detokenize{eqtools:eqtools.core.Equilibrium}]{\sphinxcrossref{\sphinxcode{\sphinxupquote{Equilibrium}}}}} abstract class.

Find single volnorm value for phinorm=0.7, t=0.26s:

\begin{sphinxVerbatim}[commandchars=\\\{\}]
\PYG{n}{volnorm\PYGZus{}val} \PYG{o}{=} \PYG{n}{Eq\PYGZus{}instance}\PYG{o}{.}\PYG{n}{phinorm2volnorm}\PYG{p}{(}\PYG{l+m+mf}{0.7}\PYG{p}{,} \PYG{l+m+mf}{0.26}\PYG{p}{)}
\end{sphinxVerbatim}

Find volnorm values at phinorm values of 0.5 and 0.7 at the single time
t=0.26s:

\begin{sphinxVerbatim}[commandchars=\\\{\}]
\PYG{n}{volnorm\PYGZus{}arr} \PYG{o}{=} \PYG{n}{Eq\PYGZus{}instance}\PYG{o}{.}\PYG{n}{phinorm2volnorm}\PYG{p}{(}\PYG{p}{[}\PYG{l+m+mf}{0.5}\PYG{p}{,} \PYG{l+m+mf}{0.7}\PYG{p}{]}\PYG{p}{,} \PYG{l+m+mf}{0.26}\PYG{p}{)}
\end{sphinxVerbatim}

Find volnorm values at phinorm=0.5 at times t={[}0.2s, 0.3s{]}:

\begin{sphinxVerbatim}[commandchars=\\\{\}]
\PYG{n}{volnorm\PYGZus{}arr} \PYG{o}{=} \PYG{n}{Eq\PYGZus{}instance}\PYG{o}{.}\PYG{n}{phinorm2volnorm}\PYG{p}{(}\PYG{l+m+mf}{0.5}\PYG{p}{,} \PYG{p}{[}\PYG{l+m+mf}{0.2}\PYG{p}{,} \PYG{l+m+mf}{0.3}\PYG{p}{]}\PYG{p}{)}
\end{sphinxVerbatim}

Find volnorm values at (phinorm, t) points (0.6, 0.2s) and (0.5, 0.3s):

\begin{sphinxVerbatim}[commandchars=\\\{\}]
\PYG{n}{volnorm\PYGZus{}arr} \PYG{o}{=} \PYG{n}{Eq\PYGZus{}instance}\PYG{o}{.}\PYG{n}{phinorm2volnorm}\PYG{p}{(}\PYG{p}{[}\PYG{l+m+mf}{0.6}\PYG{p}{,} \PYG{l+m+mf}{0.5}\PYG{p}{]}\PYG{p}{,} \PYG{p}{[}\PYG{l+m+mf}{0.2}\PYG{p}{,} \PYG{l+m+mf}{0.3}\PYG{p}{]}\PYG{p}{,} \PYG{n}{each\PYGZus{}t}\PYG{o}{=}\PYG{k+kc}{False}\PYG{p}{)}
\end{sphinxVerbatim}

\end{fulllineitems}

\index{phinorm2rmid() (eqtools.core.Equilibrium method)@\spxentry{phinorm2rmid()}\spxextra{eqtools.core.Equilibrium method}}

\begin{fulllineitems}
\phantomsection\label{\detokenize{eqtools:eqtools.core.Equilibrium.phinorm2rmid}}\pysiglinewithargsret{\sphinxbfcode{\sphinxupquote{phinorm2rmid}}}{\emph{*args}, \emph{**kwargs}}{}
Calculates the mapped outboard midplane major radius corresponding to the passed phinorm (normalized toroidal flux) values.
\begin{quote}\begin{description}
\item[{Parameters}] \leavevmode\begin{itemize}
\item {} 
\sphinxstyleliteralstrong{\sphinxupquote{phinorm}} (\sphinxstyleliteralemphasis{\sphinxupquote{Array-like}}\sphinxstyleliteralemphasis{\sphinxupquote{ or }}\sphinxstyleliteralemphasis{\sphinxupquote{scalar float}}) \textendash{} Values of the normalized
toroidal flux to map to Rmid.

\item {} 
\sphinxstyleliteralstrong{\sphinxupquote{t}} (\sphinxstyleliteralemphasis{\sphinxupquote{Array-like}}\sphinxstyleliteralemphasis{\sphinxupquote{ or }}\sphinxstyleliteralemphasis{\sphinxupquote{scalar float}}) \textendash{} Times to perform the conversion at.
If \sphinxtitleref{t} is a single value, it is used for all of the elements of
\sphinxtitleref{phinorm}. If the \sphinxtitleref{each\_t} keyword is True, then \sphinxtitleref{t} must be scalar
or have exactly one dimension. If the \sphinxtitleref{each\_t} keyword is False,
\sphinxtitleref{t} must have the same shape as \sphinxtitleref{phinorm}.

\end{itemize}

\item[{Keyword Arguments}] \leavevmode\begin{itemize}
\item {} 
\sphinxstyleliteralstrong{\sphinxupquote{sqrt}} (\sphinxstyleliteralemphasis{\sphinxupquote{Boolean}}) \textendash{} Set to True to return the square root of Rmid.
Only the square root of positive values is taken. Negative
values are replaced with zeros, consistent with Steve Wolfe’s
IDL implementation efit\_rz2rho.pro. Default is False.

\item {} 
\sphinxstyleliteralstrong{\sphinxupquote{each\_t}} (\sphinxstyleliteralemphasis{\sphinxupquote{Boolean}}) \textendash{} When True, the elements in \sphinxtitleref{phinorm} are evaluated
at each value in \sphinxtitleref{t}. If True, \sphinxtitleref{t} must have only one dimension
(or be a scalar). If False, \sphinxtitleref{t} must match the shape of \sphinxtitleref{phinorm}
or be a scalar. Default is True (evaluate ALL \sphinxtitleref{phinorm} at EACH
element in \sphinxtitleref{t}).

\item {} 
\sphinxstyleliteralstrong{\sphinxupquote{rho}} (\sphinxstyleliteralemphasis{\sphinxupquote{Boolean}}) \textendash{} Set to True to return r/a (normalized minor radius)
instead of Rmid. Default is False (return major radius, Rmid).

\item {} 
\sphinxstyleliteralstrong{\sphinxupquote{length\_unit}} (\sphinxstyleliteralemphasis{\sphinxupquote{String}}\sphinxstyleliteralemphasis{\sphinxupquote{ or }}\sphinxstyleliteralemphasis{\sphinxupquote{1}}) \textendash{} 
Length unit that \sphinxtitleref{Rmid} is returned in.
If a string is given, it must be a valid unit specifier:
\begin{quote}


\begin{savenotes}\sphinxattablestart
\centering
\begin{tabulary}{\linewidth}[t]{|T|T|}
\hline

’m’
&
meters
\\
\hline
’cm’
&
centimeters
\\
\hline
’mm’
&
millimeters
\\
\hline
’in’
&
inches
\\
\hline
’ft’
&
feet
\\
\hline
’yd’
&
yards
\\
\hline
’smoot’
&
smoots
\\
\hline
’cubit’
&
cubits
\\
\hline
’hand’
&
hands
\\
\hline
’default’
&
meters
\\
\hline
\end{tabulary}
\par
\sphinxattableend\end{savenotes}
\end{quote}

If length\_unit is 1 or None, meters are assumed. The default
value is 1 (use meters).


\item {} 
\sphinxstyleliteralstrong{\sphinxupquote{k}} (\sphinxstyleliteralemphasis{\sphinxupquote{positive int}}) \textendash{} The degree of polynomial spline interpolation to
use in converting coordinates.

\item {} 
\sphinxstyleliteralstrong{\sphinxupquote{return\_t}} (\sphinxstyleliteralemphasis{\sphinxupquote{Boolean}}) \textendash{} Set to True to return a tuple of (\sphinxtitleref{rho},
\sphinxtitleref{time\_idxs}), where \sphinxtitleref{time\_idxs} is the array of time indices
actually used in evaluating \sphinxtitleref{rho} with nearest-neighbor
interpolation. (This is mostly present as an internal helper.)
Default is False (only return \sphinxtitleref{rho}).

\end{itemize}

\item[{Returns}] \leavevmode

\sphinxtitleref{Rmid} or (\sphinxtitleref{Rmid}, \sphinxtitleref{time\_idxs})
\begin{itemize}
\item {} 
\sphinxstylestrong{Rmid} (\sphinxtitleref{Array or scalar float}) - The converted coordinates. If
all of the input arguments are scalar, then a scalar is returned.
Otherwise, a scipy Array is returned.

\item {} 
\sphinxstylestrong{time\_idxs} (Array with same shape as \sphinxtitleref{Rmid}) - The indices
(in \sphinxcode{\sphinxupquote{self.getTimeBase()}}) that were used for
nearest-neighbor interpolation. Only returned if \sphinxtitleref{return\_t} is
True.

\end{itemize}


\end{description}\end{quote}
\subsubsection*{Examples}

All assume that \sphinxtitleref{Eq\_instance} is a valid instance of the appropriate
extension of the {\hyperref[\detokenize{eqtools:eqtools.core.Equilibrium}]{\sphinxcrossref{\sphinxcode{\sphinxupquote{Equilibrium}}}}} abstract class.

Find single Rmid value for phinorm=0.7, t=0.26s:

\begin{sphinxVerbatim}[commandchars=\\\{\}]
\PYG{n}{Rmid\PYGZus{}val} \PYG{o}{=} \PYG{n}{Eq\PYGZus{}instance}\PYG{o}{.}\PYG{n}{phinorm2rmid}\PYG{p}{(}\PYG{l+m+mf}{0.7}\PYG{p}{,} \PYG{l+m+mf}{0.26}\PYG{p}{)}
\end{sphinxVerbatim}

Find Rmid values at phinorm values of 0.5 and 0.7 at the single time
t=0.26s:

\begin{sphinxVerbatim}[commandchars=\\\{\}]
\PYG{n}{Rmid\PYGZus{}arr} \PYG{o}{=} \PYG{n}{Eq\PYGZus{}instance}\PYG{o}{.}\PYG{n}{phinorm2rmid}\PYG{p}{(}\PYG{p}{[}\PYG{l+m+mf}{0.5}\PYG{p}{,} \PYG{l+m+mf}{0.7}\PYG{p}{]}\PYG{p}{,} \PYG{l+m+mf}{0.26}\PYG{p}{)}
\end{sphinxVerbatim}

Find Rmid values at phinorm=0.5 at times t={[}0.2s, 0.3s{]}:

\begin{sphinxVerbatim}[commandchars=\\\{\}]
\PYG{n}{Rmid\PYGZus{}arr} \PYG{o}{=} \PYG{n}{Eq\PYGZus{}instance}\PYG{o}{.}\PYG{n}{phinorm2rmid}\PYG{p}{(}\PYG{l+m+mf}{0.5}\PYG{p}{,} \PYG{p}{[}\PYG{l+m+mf}{0.2}\PYG{p}{,} \PYG{l+m+mf}{0.3}\PYG{p}{]}\PYG{p}{)}
\end{sphinxVerbatim}

Find Rmid values at (phinorm, t) points (0.6, 0.2s) and (0.5, 0.3s):

\begin{sphinxVerbatim}[commandchars=\\\{\}]
\PYG{n}{Rmid\PYGZus{}arr} \PYG{o}{=} \PYG{n}{Eq\PYGZus{}instance}\PYG{o}{.}\PYG{n}{phinorm2rmid}\PYG{p}{(}\PYG{p}{[}\PYG{l+m+mf}{0.6}\PYG{p}{,} \PYG{l+m+mf}{0.5}\PYG{p}{]}\PYG{p}{,} \PYG{p}{[}\PYG{l+m+mf}{0.2}\PYG{p}{,} \PYG{l+m+mf}{0.3}\PYG{p}{]}\PYG{p}{,} \PYG{n}{each\PYGZus{}t}\PYG{o}{=}\PYG{k+kc}{False}\PYG{p}{)}
\end{sphinxVerbatim}

\end{fulllineitems}

\index{phinorm2roa() (eqtools.core.Equilibrium method)@\spxentry{phinorm2roa()}\spxextra{eqtools.core.Equilibrium method}}

\begin{fulllineitems}
\phantomsection\label{\detokenize{eqtools:eqtools.core.Equilibrium.phinorm2roa}}\pysiglinewithargsret{\sphinxbfcode{\sphinxupquote{phinorm2roa}}}{\emph{phi\_norm}, \emph{t}, \emph{**kwargs}}{}
Calculates the normalized minor radius corresponding to the passed phinorm (normalized toroidal flux) values.
\begin{quote}\begin{description}
\item[{Parameters}] \leavevmode\begin{itemize}
\item {} 
\sphinxstyleliteralstrong{\sphinxupquote{phinorm}} (\sphinxstyleliteralemphasis{\sphinxupquote{Array-like}}\sphinxstyleliteralemphasis{\sphinxupquote{ or }}\sphinxstyleliteralemphasis{\sphinxupquote{scalar float}}) \textendash{} Values of the normalized
toroidal flux to map to r/a.

\item {} 
\sphinxstyleliteralstrong{\sphinxupquote{t}} (\sphinxstyleliteralemphasis{\sphinxupquote{Array-like}}\sphinxstyleliteralemphasis{\sphinxupquote{ or }}\sphinxstyleliteralemphasis{\sphinxupquote{scalar float}}) \textendash{} Times to perform the conversion at.
If \sphinxtitleref{t} is a single value, it is used for all of the elements of
\sphinxtitleref{phinorm}. If the \sphinxtitleref{each\_t} keyword is True, then \sphinxtitleref{t} must be scalar
or have exactly one dimension. If the \sphinxtitleref{each\_t} keyword is False,
\sphinxtitleref{t} must have the same shape as \sphinxtitleref{phinorm}.

\end{itemize}

\item[{Keyword Arguments}] \leavevmode\begin{itemize}
\item {} 
\sphinxstyleliteralstrong{\sphinxupquote{sqrt}} (\sphinxstyleliteralemphasis{\sphinxupquote{Boolean}}) \textendash{} Set to True to return the square root of r/a.
Only the square root of positive values is taken. Negative
values are replaced with zeros, consistent with Steve Wolfe’s
IDL implementation efit\_rz2rho.pro. Default is False.

\item {} 
\sphinxstyleliteralstrong{\sphinxupquote{each\_t}} (\sphinxstyleliteralemphasis{\sphinxupquote{Boolean}}) \textendash{} When True, the elements in \sphinxtitleref{phinorm} are evaluated
at each value in \sphinxtitleref{t}. If True, \sphinxtitleref{t} must have only one dimension
(or be a scalar). If False, \sphinxtitleref{t} must match the shape of \sphinxtitleref{phinorm}
or be a scalar. Default is True (evaluate ALL \sphinxtitleref{phinorm} at EACH
element in \sphinxtitleref{t}).

\item {} 
\sphinxstyleliteralstrong{\sphinxupquote{k}} (\sphinxstyleliteralemphasis{\sphinxupquote{positive int}}) \textendash{} The degree of polynomial spline interpolation to
use in converting coordinates.

\item {} 
\sphinxstyleliteralstrong{\sphinxupquote{return\_t}} (\sphinxstyleliteralemphasis{\sphinxupquote{Boolean}}) \textendash{} Set to True to return a tuple of (\sphinxtitleref{rho},
\sphinxtitleref{time\_idxs}), where \sphinxtitleref{time\_idxs} is the array of time indices
actually used in evaluating \sphinxtitleref{rho} with nearest-neighbor
interpolation. (This is mostly present as an internal helper.)
Default is False (only return \sphinxtitleref{rho}).

\end{itemize}

\item[{Returns}] \leavevmode

\sphinxtitleref{roa} or (\sphinxtitleref{roa}, \sphinxtitleref{time\_idxs})
\begin{itemize}
\item {} 
\sphinxstylestrong{roa} (\sphinxtitleref{Array or scalar float}) - Normalized midplane minor
radius. If all of the input arguments are scalar, then a scalar
is returned. Otherwise, a scipy Array is returned.

\item {} 
\sphinxstylestrong{time\_idxs} (Array with same shape as \sphinxtitleref{roa}) - The indices
(in \sphinxcode{\sphinxupquote{self.getTimeBase()}}) that were used for
nearest-neighbor interpolation. Only returned if \sphinxtitleref{return\_t} is
True.

\end{itemize}


\end{description}\end{quote}
\subsubsection*{Examples}

All assume that \sphinxtitleref{Eq\_instance} is a valid instance of the appropriate
extension of the {\hyperref[\detokenize{eqtools:eqtools.core.Equilibrium}]{\sphinxcrossref{\sphinxcode{\sphinxupquote{Equilibrium}}}}} abstract class.

Find single r/a value for phinorm=0.7, t=0.26s:

\begin{sphinxVerbatim}[commandchars=\\\{\}]
\PYG{n}{roa\PYGZus{}val} \PYG{o}{=} \PYG{n}{Eq\PYGZus{}instance}\PYG{o}{.}\PYG{n}{phinorm2roa}\PYG{p}{(}\PYG{l+m+mf}{0.7}\PYG{p}{,} \PYG{l+m+mf}{0.26}\PYG{p}{)}
\end{sphinxVerbatim}

Find r/a values at phinorm values of 0.5 and 0.7 at the single time
t=0.26s:

\begin{sphinxVerbatim}[commandchars=\\\{\}]
\PYG{n}{roa\PYGZus{}arr} \PYG{o}{=} \PYG{n}{Eq\PYGZus{}instance}\PYG{o}{.}\PYG{n}{phinorm2roa}\PYG{p}{(}\PYG{p}{[}\PYG{l+m+mf}{0.5}\PYG{p}{,} \PYG{l+m+mf}{0.7}\PYG{p}{]}\PYG{p}{,} \PYG{l+m+mf}{0.26}\PYG{p}{)}
\end{sphinxVerbatim}

Find r/a values at phinorm=0.5 at times t={[}0.2s, 0.3s{]}:

\begin{sphinxVerbatim}[commandchars=\\\{\}]
\PYG{n}{roa\PYGZus{}arr} \PYG{o}{=} \PYG{n}{Eq\PYGZus{}instance}\PYG{o}{.}\PYG{n}{phinorm2roa}\PYG{p}{(}\PYG{l+m+mf}{0.5}\PYG{p}{,} \PYG{p}{[}\PYG{l+m+mf}{0.2}\PYG{p}{,} \PYG{l+m+mf}{0.3}\PYG{p}{]}\PYG{p}{)}
\end{sphinxVerbatim}

Find r/a values at (phinorm, t) points (0.6, 0.2s) and (0.5, 0.3s):

\begin{sphinxVerbatim}[commandchars=\\\{\}]
\PYG{n}{roa\PYGZus{}arr} \PYG{o}{=} \PYG{n}{Eq\PYGZus{}instance}\PYG{o}{.}\PYG{n}{phinorm2roa}\PYG{p}{(}\PYG{p}{[}\PYG{l+m+mf}{0.6}\PYG{p}{,} \PYG{l+m+mf}{0.5}\PYG{p}{]}\PYG{p}{,} \PYG{p}{[}\PYG{l+m+mf}{0.2}\PYG{p}{,} \PYG{l+m+mf}{0.3}\PYG{p}{]}\PYG{p}{,} \PYG{n}{each\PYGZus{}t}\PYG{o}{=}\PYG{k+kc}{False}\PYG{p}{)}
\end{sphinxVerbatim}

\end{fulllineitems}

\index{phinorm2rho() (eqtools.core.Equilibrium method)@\spxentry{phinorm2rho()}\spxextra{eqtools.core.Equilibrium method}}

\begin{fulllineitems}
\phantomsection\label{\detokenize{eqtools:eqtools.core.Equilibrium.phinorm2rho}}\pysiglinewithargsret{\sphinxbfcode{\sphinxupquote{phinorm2rho}}}{\emph{method}, \emph{*args}, \emph{**kwargs}}{}
Convert the passed (phinorm, t) coordinates into one of several coordinates.
\begin{quote}\begin{description}
\item[{Parameters}] \leavevmode\begin{itemize}
\item {} 
\sphinxstyleliteralstrong{\sphinxupquote{method}} (\sphinxstyleliteralemphasis{\sphinxupquote{String}}) \textendash{} 
Indicates which coordinates to convert to.
Valid options are:
\begin{quote}


\begin{savenotes}\sphinxattablestart
\centering
\begin{tabulary}{\linewidth}[t]{|T|T|}
\hline

psinorm
&
Normalized poloidal flux
\\
\hline
volnorm
&
Normalized volume
\\
\hline
Rmid
&
Midplane major radius
\\
\hline
r/a
&
Normalized minor radius
\\
\hline
q
&
Safety factor
\\
\hline
F
&
Flux function \(F=RB_{\phi}\)
\\
\hline
FFPrime
&
Flux function \(FF'\)
\\
\hline
p
&
Pressure
\\
\hline
pprime
&
Pressure gradient
\\
\hline
v
&
Flux surface volume
\\
\hline
\end{tabulary}
\par
\sphinxattableend\end{savenotes}
\end{quote}

Additionally, each valid option may be prepended with ‘sqrt’
to specify the square root of the desired unit.


\item {} 
\sphinxstyleliteralstrong{\sphinxupquote{phinorm}} (\sphinxstyleliteralemphasis{\sphinxupquote{Array-like}}\sphinxstyleliteralemphasis{\sphinxupquote{ or }}\sphinxstyleliteralemphasis{\sphinxupquote{scalar float}}) \textendash{} Values of the normalized
toroidal flux to map to rho.

\item {} 
\sphinxstyleliteralstrong{\sphinxupquote{t}} (\sphinxstyleliteralemphasis{\sphinxupquote{Array-like}}\sphinxstyleliteralemphasis{\sphinxupquote{ or }}\sphinxstyleliteralemphasis{\sphinxupquote{scalar float}}) \textendash{} Times to perform the conversion at.
If \sphinxtitleref{t} is a single value, it is used for all of the elements of
\sphinxtitleref{phinorm}. If the \sphinxtitleref{each\_t} keyword is True, then \sphinxtitleref{t} must be scalar
or have exactly one dimension. If the \sphinxtitleref{each\_t} keyword is False,
\sphinxtitleref{t} must have the same shape as \sphinxtitleref{phinorm}.

\end{itemize}

\item[{Keyword Arguments}] \leavevmode\begin{itemize}
\item {} 
\sphinxstyleliteralstrong{\sphinxupquote{sqrt}} (\sphinxstyleliteralemphasis{\sphinxupquote{Boolean}}) \textendash{} Set to True to return the square root of rho. Only
the square root of positive values is taken. Negative values are
replaced with zeros, consistent with Steve Wolfe’s IDL
implementation efit\_rz2rho.pro. Default is False.

\item {} 
\sphinxstyleliteralstrong{\sphinxupquote{each\_t}} (\sphinxstyleliteralemphasis{\sphinxupquote{Boolean}}) \textendash{} When True, the elements in \sphinxtitleref{phinorm} are evaluated at
each value in \sphinxtitleref{t}. If True, \sphinxtitleref{t} must have only one dimension (or
be a scalar). If False, \sphinxtitleref{t} must match the shape of \sphinxtitleref{phinorm} or be
a scalar. Default is True (evaluate ALL \sphinxtitleref{phinorm} at EACH element in
\sphinxtitleref{t}).

\item {} 
\sphinxstyleliteralstrong{\sphinxupquote{rho}} (\sphinxstyleliteralemphasis{\sphinxupquote{Boolean}}) \textendash{} Set to True to return r/a (normalized minor radius)
instead of Rmid. Default is False (return major radius, Rmid).

\item {} 
\sphinxstyleliteralstrong{\sphinxupquote{length\_unit}} (\sphinxstyleliteralemphasis{\sphinxupquote{String}}\sphinxstyleliteralemphasis{\sphinxupquote{ or }}\sphinxstyleliteralemphasis{\sphinxupquote{1}}) \textendash{} 
Length unit that \sphinxtitleref{Rmid} is returned in.
If a string is given, it must be a valid unit specifier:
\begin{quote}


\begin{savenotes}\sphinxattablestart
\centering
\begin{tabulary}{\linewidth}[t]{|T|T|}
\hline

’m’
&
meters
\\
\hline
’cm’
&
centimeters
\\
\hline
’mm’
&
millimeters
\\
\hline
’in’
&
inches
\\
\hline
’ft’
&
feet
\\
\hline
’yd’
&
yards
\\
\hline
’smoot’
&
smoots
\\
\hline
’cubit’
&
cubits
\\
\hline
’hand’
&
hands
\\
\hline
’default’
&
meters
\\
\hline
\end{tabulary}
\par
\sphinxattableend\end{savenotes}
\end{quote}

If length\_unit is 1 or None, meters are assumed. The default
value is 1 (use meters).


\item {} 
\sphinxstyleliteralstrong{\sphinxupquote{k}} (\sphinxstyleliteralemphasis{\sphinxupquote{positive int}}) \textendash{} The degree of polynomial spline interpolation to
use in converting coordinates.

\item {} 
\sphinxstyleliteralstrong{\sphinxupquote{return\_t}} (\sphinxstyleliteralemphasis{\sphinxupquote{Boolean}}) \textendash{} Set to True to return a tuple of (\sphinxtitleref{rho},
\sphinxtitleref{time\_idxs}), where \sphinxtitleref{time\_idxs} is the array of time indices
actually used in evaluating \sphinxtitleref{rho} with nearest-neighbor
interpolation. (This is mostly present as an internal helper.)
Default is False (only return \sphinxtitleref{rho}).

\end{itemize}

\item[{Returns}] \leavevmode

\sphinxtitleref{rho} or (\sphinxtitleref{rho}, \sphinxtitleref{time\_idxs})
\begin{itemize}
\item {} 
\sphinxstylestrong{rho} (\sphinxtitleref{Array or scalar float}) - The converted coordinates. If
all of the input arguments are scalar, then a scalar is returned.
Otherwise, a scipy Array is returned.

\item {} 
\sphinxstylestrong{time\_idxs} (Array with same shape as \sphinxtitleref{rho}) - The indices
(in \sphinxcode{\sphinxupquote{self.getTimeBase()}}) that were used for
nearest-neighbor interpolation. Only returned if \sphinxtitleref{return\_t} is
True.

\end{itemize}


\item[{Raises}] \leavevmode
\sphinxstyleliteralstrong{\sphinxupquote{ValueError}} \textendash{} If \sphinxtitleref{method} is not one of the supported values.

\end{description}\end{quote}
\subsubsection*{Examples}

All assume that \sphinxtitleref{Eq\_instance} is a valid instance of the appropriate
extension of the {\hyperref[\detokenize{eqtools:eqtools.core.Equilibrium}]{\sphinxcrossref{\sphinxcode{\sphinxupquote{Equilibrium}}}}} abstract class.

Find single psinorm value at phinorm=0.6, t=0.26s:

\begin{sphinxVerbatim}[commandchars=\\\{\}]
\PYG{n}{psi\PYGZus{}val} \PYG{o}{=} \PYG{n}{Eq\PYGZus{}instance}\PYG{o}{.}\PYG{n}{phinorm2rho}\PYG{p}{(}\PYG{l+s+s1}{\PYGZsq{}}\PYG{l+s+s1}{psinorm}\PYG{l+s+s1}{\PYGZsq{}}\PYG{p}{,} \PYG{l+m+mf}{0.6}\PYG{p}{,} \PYG{l+m+mf}{0.26}\PYG{p}{)}
\end{sphinxVerbatim}

Find psinorm values at phinorm of 0.6 and 0.8 at the
single time t=0.26s:

\begin{sphinxVerbatim}[commandchars=\\\{\}]
\PYG{n}{psi\PYGZus{}arr} \PYG{o}{=} \PYG{n}{Eq\PYGZus{}instance}\PYG{o}{.}\PYG{n}{phinorm2rho}\PYG{p}{(}\PYG{l+s+s1}{\PYGZsq{}}\PYG{l+s+s1}{psinorm}\PYG{l+s+s1}{\PYGZsq{}}\PYG{p}{,} \PYG{p}{[}\PYG{l+m+mf}{0.6}\PYG{p}{,} \PYG{l+m+mf}{0.8}\PYG{p}{]}\PYG{p}{,} \PYG{l+m+mf}{0.26}\PYG{p}{)}
\end{sphinxVerbatim}

Find psinorm values at phinorm of 0.6 at times t={[}0.2s, 0.3s{]}:

\begin{sphinxVerbatim}[commandchars=\\\{\}]
\PYG{n}{psi\PYGZus{}arr} \PYG{o}{=} \PYG{n}{Eq\PYGZus{}instance}\PYG{o}{.}\PYG{n}{phinorm2rho}\PYG{p}{(}\PYG{l+s+s1}{\PYGZsq{}}\PYG{l+s+s1}{psinorm}\PYG{l+s+s1}{\PYGZsq{}}\PYG{p}{,} \PYG{l+m+mf}{0.6}\PYG{p}{,} \PYG{p}{[}\PYG{l+m+mf}{0.2}\PYG{p}{,} \PYG{l+m+mf}{0.3}\PYG{p}{]}\PYG{p}{)}
\end{sphinxVerbatim}

Find psinorm values at (phinorm, t) points (0.6, 0.2s) and (0.5m, 0.3s):

\begin{sphinxVerbatim}[commandchars=\\\{\}]
\PYG{n}{psi\PYGZus{}arr} \PYG{o}{=} \PYG{n}{Eq\PYGZus{}instance}\PYG{o}{.}\PYG{n}{phinorm2rho}\PYG{p}{(}\PYG{l+s+s1}{\PYGZsq{}}\PYG{l+s+s1}{psinorm}\PYG{l+s+s1}{\PYGZsq{}}\PYG{p}{,} \PYG{p}{[}\PYG{l+m+mf}{0.6}\PYG{p}{,} \PYG{l+m+mf}{0.5}\PYG{p}{]}\PYG{p}{,} \PYG{p}{[}\PYG{l+m+mf}{0.2}\PYG{p}{,} \PYG{l+m+mf}{0.3}\PYG{p}{]}\PYG{p}{,} \PYG{n}{each\PYGZus{}t}\PYG{o}{=}\PYG{k+kc}{False}\PYG{p}{)}
\end{sphinxVerbatim}

\end{fulllineitems}

\index{volnorm2psinorm() (eqtools.core.Equilibrium method)@\spxentry{volnorm2psinorm()}\spxextra{eqtools.core.Equilibrium method}}

\begin{fulllineitems}
\phantomsection\label{\detokenize{eqtools:eqtools.core.Equilibrium.volnorm2psinorm}}\pysiglinewithargsret{\sphinxbfcode{\sphinxupquote{volnorm2psinorm}}}{\emph{*args}, \emph{**kwargs}}{}
Calculates the normalized poloidal flux corresponding to the passed volnorm (normalized flux surface volume) values.
\begin{quote}\begin{description}
\item[{Parameters}] \leavevmode\begin{itemize}
\item {} 
\sphinxstyleliteralstrong{\sphinxupquote{volnorm}} (\sphinxstyleliteralemphasis{\sphinxupquote{Array-like}}\sphinxstyleliteralemphasis{\sphinxupquote{ or }}\sphinxstyleliteralemphasis{\sphinxupquote{scalar float}}) \textendash{} Values of the normalized
flux surface volume to map to psinorm.

\item {} 
\sphinxstyleliteralstrong{\sphinxupquote{t}} (\sphinxstyleliteralemphasis{\sphinxupquote{Array-like}}\sphinxstyleliteralemphasis{\sphinxupquote{ or }}\sphinxstyleliteralemphasis{\sphinxupquote{scalar float}}) \textendash{} Times to perform the conversion at.
If \sphinxtitleref{t} is a single value, it is used for all of the elements of
\sphinxtitleref{volnorm}. If the \sphinxtitleref{each\_t} keyword is True, then \sphinxtitleref{t} must be scalar
or have exactly one dimension. If the \sphinxtitleref{each\_t} keyword is False,
\sphinxtitleref{t} must have the same shape as \sphinxtitleref{volnorm}.

\end{itemize}

\item[{Keyword Arguments}] \leavevmode\begin{itemize}
\item {} 
\sphinxstyleliteralstrong{\sphinxupquote{sqrt}} (\sphinxstyleliteralemphasis{\sphinxupquote{Boolean}}) \textendash{} Set to True to return the square root of psinorm.
Only the square root of positive values is taken. Negative
values are replaced with zeros, consistent with Steve Wolfe’s
IDL implementation efit\_rz2rho.pro. Default is False.

\item {} 
\sphinxstyleliteralstrong{\sphinxupquote{each\_t}} (\sphinxstyleliteralemphasis{\sphinxupquote{Boolean}}) \textendash{} When True, the elements in \sphinxtitleref{volnorm} are evaluated
at each value in \sphinxtitleref{t}. If True, \sphinxtitleref{t} must have only one dimension
(or be a scalar). If False, \sphinxtitleref{t} must match the shape of \sphinxtitleref{volnorm}
or be a scalar. Default is True (evaluate ALL \sphinxtitleref{volnorm} at EACH
element in \sphinxtitleref{t}).

\item {} 
\sphinxstyleliteralstrong{\sphinxupquote{k}} (\sphinxstyleliteralemphasis{\sphinxupquote{positive int}}) \textendash{} The degree of polynomial spline interpolation to
use in converting coordinates.

\item {} 
\sphinxstyleliteralstrong{\sphinxupquote{return\_t}} (\sphinxstyleliteralemphasis{\sphinxupquote{Boolean}}) \textendash{} Set to True to return a tuple of (\sphinxtitleref{rho},
\sphinxtitleref{time\_idxs}), where \sphinxtitleref{time\_idxs} is the array of time indices
actually used in evaluating \sphinxtitleref{rho} with nearest-neighbor
interpolation. (This is mostly present as an internal helper.)
Default is False (only return \sphinxtitleref{rho}).

\end{itemize}

\item[{Returns}] \leavevmode

\sphinxtitleref{psinorm} or (\sphinxtitleref{psinorm}, \sphinxtitleref{time\_idxs})
\begin{itemize}
\item {} 
\sphinxstylestrong{psinorm} (\sphinxtitleref{Array or scalar float}) - The converted coordinates. If
all of the input arguments are scalar, then a scalar is returned.
Otherwise, a scipy Array is returned.

\item {} 
\sphinxstylestrong{time\_idxs} (Array with same shape as \sphinxtitleref{psinorm}) - The indices
(in \sphinxcode{\sphinxupquote{self.getTimeBase()}}) that were used for
nearest-neighbor interpolation. Only returned if \sphinxtitleref{return\_t} is
True.

\end{itemize}


\end{description}\end{quote}
\subsubsection*{Examples}

All assume that \sphinxtitleref{Eq\_instance} is a valid instance of the appropriate
extension of the {\hyperref[\detokenize{eqtools:eqtools.core.Equilibrium}]{\sphinxcrossref{\sphinxcode{\sphinxupquote{Equilibrium}}}}} abstract class.

Find single psinorm value for volnorm=0.7, t=0.26s:

\begin{sphinxVerbatim}[commandchars=\\\{\}]
\PYG{n}{psinorm\PYGZus{}val} \PYG{o}{=} \PYG{n}{Eq\PYGZus{}instance}\PYG{o}{.}\PYG{n}{volnorm2psinorm}\PYG{p}{(}\PYG{l+m+mf}{0.7}\PYG{p}{,} \PYG{l+m+mf}{0.26}\PYG{p}{)}
\end{sphinxVerbatim}

Find psinorm values at volnorm values of 0.5 and 0.7 at the single time
t=0.26s:

\begin{sphinxVerbatim}[commandchars=\\\{\}]
\PYG{n}{psinorm\PYGZus{}arr} \PYG{o}{=} \PYG{n}{Eq\PYGZus{}instance}\PYG{o}{.}\PYG{n}{volnorm2psinorm}\PYG{p}{(}\PYG{p}{[}\PYG{l+m+mf}{0.5}\PYG{p}{,} \PYG{l+m+mf}{0.7}\PYG{p}{]}\PYG{p}{,} \PYG{l+m+mf}{0.26}\PYG{p}{)}
\end{sphinxVerbatim}

Find psinorm values at volnorm=0.5 at times t={[}0.2s, 0.3s{]}:

\begin{sphinxVerbatim}[commandchars=\\\{\}]
\PYG{n}{psinorm\PYGZus{}arr} \PYG{o}{=} \PYG{n}{Eq\PYGZus{}instance}\PYG{o}{.}\PYG{n}{volnorm2psinorm}\PYG{p}{(}\PYG{l+m+mf}{0.5}\PYG{p}{,} \PYG{p}{[}\PYG{l+m+mf}{0.2}\PYG{p}{,} \PYG{l+m+mf}{0.3}\PYG{p}{]}\PYG{p}{)}
\end{sphinxVerbatim}

Find psinorm values at (volnorm, t) points (0.6, 0.2s) and (0.5, 0.3s):

\begin{sphinxVerbatim}[commandchars=\\\{\}]
\PYG{n}{psinorm\PYGZus{}arr} \PYG{o}{=} \PYG{n}{Eq\PYGZus{}instance}\PYG{o}{.}\PYG{n}{volnorm2psinorm}\PYG{p}{(}\PYG{p}{[}\PYG{l+m+mf}{0.6}\PYG{p}{,} \PYG{l+m+mf}{0.5}\PYG{p}{]}\PYG{p}{,} \PYG{p}{[}\PYG{l+m+mf}{0.2}\PYG{p}{,} \PYG{l+m+mf}{0.3}\PYG{p}{]}\PYG{p}{,} \PYG{n}{each\PYGZus{}t}\PYG{o}{=}\PYG{k+kc}{False}\PYG{p}{)}
\end{sphinxVerbatim}

\end{fulllineitems}

\index{volnorm2phinorm() (eqtools.core.Equilibrium method)@\spxentry{volnorm2phinorm()}\spxextra{eqtools.core.Equilibrium method}}

\begin{fulllineitems}
\phantomsection\label{\detokenize{eqtools:eqtools.core.Equilibrium.volnorm2phinorm}}\pysiglinewithargsret{\sphinxbfcode{\sphinxupquote{volnorm2phinorm}}}{\emph{*args}, \emph{**kwargs}}{}
Calculates the normalized toroidal flux corresponding to the passed volnorm (normalized flux surface volume) values.
\begin{quote}\begin{description}
\item[{Parameters}] \leavevmode\begin{itemize}
\item {} 
\sphinxstyleliteralstrong{\sphinxupquote{volnorm}} (\sphinxstyleliteralemphasis{\sphinxupquote{Array-like}}\sphinxstyleliteralemphasis{\sphinxupquote{ or }}\sphinxstyleliteralemphasis{\sphinxupquote{scalar float}}) \textendash{} Values of the normalized
flux surface volume to map to phinorm.

\item {} 
\sphinxstyleliteralstrong{\sphinxupquote{t}} (\sphinxstyleliteralemphasis{\sphinxupquote{Array-like}}\sphinxstyleliteralemphasis{\sphinxupquote{ or }}\sphinxstyleliteralemphasis{\sphinxupquote{scalar float}}) \textendash{} Times to perform the conversion at.
If \sphinxtitleref{t} is a single value, it is used for all of the elements of
\sphinxtitleref{volnorm}. If the \sphinxtitleref{each\_t} keyword is True, then \sphinxtitleref{t} must be scalar
or have exactly one dimension. If the \sphinxtitleref{each\_t} keyword is False,
\sphinxtitleref{t} must have the same shape as \sphinxtitleref{volnorm}.

\end{itemize}

\item[{Keyword Arguments}] \leavevmode\begin{itemize}
\item {} 
\sphinxstyleliteralstrong{\sphinxupquote{sqrt}} (\sphinxstyleliteralemphasis{\sphinxupquote{Boolean}}) \textendash{} Set to True to return the square root of phinorm.
Only the square root of positive values is taken. Negative
values are replaced with zeros, consistent with Steve Wolfe’s
IDL implementation efit\_rz2rho.pro. Default is False.

\item {} 
\sphinxstyleliteralstrong{\sphinxupquote{each\_t}} (\sphinxstyleliteralemphasis{\sphinxupquote{Boolean}}) \textendash{} When True, the elements in \sphinxtitleref{volnorm} are evaluated
at each value in \sphinxtitleref{t}. If True, \sphinxtitleref{t} must have only one dimension
(or be a scalar). If False, \sphinxtitleref{t} must match the shape of \sphinxtitleref{volnorm}
or be a scalar. Default is True (evaluate ALL \sphinxtitleref{volnorm} at EACH
element in \sphinxtitleref{t}).

\item {} 
\sphinxstyleliteralstrong{\sphinxupquote{k}} (\sphinxstyleliteralemphasis{\sphinxupquote{positive int}}) \textendash{} The degree of polynomial spline interpolation to
use in converting coordinates.

\item {} 
\sphinxstyleliteralstrong{\sphinxupquote{return\_t}} (\sphinxstyleliteralemphasis{\sphinxupquote{Boolean}}) \textendash{} Set to True to return a tuple of (\sphinxtitleref{rho},
\sphinxtitleref{time\_idxs}), where \sphinxtitleref{time\_idxs} is the array of time indices
actually used in evaluating \sphinxtitleref{rho} with nearest-neighbor
interpolation. (This is mostly present as an internal helper.)
Default is False (only return \sphinxtitleref{rho}).

\end{itemize}

\item[{Returns}] \leavevmode

\sphinxtitleref{phinorm} or (\sphinxtitleref{phinorm}, \sphinxtitleref{time\_idxs})
\begin{itemize}
\item {} 
\sphinxstylestrong{phinorm} (\sphinxtitleref{Array or scalar float}) - The converted coordinates. If
all of the input arguments are scalar, then a scalar is returned.
Otherwise, a scipy Array is returned.

\item {} 
\sphinxstylestrong{time\_idxs} (Array with same shape as \sphinxtitleref{phinorm}) - The indices
(in \sphinxcode{\sphinxupquote{self.getTimeBase()}}) that were used for
nearest-neighbor interpolation. Only returned if \sphinxtitleref{return\_t} is
True.

\end{itemize}


\end{description}\end{quote}
\subsubsection*{Examples}

All assume that \sphinxtitleref{Eq\_instance} is a valid instance of the appropriate
extension of the {\hyperref[\detokenize{eqtools:eqtools.core.Equilibrium}]{\sphinxcrossref{\sphinxcode{\sphinxupquote{Equilibrium}}}}} abstract class.

Find single phinorm value for volnorm=0.7, t=0.26s:

\begin{sphinxVerbatim}[commandchars=\\\{\}]
\PYG{n}{phinorm\PYGZus{}val} \PYG{o}{=} \PYG{n}{Eq\PYGZus{}instance}\PYG{o}{.}\PYG{n}{volnorm2phinorm}\PYG{p}{(}\PYG{l+m+mf}{0.7}\PYG{p}{,} \PYG{l+m+mf}{0.26}\PYG{p}{)}
\end{sphinxVerbatim}

Find phinorm values at volnorm values of 0.5 and 0.7 at the single time
t=0.26s:

\begin{sphinxVerbatim}[commandchars=\\\{\}]
\PYG{n}{phinorm\PYGZus{}arr} \PYG{o}{=} \PYG{n}{Eq\PYGZus{}instance}\PYG{o}{.}\PYG{n}{volnorm2phinorm}\PYG{p}{(}\PYG{p}{[}\PYG{l+m+mf}{0.5}\PYG{p}{,} \PYG{l+m+mf}{0.7}\PYG{p}{]}\PYG{p}{,} \PYG{l+m+mf}{0.26}\PYG{p}{)}
\end{sphinxVerbatim}

Find phinorm values at volnorm=0.5 at times t={[}0.2s, 0.3s{]}:

\begin{sphinxVerbatim}[commandchars=\\\{\}]
\PYG{n}{phinorm\PYGZus{}arr} \PYG{o}{=} \PYG{n}{Eq\PYGZus{}instance}\PYG{o}{.}\PYG{n}{volnorm2phinorm}\PYG{p}{(}\PYG{l+m+mf}{0.5}\PYG{p}{,} \PYG{p}{[}\PYG{l+m+mf}{0.2}\PYG{p}{,} \PYG{l+m+mf}{0.3}\PYG{p}{]}\PYG{p}{)}
\end{sphinxVerbatim}

Find phinorm values at (volnorm, t) points (0.6, 0.2s) and (0.5, 0.3s):

\begin{sphinxVerbatim}[commandchars=\\\{\}]
\PYG{n}{phinorm\PYGZus{}arr} \PYG{o}{=} \PYG{n}{Eq\PYGZus{}instance}\PYG{o}{.}\PYG{n}{volnorm2phinorm}\PYG{p}{(}\PYG{p}{[}\PYG{l+m+mf}{0.6}\PYG{p}{,} \PYG{l+m+mf}{0.5}\PYG{p}{]}\PYG{p}{,} \PYG{p}{[}\PYG{l+m+mf}{0.2}\PYG{p}{,} \PYG{l+m+mf}{0.3}\PYG{p}{]}\PYG{p}{,} \PYG{n}{each\PYGZus{}t}\PYG{o}{=}\PYG{k+kc}{False}\PYG{p}{)}
\end{sphinxVerbatim}

\end{fulllineitems}

\index{volnorm2rmid() (eqtools.core.Equilibrium method)@\spxentry{volnorm2rmid()}\spxextra{eqtools.core.Equilibrium method}}

\begin{fulllineitems}
\phantomsection\label{\detokenize{eqtools:eqtools.core.Equilibrium.volnorm2rmid}}\pysiglinewithargsret{\sphinxbfcode{\sphinxupquote{volnorm2rmid}}}{\emph{*args}, \emph{**kwargs}}{}
Calculates the mapped outboard midplane major radius corresponding to the passed volnorm (normalized flux surface volume) values.
\begin{quote}\begin{description}
\item[{Parameters}] \leavevmode\begin{itemize}
\item {} 
\sphinxstyleliteralstrong{\sphinxupquote{volnorm}} (\sphinxstyleliteralemphasis{\sphinxupquote{Array-like}}\sphinxstyleliteralemphasis{\sphinxupquote{ or }}\sphinxstyleliteralemphasis{\sphinxupquote{scalar float}}) \textendash{} Values of the normalized
flux surface volume to map to Rmid.

\item {} 
\sphinxstyleliteralstrong{\sphinxupquote{t}} (\sphinxstyleliteralemphasis{\sphinxupquote{Array-like}}\sphinxstyleliteralemphasis{\sphinxupquote{ or }}\sphinxstyleliteralemphasis{\sphinxupquote{scalar float}}) \textendash{} Times to perform the conversion at.
If \sphinxtitleref{t} is a single value, it is used for all of the elements of
\sphinxtitleref{volnorm}. If the \sphinxtitleref{each\_t} keyword is True, then \sphinxtitleref{t} must be scalar
or have exactly one dimension. If the \sphinxtitleref{each\_t} keyword is False,
\sphinxtitleref{t} must have the same shape as \sphinxtitleref{volnorm}.

\end{itemize}

\item[{Keyword Arguments}] \leavevmode\begin{itemize}
\item {} 
\sphinxstyleliteralstrong{\sphinxupquote{sqrt}} (\sphinxstyleliteralemphasis{\sphinxupquote{Boolean}}) \textendash{} Set to True to return the square root of Rmid.
Only the square root of positive values is taken. Negative
values are replaced with zeros, consistent with Steve Wolfe’s
IDL implementation efit\_rz2rho.pro. Default is False.

\item {} 
\sphinxstyleliteralstrong{\sphinxupquote{each\_t}} (\sphinxstyleliteralemphasis{\sphinxupquote{Boolean}}) \textendash{} When True, the elements in \sphinxtitleref{volnorm} are evaluated
at each value in \sphinxtitleref{t}. If True, \sphinxtitleref{t} must have only one dimension
(or be a scalar). If False, \sphinxtitleref{t} must match the shape of \sphinxtitleref{volnorm}
or be a scalar. Default is True (evaluate ALL \sphinxtitleref{volnorm} at EACH
element in \sphinxtitleref{t}).

\item {} 
\sphinxstyleliteralstrong{\sphinxupquote{rho}} (\sphinxstyleliteralemphasis{\sphinxupquote{Boolean}}) \textendash{} Set to True to return r/a (normalized minor radius)
instead of Rmid. Default is False (return major radius, Rmid).

\item {} 
\sphinxstyleliteralstrong{\sphinxupquote{length\_unit}} (\sphinxstyleliteralemphasis{\sphinxupquote{String}}\sphinxstyleliteralemphasis{\sphinxupquote{ or }}\sphinxstyleliteralemphasis{\sphinxupquote{1}}) \textendash{} 
Length unit that \sphinxtitleref{Rmid} is returned in.
If a string is given, it must be a valid unit specifier:
\begin{quote}


\begin{savenotes}\sphinxattablestart
\centering
\begin{tabulary}{\linewidth}[t]{|T|T|}
\hline

’m’
&
meters
\\
\hline
’cm’
&
centimeters
\\
\hline
’mm’
&
millimeters
\\
\hline
’in’
&
inches
\\
\hline
’ft’
&
feet
\\
\hline
’yd’
&
yards
\\
\hline
’smoot’
&
smoots
\\
\hline
’cubit’
&
cubits
\\
\hline
’hand’
&
hands
\\
\hline
’default’
&
meters
\\
\hline
\end{tabulary}
\par
\sphinxattableend\end{savenotes}
\end{quote}

If length\_unit is 1 or None, meters are assumed. The default
value is 1 (use meters).


\item {} 
\sphinxstyleliteralstrong{\sphinxupquote{k}} (\sphinxstyleliteralemphasis{\sphinxupquote{positive int}}) \textendash{} The degree of polynomial spline interpolation to
use in converting coordinates.

\item {} 
\sphinxstyleliteralstrong{\sphinxupquote{return\_t}} (\sphinxstyleliteralemphasis{\sphinxupquote{Boolean}}) \textendash{} Set to True to return a tuple of (\sphinxtitleref{rho},
\sphinxtitleref{time\_idxs}), where \sphinxtitleref{time\_idxs} is the array of time indices
actually used in evaluating \sphinxtitleref{rho} with nearest-neighbor
interpolation. (This is mostly present as an internal helper.)
Default is False (only return \sphinxtitleref{rho}).

\end{itemize}

\item[{Returns}] \leavevmode

\sphinxtitleref{Rmid} or (\sphinxtitleref{Rmid}, \sphinxtitleref{time\_idxs})
\begin{itemize}
\item {} 
\sphinxstylestrong{Rmid} (\sphinxtitleref{Array or scalar float}) - The converted coordinates. If
all of the input arguments are scalar, then a scalar is returned.
Otherwise, a scipy Array is returned.

\item {} 
\sphinxstylestrong{time\_idxs} (Array with same shape as \sphinxtitleref{Rmid}) - The indices
(in \sphinxcode{\sphinxupquote{self.getTimeBase()}}) that were used for
nearest-neighbor interpolation. Only returned if \sphinxtitleref{return\_t} is
True.

\end{itemize}


\end{description}\end{quote}
\subsubsection*{Examples}

All assume that \sphinxtitleref{Eq\_instance} is a valid instance of the appropriate
extension of the {\hyperref[\detokenize{eqtools:eqtools.core.Equilibrium}]{\sphinxcrossref{\sphinxcode{\sphinxupquote{Equilibrium}}}}} abstract class.

Find single Rmid value for volnorm=0.7, t=0.26s:

\begin{sphinxVerbatim}[commandchars=\\\{\}]
\PYG{n}{Rmid\PYGZus{}val} \PYG{o}{=} \PYG{n}{Eq\PYGZus{}instance}\PYG{o}{.}\PYG{n}{volnorm2rmid}\PYG{p}{(}\PYG{l+m+mf}{0.7}\PYG{p}{,} \PYG{l+m+mf}{0.26}\PYG{p}{)}
\end{sphinxVerbatim}

Find Rmid values at volnorm values of 0.5 and 0.7 at the single time
t=0.26s:

\begin{sphinxVerbatim}[commandchars=\\\{\}]
\PYG{n}{Rmid\PYGZus{}arr} \PYG{o}{=} \PYG{n}{Eq\PYGZus{}instance}\PYG{o}{.}\PYG{n}{volnorm2rmid}\PYG{p}{(}\PYG{p}{[}\PYG{l+m+mf}{0.5}\PYG{p}{,} \PYG{l+m+mf}{0.7}\PYG{p}{]}\PYG{p}{,} \PYG{l+m+mf}{0.26}\PYG{p}{)}
\end{sphinxVerbatim}

Find Rmid values at volnorm=0.5 at times t={[}0.2s, 0.3s{]}:

\begin{sphinxVerbatim}[commandchars=\\\{\}]
\PYG{n}{Rmid\PYGZus{}arr} \PYG{o}{=} \PYG{n}{Eq\PYGZus{}instance}\PYG{o}{.}\PYG{n}{volnorm2rmid}\PYG{p}{(}\PYG{l+m+mf}{0.5}\PYG{p}{,} \PYG{p}{[}\PYG{l+m+mf}{0.2}\PYG{p}{,} \PYG{l+m+mf}{0.3}\PYG{p}{]}\PYG{p}{)}
\end{sphinxVerbatim}

Find Rmid values at (volnorm, t) points (0.6, 0.2s) and (0.5, 0.3s):

\begin{sphinxVerbatim}[commandchars=\\\{\}]
\PYG{n}{Rmid\PYGZus{}arr} \PYG{o}{=} \PYG{n}{Eq\PYGZus{}instance}\PYG{o}{.}\PYG{n}{volnorm2rmid}\PYG{p}{(}\PYG{p}{[}\PYG{l+m+mf}{0.6}\PYG{p}{,} \PYG{l+m+mf}{0.5}\PYG{p}{]}\PYG{p}{,} \PYG{p}{[}\PYG{l+m+mf}{0.2}\PYG{p}{,} \PYG{l+m+mf}{0.3}\PYG{p}{]}\PYG{p}{,} \PYG{n}{each\PYGZus{}t}\PYG{o}{=}\PYG{k+kc}{False}\PYG{p}{)}
\end{sphinxVerbatim}

\end{fulllineitems}

\index{volnorm2roa() (eqtools.core.Equilibrium method)@\spxentry{volnorm2roa()}\spxextra{eqtools.core.Equilibrium method}}

\begin{fulllineitems}
\phantomsection\label{\detokenize{eqtools:eqtools.core.Equilibrium.volnorm2roa}}\pysiglinewithargsret{\sphinxbfcode{\sphinxupquote{volnorm2roa}}}{\emph{*args}, \emph{**kwargs}}{}
Calculates the normalized minor radius corresponding to the passed volnorm (normalized flux surface volume) values.
\begin{quote}\begin{description}
\item[{Parameters}] \leavevmode\begin{itemize}
\item {} 
\sphinxstyleliteralstrong{\sphinxupquote{volnorm}} (\sphinxstyleliteralemphasis{\sphinxupquote{Array-like}}\sphinxstyleliteralemphasis{\sphinxupquote{ or }}\sphinxstyleliteralemphasis{\sphinxupquote{scalar float}}) \textendash{} Values of the normalized
flux surface volume to map to r/a.

\item {} 
\sphinxstyleliteralstrong{\sphinxupquote{t}} (\sphinxstyleliteralemphasis{\sphinxupquote{Array-like}}\sphinxstyleliteralemphasis{\sphinxupquote{ or }}\sphinxstyleliteralemphasis{\sphinxupquote{scalar float}}) \textendash{} Times to perform the conversion at.
If \sphinxtitleref{t} is a single value, it is used for all of the elements of
\sphinxtitleref{volnorm}. If the \sphinxtitleref{each\_t} keyword is True, then \sphinxtitleref{t} must be scalar
or have exactly one dimension. If the \sphinxtitleref{each\_t} keyword is False,
\sphinxtitleref{t} must have the same shape as \sphinxtitleref{volnorm}.

\end{itemize}

\item[{Keyword Arguments}] \leavevmode\begin{itemize}
\item {} 
\sphinxstyleliteralstrong{\sphinxupquote{sqrt}} (\sphinxstyleliteralemphasis{\sphinxupquote{Boolean}}) \textendash{} Set to True to return the square root of r/a.
Only the square root of positive values is taken. Negative
values are replaced with zeros, consistent with Steve Wolfe’s
IDL implementation efit\_rz2rho.pro. Default is False.

\item {} 
\sphinxstyleliteralstrong{\sphinxupquote{each\_t}} (\sphinxstyleliteralemphasis{\sphinxupquote{Boolean}}) \textendash{} When True, the elements in \sphinxtitleref{volnorm} are evaluated
at each value in \sphinxtitleref{t}. If True, \sphinxtitleref{t} must have only one dimension
(or be a scalar). If False, \sphinxtitleref{t} must match the shape of \sphinxtitleref{volnorm}
or be a scalar. Default is True (evaluate ALL \sphinxtitleref{volnorm} at EACH
element in \sphinxtitleref{t}).

\item {} 
\sphinxstyleliteralstrong{\sphinxupquote{k}} (\sphinxstyleliteralemphasis{\sphinxupquote{positive int}}) \textendash{} The degree of polynomial spline interpolation to
use in converting coordinates.

\item {} 
\sphinxstyleliteralstrong{\sphinxupquote{return\_t}} (\sphinxstyleliteralemphasis{\sphinxupquote{Boolean}}) \textendash{} Set to True to return a tuple of (\sphinxtitleref{rho},
\sphinxtitleref{time\_idxs}), where \sphinxtitleref{time\_idxs} is the array of time indices
actually used in evaluating \sphinxtitleref{rho} with nearest-neighbor
interpolation. (This is mostly present as an internal helper.)
Default is False (only return \sphinxtitleref{rho}).

\end{itemize}

\item[{Returns}] \leavevmode

\sphinxtitleref{roa} or (\sphinxtitleref{roa}, \sphinxtitleref{time\_idxs})
\begin{itemize}
\item {} 
\sphinxstylestrong{roa} (\sphinxtitleref{Array or scalar float}) - The converted coordinates. If
all of the input arguments are scalar, then a scalar is returned.
Otherwise, a scipy Array is returned.

\item {} 
\sphinxstylestrong{time\_idxs} (Array with same shape as \sphinxtitleref{roa}) - The indices
(in \sphinxcode{\sphinxupquote{self.getTimeBase()}}) that were used for
nearest-neighbor interpolation. Only returned if \sphinxtitleref{return\_t} is
True.

\end{itemize}


\end{description}\end{quote}
\subsubsection*{Examples}

All assume that \sphinxtitleref{Eq\_instance} is a valid instance of the appropriate
extension of the {\hyperref[\detokenize{eqtools:eqtools.core.Equilibrium}]{\sphinxcrossref{\sphinxcode{\sphinxupquote{Equilibrium}}}}} abstract class.

Find single r/a value for volnorm=0.7, t=0.26s:

\begin{sphinxVerbatim}[commandchars=\\\{\}]
\PYG{n}{roa\PYGZus{}val} \PYG{o}{=} \PYG{n}{Eq\PYGZus{}instance}\PYG{o}{.}\PYG{n}{volnorm2roa}\PYG{p}{(}\PYG{l+m+mf}{0.7}\PYG{p}{,} \PYG{l+m+mf}{0.26}\PYG{p}{)}
\end{sphinxVerbatim}

Find r/a values at volnorm values of 0.5 and 0.7 at the single time
t=0.26s:

\begin{sphinxVerbatim}[commandchars=\\\{\}]
\PYG{n}{roa\PYGZus{}arr} \PYG{o}{=} \PYG{n}{Eq\PYGZus{}instance}\PYG{o}{.}\PYG{n}{volnorm2roa}\PYG{p}{(}\PYG{p}{[}\PYG{l+m+mf}{0.5}\PYG{p}{,} \PYG{l+m+mf}{0.7}\PYG{p}{]}\PYG{p}{,} \PYG{l+m+mf}{0.26}\PYG{p}{)}
\end{sphinxVerbatim}

Find r/a values at volnorm=0.5 at times t={[}0.2s, 0.3s{]}:

\begin{sphinxVerbatim}[commandchars=\\\{\}]
\PYG{n}{roa\PYGZus{}arr} \PYG{o}{=} \PYG{n}{Eq\PYGZus{}instance}\PYG{o}{.}\PYG{n}{volnorm2roa}\PYG{p}{(}\PYG{l+m+mf}{0.5}\PYG{p}{,} \PYG{p}{[}\PYG{l+m+mf}{0.2}\PYG{p}{,} \PYG{l+m+mf}{0.3}\PYG{p}{]}\PYG{p}{)}
\end{sphinxVerbatim}

Find r/a values at (volnorm, t) points (0.6, 0.2s) and (0.5, 0.3s):

\begin{sphinxVerbatim}[commandchars=\\\{\}]
\PYG{n}{roa\PYGZus{}arr} \PYG{o}{=} \PYG{n}{Eq\PYGZus{}instance}\PYG{o}{.}\PYG{n}{volnorm2roa}\PYG{p}{(}\PYG{p}{[}\PYG{l+m+mf}{0.6}\PYG{p}{,} \PYG{l+m+mf}{0.5}\PYG{p}{]}\PYG{p}{,} \PYG{p}{[}\PYG{l+m+mf}{0.2}\PYG{p}{,} \PYG{l+m+mf}{0.3}\PYG{p}{]}\PYG{p}{,} \PYG{n}{each\PYGZus{}t}\PYG{o}{=}\PYG{k+kc}{False}\PYG{p}{)}
\end{sphinxVerbatim}

\end{fulllineitems}

\index{volnorm2rho() (eqtools.core.Equilibrium method)@\spxentry{volnorm2rho()}\spxextra{eqtools.core.Equilibrium method}}

\begin{fulllineitems}
\phantomsection\label{\detokenize{eqtools:eqtools.core.Equilibrium.volnorm2rho}}\pysiglinewithargsret{\sphinxbfcode{\sphinxupquote{volnorm2rho}}}{\emph{method}, \emph{*args}, \emph{**kwargs}}{}
Convert the passed (volnorm, t) coordinates into one of several coordinates.
\begin{quote}\begin{description}
\item[{Parameters}] \leavevmode\begin{itemize}
\item {} 
\sphinxstyleliteralstrong{\sphinxupquote{method}} (\sphinxstyleliteralemphasis{\sphinxupquote{String}}) \textendash{} 
Indicates which coordinates to convert to.
Valid options are:
\begin{quote}


\begin{savenotes}\sphinxattablestart
\centering
\begin{tabulary}{\linewidth}[t]{|T|T|}
\hline

psinorm
&
Normalized poloidal flux
\\
\hline
phinorm
&
Normalized toroidal flux
\\
\hline
Rmid
&
Midplane major radius
\\
\hline
r/a
&
Normalized minor radius
\\
\hline
q
&
Safety factor
\\
\hline
F
&
Flux function \(F=RB_{\phi}\)
\\
\hline
FFPrime
&
Flux function \(FF'\)
\\
\hline
p
&
Pressure
\\
\hline
pprime
&
Pressure gradient
\\
\hline
v
&
Flux surface volume
\\
\hline
\end{tabulary}
\par
\sphinxattableend\end{savenotes}
\end{quote}

Additionally, each valid option may be prepended with ‘sqrt’
to specify the square root of the desired unit.


\item {} 
\sphinxstyleliteralstrong{\sphinxupquote{volnorm}} (\sphinxstyleliteralemphasis{\sphinxupquote{Array-like}}\sphinxstyleliteralemphasis{\sphinxupquote{ or }}\sphinxstyleliteralemphasis{\sphinxupquote{scalar float}}) \textendash{} Values of the normalized
flux surface volume to map to rho.

\item {} 
\sphinxstyleliteralstrong{\sphinxupquote{t}} (\sphinxstyleliteralemphasis{\sphinxupquote{Array-like}}\sphinxstyleliteralemphasis{\sphinxupquote{ or }}\sphinxstyleliteralemphasis{\sphinxupquote{scalar float}}) \textendash{} Times to perform the conversion at.
If \sphinxtitleref{t} is a single value, it is used for all of the elements of
\sphinxtitleref{volnorm}. If the \sphinxtitleref{each\_t} keyword is True, then \sphinxtitleref{t} must be scalar
or have exactly one dimension. If the \sphinxtitleref{each\_t} keyword is False,
\sphinxtitleref{t} must have the same shape as \sphinxtitleref{volnorm}.

\end{itemize}

\item[{Keyword Arguments}] \leavevmode\begin{itemize}
\item {} 
\sphinxstyleliteralstrong{\sphinxupquote{sqrt}} (\sphinxstyleliteralemphasis{\sphinxupquote{Boolean}}) \textendash{} Set to True to return the square root of rho. Only
the square root of positive values is taken. Negative values are
replaced with zeros, consistent with Steve Wolfe’s IDL
implementation efit\_rz2rho.pro. Default is False.

\item {} 
\sphinxstyleliteralstrong{\sphinxupquote{each\_t}} (\sphinxstyleliteralemphasis{\sphinxupquote{Boolean}}) \textendash{} When True, the elements in \sphinxtitleref{volnorm} are evaluated at
each value in \sphinxtitleref{t}. If True, \sphinxtitleref{t} must have only one dimension (or
be a scalar). If False, \sphinxtitleref{t} must match the shape of \sphinxtitleref{volnorm} or be
a scalar. Default is True (evaluate ALL \sphinxtitleref{volnorm} at EACH element in
\sphinxtitleref{t}).

\item {} 
\sphinxstyleliteralstrong{\sphinxupquote{rho}} (\sphinxstyleliteralemphasis{\sphinxupquote{Boolean}}) \textendash{} Set to True to return r/a (normalized minor radius)
instead of Rmid. Default is False (return major radius, Rmid).

\item {} 
\sphinxstyleliteralstrong{\sphinxupquote{length\_unit}} (\sphinxstyleliteralemphasis{\sphinxupquote{String}}\sphinxstyleliteralemphasis{\sphinxupquote{ or }}\sphinxstyleliteralemphasis{\sphinxupquote{1}}) \textendash{} 
Length unit that \sphinxtitleref{Rmid} is returned in.
If a string is given, it must be a valid unit specifier:
\begin{quote}


\begin{savenotes}\sphinxattablestart
\centering
\begin{tabulary}{\linewidth}[t]{|T|T|}
\hline

’m’
&
meters
\\
\hline
’cm’
&
centimeters
\\
\hline
’mm’
&
millimeters
\\
\hline
’in’
&
inches
\\
\hline
’ft’
&
feet
\\
\hline
’yd’
&
yards
\\
\hline
’smoot’
&
smoots
\\
\hline
’cubit’
&
cubits
\\
\hline
’hand’
&
hands
\\
\hline
’default’
&
meters
\\
\hline
\end{tabulary}
\par
\sphinxattableend\end{savenotes}
\end{quote}

If length\_unit is 1 or None, meters are assumed. The default
value is 1 (use meters).


\item {} 
\sphinxstyleliteralstrong{\sphinxupquote{k}} (\sphinxstyleliteralemphasis{\sphinxupquote{positive int}}) \textendash{} The degree of polynomial spline interpolation to
use in converting coordinates.

\item {} 
\sphinxstyleliteralstrong{\sphinxupquote{return\_t}} (\sphinxstyleliteralemphasis{\sphinxupquote{Boolean}}) \textendash{} Set to True to return a tuple of (\sphinxtitleref{rho},
\sphinxtitleref{time\_idxs}), where \sphinxtitleref{time\_idxs} is the array of time indices
actually used in evaluating \sphinxtitleref{rho} with nearest-neighbor
interpolation. (This is mostly present as an internal helper.)
Default is False (only return \sphinxtitleref{rho}).

\end{itemize}

\item[{Returns}] \leavevmode

\sphinxtitleref{rho} or (\sphinxtitleref{rho}, \sphinxtitleref{time\_idxs})
\begin{itemize}
\item {} 
\sphinxstylestrong{rho} (\sphinxtitleref{Array or scalar float}) - The converted coordinates. If
all of the input arguments are scalar, then a scalar is returned.
Otherwise, a scipy Array is returned.

\item {} 
\sphinxstylestrong{time\_idxs} (Array with same shape as \sphinxtitleref{rho}) - The indices
(in \sphinxcode{\sphinxupquote{self.getTimeBase()}}) that were used for
nearest-neighbor interpolation. Only returned if \sphinxtitleref{return\_t} is
True.

\end{itemize}


\item[{Raises}] \leavevmode
\sphinxstyleliteralstrong{\sphinxupquote{ValueError}} \textendash{} If \sphinxtitleref{method} is not one of the supported values.

\end{description}\end{quote}
\subsubsection*{Examples}

All assume that \sphinxtitleref{Eq\_instance} is a valid instance of the appropriate
extension of the {\hyperref[\detokenize{eqtools:eqtools.core.Equilibrium}]{\sphinxcrossref{\sphinxcode{\sphinxupquote{Equilibrium}}}}} abstract class.

Find single psinorm value at volnorm=0.6, t=0.26s:

\begin{sphinxVerbatim}[commandchars=\\\{\}]
\PYG{n}{psi\PYGZus{}val} \PYG{o}{=} \PYG{n}{Eq\PYGZus{}instance}\PYG{o}{.}\PYG{n}{volnorm2rho}\PYG{p}{(}\PYG{l+s+s1}{\PYGZsq{}}\PYG{l+s+s1}{psinorm}\PYG{l+s+s1}{\PYGZsq{}}\PYG{p}{,} \PYG{l+m+mf}{0.6}\PYG{p}{,} \PYG{l+m+mf}{0.26}\PYG{p}{)}
\end{sphinxVerbatim}

Find psinorm values at volnorm of 0.6 and 0.8 at the
single time t=0.26s:

\begin{sphinxVerbatim}[commandchars=\\\{\}]
\PYG{n}{psi\PYGZus{}arr} \PYG{o}{=} \PYG{n}{Eq\PYGZus{}instance}\PYG{o}{.}\PYG{n}{volnorm2rho}\PYG{p}{(}\PYG{l+s+s1}{\PYGZsq{}}\PYG{l+s+s1}{psinorm}\PYG{l+s+s1}{\PYGZsq{}}\PYG{p}{,} \PYG{p}{[}\PYG{l+m+mf}{0.6}\PYG{p}{,} \PYG{l+m+mf}{0.8}\PYG{p}{]}\PYG{p}{,} \PYG{l+m+mf}{0.26}\PYG{p}{)}
\end{sphinxVerbatim}

Find psinorm values at volnorm of 0.6 at times t={[}0.2s, 0.3s{]}:

\begin{sphinxVerbatim}[commandchars=\\\{\}]
\PYG{n}{psi\PYGZus{}arr} \PYG{o}{=} \PYG{n}{Eq\PYGZus{}instance}\PYG{o}{.}\PYG{n}{volnorm2rho}\PYG{p}{(}\PYG{l+s+s1}{\PYGZsq{}}\PYG{l+s+s1}{psinorm}\PYG{l+s+s1}{\PYGZsq{}}\PYG{p}{,} \PYG{l+m+mf}{0.6}\PYG{p}{,} \PYG{p}{[}\PYG{l+m+mf}{0.2}\PYG{p}{,} \PYG{l+m+mf}{0.3}\PYG{p}{]}\PYG{p}{)}
\end{sphinxVerbatim}

Find psinorm values at (volnorm, t) points (0.6, 0.2s) and (0.5m, 0.3s):

\begin{sphinxVerbatim}[commandchars=\\\{\}]
\PYG{n}{psi\PYGZus{}arr} \PYG{o}{=} \PYG{n}{Eq\PYGZus{}instance}\PYG{o}{.}\PYG{n}{volnorm2rho}\PYG{p}{(}\PYG{l+s+s1}{\PYGZsq{}}\PYG{l+s+s1}{psinorm}\PYG{l+s+s1}{\PYGZsq{}}\PYG{p}{,} \PYG{p}{[}\PYG{l+m+mf}{0.6}\PYG{p}{,} \PYG{l+m+mf}{0.5}\PYG{p}{]}\PYG{p}{,} \PYG{p}{[}\PYG{l+m+mf}{0.2}\PYG{p}{,} \PYG{l+m+mf}{0.3}\PYG{p}{]}\PYG{p}{,} \PYG{n}{each\PYGZus{}t}\PYG{o}{=}\PYG{k+kc}{False}\PYG{p}{)}
\end{sphinxVerbatim}

\end{fulllineitems}

\index{rz2q() (eqtools.core.Equilibrium method)@\spxentry{rz2q()}\spxextra{eqtools.core.Equilibrium method}}

\begin{fulllineitems}
\phantomsection\label{\detokenize{eqtools:eqtools.core.Equilibrium.rz2q}}\pysiglinewithargsret{\sphinxbfcode{\sphinxupquote{rz2q}}}{\emph{R}, \emph{Z}, \emph{t}, \emph{**kwargs}}{}
Calculates the safety factor (“q”) at the given (R, Z, t).

By default, EFIT only computes this inside the LCFS.
\begin{quote}\begin{description}
\item[{Parameters}] \leavevmode\begin{itemize}
\item {} 
\sphinxstyleliteralstrong{\sphinxupquote{R}} (\sphinxstyleliteralemphasis{\sphinxupquote{Array-like}}\sphinxstyleliteralemphasis{\sphinxupquote{ or }}\sphinxstyleliteralemphasis{\sphinxupquote{scalar float}}) \textendash{} Values of the radial coordinate to
map to q. If \sphinxtitleref{R} and \sphinxtitleref{Z} are both scalar values,
they are used as the coordinate pair for all of the values in
\sphinxtitleref{t}. Must have the same shape as \sphinxtitleref{Z} unless the \sphinxtitleref{make\_grid}
keyword is set. If the \sphinxtitleref{make\_grid} keyword is True, \sphinxtitleref{R} must
have exactly one dimension.

\item {} 
\sphinxstyleliteralstrong{\sphinxupquote{Z}} (\sphinxstyleliteralemphasis{\sphinxupquote{Array-like}}\sphinxstyleliteralemphasis{\sphinxupquote{ or }}\sphinxstyleliteralemphasis{\sphinxupquote{scalar float}}) \textendash{} Values of the vertical coordinate to
map to q. If \sphinxtitleref{R} and \sphinxtitleref{Z} are both scalar values,
they are used as the coordinate pair for all of the values in
\sphinxtitleref{t}. Must have the same shape as \sphinxtitleref{R} unless the \sphinxtitleref{make\_grid}
keyword is set. If the \sphinxtitleref{make\_grid} keyword is True, \sphinxtitleref{Z} must
have exactly one dimension.

\item {} 
\sphinxstyleliteralstrong{\sphinxupquote{t}} (\sphinxstyleliteralemphasis{\sphinxupquote{Array-like}}\sphinxstyleliteralemphasis{\sphinxupquote{ or }}\sphinxstyleliteralemphasis{\sphinxupquote{scalar float}}) \textendash{} Times to perform the conversion at.
If \sphinxtitleref{t} is a single value, it is used for all of the elements of
\sphinxtitleref{R}, \sphinxtitleref{Z}. If the \sphinxtitleref{each\_t} keyword is True, then \sphinxtitleref{t} must be
scalar or have exactly one dimension. If the \sphinxtitleref{each\_t} keyword is
False, \sphinxtitleref{t} must have the same shape as \sphinxtitleref{R} and \sphinxtitleref{Z} (or their
meshgrid if \sphinxtitleref{make\_grid} is True).

\end{itemize}

\item[{Keyword Arguments}] \leavevmode\begin{itemize}
\item {} 
\sphinxstyleliteralstrong{\sphinxupquote{sqrt}} (\sphinxstyleliteralemphasis{\sphinxupquote{Boolean}}) \textendash{} Set to True to return the square root of q.
Only the square root of positive values is taken. Negative
values are replaced with zeros, consistent with Steve Wolfe’s
IDL implementation efit\_rz2rho.pro. Default is False.

\item {} 
\sphinxstyleliteralstrong{\sphinxupquote{each\_t}} (\sphinxstyleliteralemphasis{\sphinxupquote{Boolean}}) \textendash{} When True, the elements in \sphinxtitleref{R}, \sphinxtitleref{Z} are evaluated
at each value in \sphinxtitleref{t}. If True, \sphinxtitleref{t} must have only one dimension
(or be a scalar). If False, \sphinxtitleref{t} must match the shape of \sphinxtitleref{R} and
\sphinxtitleref{Z} or be a scalar. Default is True (evaluate ALL \sphinxtitleref{R}, \sphinxtitleref{Z} at
EACH element in \sphinxtitleref{t}).

\item {} 
\sphinxstyleliteralstrong{\sphinxupquote{make\_grid}} (\sphinxstyleliteralemphasis{\sphinxupquote{Boolean}}) \textendash{} Set to True to pass \sphinxtitleref{R} and \sphinxtitleref{Z} through
\sphinxcode{\sphinxupquote{scipy.meshgrid()}} before evaluating. If this is set to
True, \sphinxtitleref{R} and \sphinxtitleref{Z} must each only have a single dimension, but
can have different lengths. Default is False (do not form
meshgrid).

\item {} 
\sphinxstyleliteralstrong{\sphinxupquote{length\_unit}} (\sphinxstyleliteralemphasis{\sphinxupquote{String}}\sphinxstyleliteralemphasis{\sphinxupquote{ or }}\sphinxstyleliteralemphasis{\sphinxupquote{1}}) \textendash{} 
Length unit that \sphinxtitleref{R}, \sphinxtitleref{Z} are given in.
If a string is given, it must be a valid unit specifier:
\begin{quote}


\begin{savenotes}\sphinxattablestart
\centering
\begin{tabulary}{\linewidth}[t]{|T|T|}
\hline

’m’
&
meters
\\
\hline
’cm’
&
centimeters
\\
\hline
’mm’
&
millimeters
\\
\hline
’in’
&
inches
\\
\hline
’ft’
&
feet
\\
\hline
’yd’
&
yards
\\
\hline
’smoot’
&
smoots
\\
\hline
’cubit’
&
cubits
\\
\hline
’hand’
&
hands
\\
\hline
’default’
&
meters
\\
\hline
\end{tabulary}
\par
\sphinxattableend\end{savenotes}
\end{quote}

If length\_unit is 1 or None, meters are assumed. The default
value is 1 (use meters).


\item {} 
\sphinxstyleliteralstrong{\sphinxupquote{return\_t}} (\sphinxstyleliteralemphasis{\sphinxupquote{Boolean}}) \textendash{} Set to True to return a tuple of (\sphinxtitleref{q},
\sphinxtitleref{time\_idxs}), where \sphinxtitleref{time\_idxs} is the array of time indices
actually used in evaluating \sphinxtitleref{q} with nearest-neighbor
interpolation. (This is mostly present as an internal helper.)
Default is False (only return \sphinxtitleref{q}).

\end{itemize}

\item[{Returns}] \leavevmode

\sphinxtitleref{q} or (\sphinxtitleref{q}, \sphinxtitleref{time\_idxs})
\begin{itemize}
\item {} 
\sphinxstylestrong{q} (\sphinxtitleref{Array or scalar float}) - The safety factor (“q”). If all
of the input arguments are scalar, then a scalar is
returned. Otherwise, a scipy Array is returned. If \sphinxtitleref{R} and \sphinxtitleref{Z}
both have the same shape then \sphinxtitleref{q} has this shape as well,
unless the \sphinxtitleref{make\_grid} keyword was True, in which case \sphinxtitleref{q}
has shape (len(\sphinxtitleref{Z}), len(\sphinxtitleref{R})).

\item {} 
\sphinxstylestrong{time\_idxs} (Array with same shape as \sphinxtitleref{q}) - The indices
(in \sphinxcode{\sphinxupquote{self.getTimeBase()}}) that were used for
nearest-neighbor interpolation. Only returned if \sphinxtitleref{return\_t} is
True.

\end{itemize}


\end{description}\end{quote}
\subsubsection*{Examples}

All assume that \sphinxtitleref{Eq\_instance} is a valid instance of the
appropriate extension of the {\hyperref[\detokenize{eqtools:eqtools.core.Equilibrium}]{\sphinxcrossref{\sphinxcode{\sphinxupquote{Equilibrium}}}}} abstract class.

Find single q value at R=0.6m, Z=0.0m, t=0.26s:

\begin{sphinxVerbatim}[commandchars=\\\{\}]
\PYG{n}{q\PYGZus{}val} \PYG{o}{=} \PYG{n}{Eq\PYGZus{}instance}\PYG{o}{.}\PYG{n}{rz2q}\PYG{p}{(}\PYG{l+m+mf}{0.6}\PYG{p}{,} \PYG{l+m+mi}{0}\PYG{p}{,} \PYG{l+m+mf}{0.26}\PYG{p}{)}
\end{sphinxVerbatim}

Find q values at (R, Z) points (0.6m, 0m) and (0.8m, 0m) at the
single time t=0.26s. Note that the \sphinxtitleref{Z} vector must be fully specified,
even if the values are all the same:

\begin{sphinxVerbatim}[commandchars=\\\{\}]
\PYG{n}{q\PYGZus{}arr} \PYG{o}{=} \PYG{n}{Eq\PYGZus{}instance}\PYG{o}{.}\PYG{n}{rz2q}\PYG{p}{(}\PYG{p}{[}\PYG{l+m+mf}{0.6}\PYG{p}{,} \PYG{l+m+mf}{0.8}\PYG{p}{]}\PYG{p}{,} \PYG{p}{[}\PYG{l+m+mi}{0}\PYG{p}{,} \PYG{l+m+mi}{0}\PYG{p}{]}\PYG{p}{,} \PYG{l+m+mf}{0.26}\PYG{p}{)}
\end{sphinxVerbatim}

Find q values at (R, Z) points (0.6m, 0m) at times t={[}0.2s, 0.3s{]}:

\begin{sphinxVerbatim}[commandchars=\\\{\}]
\PYG{n}{q\PYGZus{}arr} \PYG{o}{=} \PYG{n}{Eq\PYGZus{}instance}\PYG{o}{.}\PYG{n}{rz2q}\PYG{p}{(}\PYG{l+m+mf}{0.6}\PYG{p}{,} \PYG{l+m+mi}{0}\PYG{p}{,} \PYG{p}{[}\PYG{l+m+mf}{0.2}\PYG{p}{,} \PYG{l+m+mf}{0.3}\PYG{p}{]}\PYG{p}{)}
\end{sphinxVerbatim}

Find q values at (R, Z, t) points (0.6m, 0m, 0.2s) and (0.5m, 0.2m, 0.3s):

\begin{sphinxVerbatim}[commandchars=\\\{\}]
\PYG{n}{q\PYGZus{}arr} \PYG{o}{=} \PYG{n}{Eq\PYGZus{}instance}\PYG{o}{.}\PYG{n}{rz2q}\PYG{p}{(}\PYG{p}{[}\PYG{l+m+mf}{0.6}\PYG{p}{,} \PYG{l+m+mf}{0.5}\PYG{p}{]}\PYG{p}{,} \PYG{p}{[}\PYG{l+m+mi}{0}\PYG{p}{,} \PYG{l+m+mf}{0.2}\PYG{p}{]}\PYG{p}{,} \PYG{p}{[}\PYG{l+m+mf}{0.2}\PYG{p}{,} \PYG{l+m+mf}{0.3}\PYG{p}{]}\PYG{p}{,} \PYG{n}{each\PYGZus{}t}\PYG{o}{=}\PYG{k+kc}{False}\PYG{p}{)}
\end{sphinxVerbatim}

Find q values on grid defined by 1D vector of radial positions \sphinxtitleref{R}
and 1D vector of vertical positions \sphinxtitleref{Z} at time t=0.2s:

\begin{sphinxVerbatim}[commandchars=\\\{\}]
\PYG{n}{q\PYGZus{}mat} \PYG{o}{=} \PYG{n}{Eq\PYGZus{}instance}\PYG{o}{.}\PYG{n}{rz2q}\PYG{p}{(}\PYG{n}{R}\PYG{p}{,} \PYG{n}{Z}\PYG{p}{,} \PYG{l+m+mf}{0.2}\PYG{p}{,} \PYG{n}{make\PYGZus{}grid}\PYG{o}{=}\PYG{k+kc}{True}\PYG{p}{)}
\end{sphinxVerbatim}

\end{fulllineitems}

\index{rmid2q() (eqtools.core.Equilibrium method)@\spxentry{rmid2q()}\spxextra{eqtools.core.Equilibrium method}}

\begin{fulllineitems}
\phantomsection\label{\detokenize{eqtools:eqtools.core.Equilibrium.rmid2q}}\pysiglinewithargsret{\sphinxbfcode{\sphinxupquote{rmid2q}}}{\emph{R\_mid}, \emph{t}, \emph{**kwargs}}{}
Calculates the safety factor (“q”) corresponding to the passed R\_mid (mapped outboard midplane major radius) values.

By default, EFIT only computes this inside the LCFS.
\begin{quote}\begin{description}
\item[{Parameters}] \leavevmode\begin{itemize}
\item {} 
\sphinxstyleliteralstrong{\sphinxupquote{R\_mid}} (\sphinxstyleliteralemphasis{\sphinxupquote{Array-like}}\sphinxstyleliteralemphasis{\sphinxupquote{ or }}\sphinxstyleliteralemphasis{\sphinxupquote{scalar float}}) \textendash{} Values of the outboard midplane
major radius to map to q.

\item {} 
\sphinxstyleliteralstrong{\sphinxupquote{t}} (\sphinxstyleliteralemphasis{\sphinxupquote{Array-like}}\sphinxstyleliteralemphasis{\sphinxupquote{ or }}\sphinxstyleliteralemphasis{\sphinxupquote{scalar float}}) \textendash{} Times to perform the conversion at.
If \sphinxtitleref{t} is a single value, it is used for all of the elements of
\sphinxtitleref{R\_mid}. If the \sphinxtitleref{each\_t} keyword is True, then \sphinxtitleref{t} must be scalar
or have exactly one dimension. If the \sphinxtitleref{each\_t} keyword is False,
\sphinxtitleref{t} must have the same shape as \sphinxtitleref{R\_mid}.

\end{itemize}

\item[{Keyword Arguments}] \leavevmode\begin{itemize}
\item {} 
\sphinxstyleliteralstrong{\sphinxupquote{sqrt}} (\sphinxstyleliteralemphasis{\sphinxupquote{Boolean}}) \textendash{} Set to True to return the square root of q.
Only the square root of positive values is taken. Negative
values are replaced with zeros, consistent with Steve Wolfe’s
IDL implementation efit\_rz2rho.pro. Default is False.

\item {} 
\sphinxstyleliteralstrong{\sphinxupquote{each\_t}} (\sphinxstyleliteralemphasis{\sphinxupquote{Boolean}}) \textendash{} When True, the elements in \sphinxtitleref{R\_mid} are evaluated
at each value in \sphinxtitleref{t}. If True, \sphinxtitleref{t} must have only one dimension
(or be a scalar). If False, \sphinxtitleref{t} must match the shape of \sphinxtitleref{R\_mid}
or be a scalar. Default is True (evaluate ALL \sphinxtitleref{R\_mid} at EACH
element in \sphinxtitleref{t}).

\item {} 
\sphinxstyleliteralstrong{\sphinxupquote{length\_unit}} (\sphinxstyleliteralemphasis{\sphinxupquote{String}}\sphinxstyleliteralemphasis{\sphinxupquote{ or }}\sphinxstyleliteralemphasis{\sphinxupquote{1}}) \textendash{} 
Length unit that \sphinxtitleref{R\_mid} is given in.
If a string is given, it must be a valid unit specifier:
\begin{quote}


\begin{savenotes}\sphinxattablestart
\centering
\begin{tabulary}{\linewidth}[t]{|T|T|}
\hline

’m’
&
meters
\\
\hline
’cm’
&
centimeters
\\
\hline
’mm’
&
millimeters
\\
\hline
’in’
&
inches
\\
\hline
’ft’
&
feet
\\
\hline
’yd’
&
yards
\\
\hline
’smoot’
&
smoots
\\
\hline
’cubit’
&
cubits
\\
\hline
’hand’
&
hands
\\
\hline
’default’
&
meters
\\
\hline
\end{tabulary}
\par
\sphinxattableend\end{savenotes}
\end{quote}

If length\_unit is 1 or None, meters are assumed. The default
value is 1 (use meters).


\item {} 
\sphinxstyleliteralstrong{\sphinxupquote{k}} (\sphinxstyleliteralemphasis{\sphinxupquote{positive int}}) \textendash{} The degree of polynomial spline interpolation to
use in converting coordinates.

\item {} 
\sphinxstyleliteralstrong{\sphinxupquote{return\_t}} (\sphinxstyleliteralemphasis{\sphinxupquote{Boolean}}) \textendash{} Set to True to return a tuple of (\sphinxtitleref{q},
\sphinxtitleref{time\_idxs}), where \sphinxtitleref{time\_idxs} is the array of time indices
actually used in evaluating \sphinxtitleref{q} with nearest-neighbor
interpolation. (This is mostly present as an internal helper.)
Default is False (only return \sphinxtitleref{q}).

\end{itemize}

\item[{Returns}] \leavevmode

\sphinxtitleref{q} or (\sphinxtitleref{q}, \sphinxtitleref{time\_idxs})
\begin{itemize}
\item {} 
\sphinxstylestrong{q} (\sphinxtitleref{Array or scalar float}) - The safety factor (“q”).
If all of the input arguments are scalar, then a scalar is
returned. Otherwise, a scipy Array is returned.

\item {} 
\sphinxstylestrong{time\_idxs} (Array with same shape as \sphinxtitleref{q}) - The indices
(in \sphinxcode{\sphinxupquote{self.getTimeBase()}}) that were used for
nearest-neighbor interpolation. Only returned if \sphinxtitleref{return\_t} is
True.

\end{itemize}


\end{description}\end{quote}
\subsubsection*{Examples}

All assume that \sphinxtitleref{Eq\_instance} is a valid instance of the appropriate
extension of the {\hyperref[\detokenize{eqtools:eqtools.core.Equilibrium}]{\sphinxcrossref{\sphinxcode{\sphinxupquote{Equilibrium}}}}} abstract class.

Find single q value for Rmid=0.7m, t=0.26s:

\begin{sphinxVerbatim}[commandchars=\\\{\}]
\PYG{n}{q\PYGZus{}val} \PYG{o}{=} \PYG{n}{Eq\PYGZus{}instance}\PYG{o}{.}\PYG{n}{rmid2q}\PYG{p}{(}\PYG{l+m+mf}{0.7}\PYG{p}{,} \PYG{l+m+mf}{0.26}\PYG{p}{)}
\end{sphinxVerbatim}

Find q values at R\_mid values of 0.5m and 0.7m at the single time
t=0.26s:

\begin{sphinxVerbatim}[commandchars=\\\{\}]
\PYG{n}{q\PYGZus{}arr} \PYG{o}{=} \PYG{n}{Eq\PYGZus{}instance}\PYG{o}{.}\PYG{n}{rmid2q}\PYG{p}{(}\PYG{p}{[}\PYG{l+m+mf}{0.5}\PYG{p}{,} \PYG{l+m+mf}{0.7}\PYG{p}{]}\PYG{p}{,} \PYG{l+m+mf}{0.26}\PYG{p}{)}
\end{sphinxVerbatim}

Find q values at R\_mid=0.5m at times t={[}0.2s, 0.3s{]}:

\begin{sphinxVerbatim}[commandchars=\\\{\}]
\PYG{n}{q\PYGZus{}arr} \PYG{o}{=} \PYG{n}{Eq\PYGZus{}instance}\PYG{o}{.}\PYG{n}{rmid2q}\PYG{p}{(}\PYG{l+m+mf}{0.5}\PYG{p}{,} \PYG{p}{[}\PYG{l+m+mf}{0.2}\PYG{p}{,} \PYG{l+m+mf}{0.3}\PYG{p}{]}\PYG{p}{)}
\end{sphinxVerbatim}

Find q values at (R\_mid, t) points (0.6m, 0.2s) and (0.5m, 0.3s):

\begin{sphinxVerbatim}[commandchars=\\\{\}]
\PYG{n}{q\PYGZus{}arr} \PYG{o}{=} \PYG{n}{Eq\PYGZus{}instance}\PYG{o}{.}\PYG{n}{rmid2q}\PYG{p}{(}\PYG{p}{[}\PYG{l+m+mf}{0.6}\PYG{p}{,} \PYG{l+m+mf}{0.5}\PYG{p}{]}\PYG{p}{,} \PYG{p}{[}\PYG{l+m+mf}{0.2}\PYG{p}{,} \PYG{l+m+mf}{0.3}\PYG{p}{]}\PYG{p}{,} \PYG{n}{each\PYGZus{}t}\PYG{o}{=}\PYG{k+kc}{False}\PYG{p}{)}
\end{sphinxVerbatim}

\end{fulllineitems}

\index{roa2q() (eqtools.core.Equilibrium method)@\spxentry{roa2q()}\spxextra{eqtools.core.Equilibrium method}}

\begin{fulllineitems}
\phantomsection\label{\detokenize{eqtools:eqtools.core.Equilibrium.roa2q}}\pysiglinewithargsret{\sphinxbfcode{\sphinxupquote{roa2q}}}{\emph{roa}, \emph{t}, \emph{**kwargs}}{}
Convert the passed (r/a, t) coordinates into safety factor (“q”).

By default, EFIT only computes this inside the LCFS.
\begin{quote}\begin{description}
\item[{Parameters}] \leavevmode\begin{itemize}
\item {} 
\sphinxstyleliteralstrong{\sphinxupquote{roa}} (\sphinxstyleliteralemphasis{\sphinxupquote{Array-like}}\sphinxstyleliteralemphasis{\sphinxupquote{ or }}\sphinxstyleliteralemphasis{\sphinxupquote{scalar float}}) \textendash{} Values of the normalized minor
radius to map to q.

\item {} 
\sphinxstyleliteralstrong{\sphinxupquote{t}} (\sphinxstyleliteralemphasis{\sphinxupquote{Array-like}}\sphinxstyleliteralemphasis{\sphinxupquote{ or }}\sphinxstyleliteralemphasis{\sphinxupquote{scalar float}}) \textendash{} Times to perform the conversion at.
If \sphinxtitleref{t} is a single value, it is used for all of the elements of
\sphinxtitleref{roa}. If the \sphinxtitleref{each\_t} keyword is True, then \sphinxtitleref{t} must be scalar
or have exactly one dimension. If the \sphinxtitleref{each\_t} keyword is False,
\sphinxtitleref{t} must have the same shape as \sphinxtitleref{roa}.

\end{itemize}

\item[{Keyword Arguments}] \leavevmode\begin{itemize}
\item {} 
\sphinxstyleliteralstrong{\sphinxupquote{sqrt}} (\sphinxstyleliteralemphasis{\sphinxupquote{Boolean}}) \textendash{} Set to True to return the square root of q.
Only the square root of positive values is taken. Negative
values are replaced with zeros, consistent with Steve Wolfe’s
IDL implementation efit\_rz2rho.pro. Default is False.

\item {} 
\sphinxstyleliteralstrong{\sphinxupquote{each\_t}} (\sphinxstyleliteralemphasis{\sphinxupquote{Boolean}}) \textendash{} When True, the elements in \sphinxtitleref{roa} are evaluated
at each value in \sphinxtitleref{t}. If True, \sphinxtitleref{t} must have only one dimension
(or be a scalar). If False, \sphinxtitleref{t} must match the shape of \sphinxtitleref{roa}
or be a scalar. Default is True (evaluate ALL \sphinxtitleref{roa} at EACH
element in \sphinxtitleref{t}).

\item {} 
\sphinxstyleliteralstrong{\sphinxupquote{k}} (\sphinxstyleliteralemphasis{\sphinxupquote{positive int}}) \textendash{} The degree of polynomial spline interpolation to
use in converting coordinates.

\item {} 
\sphinxstyleliteralstrong{\sphinxupquote{return\_t}} (\sphinxstyleliteralemphasis{\sphinxupquote{Boolean}}) \textendash{} Set to True to return a tuple of (\sphinxtitleref{q},
\sphinxtitleref{time\_idxs}), where \sphinxtitleref{time\_idxs} is the array of time indices
actually used in evaluating \sphinxtitleref{q} with nearest-neighbor
interpolation. (This is mostly present as an internal helper.)
Default is False (only return \sphinxtitleref{q}).

\end{itemize}

\item[{Returns}] \leavevmode

\sphinxtitleref{q} or (\sphinxtitleref{q}, \sphinxtitleref{time\_idxs})
\begin{itemize}
\item {} 
\sphinxstylestrong{q} (\sphinxtitleref{Array or scalar float}) - The safety factor (“q”). If
all of the input arguments are scalar, then a scalar is returned.
Otherwise, a scipy Array is returned.

\item {} 
\sphinxstylestrong{time\_idxs} (Array with same shape as \sphinxtitleref{q}) - The indices
(in \sphinxcode{\sphinxupquote{self.getTimeBase()}}) that were used for
nearest-neighbor interpolation. Only returned if \sphinxtitleref{return\_t} is
True.

\end{itemize}


\end{description}\end{quote}
\subsubsection*{Examples}

All assume that \sphinxtitleref{Eq\_instance} is a valid instance of the appropriate
extension of the {\hyperref[\detokenize{eqtools:eqtools.core.Equilibrium}]{\sphinxcrossref{\sphinxcode{\sphinxupquote{Equilibrium}}}}} abstract class.

Find single q value at r/a=0.6, t=0.26s:

\begin{sphinxVerbatim}[commandchars=\\\{\}]
\PYG{n}{q\PYGZus{}val} \PYG{o}{=} \PYG{n}{Eq\PYGZus{}instance}\PYG{o}{.}\PYG{n}{roa2q}\PYG{p}{(}\PYG{l+m+mf}{0.6}\PYG{p}{,} \PYG{l+m+mf}{0.26}\PYG{p}{)}
\end{sphinxVerbatim}

Find q values at r/a points 0.6 and 0.8 at the
single time t=0.26s.:

\begin{sphinxVerbatim}[commandchars=\\\{\}]
\PYG{n}{q\PYGZus{}arr} \PYG{o}{=} \PYG{n}{Eq\PYGZus{}instance}\PYG{o}{.}\PYG{n}{roa2q}\PYG{p}{(}\PYG{p}{[}\PYG{l+m+mf}{0.6}\PYG{p}{,} \PYG{l+m+mf}{0.8}\PYG{p}{]}\PYG{p}{,} \PYG{l+m+mf}{0.26}\PYG{p}{)}
\end{sphinxVerbatim}

Find q values at r/a of 0.6 at times t={[}0.2s, 0.3s{]}:

\begin{sphinxVerbatim}[commandchars=\\\{\}]
\PYG{n}{q\PYGZus{}arr} \PYG{o}{=} \PYG{n}{Eq\PYGZus{}instance}\PYG{o}{.}\PYG{n}{roa2q}\PYG{p}{(}\PYG{l+m+mf}{0.6}\PYG{p}{,} \PYG{p}{[}\PYG{l+m+mf}{0.2}\PYG{p}{,} \PYG{l+m+mf}{0.3}\PYG{p}{]}\PYG{p}{)}
\end{sphinxVerbatim}

Find q values at (roa, t) points (0.6, 0.2s) and (0.5, 0.3s):

\begin{sphinxVerbatim}[commandchars=\\\{\}]
\PYG{n}{q\PYGZus{}arr} \PYG{o}{=} \PYG{n}{Eq\PYGZus{}instance}\PYG{o}{.}\PYG{n}{roa2q}\PYG{p}{(}\PYG{p}{[}\PYG{l+m+mf}{0.6}\PYG{p}{,} \PYG{l+m+mf}{0.5}\PYG{p}{]}\PYG{p}{,} \PYG{p}{[}\PYG{l+m+mf}{0.2}\PYG{p}{,} \PYG{l+m+mf}{0.3}\PYG{p}{]}\PYG{p}{,} \PYG{n}{each\PYGZus{}t}\PYG{o}{=}\PYG{k+kc}{False}\PYG{p}{)}
\end{sphinxVerbatim}

\end{fulllineitems}

\index{psinorm2q() (eqtools.core.Equilibrium method)@\spxentry{psinorm2q()}\spxextra{eqtools.core.Equilibrium method}}

\begin{fulllineitems}
\phantomsection\label{\detokenize{eqtools:eqtools.core.Equilibrium.psinorm2q}}\pysiglinewithargsret{\sphinxbfcode{\sphinxupquote{psinorm2q}}}{\emph{psinorm}, \emph{t}, \emph{**kwargs}}{}
Calculates the safety factor (“q”) corresponding to the passed psi\_norm (normalized poloidal flux) values.

By default, EFIT only computes this inside the LCFS.
\begin{quote}\begin{description}
\item[{Parameters}] \leavevmode\begin{itemize}
\item {} 
\sphinxstyleliteralstrong{\sphinxupquote{psi\_norm}} (\sphinxstyleliteralemphasis{\sphinxupquote{Array-like}}\sphinxstyleliteralemphasis{\sphinxupquote{ or }}\sphinxstyleliteralemphasis{\sphinxupquote{scalar float}}) \textendash{} Values of the normalized
poloidal flux to map to q.

\item {} 
\sphinxstyleliteralstrong{\sphinxupquote{t}} (\sphinxstyleliteralemphasis{\sphinxupquote{Array-like}}\sphinxstyleliteralemphasis{\sphinxupquote{ or }}\sphinxstyleliteralemphasis{\sphinxupquote{scalar float}}) \textendash{} Times to perform the conversion at.
If \sphinxtitleref{t} is a single value, it is used for all of the elements of
\sphinxtitleref{psi\_norm}. If the \sphinxtitleref{each\_t} keyword is True, then \sphinxtitleref{t} must be scalar
or have exactly one dimension. If the \sphinxtitleref{each\_t} keyword is False,
\sphinxtitleref{t} must have the same shape as \sphinxtitleref{psi\_norm}.

\end{itemize}

\item[{Keyword Arguments}] \leavevmode\begin{itemize}
\item {} 
\sphinxstyleliteralstrong{\sphinxupquote{sqrt}} (\sphinxstyleliteralemphasis{\sphinxupquote{Boolean}}) \textendash{} Set to True to return the square root of q. Only
the square root of positive values is taken. Negative values are
replaced with zeros, consistent with Steve Wolfe’s IDL
implementation efit\_rz2rho.pro. Default is False.

\item {} 
\sphinxstyleliteralstrong{\sphinxupquote{each\_t}} (\sphinxstyleliteralemphasis{\sphinxupquote{Boolean}}) \textendash{} When True, the elements in \sphinxtitleref{psi\_norm} are evaluated at
each value in \sphinxtitleref{t}. If True, \sphinxtitleref{t} must have only one dimension (or
be a scalar). If False, \sphinxtitleref{t} must match the shape of \sphinxtitleref{psi\_norm} or be
a scalar. Default is True (evaluate ALL \sphinxtitleref{psi\_norm} at EACH element in
\sphinxtitleref{t}).

\item {} 
\sphinxstyleliteralstrong{\sphinxupquote{k}} (\sphinxstyleliteralemphasis{\sphinxupquote{positive int}}) \textendash{} The degree of polynomial spline interpolation to
use in converting coordinates.

\item {} 
\sphinxstyleliteralstrong{\sphinxupquote{return\_t}} (\sphinxstyleliteralemphasis{\sphinxupquote{Boolean}}) \textendash{} Set to True to return a tuple of (\sphinxtitleref{q},
\sphinxtitleref{time\_idxs}), where \sphinxtitleref{time\_idxs} is the array of time indices
actually used in evaluating \sphinxtitleref{q} with nearest-neighbor
interpolation. (This is mostly present as an internal helper.)
Default is False (only return \sphinxtitleref{q}).

\end{itemize}

\item[{Returns}] \leavevmode

\sphinxtitleref{q} or (\sphinxtitleref{q}, \sphinxtitleref{time\_idxs})
\begin{itemize}
\item {} 
\sphinxstylestrong{q} (\sphinxtitleref{Array or scalar float}) - The safety factor (“q”). If
all of the input arguments are scalar, then a scalar is returned.
Otherwise, a scipy Array is returned.

\item {} 
\sphinxstylestrong{time\_idxs} (Array with same shape as \sphinxtitleref{q}) - The indices
(in \sphinxcode{\sphinxupquote{self.getTimeBase()}}) that were used for
nearest-neighbor interpolation. Only returned if \sphinxtitleref{return\_t} is
True.

\end{itemize}


\end{description}\end{quote}
\subsubsection*{Examples}

All assume that \sphinxtitleref{Eq\_instance} is a valid instance of the appropriate
extension of the {\hyperref[\detokenize{eqtools:eqtools.core.Equilibrium}]{\sphinxcrossref{\sphinxcode{\sphinxupquote{Equilibrium}}}}} abstract class.

Find single q value for psinorm=0.7, t=0.26s:

\begin{sphinxVerbatim}[commandchars=\\\{\}]
\PYG{n}{q\PYGZus{}val} \PYG{o}{=} \PYG{n}{Eq\PYGZus{}instance}\PYG{o}{.}\PYG{n}{psinorm2q}\PYG{p}{(}\PYG{l+m+mf}{0.7}\PYG{p}{,} \PYG{l+m+mf}{0.26}\PYG{p}{)}
\end{sphinxVerbatim}

Find q values at psi\_norm values of 0.5 and 0.7 at the single time
t=0.26s:

\begin{sphinxVerbatim}[commandchars=\\\{\}]
\PYG{n}{q\PYGZus{}arr} \PYG{o}{=} \PYG{n}{Eq\PYGZus{}instance}\PYG{o}{.}\PYG{n}{psinorm2q}\PYG{p}{(}\PYG{p}{[}\PYG{l+m+mf}{0.5}\PYG{p}{,} \PYG{l+m+mf}{0.7}\PYG{p}{]}\PYG{p}{,} \PYG{l+m+mf}{0.26}\PYG{p}{)}
\end{sphinxVerbatim}

Find q values at psi\_norm=0.5 at times t={[}0.2s, 0.3s{]}:

\begin{sphinxVerbatim}[commandchars=\\\{\}]
\PYG{n}{q\PYGZus{}arr} \PYG{o}{=} \PYG{n}{Eq\PYGZus{}instance}\PYG{o}{.}\PYG{n}{psinorm2q}\PYG{p}{(}\PYG{l+m+mf}{0.5}\PYG{p}{,} \PYG{p}{[}\PYG{l+m+mf}{0.2}\PYG{p}{,} \PYG{l+m+mf}{0.3}\PYG{p}{]}\PYG{p}{)}
\end{sphinxVerbatim}

Find q values at (psinorm, t) points (0.6, 0.2s) and (0.5, 0.3s):

\begin{sphinxVerbatim}[commandchars=\\\{\}]
\PYG{n}{q\PYGZus{}arr} \PYG{o}{=} \PYG{n}{Eq\PYGZus{}instance}\PYG{o}{.}\PYG{n}{psinorm2q}\PYG{p}{(}\PYG{p}{[}\PYG{l+m+mf}{0.6}\PYG{p}{,} \PYG{l+m+mf}{0.5}\PYG{p}{]}\PYG{p}{,} \PYG{p}{[}\PYG{l+m+mf}{0.2}\PYG{p}{,} \PYG{l+m+mf}{0.3}\PYG{p}{]}\PYG{p}{,} \PYG{n}{each\PYGZus{}t}\PYG{o}{=}\PYG{k+kc}{False}\PYG{p}{)}
\end{sphinxVerbatim}

\end{fulllineitems}

\index{phinorm2q() (eqtools.core.Equilibrium method)@\spxentry{phinorm2q()}\spxextra{eqtools.core.Equilibrium method}}

\begin{fulllineitems}
\phantomsection\label{\detokenize{eqtools:eqtools.core.Equilibrium.phinorm2q}}\pysiglinewithargsret{\sphinxbfcode{\sphinxupquote{phinorm2q}}}{\emph{phinorm}, \emph{t}, \emph{**kwargs}}{}
Calculates the safety factor (“q”) corresponding to the passed phinorm (normalized toroidal flux) values.

By default, EFIT only computes this inside the LCFS.
\begin{quote}\begin{description}
\item[{Parameters}] \leavevmode\begin{itemize}
\item {} 
\sphinxstyleliteralstrong{\sphinxupquote{phinorm}} (\sphinxstyleliteralemphasis{\sphinxupquote{Array-like}}\sphinxstyleliteralemphasis{\sphinxupquote{ or }}\sphinxstyleliteralemphasis{\sphinxupquote{scalar float}}) \textendash{} Values of the normalized
toroidal flux to map to q.

\item {} 
\sphinxstyleliteralstrong{\sphinxupquote{t}} (\sphinxstyleliteralemphasis{\sphinxupquote{Array-like}}\sphinxstyleliteralemphasis{\sphinxupquote{ or }}\sphinxstyleliteralemphasis{\sphinxupquote{scalar float}}) \textendash{} Times to perform the conversion at.
If \sphinxtitleref{t} is a single value, it is used for all of the elements of
\sphinxtitleref{phinorm}. If the \sphinxtitleref{each\_t} keyword is True, then \sphinxtitleref{t} must be scalar
or have exactly one dimension. If the \sphinxtitleref{each\_t} keyword is False,
\sphinxtitleref{t} must have the same shape as \sphinxtitleref{phinorm}.

\end{itemize}

\item[{Keyword Arguments}] \leavevmode\begin{itemize}
\item {} 
\sphinxstyleliteralstrong{\sphinxupquote{sqrt}} (\sphinxstyleliteralemphasis{\sphinxupquote{Boolean}}) \textendash{} Set to True to return the square root of q.
Only the square root of positive values is taken. Negative
values are replaced with zeros, consistent with Steve Wolfe’s
IDL implementation efit\_rz2rho.pro. Default is False.

\item {} 
\sphinxstyleliteralstrong{\sphinxupquote{each\_t}} (\sphinxstyleliteralemphasis{\sphinxupquote{Boolean}}) \textendash{} When True, the elements in \sphinxtitleref{phinorm} are evaluated
at each value in \sphinxtitleref{t}. If True, \sphinxtitleref{t} must have only one dimension
(or be a scalar). If False, \sphinxtitleref{t} must match the shape of \sphinxtitleref{phinorm}
or be a scalar. Default is True (evaluate ALL \sphinxtitleref{phinorm} at EACH
element in \sphinxtitleref{t}).

\item {} 
\sphinxstyleliteralstrong{\sphinxupquote{k}} (\sphinxstyleliteralemphasis{\sphinxupquote{positive int}}) \textendash{} The degree of polynomial spline interpolation to
use in converting coordinates.

\item {} 
\sphinxstyleliteralstrong{\sphinxupquote{return\_t}} (\sphinxstyleliteralemphasis{\sphinxupquote{Boolean}}) \textendash{} Set to True to return a tuple of (\sphinxtitleref{q},
\sphinxtitleref{time\_idxs}), where \sphinxtitleref{time\_idxs} is the array of time indices
actually used in evaluating \sphinxtitleref{q} with nearest-neighbor
interpolation. (This is mostly present as an internal helper.)
Default is False (only return \sphinxtitleref{q}).

\end{itemize}

\item[{Returns}] \leavevmode

\sphinxtitleref{q} or (\sphinxtitleref{q}, \sphinxtitleref{time\_idxs})
\begin{itemize}
\item {} 
\sphinxstylestrong{q} (\sphinxtitleref{Array or scalar float}) - The safety factor (“q”). If
all of the input arguments are scalar, then a scalar is returned.
Otherwise, a scipy Array is returned.

\item {} 
\sphinxstylestrong{time\_idxs} (Array with same shape as \sphinxtitleref{q}) - The indices
(in \sphinxcode{\sphinxupquote{self.getTimeBase()}}) that were used for
nearest-neighbor interpolation. Only returned if \sphinxtitleref{return\_t} is
True.

\end{itemize}


\end{description}\end{quote}
\subsubsection*{Examples}

All assume that \sphinxtitleref{Eq\_instance} is a valid instance of the appropriate
extension of the {\hyperref[\detokenize{eqtools:eqtools.core.Equilibrium}]{\sphinxcrossref{\sphinxcode{\sphinxupquote{Equilibrium}}}}} abstract class.

Find single q value for phinorm=0.7, t=0.26s:

\begin{sphinxVerbatim}[commandchars=\\\{\}]
\PYG{n}{q\PYGZus{}val} \PYG{o}{=} \PYG{n}{Eq\PYGZus{}instance}\PYG{o}{.}\PYG{n}{phinorm2q}\PYG{p}{(}\PYG{l+m+mf}{0.7}\PYG{p}{,} \PYG{l+m+mf}{0.26}\PYG{p}{)}
\end{sphinxVerbatim}

Find q values at phinorm values of 0.5 and 0.7 at the single time
t=0.26s:

\begin{sphinxVerbatim}[commandchars=\\\{\}]
\PYG{n}{q\PYGZus{}arr} \PYG{o}{=} \PYG{n}{Eq\PYGZus{}instance}\PYG{o}{.}\PYG{n}{phinorm2q}\PYG{p}{(}\PYG{p}{[}\PYG{l+m+mf}{0.5}\PYG{p}{,} \PYG{l+m+mf}{0.7}\PYG{p}{]}\PYG{p}{,} \PYG{l+m+mf}{0.26}\PYG{p}{)}
\end{sphinxVerbatim}

Find q values at phinorm=0.5 at times t={[}0.2s, 0.3s{]}:

\begin{sphinxVerbatim}[commandchars=\\\{\}]
\PYG{n}{q\PYGZus{}arr} \PYG{o}{=} \PYG{n}{Eq\PYGZus{}instance}\PYG{o}{.}\PYG{n}{phinorm2q}\PYG{p}{(}\PYG{l+m+mf}{0.5}\PYG{p}{,} \PYG{p}{[}\PYG{l+m+mf}{0.2}\PYG{p}{,} \PYG{l+m+mf}{0.3}\PYG{p}{]}\PYG{p}{)}
\end{sphinxVerbatim}

Find q values at (phinorm, t) points (0.6, 0.2s) and (0.5, 0.3s):

\begin{sphinxVerbatim}[commandchars=\\\{\}]
\PYG{n}{q\PYGZus{}arr} \PYG{o}{=} \PYG{n}{Eq\PYGZus{}instance}\PYG{o}{.}\PYG{n}{phinorm2q}\PYG{p}{(}\PYG{p}{[}\PYG{l+m+mf}{0.6}\PYG{p}{,} \PYG{l+m+mf}{0.5}\PYG{p}{]}\PYG{p}{,} \PYG{p}{[}\PYG{l+m+mf}{0.2}\PYG{p}{,} \PYG{l+m+mf}{0.3}\PYG{p}{]}\PYG{p}{,} \PYG{n}{each\PYGZus{}t}\PYG{o}{=}\PYG{k+kc}{False}\PYG{p}{)}
\end{sphinxVerbatim}

\end{fulllineitems}

\index{volnorm2q() (eqtools.core.Equilibrium method)@\spxentry{volnorm2q()}\spxextra{eqtools.core.Equilibrium method}}

\begin{fulllineitems}
\phantomsection\label{\detokenize{eqtools:eqtools.core.Equilibrium.volnorm2q}}\pysiglinewithargsret{\sphinxbfcode{\sphinxupquote{volnorm2q}}}{\emph{volnorm}, \emph{t}, \emph{**kwargs}}{}
Calculates the safety factor (“q”) corresponding to the passed volnorm (normalized flux surface volume) values.

By default, EFIT only computes this inside the LCFS.
\begin{quote}\begin{description}
\item[{Parameters}] \leavevmode\begin{itemize}
\item {} 
\sphinxstyleliteralstrong{\sphinxupquote{volnorm}} (\sphinxstyleliteralemphasis{\sphinxupquote{Array-like}}\sphinxstyleliteralemphasis{\sphinxupquote{ or }}\sphinxstyleliteralemphasis{\sphinxupquote{scalar float}}) \textendash{} Values of the normalized
flux surface volume to map to q.

\item {} 
\sphinxstyleliteralstrong{\sphinxupquote{t}} (\sphinxstyleliteralemphasis{\sphinxupquote{Array-like}}\sphinxstyleliteralemphasis{\sphinxupquote{ or }}\sphinxstyleliteralemphasis{\sphinxupquote{scalar float}}) \textendash{} Times to perform the conversion at.
If \sphinxtitleref{t} is a single value, it is used for all of the elements of
\sphinxtitleref{volnorm}. If the \sphinxtitleref{each\_t} keyword is True, then \sphinxtitleref{t} must be scalar
or have exactly one dimension. If the \sphinxtitleref{each\_t} keyword is False,
\sphinxtitleref{t} must have the same shape as \sphinxtitleref{volnorm}.

\end{itemize}

\item[{Keyword Arguments}] \leavevmode\begin{itemize}
\item {} 
\sphinxstyleliteralstrong{\sphinxupquote{sqrt}} (\sphinxstyleliteralemphasis{\sphinxupquote{Boolean}}) \textendash{} Set to True to return the square root of q.
Only the square root of positive values is taken. Negative
values are replaced with zeros, consistent with Steve Wolfe’s
IDL implementation efit\_rz2rho.pro. Default is False.

\item {} 
\sphinxstyleliteralstrong{\sphinxupquote{each\_t}} (\sphinxstyleliteralemphasis{\sphinxupquote{Boolean}}) \textendash{} When True, the elements in \sphinxtitleref{volnorm} are evaluated
at each value in \sphinxtitleref{t}. If True, \sphinxtitleref{t} must have only one dimension
(or be a scalar). If False, \sphinxtitleref{t} must match the shape of \sphinxtitleref{volnorm}
or be a scalar. Default is True (evaluate ALL \sphinxtitleref{volnorm} at EACH
element in \sphinxtitleref{t}).

\item {} 
\sphinxstyleliteralstrong{\sphinxupquote{k}} (\sphinxstyleliteralemphasis{\sphinxupquote{positive int}}) \textendash{} The degree of polynomial spline interpolation to
use in converting coordinates.

\item {} 
\sphinxstyleliteralstrong{\sphinxupquote{return\_t}} (\sphinxstyleliteralemphasis{\sphinxupquote{Boolean}}) \textendash{} Set to True to return a tuple of (\sphinxtitleref{q},
\sphinxtitleref{time\_idxs}), where \sphinxtitleref{time\_idxs} is the array of time indices
actually used in evaluating \sphinxtitleref{q} with nearest-neighbor
interpolation. (This is mostly present as an internal helper.)
Default is False (only return \sphinxtitleref{q}).

\end{itemize}

\item[{Returns}] \leavevmode

\sphinxtitleref{q} or (\sphinxtitleref{q}, \sphinxtitleref{time\_idxs})
\begin{itemize}
\item {} 
\sphinxstylestrong{q} (\sphinxtitleref{Array or scalar float}) - The safety factor (“q”). If
all of the input arguments are scalar, then a scalar is returned.
Otherwise, a scipy Array is returned.

\item {} 
\sphinxstylestrong{time\_idxs} (Array with same shape as \sphinxtitleref{q}) - The indices
(in \sphinxcode{\sphinxupquote{self.getTimeBase()}}) that were used for
nearest-neighbor interpolation. Only returned if \sphinxtitleref{return\_t} is
True.

\end{itemize}


\end{description}\end{quote}
\subsubsection*{Examples}

All assume that \sphinxtitleref{Eq\_instance} is a valid instance of the appropriate
extension of the {\hyperref[\detokenize{eqtools:eqtools.core.Equilibrium}]{\sphinxcrossref{\sphinxcode{\sphinxupquote{Equilibrium}}}}} abstract class.

Find single q value for volnorm=0.7, t=0.26s:

\begin{sphinxVerbatim}[commandchars=\\\{\}]
\PYG{n}{q\PYGZus{}val} \PYG{o}{=} \PYG{n}{Eq\PYGZus{}instance}\PYG{o}{.}\PYG{n}{volnorm2q}\PYG{p}{(}\PYG{l+m+mf}{0.7}\PYG{p}{,} \PYG{l+m+mf}{0.26}\PYG{p}{)}
\end{sphinxVerbatim}

Find q values at volnorm values of 0.5 and 0.7 at the single time
t=0.26s:

\begin{sphinxVerbatim}[commandchars=\\\{\}]
\PYG{n}{q\PYGZus{}arr} \PYG{o}{=} \PYG{n}{Eq\PYGZus{}instance}\PYG{o}{.}\PYG{n}{volnorm2q}\PYG{p}{(}\PYG{p}{[}\PYG{l+m+mf}{0.5}\PYG{p}{,} \PYG{l+m+mf}{0.7}\PYG{p}{]}\PYG{p}{,} \PYG{l+m+mf}{0.26}\PYG{p}{)}
\end{sphinxVerbatim}

Find q values at volnorm=0.5 at times t={[}0.2s, 0.3s{]}:

\begin{sphinxVerbatim}[commandchars=\\\{\}]
\PYG{n}{q\PYGZus{}arr} \PYG{o}{=} \PYG{n}{Eq\PYGZus{}instance}\PYG{o}{.}\PYG{n}{volnorm2q}\PYG{p}{(}\PYG{l+m+mf}{0.5}\PYG{p}{,} \PYG{p}{[}\PYG{l+m+mf}{0.2}\PYG{p}{,} \PYG{l+m+mf}{0.3}\PYG{p}{]}\PYG{p}{)}
\end{sphinxVerbatim}

Find q values at (volnorm, t) points (0.6, 0.2s) and (0.5, 0.3s):

\begin{sphinxVerbatim}[commandchars=\\\{\}]
\PYG{n}{q\PYGZus{}arr} \PYG{o}{=} \PYG{n}{Eq\PYGZus{}instance}\PYG{o}{.}\PYG{n}{volnorm2q}\PYG{p}{(}\PYG{p}{[}\PYG{l+m+mf}{0.6}\PYG{p}{,} \PYG{l+m+mf}{0.5}\PYG{p}{]}\PYG{p}{,} \PYG{p}{[}\PYG{l+m+mf}{0.2}\PYG{p}{,} \PYG{l+m+mf}{0.3}\PYG{p}{]}\PYG{p}{,} \PYG{n}{each\PYGZus{}t}\PYG{o}{=}\PYG{k+kc}{False}\PYG{p}{)}
\end{sphinxVerbatim}

\end{fulllineitems}

\index{rz2F() (eqtools.core.Equilibrium method)@\spxentry{rz2F()}\spxextra{eqtools.core.Equilibrium method}}

\begin{fulllineitems}
\phantomsection\label{\detokenize{eqtools:eqtools.core.Equilibrium.rz2F}}\pysiglinewithargsret{\sphinxbfcode{\sphinxupquote{rz2F}}}{\emph{R}, \emph{Z}, \emph{t}, \emph{**kwargs}}{}
Calculates the flux function \(F=RB_{\phi}\) at the given (R, Z, t).

By default, EFIT only computes this inside the LCFS.
\begin{quote}\begin{description}
\item[{Parameters}] \leavevmode\begin{itemize}
\item {} 
\sphinxstyleliteralstrong{\sphinxupquote{R}} (\sphinxstyleliteralemphasis{\sphinxupquote{Array-like}}\sphinxstyleliteralemphasis{\sphinxupquote{ or }}\sphinxstyleliteralemphasis{\sphinxupquote{scalar float}}) \textendash{} Values of the radial coordinate to
map to F. If \sphinxtitleref{R} and \sphinxtitleref{Z} are both scalar values,
they are used as the coordinate pair for all of the values in
\sphinxtitleref{t}. Must have the same shape as \sphinxtitleref{Z} unless the \sphinxtitleref{make\_grid}
keyword is set. If the \sphinxtitleref{make\_grid} keyword is True, \sphinxtitleref{R} must
have exactly one dimension.

\item {} 
\sphinxstyleliteralstrong{\sphinxupquote{Z}} (\sphinxstyleliteralemphasis{\sphinxupquote{Array-like}}\sphinxstyleliteralemphasis{\sphinxupquote{ or }}\sphinxstyleliteralemphasis{\sphinxupquote{scalar float}}) \textendash{} Values of the vertical coordinate to
map to F. If \sphinxtitleref{R} and \sphinxtitleref{Z} are both scalar values,
they are used as the coordinate pair for all of the values in
\sphinxtitleref{t}. Must have the same shape as \sphinxtitleref{R} unless the \sphinxtitleref{make\_grid}
keyword is set. If the \sphinxtitleref{make\_grid} keyword is True, \sphinxtitleref{Z} must
have exactly one dimension.

\item {} 
\sphinxstyleliteralstrong{\sphinxupquote{t}} (\sphinxstyleliteralemphasis{\sphinxupquote{Array-like}}\sphinxstyleliteralemphasis{\sphinxupquote{ or }}\sphinxstyleliteralemphasis{\sphinxupquote{scalar float}}) \textendash{} Times to perform the conversion at.
If \sphinxtitleref{t} is a single value, it is used for all of the elements of
\sphinxtitleref{R}, \sphinxtitleref{Z}. If the \sphinxtitleref{each\_t} keyword is True, then \sphinxtitleref{t} must be
scalar or have exactly one dimension. If the \sphinxtitleref{each\_t} keyword is
False, \sphinxtitleref{t} must have the same shape as \sphinxtitleref{R} and \sphinxtitleref{Z} (or their
meshgrid if \sphinxtitleref{make\_grid} is True).

\end{itemize}

\item[{Keyword Arguments}] \leavevmode\begin{itemize}
\item {} 
\sphinxstyleliteralstrong{\sphinxupquote{sqrt}} (\sphinxstyleliteralemphasis{\sphinxupquote{Boolean}}) \textendash{} Set to True to return the square root of F.
Only the square root of positive values is taken. Negative
values are replaced with zeros, consistent with Steve Wolfe’s
IDL implementation efit\_rz2rho.pro. Default is False.

\item {} 
\sphinxstyleliteralstrong{\sphinxupquote{each\_t}} (\sphinxstyleliteralemphasis{\sphinxupquote{Boolean}}) \textendash{} When True, the elements in \sphinxtitleref{R}, \sphinxtitleref{Z} are evaluated
at each value in \sphinxtitleref{t}. If True, \sphinxtitleref{t} must have only one dimension
(or be a scalar). If False, \sphinxtitleref{t} must match the shape of \sphinxtitleref{R} and
\sphinxtitleref{Z} or be a scalar. Default is True (evaluate ALL \sphinxtitleref{R}, \sphinxtitleref{Z} at
EACH element in \sphinxtitleref{t}).

\item {} 
\sphinxstyleliteralstrong{\sphinxupquote{make\_grid}} (\sphinxstyleliteralemphasis{\sphinxupquote{Boolean}}) \textendash{} Set to True to pass \sphinxtitleref{R} and \sphinxtitleref{Z} through
\sphinxcode{\sphinxupquote{scipy.meshgrid()}} before evaluating. If this is set to
True, \sphinxtitleref{R} and \sphinxtitleref{Z} must each only have a single dimension, but
can have different lengths. Default is False (do not form
meshgrid).

\item {} 
\sphinxstyleliteralstrong{\sphinxupquote{length\_unit}} (\sphinxstyleliteralemphasis{\sphinxupquote{String}}\sphinxstyleliteralemphasis{\sphinxupquote{ or }}\sphinxstyleliteralemphasis{\sphinxupquote{1}}) \textendash{} 
Length unit that \sphinxtitleref{R}, \sphinxtitleref{Z} are given in.
If a string is given, it must be a valid unit specifier:
\begin{quote}


\begin{savenotes}\sphinxattablestart
\centering
\begin{tabulary}{\linewidth}[t]{|T|T|}
\hline

’m’
&
meters
\\
\hline
’cm’
&
centimeters
\\
\hline
’mm’
&
millimeters
\\
\hline
’in’
&
inches
\\
\hline
’ft’
&
feet
\\
\hline
’yd’
&
yards
\\
\hline
’smoot’
&
smoots
\\
\hline
’cubit’
&
cubits
\\
\hline
’hand’
&
hands
\\
\hline
’default’
&
meters
\\
\hline
\end{tabulary}
\par
\sphinxattableend\end{savenotes}
\end{quote}

If length\_unit is 1 or None, meters are assumed. The default
value is 1 (use meters).


\item {} 
\sphinxstyleliteralstrong{\sphinxupquote{return\_t}} (\sphinxstyleliteralemphasis{\sphinxupquote{Boolean}}) \textendash{} Set to True to return a tuple of (\sphinxtitleref{F},
\sphinxtitleref{time\_idxs}), where \sphinxtitleref{time\_idxs} is the array of time indices
actually used in evaluating \sphinxtitleref{F} with nearest-neighbor
interpolation. (This is mostly present as an internal helper.)
Default is False (only return \sphinxtitleref{F}).

\end{itemize}

\item[{Returns}] \leavevmode

\sphinxtitleref{F} or (\sphinxtitleref{F}, \sphinxtitleref{time\_idxs})
\begin{itemize}
\item {} 
\sphinxstylestrong{F} (\sphinxtitleref{Array or scalar float}) - The flux function \(F=RB_{\phi}\).
If all of the input arguments are scalar, then a scalar is
returned. Otherwise, a scipy Array is returned. If \sphinxtitleref{R} and \sphinxtitleref{Z}
both have the same shape then \sphinxtitleref{F} has this shape as well,
unless the \sphinxtitleref{make\_grid} keyword was True, in which case \sphinxtitleref{F}
has shape (len(\sphinxtitleref{Z}), len(\sphinxtitleref{R})).

\item {} 
\sphinxstylestrong{time\_idxs} (Array with same shape as \sphinxtitleref{F}) - The indices
(in \sphinxcode{\sphinxupquote{self.getTimeBase()}}) that were used for
nearest-neighbor interpolation. Only returned if \sphinxtitleref{return\_t} is
True.

\end{itemize}


\end{description}\end{quote}
\subsubsection*{Examples}

All assume that \sphinxtitleref{Eq\_instance} is a valid instance of the
appropriate extension of the {\hyperref[\detokenize{eqtools:eqtools.core.Equilibrium}]{\sphinxcrossref{\sphinxcode{\sphinxupquote{Equilibrium}}}}} abstract class.

Find single F value at R=0.6m, Z=0.0m, t=0.26s:

\begin{sphinxVerbatim}[commandchars=\\\{\}]
\PYG{n}{F\PYGZus{}val} \PYG{o}{=} \PYG{n}{Eq\PYGZus{}instance}\PYG{o}{.}\PYG{n}{rz2F}\PYG{p}{(}\PYG{l+m+mf}{0.6}\PYG{p}{,} \PYG{l+m+mi}{0}\PYG{p}{,} \PYG{l+m+mf}{0.26}\PYG{p}{)}
\end{sphinxVerbatim}

Find F values at (R, Z) points (0.6m, 0m) and (0.8m, 0m) at the
single time t=0.26s. Note that the \sphinxtitleref{Z} vector must be fully specified,
even if the values are all the same:

\begin{sphinxVerbatim}[commandchars=\\\{\}]
\PYG{n}{F\PYGZus{}arr} \PYG{o}{=} \PYG{n}{Eq\PYGZus{}instance}\PYG{o}{.}\PYG{n}{rz2F}\PYG{p}{(}\PYG{p}{[}\PYG{l+m+mf}{0.6}\PYG{p}{,} \PYG{l+m+mf}{0.8}\PYG{p}{]}\PYG{p}{,} \PYG{p}{[}\PYG{l+m+mi}{0}\PYG{p}{,} \PYG{l+m+mi}{0}\PYG{p}{]}\PYG{p}{,} \PYG{l+m+mf}{0.26}\PYG{p}{)}
\end{sphinxVerbatim}

Find F values at (R, Z) points (0.6m, 0m) at times t={[}0.2s, 0.3s{]}:

\begin{sphinxVerbatim}[commandchars=\\\{\}]
\PYG{n}{F\PYGZus{}arr} \PYG{o}{=} \PYG{n}{Eq\PYGZus{}instance}\PYG{o}{.}\PYG{n}{rz2F}\PYG{p}{(}\PYG{l+m+mf}{0.6}\PYG{p}{,} \PYG{l+m+mi}{0}\PYG{p}{,} \PYG{p}{[}\PYG{l+m+mf}{0.2}\PYG{p}{,} \PYG{l+m+mf}{0.3}\PYG{p}{]}\PYG{p}{)}
\end{sphinxVerbatim}

Find F values at (R, Z, t) points (0.6m, 0m, 0.2s) and (0.5m, 0.2m, 0.3s):

\begin{sphinxVerbatim}[commandchars=\\\{\}]
\PYG{n}{F\PYGZus{}arr} \PYG{o}{=} \PYG{n}{Eq\PYGZus{}instance}\PYG{o}{.}\PYG{n}{rz2F}\PYG{p}{(}\PYG{p}{[}\PYG{l+m+mf}{0.6}\PYG{p}{,} \PYG{l+m+mf}{0.5}\PYG{p}{]}\PYG{p}{,} \PYG{p}{[}\PYG{l+m+mi}{0}\PYG{p}{,} \PYG{l+m+mf}{0.2}\PYG{p}{]}\PYG{p}{,} \PYG{p}{[}\PYG{l+m+mf}{0.2}\PYG{p}{,} \PYG{l+m+mf}{0.3}\PYG{p}{]}\PYG{p}{,} \PYG{n}{each\PYGZus{}t}\PYG{o}{=}\PYG{k+kc}{False}\PYG{p}{)}
\end{sphinxVerbatim}

Find F values on grid defined by 1D vector of radial positions \sphinxtitleref{R}
and 1D vector of vertical positions \sphinxtitleref{Z} at time t=0.2s:

\begin{sphinxVerbatim}[commandchars=\\\{\}]
\PYG{n}{F\PYGZus{}mat} \PYG{o}{=} \PYG{n}{Eq\PYGZus{}instance}\PYG{o}{.}\PYG{n}{rz2F}\PYG{p}{(}\PYG{n}{R}\PYG{p}{,} \PYG{n}{Z}\PYG{p}{,} \PYG{l+m+mf}{0.2}\PYG{p}{,} \PYG{n}{make\PYGZus{}grid}\PYG{o}{=}\PYG{k+kc}{True}\PYG{p}{)}
\end{sphinxVerbatim}

\end{fulllineitems}

\index{rmid2F() (eqtools.core.Equilibrium method)@\spxentry{rmid2F()}\spxextra{eqtools.core.Equilibrium method}}

\begin{fulllineitems}
\phantomsection\label{\detokenize{eqtools:eqtools.core.Equilibrium.rmid2F}}\pysiglinewithargsret{\sphinxbfcode{\sphinxupquote{rmid2F}}}{\emph{R\_mid}, \emph{t}, \emph{**kwargs}}{}
Calculates the flux function \(F=RB_{\phi}\) corresponding to the passed R\_mid (mapped outboard midplane major radius) values.

By default, EFIT only computes this inside the LCFS.
\begin{quote}\begin{description}
\item[{Parameters}] \leavevmode\begin{itemize}
\item {} 
\sphinxstyleliteralstrong{\sphinxupquote{R\_mid}} (\sphinxstyleliteralemphasis{\sphinxupquote{Array-like}}\sphinxstyleliteralemphasis{\sphinxupquote{ or }}\sphinxstyleliteralemphasis{\sphinxupquote{scalar float}}) \textendash{} Values of the outboard midplane
major radius to map to F.

\item {} 
\sphinxstyleliteralstrong{\sphinxupquote{t}} (\sphinxstyleliteralemphasis{\sphinxupquote{Array-like}}\sphinxstyleliteralemphasis{\sphinxupquote{ or }}\sphinxstyleliteralemphasis{\sphinxupquote{scalar float}}) \textendash{} Times to perform the conversion at.
If \sphinxtitleref{t} is a single value, it is used for all of the elements of
\sphinxtitleref{R\_mid}. If the \sphinxtitleref{each\_t} keyword is True, then \sphinxtitleref{t} must be scalar
or have exactly one dimension. If the \sphinxtitleref{each\_t} keyword is False,
\sphinxtitleref{t} must have the same shape as \sphinxtitleref{R\_mid}.

\end{itemize}

\item[{Keyword Arguments}] \leavevmode\begin{itemize}
\item {} 
\sphinxstyleliteralstrong{\sphinxupquote{sqrt}} (\sphinxstyleliteralemphasis{\sphinxupquote{Boolean}}) \textendash{} Set to True to return the square root of F.
Only the square root of positive values is taken. Negative
values are replaced with zeros, consistent with Steve Wolfe’s
IDL implementation efit\_rz2rho.pro. Default is False.

\item {} 
\sphinxstyleliteralstrong{\sphinxupquote{each\_t}} (\sphinxstyleliteralemphasis{\sphinxupquote{Boolean}}) \textendash{} When True, the elements in \sphinxtitleref{R\_mid} are evaluated
at each value in \sphinxtitleref{t}. If True, \sphinxtitleref{t} must have only one dimension
(or be a scalar). If False, \sphinxtitleref{t} must match the shape of \sphinxtitleref{R\_mid}
or be a scalar. Default is True (evaluate ALL \sphinxtitleref{R\_mid} at EACH
element in \sphinxtitleref{t}).

\item {} 
\sphinxstyleliteralstrong{\sphinxupquote{length\_unit}} (\sphinxstyleliteralemphasis{\sphinxupquote{String}}\sphinxstyleliteralemphasis{\sphinxupquote{ or }}\sphinxstyleliteralemphasis{\sphinxupquote{1}}) \textendash{} 
Length unit that \sphinxtitleref{R\_mid} is given in.
If a string is given, it must be a valid unit specifier:
\begin{quote}


\begin{savenotes}\sphinxattablestart
\centering
\begin{tabulary}{\linewidth}[t]{|T|T|}
\hline

’m’
&
meters
\\
\hline
’cm’
&
centimeters
\\
\hline
’mm’
&
millimeters
\\
\hline
’in’
&
inches
\\
\hline
’ft’
&
feet
\\
\hline
’yd’
&
yards
\\
\hline
’smoot’
&
smoots
\\
\hline
’cubit’
&
cubits
\\
\hline
’hand’
&
hands
\\
\hline
’default’
&
meters
\\
\hline
\end{tabulary}
\par
\sphinxattableend\end{savenotes}
\end{quote}

If length\_unit is 1 or None, meters are assumed. The default
value is 1 (use meters).


\item {} 
\sphinxstyleliteralstrong{\sphinxupquote{k}} (\sphinxstyleliteralemphasis{\sphinxupquote{positive int}}) \textendash{} The degree of polynomial spline interpolation to
use in converting coordinates.

\item {} 
\sphinxstyleliteralstrong{\sphinxupquote{return\_t}} (\sphinxstyleliteralemphasis{\sphinxupquote{Boolean}}) \textendash{} Set to True to return a tuple of (\sphinxtitleref{F},
\sphinxtitleref{time\_idxs}), where \sphinxtitleref{time\_idxs} is the array of time indices
actually used in evaluating \sphinxtitleref{F} with nearest-neighbor
interpolation. (This is mostly present as an internal helper.)
Default is False (only return \sphinxtitleref{F}).

\end{itemize}

\item[{Returns}] \leavevmode

\sphinxtitleref{F} or (\sphinxtitleref{F}, \sphinxtitleref{time\_idxs})
\begin{itemize}
\item {} 
\sphinxstylestrong{F} (\sphinxtitleref{Array or scalar float}) - The flux function \(F=RB_{\phi}\).
If all of the input arguments are scalar, then a scalar is
returned. Otherwise, a scipy Array is returned.

\item {} 
\sphinxstylestrong{time\_idxs} (Array with same shape as \sphinxtitleref{F}) - The indices
(in \sphinxcode{\sphinxupquote{self.getTimeBase()}}) that were used for
nearest-neighbor interpolation. Only returned if \sphinxtitleref{return\_t} is
True.

\end{itemize}


\end{description}\end{quote}
\subsubsection*{Examples}

All assume that \sphinxtitleref{Eq\_instance} is a valid instance of the appropriate
extension of the {\hyperref[\detokenize{eqtools:eqtools.core.Equilibrium}]{\sphinxcrossref{\sphinxcode{\sphinxupquote{Equilibrium}}}}} abstract class.

Find single F value for Rmid=0.7m, t=0.26s:

\begin{sphinxVerbatim}[commandchars=\\\{\}]
\PYG{n}{F\PYGZus{}val} \PYG{o}{=} \PYG{n}{Eq\PYGZus{}instance}\PYG{o}{.}\PYG{n}{rmid2F}\PYG{p}{(}\PYG{l+m+mf}{0.7}\PYG{p}{,} \PYG{l+m+mf}{0.26}\PYG{p}{)}
\end{sphinxVerbatim}

Find F values at R\_mid values of 0.5m and 0.7m at the single time
t=0.26s:

\begin{sphinxVerbatim}[commandchars=\\\{\}]
\PYG{n}{F\PYGZus{}arr} \PYG{o}{=} \PYG{n}{Eq\PYGZus{}instance}\PYG{o}{.}\PYG{n}{rmid2F}\PYG{p}{(}\PYG{p}{[}\PYG{l+m+mf}{0.5}\PYG{p}{,} \PYG{l+m+mf}{0.7}\PYG{p}{]}\PYG{p}{,} \PYG{l+m+mf}{0.26}\PYG{p}{)}
\end{sphinxVerbatim}

Find F values at R\_mid=0.5m at times t={[}0.2s, 0.3s{]}:

\begin{sphinxVerbatim}[commandchars=\\\{\}]
\PYG{n}{F\PYGZus{}arr} \PYG{o}{=} \PYG{n}{Eq\PYGZus{}instance}\PYG{o}{.}\PYG{n}{rmid2F}\PYG{p}{(}\PYG{l+m+mf}{0.5}\PYG{p}{,} \PYG{p}{[}\PYG{l+m+mf}{0.2}\PYG{p}{,} \PYG{l+m+mf}{0.3}\PYG{p}{]}\PYG{p}{)}
\end{sphinxVerbatim}

Find F values at (R\_mid, t) points (0.6m, 0.2s) and (0.5m, 0.3s):

\begin{sphinxVerbatim}[commandchars=\\\{\}]
\PYG{n}{F\PYGZus{}arr} \PYG{o}{=} \PYG{n}{Eq\PYGZus{}instance}\PYG{o}{.}\PYG{n}{rmid2F}\PYG{p}{(}\PYG{p}{[}\PYG{l+m+mf}{0.6}\PYG{p}{,} \PYG{l+m+mf}{0.5}\PYG{p}{]}\PYG{p}{,} \PYG{p}{[}\PYG{l+m+mf}{0.2}\PYG{p}{,} \PYG{l+m+mf}{0.3}\PYG{p}{]}\PYG{p}{,} \PYG{n}{each\PYGZus{}t}\PYG{o}{=}\PYG{k+kc}{False}\PYG{p}{)}
\end{sphinxVerbatim}

\end{fulllineitems}

\index{roa2F() (eqtools.core.Equilibrium method)@\spxentry{roa2F()}\spxextra{eqtools.core.Equilibrium method}}

\begin{fulllineitems}
\phantomsection\label{\detokenize{eqtools:eqtools.core.Equilibrium.roa2F}}\pysiglinewithargsret{\sphinxbfcode{\sphinxupquote{roa2F}}}{\emph{roa}, \emph{t}, \emph{**kwargs}}{}
Convert the passed (r/a, t) coordinates into the flux function \(F=RB_{\phi}\).

By default, EFIT only computes this inside the LCFS.
\begin{quote}\begin{description}
\item[{Parameters}] \leavevmode\begin{itemize}
\item {} 
\sphinxstyleliteralstrong{\sphinxupquote{roa}} (\sphinxstyleliteralemphasis{\sphinxupquote{Array-like}}\sphinxstyleliteralemphasis{\sphinxupquote{ or }}\sphinxstyleliteralemphasis{\sphinxupquote{scalar float}}) \textendash{} Values of the normalized minor
radius to map to F.

\item {} 
\sphinxstyleliteralstrong{\sphinxupquote{t}} (\sphinxstyleliteralemphasis{\sphinxupquote{Array-like}}\sphinxstyleliteralemphasis{\sphinxupquote{ or }}\sphinxstyleliteralemphasis{\sphinxupquote{scalar float}}) \textendash{} Times to perform the conversion at.
If \sphinxtitleref{t} is a single value, it is used for all of the elements of
\sphinxtitleref{roa}. If the \sphinxtitleref{each\_t} keyword is True, then \sphinxtitleref{t} must be scalar
or have exactly one dimension. If the \sphinxtitleref{each\_t} keyword is False,
\sphinxtitleref{t} must have the same shape as \sphinxtitleref{roa}.

\end{itemize}

\item[{Keyword Arguments}] \leavevmode\begin{itemize}
\item {} 
\sphinxstyleliteralstrong{\sphinxupquote{sqrt}} (\sphinxstyleliteralemphasis{\sphinxupquote{Boolean}}) \textendash{} Set to True to return the square root of F.
Only the square root of positive values is taken. Negative
values are replaced with zeros, consistent with Steve Wolfe’s
IDL implementation efit\_rz2rho.pro. Default is False.

\item {} 
\sphinxstyleliteralstrong{\sphinxupquote{each\_t}} (\sphinxstyleliteralemphasis{\sphinxupquote{Boolean}}) \textendash{} When True, the elements in \sphinxtitleref{roa} are evaluated
at each value in \sphinxtitleref{t}. If True, \sphinxtitleref{t} must have only one dimension
(or be a scalar). If False, \sphinxtitleref{t} must match the shape of \sphinxtitleref{roa}
or be a scalar. Default is True (evaluate ALL \sphinxtitleref{roa} at EACH
element in \sphinxtitleref{t}).

\item {} 
\sphinxstyleliteralstrong{\sphinxupquote{k}} (\sphinxstyleliteralemphasis{\sphinxupquote{positive int}}) \textendash{} The degree of polynomial spline interpolation to
use in converting coordinates.

\item {} 
\sphinxstyleliteralstrong{\sphinxupquote{return\_t}} (\sphinxstyleliteralemphasis{\sphinxupquote{Boolean}}) \textendash{} Set to True to return a tuple of (\sphinxtitleref{F},
\sphinxtitleref{time\_idxs}), where \sphinxtitleref{time\_idxs} is the array of time indices
actually used in evaluating \sphinxtitleref{F} with nearest-neighbor
interpolation. (This is mostly present as an internal helper.)
Default is False (only return \sphinxtitleref{F}).

\end{itemize}

\item[{Returns}] \leavevmode

\sphinxtitleref{F} or (\sphinxtitleref{F}, \sphinxtitleref{time\_idxs})
\begin{itemize}
\item {} 
\sphinxstylestrong{F} (\sphinxtitleref{Array or scalar float}) - The flux function \(F=RB_{\phi}\).
If all of the input arguments are scalar, then a scalar is returned.
Otherwise, a scipy Array is returned.

\item {} 
\sphinxstylestrong{time\_idxs} (Array with same shape as \sphinxtitleref{F}) - The indices
(in \sphinxcode{\sphinxupquote{self.getTimeBase()}}) that were used for
nearest-neighbor interpolation. Only returned if \sphinxtitleref{return\_t} is
True.

\end{itemize}


\end{description}\end{quote}
\subsubsection*{Examples}

All assume that \sphinxtitleref{Eq\_instance} is a valid instance of the appropriate
extension of the {\hyperref[\detokenize{eqtools:eqtools.core.Equilibrium}]{\sphinxcrossref{\sphinxcode{\sphinxupquote{Equilibrium}}}}} abstract class.

Find single F value at r/a=0.6, t=0.26s:

\begin{sphinxVerbatim}[commandchars=\\\{\}]
\PYG{n}{F\PYGZus{}val} \PYG{o}{=} \PYG{n}{Eq\PYGZus{}instance}\PYG{o}{.}\PYG{n}{roa2F}\PYG{p}{(}\PYG{l+m+mf}{0.6}\PYG{p}{,} \PYG{l+m+mf}{0.26}\PYG{p}{)}
\end{sphinxVerbatim}

Find F values at r/a points 0.6 and 0.8 at the
single time t=0.26s.:

\begin{sphinxVerbatim}[commandchars=\\\{\}]
\PYG{n}{F\PYGZus{}arr} \PYG{o}{=} \PYG{n}{Eq\PYGZus{}instance}\PYG{o}{.}\PYG{n}{roa2F}\PYG{p}{(}\PYG{p}{[}\PYG{l+m+mf}{0.6}\PYG{p}{,} \PYG{l+m+mf}{0.8}\PYG{p}{]}\PYG{p}{,} \PYG{l+m+mf}{0.26}\PYG{p}{)}
\end{sphinxVerbatim}

Find F values at r/a of 0.6 at times t={[}0.2s, 0.3s{]}:

\begin{sphinxVerbatim}[commandchars=\\\{\}]
\PYG{n}{F\PYGZus{}arr} \PYG{o}{=} \PYG{n}{Eq\PYGZus{}instance}\PYG{o}{.}\PYG{n}{roa2F}\PYG{p}{(}\PYG{l+m+mf}{0.6}\PYG{p}{,} \PYG{p}{[}\PYG{l+m+mf}{0.2}\PYG{p}{,} \PYG{l+m+mf}{0.3}\PYG{p}{]}\PYG{p}{)}
\end{sphinxVerbatim}

Find F values at (roa, t) points (0.6, 0.2s) and (0.5, 0.3s):

\begin{sphinxVerbatim}[commandchars=\\\{\}]
\PYG{n}{F\PYGZus{}arr} \PYG{o}{=} \PYG{n}{Eq\PYGZus{}instance}\PYG{o}{.}\PYG{n}{roa2F}\PYG{p}{(}\PYG{p}{[}\PYG{l+m+mf}{0.6}\PYG{p}{,} \PYG{l+m+mf}{0.5}\PYG{p}{]}\PYG{p}{,} \PYG{p}{[}\PYG{l+m+mf}{0.2}\PYG{p}{,} \PYG{l+m+mf}{0.3}\PYG{p}{]}\PYG{p}{,} \PYG{n}{each\PYGZus{}t}\PYG{o}{=}\PYG{k+kc}{False}\PYG{p}{)}
\end{sphinxVerbatim}

\end{fulllineitems}

\index{psinorm2F() (eqtools.core.Equilibrium method)@\spxentry{psinorm2F()}\spxextra{eqtools.core.Equilibrium method}}

\begin{fulllineitems}
\phantomsection\label{\detokenize{eqtools:eqtools.core.Equilibrium.psinorm2F}}\pysiglinewithargsret{\sphinxbfcode{\sphinxupquote{psinorm2F}}}{\emph{psinorm}, \emph{t}, \emph{**kwargs}}{}
Calculates the flux function \(F=RB_{\phi}\) corresponding to the passed psi\_norm (normalized poloidal flux) values.

By default, EFIT only computes this inside the LCFS.
\begin{quote}\begin{description}
\item[{Parameters}] \leavevmode\begin{itemize}
\item {} 
\sphinxstyleliteralstrong{\sphinxupquote{psi\_norm}} (\sphinxstyleliteralemphasis{\sphinxupquote{Array-like}}\sphinxstyleliteralemphasis{\sphinxupquote{ or }}\sphinxstyleliteralemphasis{\sphinxupquote{scalar float}}) \textendash{} Values of the normalized
poloidal flux to map to F.

\item {} 
\sphinxstyleliteralstrong{\sphinxupquote{t}} (\sphinxstyleliteralemphasis{\sphinxupquote{Array-like}}\sphinxstyleliteralemphasis{\sphinxupquote{ or }}\sphinxstyleliteralemphasis{\sphinxupquote{scalar float}}) \textendash{} Times to perform the conversion at.
If \sphinxtitleref{t} is a single value, it is used for all of the elements of
\sphinxtitleref{psi\_norm}. If the \sphinxtitleref{each\_t} keyword is True, then \sphinxtitleref{t} must be scalar
or have exactly one dimension. If the \sphinxtitleref{each\_t} keyword is False,
\sphinxtitleref{t} must have the same shape as \sphinxtitleref{psi\_norm}.

\end{itemize}

\item[{Keyword Arguments}] \leavevmode\begin{itemize}
\item {} 
\sphinxstyleliteralstrong{\sphinxupquote{sqrt}} (\sphinxstyleliteralemphasis{\sphinxupquote{Boolean}}) \textendash{} Set to True to return the square root of F. Only
the square root of positive values is taken. Negative values are
replaced with zeros, consistent with Steve Wolfe’s IDL
implementation efit\_rz2rho.pro. Default is False.

\item {} 
\sphinxstyleliteralstrong{\sphinxupquote{each\_t}} (\sphinxstyleliteralemphasis{\sphinxupquote{Boolean}}) \textendash{} When True, the elements in \sphinxtitleref{psi\_norm} are evaluated at
each value in \sphinxtitleref{t}. If True, \sphinxtitleref{t} must have only one dimension (or
be a scalar). If False, \sphinxtitleref{t} must match the shape of \sphinxtitleref{psi\_norm} or be
a scalar. Default is True (evaluate ALL \sphinxtitleref{psi\_norm} at EACH element in
\sphinxtitleref{t}).

\item {} 
\sphinxstyleliteralstrong{\sphinxupquote{k}} (\sphinxstyleliteralemphasis{\sphinxupquote{positive int}}) \textendash{} The degree of polynomial spline interpolation to
use in converting coordinates.

\item {} 
\sphinxstyleliteralstrong{\sphinxupquote{return\_t}} (\sphinxstyleliteralemphasis{\sphinxupquote{Boolean}}) \textendash{} Set to True to return a tuple of (\sphinxtitleref{F},
\sphinxtitleref{time\_idxs}), where \sphinxtitleref{time\_idxs} is the array of time indices
actually used in evaluating \sphinxtitleref{F} with nearest-neighbor
interpolation. (This is mostly present as an internal helper.)
Default is False (only return \sphinxtitleref{F}).

\end{itemize}

\item[{Returns}] \leavevmode

\sphinxtitleref{F} or (\sphinxtitleref{F}, \sphinxtitleref{time\_idxs})
\begin{itemize}
\item {} 
\sphinxstylestrong{F} (\sphinxtitleref{Array or scalar float}) - The flux function \(F=RB_{\phi}\).
If all of the input arguments are scalar, then a scalar is returned.
Otherwise, a scipy Array is returned.

\item {} 
\sphinxstylestrong{time\_idxs} (Array with same shape as \sphinxtitleref{F}) - The indices
(in \sphinxcode{\sphinxupquote{self.getTimeBase()}}) that were used for
nearest-neighbor interpolation. Only returned if \sphinxtitleref{return\_t} is
True.

\end{itemize}


\end{description}\end{quote}
\subsubsection*{Examples}

All assume that \sphinxtitleref{Eq\_instance} is a valid instance of the appropriate
extension of the {\hyperref[\detokenize{eqtools:eqtools.core.Equilibrium}]{\sphinxcrossref{\sphinxcode{\sphinxupquote{Equilibrium}}}}} abstract class.

Find single F value for psinorm=0.7, t=0.26s:

\begin{sphinxVerbatim}[commandchars=\\\{\}]
\PYG{n}{F\PYGZus{}val} \PYG{o}{=} \PYG{n}{Eq\PYGZus{}instance}\PYG{o}{.}\PYG{n}{psinorm2F}\PYG{p}{(}\PYG{l+m+mf}{0.7}\PYG{p}{,} \PYG{l+m+mf}{0.26}\PYG{p}{)}
\end{sphinxVerbatim}

Find F values at psi\_norm values of 0.5 and 0.7 at the single time
t=0.26s:

\begin{sphinxVerbatim}[commandchars=\\\{\}]
\PYG{n}{F\PYGZus{}arr} \PYG{o}{=} \PYG{n}{Eq\PYGZus{}instance}\PYG{o}{.}\PYG{n}{psinorm2F}\PYG{p}{(}\PYG{p}{[}\PYG{l+m+mf}{0.5}\PYG{p}{,} \PYG{l+m+mf}{0.7}\PYG{p}{]}\PYG{p}{,} \PYG{l+m+mf}{0.26}\PYG{p}{)}
\end{sphinxVerbatim}

Find F values at psi\_norm=0.5 at times t={[}0.2s, 0.3s{]}:

\begin{sphinxVerbatim}[commandchars=\\\{\}]
\PYG{n}{F\PYGZus{}arr} \PYG{o}{=} \PYG{n}{Eq\PYGZus{}instance}\PYG{o}{.}\PYG{n}{psinorm2F}\PYG{p}{(}\PYG{l+m+mf}{0.5}\PYG{p}{,} \PYG{p}{[}\PYG{l+m+mf}{0.2}\PYG{p}{,} \PYG{l+m+mf}{0.3}\PYG{p}{]}\PYG{p}{)}
\end{sphinxVerbatim}

Find F values at (psinorm, t) points (0.6, 0.2s) and (0.5, 0.3s):

\begin{sphinxVerbatim}[commandchars=\\\{\}]
\PYG{n}{F\PYGZus{}arr} \PYG{o}{=} \PYG{n}{Eq\PYGZus{}instance}\PYG{o}{.}\PYG{n}{psinorm2F}\PYG{p}{(}\PYG{p}{[}\PYG{l+m+mf}{0.6}\PYG{p}{,} \PYG{l+m+mf}{0.5}\PYG{p}{]}\PYG{p}{,} \PYG{p}{[}\PYG{l+m+mf}{0.2}\PYG{p}{,} \PYG{l+m+mf}{0.3}\PYG{p}{]}\PYG{p}{,} \PYG{n}{each\PYGZus{}t}\PYG{o}{=}\PYG{k+kc}{False}\PYG{p}{)}
\end{sphinxVerbatim}

\end{fulllineitems}

\index{phinorm2F() (eqtools.core.Equilibrium method)@\spxentry{phinorm2F()}\spxextra{eqtools.core.Equilibrium method}}

\begin{fulllineitems}
\phantomsection\label{\detokenize{eqtools:eqtools.core.Equilibrium.phinorm2F}}\pysiglinewithargsret{\sphinxbfcode{\sphinxupquote{phinorm2F}}}{\emph{phinorm}, \emph{t}, \emph{**kwargs}}{}
Calculates the flux function \(F=RB_{\phi}\) corresponding to the passed phinorm (normalized toroidal flux) values.

By default, EFIT only computes this inside the LCFS.
\begin{quote}\begin{description}
\item[{Parameters}] \leavevmode\begin{itemize}
\item {} 
\sphinxstyleliteralstrong{\sphinxupquote{phinorm}} (\sphinxstyleliteralemphasis{\sphinxupquote{Array-like}}\sphinxstyleliteralemphasis{\sphinxupquote{ or }}\sphinxstyleliteralemphasis{\sphinxupquote{scalar float}}) \textendash{} Values of the normalized
toroidal flux to map to F.

\item {} 
\sphinxstyleliteralstrong{\sphinxupquote{t}} (\sphinxstyleliteralemphasis{\sphinxupquote{Array-like}}\sphinxstyleliteralemphasis{\sphinxupquote{ or }}\sphinxstyleliteralemphasis{\sphinxupquote{scalar float}}) \textendash{} Times to perform the conversion at.
If \sphinxtitleref{t} is a single value, it is used for all of the elements of
\sphinxtitleref{phinorm}. If the \sphinxtitleref{each\_t} keyword is True, then \sphinxtitleref{t} must be scalar
or have exactly one dimension. If the \sphinxtitleref{each\_t} keyword is False,
\sphinxtitleref{t} must have the same shape as \sphinxtitleref{phinorm}.

\end{itemize}

\item[{Keyword Arguments}] \leavevmode\begin{itemize}
\item {} 
\sphinxstyleliteralstrong{\sphinxupquote{sqrt}} (\sphinxstyleliteralemphasis{\sphinxupquote{Boolean}}) \textendash{} Set to True to return the square root of F.
Only the square root of positive values is taken. Negative
values are replaced with zeros, consistent with Steve Wolfe’s
IDL implementation efit\_rz2rho.pro. Default is False.

\item {} 
\sphinxstyleliteralstrong{\sphinxupquote{each\_t}} (\sphinxstyleliteralemphasis{\sphinxupquote{Boolean}}) \textendash{} When True, the elements in \sphinxtitleref{phinorm} are evaluated
at each value in \sphinxtitleref{t}. If True, \sphinxtitleref{t} must have only one dimension
(or be a scalar). If False, \sphinxtitleref{t} must match the shape of \sphinxtitleref{phinorm}
or be a scalar. Default is True (evaluate ALL \sphinxtitleref{phinorm} at EACH
element in \sphinxtitleref{t}).

\item {} 
\sphinxstyleliteralstrong{\sphinxupquote{k}} (\sphinxstyleliteralemphasis{\sphinxupquote{positive int}}) \textendash{} The degree of polynomial spline interpolation to
use in converting coordinates.

\item {} 
\sphinxstyleliteralstrong{\sphinxupquote{return\_t}} (\sphinxstyleliteralemphasis{\sphinxupquote{Boolean}}) \textendash{} Set to True to return a tuple of (\sphinxtitleref{F},
\sphinxtitleref{time\_idxs}), where \sphinxtitleref{time\_idxs} is the array of time indices
actually used in evaluating \sphinxtitleref{F} with nearest-neighbor
interpolation. (This is mostly present as an internal helper.)
Default is False (only return \sphinxtitleref{F}).

\end{itemize}

\item[{Returns}] \leavevmode

\sphinxtitleref{F} or (\sphinxtitleref{F}, \sphinxtitleref{time\_idxs})
\begin{itemize}
\item {} 
\sphinxstylestrong{F} (\sphinxtitleref{Array or scalar float}) - The flux function \(F=RB_{\phi}\).
If all of the input arguments are scalar, then a scalar is returned.
Otherwise, a scipy Array is returned.

\item {} 
\sphinxstylestrong{time\_idxs} (Array with same shape as \sphinxtitleref{F}) - The indices
(in \sphinxcode{\sphinxupquote{self.getTimeBase()}}) that were used for
nearest-neighbor interpolation. Only returned if \sphinxtitleref{return\_t} is
True.

\end{itemize}


\end{description}\end{quote}
\subsubsection*{Examples}

All assume that \sphinxtitleref{Eq\_instance} is a valid instance of the appropriate
extension of the {\hyperref[\detokenize{eqtools:eqtools.core.Equilibrium}]{\sphinxcrossref{\sphinxcode{\sphinxupquote{Equilibrium}}}}} abstract class.

Find single F value for phinorm=0.7, t=0.26s:

\begin{sphinxVerbatim}[commandchars=\\\{\}]
\PYG{n}{F\PYGZus{}val} \PYG{o}{=} \PYG{n}{Eq\PYGZus{}instance}\PYG{o}{.}\PYG{n}{phinorm2F}\PYG{p}{(}\PYG{l+m+mf}{0.7}\PYG{p}{,} \PYG{l+m+mf}{0.26}\PYG{p}{)}
\end{sphinxVerbatim}

Find F values at phinorm values of 0.5 and 0.7 at the single time
t=0.26s:

\begin{sphinxVerbatim}[commandchars=\\\{\}]
\PYG{n}{F\PYGZus{}arr} \PYG{o}{=} \PYG{n}{Eq\PYGZus{}instance}\PYG{o}{.}\PYG{n}{phinorm2F}\PYG{p}{(}\PYG{p}{[}\PYG{l+m+mf}{0.5}\PYG{p}{,} \PYG{l+m+mf}{0.7}\PYG{p}{]}\PYG{p}{,} \PYG{l+m+mf}{0.26}\PYG{p}{)}
\end{sphinxVerbatim}

Find F values at phinorm=0.5 at times t={[}0.2s, 0.3s{]}:

\begin{sphinxVerbatim}[commandchars=\\\{\}]
\PYG{n}{F\PYGZus{}arr} \PYG{o}{=} \PYG{n}{Eq\PYGZus{}instance}\PYG{o}{.}\PYG{n}{phinorm2F}\PYG{p}{(}\PYG{l+m+mf}{0.5}\PYG{p}{,} \PYG{p}{[}\PYG{l+m+mf}{0.2}\PYG{p}{,} \PYG{l+m+mf}{0.3}\PYG{p}{]}\PYG{p}{)}
\end{sphinxVerbatim}

Find F values at (phinorm, t) points (0.6, 0.2s) and (0.5, 0.3s):

\begin{sphinxVerbatim}[commandchars=\\\{\}]
\PYG{n}{F\PYGZus{}arr} \PYG{o}{=} \PYG{n}{Eq\PYGZus{}instance}\PYG{o}{.}\PYG{n}{phinorm2F}\PYG{p}{(}\PYG{p}{[}\PYG{l+m+mf}{0.6}\PYG{p}{,} \PYG{l+m+mf}{0.5}\PYG{p}{]}\PYG{p}{,} \PYG{p}{[}\PYG{l+m+mf}{0.2}\PYG{p}{,} \PYG{l+m+mf}{0.3}\PYG{p}{]}\PYG{p}{,} \PYG{n}{each\PYGZus{}t}\PYG{o}{=}\PYG{k+kc}{False}\PYG{p}{)}
\end{sphinxVerbatim}

\end{fulllineitems}

\index{volnorm2F() (eqtools.core.Equilibrium method)@\spxentry{volnorm2F()}\spxextra{eqtools.core.Equilibrium method}}

\begin{fulllineitems}
\phantomsection\label{\detokenize{eqtools:eqtools.core.Equilibrium.volnorm2F}}\pysiglinewithargsret{\sphinxbfcode{\sphinxupquote{volnorm2F}}}{\emph{volnorm}, \emph{t}, \emph{**kwargs}}{}
Calculates the flux function \(F=RB_{\phi}\) corresponding to the passed volnorm (normalized flux surface volume) values.

By default, EFIT only computes this inside the LCFS.
\begin{quote}\begin{description}
\item[{Parameters}] \leavevmode\begin{itemize}
\item {} 
\sphinxstyleliteralstrong{\sphinxupquote{volnorm}} (\sphinxstyleliteralemphasis{\sphinxupquote{Array-like}}\sphinxstyleliteralemphasis{\sphinxupquote{ or }}\sphinxstyleliteralemphasis{\sphinxupquote{scalar float}}) \textendash{} Values of the normalized
flux surface volume to map to F.

\item {} 
\sphinxstyleliteralstrong{\sphinxupquote{t}} (\sphinxstyleliteralemphasis{\sphinxupquote{Array-like}}\sphinxstyleliteralemphasis{\sphinxupquote{ or }}\sphinxstyleliteralemphasis{\sphinxupquote{scalar float}}) \textendash{} Times to perform the conversion at.
If \sphinxtitleref{t} is a single value, it is used for all of the elements of
\sphinxtitleref{volnorm}. If the \sphinxtitleref{each\_t} keyword is True, then \sphinxtitleref{t} must be scalar
or have exactly one dimension. If the \sphinxtitleref{each\_t} keyword is False,
\sphinxtitleref{t} must have the same shape as \sphinxtitleref{volnorm}.

\end{itemize}

\item[{Keyword Arguments}] \leavevmode\begin{itemize}
\item {} 
\sphinxstyleliteralstrong{\sphinxupquote{sqrt}} (\sphinxstyleliteralemphasis{\sphinxupquote{Boolean}}) \textendash{} Set to True to return the square root of F.
Only the square root of positive values is taken. Negative
values are replaced with zeros, consistent with Steve Wolfe’s
IDL implementation efit\_rz2rho.pro. Default is False.

\item {} 
\sphinxstyleliteralstrong{\sphinxupquote{each\_t}} (\sphinxstyleliteralemphasis{\sphinxupquote{Boolean}}) \textendash{} When True, the elements in \sphinxtitleref{volnorm} are evaluated
at each value in \sphinxtitleref{t}. If True, \sphinxtitleref{t} must have only one dimension
(or be a scalar). If False, \sphinxtitleref{t} must match the shape of \sphinxtitleref{volnorm}
or be a scalar. Default is True (evaluate ALL \sphinxtitleref{volnorm} at EACH
element in \sphinxtitleref{t}).

\item {} 
\sphinxstyleliteralstrong{\sphinxupquote{k}} (\sphinxstyleliteralemphasis{\sphinxupquote{positive int}}) \textendash{} The degree of polynomial spline interpolation to
use in converting coordinates.

\item {} 
\sphinxstyleliteralstrong{\sphinxupquote{return\_t}} (\sphinxstyleliteralemphasis{\sphinxupquote{Boolean}}) \textendash{} Set to True to return a tuple of (\sphinxtitleref{F},
\sphinxtitleref{time\_idxs}), where \sphinxtitleref{time\_idxs} is the array of time indices
actually used in evaluating \sphinxtitleref{F} with nearest-neighbor
interpolation. (This is mostly present as an internal helper.)
Default is False (only return \sphinxtitleref{F}).

\end{itemize}

\item[{Returns}] \leavevmode

\sphinxtitleref{F} or (\sphinxtitleref{F}, \sphinxtitleref{time\_idxs})
\begin{itemize}
\item {} 
\sphinxstylestrong{F} (\sphinxtitleref{Array or scalar float}) - The flux function \(F=RB_{\phi}\).
If all of the input arguments are scalar, then a scalar is returned.
Otherwise, a scipy Array is returned.

\item {} 
\sphinxstylestrong{time\_idxs} (Array with same shape as \sphinxtitleref{F}) - The indices
(in \sphinxcode{\sphinxupquote{self.getTimeBase()}}) that were used for
nearest-neighbor interpolation. Only returned if \sphinxtitleref{return\_t} is
True.

\end{itemize}


\end{description}\end{quote}
\subsubsection*{Examples}

All assume that \sphinxtitleref{Eq\_instance} is a valid instance of the appropriate
extension of the {\hyperref[\detokenize{eqtools:eqtools.core.Equilibrium}]{\sphinxcrossref{\sphinxcode{\sphinxupquote{Equilibrium}}}}} abstract class.

Find single F value for volnorm=0.7, t=0.26s:

\begin{sphinxVerbatim}[commandchars=\\\{\}]
\PYG{n}{F\PYGZus{}val} \PYG{o}{=} \PYG{n}{Eq\PYGZus{}instance}\PYG{o}{.}\PYG{n}{volnorm2F}\PYG{p}{(}\PYG{l+m+mf}{0.7}\PYG{p}{,} \PYG{l+m+mf}{0.26}\PYG{p}{)}
\end{sphinxVerbatim}

Find F values at volnorm values of 0.5 and 0.7 at the single time
t=0.26s:

\begin{sphinxVerbatim}[commandchars=\\\{\}]
\PYG{n}{F\PYGZus{}arr} \PYG{o}{=} \PYG{n}{Eq\PYGZus{}instance}\PYG{o}{.}\PYG{n}{volnorm2F}\PYG{p}{(}\PYG{p}{[}\PYG{l+m+mf}{0.5}\PYG{p}{,} \PYG{l+m+mf}{0.7}\PYG{p}{]}\PYG{p}{,} \PYG{l+m+mf}{0.26}\PYG{p}{)}
\end{sphinxVerbatim}

Find F values at volnorm=0.5 at times t={[}0.2s, 0.3s{]}:

\begin{sphinxVerbatim}[commandchars=\\\{\}]
\PYG{n}{F\PYGZus{}arr} \PYG{o}{=} \PYG{n}{Eq\PYGZus{}instance}\PYG{o}{.}\PYG{n}{volnorm2F}\PYG{p}{(}\PYG{l+m+mf}{0.5}\PYG{p}{,} \PYG{p}{[}\PYG{l+m+mf}{0.2}\PYG{p}{,} \PYG{l+m+mf}{0.3}\PYG{p}{]}\PYG{p}{)}
\end{sphinxVerbatim}

Find F values at (volnorm, t) points (0.6, 0.2s) and (0.5, 0.3s):

\begin{sphinxVerbatim}[commandchars=\\\{\}]
\PYG{n}{F\PYGZus{}arr} \PYG{o}{=} \PYG{n}{Eq\PYGZus{}instance}\PYG{o}{.}\PYG{n}{volnorm2F}\PYG{p}{(}\PYG{p}{[}\PYG{l+m+mf}{0.6}\PYG{p}{,} \PYG{l+m+mf}{0.5}\PYG{p}{]}\PYG{p}{,} \PYG{p}{[}\PYG{l+m+mf}{0.2}\PYG{p}{,} \PYG{l+m+mf}{0.3}\PYG{p}{]}\PYG{p}{,} \PYG{n}{each\PYGZus{}t}\PYG{o}{=}\PYG{k+kc}{False}\PYG{p}{)}
\end{sphinxVerbatim}

\end{fulllineitems}

\index{Fnorm2psinorm() (eqtools.core.Equilibrium method)@\spxentry{Fnorm2psinorm()}\spxextra{eqtools.core.Equilibrium method}}

\begin{fulllineitems}
\phantomsection\label{\detokenize{eqtools:eqtools.core.Equilibrium.Fnorm2psinorm}}\pysiglinewithargsret{\sphinxbfcode{\sphinxupquote{Fnorm2psinorm}}}{\emph{F}, \emph{t}, \emph{**kwargs}}{}
Calculates the psinorm (normalized poloidal flux) corresponding to the passed normalized flux function \(F=RB_{\phi}\) values.

This is provided as a convenience method to plot current lines with the
correct spacing: current lines launched from a grid uniformly-spaced in
Fnorm will have spacing directly proportional to the magnitude.

By default, EFIT only computes this inside the LCFS. Furthermore, it is
truncated at the radius at which is becomes non-monotonic.
\begin{quote}\begin{description}
\item[{Parameters}] \leavevmode\begin{itemize}
\item {} 
\sphinxstyleliteralstrong{\sphinxupquote{F}} (\sphinxstyleliteralemphasis{\sphinxupquote{Array-like}}\sphinxstyleliteralemphasis{\sphinxupquote{ or }}\sphinxstyleliteralemphasis{\sphinxupquote{scalar float}}) \textendash{} Values of F to map to psinorm.

\item {} 
\sphinxstyleliteralstrong{\sphinxupquote{t}} (\sphinxstyleliteralemphasis{\sphinxupquote{Array-like}}\sphinxstyleliteralemphasis{\sphinxupquote{ or }}\sphinxstyleliteralemphasis{\sphinxupquote{scalar float}}) \textendash{} Times to perform the conversion at.
If \sphinxtitleref{t} is a single value, it is used for all of the elements of
\sphinxtitleref{volnorm}. If the \sphinxtitleref{each\_t} keyword is True, then \sphinxtitleref{t} must be scalar
or have exactly one dimension. If the \sphinxtitleref{each\_t} keyword is False,
\sphinxtitleref{t} must have the same shape as \sphinxtitleref{volnorm}.

\end{itemize}

\item[{Keyword Arguments}] \leavevmode\begin{itemize}
\item {} 
\sphinxstyleliteralstrong{\sphinxupquote{sqrt}} (\sphinxstyleliteralemphasis{\sphinxupquote{Boolean}}) \textendash{} Set to True to return the square root of psinorm.
Only the square root of positive values is taken. Negative
values are replaced with zeros, consistent with Steve Wolfe’s
IDL implementation efit\_rz2rho.pro. Default is False.

\item {} 
\sphinxstyleliteralstrong{\sphinxupquote{each\_t}} (\sphinxstyleliteralemphasis{\sphinxupquote{Boolean}}) \textendash{} When True, the elements in \sphinxtitleref{F} are evaluated at
each value in \sphinxtitleref{t}. If True, \sphinxtitleref{t} must have only one dimension (or
be a scalar). If False, \sphinxtitleref{t} must match the shape of \sphinxtitleref{F} or be a
scalar. Default is True (evaluate ALL \sphinxtitleref{volnorm} at EACH element
in \sphinxtitleref{t}).

\item {} 
\sphinxstyleliteralstrong{\sphinxupquote{k}} (\sphinxstyleliteralemphasis{\sphinxupquote{positive int}}) \textendash{} The degree of polynomial spline interpolation to
use in converting coordinates.

\item {} 
\sphinxstyleliteralstrong{\sphinxupquote{return\_t}} (\sphinxstyleliteralemphasis{\sphinxupquote{Boolean}}) \textendash{} Set to True to return a tuple of (\sphinxtitleref{psinorm},
\sphinxtitleref{time\_idxs}), where \sphinxtitleref{time\_idxs} is the array of time indices
actually used in evaluating \sphinxtitleref{psinorm} with nearest-neighbor
interpolation. (This is mostly present as an internal helper.)
Default is False (only return \sphinxtitleref{psinorm}).

\end{itemize}

\item[{Returns}] \leavevmode

\sphinxtitleref{psinorm} or (\sphinxtitleref{psinorm}, \sphinxtitleref{time\_idxs})
\begin{itemize}
\item {} 
\sphinxstylestrong{psinorm} (\sphinxtitleref{Array or scalar float}) - The normalized poloidal
flux. If all of the input arguments are scalar, then a scalar is
returned. Otherwise, a scipy Array is returned.

\item {} 
\sphinxstylestrong{time\_idxs} (Array with same shape as \sphinxtitleref{psinorm}) - The indices
(in \sphinxcode{\sphinxupquote{self.getTimeBase()}}) that were used for
nearest-neighbor interpolation. Only returned if \sphinxtitleref{return\_t} is
True.

\end{itemize}


\end{description}\end{quote}
\subsubsection*{Examples}

All assume that \sphinxtitleref{Eq\_instance} is a valid instance of the appropriate
extension of the {\hyperref[\detokenize{eqtools:eqtools.core.Equilibrium}]{\sphinxcrossref{\sphinxcode{\sphinxupquote{Equilibrium}}}}} abstract class.

Find single psinorm value for F=0.7, t=0.26s:

\begin{sphinxVerbatim}[commandchars=\\\{\}]
\PYG{n}{psinorm\PYGZus{}val} \PYG{o}{=} \PYG{n}{Eq\PYGZus{}instance}\PYG{o}{.}\PYG{n}{F2psinorm}\PYG{p}{(}\PYG{l+m+mf}{0.7}\PYG{p}{,} \PYG{l+m+mf}{0.26}\PYG{p}{)}
\end{sphinxVerbatim}

Find psinorm values at F values of 0.5 and 0.7 at the single time
t=0.26s:

\begin{sphinxVerbatim}[commandchars=\\\{\}]
\PYG{n}{psinorm\PYGZus{}arr} \PYG{o}{=} \PYG{n}{Eq\PYGZus{}instance}\PYG{o}{.}\PYG{n}{F2psinorm}\PYG{p}{(}\PYG{p}{[}\PYG{l+m+mf}{0.5}\PYG{p}{,} \PYG{l+m+mf}{0.7}\PYG{p}{]}\PYG{p}{,} \PYG{l+m+mf}{0.26}\PYG{p}{)}
\end{sphinxVerbatim}

Find psinorm values at F=0.5 at times t={[}0.2s, 0.3s{]}:

\begin{sphinxVerbatim}[commandchars=\\\{\}]
\PYG{n}{psinorm\PYGZus{}arr} \PYG{o}{=} \PYG{n}{Eq\PYGZus{}instance}\PYG{o}{.}\PYG{n}{F2psinorm}\PYG{p}{(}\PYG{l+m+mf}{0.5}\PYG{p}{,} \PYG{p}{[}\PYG{l+m+mf}{0.2}\PYG{p}{,} \PYG{l+m+mf}{0.3}\PYG{p}{]}\PYG{p}{)}
\end{sphinxVerbatim}

Find psinorm values at (F, t) points (0.6, 0.2s) and (0.5, 0.3s):

\begin{sphinxVerbatim}[commandchars=\\\{\}]
\PYG{n}{psinorm\PYGZus{}arr} \PYG{o}{=} \PYG{n}{Eq\PYGZus{}instance}\PYG{o}{.}\PYG{n}{F2psinorm}\PYG{p}{(}\PYG{p}{[}\PYG{l+m+mf}{0.6}\PYG{p}{,} \PYG{l+m+mf}{0.5}\PYG{p}{]}\PYG{p}{,} \PYG{p}{[}\PYG{l+m+mf}{0.2}\PYG{p}{,} \PYG{l+m+mf}{0.3}\PYG{p}{]}\PYG{p}{,} \PYG{n}{each\PYGZus{}t}\PYG{o}{=}\PYG{k+kc}{False}\PYG{p}{)}
\end{sphinxVerbatim}

\end{fulllineitems}

\index{rz2FFPrime() (eqtools.core.Equilibrium method)@\spxentry{rz2FFPrime()}\spxextra{eqtools.core.Equilibrium method}}

\begin{fulllineitems}
\phantomsection\label{\detokenize{eqtools:eqtools.core.Equilibrium.rz2FFPrime}}\pysiglinewithargsret{\sphinxbfcode{\sphinxupquote{rz2FFPrime}}}{\emph{R}, \emph{Z}, \emph{t}, \emph{**kwargs}}{}
Calculates the flux function \(FF'\) at the given (R, Z, t).

By default, EFIT only computes this inside the LCFS.
\begin{quote}\begin{description}
\item[{Parameters}] \leavevmode\begin{itemize}
\item {} 
\sphinxstyleliteralstrong{\sphinxupquote{R}} (\sphinxstyleliteralemphasis{\sphinxupquote{Array-like}}\sphinxstyleliteralemphasis{\sphinxupquote{ or }}\sphinxstyleliteralemphasis{\sphinxupquote{scalar float}}) \textendash{} Values of the radial coordinate to
map to FFPrime. If \sphinxtitleref{R} and \sphinxtitleref{Z} are both scalar values,
they are used as the coordinate pair for all of the values in
\sphinxtitleref{t}. Must have the same shape as \sphinxtitleref{Z} unless the \sphinxtitleref{make\_grid}
keyword is set. If the \sphinxtitleref{make\_grid} keyword is True, \sphinxtitleref{R} must
have exactly one dimension.

\item {} 
\sphinxstyleliteralstrong{\sphinxupquote{Z}} (\sphinxstyleliteralemphasis{\sphinxupquote{Array-like}}\sphinxstyleliteralemphasis{\sphinxupquote{ or }}\sphinxstyleliteralemphasis{\sphinxupquote{scalar float}}) \textendash{} Values of the vertical coordinate to
map to FFPrime. If \sphinxtitleref{R} and \sphinxtitleref{Z} are both scalar values,
they are used as the coordinate pair for all of the values in
\sphinxtitleref{t}. Must have the same shape as \sphinxtitleref{R} unless the \sphinxtitleref{make\_grid}
keyword is set. If the \sphinxtitleref{make\_grid} keyword is True, \sphinxtitleref{Z} must
have exactly one dimension.

\item {} 
\sphinxstyleliteralstrong{\sphinxupquote{t}} (\sphinxstyleliteralemphasis{\sphinxupquote{Array-like}}\sphinxstyleliteralemphasis{\sphinxupquote{ or }}\sphinxstyleliteralemphasis{\sphinxupquote{scalar float}}) \textendash{} Times to perform the conversion at.
If \sphinxtitleref{t} is a single value, it is used for all of the elements of
\sphinxtitleref{R}, \sphinxtitleref{Z}. If the \sphinxtitleref{each\_t} keyword is True, then \sphinxtitleref{t} must be
scalar or have exactly one dimension. If the \sphinxtitleref{each\_t} keyword is
False, \sphinxtitleref{t} must have the same shape as \sphinxtitleref{R} and \sphinxtitleref{Z} (or their
meshgrid if \sphinxtitleref{make\_grid} is True).

\end{itemize}

\item[{Keyword Arguments}] \leavevmode\begin{itemize}
\item {} 
\sphinxstyleliteralstrong{\sphinxupquote{sqrt}} (\sphinxstyleliteralemphasis{\sphinxupquote{Boolean}}) \textendash{} Set to True to return the square root of FFPrime.
Only the square root of positive values is taken. Negative
values are replaced with zeros, consistent with Steve Wolfe’s
IDL implementation efit\_rz2rho.pro. Default is False.

\item {} 
\sphinxstyleliteralstrong{\sphinxupquote{each\_t}} (\sphinxstyleliteralemphasis{\sphinxupquote{Boolean}}) \textendash{} When True, the elements in \sphinxtitleref{R}, \sphinxtitleref{Z} are evaluated
at each value in \sphinxtitleref{t}. If True, \sphinxtitleref{t} must have only one dimension
(or be a scalar). If False, \sphinxtitleref{t} must match the shape of \sphinxtitleref{R} and
\sphinxtitleref{Z} or be a scalar. Default is True (evaluate ALL \sphinxtitleref{R}, \sphinxtitleref{Z} at
EACH element in \sphinxtitleref{t}).

\item {} 
\sphinxstyleliteralstrong{\sphinxupquote{make\_grid}} (\sphinxstyleliteralemphasis{\sphinxupquote{Boolean}}) \textendash{} Set to True to pass \sphinxtitleref{R} and \sphinxtitleref{Z} through
\sphinxcode{\sphinxupquote{scipy.meshgrid()}} before evaluating. If this is set to
True, \sphinxtitleref{R} and \sphinxtitleref{Z} must each only have a single dimension, but
can have different lengths. Default is False (do not form
meshgrid).

\item {} 
\sphinxstyleliteralstrong{\sphinxupquote{length\_unit}} (\sphinxstyleliteralemphasis{\sphinxupquote{String}}\sphinxstyleliteralemphasis{\sphinxupquote{ or }}\sphinxstyleliteralemphasis{\sphinxupquote{1}}) \textendash{} 
Length unit that \sphinxtitleref{R}, \sphinxtitleref{Z} are given in.
If a string is given, it must be a valid unit specifier:
\begin{quote}


\begin{savenotes}\sphinxattablestart
\centering
\begin{tabulary}{\linewidth}[t]{|T|T|}
\hline

’m’
&
meters
\\
\hline
’cm’
&
centimeters
\\
\hline
’mm’
&
millimeters
\\
\hline
’in’
&
inches
\\
\hline
’ft’
&
feet
\\
\hline
’yd’
&
yards
\\
\hline
’smoot’
&
smoots
\\
\hline
’cubit’
&
cubits
\\
\hline
’hand’
&
hands
\\
\hline
’default’
&
meters
\\
\hline
\end{tabulary}
\par
\sphinxattableend\end{savenotes}
\end{quote}

If length\_unit is 1 or None, meters are assumed. The default
value is 1 (use meters).


\item {} 
\sphinxstyleliteralstrong{\sphinxupquote{return\_t}} (\sphinxstyleliteralemphasis{\sphinxupquote{Boolean}}) \textendash{} Set to True to return a tuple of (\sphinxtitleref{FFPrime},
\sphinxtitleref{time\_idxs}), where \sphinxtitleref{time\_idxs} is the array of time indices
actually used in evaluating \sphinxtitleref{FFPrime} with nearest-neighbor
interpolation. (This is mostly present as an internal helper.)
Default is False (only return \sphinxtitleref{FFPrime}).

\end{itemize}

\item[{Returns}] \leavevmode

\sphinxtitleref{FFPrime} or (\sphinxtitleref{FFPrime}, \sphinxtitleref{time\_idxs})
\begin{itemize}
\item {} 
\sphinxstylestrong{FFPrime} (\sphinxtitleref{Array or scalar float}) - The flux function \(FF'\).
If all of the input arguments are scalar, then a scalar is
returned. Otherwise, a scipy Array is returned. If \sphinxtitleref{R} and \sphinxtitleref{Z}
both have the same shape then \sphinxtitleref{FFPrime} has this shape as well,
unless the \sphinxtitleref{make\_grid} keyword was True, in which case \sphinxtitleref{FFPrime}
has shape (len(\sphinxtitleref{Z}), len(\sphinxtitleref{R})).

\item {} 
\sphinxstylestrong{time\_idxs} (Array with same shape as \sphinxtitleref{FFPrime}) - The indices
(in \sphinxcode{\sphinxupquote{self.getTimeBase()}}) that were used for
nearest-neighbor interpolation. Only returned if \sphinxtitleref{return\_t} is
True.

\end{itemize}


\end{description}\end{quote}
\subsubsection*{Examples}

All assume that \sphinxtitleref{Eq\_instance} is a valid instance of the
appropriate extension of the {\hyperref[\detokenize{eqtools:eqtools.core.Equilibrium}]{\sphinxcrossref{\sphinxcode{\sphinxupquote{Equilibrium}}}}} abstract class.

Find single FFPrime value at R=0.6m, Z=0.0m, t=0.26s:

\begin{sphinxVerbatim}[commandchars=\\\{\}]
\PYG{n}{FFPrime\PYGZus{}val} \PYG{o}{=} \PYG{n}{Eq\PYGZus{}instance}\PYG{o}{.}\PYG{n}{rz2FFPrime}\PYG{p}{(}\PYG{l+m+mf}{0.6}\PYG{p}{,} \PYG{l+m+mi}{0}\PYG{p}{,} \PYG{l+m+mf}{0.26}\PYG{p}{)}
\end{sphinxVerbatim}

Find FFPrime values at (R, Z) points (0.6m, 0m) and (0.8m, 0m) at the
single time t=0.26s. Note that the \sphinxtitleref{Z} vector must be fully specified,
even if the values are all the same:

\begin{sphinxVerbatim}[commandchars=\\\{\}]
\PYG{n}{FFPrime\PYGZus{}arr} \PYG{o}{=} \PYG{n}{Eq\PYGZus{}instance}\PYG{o}{.}\PYG{n}{rz2FFPrime}\PYG{p}{(}\PYG{p}{[}\PYG{l+m+mf}{0.6}\PYG{p}{,} \PYG{l+m+mf}{0.8}\PYG{p}{]}\PYG{p}{,} \PYG{p}{[}\PYG{l+m+mi}{0}\PYG{p}{,} \PYG{l+m+mi}{0}\PYG{p}{]}\PYG{p}{,} \PYG{l+m+mf}{0.26}\PYG{p}{)}
\end{sphinxVerbatim}

Find FFPrime values at (R, Z) points (0.6m, 0m) at times t={[}0.2s, 0.3s{]}:

\begin{sphinxVerbatim}[commandchars=\\\{\}]
\PYG{n}{FFPrime\PYGZus{}arr} \PYG{o}{=} \PYG{n}{Eq\PYGZus{}instance}\PYG{o}{.}\PYG{n}{rz2FFPrime}\PYG{p}{(}\PYG{l+m+mf}{0.6}\PYG{p}{,} \PYG{l+m+mi}{0}\PYG{p}{,} \PYG{p}{[}\PYG{l+m+mf}{0.2}\PYG{p}{,} \PYG{l+m+mf}{0.3}\PYG{p}{]}\PYG{p}{)}
\end{sphinxVerbatim}

Find FFPrime values at (R, Z, t) points (0.6m, 0m, 0.2s) and (0.5m, 0.2m, 0.3s):

\begin{sphinxVerbatim}[commandchars=\\\{\}]
\PYG{n}{FFPrime\PYGZus{}arr} \PYG{o}{=} \PYG{n}{Eq\PYGZus{}instance}\PYG{o}{.}\PYG{n}{rz2FFPrime}\PYG{p}{(}\PYG{p}{[}\PYG{l+m+mf}{0.6}\PYG{p}{,} \PYG{l+m+mf}{0.5}\PYG{p}{]}\PYG{p}{,} \PYG{p}{[}\PYG{l+m+mi}{0}\PYG{p}{,} \PYG{l+m+mf}{0.2}\PYG{p}{]}\PYG{p}{,} \PYG{p}{[}\PYG{l+m+mf}{0.2}\PYG{p}{,} \PYG{l+m+mf}{0.3}\PYG{p}{]}\PYG{p}{,} \PYG{n}{each\PYGZus{}t}\PYG{o}{=}\PYG{k+kc}{False}\PYG{p}{)}
\end{sphinxVerbatim}

Find FFPrime values on grid defined by 1D vector of radial positions \sphinxtitleref{R}
and 1D vector of vertical positions \sphinxtitleref{Z} at time t=0.2s:

\begin{sphinxVerbatim}[commandchars=\\\{\}]
\PYG{n}{FFPrime\PYGZus{}mat} \PYG{o}{=} \PYG{n}{Eq\PYGZus{}instance}\PYG{o}{.}\PYG{n}{rz2FFPrime}\PYG{p}{(}\PYG{n}{R}\PYG{p}{,} \PYG{n}{Z}\PYG{p}{,} \PYG{l+m+mf}{0.2}\PYG{p}{,} \PYG{n}{make\PYGZus{}grid}\PYG{o}{=}\PYG{k+kc}{True}\PYG{p}{)}
\end{sphinxVerbatim}

\end{fulllineitems}

\index{rmid2FFPrime() (eqtools.core.Equilibrium method)@\spxentry{rmid2FFPrime()}\spxextra{eqtools.core.Equilibrium method}}

\begin{fulllineitems}
\phantomsection\label{\detokenize{eqtools:eqtools.core.Equilibrium.rmid2FFPrime}}\pysiglinewithargsret{\sphinxbfcode{\sphinxupquote{rmid2FFPrime}}}{\emph{R\_mid}, \emph{t}, \emph{**kwargs}}{}
Calculates the flux function \(FF'\) corresponding to the passed R\_mid (mapped outboard midplane major radius) values.

By default, EFIT only computes this inside the LCFS.
\begin{quote}\begin{description}
\item[{Parameters}] \leavevmode\begin{itemize}
\item {} 
\sphinxstyleliteralstrong{\sphinxupquote{R\_mid}} (\sphinxstyleliteralemphasis{\sphinxupquote{Array-like}}\sphinxstyleliteralemphasis{\sphinxupquote{ or }}\sphinxstyleliteralemphasis{\sphinxupquote{scalar float}}) \textendash{} Values of the outboard midplane
major radius to map to FFPrime.

\item {} 
\sphinxstyleliteralstrong{\sphinxupquote{t}} (\sphinxstyleliteralemphasis{\sphinxupquote{Array-like}}\sphinxstyleliteralemphasis{\sphinxupquote{ or }}\sphinxstyleliteralemphasis{\sphinxupquote{scalar float}}) \textendash{} Times to perform the conversion at.
If \sphinxtitleref{t} is a single value, it is used for all of the elements of
\sphinxtitleref{R\_mid}. If the \sphinxtitleref{each\_t} keyword is True, then \sphinxtitleref{t} must be scalar
or have exactly one dimension. If the \sphinxtitleref{each\_t} keyword is False,
\sphinxtitleref{t} must have the same shape as \sphinxtitleref{R\_mid}.

\end{itemize}

\item[{Keyword Arguments}] \leavevmode\begin{itemize}
\item {} 
\sphinxstyleliteralstrong{\sphinxupquote{sqrt}} (\sphinxstyleliteralemphasis{\sphinxupquote{Boolean}}) \textendash{} Set to True to return the square root of FFPrime.
Only the square root of positive values is taken. Negative
values are replaced with zeros, consistent with Steve Wolfe’s
IDL implementation efit\_rz2rho.pro. Default is False.

\item {} 
\sphinxstyleliteralstrong{\sphinxupquote{each\_t}} (\sphinxstyleliteralemphasis{\sphinxupquote{Boolean}}) \textendash{} When True, the elements in \sphinxtitleref{R\_mid} are evaluated
at each value in \sphinxtitleref{t}. If True, \sphinxtitleref{t} must have only one dimension
(or be a scalar). If False, \sphinxtitleref{t} must match the shape of \sphinxtitleref{R\_mid}
or be a scalar. Default is True (evaluate ALL \sphinxtitleref{R\_mid} at EACH
element in \sphinxtitleref{t}).

\item {} 
\sphinxstyleliteralstrong{\sphinxupquote{length\_unit}} (\sphinxstyleliteralemphasis{\sphinxupquote{String}}\sphinxstyleliteralemphasis{\sphinxupquote{ or }}\sphinxstyleliteralemphasis{\sphinxupquote{1}}) \textendash{} 
Length unit that \sphinxtitleref{R\_mid} is given in.
If a string is given, it must be a valid unit specifier:
\begin{quote}


\begin{savenotes}\sphinxattablestart
\centering
\begin{tabulary}{\linewidth}[t]{|T|T|}
\hline

’m’
&
meters
\\
\hline
’cm’
&
centimeters
\\
\hline
’mm’
&
millimeters
\\
\hline
’in’
&
inches
\\
\hline
’ft’
&
feet
\\
\hline
’yd’
&
yards
\\
\hline
’smoot’
&
smoots
\\
\hline
’cubit’
&
cubits
\\
\hline
’hand’
&
hands
\\
\hline
’default’
&
meters
\\
\hline
\end{tabulary}
\par
\sphinxattableend\end{savenotes}
\end{quote}

If length\_unit is 1 or None, meters are assumed. The default
value is 1 (use meters).


\item {} 
\sphinxstyleliteralstrong{\sphinxupquote{k}} (\sphinxstyleliteralemphasis{\sphinxupquote{positive int}}) \textendash{} The degree of polynomial spline interpolation to
use in converting coordinates.

\item {} 
\sphinxstyleliteralstrong{\sphinxupquote{return\_t}} (\sphinxstyleliteralemphasis{\sphinxupquote{Boolean}}) \textendash{} Set to True to return a tuple of (\sphinxtitleref{FFPrime},
\sphinxtitleref{time\_idxs}), where \sphinxtitleref{time\_idxs} is the array of time indices
actually used in evaluating \sphinxtitleref{FFPrime} with nearest-neighbor
interpolation. (This is mostly present as an internal helper.)
Default is False (only return \sphinxtitleref{FFPrime}).

\end{itemize}

\item[{Returns}] \leavevmode

\sphinxtitleref{FFPrime} or (\sphinxtitleref{FFPrime}, \sphinxtitleref{time\_idxs})
\begin{itemize}
\item {} 
\sphinxstylestrong{FFPrime} (\sphinxtitleref{Array or scalar float}) - The flux function \(FF'\).
If all of the input arguments are scalar, then a scalar is
returned. Otherwise, a scipy Array is returned.

\item {} 
\sphinxstylestrong{time\_idxs} (Array with same shape as \sphinxtitleref{FFPrime}) - The indices
(in \sphinxcode{\sphinxupquote{self.getTimeBase()}}) that were used for
nearest-neighbor interpolation. Only returned if \sphinxtitleref{return\_t} is
True.

\end{itemize}


\end{description}\end{quote}
\subsubsection*{Examples}

All assume that \sphinxtitleref{Eq\_instance} is a valid instance of the appropriate
extension of the {\hyperref[\detokenize{eqtools:eqtools.core.Equilibrium}]{\sphinxcrossref{\sphinxcode{\sphinxupquote{Equilibrium}}}}} abstract class.

Find single FFPrime value for Rmid=0.7m, t=0.26s:

\begin{sphinxVerbatim}[commandchars=\\\{\}]
\PYG{n}{FFPrime\PYGZus{}val} \PYG{o}{=} \PYG{n}{Eq\PYGZus{}instance}\PYG{o}{.}\PYG{n}{rmid2FFPrime}\PYG{p}{(}\PYG{l+m+mf}{0.7}\PYG{p}{,} \PYG{l+m+mf}{0.26}\PYG{p}{)}
\end{sphinxVerbatim}

Find FFPrime values at R\_mid values of 0.5m and 0.7m at the single time
t=0.26s:

\begin{sphinxVerbatim}[commandchars=\\\{\}]
\PYG{n}{FFPrime\PYGZus{}arr} \PYG{o}{=} \PYG{n}{Eq\PYGZus{}instance}\PYG{o}{.}\PYG{n}{rmid2FFPrime}\PYG{p}{(}\PYG{p}{[}\PYG{l+m+mf}{0.5}\PYG{p}{,} \PYG{l+m+mf}{0.7}\PYG{p}{]}\PYG{p}{,} \PYG{l+m+mf}{0.26}\PYG{p}{)}
\end{sphinxVerbatim}

Find FFPrime values at R\_mid=0.5m at times t={[}0.2s, 0.3s{]}:

\begin{sphinxVerbatim}[commandchars=\\\{\}]
\PYG{n}{FFPrime\PYGZus{}arr} \PYG{o}{=} \PYG{n}{Eq\PYGZus{}instance}\PYG{o}{.}\PYG{n}{rmid2FFPrime}\PYG{p}{(}\PYG{l+m+mf}{0.5}\PYG{p}{,} \PYG{p}{[}\PYG{l+m+mf}{0.2}\PYG{p}{,} \PYG{l+m+mf}{0.3}\PYG{p}{]}\PYG{p}{)}
\end{sphinxVerbatim}

Find FFPrime values at (R\_mid, t) points (0.6m, 0.2s) and (0.5m, 0.3s):

\begin{sphinxVerbatim}[commandchars=\\\{\}]
\PYG{n}{FFPrime\PYGZus{}arr} \PYG{o}{=} \PYG{n}{Eq\PYGZus{}instance}\PYG{o}{.}\PYG{n}{rmid2FFPrime}\PYG{p}{(}\PYG{p}{[}\PYG{l+m+mf}{0.6}\PYG{p}{,} \PYG{l+m+mf}{0.5}\PYG{p}{]}\PYG{p}{,} \PYG{p}{[}\PYG{l+m+mf}{0.2}\PYG{p}{,} \PYG{l+m+mf}{0.3}\PYG{p}{]}\PYG{p}{,} \PYG{n}{each\PYGZus{}t}\PYG{o}{=}\PYG{k+kc}{False}\PYG{p}{)}
\end{sphinxVerbatim}

\end{fulllineitems}

\index{roa2FFPrime() (eqtools.core.Equilibrium method)@\spxentry{roa2FFPrime()}\spxextra{eqtools.core.Equilibrium method}}

\begin{fulllineitems}
\phantomsection\label{\detokenize{eqtools:eqtools.core.Equilibrium.roa2FFPrime}}\pysiglinewithargsret{\sphinxbfcode{\sphinxupquote{roa2FFPrime}}}{\emph{roa}, \emph{t}, \emph{**kwargs}}{}
Convert the passed (r/a, t) coordinates into the flux function \(FF'\).

By default, EFIT only computes this inside the LCFS.
\begin{quote}\begin{description}
\item[{Parameters}] \leavevmode\begin{itemize}
\item {} 
\sphinxstyleliteralstrong{\sphinxupquote{roa}} (\sphinxstyleliteralemphasis{\sphinxupquote{Array-like}}\sphinxstyleliteralemphasis{\sphinxupquote{ or }}\sphinxstyleliteralemphasis{\sphinxupquote{scalar float}}) \textendash{} Values of the normalized minor
radius to map to FFPrime.

\item {} 
\sphinxstyleliteralstrong{\sphinxupquote{t}} (\sphinxstyleliteralemphasis{\sphinxupquote{Array-like}}\sphinxstyleliteralemphasis{\sphinxupquote{ or }}\sphinxstyleliteralemphasis{\sphinxupquote{scalar float}}) \textendash{} Times to perform the conversion at.
If \sphinxtitleref{t} is a single value, it is used for all of the elements of
\sphinxtitleref{roa}. If the \sphinxtitleref{each\_t} keyword is True, then \sphinxtitleref{t} must be scalar
or have exactly one dimension. If the \sphinxtitleref{each\_t} keyword is False,
\sphinxtitleref{t} must have the same shape as \sphinxtitleref{roa}.

\end{itemize}

\item[{Keyword Arguments}] \leavevmode\begin{itemize}
\item {} 
\sphinxstyleliteralstrong{\sphinxupquote{sqrt}} (\sphinxstyleliteralemphasis{\sphinxupquote{Boolean}}) \textendash{} Set to True to return the square root of FFPrime.
Only the square root of positive values is taken. Negative
values are replaced with zeros, consistent with Steve Wolfe’s
IDL implementation efit\_rz2rho.pro. Default is False.

\item {} 
\sphinxstyleliteralstrong{\sphinxupquote{each\_t}} (\sphinxstyleliteralemphasis{\sphinxupquote{Boolean}}) \textendash{} When True, the elements in \sphinxtitleref{roa} are evaluated
at each value in \sphinxtitleref{t}. If True, \sphinxtitleref{t} must have only one dimension
(or be a scalar). If False, \sphinxtitleref{t} must match the shape of \sphinxtitleref{roa}
or be a scalar. Default is True (evaluate ALL \sphinxtitleref{roa} at EACH
element in \sphinxtitleref{t}).

\item {} 
\sphinxstyleliteralstrong{\sphinxupquote{k}} (\sphinxstyleliteralemphasis{\sphinxupquote{positive int}}) \textendash{} The degree of polynomial spline interpolation to
use in converting coordinates.

\item {} 
\sphinxstyleliteralstrong{\sphinxupquote{return\_t}} (\sphinxstyleliteralemphasis{\sphinxupquote{Boolean}}) \textendash{} Set to True to return a tuple of (\sphinxtitleref{FFPrime},
\sphinxtitleref{time\_idxs}), where \sphinxtitleref{time\_idxs} is the array of time indices
actually used in evaluating \sphinxtitleref{FFPrime} with nearest-neighbor
interpolation. (This is mostly present as an internal helper.)
Default is False (only return \sphinxtitleref{FFPrime}).

\end{itemize}

\item[{Returns}] \leavevmode

\sphinxtitleref{FFPrime} or (\sphinxtitleref{FFPrime}, \sphinxtitleref{time\_idxs})
\begin{itemize}
\item {} 
\sphinxstylestrong{FFPrime} (\sphinxtitleref{Array or scalar float}) - The flux function \(FF'\).
If all of the input arguments are scalar, then a scalar is returned.
Otherwise, a scipy Array is returned.

\item {} 
\sphinxstylestrong{time\_idxs} (Array with same shape as \sphinxtitleref{FFPrime}) - The indices
(in \sphinxcode{\sphinxupquote{self.getTimeBase()}}) that were used for
nearest-neighbor interpolation. Only returned if \sphinxtitleref{return\_t} is
True.

\end{itemize}


\end{description}\end{quote}
\subsubsection*{Examples}

All assume that \sphinxtitleref{Eq\_instance} is a valid instance of the appropriate
extension of the {\hyperref[\detokenize{eqtools:eqtools.core.Equilibrium}]{\sphinxcrossref{\sphinxcode{\sphinxupquote{Equilibrium}}}}} abstract class.

Find single FFPrime value at r/a=0.6, t=0.26s:

\begin{sphinxVerbatim}[commandchars=\\\{\}]
\PYG{n}{FFPrime\PYGZus{}val} \PYG{o}{=} \PYG{n}{Eq\PYGZus{}instance}\PYG{o}{.}\PYG{n}{roa2FFPrime}\PYG{p}{(}\PYG{l+m+mf}{0.6}\PYG{p}{,} \PYG{l+m+mf}{0.26}\PYG{p}{)}
\end{sphinxVerbatim}

Find FFPrime values at r/a points 0.6 and 0.8 at the
single time t=0.26s.:

\begin{sphinxVerbatim}[commandchars=\\\{\}]
\PYG{n}{FFPrime\PYGZus{}arr} \PYG{o}{=} \PYG{n}{Eq\PYGZus{}instance}\PYG{o}{.}\PYG{n}{roa2FFPrime}\PYG{p}{(}\PYG{p}{[}\PYG{l+m+mf}{0.6}\PYG{p}{,} \PYG{l+m+mf}{0.8}\PYG{p}{]}\PYG{p}{,} \PYG{l+m+mf}{0.26}\PYG{p}{)}
\end{sphinxVerbatim}

Find FFPrime values at r/a of 0.6 at times t={[}0.2s, 0.3s{]}:

\begin{sphinxVerbatim}[commandchars=\\\{\}]
\PYG{n}{FFPrime\PYGZus{}arr} \PYG{o}{=} \PYG{n}{Eq\PYGZus{}instance}\PYG{o}{.}\PYG{n}{roa2FFPrime}\PYG{p}{(}\PYG{l+m+mf}{0.6}\PYG{p}{,} \PYG{p}{[}\PYG{l+m+mf}{0.2}\PYG{p}{,} \PYG{l+m+mf}{0.3}\PYG{p}{]}\PYG{p}{)}
\end{sphinxVerbatim}

Find FFPrime values at (roa, t) points (0.6, 0.2s) and (0.5, 0.3s):

\begin{sphinxVerbatim}[commandchars=\\\{\}]
\PYG{n}{FFPrime\PYGZus{}arr} \PYG{o}{=} \PYG{n}{Eq\PYGZus{}instance}\PYG{o}{.}\PYG{n}{roa2FFPrime}\PYG{p}{(}\PYG{p}{[}\PYG{l+m+mf}{0.6}\PYG{p}{,} \PYG{l+m+mf}{0.5}\PYG{p}{]}\PYG{p}{,} \PYG{p}{[}\PYG{l+m+mf}{0.2}\PYG{p}{,} \PYG{l+m+mf}{0.3}\PYG{p}{]}\PYG{p}{,} \PYG{n}{each\PYGZus{}t}\PYG{o}{=}\PYG{k+kc}{False}\PYG{p}{)}
\end{sphinxVerbatim}

\end{fulllineitems}

\index{psinorm2FFPrime() (eqtools.core.Equilibrium method)@\spxentry{psinorm2FFPrime()}\spxextra{eqtools.core.Equilibrium method}}

\begin{fulllineitems}
\phantomsection\label{\detokenize{eqtools:eqtools.core.Equilibrium.psinorm2FFPrime}}\pysiglinewithargsret{\sphinxbfcode{\sphinxupquote{psinorm2FFPrime}}}{\emph{psinorm}, \emph{t}, \emph{**kwargs}}{}
Calculates the flux function \(FF'\) corresponding to the passed psi\_norm (normalized poloidal flux) values.

By default, EFIT only computes this inside the LCFS.
\begin{quote}\begin{description}
\item[{Parameters}] \leavevmode\begin{itemize}
\item {} 
\sphinxstyleliteralstrong{\sphinxupquote{psi\_norm}} (\sphinxstyleliteralemphasis{\sphinxupquote{Array-like}}\sphinxstyleliteralemphasis{\sphinxupquote{ or }}\sphinxstyleliteralemphasis{\sphinxupquote{scalar float}}) \textendash{} Values of the normalized
poloidal flux to map to FFPrime.

\item {} 
\sphinxstyleliteralstrong{\sphinxupquote{t}} (\sphinxstyleliteralemphasis{\sphinxupquote{Array-like}}\sphinxstyleliteralemphasis{\sphinxupquote{ or }}\sphinxstyleliteralemphasis{\sphinxupquote{scalar float}}) \textendash{} Times to perform the conversion at.
If \sphinxtitleref{t} is a single value, it is used for all of the elements of
\sphinxtitleref{psi\_norm}. If the \sphinxtitleref{each\_t} keyword is True, then \sphinxtitleref{t} must be scalar
or have exactly one dimension. If the \sphinxtitleref{each\_t} keyword is False,
\sphinxtitleref{t} must have the same shape as \sphinxtitleref{psi\_norm}.

\end{itemize}

\item[{Keyword Arguments}] \leavevmode\begin{itemize}
\item {} 
\sphinxstyleliteralstrong{\sphinxupquote{sqrt}} (\sphinxstyleliteralemphasis{\sphinxupquote{Boolean}}) \textendash{} Set to True to return the square root of FFPrime. Only
the square root of positive values is taken. Negative values are
replaced with zeros, consistent with Steve Wolfe’s IDL
implementation efit\_rz2rho.pro. Default is False.

\item {} 
\sphinxstyleliteralstrong{\sphinxupquote{each\_t}} (\sphinxstyleliteralemphasis{\sphinxupquote{Boolean}}) \textendash{} When True, the elements in \sphinxtitleref{psi\_norm} are evaluated at
each value in \sphinxtitleref{t}. If True, \sphinxtitleref{t} must have only one dimension (or
be a scalar). If False, \sphinxtitleref{t} must match the shape of \sphinxtitleref{psi\_norm} or be
a scalar. Default is True (evaluate ALL \sphinxtitleref{psi\_norm} at EACH element in
\sphinxtitleref{t}).

\item {} 
\sphinxstyleliteralstrong{\sphinxupquote{k}} (\sphinxstyleliteralemphasis{\sphinxupquote{positive int}}) \textendash{} The degree of polynomial spline interpolation to
use in converting coordinates.

\item {} 
\sphinxstyleliteralstrong{\sphinxupquote{return\_t}} (\sphinxstyleliteralemphasis{\sphinxupquote{Boolean}}) \textendash{} Set to True to return a tuple of (\sphinxtitleref{FFPrime},
\sphinxtitleref{time\_idxs}), where \sphinxtitleref{time\_idxs} is the array of time indices
actually used in evaluating \sphinxtitleref{FFPrime} with nearest-neighbor
interpolation. (This is mostly present as an internal helper.)
Default is False (only return \sphinxtitleref{FFPrime}).

\end{itemize}

\item[{Returns}] \leavevmode

\sphinxtitleref{FFPrime} or (\sphinxtitleref{FFPrime}, \sphinxtitleref{time\_idxs})
\begin{itemize}
\item {} 
\sphinxstylestrong{FFPrime} (\sphinxtitleref{Array or scalar float}) - The flux function \(FF'\).
If all of the input arguments are scalar, then a scalar is returned.
Otherwise, a scipy Array is returned.

\item {} 
\sphinxstylestrong{time\_idxs} (Array with same shape as \sphinxtitleref{FFPrime}) - The indices
(in \sphinxcode{\sphinxupquote{self.getTimeBase()}}) that were used for
nearest-neighbor interpolation. Only returned if \sphinxtitleref{return\_t} is
True.

\end{itemize}


\end{description}\end{quote}
\subsubsection*{Examples}

All assume that \sphinxtitleref{Eq\_instance} is a valid instance of the appropriate
extension of the {\hyperref[\detokenize{eqtools:eqtools.core.Equilibrium}]{\sphinxcrossref{\sphinxcode{\sphinxupquote{Equilibrium}}}}} abstract class.

Find single FFPrime value for psinorm=0.7, t=0.26s:

\begin{sphinxVerbatim}[commandchars=\\\{\}]
\PYG{n}{FFPrime\PYGZus{}val} \PYG{o}{=} \PYG{n}{Eq\PYGZus{}instance}\PYG{o}{.}\PYG{n}{psinorm2FFPrime}\PYG{p}{(}\PYG{l+m+mf}{0.7}\PYG{p}{,} \PYG{l+m+mf}{0.26}\PYG{p}{)}
\end{sphinxVerbatim}

Find FFPrime values at psi\_norm values of 0.5 and 0.7 at the single time
t=0.26s:

\begin{sphinxVerbatim}[commandchars=\\\{\}]
\PYG{n}{FFPrime\PYGZus{}arr} \PYG{o}{=} \PYG{n}{Eq\PYGZus{}instance}\PYG{o}{.}\PYG{n}{psinorm2FFPrime}\PYG{p}{(}\PYG{p}{[}\PYG{l+m+mf}{0.5}\PYG{p}{,} \PYG{l+m+mf}{0.7}\PYG{p}{]}\PYG{p}{,} \PYG{l+m+mf}{0.26}\PYG{p}{)}
\end{sphinxVerbatim}

Find FFPrime values at psi\_norm=0.5 at times t={[}0.2s, 0.3s{]}:

\begin{sphinxVerbatim}[commandchars=\\\{\}]
\PYG{n}{FFPrime\PYGZus{}arr} \PYG{o}{=} \PYG{n}{Eq\PYGZus{}instance}\PYG{o}{.}\PYG{n}{psinorm2FFPrime}\PYG{p}{(}\PYG{l+m+mf}{0.5}\PYG{p}{,} \PYG{p}{[}\PYG{l+m+mf}{0.2}\PYG{p}{,} \PYG{l+m+mf}{0.3}\PYG{p}{]}\PYG{p}{)}
\end{sphinxVerbatim}

Find FFPrime values at (psinorm, t) points (0.6, 0.2s) and (0.5, 0.3s):

\begin{sphinxVerbatim}[commandchars=\\\{\}]
\PYG{n}{FFPrime\PYGZus{}arr} \PYG{o}{=} \PYG{n}{Eq\PYGZus{}instance}\PYG{o}{.}\PYG{n}{psinorm2FFPrime}\PYG{p}{(}\PYG{p}{[}\PYG{l+m+mf}{0.6}\PYG{p}{,} \PYG{l+m+mf}{0.5}\PYG{p}{]}\PYG{p}{,} \PYG{p}{[}\PYG{l+m+mf}{0.2}\PYG{p}{,} \PYG{l+m+mf}{0.3}\PYG{p}{]}\PYG{p}{,} \PYG{n}{each\PYGZus{}t}\PYG{o}{=}\PYG{k+kc}{False}\PYG{p}{)}
\end{sphinxVerbatim}

\end{fulllineitems}

\index{phinorm2FFPrime() (eqtools.core.Equilibrium method)@\spxentry{phinorm2FFPrime()}\spxextra{eqtools.core.Equilibrium method}}

\begin{fulllineitems}
\phantomsection\label{\detokenize{eqtools:eqtools.core.Equilibrium.phinorm2FFPrime}}\pysiglinewithargsret{\sphinxbfcode{\sphinxupquote{phinorm2FFPrime}}}{\emph{phinorm}, \emph{t}, \emph{**kwargs}}{}
Calculates the flux function \(FF'\) corresponding to the passed phinorm (normalized toroidal flux) values.

By default, EFIT only computes this inside the LCFS.
\begin{quote}\begin{description}
\item[{Parameters}] \leavevmode\begin{itemize}
\item {} 
\sphinxstyleliteralstrong{\sphinxupquote{phinorm}} (\sphinxstyleliteralemphasis{\sphinxupquote{Array-like}}\sphinxstyleliteralemphasis{\sphinxupquote{ or }}\sphinxstyleliteralemphasis{\sphinxupquote{scalar float}}) \textendash{} Values of the normalized
toroidal flux to map to FFPrime.

\item {} 
\sphinxstyleliteralstrong{\sphinxupquote{t}} (\sphinxstyleliteralemphasis{\sphinxupquote{Array-like}}\sphinxstyleliteralemphasis{\sphinxupquote{ or }}\sphinxstyleliteralemphasis{\sphinxupquote{scalar float}}) \textendash{} Times to perform the conversion at.
If \sphinxtitleref{t} is a single value, it is used for all of the elements of
\sphinxtitleref{phinorm}. If the \sphinxtitleref{each\_t} keyword is True, then \sphinxtitleref{t} must be scalar
or have exactly one dimension. If the \sphinxtitleref{each\_t} keyword is False,
\sphinxtitleref{t} must have the same shape as \sphinxtitleref{phinorm}.

\end{itemize}

\item[{Keyword Arguments}] \leavevmode\begin{itemize}
\item {} 
\sphinxstyleliteralstrong{\sphinxupquote{sqrt}} (\sphinxstyleliteralemphasis{\sphinxupquote{Boolean}}) \textendash{} Set to True to return the square root of FFPrime.
Only the square root of positive values is taken. Negative
values are replaced with zeros, consistent with Steve Wolfe’s
IDL implementation efit\_rz2rho.pro. Default is False.

\item {} 
\sphinxstyleliteralstrong{\sphinxupquote{each\_t}} (\sphinxstyleliteralemphasis{\sphinxupquote{Boolean}}) \textendash{} When True, the elements in \sphinxtitleref{phinorm} are evaluated
at each value in \sphinxtitleref{t}. If True, \sphinxtitleref{t} must have only one dimension
(or be a scalar). If False, \sphinxtitleref{t} must match the shape of \sphinxtitleref{phinorm}
or be a scalar. Default is True (evaluate ALL \sphinxtitleref{phinorm} at EACH
element in \sphinxtitleref{t}).

\item {} 
\sphinxstyleliteralstrong{\sphinxupquote{k}} (\sphinxstyleliteralemphasis{\sphinxupquote{positive int}}) \textendash{} The degree of polynomial spline interpolation to
use in converting coordinates.

\item {} 
\sphinxstyleliteralstrong{\sphinxupquote{return\_t}} (\sphinxstyleliteralemphasis{\sphinxupquote{Boolean}}) \textendash{} Set to True to return a tuple of (\sphinxtitleref{FFPrime},
\sphinxtitleref{time\_idxs}), where \sphinxtitleref{time\_idxs} is the array of time indices
actually used in evaluating \sphinxtitleref{FFPrime} with nearest-neighbor
interpolation. (This is mostly present as an internal helper.)
Default is False (only return \sphinxtitleref{FFPrime}).

\end{itemize}

\item[{Returns}] \leavevmode

\sphinxtitleref{FFPrime} or (\sphinxtitleref{FFPrime}, \sphinxtitleref{time\_idxs})
\begin{itemize}
\item {} 
\sphinxstylestrong{FFPrime} (\sphinxtitleref{Array or scalar float}) - The flux function \(FF'\).
If all of the input arguments are scalar, then a scalar is returned.
Otherwise, a scipy Array is returned.

\item {} 
\sphinxstylestrong{time\_idxs} (Array with same shape as \sphinxtitleref{FFPrime}) - The indices
(in \sphinxcode{\sphinxupquote{self.getTimeBase()}}) that were used for
nearest-neighbor interpolation. Only returned if \sphinxtitleref{return\_t} is
True.

\end{itemize}


\end{description}\end{quote}
\subsubsection*{Examples}

All assume that \sphinxtitleref{Eq\_instance} is a valid instance of the appropriate
extension of the {\hyperref[\detokenize{eqtools:eqtools.core.Equilibrium}]{\sphinxcrossref{\sphinxcode{\sphinxupquote{Equilibrium}}}}} abstract class.

Find single FFPrime value for phinorm=0.7, t=0.26s:

\begin{sphinxVerbatim}[commandchars=\\\{\}]
\PYG{n}{FFPrime\PYGZus{}val} \PYG{o}{=} \PYG{n}{Eq\PYGZus{}instance}\PYG{o}{.}\PYG{n}{phinorm2FFPrime}\PYG{p}{(}\PYG{l+m+mf}{0.7}\PYG{p}{,} \PYG{l+m+mf}{0.26}\PYG{p}{)}
\end{sphinxVerbatim}

Find FFPrime values at phinorm values of 0.5 and 0.7 at the single time
t=0.26s:

\begin{sphinxVerbatim}[commandchars=\\\{\}]
\PYG{n}{FFPrime\PYGZus{}arr} \PYG{o}{=} \PYG{n}{Eq\PYGZus{}instance}\PYG{o}{.}\PYG{n}{phinorm2FFPrime}\PYG{p}{(}\PYG{p}{[}\PYG{l+m+mf}{0.5}\PYG{p}{,} \PYG{l+m+mf}{0.7}\PYG{p}{]}\PYG{p}{,} \PYG{l+m+mf}{0.26}\PYG{p}{)}
\end{sphinxVerbatim}

Find FFPrime values at phinorm=0.5 at times t={[}0.2s, 0.3s{]}:

\begin{sphinxVerbatim}[commandchars=\\\{\}]
\PYG{n}{FFPrime\PYGZus{}arr} \PYG{o}{=} \PYG{n}{Eq\PYGZus{}instance}\PYG{o}{.}\PYG{n}{phinorm2FFPrime}\PYG{p}{(}\PYG{l+m+mf}{0.5}\PYG{p}{,} \PYG{p}{[}\PYG{l+m+mf}{0.2}\PYG{p}{,} \PYG{l+m+mf}{0.3}\PYG{p}{]}\PYG{p}{)}
\end{sphinxVerbatim}

Find FFPrime values at (phinorm, t) points (0.6, 0.2s) and (0.5, 0.3s):

\begin{sphinxVerbatim}[commandchars=\\\{\}]
\PYG{n}{FFPrime\PYGZus{}arr} \PYG{o}{=} \PYG{n}{Eq\PYGZus{}instance}\PYG{o}{.}\PYG{n}{phinorm2FFPrime}\PYG{p}{(}\PYG{p}{[}\PYG{l+m+mf}{0.6}\PYG{p}{,} \PYG{l+m+mf}{0.5}\PYG{p}{]}\PYG{p}{,} \PYG{p}{[}\PYG{l+m+mf}{0.2}\PYG{p}{,} \PYG{l+m+mf}{0.3}\PYG{p}{]}\PYG{p}{,} \PYG{n}{each\PYGZus{}t}\PYG{o}{=}\PYG{k+kc}{False}\PYG{p}{)}
\end{sphinxVerbatim}

\end{fulllineitems}

\index{volnorm2FFPrime() (eqtools.core.Equilibrium method)@\spxentry{volnorm2FFPrime()}\spxextra{eqtools.core.Equilibrium method}}

\begin{fulllineitems}
\phantomsection\label{\detokenize{eqtools:eqtools.core.Equilibrium.volnorm2FFPrime}}\pysiglinewithargsret{\sphinxbfcode{\sphinxupquote{volnorm2FFPrime}}}{\emph{volnorm}, \emph{t}, \emph{**kwargs}}{}
Calculates the flux function \(FF'\) corresponding to the passed volnorm (normalized flux surface volume) values.

By default, EFIT only computes this inside the LCFS.
\begin{quote}\begin{description}
\item[{Parameters}] \leavevmode\begin{itemize}
\item {} 
\sphinxstyleliteralstrong{\sphinxupquote{volnorm}} (\sphinxstyleliteralemphasis{\sphinxupquote{Array-like}}\sphinxstyleliteralemphasis{\sphinxupquote{ or }}\sphinxstyleliteralemphasis{\sphinxupquote{scalar float}}) \textendash{} Values of the normalized
flux surface volume to map to FFPrime.

\item {} 
\sphinxstyleliteralstrong{\sphinxupquote{t}} (\sphinxstyleliteralemphasis{\sphinxupquote{Array-like}}\sphinxstyleliteralemphasis{\sphinxupquote{ or }}\sphinxstyleliteralemphasis{\sphinxupquote{scalar float}}) \textendash{} Times to perform the conversion at.
If \sphinxtitleref{t} is a single value, it is used for all of the elements of
\sphinxtitleref{volnorm}. If the \sphinxtitleref{each\_t} keyword is True, then \sphinxtitleref{t} must be scalar
or have exactly one dimension. If the \sphinxtitleref{each\_t} keyword is False,
\sphinxtitleref{t} must have the same shape as \sphinxtitleref{volnorm}.

\end{itemize}

\item[{Keyword Arguments}] \leavevmode\begin{itemize}
\item {} 
\sphinxstyleliteralstrong{\sphinxupquote{sqrt}} (\sphinxstyleliteralemphasis{\sphinxupquote{Boolean}}) \textendash{} Set to True to return the square root of FFPrime.
Only the square root of positive values is taken. Negative
values are replaced with zeros, consistent with Steve Wolfe’s
IDL implementation efit\_rz2rho.pro. Default is False.

\item {} 
\sphinxstyleliteralstrong{\sphinxupquote{each\_t}} (\sphinxstyleliteralemphasis{\sphinxupquote{Boolean}}) \textendash{} When True, the elements in \sphinxtitleref{volnorm} are evaluated
at each value in \sphinxtitleref{t}. If True, \sphinxtitleref{t} must have only one dimension
(or be a scalar). If False, \sphinxtitleref{t} must match the shape of \sphinxtitleref{volnorm}
or be a scalar. Default is True (evaluate ALL \sphinxtitleref{volnorm} at EACH
element in \sphinxtitleref{t}).

\item {} 
\sphinxstyleliteralstrong{\sphinxupquote{k}} (\sphinxstyleliteralemphasis{\sphinxupquote{positive int}}) \textendash{} The degree of polynomial spline interpolation to
use in converting coordinates.

\item {} 
\sphinxstyleliteralstrong{\sphinxupquote{return\_t}} (\sphinxstyleliteralemphasis{\sphinxupquote{Boolean}}) \textendash{} Set to True to return a tuple of (\sphinxtitleref{FFPrime},
\sphinxtitleref{time\_idxs}), where \sphinxtitleref{time\_idxs} is the array of time indices
actually used in evaluating \sphinxtitleref{FFPrime} with nearest-neighbor
interpolation. (This is mostly present as an internal helper.)
Default is False (only return \sphinxtitleref{FFPrime}).

\end{itemize}

\item[{Returns}] \leavevmode

\sphinxtitleref{FFPrime} or (\sphinxtitleref{FFPrime}, \sphinxtitleref{time\_idxs})
\begin{itemize}
\item {} 
\sphinxstylestrong{FFPrime} (\sphinxtitleref{Array or scalar float}) - The flux function \(FF'\).
If all of the input arguments are scalar, then a scalar is returned.
Otherwise, a scipy Array is returned.

\item {} 
\sphinxstylestrong{time\_idxs} (Array with same shape as \sphinxtitleref{FFPrime}) - The indices
(in \sphinxcode{\sphinxupquote{self.getTimeBase()}}) that were used for
nearest-neighbor interpolation. Only returned if \sphinxtitleref{return\_t} is
True.

\end{itemize}


\end{description}\end{quote}
\subsubsection*{Examples}

All assume that \sphinxtitleref{Eq\_instance} is a valid instance of the appropriate
extension of the {\hyperref[\detokenize{eqtools:eqtools.core.Equilibrium}]{\sphinxcrossref{\sphinxcode{\sphinxupquote{Equilibrium}}}}} abstract class.

Find single FFPrime value for volnorm=0.7, t=0.26s:

\begin{sphinxVerbatim}[commandchars=\\\{\}]
\PYG{n}{FFPrime\PYGZus{}val} \PYG{o}{=} \PYG{n}{Eq\PYGZus{}instance}\PYG{o}{.}\PYG{n}{volnorm2FFPrime}\PYG{p}{(}\PYG{l+m+mf}{0.7}\PYG{p}{,} \PYG{l+m+mf}{0.26}\PYG{p}{)}
\end{sphinxVerbatim}

Find FFPrime values at volnorm values of 0.5 and 0.7 at the single time
t=0.26s:

\begin{sphinxVerbatim}[commandchars=\\\{\}]
\PYG{n}{FFPrime\PYGZus{}arr} \PYG{o}{=} \PYG{n}{Eq\PYGZus{}instance}\PYG{o}{.}\PYG{n}{volnorm2FFPrime}\PYG{p}{(}\PYG{p}{[}\PYG{l+m+mf}{0.5}\PYG{p}{,} \PYG{l+m+mf}{0.7}\PYG{p}{]}\PYG{p}{,} \PYG{l+m+mf}{0.26}\PYG{p}{)}
\end{sphinxVerbatim}

Find FFPrime values at volnorm=0.5 at times t={[}0.2s, 0.3s{]}:

\begin{sphinxVerbatim}[commandchars=\\\{\}]
\PYG{n}{FFPrime\PYGZus{}arr} \PYG{o}{=} \PYG{n}{Eq\PYGZus{}instance}\PYG{o}{.}\PYG{n}{volnorm2FFPrime}\PYG{p}{(}\PYG{l+m+mf}{0.5}\PYG{p}{,} \PYG{p}{[}\PYG{l+m+mf}{0.2}\PYG{p}{,} \PYG{l+m+mf}{0.3}\PYG{p}{]}\PYG{p}{)}
\end{sphinxVerbatim}

Find FFPrime values at (volnorm, t) points (0.6, 0.2s) and (0.5, 0.3s):

\begin{sphinxVerbatim}[commandchars=\\\{\}]
\PYG{n}{FFPrime\PYGZus{}arr} \PYG{o}{=} \PYG{n}{Eq\PYGZus{}instance}\PYG{o}{.}\PYG{n}{volnorm2FFPrime}\PYG{p}{(}\PYG{p}{[}\PYG{l+m+mf}{0.6}\PYG{p}{,} \PYG{l+m+mf}{0.5}\PYG{p}{]}\PYG{p}{,} \PYG{p}{[}\PYG{l+m+mf}{0.2}\PYG{p}{,} \PYG{l+m+mf}{0.3}\PYG{p}{]}\PYG{p}{,} \PYG{n}{each\PYGZus{}t}\PYG{o}{=}\PYG{k+kc}{False}\PYG{p}{)}
\end{sphinxVerbatim}

\end{fulllineitems}

\index{rz2p() (eqtools.core.Equilibrium method)@\spxentry{rz2p()}\spxextra{eqtools.core.Equilibrium method}}

\begin{fulllineitems}
\phantomsection\label{\detokenize{eqtools:eqtools.core.Equilibrium.rz2p}}\pysiglinewithargsret{\sphinxbfcode{\sphinxupquote{rz2p}}}{\emph{R}, \emph{Z}, \emph{t}, \emph{**kwargs}}{}
Calculates the pressure at the given (R, Z, t).

By default, EFIT only computes this inside the LCFS.
\begin{quote}\begin{description}
\item[{Parameters}] \leavevmode\begin{itemize}
\item {} 
\sphinxstyleliteralstrong{\sphinxupquote{R}} (\sphinxstyleliteralemphasis{\sphinxupquote{Array-like}}\sphinxstyleliteralemphasis{\sphinxupquote{ or }}\sphinxstyleliteralemphasis{\sphinxupquote{scalar float}}) \textendash{} Values of the radial coordinate to
map to p. If \sphinxtitleref{R} and \sphinxtitleref{Z} are both scalar values,
they are used as the coordinate pair for all of the values in
\sphinxtitleref{t}. Must have the same shape as \sphinxtitleref{Z} unless the \sphinxtitleref{make\_grid}
keyword is set. If the \sphinxtitleref{make\_grid} keyword is True, \sphinxtitleref{R} must
have exactly one dimension.

\item {} 
\sphinxstyleliteralstrong{\sphinxupquote{Z}} (\sphinxstyleliteralemphasis{\sphinxupquote{Array-like}}\sphinxstyleliteralemphasis{\sphinxupquote{ or }}\sphinxstyleliteralemphasis{\sphinxupquote{scalar float}}) \textendash{} Values of the vertical coordinate to
map to p. If \sphinxtitleref{R} and \sphinxtitleref{Z} are both scalar values,
they are used as the coordinate pair for all of the values in
\sphinxtitleref{t}. Must have the same shape as \sphinxtitleref{R} unless the \sphinxtitleref{make\_grid}
keyword is set. If the \sphinxtitleref{make\_grid} keyword is True, \sphinxtitleref{Z} must
have exactly one dimension.

\item {} 
\sphinxstyleliteralstrong{\sphinxupquote{t}} (\sphinxstyleliteralemphasis{\sphinxupquote{Array-like}}\sphinxstyleliteralemphasis{\sphinxupquote{ or }}\sphinxstyleliteralemphasis{\sphinxupquote{scalar float}}) \textendash{} Times to perform the conversion at.
If \sphinxtitleref{t} is a single value, it is used for all of the elements of
\sphinxtitleref{R}, \sphinxtitleref{Z}. If the \sphinxtitleref{each\_t} keyword is True, then \sphinxtitleref{t} must be
scalar or have exactly one dimension. If the \sphinxtitleref{each\_t} keyword is
False, \sphinxtitleref{t} must have the same shape as \sphinxtitleref{R} and \sphinxtitleref{Z} (or their
meshgrid if \sphinxtitleref{make\_grid} is True).

\end{itemize}

\item[{Keyword Arguments}] \leavevmode\begin{itemize}
\item {} 
\sphinxstyleliteralstrong{\sphinxupquote{sqrt}} (\sphinxstyleliteralemphasis{\sphinxupquote{Boolean}}) \textendash{} Set to True to return the square root of p.
Only the square root of positive values is taken. Negative
values are replaced with zeros, consistent with Steve Wolfe’s
IDL implementation efit\_rz2rho.pro. Default is False.

\item {} 
\sphinxstyleliteralstrong{\sphinxupquote{each\_t}} (\sphinxstyleliteralemphasis{\sphinxupquote{Boolean}}) \textendash{} When True, the elements in \sphinxtitleref{R}, \sphinxtitleref{Z} are evaluated
at each value in \sphinxtitleref{t}. If True, \sphinxtitleref{t} must have only one dimension
(or be a scalar). If False, \sphinxtitleref{t} must match the shape of \sphinxtitleref{R} and
\sphinxtitleref{Z} or be a scalar. Default is True (evaluate ALL \sphinxtitleref{R}, \sphinxtitleref{Z} at
EACH element in \sphinxtitleref{t}).

\item {} 
\sphinxstyleliteralstrong{\sphinxupquote{make\_grid}} (\sphinxstyleliteralemphasis{\sphinxupquote{Boolean}}) \textendash{} Set to True to pass \sphinxtitleref{R} and \sphinxtitleref{Z} through
\sphinxcode{\sphinxupquote{scipy.meshgrid()}} before evaluating. If this is set to
True, \sphinxtitleref{R} and \sphinxtitleref{Z} must each only have a single dimension, but
can have different lengths. Default is False (do not form
meshgrid).

\item {} 
\sphinxstyleliteralstrong{\sphinxupquote{length\_unit}} (\sphinxstyleliteralemphasis{\sphinxupquote{String}}\sphinxstyleliteralemphasis{\sphinxupquote{ or }}\sphinxstyleliteralemphasis{\sphinxupquote{1}}) \textendash{} 
Length unit that \sphinxtitleref{R}, \sphinxtitleref{Z} are given in.
If a string is given, it must be a valid unit specifier:
\begin{quote}


\begin{savenotes}\sphinxattablestart
\centering
\begin{tabulary}{\linewidth}[t]{|T|T|}
\hline

’m’
&
meters
\\
\hline
’cm’
&
centimeters
\\
\hline
’mm’
&
millimeters
\\
\hline
’in’
&
inches
\\
\hline
’ft’
&
feet
\\
\hline
’yd’
&
yards
\\
\hline
’smoot’
&
smoots
\\
\hline
’cubit’
&
cubits
\\
\hline
’hand’
&
hands
\\
\hline
’default’
&
meters
\\
\hline
\end{tabulary}
\par
\sphinxattableend\end{savenotes}
\end{quote}

If length\_unit is 1 or None, meters are assumed. The default
value is 1 (use meters).


\item {} 
\sphinxstyleliteralstrong{\sphinxupquote{return\_t}} (\sphinxstyleliteralemphasis{\sphinxupquote{Boolean}}) \textendash{} Set to True to return a tuple of (\sphinxtitleref{p},
\sphinxtitleref{time\_idxs}), where \sphinxtitleref{time\_idxs} is the array of time indices
actually used in evaluating \sphinxtitleref{p} with nearest-neighbor
interpolation. (This is mostly present as an internal helper.)
Default is False (only return \sphinxtitleref{p}).

\end{itemize}

\item[{Returns}] \leavevmode

\sphinxtitleref{p} or (\sphinxtitleref{p}, \sphinxtitleref{time\_idxs})
\begin{itemize}
\item {} 
\sphinxstylestrong{p} (\sphinxtitleref{Array or scalar float}) - The pressure. If all
of the input arguments are scalar, then a scalar is
returned. Otherwise, a scipy Array is returned. If \sphinxtitleref{R} and \sphinxtitleref{Z}
both have the same shape then \sphinxtitleref{p} has this shape as well,
unless the \sphinxtitleref{make\_grid} keyword was True, in which case \sphinxtitleref{p}
has shape (len(\sphinxtitleref{Z}), len(\sphinxtitleref{R})).

\item {} 
\sphinxstylestrong{time\_idxs} (Array with same shape as \sphinxtitleref{p}) - The indices
(in \sphinxcode{\sphinxupquote{self.getTimeBase()}}) that were used for
nearest-neighbor interpolation. Only returned if \sphinxtitleref{return\_t} is
True.

\end{itemize}


\end{description}\end{quote}
\subsubsection*{Examples}

All assume that \sphinxtitleref{Eq\_instance} is a valid instance of the
appropriate extension of the {\hyperref[\detokenize{eqtools:eqtools.core.Equilibrium}]{\sphinxcrossref{\sphinxcode{\sphinxupquote{Equilibrium}}}}} abstract class.

Find single p value at R=0.6m, Z=0.0m, t=0.26s:

\begin{sphinxVerbatim}[commandchars=\\\{\}]
\PYG{n}{p\PYGZus{}val} \PYG{o}{=} \PYG{n}{Eq\PYGZus{}instance}\PYG{o}{.}\PYG{n}{rz2p}\PYG{p}{(}\PYG{l+m+mf}{0.6}\PYG{p}{,} \PYG{l+m+mi}{0}\PYG{p}{,} \PYG{l+m+mf}{0.26}\PYG{p}{)}
\end{sphinxVerbatim}

Find p values at (R, Z) points (0.6m, 0m) and (0.8m, 0m) at the
single time t=0.26s. Note that the \sphinxtitleref{Z} vector must be fully specified,
even if the values are all the same:

\begin{sphinxVerbatim}[commandchars=\\\{\}]
\PYG{n}{p\PYGZus{}arr} \PYG{o}{=} \PYG{n}{Eq\PYGZus{}instance}\PYG{o}{.}\PYG{n}{rz2p}\PYG{p}{(}\PYG{p}{[}\PYG{l+m+mf}{0.6}\PYG{p}{,} \PYG{l+m+mf}{0.8}\PYG{p}{]}\PYG{p}{,} \PYG{p}{[}\PYG{l+m+mi}{0}\PYG{p}{,} \PYG{l+m+mi}{0}\PYG{p}{]}\PYG{p}{,} \PYG{l+m+mf}{0.26}\PYG{p}{)}
\end{sphinxVerbatim}

Find p values at (R, Z) points (0.6m, 0m) at times t={[}0.2s, 0.3s{]}:

\begin{sphinxVerbatim}[commandchars=\\\{\}]
\PYG{n}{p\PYGZus{}arr} \PYG{o}{=} \PYG{n}{Eq\PYGZus{}instance}\PYG{o}{.}\PYG{n}{rz2p}\PYG{p}{(}\PYG{l+m+mf}{0.6}\PYG{p}{,} \PYG{l+m+mi}{0}\PYG{p}{,} \PYG{p}{[}\PYG{l+m+mf}{0.2}\PYG{p}{,} \PYG{l+m+mf}{0.3}\PYG{p}{]}\PYG{p}{)}
\end{sphinxVerbatim}

Find p values at (R, Z, t) points (0.6m, 0m, 0.2s) and (0.5m, 0.2m, 0.3s):

\begin{sphinxVerbatim}[commandchars=\\\{\}]
\PYG{n}{p\PYGZus{}arr} \PYG{o}{=} \PYG{n}{Eq\PYGZus{}instance}\PYG{o}{.}\PYG{n}{rz2p}\PYG{p}{(}\PYG{p}{[}\PYG{l+m+mf}{0.6}\PYG{p}{,} \PYG{l+m+mf}{0.5}\PYG{p}{]}\PYG{p}{,} \PYG{p}{[}\PYG{l+m+mi}{0}\PYG{p}{,} \PYG{l+m+mf}{0.2}\PYG{p}{]}\PYG{p}{,} \PYG{p}{[}\PYG{l+m+mf}{0.2}\PYG{p}{,} \PYG{l+m+mf}{0.3}\PYG{p}{]}\PYG{p}{,} \PYG{n}{each\PYGZus{}t}\PYG{o}{=}\PYG{k+kc}{False}\PYG{p}{)}
\end{sphinxVerbatim}

Find p values on grid defined by 1D vector of radial positions \sphinxtitleref{R}
and 1D vector of vertical positions \sphinxtitleref{Z} at time t=0.2s:

\begin{sphinxVerbatim}[commandchars=\\\{\}]
\PYG{n}{p\PYGZus{}mat} \PYG{o}{=} \PYG{n}{Eq\PYGZus{}instance}\PYG{o}{.}\PYG{n}{rz2p}\PYG{p}{(}\PYG{n}{R}\PYG{p}{,} \PYG{n}{Z}\PYG{p}{,} \PYG{l+m+mf}{0.2}\PYG{p}{,} \PYG{n}{make\PYGZus{}grid}\PYG{o}{=}\PYG{k+kc}{True}\PYG{p}{)}
\end{sphinxVerbatim}

\end{fulllineitems}

\index{rmid2p() (eqtools.core.Equilibrium method)@\spxentry{rmid2p()}\spxextra{eqtools.core.Equilibrium method}}

\begin{fulllineitems}
\phantomsection\label{\detokenize{eqtools:eqtools.core.Equilibrium.rmid2p}}\pysiglinewithargsret{\sphinxbfcode{\sphinxupquote{rmid2p}}}{\emph{R\_mid}, \emph{t}, \emph{**kwargs}}{}
Calculates the pressure corresponding to the passed R\_mid (mapped outboard midplane major radius) values.

By default, EFIT only computes this inside the LCFS.
\begin{quote}\begin{description}
\item[{Parameters}] \leavevmode\begin{itemize}
\item {} 
\sphinxstyleliteralstrong{\sphinxupquote{R\_mid}} (\sphinxstyleliteralemphasis{\sphinxupquote{Array-like}}\sphinxstyleliteralemphasis{\sphinxupquote{ or }}\sphinxstyleliteralemphasis{\sphinxupquote{scalar float}}) \textendash{} Values of the outboard midplane
major radius to map to p.

\item {} 
\sphinxstyleliteralstrong{\sphinxupquote{t}} (\sphinxstyleliteralemphasis{\sphinxupquote{Array-like}}\sphinxstyleliteralemphasis{\sphinxupquote{ or }}\sphinxstyleliteralemphasis{\sphinxupquote{scalar float}}) \textendash{} Times to perform the conversion at.
If \sphinxtitleref{t} is a single value, it is used for all of the elements of
\sphinxtitleref{R\_mid}. If the \sphinxtitleref{each\_t} keyword is True, then \sphinxtitleref{t} must be scalar
or have exactly one dimension. If the \sphinxtitleref{each\_t} keyword is False,
\sphinxtitleref{t} must have the same shape as \sphinxtitleref{R\_mid}.

\end{itemize}

\item[{Keyword Arguments}] \leavevmode\begin{itemize}
\item {} 
\sphinxstyleliteralstrong{\sphinxupquote{sqrt}} (\sphinxstyleliteralemphasis{\sphinxupquote{Boolean}}) \textendash{} Set to True to return the square root of p.
Only the square root of positive values is taken. Negative
values are replaced with zeros, consistent with Steve Wolfe’s
IDL implementation efit\_rz2rho.pro. Default is False.

\item {} 
\sphinxstyleliteralstrong{\sphinxupquote{each\_t}} (\sphinxstyleliteralemphasis{\sphinxupquote{Boolean}}) \textendash{} When True, the elements in \sphinxtitleref{R\_mid} are evaluated
at each value in \sphinxtitleref{t}. If True, \sphinxtitleref{t} must have only one dimension
(or be a scalar). If False, \sphinxtitleref{t} must match the shape of \sphinxtitleref{R\_mid}
or be a scalar. Default is True (evaluate ALL \sphinxtitleref{R\_mid} at EACH
element in \sphinxtitleref{t}).

\item {} 
\sphinxstyleliteralstrong{\sphinxupquote{length\_unit}} (\sphinxstyleliteralemphasis{\sphinxupquote{String}}\sphinxstyleliteralemphasis{\sphinxupquote{ or }}\sphinxstyleliteralemphasis{\sphinxupquote{1}}) \textendash{} 
Length unit that \sphinxtitleref{R\_mid} is given in.
If a string is given, it must be a valid unit specifier:
\begin{quote}


\begin{savenotes}\sphinxattablestart
\centering
\begin{tabulary}{\linewidth}[t]{|T|T|}
\hline

’m’
&
meters
\\
\hline
’cm’
&
centimeters
\\
\hline
’mm’
&
millimeters
\\
\hline
’in’
&
inches
\\
\hline
’ft’
&
feet
\\
\hline
’yd’
&
yards
\\
\hline
’smoot’
&
smoots
\\
\hline
’cubit’
&
cubits
\\
\hline
’hand’
&
hands
\\
\hline
’default’
&
meters
\\
\hline
\end{tabulary}
\par
\sphinxattableend\end{savenotes}
\end{quote}

If length\_unit is 1 or None, meters are assumed. The default
value is 1 (use meters).


\item {} 
\sphinxstyleliteralstrong{\sphinxupquote{k}} (\sphinxstyleliteralemphasis{\sphinxupquote{positive int}}) \textendash{} The degree of polynomial spline interpolation to
use in converting coordinates.

\item {} 
\sphinxstyleliteralstrong{\sphinxupquote{return\_t}} (\sphinxstyleliteralemphasis{\sphinxupquote{Boolean}}) \textendash{} Set to True to return a tuple of (\sphinxtitleref{p},
\sphinxtitleref{time\_idxs}), where \sphinxtitleref{time\_idxs} is the array of time indices
actually used in evaluating \sphinxtitleref{p} with nearest-neighbor
interpolation. (This is mostly present as an internal helper.)
Default is False (only return \sphinxtitleref{p}).

\end{itemize}

\item[{Returns}] \leavevmode

\sphinxtitleref{p} or (\sphinxtitleref{p}, \sphinxtitleref{time\_idxs})
\begin{itemize}
\item {} 
\sphinxstylestrong{p} (\sphinxtitleref{Array or scalar float}) - The pressure.
If all of the input arguments are scalar, then a scalar is
returned. Otherwise, a scipy Array is returned.

\item {} 
\sphinxstylestrong{time\_idxs} (Array with same shape as \sphinxtitleref{p}) - The indices
(in \sphinxcode{\sphinxupquote{self.getTimeBase()}}) that were used for
nearest-neighbor interpolation. Only returned if \sphinxtitleref{return\_t} is
True.

\end{itemize}


\end{description}\end{quote}
\subsubsection*{Examples}

All assume that \sphinxtitleref{Eq\_instance} is a valid instance of the appropriate
extension of the {\hyperref[\detokenize{eqtools:eqtools.core.Equilibrium}]{\sphinxcrossref{\sphinxcode{\sphinxupquote{Equilibrium}}}}} abstract class.

Find single p value for Rmid=0.7m, t=0.26s:

\begin{sphinxVerbatim}[commandchars=\\\{\}]
\PYG{n}{p\PYGZus{}val} \PYG{o}{=} \PYG{n}{Eq\PYGZus{}instance}\PYG{o}{.}\PYG{n}{rmid2p}\PYG{p}{(}\PYG{l+m+mf}{0.7}\PYG{p}{,} \PYG{l+m+mf}{0.26}\PYG{p}{)}
\end{sphinxVerbatim}

Find p values at R\_mid values of 0.5m and 0.7m at the single time
t=0.26s:

\begin{sphinxVerbatim}[commandchars=\\\{\}]
\PYG{n}{p\PYGZus{}arr} \PYG{o}{=} \PYG{n}{Eq\PYGZus{}instance}\PYG{o}{.}\PYG{n}{rmid2p}\PYG{p}{(}\PYG{p}{[}\PYG{l+m+mf}{0.5}\PYG{p}{,} \PYG{l+m+mf}{0.7}\PYG{p}{]}\PYG{p}{,} \PYG{l+m+mf}{0.26}\PYG{p}{)}
\end{sphinxVerbatim}

Find p values at R\_mid=0.5m at times t={[}0.2s, 0.3s{]}:

\begin{sphinxVerbatim}[commandchars=\\\{\}]
\PYG{n}{p\PYGZus{}arr} \PYG{o}{=} \PYG{n}{Eq\PYGZus{}instance}\PYG{o}{.}\PYG{n}{rmid2p}\PYG{p}{(}\PYG{l+m+mf}{0.5}\PYG{p}{,} \PYG{p}{[}\PYG{l+m+mf}{0.2}\PYG{p}{,} \PYG{l+m+mf}{0.3}\PYG{p}{]}\PYG{p}{)}
\end{sphinxVerbatim}

Find p values at (R\_mid, t) points (0.6m, 0.2s) and (0.5m, 0.3s):

\begin{sphinxVerbatim}[commandchars=\\\{\}]
\PYG{n}{p\PYGZus{}arr} \PYG{o}{=} \PYG{n}{Eq\PYGZus{}instance}\PYG{o}{.}\PYG{n}{rmid2p}\PYG{p}{(}\PYG{p}{[}\PYG{l+m+mf}{0.6}\PYG{p}{,} \PYG{l+m+mf}{0.5}\PYG{p}{]}\PYG{p}{,} \PYG{p}{[}\PYG{l+m+mf}{0.2}\PYG{p}{,} \PYG{l+m+mf}{0.3}\PYG{p}{]}\PYG{p}{,} \PYG{n}{each\PYGZus{}t}\PYG{o}{=}\PYG{k+kc}{False}\PYG{p}{)}
\end{sphinxVerbatim}

\end{fulllineitems}

\index{roa2p() (eqtools.core.Equilibrium method)@\spxentry{roa2p()}\spxextra{eqtools.core.Equilibrium method}}

\begin{fulllineitems}
\phantomsection\label{\detokenize{eqtools:eqtools.core.Equilibrium.roa2p}}\pysiglinewithargsret{\sphinxbfcode{\sphinxupquote{roa2p}}}{\emph{roa}, \emph{t}, \emph{**kwargs}}{}
Convert the passed (r/a, t) coordinates into pressure.

By default, EFIT only computes this inside the LCFS.
\begin{quote}\begin{description}
\item[{Parameters}] \leavevmode\begin{itemize}
\item {} 
\sphinxstyleliteralstrong{\sphinxupquote{roa}} (\sphinxstyleliteralemphasis{\sphinxupquote{Array-like}}\sphinxstyleliteralemphasis{\sphinxupquote{ or }}\sphinxstyleliteralemphasis{\sphinxupquote{scalar float}}) \textendash{} Values of the normalized minor
radius to map to p.

\item {} 
\sphinxstyleliteralstrong{\sphinxupquote{t}} (\sphinxstyleliteralemphasis{\sphinxupquote{Array-like}}\sphinxstyleliteralemphasis{\sphinxupquote{ or }}\sphinxstyleliteralemphasis{\sphinxupquote{scalar float}}) \textendash{} Times to perform the conversion at.
If \sphinxtitleref{t} is a single value, it is used for all of the elements of
\sphinxtitleref{roa}. If the \sphinxtitleref{each\_t} keyword is True, then \sphinxtitleref{t} must be scalar
or have exactly one dimension. If the \sphinxtitleref{each\_t} keyword is False,
\sphinxtitleref{t} must have the same shape as \sphinxtitleref{roa}.

\end{itemize}

\item[{Keyword Arguments}] \leavevmode\begin{itemize}
\item {} 
\sphinxstyleliteralstrong{\sphinxupquote{sqrt}} (\sphinxstyleliteralemphasis{\sphinxupquote{Boolean}}) \textendash{} Set to True to return the square root of p.
Only the square root of positive values is taken. Negative
values are replaced with zeros, consistent with Steve Wolfe’s
IDL implementation efit\_rz2rho.pro. Default is False.

\item {} 
\sphinxstyleliteralstrong{\sphinxupquote{each\_t}} (\sphinxstyleliteralemphasis{\sphinxupquote{Boolean}}) \textendash{} When True, the elements in \sphinxtitleref{roa} are evaluated
at each value in \sphinxtitleref{t}. If True, \sphinxtitleref{t} must have only one dimension
(or be a scalar). If False, \sphinxtitleref{t} must match the shape of \sphinxtitleref{roa}
or be a scalar. Default is True (evaluate ALL \sphinxtitleref{roa} at EACH
element in \sphinxtitleref{t}).

\item {} 
\sphinxstyleliteralstrong{\sphinxupquote{k}} (\sphinxstyleliteralemphasis{\sphinxupquote{positive int}}) \textendash{} The degree of polynomial spline interpolation to
use in converting coordinates.

\item {} 
\sphinxstyleliteralstrong{\sphinxupquote{return\_t}} (\sphinxstyleliteralemphasis{\sphinxupquote{Boolean}}) \textendash{} Set to True to return a tuple of (\sphinxtitleref{p},
\sphinxtitleref{time\_idxs}), where \sphinxtitleref{time\_idxs} is the array of time indices
actually used in evaluating \sphinxtitleref{p} with nearest-neighbor
interpolation. (This is mostly present as an internal helper.)
Default is False (only return \sphinxtitleref{p}).

\end{itemize}

\item[{Returns}] \leavevmode

\sphinxtitleref{p} or (\sphinxtitleref{p}, \sphinxtitleref{time\_idxs})
\begin{itemize}
\item {} 
\sphinxstylestrong{p} (\sphinxtitleref{Array or scalar float}) - The pressure. If
all of the input arguments are scalar, then a scalar is returned.
Otherwise, a scipy Array is returned.

\item {} 
\sphinxstylestrong{time\_idxs} (Array with same shape as \sphinxtitleref{p}) - The indices
(in \sphinxcode{\sphinxupquote{self.getTimeBase()}}) that were used for
nearest-neighbor interpolation. Only returned if \sphinxtitleref{return\_t} is
True.

\end{itemize}


\end{description}\end{quote}
\subsubsection*{Examples}

All assume that \sphinxtitleref{Eq\_instance} is a valid instance of the appropriate
extension of the {\hyperref[\detokenize{eqtools:eqtools.core.Equilibrium}]{\sphinxcrossref{\sphinxcode{\sphinxupquote{Equilibrium}}}}} abstract class.

Find single p value at r/a=0.6, t=0.26s:

\begin{sphinxVerbatim}[commandchars=\\\{\}]
\PYG{n}{p\PYGZus{}val} \PYG{o}{=} \PYG{n}{Eq\PYGZus{}instance}\PYG{o}{.}\PYG{n}{roa2p}\PYG{p}{(}\PYG{l+m+mf}{0.6}\PYG{p}{,} \PYG{l+m+mf}{0.26}\PYG{p}{)}
\end{sphinxVerbatim}

Find p values at r/a points 0.6 and 0.8 at the
single time t=0.26s.:

\begin{sphinxVerbatim}[commandchars=\\\{\}]
\PYG{n}{p\PYGZus{}arr} \PYG{o}{=} \PYG{n}{Eq\PYGZus{}instance}\PYG{o}{.}\PYG{n}{roa2p}\PYG{p}{(}\PYG{p}{[}\PYG{l+m+mf}{0.6}\PYG{p}{,} \PYG{l+m+mf}{0.8}\PYG{p}{]}\PYG{p}{,} \PYG{l+m+mf}{0.26}\PYG{p}{)}
\end{sphinxVerbatim}

Find p values at r/a of 0.6 at times t={[}0.2s, 0.3s{]}:

\begin{sphinxVerbatim}[commandchars=\\\{\}]
\PYG{n}{p\PYGZus{}arr} \PYG{o}{=} \PYG{n}{Eq\PYGZus{}instance}\PYG{o}{.}\PYG{n}{roa2p}\PYG{p}{(}\PYG{l+m+mf}{0.6}\PYG{p}{,} \PYG{p}{[}\PYG{l+m+mf}{0.2}\PYG{p}{,} \PYG{l+m+mf}{0.3}\PYG{p}{]}\PYG{p}{)}
\end{sphinxVerbatim}

Find p values at (roa, t) points (0.6, 0.2s) and (0.5, 0.3s):

\begin{sphinxVerbatim}[commandchars=\\\{\}]
\PYG{n}{p\PYGZus{}arr} \PYG{o}{=} \PYG{n}{Eq\PYGZus{}instance}\PYG{o}{.}\PYG{n}{roa2p}\PYG{p}{(}\PYG{p}{[}\PYG{l+m+mf}{0.6}\PYG{p}{,} \PYG{l+m+mf}{0.5}\PYG{p}{]}\PYG{p}{,} \PYG{p}{[}\PYG{l+m+mf}{0.2}\PYG{p}{,} \PYG{l+m+mf}{0.3}\PYG{p}{]}\PYG{p}{,} \PYG{n}{each\PYGZus{}t}\PYG{o}{=}\PYG{k+kc}{False}\PYG{p}{)}
\end{sphinxVerbatim}

\end{fulllineitems}

\index{psinorm2p() (eqtools.core.Equilibrium method)@\spxentry{psinorm2p()}\spxextra{eqtools.core.Equilibrium method}}

\begin{fulllineitems}
\phantomsection\label{\detokenize{eqtools:eqtools.core.Equilibrium.psinorm2p}}\pysiglinewithargsret{\sphinxbfcode{\sphinxupquote{psinorm2p}}}{\emph{psinorm}, \emph{t}, \emph{**kwargs}}{}
Calculates the pressure corresponding to the passed psi\_norm (normalized poloidal flux) values.

By default, EFIT only computes this inside the LCFS.
\begin{quote}\begin{description}
\item[{Parameters}] \leavevmode\begin{itemize}
\item {} 
\sphinxstyleliteralstrong{\sphinxupquote{psi\_norm}} (\sphinxstyleliteralemphasis{\sphinxupquote{Array-like}}\sphinxstyleliteralemphasis{\sphinxupquote{ or }}\sphinxstyleliteralemphasis{\sphinxupquote{scalar float}}) \textendash{} Values of the normalized
poloidal flux to map to p.

\item {} 
\sphinxstyleliteralstrong{\sphinxupquote{t}} (\sphinxstyleliteralemphasis{\sphinxupquote{Array-like}}\sphinxstyleliteralemphasis{\sphinxupquote{ or }}\sphinxstyleliteralemphasis{\sphinxupquote{scalar float}}) \textendash{} Times to perform the conversion at.
If \sphinxtitleref{t} is a single value, it is used for all of the elements of
\sphinxtitleref{psi\_norm}. If the \sphinxtitleref{each\_t} keyword is True, then \sphinxtitleref{t} must be scalar
or have exactly one dimension. If the \sphinxtitleref{each\_t} keyword is False,
\sphinxtitleref{t} must have the same shape as \sphinxtitleref{psi\_norm}.

\end{itemize}

\item[{Keyword Arguments}] \leavevmode\begin{itemize}
\item {} 
\sphinxstyleliteralstrong{\sphinxupquote{sqrt}} (\sphinxstyleliteralemphasis{\sphinxupquote{Boolean}}) \textendash{} Set to True to return the square root of p. Only
the square root of positive values is taken. Negative values are
replaced with zeros, consistent with Steve Wolfe’s IDL
implementation efit\_rz2rho.pro. Default is False.

\item {} 
\sphinxstyleliteralstrong{\sphinxupquote{each\_t}} (\sphinxstyleliteralemphasis{\sphinxupquote{Boolean}}) \textendash{} When True, the elements in \sphinxtitleref{psi\_norm} are evaluated at
each value in \sphinxtitleref{t}. If True, \sphinxtitleref{t} must have only one dimension (or
be a scalar). If False, \sphinxtitleref{t} must match the shape of \sphinxtitleref{psi\_norm} or be
a scalar. Default is True (evaluate ALL \sphinxtitleref{psi\_norm} at EACH element in
\sphinxtitleref{t}).

\item {} 
\sphinxstyleliteralstrong{\sphinxupquote{k}} (\sphinxstyleliteralemphasis{\sphinxupquote{positive int}}) \textendash{} The degree of polynomial spline interpolation to
use in converting coordinates.

\item {} 
\sphinxstyleliteralstrong{\sphinxupquote{return\_t}} (\sphinxstyleliteralemphasis{\sphinxupquote{Boolean}}) \textendash{} Set to True to return a tuple of (\sphinxtitleref{p},
\sphinxtitleref{time\_idxs}), where \sphinxtitleref{time\_idxs} is the array of time indices
actually used in evaluating \sphinxtitleref{p} with nearest-neighbor
interpolation. (This is mostly present as an internal helper.)
Default is False (only return \sphinxtitleref{p}).

\end{itemize}

\item[{Returns}] \leavevmode

\sphinxtitleref{p} or (\sphinxtitleref{p}, \sphinxtitleref{time\_idxs})
\begin{itemize}
\item {} 
\sphinxstylestrong{p} (\sphinxtitleref{Array or scalar float}) - The pressure. If
all of the input arguments are scalar, then a scalar is returned.
Otherwise, a scipy Array is returned.

\item {} 
\sphinxstylestrong{time\_idxs} (Array with same shape as \sphinxtitleref{p}) - The indices
(in \sphinxcode{\sphinxupquote{self.getTimeBase()}}) that were used for
nearest-neighbor interpolation. Only returned if \sphinxtitleref{return\_t} is
True.

\end{itemize}


\end{description}\end{quote}
\subsubsection*{Examples}

All assume that \sphinxtitleref{Eq\_instance} is a valid instance of the appropriate
extension of the {\hyperref[\detokenize{eqtools:eqtools.core.Equilibrium}]{\sphinxcrossref{\sphinxcode{\sphinxupquote{Equilibrium}}}}} abstract class.

Find single p value for psinorm=0.7, t=0.26s:

\begin{sphinxVerbatim}[commandchars=\\\{\}]
\PYG{n}{p\PYGZus{}val} \PYG{o}{=} \PYG{n}{Eq\PYGZus{}instance}\PYG{o}{.}\PYG{n}{psinorm2p}\PYG{p}{(}\PYG{l+m+mf}{0.7}\PYG{p}{,} \PYG{l+m+mf}{0.26}\PYG{p}{)}
\end{sphinxVerbatim}

Find p values at psi\_norm values of 0.5 and 0.7 at the single time
t=0.26s:

\begin{sphinxVerbatim}[commandchars=\\\{\}]
\PYG{n}{p\PYGZus{}arr} \PYG{o}{=} \PYG{n}{Eq\PYGZus{}instance}\PYG{o}{.}\PYG{n}{psinorm2p}\PYG{p}{(}\PYG{p}{[}\PYG{l+m+mf}{0.5}\PYG{p}{,} \PYG{l+m+mf}{0.7}\PYG{p}{]}\PYG{p}{,} \PYG{l+m+mf}{0.26}\PYG{p}{)}
\end{sphinxVerbatim}

Find p values at psi\_norm=0.5 at times t={[}0.2s, 0.3s{]}:

\begin{sphinxVerbatim}[commandchars=\\\{\}]
\PYG{n}{p\PYGZus{}arr} \PYG{o}{=} \PYG{n}{Eq\PYGZus{}instance}\PYG{o}{.}\PYG{n}{psinorm2p}\PYG{p}{(}\PYG{l+m+mf}{0.5}\PYG{p}{,} \PYG{p}{[}\PYG{l+m+mf}{0.2}\PYG{p}{,} \PYG{l+m+mf}{0.3}\PYG{p}{]}\PYG{p}{)}
\end{sphinxVerbatim}

Find p values at (psinorm, t) points (0.6, 0.2s) and (0.5, 0.3s):

\begin{sphinxVerbatim}[commandchars=\\\{\}]
\PYG{n}{p\PYGZus{}arr} \PYG{o}{=} \PYG{n}{Eq\PYGZus{}instance}\PYG{o}{.}\PYG{n}{psinorm2p}\PYG{p}{(}\PYG{p}{[}\PYG{l+m+mf}{0.6}\PYG{p}{,} \PYG{l+m+mf}{0.5}\PYG{p}{]}\PYG{p}{,} \PYG{p}{[}\PYG{l+m+mf}{0.2}\PYG{p}{,} \PYG{l+m+mf}{0.3}\PYG{p}{]}\PYG{p}{,} \PYG{n}{each\PYGZus{}t}\PYG{o}{=}\PYG{k+kc}{False}\PYG{p}{)}
\end{sphinxVerbatim}

\end{fulllineitems}

\index{phinorm2p() (eqtools.core.Equilibrium method)@\spxentry{phinorm2p()}\spxextra{eqtools.core.Equilibrium method}}

\begin{fulllineitems}
\phantomsection\label{\detokenize{eqtools:eqtools.core.Equilibrium.phinorm2p}}\pysiglinewithargsret{\sphinxbfcode{\sphinxupquote{phinorm2p}}}{\emph{phinorm}, \emph{t}, \emph{**kwargs}}{}
Calculates the pressure corresponding to the passed phinorm (normalized toroidal flux) values.

By default, EFIT only computes this inside the LCFS.
\begin{quote}\begin{description}
\item[{Parameters}] \leavevmode\begin{itemize}
\item {} 
\sphinxstyleliteralstrong{\sphinxupquote{phinorm}} (\sphinxstyleliteralemphasis{\sphinxupquote{Array-like}}\sphinxstyleliteralemphasis{\sphinxupquote{ or }}\sphinxstyleliteralemphasis{\sphinxupquote{scalar float}}) \textendash{} Values of the normalized
toroidal flux to map to p.

\item {} 
\sphinxstyleliteralstrong{\sphinxupquote{t}} (\sphinxstyleliteralemphasis{\sphinxupquote{Array-like}}\sphinxstyleliteralemphasis{\sphinxupquote{ or }}\sphinxstyleliteralemphasis{\sphinxupquote{scalar float}}) \textendash{} Times to perform the conversion at.
If \sphinxtitleref{t} is a single value, it is used for all of the elements of
\sphinxtitleref{phinorm}. If the \sphinxtitleref{each\_t} keyword is True, then \sphinxtitleref{t} must be scalar
or have exactly one dimension. If the \sphinxtitleref{each\_t} keyword is False,
\sphinxtitleref{t} must have the same shape as \sphinxtitleref{phinorm}.

\end{itemize}

\item[{Keyword Arguments}] \leavevmode\begin{itemize}
\item {} 
\sphinxstyleliteralstrong{\sphinxupquote{sqrt}} (\sphinxstyleliteralemphasis{\sphinxupquote{Boolean}}) \textendash{} Set to True to return the square root of p.
Only the square root of positive values is taken. Negative
values are replaced with zeros, consistent with Steve Wolfe’s
IDL implementation efit\_rz2rho.pro. Default is False.

\item {} 
\sphinxstyleliteralstrong{\sphinxupquote{each\_t}} (\sphinxstyleliteralemphasis{\sphinxupquote{Boolean}}) \textendash{} When True, the elements in \sphinxtitleref{phinorm} are evaluated
at each value in \sphinxtitleref{t}. If True, \sphinxtitleref{t} must have only one dimension
(or be a scalar). If False, \sphinxtitleref{t} must match the shape of \sphinxtitleref{phinorm}
or be a scalar. Default is True (evaluate ALL \sphinxtitleref{phinorm} at EACH
element in \sphinxtitleref{t}).

\item {} 
\sphinxstyleliteralstrong{\sphinxupquote{k}} (\sphinxstyleliteralemphasis{\sphinxupquote{positive int}}) \textendash{} The degree of polynomial spline interpolation to
use in converting coordinates.

\item {} 
\sphinxstyleliteralstrong{\sphinxupquote{return\_t}} (\sphinxstyleliteralemphasis{\sphinxupquote{Boolean}}) \textendash{} Set to True to return a tuple of (\sphinxtitleref{p},
\sphinxtitleref{time\_idxs}), where \sphinxtitleref{time\_idxs} is the array of time indices
actually used in evaluating \sphinxtitleref{p} with nearest-neighbor
interpolation. (This is mostly present as an internal helper.)
Default is False (only return \sphinxtitleref{p}).

\end{itemize}

\item[{Returns}] \leavevmode

\sphinxtitleref{p} or (\sphinxtitleref{p}, \sphinxtitleref{time\_idxs})
\begin{itemize}
\item {} 
\sphinxstylestrong{p} (\sphinxtitleref{Array or scalar float}) - The pressure. If
all of the input arguments are scalar, then a scalar is returned.
Otherwise, a scipy Array is returned.

\item {} 
\sphinxstylestrong{time\_idxs} (Array with same shape as \sphinxtitleref{p}) - The indices
(in \sphinxcode{\sphinxupquote{self.getTimeBase()}}) that were used for
nearest-neighbor interpolation. Only returned if \sphinxtitleref{return\_t} is
True.

\end{itemize}


\end{description}\end{quote}
\subsubsection*{Examples}

All assume that \sphinxtitleref{Eq\_instance} is a valid instance of the appropriate
extension of the {\hyperref[\detokenize{eqtools:eqtools.core.Equilibrium}]{\sphinxcrossref{\sphinxcode{\sphinxupquote{Equilibrium}}}}} abstract class.

Find single p value for phinorm=0.7, t=0.26s:

\begin{sphinxVerbatim}[commandchars=\\\{\}]
\PYG{n}{p\PYGZus{}val} \PYG{o}{=} \PYG{n}{Eq\PYGZus{}instance}\PYG{o}{.}\PYG{n}{phinorm2p}\PYG{p}{(}\PYG{l+m+mf}{0.7}\PYG{p}{,} \PYG{l+m+mf}{0.26}\PYG{p}{)}
\end{sphinxVerbatim}

Find p values at phinorm values of 0.5 and 0.7 at the single time
t=0.26s:

\begin{sphinxVerbatim}[commandchars=\\\{\}]
\PYG{n}{p\PYGZus{}arr} \PYG{o}{=} \PYG{n}{Eq\PYGZus{}instance}\PYG{o}{.}\PYG{n}{phinorm2p}\PYG{p}{(}\PYG{p}{[}\PYG{l+m+mf}{0.5}\PYG{p}{,} \PYG{l+m+mf}{0.7}\PYG{p}{]}\PYG{p}{,} \PYG{l+m+mf}{0.26}\PYG{p}{)}
\end{sphinxVerbatim}

Find p values at phinorm=0.5 at times t={[}0.2s, 0.3s{]}:

\begin{sphinxVerbatim}[commandchars=\\\{\}]
\PYG{n}{p\PYGZus{}arr} \PYG{o}{=} \PYG{n}{Eq\PYGZus{}instance}\PYG{o}{.}\PYG{n}{phinorm2p}\PYG{p}{(}\PYG{l+m+mf}{0.5}\PYG{p}{,} \PYG{p}{[}\PYG{l+m+mf}{0.2}\PYG{p}{,} \PYG{l+m+mf}{0.3}\PYG{p}{]}\PYG{p}{)}
\end{sphinxVerbatim}

Find p values at (phinorm, t) points (0.6, 0.2s) and (0.5, 0.3s):

\begin{sphinxVerbatim}[commandchars=\\\{\}]
\PYG{n}{p\PYGZus{}arr} \PYG{o}{=} \PYG{n}{Eq\PYGZus{}instance}\PYG{o}{.}\PYG{n}{phinorm2p}\PYG{p}{(}\PYG{p}{[}\PYG{l+m+mf}{0.6}\PYG{p}{,} \PYG{l+m+mf}{0.5}\PYG{p}{]}\PYG{p}{,} \PYG{p}{[}\PYG{l+m+mf}{0.2}\PYG{p}{,} \PYG{l+m+mf}{0.3}\PYG{p}{]}\PYG{p}{,} \PYG{n}{each\PYGZus{}t}\PYG{o}{=}\PYG{k+kc}{False}\PYG{p}{)}
\end{sphinxVerbatim}

\end{fulllineitems}

\index{volnorm2p() (eqtools.core.Equilibrium method)@\spxentry{volnorm2p()}\spxextra{eqtools.core.Equilibrium method}}

\begin{fulllineitems}
\phantomsection\label{\detokenize{eqtools:eqtools.core.Equilibrium.volnorm2p}}\pysiglinewithargsret{\sphinxbfcode{\sphinxupquote{volnorm2p}}}{\emph{volnorm}, \emph{t}, \emph{**kwargs}}{}
Calculates the pressure corresponding to the passed volnorm (normalized flux surface volume) values.

By default, EFIT only computes this inside the LCFS.
\begin{quote}\begin{description}
\item[{Parameters}] \leavevmode\begin{itemize}
\item {} 
\sphinxstyleliteralstrong{\sphinxupquote{volnorm}} (\sphinxstyleliteralemphasis{\sphinxupquote{Array-like}}\sphinxstyleliteralemphasis{\sphinxupquote{ or }}\sphinxstyleliteralemphasis{\sphinxupquote{scalar float}}) \textendash{} Values of the normalized
flux surface volume to map to p.

\item {} 
\sphinxstyleliteralstrong{\sphinxupquote{t}} (\sphinxstyleliteralemphasis{\sphinxupquote{Array-like}}\sphinxstyleliteralemphasis{\sphinxupquote{ or }}\sphinxstyleliteralemphasis{\sphinxupquote{scalar float}}) \textendash{} Times to perform the conversion at.
If \sphinxtitleref{t} is a single value, it is used for all of the elements of
\sphinxtitleref{volnorm}. If the \sphinxtitleref{each\_t} keyword is True, then \sphinxtitleref{t} must be scalar
or have exactly one dimension. If the \sphinxtitleref{each\_t} keyword is False,
\sphinxtitleref{t} must have the same shape as \sphinxtitleref{volnorm}.

\end{itemize}

\item[{Keyword Arguments}] \leavevmode\begin{itemize}
\item {} 
\sphinxstyleliteralstrong{\sphinxupquote{sqrt}} (\sphinxstyleliteralemphasis{\sphinxupquote{Boolean}}) \textendash{} Set to True to return the square root of p.
Only the square root of positive values is taken. Negative
values are replaced with zeros, consistent with Steve Wolfe’s
IDL implementation efit\_rz2rho.pro. Default is False.

\item {} 
\sphinxstyleliteralstrong{\sphinxupquote{each\_t}} (\sphinxstyleliteralemphasis{\sphinxupquote{Boolean}}) \textendash{} When True, the elements in \sphinxtitleref{volnorm} are evaluated
at each value in \sphinxtitleref{t}. If True, \sphinxtitleref{t} must have only one dimension
(or be a scalar). If False, \sphinxtitleref{t} must match the shape of \sphinxtitleref{volnorm}
or be a scalar. Default is True (evaluate ALL \sphinxtitleref{volnorm} at EACH
element in \sphinxtitleref{t}).

\item {} 
\sphinxstyleliteralstrong{\sphinxupquote{k}} (\sphinxstyleliteralemphasis{\sphinxupquote{positive int}}) \textendash{} The degree of polynomial spline interpolation to
use in converting coordinates.

\item {} 
\sphinxstyleliteralstrong{\sphinxupquote{return\_t}} (\sphinxstyleliteralemphasis{\sphinxupquote{Boolean}}) \textendash{} Set to True to return a tuple of (\sphinxtitleref{p},
\sphinxtitleref{time\_idxs}), where \sphinxtitleref{time\_idxs} is the array of time indices
actually used in evaluating \sphinxtitleref{p} with nearest-neighbor
interpolation. (This is mostly present as an internal helper.)
Default is False (only return \sphinxtitleref{p}).

\end{itemize}

\item[{Returns}] \leavevmode

\sphinxtitleref{p} or (\sphinxtitleref{p}, \sphinxtitleref{time\_idxs})
\begin{itemize}
\item {} 
\sphinxstylestrong{p} (\sphinxtitleref{Array or scalar float}) - The pressure. If
all of the input arguments are scalar, then a scalar is returned.
Otherwise, a scipy Array is returned.

\item {} 
\sphinxstylestrong{time\_idxs} (Array with same shape as \sphinxtitleref{p}) - The indices
(in \sphinxcode{\sphinxupquote{self.getTimeBase()}}) that were used for
nearest-neighbor interpolation. Only returned if \sphinxtitleref{return\_t} is
True.

\end{itemize}


\end{description}\end{quote}
\subsubsection*{Examples}

All assume that \sphinxtitleref{Eq\_instance} is a valid instance of the appropriate
extension of the {\hyperref[\detokenize{eqtools:eqtools.core.Equilibrium}]{\sphinxcrossref{\sphinxcode{\sphinxupquote{Equilibrium}}}}} abstract class.

Find single p value for volnorm=0.7, t=0.26s:

\begin{sphinxVerbatim}[commandchars=\\\{\}]
\PYG{n}{p\PYGZus{}val} \PYG{o}{=} \PYG{n}{Eq\PYGZus{}instance}\PYG{o}{.}\PYG{n}{volnorm2p}\PYG{p}{(}\PYG{l+m+mf}{0.7}\PYG{p}{,} \PYG{l+m+mf}{0.26}\PYG{p}{)}
\end{sphinxVerbatim}

Find p values at volnorm values of 0.5 and 0.7 at the single time
t=0.26s:

\begin{sphinxVerbatim}[commandchars=\\\{\}]
\PYG{n}{p\PYGZus{}arr} \PYG{o}{=} \PYG{n}{Eq\PYGZus{}instance}\PYG{o}{.}\PYG{n}{volnorm2p}\PYG{p}{(}\PYG{p}{[}\PYG{l+m+mf}{0.5}\PYG{p}{,} \PYG{l+m+mf}{0.7}\PYG{p}{]}\PYG{p}{,} \PYG{l+m+mf}{0.26}\PYG{p}{)}
\end{sphinxVerbatim}

Find p values at volnorm=0.5 at times t={[}0.2s, 0.3s{]}:

\begin{sphinxVerbatim}[commandchars=\\\{\}]
\PYG{n}{p\PYGZus{}arr} \PYG{o}{=} \PYG{n}{Eq\PYGZus{}instance}\PYG{o}{.}\PYG{n}{volnorm2p}\PYG{p}{(}\PYG{l+m+mf}{0.5}\PYG{p}{,} \PYG{p}{[}\PYG{l+m+mf}{0.2}\PYG{p}{,} \PYG{l+m+mf}{0.3}\PYG{p}{]}\PYG{p}{)}
\end{sphinxVerbatim}

Find p values at (volnorm, t) points (0.6, 0.2s) and (0.5, 0.3s):

\begin{sphinxVerbatim}[commandchars=\\\{\}]
\PYG{n}{p\PYGZus{}arr} \PYG{o}{=} \PYG{n}{Eq\PYGZus{}instance}\PYG{o}{.}\PYG{n}{volnorm2p}\PYG{p}{(}\PYG{p}{[}\PYG{l+m+mf}{0.6}\PYG{p}{,} \PYG{l+m+mf}{0.5}\PYG{p}{]}\PYG{p}{,} \PYG{p}{[}\PYG{l+m+mf}{0.2}\PYG{p}{,} \PYG{l+m+mf}{0.3}\PYG{p}{]}\PYG{p}{,} \PYG{n}{each\PYGZus{}t}\PYG{o}{=}\PYG{k+kc}{False}\PYG{p}{)}
\end{sphinxVerbatim}

\end{fulllineitems}

\index{rz2pprime() (eqtools.core.Equilibrium method)@\spxentry{rz2pprime()}\spxextra{eqtools.core.Equilibrium method}}

\begin{fulllineitems}
\phantomsection\label{\detokenize{eqtools:eqtools.core.Equilibrium.rz2pprime}}\pysiglinewithargsret{\sphinxbfcode{\sphinxupquote{rz2pprime}}}{\emph{R}, \emph{Z}, \emph{t}, \emph{**kwargs}}{}
Calculates the pressure gradient at the given (R, Z, t).

By default, EFIT only computes this inside the LCFS.
\begin{quote}\begin{description}
\item[{Parameters}] \leavevmode\begin{itemize}
\item {} 
\sphinxstyleliteralstrong{\sphinxupquote{R}} (\sphinxstyleliteralemphasis{\sphinxupquote{Array-like}}\sphinxstyleliteralemphasis{\sphinxupquote{ or }}\sphinxstyleliteralemphasis{\sphinxupquote{scalar float}}) \textendash{} Values of the radial coordinate to
map to pprime. If \sphinxtitleref{R} and \sphinxtitleref{Z} are both scalar values,
they are used as the coordinate pair for all of the values in
\sphinxtitleref{t}. Must have the same shape as \sphinxtitleref{Z} unless the \sphinxtitleref{make\_grid}
keyword is set. If the \sphinxtitleref{make\_grid} keyword is True, \sphinxtitleref{R} must
have exactly one dimension.

\item {} 
\sphinxstyleliteralstrong{\sphinxupquote{Z}} (\sphinxstyleliteralemphasis{\sphinxupquote{Array-like}}\sphinxstyleliteralemphasis{\sphinxupquote{ or }}\sphinxstyleliteralemphasis{\sphinxupquote{scalar float}}) \textendash{} Values of the vertical coordinate to
map to pprime. If \sphinxtitleref{R} and \sphinxtitleref{Z} are both scalar values,
they are used as the coordinate pair for all of the values in
\sphinxtitleref{t}. Must have the same shape as \sphinxtitleref{R} unless the \sphinxtitleref{make\_grid}
keyword is set. If the \sphinxtitleref{make\_grid} keyword is True, \sphinxtitleref{Z} must
have exactly one dimension.

\item {} 
\sphinxstyleliteralstrong{\sphinxupquote{t}} (\sphinxstyleliteralemphasis{\sphinxupquote{Array-like}}\sphinxstyleliteralemphasis{\sphinxupquote{ or }}\sphinxstyleliteralemphasis{\sphinxupquote{scalar float}}) \textendash{} Times to perform the conversion at.
If \sphinxtitleref{t} is a single value, it is used for all of the elements of
\sphinxtitleref{R}, \sphinxtitleref{Z}. If the \sphinxtitleref{each\_t} keyword is True, then \sphinxtitleref{t} must be
scalar or have exactly one dimension. If the \sphinxtitleref{each\_t} keyword is
False, \sphinxtitleref{t} must have the same shape as \sphinxtitleref{R} and \sphinxtitleref{Z} (or their
meshgrid if \sphinxtitleref{make\_grid} is True).

\end{itemize}

\item[{Keyword Arguments}] \leavevmode\begin{itemize}
\item {} 
\sphinxstyleliteralstrong{\sphinxupquote{sqrt}} (\sphinxstyleliteralemphasis{\sphinxupquote{Boolean}}) \textendash{} Set to True to return the square root of pprime.
Only the square root of positive values is taken. Negative
values are replaced with zeros, consistent with Steve Wolfe’s
IDL implementation efit\_rz2rho.pro. Default is False.

\item {} 
\sphinxstyleliteralstrong{\sphinxupquote{each\_t}} (\sphinxstyleliteralemphasis{\sphinxupquote{Boolean}}) \textendash{} When True, the elements in \sphinxtitleref{R}, \sphinxtitleref{Z} are evaluated
at each value in \sphinxtitleref{t}. If True, \sphinxtitleref{t} must have only one dimension
(or be a scalar). If False, \sphinxtitleref{t} must match the shape of \sphinxtitleref{R} and
\sphinxtitleref{Z} or be a scalar. Default is True (evaluate ALL \sphinxtitleref{R}, \sphinxtitleref{Z} at
EACH element in \sphinxtitleref{t}).

\item {} 
\sphinxstyleliteralstrong{\sphinxupquote{make\_grid}} (\sphinxstyleliteralemphasis{\sphinxupquote{Boolean}}) \textendash{} Set to True to pass \sphinxtitleref{R} and \sphinxtitleref{Z} through
\sphinxcode{\sphinxupquote{scipy.meshgrid()}} before evaluating. If this is set to
True, \sphinxtitleref{R} and \sphinxtitleref{Z} must each only have a single dimension, but
can have different lengths. Default is False (do not form
meshgrid).

\item {} 
\sphinxstyleliteralstrong{\sphinxupquote{length\_unit}} (\sphinxstyleliteralemphasis{\sphinxupquote{String}}\sphinxstyleliteralemphasis{\sphinxupquote{ or }}\sphinxstyleliteralemphasis{\sphinxupquote{1}}) \textendash{} 
Length unit that \sphinxtitleref{R}, \sphinxtitleref{Z} are given in.
If a string is given, it must be a valid unit specifier:
\begin{quote}


\begin{savenotes}\sphinxattablestart
\centering
\begin{tabulary}{\linewidth}[t]{|T|T|}
\hline

’m’
&
meters
\\
\hline
’cm’
&
centimeters
\\
\hline
’mm’
&
millimeters
\\
\hline
’in’
&
inches
\\
\hline
’ft’
&
feet
\\
\hline
’yd’
&
yards
\\
\hline
’smoot’
&
smoots
\\
\hline
’cubit’
&
cubits
\\
\hline
’hand’
&
hands
\\
\hline
’default’
&
meters
\\
\hline
\end{tabulary}
\par
\sphinxattableend\end{savenotes}
\end{quote}

If length\_unit is 1 or None, meters are assumed. The default
value is 1 (use meters).


\item {} 
\sphinxstyleliteralstrong{\sphinxupquote{return\_t}} (\sphinxstyleliteralemphasis{\sphinxupquote{Boolean}}) \textendash{} Set to True to return a tuple of (\sphinxtitleref{pprime},
\sphinxtitleref{time\_idxs}), where \sphinxtitleref{time\_idxs} is the array of time indices
actually used in evaluating \sphinxtitleref{pprime} with nearest-neighbor
interpolation. (This is mostly present as an internal helper.)
Default is False (only return \sphinxtitleref{pprime}).

\end{itemize}

\item[{Returns}] \leavevmode

\sphinxtitleref{pprime} or (\sphinxtitleref{pprime}, \sphinxtitleref{time\_idxs})
\begin{itemize}
\item {} 
\sphinxstylestrong{pprime} (\sphinxtitleref{Array or scalar float}) - The pressure gradient. If
all of the input arguments are scalar, then a scalar is
returned. Otherwise, a scipy Array is returned. If \sphinxtitleref{R} and \sphinxtitleref{Z}
both have the same shape then \sphinxtitleref{p} has this shape as well,
unless the \sphinxtitleref{make\_grid} keyword was True, in which case \sphinxtitleref{p}
has shape (len(\sphinxtitleref{Z}), len(\sphinxtitleref{R})).

\item {} 
\sphinxstylestrong{time\_idxs} (Array with same shape as \sphinxtitleref{pprime}) - The indices
(in \sphinxcode{\sphinxupquote{self.getTimeBase()}}) that were used for
nearest-neighbor interpolation. Only returned if \sphinxtitleref{return\_t} is
True.

\end{itemize}


\end{description}\end{quote}
\subsubsection*{Examples}

All assume that \sphinxtitleref{Eq\_instance} is a valid instance of the
appropriate extension of the {\hyperref[\detokenize{eqtools:eqtools.core.Equilibrium}]{\sphinxcrossref{\sphinxcode{\sphinxupquote{Equilibrium}}}}} abstract class.

Find single pprime value at R=0.6m, Z=0.0m, t=0.26s:

\begin{sphinxVerbatim}[commandchars=\\\{\}]
\PYG{n}{pprime\PYGZus{}val} \PYG{o}{=} \PYG{n}{Eq\PYGZus{}instance}\PYG{o}{.}\PYG{n}{rz2pprime}\PYG{p}{(}\PYG{l+m+mf}{0.6}\PYG{p}{,} \PYG{l+m+mi}{0}\PYG{p}{,} \PYG{l+m+mf}{0.26}\PYG{p}{)}
\end{sphinxVerbatim}

Find pprime values at (R, Z) points (0.6m, 0m) and (0.8m, 0m) at the
single time t=0.26s. Note that the \sphinxtitleref{Z} vector must be fully specified,
even if the values are all the same:

\begin{sphinxVerbatim}[commandchars=\\\{\}]
\PYG{n}{pprime\PYGZus{}arr} \PYG{o}{=} \PYG{n}{Eq\PYGZus{}instance}\PYG{o}{.}\PYG{n}{rz2pprime}\PYG{p}{(}\PYG{p}{[}\PYG{l+m+mf}{0.6}\PYG{p}{,} \PYG{l+m+mf}{0.8}\PYG{p}{]}\PYG{p}{,} \PYG{p}{[}\PYG{l+m+mi}{0}\PYG{p}{,} \PYG{l+m+mi}{0}\PYG{p}{]}\PYG{p}{,} \PYG{l+m+mf}{0.26}\PYG{p}{)}
\end{sphinxVerbatim}

Find pprime values at (R, Z) points (0.6m, 0m) at times t={[}0.2s, 0.3s{]}:

\begin{sphinxVerbatim}[commandchars=\\\{\}]
\PYG{n}{pprime\PYGZus{}arr} \PYG{o}{=} \PYG{n}{Eq\PYGZus{}instance}\PYG{o}{.}\PYG{n}{rz2pprime}\PYG{p}{(}\PYG{l+m+mf}{0.6}\PYG{p}{,} \PYG{l+m+mi}{0}\PYG{p}{,} \PYG{p}{[}\PYG{l+m+mf}{0.2}\PYG{p}{,} \PYG{l+m+mf}{0.3}\PYG{p}{]}\PYG{p}{)}
\end{sphinxVerbatim}

Find pprime values at (R, Z, t) points (0.6m, 0m, 0.2s) and (0.5m, 0.2m, 0.3s):

\begin{sphinxVerbatim}[commandchars=\\\{\}]
\PYG{n}{pprime\PYGZus{}arr} \PYG{o}{=} \PYG{n}{Eq\PYGZus{}instance}\PYG{o}{.}\PYG{n}{rz2pprime}\PYG{p}{(}\PYG{p}{[}\PYG{l+m+mf}{0.6}\PYG{p}{,} \PYG{l+m+mf}{0.5}\PYG{p}{]}\PYG{p}{,} \PYG{p}{[}\PYG{l+m+mi}{0}\PYG{p}{,} \PYG{l+m+mf}{0.2}\PYG{p}{]}\PYG{p}{,} \PYG{p}{[}\PYG{l+m+mf}{0.2}\PYG{p}{,} \PYG{l+m+mf}{0.3}\PYG{p}{]}\PYG{p}{,} \PYG{n}{each\PYGZus{}t}\PYG{o}{=}\PYG{k+kc}{False}\PYG{p}{)}
\end{sphinxVerbatim}

Find pprime values on grid defined by 1D vector of radial positions \sphinxtitleref{R}
and 1D vector of vertical positions \sphinxtitleref{Z} at time t=0.2s:

\begin{sphinxVerbatim}[commandchars=\\\{\}]
\PYG{n}{pprime\PYGZus{}mat} \PYG{o}{=} \PYG{n}{Eq\PYGZus{}instance}\PYG{o}{.}\PYG{n}{rz2pprime}\PYG{p}{(}\PYG{n}{R}\PYG{p}{,} \PYG{n}{Z}\PYG{p}{,} \PYG{l+m+mf}{0.2}\PYG{p}{,} \PYG{n}{make\PYGZus{}grid}\PYG{o}{=}\PYG{k+kc}{True}\PYG{p}{)}
\end{sphinxVerbatim}

\end{fulllineitems}

\index{rmid2pprime() (eqtools.core.Equilibrium method)@\spxentry{rmid2pprime()}\spxextra{eqtools.core.Equilibrium method}}

\begin{fulllineitems}
\phantomsection\label{\detokenize{eqtools:eqtools.core.Equilibrium.rmid2pprime}}\pysiglinewithargsret{\sphinxbfcode{\sphinxupquote{rmid2pprime}}}{\emph{R\_mid}, \emph{t}, \emph{**kwargs}}{}
Calculates the pressure gradient corresponding to the passed R\_mid (mapped outboard midplane major radius) values.

By default, EFIT only computes this inside the LCFS.
\begin{quote}\begin{description}
\item[{Parameters}] \leavevmode\begin{itemize}
\item {} 
\sphinxstyleliteralstrong{\sphinxupquote{R\_mid}} (\sphinxstyleliteralemphasis{\sphinxupquote{Array-like}}\sphinxstyleliteralemphasis{\sphinxupquote{ or }}\sphinxstyleliteralemphasis{\sphinxupquote{scalar float}}) \textendash{} Values of the outboard midplane
major radius to map to pprime.

\item {} 
\sphinxstyleliteralstrong{\sphinxupquote{t}} (\sphinxstyleliteralemphasis{\sphinxupquote{Array-like}}\sphinxstyleliteralemphasis{\sphinxupquote{ or }}\sphinxstyleliteralemphasis{\sphinxupquote{scalar float}}) \textendash{} Times to perform the conversion at.
If \sphinxtitleref{t} is a single value, it is used for all of the elements of
\sphinxtitleref{R\_mid}. If the \sphinxtitleref{each\_t} keyword is True, then \sphinxtitleref{t} must be scalar
or have exactly one dimension. If the \sphinxtitleref{each\_t} keyword is False,
\sphinxtitleref{t} must have the same shape as \sphinxtitleref{R\_mid}.

\end{itemize}

\item[{Keyword Arguments}] \leavevmode\begin{itemize}
\item {} 
\sphinxstyleliteralstrong{\sphinxupquote{sqrt}} (\sphinxstyleliteralemphasis{\sphinxupquote{Boolean}}) \textendash{} Set to True to return the square root of pprime.
Only the square root of positive values is taken. Negative
values are replaced with zeros, consistent with Steve Wolfe’s
IDL implementation efit\_rz2rho.pro. Default is False.

\item {} 
\sphinxstyleliteralstrong{\sphinxupquote{each\_t}} (\sphinxstyleliteralemphasis{\sphinxupquote{Boolean}}) \textendash{} When True, the elements in \sphinxtitleref{R\_mid} are evaluated
at each value in \sphinxtitleref{t}. If True, \sphinxtitleref{t} must have only one dimension
(or be a scalar). If False, \sphinxtitleref{t} must match the shape of \sphinxtitleref{R\_mid}
or be a scalar. Default is True (evaluate ALL \sphinxtitleref{R\_mid} at EACH
element in \sphinxtitleref{t}).

\item {} 
\sphinxstyleliteralstrong{\sphinxupquote{length\_unit}} (\sphinxstyleliteralemphasis{\sphinxupquote{String}}\sphinxstyleliteralemphasis{\sphinxupquote{ or }}\sphinxstyleliteralemphasis{\sphinxupquote{1}}) \textendash{} 
Length unit that \sphinxtitleref{R\_mid} is given in.
If a string is given, it must be a valid unit specifier:
\begin{quote}


\begin{savenotes}\sphinxattablestart
\centering
\begin{tabulary}{\linewidth}[t]{|T|T|}
\hline

’m’
&
meters
\\
\hline
’cm’
&
centimeters
\\
\hline
’mm’
&
millimeters
\\
\hline
’in’
&
inches
\\
\hline
’ft’
&
feet
\\
\hline
’yd’
&
yards
\\
\hline
’smoot’
&
smoots
\\
\hline
’cubit’
&
cubits
\\
\hline
’hand’
&
hands
\\
\hline
’default’
&
meters
\\
\hline
\end{tabulary}
\par
\sphinxattableend\end{savenotes}
\end{quote}

If length\_unit is 1 or None, meters are assumed. The default
value is 1 (use meters).


\item {} 
\sphinxstyleliteralstrong{\sphinxupquote{k}} (\sphinxstyleliteralemphasis{\sphinxupquote{positive int}}) \textendash{} The degree of polynomial spline interpolation to
use in converting coordinates.

\item {} 
\sphinxstyleliteralstrong{\sphinxupquote{return\_t}} (\sphinxstyleliteralemphasis{\sphinxupquote{Boolean}}) \textendash{} Set to True to return a tuple of (\sphinxtitleref{pprime},
\sphinxtitleref{time\_idxs}), where \sphinxtitleref{time\_idxs} is the array of time indices
actually used in evaluating \sphinxtitleref{pprime} with nearest-neighbor
interpolation. (This is mostly present as an internal helper.)
Default is False (only return \sphinxtitleref{pprime}).

\end{itemize}

\item[{Returns}] \leavevmode

\sphinxtitleref{pprime} or (\sphinxtitleref{pprime}, \sphinxtitleref{time\_idxs})
\begin{itemize}
\item {} 
\sphinxstylestrong{pprime} (\sphinxtitleref{Array or scalar float}) - The pressure gradient.
If all of the input arguments are scalar, then a scalar is
returned. Otherwise, a scipy Array is returned.

\item {} 
\sphinxstylestrong{time\_idxs} (Array with same shape as \sphinxtitleref{pprime}) - The indices
(in \sphinxcode{\sphinxupquote{self.getTimeBase()}}) that were used for
nearest-neighbor interpolation. Only returned if \sphinxtitleref{return\_t} is
True.

\end{itemize}


\end{description}\end{quote}
\subsubsection*{Examples}

All assume that \sphinxtitleref{Eq\_instance} is a valid instance of the appropriate
extension of the {\hyperref[\detokenize{eqtools:eqtools.core.Equilibrium}]{\sphinxcrossref{\sphinxcode{\sphinxupquote{Equilibrium}}}}} abstract class.

Find single pprime value for Rmid=0.7m, t=0.26s:

\begin{sphinxVerbatim}[commandchars=\\\{\}]
\PYG{n}{pprime\PYGZus{}val} \PYG{o}{=} \PYG{n}{Eq\PYGZus{}instance}\PYG{o}{.}\PYG{n}{rmid2pprime}\PYG{p}{(}\PYG{l+m+mf}{0.7}\PYG{p}{,} \PYG{l+m+mf}{0.26}\PYG{p}{)}
\end{sphinxVerbatim}

Find pprime values at R\_mid values of 0.5m and 0.7m at the single time
t=0.26s:

\begin{sphinxVerbatim}[commandchars=\\\{\}]
\PYG{n}{pprime\PYGZus{}arr} \PYG{o}{=} \PYG{n}{Eq\PYGZus{}instance}\PYG{o}{.}\PYG{n}{rmid2pprime}\PYG{p}{(}\PYG{p}{[}\PYG{l+m+mf}{0.5}\PYG{p}{,} \PYG{l+m+mf}{0.7}\PYG{p}{]}\PYG{p}{,} \PYG{l+m+mf}{0.26}\PYG{p}{)}
\end{sphinxVerbatim}

Find pprime values at R\_mid=0.5m at times t={[}0.2s, 0.3s{]}:

\begin{sphinxVerbatim}[commandchars=\\\{\}]
\PYG{n}{pprime\PYGZus{}arr} \PYG{o}{=} \PYG{n}{Eq\PYGZus{}instance}\PYG{o}{.}\PYG{n}{rmid2pprime}\PYG{p}{(}\PYG{l+m+mf}{0.5}\PYG{p}{,} \PYG{p}{[}\PYG{l+m+mf}{0.2}\PYG{p}{,} \PYG{l+m+mf}{0.3}\PYG{p}{]}\PYG{p}{)}
\end{sphinxVerbatim}

Find pprime values at (R\_mid, t) points (0.6m, 0.2s) and (0.5m, 0.3s):

\begin{sphinxVerbatim}[commandchars=\\\{\}]
\PYG{n}{pprime\PYGZus{}arr} \PYG{o}{=} \PYG{n}{Eq\PYGZus{}instance}\PYG{o}{.}\PYG{n}{rmid2pprime}\PYG{p}{(}\PYG{p}{[}\PYG{l+m+mf}{0.6}\PYG{p}{,} \PYG{l+m+mf}{0.5}\PYG{p}{]}\PYG{p}{,} \PYG{p}{[}\PYG{l+m+mf}{0.2}\PYG{p}{,} \PYG{l+m+mf}{0.3}\PYG{p}{]}\PYG{p}{,} \PYG{n}{each\PYGZus{}t}\PYG{o}{=}\PYG{k+kc}{False}\PYG{p}{)}
\end{sphinxVerbatim}

\end{fulllineitems}

\index{roa2pprime() (eqtools.core.Equilibrium method)@\spxentry{roa2pprime()}\spxextra{eqtools.core.Equilibrium method}}

\begin{fulllineitems}
\phantomsection\label{\detokenize{eqtools:eqtools.core.Equilibrium.roa2pprime}}\pysiglinewithargsret{\sphinxbfcode{\sphinxupquote{roa2pprime}}}{\emph{roa}, \emph{t}, \emph{**kwargs}}{}
Convert the passed (r/a, t) coordinates into pressure gradient.

By default, EFIT only computes this inside the LCFS.
\begin{quote}\begin{description}
\item[{Parameters}] \leavevmode\begin{itemize}
\item {} 
\sphinxstyleliteralstrong{\sphinxupquote{roa}} (\sphinxstyleliteralemphasis{\sphinxupquote{Array-like}}\sphinxstyleliteralemphasis{\sphinxupquote{ or }}\sphinxstyleliteralemphasis{\sphinxupquote{scalar float}}) \textendash{} Values of the normalized minor
radius to map to pprime.

\item {} 
\sphinxstyleliteralstrong{\sphinxupquote{t}} (\sphinxstyleliteralemphasis{\sphinxupquote{Array-like}}\sphinxstyleliteralemphasis{\sphinxupquote{ or }}\sphinxstyleliteralemphasis{\sphinxupquote{scalar float}}) \textendash{} Times to perform the conversion at.
If \sphinxtitleref{t} is a single value, it is used for all of the elements of
\sphinxtitleref{roa}. If the \sphinxtitleref{each\_t} keyword is True, then \sphinxtitleref{t} must be scalar
or have exactly one dimension. If the \sphinxtitleref{each\_t} keyword is False,
\sphinxtitleref{t} must have the same shape as \sphinxtitleref{roa}.

\end{itemize}

\item[{Keyword Arguments}] \leavevmode\begin{itemize}
\item {} 
\sphinxstyleliteralstrong{\sphinxupquote{sqrt}} (\sphinxstyleliteralemphasis{\sphinxupquote{Boolean}}) \textendash{} Set to True to return the square root of pprime.
Only the square root of positive values is taken. Negative
values are replaced with zeros, consistent with Steve Wolfe’s
IDL implementation efit\_rz2rho.pro. Default is False.

\item {} 
\sphinxstyleliteralstrong{\sphinxupquote{each\_t}} (\sphinxstyleliteralemphasis{\sphinxupquote{Boolean}}) \textendash{} When True, the elements in \sphinxtitleref{roa} are evaluated
at each value in \sphinxtitleref{t}. If True, \sphinxtitleref{t} must have only one dimension
(or be a scalar). If False, \sphinxtitleref{t} must match the shape of \sphinxtitleref{roa}
or be a scalar. Default is True (evaluate ALL \sphinxtitleref{roa} at EACH
element in \sphinxtitleref{t}).

\item {} 
\sphinxstyleliteralstrong{\sphinxupquote{k}} (\sphinxstyleliteralemphasis{\sphinxupquote{positive int}}) \textendash{} The degree of polynomial spline interpolation to
use in converting coordinates.

\item {} 
\sphinxstyleliteralstrong{\sphinxupquote{return\_t}} (\sphinxstyleliteralemphasis{\sphinxupquote{Boolean}}) \textendash{} Set to True to return a tuple of (\sphinxtitleref{pprime},
\sphinxtitleref{time\_idxs}), where \sphinxtitleref{time\_idxs} is the array of time indices
actually used in evaluating \sphinxtitleref{pprime} with nearest-neighbor
interpolation. (This is mostly present as an internal helper.)
Default is False (only return \sphinxtitleref{pprime}).

\end{itemize}

\item[{Returns}] \leavevmode

\sphinxtitleref{pprime} or (\sphinxtitleref{pprime}, \sphinxtitleref{time\_idxs})
\begin{itemize}
\item {} 
\sphinxstylestrong{pprime} (\sphinxtitleref{Array or scalar float}) - The pressure gradient. If
all of the input arguments are scalar, then a scalar is returned.
Otherwise, a scipy Array is returned.

\item {} 
\sphinxstylestrong{time\_idxs} (Array with same shape as \sphinxtitleref{pprime}) - The indices
(in \sphinxcode{\sphinxupquote{self.getTimeBase()}}) that were used for
nearest-neighbor interpolation. Only returned if \sphinxtitleref{return\_t} is
True.

\end{itemize}


\end{description}\end{quote}
\subsubsection*{Examples}

All assume that \sphinxtitleref{Eq\_instance} is a valid instance of the appropriate
extension of the {\hyperref[\detokenize{eqtools:eqtools.core.Equilibrium}]{\sphinxcrossref{\sphinxcode{\sphinxupquote{Equilibrium}}}}} abstract class.

Find single pprime value at r/a=0.6, t=0.26s:

\begin{sphinxVerbatim}[commandchars=\\\{\}]
\PYG{n}{pprime\PYGZus{}val} \PYG{o}{=} \PYG{n}{Eq\PYGZus{}instance}\PYG{o}{.}\PYG{n}{roa2pprime}\PYG{p}{(}\PYG{l+m+mf}{0.6}\PYG{p}{,} \PYG{l+m+mf}{0.26}\PYG{p}{)}
\end{sphinxVerbatim}

Find pprime values at r/a points 0.6 and 0.8 at the
single time t=0.26s.:

\begin{sphinxVerbatim}[commandchars=\\\{\}]
\PYG{n}{pprime\PYGZus{}arr} \PYG{o}{=} \PYG{n}{Eq\PYGZus{}instance}\PYG{o}{.}\PYG{n}{roa2pprime}\PYG{p}{(}\PYG{p}{[}\PYG{l+m+mf}{0.6}\PYG{p}{,} \PYG{l+m+mf}{0.8}\PYG{p}{]}\PYG{p}{,} \PYG{l+m+mf}{0.26}\PYG{p}{)}
\end{sphinxVerbatim}

Find pprime values at r/a of 0.6 at times t={[}0.2s, 0.3s{]}:

\begin{sphinxVerbatim}[commandchars=\\\{\}]
\PYG{n}{pprime\PYGZus{}arr} \PYG{o}{=} \PYG{n}{Eq\PYGZus{}instance}\PYG{o}{.}\PYG{n}{roa2pprime}\PYG{p}{(}\PYG{l+m+mf}{0.6}\PYG{p}{,} \PYG{p}{[}\PYG{l+m+mf}{0.2}\PYG{p}{,} \PYG{l+m+mf}{0.3}\PYG{p}{]}\PYG{p}{)}
\end{sphinxVerbatim}

Find pprime values at (roa, t) points (0.6, 0.2s) and (0.5, 0.3s):

\begin{sphinxVerbatim}[commandchars=\\\{\}]
\PYG{n}{pprime\PYGZus{}arr} \PYG{o}{=} \PYG{n}{Eq\PYGZus{}instance}\PYG{o}{.}\PYG{n}{roa2pprime}\PYG{p}{(}\PYG{p}{[}\PYG{l+m+mf}{0.6}\PYG{p}{,} \PYG{l+m+mf}{0.5}\PYG{p}{]}\PYG{p}{,} \PYG{p}{[}\PYG{l+m+mf}{0.2}\PYG{p}{,} \PYG{l+m+mf}{0.3}\PYG{p}{]}\PYG{p}{,} \PYG{n}{each\PYGZus{}t}\PYG{o}{=}\PYG{k+kc}{False}\PYG{p}{)}
\end{sphinxVerbatim}

\end{fulllineitems}

\index{psinorm2pprime() (eqtools.core.Equilibrium method)@\spxentry{psinorm2pprime()}\spxextra{eqtools.core.Equilibrium method}}

\begin{fulllineitems}
\phantomsection\label{\detokenize{eqtools:eqtools.core.Equilibrium.psinorm2pprime}}\pysiglinewithargsret{\sphinxbfcode{\sphinxupquote{psinorm2pprime}}}{\emph{psinorm}, \emph{t}, \emph{**kwargs}}{}
Calculates the pressure gradient corresponding to the passed psi\_norm (normalized poloidal flux) values.

By default, EFIT only computes this inside the LCFS.
\begin{quote}\begin{description}
\item[{Parameters}] \leavevmode\begin{itemize}
\item {} 
\sphinxstyleliteralstrong{\sphinxupquote{psi\_norm}} (\sphinxstyleliteralemphasis{\sphinxupquote{Array-like}}\sphinxstyleliteralemphasis{\sphinxupquote{ or }}\sphinxstyleliteralemphasis{\sphinxupquote{scalar float}}) \textendash{} Values of the normalized
poloidal flux to map to pprime.

\item {} 
\sphinxstyleliteralstrong{\sphinxupquote{t}} (\sphinxstyleliteralemphasis{\sphinxupquote{Array-like}}\sphinxstyleliteralemphasis{\sphinxupquote{ or }}\sphinxstyleliteralemphasis{\sphinxupquote{scalar float}}) \textendash{} Times to perform the conversion at.
If \sphinxtitleref{t} is a single value, it is used for all of the elements of
\sphinxtitleref{psi\_norm}. If the \sphinxtitleref{each\_t} keyword is True, then \sphinxtitleref{t} must be scalar
or have exactly one dimension. If the \sphinxtitleref{each\_t} keyword is False,
\sphinxtitleref{t} must have the same shape as \sphinxtitleref{psi\_norm}.

\end{itemize}

\item[{Keyword Arguments}] \leavevmode\begin{itemize}
\item {} 
\sphinxstyleliteralstrong{\sphinxupquote{sqrt}} (\sphinxstyleliteralemphasis{\sphinxupquote{Boolean}}) \textendash{} Set to True to return the square root of pprime. Only
the square root of positive values is taken. Negative values are
replaced with zeros, consistent with Steve Wolfe’s IDL
implementation efit\_rz2rho.pro. Default is False.

\item {} 
\sphinxstyleliteralstrong{\sphinxupquote{each\_t}} (\sphinxstyleliteralemphasis{\sphinxupquote{Boolean}}) \textendash{} When True, the elements in \sphinxtitleref{psi\_norm} are evaluated at
each value in \sphinxtitleref{t}. If True, \sphinxtitleref{t} must have only one dimension (or
be a scalar). If False, \sphinxtitleref{t} must match the shape of \sphinxtitleref{psi\_norm} or be
a scalar. Default is True (evaluate ALL \sphinxtitleref{psi\_norm} at EACH element in
\sphinxtitleref{t}).

\item {} 
\sphinxstyleliteralstrong{\sphinxupquote{k}} (\sphinxstyleliteralemphasis{\sphinxupquote{positive int}}) \textendash{} The degree of polynomial spline interpolation to
use in converting coordinates.

\item {} 
\sphinxstyleliteralstrong{\sphinxupquote{return\_t}} (\sphinxstyleliteralemphasis{\sphinxupquote{Boolean}}) \textendash{} Set to True to return a tuple of (\sphinxtitleref{pprime},
\sphinxtitleref{time\_idxs}), where \sphinxtitleref{time\_idxs} is the array of time indices
actually used in evaluating \sphinxtitleref{pprime} with nearest-neighbor
interpolation. (This is mostly present as an internal helper.)
Default is False (only return \sphinxtitleref{pprime}).

\end{itemize}

\item[{Returns}] \leavevmode

\sphinxtitleref{pprime} or (\sphinxtitleref{pprime}, \sphinxtitleref{time\_idxs})
\begin{itemize}
\item {} 
\sphinxstylestrong{pprime} (\sphinxtitleref{Array or scalar float}) - The pressure gradient. If
all of the input arguments are scalar, then a scalar is returned.
Otherwise, a scipy Array is returned.

\item {} 
\sphinxstylestrong{time\_idxs} (Array with same shape as \sphinxtitleref{pprime}) - The indices
(in \sphinxcode{\sphinxupquote{self.getTimeBase()}}) that were used for
nearest-neighbor interpolation. Only returned if \sphinxtitleref{return\_t} is
True.

\end{itemize}


\end{description}\end{quote}
\subsubsection*{Examples}

All assume that \sphinxtitleref{Eq\_instance} is a valid instance of the appropriate
extension of the {\hyperref[\detokenize{eqtools:eqtools.core.Equilibrium}]{\sphinxcrossref{\sphinxcode{\sphinxupquote{Equilibrium}}}}} abstract class.

Find single pprime value for psinorm=0.7, t=0.26s:

\begin{sphinxVerbatim}[commandchars=\\\{\}]
\PYG{n}{pprime\PYGZus{}val} \PYG{o}{=} \PYG{n}{Eq\PYGZus{}instance}\PYG{o}{.}\PYG{n}{psinorm2pprime}\PYG{p}{(}\PYG{l+m+mf}{0.7}\PYG{p}{,} \PYG{l+m+mf}{0.26}\PYG{p}{)}
\end{sphinxVerbatim}

Find pprime values at psi\_norm values of 0.5 and 0.7 at the single time
t=0.26s:

\begin{sphinxVerbatim}[commandchars=\\\{\}]
\PYG{n}{pprime\PYGZus{}arr} \PYG{o}{=} \PYG{n}{Eq\PYGZus{}instance}\PYG{o}{.}\PYG{n}{psinorm2pprime}\PYG{p}{(}\PYG{p}{[}\PYG{l+m+mf}{0.5}\PYG{p}{,} \PYG{l+m+mf}{0.7}\PYG{p}{]}\PYG{p}{,} \PYG{l+m+mf}{0.26}\PYG{p}{)}
\end{sphinxVerbatim}

Find pprime values at psi\_norm=0.5 at times t={[}0.2s, 0.3s{]}:

\begin{sphinxVerbatim}[commandchars=\\\{\}]
\PYG{n}{pprime\PYGZus{}arr} \PYG{o}{=} \PYG{n}{Eq\PYGZus{}instance}\PYG{o}{.}\PYG{n}{psinorm2pprime}\PYG{p}{(}\PYG{l+m+mf}{0.5}\PYG{p}{,} \PYG{p}{[}\PYG{l+m+mf}{0.2}\PYG{p}{,} \PYG{l+m+mf}{0.3}\PYG{p}{]}\PYG{p}{)}
\end{sphinxVerbatim}

Find pprime values at (psinorm, t) points (0.6, 0.2s) and (0.5, 0.3s):

\begin{sphinxVerbatim}[commandchars=\\\{\}]
\PYG{n}{pprime\PYGZus{}arr} \PYG{o}{=} \PYG{n}{Eq\PYGZus{}instance}\PYG{o}{.}\PYG{n}{psinorm2pprime}\PYG{p}{(}\PYG{p}{[}\PYG{l+m+mf}{0.6}\PYG{p}{,} \PYG{l+m+mf}{0.5}\PYG{p}{]}\PYG{p}{,} \PYG{p}{[}\PYG{l+m+mf}{0.2}\PYG{p}{,} \PYG{l+m+mf}{0.3}\PYG{p}{]}\PYG{p}{,} \PYG{n}{each\PYGZus{}t}\PYG{o}{=}\PYG{k+kc}{False}\PYG{p}{)}
\end{sphinxVerbatim}

\end{fulllineitems}

\index{phinorm2pprime() (eqtools.core.Equilibrium method)@\spxentry{phinorm2pprime()}\spxextra{eqtools.core.Equilibrium method}}

\begin{fulllineitems}
\phantomsection\label{\detokenize{eqtools:eqtools.core.Equilibrium.phinorm2pprime}}\pysiglinewithargsret{\sphinxbfcode{\sphinxupquote{phinorm2pprime}}}{\emph{phinorm}, \emph{t}, \emph{**kwargs}}{}
Calculates the pressure gradient corresponding to the passed phinorm (normalized toroidal flux) values.

By default, EFIT only computes this inside the LCFS.
\begin{quote}\begin{description}
\item[{Parameters}] \leavevmode\begin{itemize}
\item {} 
\sphinxstyleliteralstrong{\sphinxupquote{phinorm}} (\sphinxstyleliteralemphasis{\sphinxupquote{Array-like}}\sphinxstyleliteralemphasis{\sphinxupquote{ or }}\sphinxstyleliteralemphasis{\sphinxupquote{scalar float}}) \textendash{} Values of the normalized
toroidal flux to map to pprime.

\item {} 
\sphinxstyleliteralstrong{\sphinxupquote{t}} (\sphinxstyleliteralemphasis{\sphinxupquote{Array-like}}\sphinxstyleliteralemphasis{\sphinxupquote{ or }}\sphinxstyleliteralemphasis{\sphinxupquote{scalar float}}) \textendash{} Times to perform the conversion at.
If \sphinxtitleref{t} is a single value, it is used for all of the elements of
\sphinxtitleref{phinorm}. If the \sphinxtitleref{each\_t} keyword is True, then \sphinxtitleref{t} must be scalar
or have exactly one dimension. If the \sphinxtitleref{each\_t} keyword is False,
\sphinxtitleref{t} must have the same shape as \sphinxtitleref{phinorm}.

\end{itemize}

\item[{Keyword Arguments}] \leavevmode\begin{itemize}
\item {} 
\sphinxstyleliteralstrong{\sphinxupquote{sqrt}} (\sphinxstyleliteralemphasis{\sphinxupquote{Boolean}}) \textendash{} Set to True to return the square root of pprime.
Only the square root of positive values is taken. Negative
values are replaced with zeros, consistent with Steve Wolfe’s
IDL implementation efit\_rz2rho.pro. Default is False.

\item {} 
\sphinxstyleliteralstrong{\sphinxupquote{each\_t}} (\sphinxstyleliteralemphasis{\sphinxupquote{Boolean}}) \textendash{} When True, the elements in \sphinxtitleref{phinorm} are evaluated
at each value in \sphinxtitleref{t}. If True, \sphinxtitleref{t} must have only one dimension
(or be a scalar). If False, \sphinxtitleref{t} must match the shape of \sphinxtitleref{phinorm}
or be a scalar. Default is True (evaluate ALL \sphinxtitleref{phinorm} at EACH
element in \sphinxtitleref{t}).

\item {} 
\sphinxstyleliteralstrong{\sphinxupquote{k}} (\sphinxstyleliteralemphasis{\sphinxupquote{positive int}}) \textendash{} The degree of polynomial spline interpolation to
use in converting coordinates.

\item {} 
\sphinxstyleliteralstrong{\sphinxupquote{return\_t}} (\sphinxstyleliteralemphasis{\sphinxupquote{Boolean}}) \textendash{} Set to True to return a tuple of (\sphinxtitleref{pprime},
\sphinxtitleref{time\_idxs}), where \sphinxtitleref{time\_idxs} is the array of time indices
actually used in evaluating \sphinxtitleref{pprime} with nearest-neighbor
interpolation. (This is mostly present as an internal helper.)
Default is False (only return \sphinxtitleref{pprime}).

\end{itemize}

\item[{Returns}] \leavevmode

\sphinxtitleref{pprime} or (\sphinxtitleref{pprime}, \sphinxtitleref{time\_idxs})
\begin{itemize}
\item {} 
\sphinxstylestrong{pprime} (\sphinxtitleref{Array or scalar float}) - The pressure gradient. If
all of the input arguments are scalar, then a scalar is returned.
Otherwise, a scipy Array is returned.

\item {} 
\sphinxstylestrong{time\_idxs} (Array with same shape as \sphinxtitleref{pprime}) - The indices
(in \sphinxcode{\sphinxupquote{self.getTimeBase()}}) that were used for
nearest-neighbor interpolation. Only returned if \sphinxtitleref{return\_t} is
True.

\end{itemize}


\end{description}\end{quote}
\subsubsection*{Examples}

All assume that \sphinxtitleref{Eq\_instance} is a valid instance of the appropriate
extension of the {\hyperref[\detokenize{eqtools:eqtools.core.Equilibrium}]{\sphinxcrossref{\sphinxcode{\sphinxupquote{Equilibrium}}}}} abstract class.

Find single pprime value for phinorm=0.7, t=0.26s:

\begin{sphinxVerbatim}[commandchars=\\\{\}]
\PYG{n}{pprime\PYGZus{}val} \PYG{o}{=} \PYG{n}{Eq\PYGZus{}instance}\PYG{o}{.}\PYG{n}{phinorm2pprime}\PYG{p}{(}\PYG{l+m+mf}{0.7}\PYG{p}{,} \PYG{l+m+mf}{0.26}\PYG{p}{)}
\end{sphinxVerbatim}

Find pprime values at phinorm values of 0.5 and 0.7 at the single time
t=0.26s:

\begin{sphinxVerbatim}[commandchars=\\\{\}]
\PYG{n}{pprime\PYGZus{}arr} \PYG{o}{=} \PYG{n}{Eq\PYGZus{}instance}\PYG{o}{.}\PYG{n}{phinorm2pprime}\PYG{p}{(}\PYG{p}{[}\PYG{l+m+mf}{0.5}\PYG{p}{,} \PYG{l+m+mf}{0.7}\PYG{p}{]}\PYG{p}{,} \PYG{l+m+mf}{0.26}\PYG{p}{)}
\end{sphinxVerbatim}

Find pprime values at phinorm=0.5 at times t={[}0.2s, 0.3s{]}:

\begin{sphinxVerbatim}[commandchars=\\\{\}]
\PYG{n}{pprime\PYGZus{}arr} \PYG{o}{=} \PYG{n}{Eq\PYGZus{}instance}\PYG{o}{.}\PYG{n}{phinorm2pprime}\PYG{p}{(}\PYG{l+m+mf}{0.5}\PYG{p}{,} \PYG{p}{[}\PYG{l+m+mf}{0.2}\PYG{p}{,} \PYG{l+m+mf}{0.3}\PYG{p}{]}\PYG{p}{)}
\end{sphinxVerbatim}

Find pprime values at (phinorm, t) points (0.6, 0.2s) and (0.5, 0.3s):

\begin{sphinxVerbatim}[commandchars=\\\{\}]
\PYG{n}{pprime\PYGZus{}arr} \PYG{o}{=} \PYG{n}{Eq\PYGZus{}instance}\PYG{o}{.}\PYG{n}{phinorm2pprime}\PYG{p}{(}\PYG{p}{[}\PYG{l+m+mf}{0.6}\PYG{p}{,} \PYG{l+m+mf}{0.5}\PYG{p}{]}\PYG{p}{,} \PYG{p}{[}\PYG{l+m+mf}{0.2}\PYG{p}{,} \PYG{l+m+mf}{0.3}\PYG{p}{]}\PYG{p}{,} \PYG{n}{each\PYGZus{}t}\PYG{o}{=}\PYG{k+kc}{False}\PYG{p}{)}
\end{sphinxVerbatim}

\end{fulllineitems}

\index{volnorm2pprime() (eqtools.core.Equilibrium method)@\spxentry{volnorm2pprime()}\spxextra{eqtools.core.Equilibrium method}}

\begin{fulllineitems}
\phantomsection\label{\detokenize{eqtools:eqtools.core.Equilibrium.volnorm2pprime}}\pysiglinewithargsret{\sphinxbfcode{\sphinxupquote{volnorm2pprime}}}{\emph{volnorm}, \emph{t}, \emph{**kwargs}}{}
Calculates the pressure gradient corresponding to the passed volnorm (normalized flux surface volume) values.

By default, EFIT only computes this inside the LCFS.
\begin{quote}\begin{description}
\item[{Parameters}] \leavevmode\begin{itemize}
\item {} 
\sphinxstyleliteralstrong{\sphinxupquote{volnorm}} (\sphinxstyleliteralemphasis{\sphinxupquote{Array-like}}\sphinxstyleliteralemphasis{\sphinxupquote{ or }}\sphinxstyleliteralemphasis{\sphinxupquote{scalar float}}) \textendash{} Values of the normalized
flux surface volume to map to pprime.

\item {} 
\sphinxstyleliteralstrong{\sphinxupquote{t}} (\sphinxstyleliteralemphasis{\sphinxupquote{Array-like}}\sphinxstyleliteralemphasis{\sphinxupquote{ or }}\sphinxstyleliteralemphasis{\sphinxupquote{scalar float}}) \textendash{} Times to perform the conversion at.
If \sphinxtitleref{t} is a single value, it is used for all of the elements of
\sphinxtitleref{volnorm}. If the \sphinxtitleref{each\_t} keyword is True, then \sphinxtitleref{t} must be scalar
or have exactly one dimension. If the \sphinxtitleref{each\_t} keyword is False,
\sphinxtitleref{t} must have the same shape as \sphinxtitleref{volnorm}.

\end{itemize}

\item[{Keyword Arguments}] \leavevmode\begin{itemize}
\item {} 
\sphinxstyleliteralstrong{\sphinxupquote{sqrt}} (\sphinxstyleliteralemphasis{\sphinxupquote{Boolean}}) \textendash{} Set to True to return the square root of pprime.
Only the square root of positive values is taken. Negative
values are replaced with zeros, consistent with Steve Wolfe’s
IDL implementation efit\_rz2rho.pro. Default is False.

\item {} 
\sphinxstyleliteralstrong{\sphinxupquote{each\_t}} (\sphinxstyleliteralemphasis{\sphinxupquote{Boolean}}) \textendash{} When True, the elements in \sphinxtitleref{volnorm} are evaluated
at each value in \sphinxtitleref{t}. If True, \sphinxtitleref{t} must have only one dimension
(or be a scalar). If False, \sphinxtitleref{t} must match the shape of \sphinxtitleref{volnorm}
or be a scalar. Default is True (evaluate ALL \sphinxtitleref{volnorm} at EACH
element in \sphinxtitleref{t}).

\item {} 
\sphinxstyleliteralstrong{\sphinxupquote{k}} (\sphinxstyleliteralemphasis{\sphinxupquote{positive int}}) \textendash{} The degree of polynomial spline interpolation to
use in converting coordinates.

\item {} 
\sphinxstyleliteralstrong{\sphinxupquote{return\_t}} (\sphinxstyleliteralemphasis{\sphinxupquote{Boolean}}) \textendash{} Set to True to return a tuple of (\sphinxtitleref{pprime},
\sphinxtitleref{time\_idxs}), where \sphinxtitleref{time\_idxs} is the array of time indices
actually used in evaluating \sphinxtitleref{pprime} with nearest-neighbor
interpolation. (This is mostly present as an internal helper.)
Default is False (only return \sphinxtitleref{pprime}).

\end{itemize}

\item[{Returns}] \leavevmode

\sphinxtitleref{pprime} or (\sphinxtitleref{pprime}, \sphinxtitleref{time\_idxs})
\begin{itemize}
\item {} 
\sphinxstylestrong{pprime} (\sphinxtitleref{Array or scalar float}) - The pressure gradient. If
all of the input arguments are scalar, then a scalar is returned.
Otherwise, a scipy Array is returned.

\item {} 
\sphinxstylestrong{time\_idxs} (Array with same shape as \sphinxtitleref{pprime}) - The indices
(in \sphinxcode{\sphinxupquote{self.getTimeBase()}}) that were used for
nearest-neighbor interpolation. Only returned if \sphinxtitleref{return\_t} is
True.

\end{itemize}


\end{description}\end{quote}
\subsubsection*{Examples}

All assume that \sphinxtitleref{Eq\_instance} is a valid instance of the appropriate
extension of the {\hyperref[\detokenize{eqtools:eqtools.core.Equilibrium}]{\sphinxcrossref{\sphinxcode{\sphinxupquote{Equilibrium}}}}} abstract class.

Find single pprime value for volnorm=0.7, t=0.26s:

\begin{sphinxVerbatim}[commandchars=\\\{\}]
\PYG{n}{pprime\PYGZus{}val} \PYG{o}{=} \PYG{n}{Eq\PYGZus{}instance}\PYG{o}{.}\PYG{n}{volnorm2pprime}\PYG{p}{(}\PYG{l+m+mf}{0.7}\PYG{p}{,} \PYG{l+m+mf}{0.26}\PYG{p}{)}
\end{sphinxVerbatim}

Find pprime values at volnorm values of 0.5 and 0.7 at the single time
t=0.26s:

\begin{sphinxVerbatim}[commandchars=\\\{\}]
\PYG{n}{pprime\PYGZus{}arr} \PYG{o}{=} \PYG{n}{Eq\PYGZus{}instance}\PYG{o}{.}\PYG{n}{volnorm2pprime}\PYG{p}{(}\PYG{p}{[}\PYG{l+m+mf}{0.5}\PYG{p}{,} \PYG{l+m+mf}{0.7}\PYG{p}{]}\PYG{p}{,} \PYG{l+m+mf}{0.26}\PYG{p}{)}
\end{sphinxVerbatim}

Find pprime values at volnorm=0.5 at times t={[}0.2s, 0.3s{]}:

\begin{sphinxVerbatim}[commandchars=\\\{\}]
\PYG{n}{pprime\PYGZus{}arr} \PYG{o}{=} \PYG{n}{Eq\PYGZus{}instance}\PYG{o}{.}\PYG{n}{volnorm2pprime}\PYG{p}{(}\PYG{l+m+mf}{0.5}\PYG{p}{,} \PYG{p}{[}\PYG{l+m+mf}{0.2}\PYG{p}{,} \PYG{l+m+mf}{0.3}\PYG{p}{]}\PYG{p}{)}
\end{sphinxVerbatim}

Find pprime values at (volnorm, t) points (0.6, 0.2s) and (0.5, 0.3s):

\begin{sphinxVerbatim}[commandchars=\\\{\}]
\PYG{n}{pprime\PYGZus{}arr} \PYG{o}{=} \PYG{n}{Eq\PYGZus{}instance}\PYG{o}{.}\PYG{n}{volnorm2pprime}\PYG{p}{(}\PYG{p}{[}\PYG{l+m+mf}{0.6}\PYG{p}{,} \PYG{l+m+mf}{0.5}\PYG{p}{]}\PYG{p}{,} \PYG{p}{[}\PYG{l+m+mf}{0.2}\PYG{p}{,} \PYG{l+m+mf}{0.3}\PYG{p}{]}\PYG{p}{,} \PYG{n}{each\PYGZus{}t}\PYG{o}{=}\PYG{k+kc}{False}\PYG{p}{)}
\end{sphinxVerbatim}

\end{fulllineitems}

\index{rz2v() (eqtools.core.Equilibrium method)@\spxentry{rz2v()}\spxextra{eqtools.core.Equilibrium method}}

\begin{fulllineitems}
\phantomsection\label{\detokenize{eqtools:eqtools.core.Equilibrium.rz2v}}\pysiglinewithargsret{\sphinxbfcode{\sphinxupquote{rz2v}}}{\emph{R}, \emph{Z}, \emph{t}, \emph{**kwargs}}{}
Calculates the flux surface volume at the given (R, Z, t).

By default, EFIT only computes this inside the LCFS.
\begin{quote}\begin{description}
\item[{Parameters}] \leavevmode\begin{itemize}
\item {} 
\sphinxstyleliteralstrong{\sphinxupquote{R}} (\sphinxstyleliteralemphasis{\sphinxupquote{Array-like}}\sphinxstyleliteralemphasis{\sphinxupquote{ or }}\sphinxstyleliteralemphasis{\sphinxupquote{scalar float}}) \textendash{} Values of the radial coordinate to
map to v. If \sphinxtitleref{R} and \sphinxtitleref{Z} are both scalar values,
they are used as the coordinate pair for all of the values in
\sphinxtitleref{t}. Must have the same shape as \sphinxtitleref{Z} unless the \sphinxtitleref{make\_grid}
keyword is set. If the \sphinxtitleref{make\_grid} keyword is True, \sphinxtitleref{R} must
have exactly one dimension.

\item {} 
\sphinxstyleliteralstrong{\sphinxupquote{Z}} (\sphinxstyleliteralemphasis{\sphinxupquote{Array-like}}\sphinxstyleliteralemphasis{\sphinxupquote{ or }}\sphinxstyleliteralemphasis{\sphinxupquote{scalar float}}) \textendash{} Values of the vertical coordinate to
map to v. If \sphinxtitleref{R} and \sphinxtitleref{Z} are both scalar values,
they are used as the coordinate pair for all of the values in
\sphinxtitleref{t}. Must have the same shape as \sphinxtitleref{R} unless the \sphinxtitleref{make\_grid}
keyword is set. If the \sphinxtitleref{make\_grid} keyword is True, \sphinxtitleref{Z} must
have exactly one dimension.

\item {} 
\sphinxstyleliteralstrong{\sphinxupquote{t}} (\sphinxstyleliteralemphasis{\sphinxupquote{Array-like}}\sphinxstyleliteralemphasis{\sphinxupquote{ or }}\sphinxstyleliteralemphasis{\sphinxupquote{scalar float}}) \textendash{} Times to perform the conversion at.
If \sphinxtitleref{t} is a single value, it is used for all of the elements of
\sphinxtitleref{R}, \sphinxtitleref{Z}. If the \sphinxtitleref{each\_t} keyword is True, then \sphinxtitleref{t} must be
scalar or have exactly one dimension. If the \sphinxtitleref{each\_t} keyword is
False, \sphinxtitleref{t} must have the same shape as \sphinxtitleref{R} and \sphinxtitleref{Z} (or their
meshgrid if \sphinxtitleref{make\_grid} is True).

\end{itemize}

\item[{Keyword Arguments}] \leavevmode\begin{itemize}
\item {} 
\sphinxstyleliteralstrong{\sphinxupquote{sqrt}} (\sphinxstyleliteralemphasis{\sphinxupquote{Boolean}}) \textendash{} Set to True to return the square root of v.
Only the square root of positive values is taken. Negative
values are replaced with zeros, consistent with Steve Wolfe’s
IDL implementation efit\_rz2rho.pro. Default is False.

\item {} 
\sphinxstyleliteralstrong{\sphinxupquote{each\_t}} (\sphinxstyleliteralemphasis{\sphinxupquote{Boolean}}) \textendash{} When True, the elements in \sphinxtitleref{R}, \sphinxtitleref{Z} are evaluated
at each value in \sphinxtitleref{t}. If True, \sphinxtitleref{t} must have only one dimension
(or be a scalar). If False, \sphinxtitleref{t} must match the shape of \sphinxtitleref{R} and
\sphinxtitleref{Z} or be a scalar. Default is True (evaluate ALL \sphinxtitleref{R}, \sphinxtitleref{Z} at
EACH element in \sphinxtitleref{t}).

\item {} 
\sphinxstyleliteralstrong{\sphinxupquote{make\_grid}} (\sphinxstyleliteralemphasis{\sphinxupquote{Boolean}}) \textendash{} Set to True to pass \sphinxtitleref{R} and \sphinxtitleref{Z} through
\sphinxcode{\sphinxupquote{scipy.meshgrid()}} before evaluating. If this is set to
True, \sphinxtitleref{R} and \sphinxtitleref{Z} must each only have a single dimension, but
can have different lengths. Default is False (do not form
meshgrid).

\item {} 
\sphinxstyleliteralstrong{\sphinxupquote{length\_unit}} (\sphinxstyleliteralemphasis{\sphinxupquote{String}}\sphinxstyleliteralemphasis{\sphinxupquote{ or }}\sphinxstyleliteralemphasis{\sphinxupquote{1}}) \textendash{} 
Length unit that \sphinxtitleref{R}, \sphinxtitleref{Z} are given in.
If a string is given, it must be a valid unit specifier:
\begin{quote}


\begin{savenotes}\sphinxattablestart
\centering
\begin{tabulary}{\linewidth}[t]{|T|T|}
\hline

’m’
&
meters
\\
\hline
’cm’
&
centimeters
\\
\hline
’mm’
&
millimeters
\\
\hline
’in’
&
inches
\\
\hline
’ft’
&
feet
\\
\hline
’yd’
&
yards
\\
\hline
’smoot’
&
smoots
\\
\hline
’cubit’
&
cubits
\\
\hline
’hand’
&
hands
\\
\hline
’default’
&
meters
\\
\hline
\end{tabulary}
\par
\sphinxattableend\end{savenotes}
\end{quote}

If length\_unit is 1 or None, meters are assumed. The default
value is 1 (use meters).


\item {} 
\sphinxstyleliteralstrong{\sphinxupquote{return\_t}} (\sphinxstyleliteralemphasis{\sphinxupquote{Boolean}}) \textendash{} Set to True to return a tuple of (\sphinxtitleref{v},
\sphinxtitleref{time\_idxs}), where \sphinxtitleref{time\_idxs} is the array of time indices
actually used in evaluating \sphinxtitleref{v} with nearest-neighbor
interpolation. (This is mostly present as an internal helper.)
Default is False (only return \sphinxtitleref{v}).

\end{itemize}

\item[{Returns}] \leavevmode

\sphinxtitleref{v} or (\sphinxtitleref{v}, \sphinxtitleref{time\_idxs})
\begin{itemize}
\item {} 
\sphinxstylestrong{v} (\sphinxtitleref{Array or scalar float}) - The flux surface volume. If all
of the input arguments are scalar, then a scalar is
returned. Otherwise, a scipy Array is returned. If \sphinxtitleref{R} and \sphinxtitleref{Z}
both have the same shape then \sphinxtitleref{v} has this shape as well,
unless the \sphinxtitleref{make\_grid} keyword was True, in which case \sphinxtitleref{v}
has shape (len(\sphinxtitleref{Z}), len(\sphinxtitleref{R})).

\item {} 
\sphinxstylestrong{time\_idxs} (Array with same shape as \sphinxtitleref{v}) - The indices
(in \sphinxcode{\sphinxupquote{self.getTimeBase()}}) that were used for
nearest-neighbor interpolation. Only returned if \sphinxtitleref{return\_t} is
True.

\end{itemize}


\end{description}\end{quote}
\subsubsection*{Examples}

All assume that \sphinxtitleref{Eq\_instance} is a valid instance of the
appropriate extension of the {\hyperref[\detokenize{eqtools:eqtools.core.Equilibrium}]{\sphinxcrossref{\sphinxcode{\sphinxupquote{Equilibrium}}}}} abstract class.

Find single v value at R=0.6m, Z=0.0m, t=0.26s:

\begin{sphinxVerbatim}[commandchars=\\\{\}]
\PYG{n}{v\PYGZus{}val} \PYG{o}{=} \PYG{n}{Eq\PYGZus{}instance}\PYG{o}{.}\PYG{n}{rz2v}\PYG{p}{(}\PYG{l+m+mf}{0.6}\PYG{p}{,} \PYG{l+m+mi}{0}\PYG{p}{,} \PYG{l+m+mf}{0.26}\PYG{p}{)}
\end{sphinxVerbatim}

Find v values at (R, Z) points (0.6m, 0m) and (0.8m, 0m) at the
single time t=0.26s. Note that the \sphinxtitleref{Z} vector must be fully specified,
even if the values are all the same:

\begin{sphinxVerbatim}[commandchars=\\\{\}]
\PYG{n}{v\PYGZus{}arr} \PYG{o}{=} \PYG{n}{Eq\PYGZus{}instance}\PYG{o}{.}\PYG{n}{rz2v}\PYG{p}{(}\PYG{p}{[}\PYG{l+m+mf}{0.6}\PYG{p}{,} \PYG{l+m+mf}{0.8}\PYG{p}{]}\PYG{p}{,} \PYG{p}{[}\PYG{l+m+mi}{0}\PYG{p}{,} \PYG{l+m+mi}{0}\PYG{p}{]}\PYG{p}{,} \PYG{l+m+mf}{0.26}\PYG{p}{)}
\end{sphinxVerbatim}

Find v values at (R, Z) points (0.6m, 0m) at times t={[}0.2s, 0.3s{]}:

\begin{sphinxVerbatim}[commandchars=\\\{\}]
\PYG{n}{v\PYGZus{}arr} \PYG{o}{=} \PYG{n}{Eq\PYGZus{}instance}\PYG{o}{.}\PYG{n}{rz2v}\PYG{p}{(}\PYG{l+m+mf}{0.6}\PYG{p}{,} \PYG{l+m+mi}{0}\PYG{p}{,} \PYG{p}{[}\PYG{l+m+mf}{0.2}\PYG{p}{,} \PYG{l+m+mf}{0.3}\PYG{p}{]}\PYG{p}{)}
\end{sphinxVerbatim}

Find v values at (R, Z, t) points (0.6m, 0m, 0.2s) and (0.5m, 0.2m, 0.3s):

\begin{sphinxVerbatim}[commandchars=\\\{\}]
\PYG{n}{v\PYGZus{}arr} \PYG{o}{=} \PYG{n}{Eq\PYGZus{}instance}\PYG{o}{.}\PYG{n}{rz2v}\PYG{p}{(}\PYG{p}{[}\PYG{l+m+mf}{0.6}\PYG{p}{,} \PYG{l+m+mf}{0.5}\PYG{p}{]}\PYG{p}{,} \PYG{p}{[}\PYG{l+m+mi}{0}\PYG{p}{,} \PYG{l+m+mf}{0.2}\PYG{p}{]}\PYG{p}{,} \PYG{p}{[}\PYG{l+m+mf}{0.2}\PYG{p}{,} \PYG{l+m+mf}{0.3}\PYG{p}{]}\PYG{p}{,} \PYG{n}{each\PYGZus{}t}\PYG{o}{=}\PYG{k+kc}{False}\PYG{p}{)}
\end{sphinxVerbatim}

Find v values on grid defined by 1D vector of radial positions \sphinxtitleref{R}
and 1D vector of vertical positions \sphinxtitleref{Z} at time t=0.2s:

\begin{sphinxVerbatim}[commandchars=\\\{\}]
\PYG{n}{v\PYGZus{}mat} \PYG{o}{=} \PYG{n}{Eq\PYGZus{}instance}\PYG{o}{.}\PYG{n}{rz2v}\PYG{p}{(}\PYG{n}{R}\PYG{p}{,} \PYG{n}{Z}\PYG{p}{,} \PYG{l+m+mf}{0.2}\PYG{p}{,} \PYG{n}{make\PYGZus{}grid}\PYG{o}{=}\PYG{k+kc}{True}\PYG{p}{)}
\end{sphinxVerbatim}

\end{fulllineitems}

\index{rmid2v() (eqtools.core.Equilibrium method)@\spxentry{rmid2v()}\spxextra{eqtools.core.Equilibrium method}}

\begin{fulllineitems}
\phantomsection\label{\detokenize{eqtools:eqtools.core.Equilibrium.rmid2v}}\pysiglinewithargsret{\sphinxbfcode{\sphinxupquote{rmid2v}}}{\emph{R\_mid}, \emph{t}, \emph{**kwargs}}{}
Calculates the flux surface volume corresponding to the passed R\_mid (mapped outboard midplane major radius) values.

By default, EFIT only computes this inside the LCFS.
\begin{quote}\begin{description}
\item[{Parameters}] \leavevmode\begin{itemize}
\item {} 
\sphinxstyleliteralstrong{\sphinxupquote{R\_mid}} (\sphinxstyleliteralemphasis{\sphinxupquote{Array-like}}\sphinxstyleliteralemphasis{\sphinxupquote{ or }}\sphinxstyleliteralemphasis{\sphinxupquote{scalar float}}) \textendash{} Values of the outboard midplane
major radius to map to v.

\item {} 
\sphinxstyleliteralstrong{\sphinxupquote{t}} (\sphinxstyleliteralemphasis{\sphinxupquote{Array-like}}\sphinxstyleliteralemphasis{\sphinxupquote{ or }}\sphinxstyleliteralemphasis{\sphinxupquote{scalar float}}) \textendash{} Times to perform the conversion at.
If \sphinxtitleref{t} is a single value, it is used for all of the elements of
\sphinxtitleref{R\_mid}. If the \sphinxtitleref{each\_t} keyword is True, then \sphinxtitleref{t} must be scalar
or have exactly one dimension. If the \sphinxtitleref{each\_t} keyword is False,
\sphinxtitleref{t} must have the same shape as \sphinxtitleref{R\_mid}.

\end{itemize}

\item[{Keyword Arguments}] \leavevmode\begin{itemize}
\item {} 
\sphinxstyleliteralstrong{\sphinxupquote{sqrt}} (\sphinxstyleliteralemphasis{\sphinxupquote{Boolean}}) \textendash{} Set to True to return the square root of v.
Only the square root of positive values is taken. Negative
values are replaced with zeros, consistent with Steve Wolfe’s
IDL implementation efit\_rz2rho.pro. Default is False.

\item {} 
\sphinxstyleliteralstrong{\sphinxupquote{each\_t}} (\sphinxstyleliteralemphasis{\sphinxupquote{Boolean}}) \textendash{} When True, the elements in \sphinxtitleref{R\_mid} are evaluated
at each value in \sphinxtitleref{t}. If True, \sphinxtitleref{t} must have only one dimension
(or be a scalar). If False, \sphinxtitleref{t} must match the shape of \sphinxtitleref{R\_mid}
or be a scalar. Default is True (evaluate ALL \sphinxtitleref{R\_mid} at EACH
element in \sphinxtitleref{t}).

\item {} 
\sphinxstyleliteralstrong{\sphinxupquote{length\_unit}} (\sphinxstyleliteralemphasis{\sphinxupquote{String}}\sphinxstyleliteralemphasis{\sphinxupquote{ or }}\sphinxstyleliteralemphasis{\sphinxupquote{1}}) \textendash{} 
Length unit that \sphinxtitleref{R\_mid} is given in.
If a string is given, it must be a valid unit specifier:
\begin{quote}


\begin{savenotes}\sphinxattablestart
\centering
\begin{tabulary}{\linewidth}[t]{|T|T|}
\hline

’m’
&
meters
\\
\hline
’cm’
&
centimeters
\\
\hline
’mm’
&
millimeters
\\
\hline
’in’
&
inches
\\
\hline
’ft’
&
feet
\\
\hline
’yd’
&
yards
\\
\hline
’smoot’
&
smoots
\\
\hline
’cubit’
&
cubits
\\
\hline
’hand’
&
hands
\\
\hline
’default’
&
meters
\\
\hline
\end{tabulary}
\par
\sphinxattableend\end{savenotes}
\end{quote}

If length\_unit is 1 or None, meters are assumed. The default
value is 1 (use meters).


\item {} 
\sphinxstyleliteralstrong{\sphinxupquote{k}} (\sphinxstyleliteralemphasis{\sphinxupquote{positive int}}) \textendash{} The degree of polynomial spline interpolation to
use in converting coordinates.

\item {} 
\sphinxstyleliteralstrong{\sphinxupquote{return\_t}} (\sphinxstyleliteralemphasis{\sphinxupquote{Boolean}}) \textendash{} Set to True to return a tuple of (\sphinxtitleref{p},
\sphinxtitleref{time\_idxs}), where \sphinxtitleref{time\_idxs} is the array of time indices
actually used in evaluating \sphinxtitleref{p} with nearest-neighbor
interpolation. (This is mostly present as an internal helper.)
Default is False (only return \sphinxtitleref{p}).

\end{itemize}

\item[{Returns}] \leavevmode

\sphinxtitleref{v} or (\sphinxtitleref{v}, \sphinxtitleref{time\_idxs})
\begin{itemize}
\item {} 
\sphinxstylestrong{v} (\sphinxtitleref{Array or scalar float}) - The flux surface volume.
If all of the input arguments are scalar, then a scalar is
returned. Otherwise, a scipy Array is returned.

\item {} 
\sphinxstylestrong{time\_idxs} (Array with same shape as \sphinxtitleref{v}) - The indices
(in \sphinxcode{\sphinxupquote{self.getTimeBase()}}) that were used for
nearest-neighbor interpolation. Only returned if \sphinxtitleref{return\_t} is
True.

\end{itemize}


\end{description}\end{quote}
\subsubsection*{Examples}

All assume that \sphinxtitleref{Eq\_instance} is a valid instance of the appropriate
extension of the {\hyperref[\detokenize{eqtools:eqtools.core.Equilibrium}]{\sphinxcrossref{\sphinxcode{\sphinxupquote{Equilibrium}}}}} abstract class.

Find single v value for Rmid=0.7m, t=0.26s:

\begin{sphinxVerbatim}[commandchars=\\\{\}]
\PYG{n}{v\PYGZus{}val} \PYG{o}{=} \PYG{n}{Eq\PYGZus{}instance}\PYG{o}{.}\PYG{n}{rmid2v}\PYG{p}{(}\PYG{l+m+mf}{0.7}\PYG{p}{,} \PYG{l+m+mf}{0.26}\PYG{p}{)}
\end{sphinxVerbatim}

Find v values at R\_mid values of 0.5m and 0.7m at the single time
t=0.26s:

\begin{sphinxVerbatim}[commandchars=\\\{\}]
\PYG{n}{v\PYGZus{}arr} \PYG{o}{=} \PYG{n}{Eq\PYGZus{}instance}\PYG{o}{.}\PYG{n}{rmid2v}\PYG{p}{(}\PYG{p}{[}\PYG{l+m+mf}{0.5}\PYG{p}{,} \PYG{l+m+mf}{0.7}\PYG{p}{]}\PYG{p}{,} \PYG{l+m+mf}{0.26}\PYG{p}{)}
\end{sphinxVerbatim}

Find v values at R\_mid=0.5m at times t={[}0.2s, 0.3s{]}:

\begin{sphinxVerbatim}[commandchars=\\\{\}]
\PYG{n}{v\PYGZus{}arr} \PYG{o}{=} \PYG{n}{Eq\PYGZus{}instance}\PYG{o}{.}\PYG{n}{rmid2v}\PYG{p}{(}\PYG{l+m+mf}{0.5}\PYG{p}{,} \PYG{p}{[}\PYG{l+m+mf}{0.2}\PYG{p}{,} \PYG{l+m+mf}{0.3}\PYG{p}{]}\PYG{p}{)}
\end{sphinxVerbatim}

Find v values at (R\_mid, t) points (0.6m, 0.2s) and (0.5m, 0.3s):

\begin{sphinxVerbatim}[commandchars=\\\{\}]
\PYG{n}{v\PYGZus{}arr} \PYG{o}{=} \PYG{n}{Eq\PYGZus{}instance}\PYG{o}{.}\PYG{n}{rmid2v}\PYG{p}{(}\PYG{p}{[}\PYG{l+m+mf}{0.6}\PYG{p}{,} \PYG{l+m+mf}{0.5}\PYG{p}{]}\PYG{p}{,} \PYG{p}{[}\PYG{l+m+mf}{0.2}\PYG{p}{,} \PYG{l+m+mf}{0.3}\PYG{p}{]}\PYG{p}{,} \PYG{n}{each\PYGZus{}t}\PYG{o}{=}\PYG{k+kc}{False}\PYG{p}{)}
\end{sphinxVerbatim}

\end{fulllineitems}

\index{roa2v() (eqtools.core.Equilibrium method)@\spxentry{roa2v()}\spxextra{eqtools.core.Equilibrium method}}

\begin{fulllineitems}
\phantomsection\label{\detokenize{eqtools:eqtools.core.Equilibrium.roa2v}}\pysiglinewithargsret{\sphinxbfcode{\sphinxupquote{roa2v}}}{\emph{roa}, \emph{t}, \emph{**kwargs}}{}
Convert the passed (r/a, t) coordinates into flux surface volume.

By default, EFIT only computes this inside the LCFS.
\begin{quote}\begin{description}
\item[{Parameters}] \leavevmode\begin{itemize}
\item {} 
\sphinxstyleliteralstrong{\sphinxupquote{roa}} (\sphinxstyleliteralemphasis{\sphinxupquote{Array-like}}\sphinxstyleliteralemphasis{\sphinxupquote{ or }}\sphinxstyleliteralemphasis{\sphinxupquote{scalar float}}) \textendash{} Values of the normalized minor
radius to map to v.

\item {} 
\sphinxstyleliteralstrong{\sphinxupquote{t}} (\sphinxstyleliteralemphasis{\sphinxupquote{Array-like}}\sphinxstyleliteralemphasis{\sphinxupquote{ or }}\sphinxstyleliteralemphasis{\sphinxupquote{scalar float}}) \textendash{} Times to perform the conversion at.
If \sphinxtitleref{t} is a single value, it is used for all of the elements of
\sphinxtitleref{roa}. If the \sphinxtitleref{each\_t} keyword is True, then \sphinxtitleref{t} must be scalar
or have exactly one dimension. If the \sphinxtitleref{each\_t} keyword is False,
\sphinxtitleref{t} must have the same shape as \sphinxtitleref{roa}.

\end{itemize}

\item[{Keyword Arguments}] \leavevmode\begin{itemize}
\item {} 
\sphinxstyleliteralstrong{\sphinxupquote{sqrt}} (\sphinxstyleliteralemphasis{\sphinxupquote{Boolean}}) \textendash{} Set to True to return the square root of v.
Only the square root of positive values is taken. Negative
values are replaced with zeros, consistent with Steve Wolfe’s
IDL implementation efit\_rz2rho.pro. Default is False.

\item {} 
\sphinxstyleliteralstrong{\sphinxupquote{each\_t}} (\sphinxstyleliteralemphasis{\sphinxupquote{Boolean}}) \textendash{} When True, the elements in \sphinxtitleref{roa} are evaluated
at each value in \sphinxtitleref{t}. If True, \sphinxtitleref{t} must have only one dimension
(or be a scalar). If False, \sphinxtitleref{t} must match the shape of \sphinxtitleref{roa}
or be a scalar. Default is True (evaluate ALL \sphinxtitleref{roa} at EACH
element in \sphinxtitleref{t}).

\item {} 
\sphinxstyleliteralstrong{\sphinxupquote{k}} (\sphinxstyleliteralemphasis{\sphinxupquote{positive int}}) \textendash{} The degree of polynomial spline interpolation to
use in converting coordinates.

\item {} 
\sphinxstyleliteralstrong{\sphinxupquote{return\_t}} (\sphinxstyleliteralemphasis{\sphinxupquote{Boolean}}) \textendash{} Set to True to return a tuple of (\sphinxtitleref{v},
\sphinxtitleref{time\_idxs}), where \sphinxtitleref{time\_idxs} is the array of time indices
actually used in evaluating \sphinxtitleref{v} with nearest-neighbor
interpolation. (This is mostly present as an internal helper.)
Default is False (only return \sphinxtitleref{v}).

\end{itemize}

\item[{Returns}] \leavevmode

\sphinxtitleref{v} or (\sphinxtitleref{v}, \sphinxtitleref{time\_idxs})
\begin{itemize}
\item {} 
\sphinxstylestrong{v} (\sphinxtitleref{Array or scalar float}) - The flux surface volume. If
all of the input arguments are scalar, then a scalar is returned.
Otherwise, a scipy Array is returned.

\item {} 
\sphinxstylestrong{time\_idxs} (Array with same shape as \sphinxtitleref{v}) - The indices
(in \sphinxcode{\sphinxupquote{self.getTimeBase()}}) that were used for
nearest-neighbor interpolation. Only returned if \sphinxtitleref{return\_t} is
True.

\end{itemize}


\end{description}\end{quote}
\subsubsection*{Examples}

All assume that \sphinxtitleref{Eq\_instance} is a valid instance of the appropriate
extension of the {\hyperref[\detokenize{eqtools:eqtools.core.Equilibrium}]{\sphinxcrossref{\sphinxcode{\sphinxupquote{Equilibrium}}}}} abstract class.

Find single v value at r/a=0.6, t=0.26s:

\begin{sphinxVerbatim}[commandchars=\\\{\}]
\PYG{n}{v\PYGZus{}val} \PYG{o}{=} \PYG{n}{Eq\PYGZus{}instance}\PYG{o}{.}\PYG{n}{roa2v}\PYG{p}{(}\PYG{l+m+mf}{0.6}\PYG{p}{,} \PYG{l+m+mf}{0.26}\PYG{p}{)}
\end{sphinxVerbatim}

Find v values at r/a points 0.6 and 0.8 at the
single time t=0.26s.:

\begin{sphinxVerbatim}[commandchars=\\\{\}]
\PYG{n}{v\PYGZus{}arr} \PYG{o}{=} \PYG{n}{Eq\PYGZus{}instance}\PYG{o}{.}\PYG{n}{roa2v}\PYG{p}{(}\PYG{p}{[}\PYG{l+m+mf}{0.6}\PYG{p}{,} \PYG{l+m+mf}{0.8}\PYG{p}{]}\PYG{p}{,} \PYG{l+m+mf}{0.26}\PYG{p}{)}
\end{sphinxVerbatim}

Find v values at r/a of 0.6 at times t={[}0.2s, 0.3s{]}:

\begin{sphinxVerbatim}[commandchars=\\\{\}]
\PYG{n}{v\PYGZus{}arr} \PYG{o}{=} \PYG{n}{Eq\PYGZus{}instance}\PYG{o}{.}\PYG{n}{roa2v}\PYG{p}{(}\PYG{l+m+mf}{0.6}\PYG{p}{,} \PYG{p}{[}\PYG{l+m+mf}{0.2}\PYG{p}{,} \PYG{l+m+mf}{0.3}\PYG{p}{]}\PYG{p}{)}
\end{sphinxVerbatim}

Find v values at (roa, t) points (0.6, 0.2s) and (0.5, 0.3s):

\begin{sphinxVerbatim}[commandchars=\\\{\}]
\PYG{n}{v\PYGZus{}arr} \PYG{o}{=} \PYG{n}{Eq\PYGZus{}instance}\PYG{o}{.}\PYG{n}{roa2v}\PYG{p}{(}\PYG{p}{[}\PYG{l+m+mf}{0.6}\PYG{p}{,} \PYG{l+m+mf}{0.5}\PYG{p}{]}\PYG{p}{,} \PYG{p}{[}\PYG{l+m+mf}{0.2}\PYG{p}{,} \PYG{l+m+mf}{0.3}\PYG{p}{]}\PYG{p}{,} \PYG{n}{each\PYGZus{}t}\PYG{o}{=}\PYG{k+kc}{False}\PYG{p}{)}
\end{sphinxVerbatim}

\end{fulllineitems}

\index{psinorm2v() (eqtools.core.Equilibrium method)@\spxentry{psinorm2v()}\spxextra{eqtools.core.Equilibrium method}}

\begin{fulllineitems}
\phantomsection\label{\detokenize{eqtools:eqtools.core.Equilibrium.psinorm2v}}\pysiglinewithargsret{\sphinxbfcode{\sphinxupquote{psinorm2v}}}{\emph{psinorm}, \emph{t}, \emph{**kwargs}}{}
Calculates the flux surface volume corresponding to the passed psi\_norm (normalized poloidal flux) values.

By default, EFIT only computes this inside the LCFS.
\begin{quote}\begin{description}
\item[{Parameters}] \leavevmode\begin{itemize}
\item {} 
\sphinxstyleliteralstrong{\sphinxupquote{psi\_norm}} (\sphinxstyleliteralemphasis{\sphinxupquote{Array-like}}\sphinxstyleliteralemphasis{\sphinxupquote{ or }}\sphinxstyleliteralemphasis{\sphinxupquote{scalar float}}) \textendash{} Values of the normalized
poloidal flux to map to v.

\item {} 
\sphinxstyleliteralstrong{\sphinxupquote{t}} (\sphinxstyleliteralemphasis{\sphinxupquote{Array-like}}\sphinxstyleliteralemphasis{\sphinxupquote{ or }}\sphinxstyleliteralemphasis{\sphinxupquote{scalar float}}) \textendash{} Times to perform the conversion at.
If \sphinxtitleref{t} is a single value, it is used for all of the elements of
\sphinxtitleref{psi\_norm}. If the \sphinxtitleref{each\_t} keyword is True, then \sphinxtitleref{t} must be scalar
or have exactly one dimension. If the \sphinxtitleref{each\_t} keyword is False,
\sphinxtitleref{t} must have the same shape as \sphinxtitleref{psi\_norm}.

\end{itemize}

\item[{Keyword Arguments}] \leavevmode\begin{itemize}
\item {} 
\sphinxstyleliteralstrong{\sphinxupquote{sqrt}} (\sphinxstyleliteralemphasis{\sphinxupquote{Boolean}}) \textendash{} Set to True to return the square root of v. Only
the square root of positive values is taken. Negative values are
replaced with zeros, consistent with Steve Wolfe’s IDL
implementation efit\_rz2rho.pro. Default is False.

\item {} 
\sphinxstyleliteralstrong{\sphinxupquote{each\_t}} (\sphinxstyleliteralemphasis{\sphinxupquote{Boolean}}) \textendash{} When True, the elements in \sphinxtitleref{psi\_norm} are evaluated at
each value in \sphinxtitleref{t}. If True, \sphinxtitleref{t} must have only one dimension (or
be a scalar). If False, \sphinxtitleref{t} must match the shape of \sphinxtitleref{psi\_norm} or be
a scalar. Default is True (evaluate ALL \sphinxtitleref{psi\_norm} at EACH element in
\sphinxtitleref{t}).

\item {} 
\sphinxstyleliteralstrong{\sphinxupquote{k}} (\sphinxstyleliteralemphasis{\sphinxupquote{positive int}}) \textendash{} The degree of polynomial spline interpolation to
use in converting coordinates.

\item {} 
\sphinxstyleliteralstrong{\sphinxupquote{return\_t}} (\sphinxstyleliteralemphasis{\sphinxupquote{Boolean}}) \textendash{} Set to True to return a tuple of (\sphinxtitleref{v},
\sphinxtitleref{time\_idxs}), where \sphinxtitleref{time\_idxs} is the array of time indices
actually used in evaluating \sphinxtitleref{v} with nearest-neighbor
interpolation. (This is mostly present as an internal helper.)
Default is False (only return \sphinxtitleref{v}).

\end{itemize}

\item[{Returns}] \leavevmode

\sphinxtitleref{v} or (\sphinxtitleref{v}, \sphinxtitleref{time\_idxs})
\begin{itemize}
\item {} 
\sphinxstylestrong{v} (\sphinxtitleref{Array or scalar float}) - The pressure. If
all of the input arguments are scalar, then a scalar is returned.
Otherwise, a scipy Array is returned.

\item {} 
\sphinxstylestrong{time\_idxs} (Array with same shape as \sphinxtitleref{v}) - The indices
(in \sphinxcode{\sphinxupquote{self.getTimeBase()}}) that were used for
nearest-neighbor interpolation. Only returned if \sphinxtitleref{return\_t} is
True.

\end{itemize}


\end{description}\end{quote}
\subsubsection*{Examples}

All assume that \sphinxtitleref{Eq\_instance} is a valid instance of the appropriate
extension of the {\hyperref[\detokenize{eqtools:eqtools.core.Equilibrium}]{\sphinxcrossref{\sphinxcode{\sphinxupquote{Equilibrium}}}}} abstract class.

Find single v value for psinorm=0.7, t=0.26s:

\begin{sphinxVerbatim}[commandchars=\\\{\}]
\PYG{n}{v\PYGZus{}val} \PYG{o}{=} \PYG{n}{Eq\PYGZus{}instance}\PYG{o}{.}\PYG{n}{psinorm2v}\PYG{p}{(}\PYG{l+m+mf}{0.7}\PYG{p}{,} \PYG{l+m+mf}{0.26}\PYG{p}{)}
\end{sphinxVerbatim}

Find v values at psi\_norm values of 0.5 and 0.7 at the single time
t=0.26s:

\begin{sphinxVerbatim}[commandchars=\\\{\}]
\PYG{n}{v\PYGZus{}arr} \PYG{o}{=} \PYG{n}{Eq\PYGZus{}instance}\PYG{o}{.}\PYG{n}{psinorm2v}\PYG{p}{(}\PYG{p}{[}\PYG{l+m+mf}{0.5}\PYG{p}{,} \PYG{l+m+mf}{0.7}\PYG{p}{]}\PYG{p}{,} \PYG{l+m+mf}{0.26}\PYG{p}{)}
\end{sphinxVerbatim}

Find v values at psi\_norm=0.5 at times t={[}0.2s, 0.3s{]}:

\begin{sphinxVerbatim}[commandchars=\\\{\}]
\PYG{n}{v\PYGZus{}arr} \PYG{o}{=} \PYG{n}{Eq\PYGZus{}instance}\PYG{o}{.}\PYG{n}{psinorm2v}\PYG{p}{(}\PYG{l+m+mf}{0.5}\PYG{p}{,} \PYG{p}{[}\PYG{l+m+mf}{0.2}\PYG{p}{,} \PYG{l+m+mf}{0.3}\PYG{p}{]}\PYG{p}{)}
\end{sphinxVerbatim}

Find v values at (psinorm, t) points (0.6, 0.2s) and (0.5, 0.3s):

\begin{sphinxVerbatim}[commandchars=\\\{\}]
\PYG{n}{v\PYGZus{}arr} \PYG{o}{=} \PYG{n}{Eq\PYGZus{}instance}\PYG{o}{.}\PYG{n}{psinorm2v}\PYG{p}{(}\PYG{p}{[}\PYG{l+m+mf}{0.6}\PYG{p}{,} \PYG{l+m+mf}{0.5}\PYG{p}{]}\PYG{p}{,} \PYG{p}{[}\PYG{l+m+mf}{0.2}\PYG{p}{,} \PYG{l+m+mf}{0.3}\PYG{p}{]}\PYG{p}{,} \PYG{n}{each\PYGZus{}t}\PYG{o}{=}\PYG{k+kc}{False}\PYG{p}{)}
\end{sphinxVerbatim}

\end{fulllineitems}

\index{phinorm2v() (eqtools.core.Equilibrium method)@\spxentry{phinorm2v()}\spxextra{eqtools.core.Equilibrium method}}

\begin{fulllineitems}
\phantomsection\label{\detokenize{eqtools:eqtools.core.Equilibrium.phinorm2v}}\pysiglinewithargsret{\sphinxbfcode{\sphinxupquote{phinorm2v}}}{\emph{phinorm}, \emph{t}, \emph{**kwargs}}{}
Calculates the flux surface volume corresponding to the passed phinorm (normalized toroidal flux) values.

By default, EFIT only computes this inside the LCFS.
\begin{quote}\begin{description}
\item[{Parameters}] \leavevmode\begin{itemize}
\item {} 
\sphinxstyleliteralstrong{\sphinxupquote{phinorm}} (\sphinxstyleliteralemphasis{\sphinxupquote{Array-like}}\sphinxstyleliteralemphasis{\sphinxupquote{ or }}\sphinxstyleliteralemphasis{\sphinxupquote{scalar float}}) \textendash{} Values of the normalized
toroidal flux to map to v.

\item {} 
\sphinxstyleliteralstrong{\sphinxupquote{t}} (\sphinxstyleliteralemphasis{\sphinxupquote{Array-like}}\sphinxstyleliteralemphasis{\sphinxupquote{ or }}\sphinxstyleliteralemphasis{\sphinxupquote{scalar float}}) \textendash{} Times to perform the conversion at.
If \sphinxtitleref{t} is a single value, it is used for all of the elements of
\sphinxtitleref{phinorm}. If the \sphinxtitleref{each\_t} keyword is True, then \sphinxtitleref{t} must be scalar
or have exactly one dimension. If the \sphinxtitleref{each\_t} keyword is False,
\sphinxtitleref{t} must have the same shape as \sphinxtitleref{phinorm}.

\end{itemize}

\item[{Keyword Arguments}] \leavevmode\begin{itemize}
\item {} 
\sphinxstyleliteralstrong{\sphinxupquote{sqrt}} (\sphinxstyleliteralemphasis{\sphinxupquote{Boolean}}) \textendash{} Set to True to return the square root of v.
Only the square root of positive values is taken. Negative
values are replaced with zeros, consistent with Steve Wolfe’s
IDL implementation efit\_rz2rho.pro. Default is False.

\item {} 
\sphinxstyleliteralstrong{\sphinxupquote{each\_t}} (\sphinxstyleliteralemphasis{\sphinxupquote{Boolean}}) \textendash{} When True, the elements in \sphinxtitleref{phinorm} are evaluated
at each value in \sphinxtitleref{t}. If True, \sphinxtitleref{t} must have only one dimension
(or be a scalar). If False, \sphinxtitleref{t} must match the shape of \sphinxtitleref{phinorm}
or be a scalar. Default is True (evaluate ALL \sphinxtitleref{phinorm} at EACH
element in \sphinxtitleref{t}).

\item {} 
\sphinxstyleliteralstrong{\sphinxupquote{k}} (\sphinxstyleliteralemphasis{\sphinxupquote{positive int}}) \textendash{} The degree of polynomial spline interpolation to
use in converting coordinates.

\item {} 
\sphinxstyleliteralstrong{\sphinxupquote{return\_t}} (\sphinxstyleliteralemphasis{\sphinxupquote{Boolean}}) \textendash{} Set to True to return a tuple of (\sphinxtitleref{v},
\sphinxtitleref{time\_idxs}), where \sphinxtitleref{time\_idxs} is the array of time indices
actually used in evaluating \sphinxtitleref{v} with nearest-neighbor
interpolation. (This is mostly present as an internal helper.)
Default is False (only return \sphinxtitleref{v}).

\end{itemize}

\item[{Returns}] \leavevmode

\sphinxtitleref{v} or (\sphinxtitleref{v}, \sphinxtitleref{time\_idxs})
\begin{itemize}
\item {} 
\sphinxstylestrong{v} (\sphinxtitleref{Array or scalar float}) - The flux surface volume. If
all of the input arguments are scalar, then a scalar is returned.
Otherwise, a scipy Array is returned.

\item {} 
\sphinxstylestrong{time\_idxs} (Array with same shape as \sphinxtitleref{v}) - The indices
(in \sphinxcode{\sphinxupquote{self.getTimeBase()}}) that were used for
nearest-neighbor interpolation. Only returned if \sphinxtitleref{return\_t} is
True.

\end{itemize}


\end{description}\end{quote}
\subsubsection*{Examples}

All assume that \sphinxtitleref{Eq\_instance} is a valid instance of the appropriate
extension of the {\hyperref[\detokenize{eqtools:eqtools.core.Equilibrium}]{\sphinxcrossref{\sphinxcode{\sphinxupquote{Equilibrium}}}}} abstract class.

Find single v value for phinorm=0.7, t=0.26s:

\begin{sphinxVerbatim}[commandchars=\\\{\}]
\PYG{n}{v\PYGZus{}val} \PYG{o}{=} \PYG{n}{Eq\PYGZus{}instance}\PYG{o}{.}\PYG{n}{phinorm2v}\PYG{p}{(}\PYG{l+m+mf}{0.7}\PYG{p}{,} \PYG{l+m+mf}{0.26}\PYG{p}{)}
\end{sphinxVerbatim}

Find v values at phinorm values of 0.5 and 0.7 at the single time
t=0.26s:

\begin{sphinxVerbatim}[commandchars=\\\{\}]
\PYG{n}{v\PYGZus{}arr} \PYG{o}{=} \PYG{n}{Eq\PYGZus{}instance}\PYG{o}{.}\PYG{n}{phinorm2v}\PYG{p}{(}\PYG{p}{[}\PYG{l+m+mf}{0.5}\PYG{p}{,} \PYG{l+m+mf}{0.7}\PYG{p}{]}\PYG{p}{,} \PYG{l+m+mf}{0.26}\PYG{p}{)}
\end{sphinxVerbatim}

Find v values at phinorm=0.5 at times t={[}0.2s, 0.3s{]}:

\begin{sphinxVerbatim}[commandchars=\\\{\}]
\PYG{n}{v\PYGZus{}arr} \PYG{o}{=} \PYG{n}{Eq\PYGZus{}instance}\PYG{o}{.}\PYG{n}{phinorm2v}\PYG{p}{(}\PYG{l+m+mf}{0.5}\PYG{p}{,} \PYG{p}{[}\PYG{l+m+mf}{0.2}\PYG{p}{,} \PYG{l+m+mf}{0.3}\PYG{p}{]}\PYG{p}{)}
\end{sphinxVerbatim}

Find v values at (phinorm, t) points (0.6, 0.2s) and (0.5, 0.3s):

\begin{sphinxVerbatim}[commandchars=\\\{\}]
\PYG{n}{v\PYGZus{}arr} \PYG{o}{=} \PYG{n}{Eq\PYGZus{}instance}\PYG{o}{.}\PYG{n}{phinorm2v}\PYG{p}{(}\PYG{p}{[}\PYG{l+m+mf}{0.6}\PYG{p}{,} \PYG{l+m+mf}{0.5}\PYG{p}{]}\PYG{p}{,} \PYG{p}{[}\PYG{l+m+mf}{0.2}\PYG{p}{,} \PYG{l+m+mf}{0.3}\PYG{p}{]}\PYG{p}{,} \PYG{n}{each\PYGZus{}t}\PYG{o}{=}\PYG{k+kc}{False}\PYG{p}{)}
\end{sphinxVerbatim}

\end{fulllineitems}

\index{volnorm2v() (eqtools.core.Equilibrium method)@\spxentry{volnorm2v()}\spxextra{eqtools.core.Equilibrium method}}

\begin{fulllineitems}
\phantomsection\label{\detokenize{eqtools:eqtools.core.Equilibrium.volnorm2v}}\pysiglinewithargsret{\sphinxbfcode{\sphinxupquote{volnorm2v}}}{\emph{volnorm}, \emph{t}, \emph{**kwargs}}{}
Calculates the flux surface volume corresponding to the passed volnorm (normalized flux surface volume) values.

By default, EFIT only computes this inside the LCFS.
\begin{quote}\begin{description}
\item[{Parameters}] \leavevmode\begin{itemize}
\item {} 
\sphinxstyleliteralstrong{\sphinxupquote{volnorm}} (\sphinxstyleliteralemphasis{\sphinxupquote{Array-like}}\sphinxstyleliteralemphasis{\sphinxupquote{ or }}\sphinxstyleliteralemphasis{\sphinxupquote{scalar float}}) \textendash{} Values of the normalized
flux surface volume to map to v.

\item {} 
\sphinxstyleliteralstrong{\sphinxupquote{t}} (\sphinxstyleliteralemphasis{\sphinxupquote{Array-like}}\sphinxstyleliteralemphasis{\sphinxupquote{ or }}\sphinxstyleliteralemphasis{\sphinxupquote{scalar float}}) \textendash{} Times to perform the conversion at.
If \sphinxtitleref{t} is a single value, it is used for all of the elements of
\sphinxtitleref{volnorm}. If the \sphinxtitleref{each\_t} keyword is True, then \sphinxtitleref{t} must be scalar
or have exactly one dimension. If the \sphinxtitleref{each\_t} keyword is False,
\sphinxtitleref{t} must have the same shape as \sphinxtitleref{volnorm}.

\end{itemize}

\item[{Keyword Arguments}] \leavevmode\begin{itemize}
\item {} 
\sphinxstyleliteralstrong{\sphinxupquote{sqrt}} (\sphinxstyleliteralemphasis{\sphinxupquote{Boolean}}) \textendash{} Set to True to return the square root of v.
Only the square root of positive values is taken. Negative
values are replaced with zeros, consistent with Steve Wolfe’s
IDL implementation efit\_rz2rho.pro. Default is False.

\item {} 
\sphinxstyleliteralstrong{\sphinxupquote{each\_t}} (\sphinxstyleliteralemphasis{\sphinxupquote{Boolean}}) \textendash{} When True, the elements in \sphinxtitleref{volnorm} are evaluated
at each value in \sphinxtitleref{t}. If True, \sphinxtitleref{t} must have only one dimension
(or be a scalar). If False, \sphinxtitleref{t} must match the shape of \sphinxtitleref{volnorm}
or be a scalar. Default is True (evaluate ALL \sphinxtitleref{volnorm} at EACH
element in \sphinxtitleref{t}).

\item {} 
\sphinxstyleliteralstrong{\sphinxupquote{k}} (\sphinxstyleliteralemphasis{\sphinxupquote{positive int}}) \textendash{} The degree of polynomial spline interpolation to
use in converting coordinates.

\item {} 
\sphinxstyleliteralstrong{\sphinxupquote{return\_t}} (\sphinxstyleliteralemphasis{\sphinxupquote{Boolean}}) \textendash{} Set to True to return a tuple of (\sphinxtitleref{v},
\sphinxtitleref{time\_idxs}), where \sphinxtitleref{time\_idxs} is the array of time indices
actually used in evaluating \sphinxtitleref{v} with nearest-neighbor
interpolation. (This is mostly present as an internal helper.)
Default is False (only return \sphinxtitleref{v}).

\end{itemize}

\item[{Returns}] \leavevmode

\sphinxtitleref{v} or (\sphinxtitleref{v}, \sphinxtitleref{time\_idxs})
\begin{itemize}
\item {} 
\sphinxstylestrong{v} (\sphinxtitleref{Array or scalar float}) - The flux surface volume. If
all of the input arguments are scalar, then a scalar is returned.
Otherwise, a scipy Array is returned.

\item {} 
\sphinxstylestrong{time\_idxs} (Array with same shape as \sphinxtitleref{v}) - The indices
(in \sphinxcode{\sphinxupquote{self.getTimeBase()}}) that were used for
nearest-neighbor interpolation. Only returned if \sphinxtitleref{return\_t} is
True.

\end{itemize}


\end{description}\end{quote}
\subsubsection*{Examples}

All assume that \sphinxtitleref{Eq\_instance} is a valid instance of the appropriate
extension of the {\hyperref[\detokenize{eqtools:eqtools.core.Equilibrium}]{\sphinxcrossref{\sphinxcode{\sphinxupquote{Equilibrium}}}}} abstract class.

Find single v value for volnorm=0.7, t=0.26s:

\begin{sphinxVerbatim}[commandchars=\\\{\}]
\PYG{n}{v\PYGZus{}val} \PYG{o}{=} \PYG{n}{Eq\PYGZus{}instance}\PYG{o}{.}\PYG{n}{volnorm2p}\PYG{p}{(}\PYG{l+m+mf}{0.7}\PYG{p}{,} \PYG{l+m+mf}{0.26}\PYG{p}{)}
\end{sphinxVerbatim}

Find v values at volnorm values of 0.5 and 0.7 at the single time
t=0.26s:

\begin{sphinxVerbatim}[commandchars=\\\{\}]
\PYG{n}{v\PYGZus{}arr} \PYG{o}{=} \PYG{n}{Eq\PYGZus{}instance}\PYG{o}{.}\PYG{n}{volnorm2v}\PYG{p}{(}\PYG{p}{[}\PYG{l+m+mf}{0.5}\PYG{p}{,} \PYG{l+m+mf}{0.7}\PYG{p}{]}\PYG{p}{,} \PYG{l+m+mf}{0.26}\PYG{p}{)}
\end{sphinxVerbatim}

Find v values at volnorm=0.5 at times t={[}0.2s, 0.3s{]}:

\begin{sphinxVerbatim}[commandchars=\\\{\}]
\PYG{n}{v\PYGZus{}arr} \PYG{o}{=} \PYG{n}{Eq\PYGZus{}instance}\PYG{o}{.}\PYG{n}{volnorm2v}\PYG{p}{(}\PYG{l+m+mf}{0.5}\PYG{p}{,} \PYG{p}{[}\PYG{l+m+mf}{0.2}\PYG{p}{,} \PYG{l+m+mf}{0.3}\PYG{p}{]}\PYG{p}{)}
\end{sphinxVerbatim}

Find v values at (volnorm, t) points (0.6, 0.2s) and (0.5, 0.3s):

\begin{sphinxVerbatim}[commandchars=\\\{\}]
\PYG{n}{v\PYGZus{}arr} \PYG{o}{=} \PYG{n}{Eq\PYGZus{}instance}\PYG{o}{.}\PYG{n}{volnorm2v}\PYG{p}{(}\PYG{p}{[}\PYG{l+m+mf}{0.6}\PYG{p}{,} \PYG{l+m+mf}{0.5}\PYG{p}{]}\PYG{p}{,} \PYG{p}{[}\PYG{l+m+mf}{0.2}\PYG{p}{,} \PYG{l+m+mf}{0.3}\PYG{p}{]}\PYG{p}{,} \PYG{n}{each\PYGZus{}t}\PYG{o}{=}\PYG{k+kc}{False}\PYG{p}{)}
\end{sphinxVerbatim}

\end{fulllineitems}

\index{rz2BR() (eqtools.core.Equilibrium method)@\spxentry{rz2BR()}\spxextra{eqtools.core.Equilibrium method}}

\begin{fulllineitems}
\phantomsection\label{\detokenize{eqtools:eqtools.core.Equilibrium.rz2BR}}\pysiglinewithargsret{\sphinxbfcode{\sphinxupquote{rz2BR}}}{\emph{R}, \emph{Z}, \emph{t}, \emph{return\_t=False}, \emph{make\_grid=False}, \emph{each\_t=True}, \emph{length\_unit=1}}{}
Calculates the major radial component of the magnetic field at the given (R, Z, t) coordinates.

Uses
\begin{equation*}
\begin{split}B_R = -\frac{1}{R}\frac{\partial \psi}{\partial Z}\end{split}
\end{equation*}\begin{quote}\begin{description}
\item[{Parameters}] \leavevmode\begin{itemize}
\item {} 
\sphinxstyleliteralstrong{\sphinxupquote{R}} (\sphinxstyleliteralemphasis{\sphinxupquote{Array-like}}\sphinxstyleliteralemphasis{\sphinxupquote{ or }}\sphinxstyleliteralemphasis{\sphinxupquote{scalar float}}) \textendash{} Values of the radial coordinate to
map to radial field. If \sphinxtitleref{R} and \sphinxtitleref{Z} are both scalar values,
they are used as the coordinate pair for all of the values in
\sphinxtitleref{t}. Must have the same shape as \sphinxtitleref{Z} unless the \sphinxtitleref{make\_grid}
keyword is set. If the \sphinxtitleref{make\_grid} keyword is True, \sphinxtitleref{R} must
have exactly one dimension.

\item {} 
\sphinxstyleliteralstrong{\sphinxupquote{Z}} (\sphinxstyleliteralemphasis{\sphinxupquote{Array-like}}\sphinxstyleliteralemphasis{\sphinxupquote{ or }}\sphinxstyleliteralemphasis{\sphinxupquote{scalar float}}) \textendash{} Values of the vertical coordinate to
map to radial field. If \sphinxtitleref{R} and \sphinxtitleref{Z} are both scalar values,
they are used as the coordinate pair for all of the values in
\sphinxtitleref{t}. Must have the same shape as \sphinxtitleref{R} unless the \sphinxtitleref{make\_grid}
keyword is set. If the \sphinxtitleref{make\_grid} keyword is True, \sphinxtitleref{Z} must
have exactly one dimension.

\item {} 
\sphinxstyleliteralstrong{\sphinxupquote{t}} (\sphinxstyleliteralemphasis{\sphinxupquote{Array-like}}\sphinxstyleliteralemphasis{\sphinxupquote{ or }}\sphinxstyleliteralemphasis{\sphinxupquote{scalar float}}) \textendash{} Times to perform the conversion at.
If \sphinxtitleref{t} is a single value, it is used for all of the elements of
\sphinxtitleref{R}, \sphinxtitleref{Z}. If the \sphinxtitleref{each\_t} keyword is True, then \sphinxtitleref{t} must be
scalar or have exactly one dimension. If the \sphinxtitleref{each\_t} keyword is
False, \sphinxtitleref{t} must have the same shape as \sphinxtitleref{R} and \sphinxtitleref{Z} (or their
meshgrid if \sphinxtitleref{make\_grid} is True).

\end{itemize}

\item[{Keyword Arguments}] \leavevmode\begin{itemize}
\item {} 
\sphinxstyleliteralstrong{\sphinxupquote{each\_t}} (\sphinxstyleliteralemphasis{\sphinxupquote{Boolean}}) \textendash{} When True, the elements in \sphinxtitleref{R}, \sphinxtitleref{Z} are evaluated
at each value in \sphinxtitleref{t}. If True, \sphinxtitleref{t} must have only one dimension
(or be a scalar). If False, \sphinxtitleref{t} must match the shape of \sphinxtitleref{R} and
\sphinxtitleref{Z} or be a scalar. Default is True (evaluate ALL \sphinxtitleref{R}, \sphinxtitleref{Z} at
EACH element in \sphinxtitleref{t}).

\item {} 
\sphinxstyleliteralstrong{\sphinxupquote{make\_grid}} (\sphinxstyleliteralemphasis{\sphinxupquote{Boolean}}) \textendash{} Set to True to pass \sphinxtitleref{R} and \sphinxtitleref{Z} through
\sphinxcode{\sphinxupquote{scipy.meshgrid()}} before evaluating. If this is set to
True, \sphinxtitleref{R} and \sphinxtitleref{Z} must each only have a single dimension, but
can have different lengths. Default is False (do not form
meshgrid).

\item {} 
\sphinxstyleliteralstrong{\sphinxupquote{length\_unit}} (\sphinxstyleliteralemphasis{\sphinxupquote{String}}\sphinxstyleliteralemphasis{\sphinxupquote{ or }}\sphinxstyleliteralemphasis{\sphinxupquote{1}}) \textendash{} 
Length unit that \sphinxtitleref{R}, \sphinxtitleref{Z} are given in.
If a string is given, it must be a valid unit specifier:
\begin{quote}


\begin{savenotes}\sphinxattablestart
\centering
\begin{tabulary}{\linewidth}[t]{|T|T|}
\hline

’m’
&
meters
\\
\hline
’cm’
&
centimeters
\\
\hline
’mm’
&
millimeters
\\
\hline
’in’
&
inches
\\
\hline
’ft’
&
feet
\\
\hline
’yd’
&
yards
\\
\hline
’smoot’
&
smoots
\\
\hline
’cubit’
&
cubits
\\
\hline
’hand’
&
hands
\\
\hline
’default’
&
meters
\\
\hline
\end{tabulary}
\par
\sphinxattableend\end{savenotes}
\end{quote}

If length\_unit is 1 or None, meters are assumed. The default
value is 1 (use meters).


\item {} 
\sphinxstyleliteralstrong{\sphinxupquote{return\_t}} (\sphinxstyleliteralemphasis{\sphinxupquote{Boolean}}) \textendash{} Set to True to return a tuple of (\sphinxtitleref{BR},
\sphinxtitleref{time\_idxs}), where \sphinxtitleref{time\_idxs} is the array of time indices
actually used in evaluating \sphinxtitleref{BR} with nearest-neighbor
interpolation. (This is mostly present as an internal helper.)
Default is False (only return \sphinxtitleref{BR}).

\end{itemize}

\item[{Returns}] \leavevmode

\sphinxtitleref{BR} or (\sphinxtitleref{BR}, \sphinxtitleref{time\_idxs})
\begin{itemize}
\item {} 
\sphinxstylestrong{BR} (\sphinxtitleref{Array or scalar float}) - The major radial component of
the magnetic field. If all of the input arguments are scalar, then
a scalar is returned. Otherwise, a scipy Array is returned. If \sphinxtitleref{R}
and \sphinxtitleref{Z} both have the same shape then \sphinxtitleref{BR} has this shape as well,
unless the \sphinxtitleref{make\_grid} keyword was True, in which case \sphinxtitleref{BR} has
shape (len(\sphinxtitleref{Z}), len(\sphinxtitleref{R})).

\item {} 
\sphinxstylestrong{time\_idxs} (Array with same shape as \sphinxtitleref{BR}) - The indices
(in \sphinxcode{\sphinxupquote{self.getTimeBase()}}) that were used for
nearest-neighbor interpolation. Only returned if \sphinxtitleref{return\_t} is
True.

\end{itemize}


\end{description}\end{quote}
\subsubsection*{Examples}

All assume that \sphinxtitleref{Eq\_instance} is a valid instance of the appropriate
extension of the {\hyperref[\detokenize{eqtools:eqtools.core.Equilibrium}]{\sphinxcrossref{\sphinxcode{\sphinxupquote{Equilibrium}}}}} abstract class.

Find single BR value at R=0.6m, Z=0.0m, t=0.26s:

\begin{sphinxVerbatim}[commandchars=\\\{\}]
\PYG{n}{BR\PYGZus{}val} \PYG{o}{=} \PYG{n}{Eq\PYGZus{}instance}\PYG{o}{.}\PYG{n}{rz2BR}\PYG{p}{(}\PYG{l+m+mf}{0.6}\PYG{p}{,} \PYG{l+m+mi}{0}\PYG{p}{,} \PYG{l+m+mf}{0.26}\PYG{p}{)}
\end{sphinxVerbatim}

Find BR values at (R, Z) points (0.6m, 0m) and (0.8m, 0m) at the
single time t=0.26s. Note that the \sphinxtitleref{Z} vector must be fully
specified, even if the values are all the same:

\begin{sphinxVerbatim}[commandchars=\\\{\}]
\PYG{n}{BR\PYGZus{}arr} \PYG{o}{=} \PYG{n}{Eq\PYGZus{}instance}\PYG{o}{.}\PYG{n}{rz2BR}\PYG{p}{(}\PYG{p}{[}\PYG{l+m+mf}{0.6}\PYG{p}{,} \PYG{l+m+mf}{0.8}\PYG{p}{]}\PYG{p}{,} \PYG{p}{[}\PYG{l+m+mi}{0}\PYG{p}{,} \PYG{l+m+mi}{0}\PYG{p}{]}\PYG{p}{,} \PYG{l+m+mf}{0.26}\PYG{p}{)}
\end{sphinxVerbatim}

Find BR values at (R, Z) points (0.6m, 0m) at times t={[}0.2s, 0.3s{]}:

\begin{sphinxVerbatim}[commandchars=\\\{\}]
\PYG{n}{BR\PYGZus{}arr} \PYG{o}{=} \PYG{n}{Eq\PYGZus{}instance}\PYG{o}{.}\PYG{n}{rz2BR}\PYG{p}{(}\PYG{l+m+mf}{0.6}\PYG{p}{,} \PYG{l+m+mi}{0}\PYG{p}{,} \PYG{p}{[}\PYG{l+m+mf}{0.2}\PYG{p}{,} \PYG{l+m+mf}{0.3}\PYG{p}{]}\PYG{p}{)}
\end{sphinxVerbatim}

Find BR values at (R, Z, t) points (0.6m, 0m, 0.2s) and
(0.5m, 0.2m, 0.3s):

\begin{sphinxVerbatim}[commandchars=\\\{\}]
\PYG{n}{BR\PYGZus{}arr} \PYG{o}{=} \PYG{n}{Eq\PYGZus{}instance}\PYG{o}{.}\PYG{n}{rz2BR}\PYG{p}{(}\PYG{p}{[}\PYG{l+m+mf}{0.6}\PYG{p}{,} \PYG{l+m+mf}{0.5}\PYG{p}{]}\PYG{p}{,} \PYG{p}{[}\PYG{l+m+mi}{0}\PYG{p}{,} \PYG{l+m+mf}{0.2}\PYG{p}{]}\PYG{p}{,} \PYG{p}{[}\PYG{l+m+mf}{0.2}\PYG{p}{,} \PYG{l+m+mf}{0.3}\PYG{p}{]}\PYG{p}{,} \PYG{n}{each\PYGZus{}t}\PYG{o}{=}\PYG{k+kc}{False}\PYG{p}{)}
\end{sphinxVerbatim}

Find BR values on grid defined by 1D vector of radial positions \sphinxtitleref{R}
and 1D vector of vertical positions \sphinxtitleref{Z} at time t=0.2s:

\begin{sphinxVerbatim}[commandchars=\\\{\}]
\PYG{n}{BR\PYGZus{}mat} \PYG{o}{=} \PYG{n}{Eq\PYGZus{}instance}\PYG{o}{.}\PYG{n}{rz2BR}\PYG{p}{(}\PYG{n}{R}\PYG{p}{,} \PYG{n}{Z}\PYG{p}{,} \PYG{l+m+mf}{0.2}\PYG{p}{,} \PYG{n}{make\PYGZus{}grid}\PYG{o}{=}\PYG{k+kc}{True}\PYG{p}{)}
\end{sphinxVerbatim}

\end{fulllineitems}

\index{rz2BZ() (eqtools.core.Equilibrium method)@\spxentry{rz2BZ()}\spxextra{eqtools.core.Equilibrium method}}

\begin{fulllineitems}
\phantomsection\label{\detokenize{eqtools:eqtools.core.Equilibrium.rz2BZ}}\pysiglinewithargsret{\sphinxbfcode{\sphinxupquote{rz2BZ}}}{\emph{R}, \emph{Z}, \emph{t}, \emph{return\_t=False}, \emph{make\_grid=False}, \emph{each\_t=True}, \emph{length\_unit=1}}{}
Calculates the vertical component of the magnetic field at the given (R, Z, t) coordinates.

Uses
\begin{equation*}
\begin{split}B_Z = \frac{1}{R}\frac{\partial \psi}{\partial R}\end{split}
\end{equation*}\begin{quote}\begin{description}
\item[{Parameters}] \leavevmode\begin{itemize}
\item {} 
\sphinxstyleliteralstrong{\sphinxupquote{R}} (\sphinxstyleliteralemphasis{\sphinxupquote{Array-like}}\sphinxstyleliteralemphasis{\sphinxupquote{ or }}\sphinxstyleliteralemphasis{\sphinxupquote{scalar float}}) \textendash{} Values of the radial coordinate to
map to vertical field. If \sphinxtitleref{R} and \sphinxtitleref{Z} are both scalar values,
they are used as the coordinate pair for all of the values in
\sphinxtitleref{t}. Must have the same shape as \sphinxtitleref{Z} unless the \sphinxtitleref{make\_grid}
keyword is set. If the \sphinxtitleref{make\_grid} keyword is True, \sphinxtitleref{R} must
have exactly one dimension.

\item {} 
\sphinxstyleliteralstrong{\sphinxupquote{Z}} (\sphinxstyleliteralemphasis{\sphinxupquote{Array-like}}\sphinxstyleliteralemphasis{\sphinxupquote{ or }}\sphinxstyleliteralemphasis{\sphinxupquote{scalar float}}) \textendash{} Values of the vertical coordinate to
map to vertical field. If \sphinxtitleref{R} and \sphinxtitleref{Z} are both scalar values,
they are used as the coordinate pair for all of the values in
\sphinxtitleref{t}. Must have the same shape as \sphinxtitleref{R} unless the \sphinxtitleref{make\_grid}
keyword is set. If the \sphinxtitleref{make\_grid} keyword is True, \sphinxtitleref{Z} must
have exactly one dimension.

\item {} 
\sphinxstyleliteralstrong{\sphinxupquote{t}} (\sphinxstyleliteralemphasis{\sphinxupquote{Array-like}}\sphinxstyleliteralemphasis{\sphinxupquote{ or }}\sphinxstyleliteralemphasis{\sphinxupquote{scalar float}}) \textendash{} Times to perform the conversion at.
If \sphinxtitleref{t} is a single value, it is used for all of the elements of
\sphinxtitleref{R}, \sphinxtitleref{Z}. If the \sphinxtitleref{each\_t} keyword is True, then \sphinxtitleref{t} must be
scalar or have exactly one dimension. If the \sphinxtitleref{each\_t} keyword is
False, \sphinxtitleref{t} must have the same shape as \sphinxtitleref{R} and \sphinxtitleref{Z} (or their
meshgrid if \sphinxtitleref{make\_grid} is True).

\end{itemize}

\item[{Keyword Arguments}] \leavevmode\begin{itemize}
\item {} 
\sphinxstyleliteralstrong{\sphinxupquote{each\_t}} (\sphinxstyleliteralemphasis{\sphinxupquote{Boolean}}) \textendash{} When True, the elements in \sphinxtitleref{R}, \sphinxtitleref{Z} are evaluated
at each value in \sphinxtitleref{t}. If True, \sphinxtitleref{t} must have only one dimension
(or be a scalar). If False, \sphinxtitleref{t} must match the shape of \sphinxtitleref{R} and
\sphinxtitleref{Z} or be a scalar. Default is True (evaluate ALL \sphinxtitleref{R}, \sphinxtitleref{Z} at
EACH element in \sphinxtitleref{t}).

\item {} 
\sphinxstyleliteralstrong{\sphinxupquote{make\_grid}} (\sphinxstyleliteralemphasis{\sphinxupquote{Boolean}}) \textendash{} Set to True to pass \sphinxtitleref{R} and \sphinxtitleref{Z} through
\sphinxcode{\sphinxupquote{scipy.meshgrid()}} before evaluating. If this is set to
True, \sphinxtitleref{R} and \sphinxtitleref{Z} must each only have a single dimension, but
can have different lengths. Default is False (do not form
meshgrid).

\item {} 
\sphinxstyleliteralstrong{\sphinxupquote{length\_unit}} (\sphinxstyleliteralemphasis{\sphinxupquote{String}}\sphinxstyleliteralemphasis{\sphinxupquote{ or }}\sphinxstyleliteralemphasis{\sphinxupquote{1}}) \textendash{} 
Length unit that \sphinxtitleref{R}, \sphinxtitleref{Z} are given in.
If a string is given, it must be a valid unit specifier:
\begin{quote}


\begin{savenotes}\sphinxattablestart
\centering
\begin{tabulary}{\linewidth}[t]{|T|T|}
\hline

’m’
&
meters
\\
\hline
’cm’
&
centimeters
\\
\hline
’mm’
&
millimeters
\\
\hline
’in’
&
inches
\\
\hline
’ft’
&
feet
\\
\hline
’yd’
&
yards
\\
\hline
’smoot’
&
smoots
\\
\hline
’cubit’
&
cubits
\\
\hline
’hand’
&
hands
\\
\hline
’default’
&
meters
\\
\hline
\end{tabulary}
\par
\sphinxattableend\end{savenotes}
\end{quote}

If length\_unit is 1 or None, meters are assumed. The default
value is 1 (use meters).


\item {} 
\sphinxstyleliteralstrong{\sphinxupquote{return\_t}} (\sphinxstyleliteralemphasis{\sphinxupquote{Boolean}}) \textendash{} Set to True to return a tuple of (\sphinxtitleref{BZ},
\sphinxtitleref{time\_idxs}), where \sphinxtitleref{time\_idxs} is the array of time indices
actually used in evaluating \sphinxtitleref{BZ} with nearest-neighbor
interpolation. (This is mostly present as an internal helper.)
Default is False (only return \sphinxtitleref{BZ}).

\end{itemize}

\item[{Returns}] \leavevmode

\sphinxtitleref{BZ} or (\sphinxtitleref{BZ}, \sphinxtitleref{time\_idxs})
\begin{itemize}
\item {} 
\sphinxstylestrong{BZ} (\sphinxtitleref{Array or scalar float}) - The vertical component of the
magnetic field. If all of the input arguments are scalar, then a
scalar is returned. Otherwise, a scipy Array is returned. If \sphinxtitleref{R}
and \sphinxtitleref{Z} both have the same shape then \sphinxtitleref{BZ} has this shape as well,
unless the \sphinxtitleref{make\_grid} keyword was True, in which case \sphinxtitleref{BZ} has
shape (len(\sphinxtitleref{Z}), len(\sphinxtitleref{R})).

\item {} 
\sphinxstylestrong{time\_idxs} (Array with same shape as \sphinxtitleref{BZ}) - The indices
(in \sphinxcode{\sphinxupquote{self.getTimeBase()}}) that were used for
nearest-neighbor interpolation. Only returned if \sphinxtitleref{return\_t} is
True.

\end{itemize}


\end{description}\end{quote}
\subsubsection*{Examples}

All assume that \sphinxtitleref{Eq\_instance} is a valid instance of the appropriate
extension of the {\hyperref[\detokenize{eqtools:eqtools.core.Equilibrium}]{\sphinxcrossref{\sphinxcode{\sphinxupquote{Equilibrium}}}}} abstract class.

Find single BZ value at R=0.6m, Z=0.0m, t=0.26s:

\begin{sphinxVerbatim}[commandchars=\\\{\}]
\PYG{n}{BZ\PYGZus{}val} \PYG{o}{=} \PYG{n}{Eq\PYGZus{}instance}\PYG{o}{.}\PYG{n}{rz2BZ}\PYG{p}{(}\PYG{l+m+mf}{0.6}\PYG{p}{,} \PYG{l+m+mi}{0}\PYG{p}{,} \PYG{l+m+mf}{0.26}\PYG{p}{)}
\end{sphinxVerbatim}

Find BZ values at (R, Z) points (0.6m, 0m) and (0.8m, 0m) at the
single time t=0.26s. Note that the \sphinxtitleref{Z} vector must be fully
specified, even if the values are all the same:

\begin{sphinxVerbatim}[commandchars=\\\{\}]
\PYG{n}{BZ\PYGZus{}arr} \PYG{o}{=} \PYG{n}{Eq\PYGZus{}instance}\PYG{o}{.}\PYG{n}{rz2BZ}\PYG{p}{(}\PYG{p}{[}\PYG{l+m+mf}{0.6}\PYG{p}{,} \PYG{l+m+mf}{0.8}\PYG{p}{]}\PYG{p}{,} \PYG{p}{[}\PYG{l+m+mi}{0}\PYG{p}{,} \PYG{l+m+mi}{0}\PYG{p}{]}\PYG{p}{,} \PYG{l+m+mf}{0.26}\PYG{p}{)}
\end{sphinxVerbatim}

Find BZ values at (R, Z) points (0.6m, 0m) at times t={[}0.2s, 0.3s{]}:

\begin{sphinxVerbatim}[commandchars=\\\{\}]
\PYG{n}{BZ\PYGZus{}arr} \PYG{o}{=} \PYG{n}{Eq\PYGZus{}instance}\PYG{o}{.}\PYG{n}{rz2BZ}\PYG{p}{(}\PYG{l+m+mf}{0.6}\PYG{p}{,} \PYG{l+m+mi}{0}\PYG{p}{,} \PYG{p}{[}\PYG{l+m+mf}{0.2}\PYG{p}{,} \PYG{l+m+mf}{0.3}\PYG{p}{]}\PYG{p}{)}
\end{sphinxVerbatim}

Find BZ values at (R, Z, t) points (0.6m, 0m, 0.2s) and
(0.5m, 0.2m, 0.3s):

\begin{sphinxVerbatim}[commandchars=\\\{\}]
\PYG{n}{BZ\PYGZus{}arr} \PYG{o}{=} \PYG{n}{Eq\PYGZus{}instance}\PYG{o}{.}\PYG{n}{rz2BZ}\PYG{p}{(}\PYG{p}{[}\PYG{l+m+mf}{0.6}\PYG{p}{,} \PYG{l+m+mf}{0.5}\PYG{p}{]}\PYG{p}{,} \PYG{p}{[}\PYG{l+m+mi}{0}\PYG{p}{,} \PYG{l+m+mf}{0.2}\PYG{p}{]}\PYG{p}{,} \PYG{p}{[}\PYG{l+m+mf}{0.2}\PYG{p}{,} \PYG{l+m+mf}{0.3}\PYG{p}{]}\PYG{p}{,} \PYG{n}{each\PYGZus{}t}\PYG{o}{=}\PYG{k+kc}{False}\PYG{p}{)}
\end{sphinxVerbatim}

Find BZ values on grid defined by 1D vector of radial positions \sphinxtitleref{R}
and 1D vector of vertical positions \sphinxtitleref{Z} at time t=0.2s:

\begin{sphinxVerbatim}[commandchars=\\\{\}]
\PYG{n}{BZ\PYGZus{}mat} \PYG{o}{=} \PYG{n}{Eq\PYGZus{}instance}\PYG{o}{.}\PYG{n}{rz2BZ}\PYG{p}{(}\PYG{n}{R}\PYG{p}{,} \PYG{n}{Z}\PYG{p}{,} \PYG{l+m+mf}{0.2}\PYG{p}{,} \PYG{n}{make\PYGZus{}grid}\PYG{o}{=}\PYG{k+kc}{True}\PYG{p}{)}
\end{sphinxVerbatim}

\end{fulllineitems}

\index{rz2BT() (eqtools.core.Equilibrium method)@\spxentry{rz2BT()}\spxextra{eqtools.core.Equilibrium method}}

\begin{fulllineitems}
\phantomsection\label{\detokenize{eqtools:eqtools.core.Equilibrium.rz2BT}}\pysiglinewithargsret{\sphinxbfcode{\sphinxupquote{rz2BT}}}{\emph{R}, \emph{Z}, \emph{t}, \emph{**kwargs}}{}
Calculates the toroidal component of the magnetic field at the given (R, Z, t).

Uses \(B_\phi = F / R\).

By default, EFIT only computes this inside the LCFS. To approximate the
field outside of the LCFS, \(B_\phi \approx B_{t, vac} R_0 / R\) is
used, where \(B_{t, vac}\) is obtained with {\hyperref[\detokenize{eqtools:eqtools.core.Equilibrium.getBtVac}]{\sphinxcrossref{\sphinxcode{\sphinxupquote{getBtVac()}}}}} and
\(R_0\) is the major radius of the magnetic axis obtained from
{\hyperref[\detokenize{eqtools:eqtools.core.Equilibrium.getMagR}]{\sphinxcrossref{\sphinxcode{\sphinxupquote{getMagR()}}}}}.

The coordinate system used is right-handed, such that “forward” field on
Alcator C-Mod (clockwise when seen from above) has negative BT.
\begin{quote}\begin{description}
\item[{Parameters}] \leavevmode\begin{itemize}
\item {} 
\sphinxstyleliteralstrong{\sphinxupquote{R}} (\sphinxstyleliteralemphasis{\sphinxupquote{Array-like}}\sphinxstyleliteralemphasis{\sphinxupquote{ or }}\sphinxstyleliteralemphasis{\sphinxupquote{scalar float}}) \textendash{} Values of the radial coordinate to
map to BT. If \sphinxtitleref{R} and \sphinxtitleref{Z} are both scalar values,
they are used as the coordinate pair for all of the values in
\sphinxtitleref{t}. Must have the same shape as \sphinxtitleref{Z} unless the \sphinxtitleref{make\_grid}
keyword is set. If the \sphinxtitleref{make\_grid} keyword is True, \sphinxtitleref{R} must
have exactly one dimension.

\item {} 
\sphinxstyleliteralstrong{\sphinxupquote{Z}} (\sphinxstyleliteralemphasis{\sphinxupquote{Array-like}}\sphinxstyleliteralemphasis{\sphinxupquote{ or }}\sphinxstyleliteralemphasis{\sphinxupquote{scalar float}}) \textendash{} Values of the vertical coordinate to
map to BT. If \sphinxtitleref{R} and \sphinxtitleref{Z} are both scalar values,
they are used as the coordinate pair for all of the values in
\sphinxtitleref{t}. Must have the same shape as \sphinxtitleref{R} unless the \sphinxtitleref{make\_grid}
keyword is set. If the \sphinxtitleref{make\_grid} keyword is True, \sphinxtitleref{Z} must
have exactly one dimension.

\item {} 
\sphinxstyleliteralstrong{\sphinxupquote{t}} (\sphinxstyleliteralemphasis{\sphinxupquote{Array-like}}\sphinxstyleliteralemphasis{\sphinxupquote{ or }}\sphinxstyleliteralemphasis{\sphinxupquote{scalar float}}) \textendash{} Times to perform the conversion at.
If \sphinxtitleref{t} is a single value, it is used for all of the elements of
\sphinxtitleref{R}, \sphinxtitleref{Z}. If the \sphinxtitleref{each\_t} keyword is True, then \sphinxtitleref{t} must be
scalar or have exactly one dimension. If the \sphinxtitleref{each\_t} keyword is
False, \sphinxtitleref{t} must have the same shape as \sphinxtitleref{R} and \sphinxtitleref{Z} (or their
meshgrid if \sphinxtitleref{make\_grid} is True).

\end{itemize}

\item[{Keyword Arguments}] \leavevmode\begin{itemize}
\item {} 
\sphinxstyleliteralstrong{\sphinxupquote{each\_t}} (\sphinxstyleliteralemphasis{\sphinxupquote{Boolean}}) \textendash{} When True, the elements in \sphinxtitleref{R}, \sphinxtitleref{Z} are evaluated
at each value in \sphinxtitleref{t}. If True, \sphinxtitleref{t} must have only one dimension
(or be a scalar). If False, \sphinxtitleref{t} must match the shape of \sphinxtitleref{R} and
\sphinxtitleref{Z} or be a scalar. Default is True (evaluate ALL \sphinxtitleref{R}, \sphinxtitleref{Z} at
EACH element in \sphinxtitleref{t}).

\item {} 
\sphinxstyleliteralstrong{\sphinxupquote{make\_grid}} (\sphinxstyleliteralemphasis{\sphinxupquote{Boolean}}) \textendash{} Set to True to pass \sphinxtitleref{R} and \sphinxtitleref{Z} through
\sphinxcode{\sphinxupquote{scipy.meshgrid()}} before evaluating. If this is set to
True, \sphinxtitleref{R} and \sphinxtitleref{Z} must each only have a single dimension, but
can have different lengths. Default is False (do not form
meshgrid).

\item {} 
\sphinxstyleliteralstrong{\sphinxupquote{length\_unit}} (\sphinxstyleliteralemphasis{\sphinxupquote{String}}\sphinxstyleliteralemphasis{\sphinxupquote{ or }}\sphinxstyleliteralemphasis{\sphinxupquote{1}}) \textendash{} 
Length unit that \sphinxtitleref{R}, \sphinxtitleref{Z} are given in.
If a string is given, it must be a valid unit specifier:
\begin{quote}


\begin{savenotes}\sphinxattablestart
\centering
\begin{tabulary}{\linewidth}[t]{|T|T|}
\hline

’m’
&
meters
\\
\hline
’cm’
&
centimeters
\\
\hline
’mm’
&
millimeters
\\
\hline
’in’
&
inches
\\
\hline
’ft’
&
feet
\\
\hline
’yd’
&
yards
\\
\hline
’smoot’
&
smoots
\\
\hline
’cubit’
&
cubits
\\
\hline
’hand’
&
hands
\\
\hline
’default’
&
meters
\\
\hline
\end{tabulary}
\par
\sphinxattableend\end{savenotes}
\end{quote}

If length\_unit is 1 or None, meters are assumed. The default
value is 1 (use meters).


\item {} 
\sphinxstyleliteralstrong{\sphinxupquote{return\_t}} (\sphinxstyleliteralemphasis{\sphinxupquote{Boolean}}) \textendash{} Set to True to return a tuple of (\sphinxtitleref{BT},
\sphinxtitleref{time\_idxs}), where \sphinxtitleref{time\_idxs} is the array of time indices
actually used in evaluating \sphinxtitleref{BT} with nearest-neighbor
interpolation. (This is mostly present as an internal helper.)
Default is False (only return \sphinxtitleref{BT}).

\end{itemize}

\item[{Returns}] \leavevmode

\sphinxtitleref{BT} or (\sphinxtitleref{BT}, \sphinxtitleref{time\_idxs})
\begin{itemize}
\item {} 
\sphinxstylestrong{BT} (\sphinxtitleref{Array or scalar float}) - The toroidal magnetic field.
If all of the input arguments are scalar, then a scalar is
returned. Otherwise, a scipy Array is returned. If \sphinxtitleref{R} and \sphinxtitleref{Z}
both have the same shape then \sphinxtitleref{BT} has this shape as well,
unless the \sphinxtitleref{make\_grid} keyword was True, in which case \sphinxtitleref{BT}
has shape (len(\sphinxtitleref{Z}), len(\sphinxtitleref{R})).

\item {} 
\sphinxstylestrong{time\_idxs} (Array with same shape as \sphinxtitleref{BT}) - The indices
(in \sphinxcode{\sphinxupquote{self.getTimeBase()}}) that were used for
nearest-neighbor interpolation. Only returned if \sphinxtitleref{return\_t} is
True.

\end{itemize}


\end{description}\end{quote}
\subsubsection*{Examples}

All assume that \sphinxtitleref{Eq\_instance} is a valid instance of the
appropriate extension of the {\hyperref[\detokenize{eqtools:eqtools.core.Equilibrium}]{\sphinxcrossref{\sphinxcode{\sphinxupquote{Equilibrium}}}}} abstract class.

Find single BT value at R=0.6m, Z=0.0m, t=0.26s:

\begin{sphinxVerbatim}[commandchars=\\\{\}]
\PYG{n}{BT\PYGZus{}val} \PYG{o}{=} \PYG{n}{Eq\PYGZus{}instance}\PYG{o}{.}\PYG{n}{rz2BT}\PYG{p}{(}\PYG{l+m+mf}{0.6}\PYG{p}{,} \PYG{l+m+mi}{0}\PYG{p}{,} \PYG{l+m+mf}{0.26}\PYG{p}{)}
\end{sphinxVerbatim}

Find BT values at (R, Z) points (0.6m, 0m) and (0.8m, 0m) at the
single time t=0.26s. Note that the \sphinxtitleref{Z} vector must be fully specified,
even if the values are all the same:

\begin{sphinxVerbatim}[commandchars=\\\{\}]
\PYG{n}{BT\PYGZus{}arr} \PYG{o}{=} \PYG{n}{Eq\PYGZus{}instance}\PYG{o}{.}\PYG{n}{rz2BT}\PYG{p}{(}\PYG{p}{[}\PYG{l+m+mf}{0.6}\PYG{p}{,} \PYG{l+m+mf}{0.8}\PYG{p}{]}\PYG{p}{,} \PYG{p}{[}\PYG{l+m+mi}{0}\PYG{p}{,} \PYG{l+m+mi}{0}\PYG{p}{]}\PYG{p}{,} \PYG{l+m+mf}{0.26}\PYG{p}{)}
\end{sphinxVerbatim}

Find BT values at (R, Z) points (0.6m, 0m) at times t={[}0.2s, 0.3s{]}:

\begin{sphinxVerbatim}[commandchars=\\\{\}]
\PYG{n}{BT\PYGZus{}arr} \PYG{o}{=} \PYG{n}{Eq\PYGZus{}instance}\PYG{o}{.}\PYG{n}{rz2BT}\PYG{p}{(}\PYG{l+m+mf}{0.6}\PYG{p}{,} \PYG{l+m+mi}{0}\PYG{p}{,} \PYG{p}{[}\PYG{l+m+mf}{0.2}\PYG{p}{,} \PYG{l+m+mf}{0.3}\PYG{p}{]}\PYG{p}{)}
\end{sphinxVerbatim}

Find BT values at (R, Z, t) points (0.6m, 0m, 0.2s) and (0.5m, 0.2m, 0.3s):

\begin{sphinxVerbatim}[commandchars=\\\{\}]
\PYG{n}{BT\PYGZus{}arr} \PYG{o}{=} \PYG{n}{Eq\PYGZus{}instance}\PYG{o}{.}\PYG{n}{rz2BT}\PYG{p}{(}\PYG{p}{[}\PYG{l+m+mf}{0.6}\PYG{p}{,} \PYG{l+m+mf}{0.5}\PYG{p}{]}\PYG{p}{,} \PYG{p}{[}\PYG{l+m+mi}{0}\PYG{p}{,} \PYG{l+m+mf}{0.2}\PYG{p}{]}\PYG{p}{,} \PYG{p}{[}\PYG{l+m+mf}{0.2}\PYG{p}{,} \PYG{l+m+mf}{0.3}\PYG{p}{]}\PYG{p}{,} \PYG{n}{each\PYGZus{}t}\PYG{o}{=}\PYG{k+kc}{False}\PYG{p}{)}
\end{sphinxVerbatim}

Find BT values on grid defined by 1D vector of radial positions \sphinxtitleref{R}
and 1D vector of vertical positions \sphinxtitleref{Z} at time t=0.2s:

\begin{sphinxVerbatim}[commandchars=\\\{\}]
\PYG{n}{BT\PYGZus{}mat} \PYG{o}{=} \PYG{n}{Eq\PYGZus{}instance}\PYG{o}{.}\PYG{n}{rz2BT}\PYG{p}{(}\PYG{n}{R}\PYG{p}{,} \PYG{n}{Z}\PYG{p}{,} \PYG{l+m+mf}{0.2}\PYG{p}{,} \PYG{n}{make\PYGZus{}grid}\PYG{o}{=}\PYG{k+kc}{True}\PYG{p}{)}
\end{sphinxVerbatim}

\end{fulllineitems}

\index{rz2B() (eqtools.core.Equilibrium method)@\spxentry{rz2B()}\spxextra{eqtools.core.Equilibrium method}}

\begin{fulllineitems}
\phantomsection\label{\detokenize{eqtools:eqtools.core.Equilibrium.rz2B}}\pysiglinewithargsret{\sphinxbfcode{\sphinxupquote{rz2B}}}{\emph{R}, \emph{Z}, \emph{t}, \emph{**kwargs}}{}
Calculates the magnitude of the magnetic field at the given (R, Z, t).
\begin{quote}\begin{description}
\item[{Parameters}] \leavevmode\begin{itemize}
\item {} 
\sphinxstyleliteralstrong{\sphinxupquote{R}} (\sphinxstyleliteralemphasis{\sphinxupquote{Array-like}}\sphinxstyleliteralemphasis{\sphinxupquote{ or }}\sphinxstyleliteralemphasis{\sphinxupquote{scalar float}}) \textendash{} Values of the radial coordinate to
map to B. If \sphinxtitleref{R} and \sphinxtitleref{Z} are both scalar values, they are used
as the coordinate pair for all of the values in \sphinxtitleref{t}. Must have
the same shape as \sphinxtitleref{Z} unless the \sphinxtitleref{make\_grid} keyword is set. If
the \sphinxtitleref{make\_grid} keyword is True, \sphinxtitleref{R} must have exactly one
dimension.

\item {} 
\sphinxstyleliteralstrong{\sphinxupquote{Z}} (\sphinxstyleliteralemphasis{\sphinxupquote{Array-like}}\sphinxstyleliteralemphasis{\sphinxupquote{ or }}\sphinxstyleliteralemphasis{\sphinxupquote{scalar float}}) \textendash{} Values of the vertical coordinate to
map to B. If \sphinxtitleref{R} and \sphinxtitleref{Z} are both scalar values, they are used
as the coordinate pair for all of the values in \sphinxtitleref{t}. Must have
the same shape as \sphinxtitleref{R} unless the \sphinxtitleref{make\_grid} keyword is set. If
the \sphinxtitleref{make\_grid} keyword is True, \sphinxtitleref{Z} must have exactly one
dimension.

\item {} 
\sphinxstyleliteralstrong{\sphinxupquote{t}} (\sphinxstyleliteralemphasis{\sphinxupquote{Array-like}}\sphinxstyleliteralemphasis{\sphinxupquote{ or }}\sphinxstyleliteralemphasis{\sphinxupquote{scalar float}}) \textendash{} Times to perform the conversion at.
If \sphinxtitleref{t} is a single value, it is used for all of the elements of
\sphinxtitleref{R}, \sphinxtitleref{Z}. If the \sphinxtitleref{each\_t} keyword is True, then \sphinxtitleref{t} must be
scalar or have exactly one dimension. If the \sphinxtitleref{each\_t} keyword is
False, \sphinxtitleref{t} must have the same shape as \sphinxtitleref{R} and \sphinxtitleref{Z} (or their
meshgrid if \sphinxtitleref{make\_grid} is True).

\end{itemize}

\item[{Keyword Arguments}] \leavevmode\begin{itemize}
\item {} 
\sphinxstyleliteralstrong{\sphinxupquote{each\_t}} (\sphinxstyleliteralemphasis{\sphinxupquote{Boolean}}) \textendash{} When True, the elements in \sphinxtitleref{R}, \sphinxtitleref{Z} are evaluated
at each value in \sphinxtitleref{t}. If True, \sphinxtitleref{t} must have only one dimension
(or be a scalar). If False, \sphinxtitleref{t} must match the shape of \sphinxtitleref{R} and
\sphinxtitleref{Z} or be a scalar. Default is True (evaluate ALL \sphinxtitleref{R}, \sphinxtitleref{Z} at
EACH element in \sphinxtitleref{t}).

\item {} 
\sphinxstyleliteralstrong{\sphinxupquote{make\_grid}} (\sphinxstyleliteralemphasis{\sphinxupquote{Boolean}}) \textendash{} Set to True to pass \sphinxtitleref{R} and \sphinxtitleref{Z} through
\sphinxcode{\sphinxupquote{scipy.meshgrid()}} before evaluating. If this is set to
True, \sphinxtitleref{R} and \sphinxtitleref{Z} must each only have a single dimension, but
can have different lengths. Default is False (do not form
meshgrid).

\item {} 
\sphinxstyleliteralstrong{\sphinxupquote{length\_unit}} (\sphinxstyleliteralemphasis{\sphinxupquote{String}}\sphinxstyleliteralemphasis{\sphinxupquote{ or }}\sphinxstyleliteralemphasis{\sphinxupquote{1}}) \textendash{} 
Length unit that \sphinxtitleref{R}, \sphinxtitleref{Z} are given in.
If a string is given, it must be a valid unit specifier:
\begin{quote}


\begin{savenotes}\sphinxattablestart
\centering
\begin{tabulary}{\linewidth}[t]{|T|T|}
\hline

’m’
&
meters
\\
\hline
’cm’
&
centimeters
\\
\hline
’mm’
&
millimeters
\\
\hline
’in’
&
inches
\\
\hline
’ft’
&
feet
\\
\hline
’yd’
&
yards
\\
\hline
’smoot’
&
smoots
\\
\hline
’cubit’
&
cubits
\\
\hline
’hand’
&
hands
\\
\hline
’default’
&
meters
\\
\hline
\end{tabulary}
\par
\sphinxattableend\end{savenotes}
\end{quote}

If length\_unit is 1 or None, meters are assumed. The default
value is 1 (use meters).


\item {} 
\sphinxstyleliteralstrong{\sphinxupquote{return\_t}} (\sphinxstyleliteralemphasis{\sphinxupquote{Boolean}}) \textendash{} Set to True to return a tuple of (\sphinxtitleref{B},
\sphinxtitleref{time\_idxs}), where \sphinxtitleref{time\_idxs} is the array of time indices
actually used in evaluating \sphinxtitleref{B} with nearest-neighbor
interpolation. (This is mostly present as an internal helper.)
Default is False (only return \sphinxtitleref{B}).

\end{itemize}

\item[{Returns}] \leavevmode

\sphinxtitleref{B} or (\sphinxtitleref{B}, \sphinxtitleref{time\_idxs})
\begin{itemize}
\item {} 
\sphinxstylestrong{B} (\sphinxtitleref{Array or scalar float}) - The magnitude of the magnetic
field. If all of the input arguments are scalar, then a scalar is
returned. Otherwise, a scipy Array is returned. If \sphinxtitleref{R} and \sphinxtitleref{Z}
both have the same shape then \sphinxtitleref{B} has this shape as well, unless
the \sphinxtitleref{make\_grid} keyword was True, in which case \sphinxtitleref{B} has shape
(len(\sphinxtitleref{Z}), len(\sphinxtitleref{R})).

\item {} 
\sphinxstylestrong{time\_idxs} (Array with same shape as \sphinxtitleref{B}) - The indices
(in \sphinxcode{\sphinxupquote{self.getTimeBase()}}) that were used for
nearest-neighbor interpolation. Only returned if \sphinxtitleref{return\_t} is
True.

\end{itemize}


\end{description}\end{quote}
\subsubsection*{Examples}

All assume that \sphinxtitleref{Eq\_instance} is a valid instance of the
appropriate extension of the {\hyperref[\detokenize{eqtools:eqtools.core.Equilibrium}]{\sphinxcrossref{\sphinxcode{\sphinxupquote{Equilibrium}}}}} abstract class.

Find single B value at R=0.6m, Z=0.0m, t=0.26s:

\begin{sphinxVerbatim}[commandchars=\\\{\}]
\PYG{n}{B\PYGZus{}val} \PYG{o}{=} \PYG{n}{Eq\PYGZus{}instance}\PYG{o}{.}\PYG{n}{rz2B}\PYG{p}{(}\PYG{l+m+mf}{0.6}\PYG{p}{,} \PYG{l+m+mi}{0}\PYG{p}{,} \PYG{l+m+mf}{0.26}\PYG{p}{)}
\end{sphinxVerbatim}

Find B values at (R, Z) points (0.6m, 0m) and (0.8m, 0m) at the
single time t=0.26s. Note that the \sphinxtitleref{Z} vector must be fully specified,
even if the values are all the same:

\begin{sphinxVerbatim}[commandchars=\\\{\}]
\PYG{n}{B\PYGZus{}arr} \PYG{o}{=} \PYG{n}{Eq\PYGZus{}instance}\PYG{o}{.}\PYG{n}{rz2B}\PYG{p}{(}\PYG{p}{[}\PYG{l+m+mf}{0.6}\PYG{p}{,} \PYG{l+m+mf}{0.8}\PYG{p}{]}\PYG{p}{,} \PYG{p}{[}\PYG{l+m+mi}{0}\PYG{p}{,} \PYG{l+m+mi}{0}\PYG{p}{]}\PYG{p}{,} \PYG{l+m+mf}{0.26}\PYG{p}{)}
\end{sphinxVerbatim}

Find B values at (R, Z) points (0.6m, 0m) at times t={[}0.2s, 0.3s{]}:

\begin{sphinxVerbatim}[commandchars=\\\{\}]
\PYG{n}{B\PYGZus{}arr} \PYG{o}{=} \PYG{n}{Eq\PYGZus{}instance}\PYG{o}{.}\PYG{n}{rz2B}\PYG{p}{(}\PYG{l+m+mf}{0.6}\PYG{p}{,} \PYG{l+m+mi}{0}\PYG{p}{,} \PYG{p}{[}\PYG{l+m+mf}{0.2}\PYG{p}{,} \PYG{l+m+mf}{0.3}\PYG{p}{]}\PYG{p}{)}
\end{sphinxVerbatim}

Find B values at (R, Z, t) points (0.6m, 0m, 0.2s) and (0.5m, 0.2m, 0.3s):

\begin{sphinxVerbatim}[commandchars=\\\{\}]
\PYG{n}{B\PYGZus{}arr} \PYG{o}{=} \PYG{n}{Eq\PYGZus{}instance}\PYG{o}{.}\PYG{n}{rz2B}\PYG{p}{(}\PYG{p}{[}\PYG{l+m+mf}{0.6}\PYG{p}{,} \PYG{l+m+mf}{0.5}\PYG{p}{]}\PYG{p}{,} \PYG{p}{[}\PYG{l+m+mi}{0}\PYG{p}{,} \PYG{l+m+mf}{0.2}\PYG{p}{]}\PYG{p}{,} \PYG{p}{[}\PYG{l+m+mf}{0.2}\PYG{p}{,} \PYG{l+m+mf}{0.3}\PYG{p}{]}\PYG{p}{,} \PYG{n}{each\PYGZus{}t}\PYG{o}{=}\PYG{k+kc}{False}\PYG{p}{)}
\end{sphinxVerbatim}

Find B values on grid defined by 1D vector of radial positions \sphinxtitleref{R}
and 1D vector of vertical positions \sphinxtitleref{Z} at time t=0.2s:

\begin{sphinxVerbatim}[commandchars=\\\{\}]
\PYG{n}{B\PYGZus{}mat} \PYG{o}{=} \PYG{n}{Eq\PYGZus{}instance}\PYG{o}{.}\PYG{n}{rz2B}\PYG{p}{(}\PYG{n}{R}\PYG{p}{,} \PYG{n}{Z}\PYG{p}{,} \PYG{l+m+mf}{0.2}\PYG{p}{,} \PYG{n}{make\PYGZus{}grid}\PYG{o}{=}\PYG{k+kc}{True}\PYG{p}{)}
\end{sphinxVerbatim}

\end{fulllineitems}

\index{rz2jR() (eqtools.core.Equilibrium method)@\spxentry{rz2jR()}\spxextra{eqtools.core.Equilibrium method}}

\begin{fulllineitems}
\phantomsection\label{\detokenize{eqtools:eqtools.core.Equilibrium.rz2jR}}\pysiglinewithargsret{\sphinxbfcode{\sphinxupquote{rz2jR}}}{\emph{R}, \emph{Z}, \emph{t}, \emph{**kwargs}}{}
Calculates the major radial component of the current density at the given (R, Z, t) coordinates.
\begin{equation*}
\begin{split}j_R = -\frac{1}{\mu_0 R}F'\frac{\partial \psi}{\partial Z} = \frac{F' B_R}{\mu_0}\end{split}
\end{equation*}\begin{quote}\begin{description}
\item[{Parameters}] \leavevmode\begin{itemize}
\item {} 
\sphinxstyleliteralstrong{\sphinxupquote{R}} (\sphinxstyleliteralemphasis{\sphinxupquote{Array-like}}\sphinxstyleliteralemphasis{\sphinxupquote{ or }}\sphinxstyleliteralemphasis{\sphinxupquote{scalar float}}) \textendash{} Values of the radial coordinate to
map to radial current density. If \sphinxtitleref{R} and \sphinxtitleref{Z} are both scalar
values, they are used as the coordinate pair for all of the
values in \sphinxtitleref{t}. Must have the same shape as \sphinxtitleref{Z} unless the
\sphinxtitleref{make\_grid} keyword is set. If the \sphinxtitleref{make\_grid} keyword is True,
\sphinxtitleref{R} must have exactly one dimension.

\item {} 
\sphinxstyleliteralstrong{\sphinxupquote{Z}} (\sphinxstyleliteralemphasis{\sphinxupquote{Array-like}}\sphinxstyleliteralemphasis{\sphinxupquote{ or }}\sphinxstyleliteralemphasis{\sphinxupquote{scalar float}}) \textendash{} Values of the vertical coordinate to
map to radial current density. If \sphinxtitleref{R} and \sphinxtitleref{Z} are both scalar
values, they are used as the coordinate pair for all of the
values in \sphinxtitleref{t}. Must have the same shape as \sphinxtitleref{R} unless the
\sphinxtitleref{make\_grid} keyword is set. If the \sphinxtitleref{make\_grid} keyword is True,
\sphinxtitleref{Z} must have exactly one dimension.

\item {} 
\sphinxstyleliteralstrong{\sphinxupquote{t}} (\sphinxstyleliteralemphasis{\sphinxupquote{Array-like}}\sphinxstyleliteralemphasis{\sphinxupquote{ or }}\sphinxstyleliteralemphasis{\sphinxupquote{scalar float}}) \textendash{} Times to perform the conversion at.
If \sphinxtitleref{t} is a single value, it is used for all of the elements of
\sphinxtitleref{R}, \sphinxtitleref{Z}. If the \sphinxtitleref{each\_t} keyword is True, then \sphinxtitleref{t} must be
scalar or have exactly one dimension. If the \sphinxtitleref{each\_t} keyword is
False, \sphinxtitleref{t} must have the same shape as \sphinxtitleref{R} and \sphinxtitleref{Z} (or their
meshgrid if \sphinxtitleref{make\_grid} is True).

\end{itemize}

\item[{Keyword Arguments}] \leavevmode\begin{itemize}
\item {} 
\sphinxstyleliteralstrong{\sphinxupquote{each\_t}} (\sphinxstyleliteralemphasis{\sphinxupquote{Boolean}}) \textendash{} When True, the elements in \sphinxtitleref{R}, \sphinxtitleref{Z} are evaluated
at each value in \sphinxtitleref{t}. If True, \sphinxtitleref{t} must have only one dimension
(or be a scalar). If False, \sphinxtitleref{t} must match the shape of \sphinxtitleref{R} and
\sphinxtitleref{Z} or be a scalar. Default is True (evaluate ALL \sphinxtitleref{R}, \sphinxtitleref{Z} at
EACH element in \sphinxtitleref{t}).

\item {} 
\sphinxstyleliteralstrong{\sphinxupquote{make\_grid}} (\sphinxstyleliteralemphasis{\sphinxupquote{Boolean}}) \textendash{} Set to True to pass \sphinxtitleref{R} and \sphinxtitleref{Z} through
\sphinxcode{\sphinxupquote{scipy.meshgrid()}} before evaluating. If this is set to
True, \sphinxtitleref{R} and \sphinxtitleref{Z} must each only have a single dimension, but
can have different lengths. Default is False (do not form
meshgrid).

\item {} 
\sphinxstyleliteralstrong{\sphinxupquote{length\_unit}} (\sphinxstyleliteralemphasis{\sphinxupquote{String}}\sphinxstyleliteralemphasis{\sphinxupquote{ or }}\sphinxstyleliteralemphasis{\sphinxupquote{1}}) \textendash{} 
Length unit that \sphinxtitleref{R}, \sphinxtitleref{Z} are given in.
If a string is given, it must be a valid unit specifier:
\begin{quote}


\begin{savenotes}\sphinxattablestart
\centering
\begin{tabulary}{\linewidth}[t]{|T|T|}
\hline

’m’
&
meters
\\
\hline
’cm’
&
centimeters
\\
\hline
’mm’
&
millimeters
\\
\hline
’in’
&
inches
\\
\hline
’ft’
&
feet
\\
\hline
’yd’
&
yards
\\
\hline
’smoot’
&
smoots
\\
\hline
’cubit’
&
cubits
\\
\hline
’hand’
&
hands
\\
\hline
’default’
&
meters
\\
\hline
\end{tabulary}
\par
\sphinxattableend\end{savenotes}
\end{quote}

If length\_unit is 1 or None, meters are assumed. The default
value is 1 (use meters).


\item {} 
\sphinxstyleliteralstrong{\sphinxupquote{return\_t}} (\sphinxstyleliteralemphasis{\sphinxupquote{Boolean}}) \textendash{} Set to True to return a tuple of (\sphinxtitleref{jR},
\sphinxtitleref{time\_idxs}), where \sphinxtitleref{time\_idxs} is the array of time indices
actually used in evaluating \sphinxtitleref{jR} with nearest-neighbor
interpolation. (This is mostly present as an internal helper.)
Default is False (only return \sphinxtitleref{jR}).

\end{itemize}

\item[{Returns}] \leavevmode

\sphinxtitleref{jR} or (\sphinxtitleref{jR}, \sphinxtitleref{time\_idxs})
\begin{itemize}
\item {} 
\sphinxstylestrong{jR} (\sphinxtitleref{Array or scalar float}) - The major radial component of
the current density. If all of the input arguments are scalar, then
a scalar is returned. Otherwise, a scipy Array is returned. If \sphinxtitleref{R}
and \sphinxtitleref{Z} both have the same shape then \sphinxtitleref{jR} has this shape as well,
unless the \sphinxtitleref{make\_grid} keyword was True, in which case \sphinxtitleref{jR} has
shape (len(\sphinxtitleref{Z}), len(\sphinxtitleref{R})).

\item {} 
\sphinxstylestrong{time\_idxs} (Array with same shape as \sphinxtitleref{jR}) - The indices
(in \sphinxcode{\sphinxupquote{self.getTimeBase()}}) that were used for
nearest-neighbor interpolation. Only returned if \sphinxtitleref{return\_t} is
True.

\end{itemize}


\end{description}\end{quote}
\subsubsection*{Examples}

All assume that \sphinxtitleref{Eq\_instance} is a valid instance of the appropriate
extension of the {\hyperref[\detokenize{eqtools:eqtools.core.Equilibrium}]{\sphinxcrossref{\sphinxcode{\sphinxupquote{Equilibrium}}}}} abstract class.

Find single jR value at R=0.6m, Z=0.0m, t=0.26s:

\begin{sphinxVerbatim}[commandchars=\\\{\}]
\PYG{n}{jR\PYGZus{}val} \PYG{o}{=} \PYG{n}{Eq\PYGZus{}instance}\PYG{o}{.}\PYG{n}{rz2jR}\PYG{p}{(}\PYG{l+m+mf}{0.6}\PYG{p}{,} \PYG{l+m+mi}{0}\PYG{p}{,} \PYG{l+m+mf}{0.26}\PYG{p}{)}
\end{sphinxVerbatim}

Find jR values at (R, Z) points (0.6m, 0m) and (0.8m, 0m) at the
single time t=0.26s. Note that the \sphinxtitleref{Z} vector must be fully
specified, even if the values are all the same:

\begin{sphinxVerbatim}[commandchars=\\\{\}]
\PYG{n}{jR\PYGZus{}arr} \PYG{o}{=} \PYG{n}{Eq\PYGZus{}instance}\PYG{o}{.}\PYG{n}{rz2jR}\PYG{p}{(}\PYG{p}{[}\PYG{l+m+mf}{0.6}\PYG{p}{,} \PYG{l+m+mf}{0.8}\PYG{p}{]}\PYG{p}{,} \PYG{p}{[}\PYG{l+m+mi}{0}\PYG{p}{,} \PYG{l+m+mi}{0}\PYG{p}{]}\PYG{p}{,} \PYG{l+m+mf}{0.26}\PYG{p}{)}
\end{sphinxVerbatim}

Find jR values at (R, Z) points (0.6m, 0m) at times t={[}0.2s, 0.3s{]}:

\begin{sphinxVerbatim}[commandchars=\\\{\}]
\PYG{n}{jR\PYGZus{}arr} \PYG{o}{=} \PYG{n}{Eq\PYGZus{}instance}\PYG{o}{.}\PYG{n}{rz2jR}\PYG{p}{(}\PYG{l+m+mf}{0.6}\PYG{p}{,} \PYG{l+m+mi}{0}\PYG{p}{,} \PYG{p}{[}\PYG{l+m+mf}{0.2}\PYG{p}{,} \PYG{l+m+mf}{0.3}\PYG{p}{]}\PYG{p}{)}
\end{sphinxVerbatim}

Find jR values at (R, Z, t) points (0.6m, 0m, 0.2s) and
(0.5m, 0.2m, 0.3s):

\begin{sphinxVerbatim}[commandchars=\\\{\}]
\PYG{n}{jR\PYGZus{}arr} \PYG{o}{=} \PYG{n}{Eq\PYGZus{}instance}\PYG{o}{.}\PYG{n}{rz2jR}\PYG{p}{(}\PYG{p}{[}\PYG{l+m+mf}{0.6}\PYG{p}{,} \PYG{l+m+mf}{0.5}\PYG{p}{]}\PYG{p}{,} \PYG{p}{[}\PYG{l+m+mi}{0}\PYG{p}{,} \PYG{l+m+mf}{0.2}\PYG{p}{]}\PYG{p}{,} \PYG{p}{[}\PYG{l+m+mf}{0.2}\PYG{p}{,} \PYG{l+m+mf}{0.3}\PYG{p}{]}\PYG{p}{,} \PYG{n}{each\PYGZus{}t}\PYG{o}{=}\PYG{k+kc}{False}\PYG{p}{)}
\end{sphinxVerbatim}

Find jR values on grid defined by 1D vector of radial positions \sphinxtitleref{R}
and 1D vector of vertical positions \sphinxtitleref{Z} at time t=0.2s:

\begin{sphinxVerbatim}[commandchars=\\\{\}]
\PYG{n}{jR\PYGZus{}mat} \PYG{o}{=} \PYG{n}{Eq\PYGZus{}instance}\PYG{o}{.}\PYG{n}{rz2jR}\PYG{p}{(}\PYG{n}{R}\PYG{p}{,} \PYG{n}{Z}\PYG{p}{,} \PYG{l+m+mf}{0.2}\PYG{p}{,} \PYG{n}{make\PYGZus{}grid}\PYG{o}{=}\PYG{k+kc}{True}\PYG{p}{)}
\end{sphinxVerbatim}

\end{fulllineitems}

\index{rz2jZ() (eqtools.core.Equilibrium method)@\spxentry{rz2jZ()}\spxextra{eqtools.core.Equilibrium method}}

\begin{fulllineitems}
\phantomsection\label{\detokenize{eqtools:eqtools.core.Equilibrium.rz2jZ}}\pysiglinewithargsret{\sphinxbfcode{\sphinxupquote{rz2jZ}}}{\emph{R}, \emph{Z}, \emph{t}, \emph{**kwargs}}{}
Calculates the vertical component of the current density at the given (R, Z, t) coordinates.

Uses
\begin{equation*}
\begin{split}j_Z = \frac{1}{\mu_0 R}F'\frac{\partial \psi}{\partial R} = \frac{F' B_Z}{\mu_0}\end{split}
\end{equation*}
Note that this function includes a factor of -1 to correct the FF’ from
Alcator C-Mod’s EFIT implementation. You should check the sign of your
data.
\begin{quote}\begin{description}
\item[{Parameters}] \leavevmode\begin{itemize}
\item {} 
\sphinxstyleliteralstrong{\sphinxupquote{R}} (\sphinxstyleliteralemphasis{\sphinxupquote{Array-like}}\sphinxstyleliteralemphasis{\sphinxupquote{ or }}\sphinxstyleliteralemphasis{\sphinxupquote{scalar float}}) \textendash{} Values of the radial coordinate to
map to vertical current density. If \sphinxtitleref{R} and \sphinxtitleref{Z} are both scalar
values, they are used as the coordinate pair for all of the
values in \sphinxtitleref{t}. Must have the same shape as \sphinxtitleref{Z} unless the
\sphinxtitleref{make\_grid} keyword is set. If the \sphinxtitleref{make\_grid} keyword is True,
\sphinxtitleref{R} must have exactly one dimension.

\item {} 
\sphinxstyleliteralstrong{\sphinxupquote{Z}} (\sphinxstyleliteralemphasis{\sphinxupquote{Array-like}}\sphinxstyleliteralemphasis{\sphinxupquote{ or }}\sphinxstyleliteralemphasis{\sphinxupquote{scalar float}}) \textendash{} Values of the vertical coordinate to
map to vertical current density. If \sphinxtitleref{R} and \sphinxtitleref{Z} are both scalar
values, they are used as the coordinate pair for all of the
values in \sphinxtitleref{t}. Must have the same shape as \sphinxtitleref{R} unless the
\sphinxtitleref{make\_grid} keyword is set. If the \sphinxtitleref{make\_grid} keyword is True,
\sphinxtitleref{Z} must have exactly one dimension.

\item {} 
\sphinxstyleliteralstrong{\sphinxupquote{t}} (\sphinxstyleliteralemphasis{\sphinxupquote{Array-like}}\sphinxstyleliteralemphasis{\sphinxupquote{ or }}\sphinxstyleliteralemphasis{\sphinxupquote{scalar float}}) \textendash{} Times to perform the conversion at.
If \sphinxtitleref{t} is a single value, it is used for all of the elements of
\sphinxtitleref{R}, \sphinxtitleref{Z}. If the \sphinxtitleref{each\_t} keyword is True, then \sphinxtitleref{t} must be
scalar or have exactly one dimension. If the \sphinxtitleref{each\_t} keyword is
False, \sphinxtitleref{t} must have the same shape as \sphinxtitleref{R} and \sphinxtitleref{Z} (or their
meshgrid if \sphinxtitleref{make\_grid} is True).

\end{itemize}

\item[{Keyword Arguments}] \leavevmode\begin{itemize}
\item {} 
\sphinxstyleliteralstrong{\sphinxupquote{each\_t}} (\sphinxstyleliteralemphasis{\sphinxupquote{Boolean}}) \textendash{} When True, the elements in \sphinxtitleref{R}, \sphinxtitleref{Z} are evaluated
at each value in \sphinxtitleref{t}. If True, \sphinxtitleref{t} must have only one dimension
(or be a scalar). If False, \sphinxtitleref{t} must match the shape of \sphinxtitleref{R} and
\sphinxtitleref{Z} or be a scalar. Default is True (evaluate ALL \sphinxtitleref{R}, \sphinxtitleref{Z} at
EACH element in \sphinxtitleref{t}).

\item {} 
\sphinxstyleliteralstrong{\sphinxupquote{make\_grid}} (\sphinxstyleliteralemphasis{\sphinxupquote{Boolean}}) \textendash{} Set to True to pass \sphinxtitleref{R} and \sphinxtitleref{Z} through
\sphinxcode{\sphinxupquote{scipy.meshgrid()}} before evaluating. If this is set to
True, \sphinxtitleref{R} and \sphinxtitleref{Z} must each only have a single dimension, but
can have different lengths. Default is False (do not form
meshgrid).

\item {} 
\sphinxstyleliteralstrong{\sphinxupquote{length\_unit}} (\sphinxstyleliteralemphasis{\sphinxupquote{String}}\sphinxstyleliteralemphasis{\sphinxupquote{ or }}\sphinxstyleliteralemphasis{\sphinxupquote{1}}) \textendash{} 
Length unit that \sphinxtitleref{R}, \sphinxtitleref{Z} are given in.
If a string is given, it must be a valid unit specifier:
\begin{quote}


\begin{savenotes}\sphinxattablestart
\centering
\begin{tabulary}{\linewidth}[t]{|T|T|}
\hline

’m’
&
meters
\\
\hline
’cm’
&
centimeters
\\
\hline
’mm’
&
millimeters
\\
\hline
’in’
&
inches
\\
\hline
’ft’
&
feet
\\
\hline
’yd’
&
yards
\\
\hline
’smoot’
&
smoots
\\
\hline
’cubit’
&
cubits
\\
\hline
’hand’
&
hands
\\
\hline
’default’
&
meters
\\
\hline
\end{tabulary}
\par
\sphinxattableend\end{savenotes}
\end{quote}

If length\_unit is 1 or None, meters are assumed. The default
value is 1 (use meters).


\item {} 
\sphinxstyleliteralstrong{\sphinxupquote{return\_t}} (\sphinxstyleliteralemphasis{\sphinxupquote{Boolean}}) \textendash{} Set to True to return a tuple of (\sphinxtitleref{jZ},
\sphinxtitleref{time\_idxs}), where \sphinxtitleref{time\_idxs} is the array of time indices
actually used in evaluating \sphinxtitleref{jZ} with nearest-neighbor
interpolation. (This is mostly present as an internal helper.)
Default is False (only return \sphinxtitleref{jZ}).

\end{itemize}

\item[{Returns}] \leavevmode

\sphinxtitleref{jZ} or (\sphinxtitleref{jZ}, \sphinxtitleref{time\_idxs})
\begin{itemize}
\item {} 
\sphinxstylestrong{jZ} (\sphinxtitleref{Array or scalar float}) - The vertical component of the
current density. If all of the input arguments are scalar, then a
scalar is returned. Otherwise, a scipy Array is returned. If \sphinxtitleref{R}
and \sphinxtitleref{Z} both have the same shape then \sphinxtitleref{jZ} has this shape as well,
unless the \sphinxtitleref{make\_grid} keyword was True, in which case \sphinxtitleref{jZ} has
shape (len(\sphinxtitleref{Z}), len(\sphinxtitleref{R})).

\item {} 
\sphinxstylestrong{time\_idxs} (Array with same shape as \sphinxtitleref{jZ}) - The indices
(in \sphinxcode{\sphinxupquote{self.getTimeBase()}}) that were used for
nearest-neighbor interpolation. Only returned if \sphinxtitleref{return\_t} is
True.

\end{itemize}


\end{description}\end{quote}
\subsubsection*{Examples}

All assume that \sphinxtitleref{Eq\_instance} is a valid instance of the appropriate
extension of the {\hyperref[\detokenize{eqtools:eqtools.core.Equilibrium}]{\sphinxcrossref{\sphinxcode{\sphinxupquote{Equilibrium}}}}} abstract class.

Find single jZ value at R=0.6m, Z=0.0m, t=0.26s:

\begin{sphinxVerbatim}[commandchars=\\\{\}]
\PYG{n}{jZ\PYGZus{}val} \PYG{o}{=} \PYG{n}{Eq\PYGZus{}instance}\PYG{o}{.}\PYG{n}{rz2jZ}\PYG{p}{(}\PYG{l+m+mf}{0.6}\PYG{p}{,} \PYG{l+m+mi}{0}\PYG{p}{,} \PYG{l+m+mf}{0.26}\PYG{p}{)}
\end{sphinxVerbatim}

Find jZ values at (R, Z) points (0.6m, 0m) and (0.8m, 0m) at the
single time t=0.26s. Note that the \sphinxtitleref{Z} vector must be fully
specified, even if the values are all the same:

\begin{sphinxVerbatim}[commandchars=\\\{\}]
\PYG{n}{jZ\PYGZus{}arr} \PYG{o}{=} \PYG{n}{Eq\PYGZus{}instance}\PYG{o}{.}\PYG{n}{rz2jZ}\PYG{p}{(}\PYG{p}{[}\PYG{l+m+mf}{0.6}\PYG{p}{,} \PYG{l+m+mf}{0.8}\PYG{p}{]}\PYG{p}{,} \PYG{p}{[}\PYG{l+m+mi}{0}\PYG{p}{,} \PYG{l+m+mi}{0}\PYG{p}{]}\PYG{p}{,} \PYG{l+m+mf}{0.26}\PYG{p}{)}
\end{sphinxVerbatim}

Find jZ values at (R, Z) points (0.6m, 0m) at times t={[}0.2s, 0.3s{]}:

\begin{sphinxVerbatim}[commandchars=\\\{\}]
\PYG{n}{jZ\PYGZus{}arr} \PYG{o}{=} \PYG{n}{Eq\PYGZus{}instance}\PYG{o}{.}\PYG{n}{rz2jZ}\PYG{p}{(}\PYG{l+m+mf}{0.6}\PYG{p}{,} \PYG{l+m+mi}{0}\PYG{p}{,} \PYG{p}{[}\PYG{l+m+mf}{0.2}\PYG{p}{,} \PYG{l+m+mf}{0.3}\PYG{p}{]}\PYG{p}{)}
\end{sphinxVerbatim}

Find jZ values at (R, Z, t) points (0.6m, 0m, 0.2s) and
(0.5m, 0.2m, 0.3s):

\begin{sphinxVerbatim}[commandchars=\\\{\}]
\PYG{n}{jZ\PYGZus{}arr} \PYG{o}{=} \PYG{n}{Eq\PYGZus{}instance}\PYG{o}{.}\PYG{n}{rz2jZ}\PYG{p}{(}\PYG{p}{[}\PYG{l+m+mf}{0.6}\PYG{p}{,} \PYG{l+m+mf}{0.5}\PYG{p}{]}\PYG{p}{,} \PYG{p}{[}\PYG{l+m+mi}{0}\PYG{p}{,} \PYG{l+m+mf}{0.2}\PYG{p}{]}\PYG{p}{,} \PYG{p}{[}\PYG{l+m+mf}{0.2}\PYG{p}{,} \PYG{l+m+mf}{0.3}\PYG{p}{]}\PYG{p}{,} \PYG{n}{each\PYGZus{}t}\PYG{o}{=}\PYG{k+kc}{False}\PYG{p}{)}
\end{sphinxVerbatim}

Find jZ values on grid defined by 1D vector of radial positions \sphinxtitleref{R}
and 1D vector of vertical positions \sphinxtitleref{Z} at time t=0.2s:

\begin{sphinxVerbatim}[commandchars=\\\{\}]
\PYG{n}{jZ\PYGZus{}mat} \PYG{o}{=} \PYG{n}{Eq\PYGZus{}instance}\PYG{o}{.}\PYG{n}{rz2jZ}\PYG{p}{(}\PYG{n}{R}\PYG{p}{,} \PYG{n}{Z}\PYG{p}{,} \PYG{l+m+mf}{0.2}\PYG{p}{,} \PYG{n}{make\PYGZus{}grid}\PYG{o}{=}\PYG{k+kc}{True}\PYG{p}{)}
\end{sphinxVerbatim}

\end{fulllineitems}

\index{rz2jT() (eqtools.core.Equilibrium method)@\spxentry{rz2jT()}\spxextra{eqtools.core.Equilibrium method}}

\begin{fulllineitems}
\phantomsection\label{\detokenize{eqtools:eqtools.core.Equilibrium.rz2jT}}\pysiglinewithargsret{\sphinxbfcode{\sphinxupquote{rz2jT}}}{\emph{R}, \emph{Z}, \emph{t}, \emph{**kwargs}}{}
Calculates the toroidal component of the current density at the given (R, Z, t) coordinates.

Uses
\begin{equation*}
\begin{split}j_\phi = Rp' + \frac{FF'}{\mu_0 R}\end{split}
\end{equation*}
The coordinate system used is right-handed, such that “forward” field on
Alcator C-Mod (clockwise when seen from above) has negative jT.
\begin{quote}\begin{description}
\item[{Parameters}] \leavevmode\begin{itemize}
\item {} 
\sphinxstyleliteralstrong{\sphinxupquote{R}} (\sphinxstyleliteralemphasis{\sphinxupquote{Array-like}}\sphinxstyleliteralemphasis{\sphinxupquote{ or }}\sphinxstyleliteralemphasis{\sphinxupquote{scalar float}}) \textendash{} Values of the radial coordinate to
map to toroidal current density. If \sphinxtitleref{R} and \sphinxtitleref{Z} are both scalar
values, they are used as the coordinate pair for all of the
values in \sphinxtitleref{t}. Must have the same shape as \sphinxtitleref{Z} unless the
\sphinxtitleref{make\_grid} keyword is set. If the \sphinxtitleref{make\_grid} keyword is True,
\sphinxtitleref{R} must have exactly one dimension.

\item {} 
\sphinxstyleliteralstrong{\sphinxupquote{Z}} (\sphinxstyleliteralemphasis{\sphinxupquote{Array-like}}\sphinxstyleliteralemphasis{\sphinxupquote{ or }}\sphinxstyleliteralemphasis{\sphinxupquote{scalar float}}) \textendash{} Values of the vertical coordinate to
map to toroidal current density. If \sphinxtitleref{R} and \sphinxtitleref{Z} are both scalar
values, they are used as the coordinate pair for all of the
values in \sphinxtitleref{t}. Must have the same shape as \sphinxtitleref{R} unless the
\sphinxtitleref{make\_grid} keyword is set. If the \sphinxtitleref{make\_grid} keyword is True,
\sphinxtitleref{Z} must have exactly one dimension.

\item {} 
\sphinxstyleliteralstrong{\sphinxupquote{t}} (\sphinxstyleliteralemphasis{\sphinxupquote{Array-like}}\sphinxstyleliteralemphasis{\sphinxupquote{ or }}\sphinxstyleliteralemphasis{\sphinxupquote{scalar float}}) \textendash{} Times to perform the conversion at.
If \sphinxtitleref{t} is a single value, it is used for all of the elements of
\sphinxtitleref{R}, \sphinxtitleref{Z}. If the \sphinxtitleref{each\_t} keyword is True, then \sphinxtitleref{t} must be
scalar or have exactly one dimension. If the \sphinxtitleref{each\_t} keyword is
False, \sphinxtitleref{t} must have the same shape as \sphinxtitleref{R} and \sphinxtitleref{Z} (or their
meshgrid if \sphinxtitleref{make\_grid} is True).

\end{itemize}

\item[{Keyword Arguments}] \leavevmode\begin{itemize}
\item {} 
\sphinxstyleliteralstrong{\sphinxupquote{each\_t}} (\sphinxstyleliteralemphasis{\sphinxupquote{Boolean}}) \textendash{} When True, the elements in \sphinxtitleref{R}, \sphinxtitleref{Z} are evaluated
at each value in \sphinxtitleref{t}. If True, \sphinxtitleref{t} must have only one dimension
(or be a scalar). If False, \sphinxtitleref{t} must match the shape of \sphinxtitleref{R} and
\sphinxtitleref{Z} or be a scalar. Default is True (evaluate ALL \sphinxtitleref{R}, \sphinxtitleref{Z} at
EACH element in \sphinxtitleref{t}).

\item {} 
\sphinxstyleliteralstrong{\sphinxupquote{make\_grid}} (\sphinxstyleliteralemphasis{\sphinxupquote{Boolean}}) \textendash{} Set to True to pass \sphinxtitleref{R} and \sphinxtitleref{Z} through
\sphinxcode{\sphinxupquote{scipy.meshgrid()}} before evaluating. If this is set to
True, \sphinxtitleref{R} and \sphinxtitleref{Z} must each only have a single dimension, but
can have different lengths. Default is False (do not form
meshgrid).

\item {} 
\sphinxstyleliteralstrong{\sphinxupquote{length\_unit}} (\sphinxstyleliteralemphasis{\sphinxupquote{String}}\sphinxstyleliteralemphasis{\sphinxupquote{ or }}\sphinxstyleliteralemphasis{\sphinxupquote{1}}) \textendash{} 
Length unit that \sphinxtitleref{R}, \sphinxtitleref{Z} are given in.
If a string is given, it must be a valid unit specifier:
\begin{quote}


\begin{savenotes}\sphinxattablestart
\centering
\begin{tabulary}{\linewidth}[t]{|T|T|}
\hline

’m’
&
meters
\\
\hline
’cm’
&
centimeters
\\
\hline
’mm’
&
millimeters
\\
\hline
’in’
&
inches
\\
\hline
’ft’
&
feet
\\
\hline
’yd’
&
yards
\\
\hline
’smoot’
&
smoots
\\
\hline
’cubit’
&
cubits
\\
\hline
’hand’
&
hands
\\
\hline
’default’
&
meters
\\
\hline
\end{tabulary}
\par
\sphinxattableend\end{savenotes}
\end{quote}

If length\_unit is 1 or None, meters are assumed. The default
value is 1 (use meters).


\item {} 
\sphinxstyleliteralstrong{\sphinxupquote{return\_t}} (\sphinxstyleliteralemphasis{\sphinxupquote{Boolean}}) \textendash{} Set to True to return a tuple of (\sphinxtitleref{jT},
\sphinxtitleref{time\_idxs}), where \sphinxtitleref{time\_idxs} is the array of time indices
actually used in evaluating \sphinxtitleref{jT} with nearest-neighbor
interpolation. (This is mostly present as an internal helper.)
Default is False (only return \sphinxtitleref{jT}).

\end{itemize}

\item[{Returns}] \leavevmode

\sphinxtitleref{jT} or (\sphinxtitleref{jT}, \sphinxtitleref{time\_idxs})
\begin{itemize}
\item {} 
\sphinxstylestrong{jT} (\sphinxtitleref{Array or scalar float}) - The major radial component of
the current density. If all of the input arguments are scalar,
then a scalar is returned. Otherwise, a scipy Array is returned.
If \sphinxtitleref{R} and \sphinxtitleref{Z} both have the same shape then \sphinxtitleref{jT} has this shape
as well, unless the \sphinxtitleref{make\_grid} keyword was True, in which case
\sphinxtitleref{jT} has shape (len(\sphinxtitleref{Z}), len(\sphinxtitleref{R})).

\item {} 
\sphinxstylestrong{time\_idxs} (Array with same shape as \sphinxtitleref{jT}) - The indices
(in \sphinxcode{\sphinxupquote{self.getTimeBase()}}) that were used for
nearest-neighbor interpolation. Only returned if \sphinxtitleref{return\_t} is
True.

\end{itemize}


\end{description}\end{quote}
\subsubsection*{Examples}

All assume that \sphinxtitleref{Eq\_instance} is a valid instance of the appropriate
extension of the {\hyperref[\detokenize{eqtools:eqtools.core.Equilibrium}]{\sphinxcrossref{\sphinxcode{\sphinxupquote{Equilibrium}}}}} abstract class.

Find single jT value at R=0.6m, Z=0.0m, t=0.26s:

\begin{sphinxVerbatim}[commandchars=\\\{\}]
\PYG{n}{jT\PYGZus{}val} \PYG{o}{=} \PYG{n}{Eq\PYGZus{}instance}\PYG{o}{.}\PYG{n}{rz2jT}\PYG{p}{(}\PYG{l+m+mf}{0.6}\PYG{p}{,} \PYG{l+m+mi}{0}\PYG{p}{,} \PYG{l+m+mf}{0.26}\PYG{p}{)}
\end{sphinxVerbatim}

Find jT values at (R, Z) points (0.6m, 0m) and (0.8m, 0m) at the
single time t=0.26s. Note that the \sphinxtitleref{Z} vector must be fully
specified, even if the values are all the same:

\begin{sphinxVerbatim}[commandchars=\\\{\}]
\PYG{n}{jT\PYGZus{}arr} \PYG{o}{=} \PYG{n}{Eq\PYGZus{}instance}\PYG{o}{.}\PYG{n}{rz2jT}\PYG{p}{(}\PYG{p}{[}\PYG{l+m+mf}{0.6}\PYG{p}{,} \PYG{l+m+mf}{0.8}\PYG{p}{]}\PYG{p}{,} \PYG{p}{[}\PYG{l+m+mi}{0}\PYG{p}{,} \PYG{l+m+mi}{0}\PYG{p}{]}\PYG{p}{,} \PYG{l+m+mf}{0.26}\PYG{p}{)}
\end{sphinxVerbatim}

Find jT values at (R, Z) points (0.6m, 0m) at times t={[}0.2s, 0.3s{]}:

\begin{sphinxVerbatim}[commandchars=\\\{\}]
\PYG{n}{jT\PYGZus{}arr} \PYG{o}{=} \PYG{n}{Eq\PYGZus{}instance}\PYG{o}{.}\PYG{n}{rz2jT}\PYG{p}{(}\PYG{l+m+mf}{0.6}\PYG{p}{,} \PYG{l+m+mi}{0}\PYG{p}{,} \PYG{p}{[}\PYG{l+m+mf}{0.2}\PYG{p}{,} \PYG{l+m+mf}{0.3}\PYG{p}{]}\PYG{p}{)}
\end{sphinxVerbatim}

Find jT values at (R, Z, t) points (0.6m, 0m, 0.2s) and
(0.5m, 0.2m, 0.3s):

\begin{sphinxVerbatim}[commandchars=\\\{\}]
\PYG{n}{jT\PYGZus{}arr} \PYG{o}{=} \PYG{n}{Eq\PYGZus{}instance}\PYG{o}{.}\PYG{n}{rz2jT}\PYG{p}{(}\PYG{p}{[}\PYG{l+m+mf}{0.6}\PYG{p}{,} \PYG{l+m+mf}{0.5}\PYG{p}{]}\PYG{p}{,} \PYG{p}{[}\PYG{l+m+mi}{0}\PYG{p}{,} \PYG{l+m+mf}{0.2}\PYG{p}{]}\PYG{p}{,} \PYG{p}{[}\PYG{l+m+mf}{0.2}\PYG{p}{,} \PYG{l+m+mf}{0.3}\PYG{p}{]}\PYG{p}{,} \PYG{n}{each\PYGZus{}t}\PYG{o}{=}\PYG{k+kc}{False}\PYG{p}{)}
\end{sphinxVerbatim}

Find jT values on grid defined by 1D vector of radial positions \sphinxtitleref{R}
and 1D vector of vertical positions \sphinxtitleref{Z} at time t=0.2s:

\begin{sphinxVerbatim}[commandchars=\\\{\}]
\PYG{n}{jT\PYGZus{}mat} \PYG{o}{=} \PYG{n}{Eq\PYGZus{}instance}\PYG{o}{.}\PYG{n}{rz2jT}\PYG{p}{(}\PYG{n}{R}\PYG{p}{,} \PYG{n}{Z}\PYG{p}{,} \PYG{l+m+mf}{0.2}\PYG{p}{,} \PYG{n}{make\PYGZus{}grid}\PYG{o}{=}\PYG{k+kc}{True}\PYG{p}{)}
\end{sphinxVerbatim}

\end{fulllineitems}

\index{rz2j() (eqtools.core.Equilibrium method)@\spxentry{rz2j()}\spxextra{eqtools.core.Equilibrium method}}

\begin{fulllineitems}
\phantomsection\label{\detokenize{eqtools:eqtools.core.Equilibrium.rz2j}}\pysiglinewithargsret{\sphinxbfcode{\sphinxupquote{rz2j}}}{\emph{R}, \emph{Z}, \emph{t}, \emph{**kwargs}}{}
Calculates the magnitude of the current density at the given (R, Z, t) coordinates.
\begin{quote}\begin{description}
\item[{Parameters}] \leavevmode\begin{itemize}
\item {} 
\sphinxstyleliteralstrong{\sphinxupquote{R}} (\sphinxstyleliteralemphasis{\sphinxupquote{Array-like}}\sphinxstyleliteralemphasis{\sphinxupquote{ or }}\sphinxstyleliteralemphasis{\sphinxupquote{scalar float}}) \textendash{} Values of the radial coordinate to
map to current density magnitude. If \sphinxtitleref{R} and \sphinxtitleref{Z} are both scalar
values, they are used as the coordinate pair for all of the
values in \sphinxtitleref{t}. Must have the same shape as \sphinxtitleref{Z} unless the
\sphinxtitleref{make\_grid} keyword is set. If the \sphinxtitleref{make\_grid} keyword is True,
\sphinxtitleref{R} must have exactly one dimension.

\item {} 
\sphinxstyleliteralstrong{\sphinxupquote{Z}} (\sphinxstyleliteralemphasis{\sphinxupquote{Array-like}}\sphinxstyleliteralemphasis{\sphinxupquote{ or }}\sphinxstyleliteralemphasis{\sphinxupquote{scalar float}}) \textendash{} Values of the vertical coordinate to
map to current density magnitude. If \sphinxtitleref{R} and \sphinxtitleref{Z} are both scalar
values, they are used as the coordinate pair for all of the
values in \sphinxtitleref{t}. Must have the same shape as \sphinxtitleref{R} unless the
\sphinxtitleref{make\_grid} keyword is set. If the \sphinxtitleref{make\_grid} keyword is True,
\sphinxtitleref{Z} must have exactly one dimension.

\item {} 
\sphinxstyleliteralstrong{\sphinxupquote{t}} (\sphinxstyleliteralemphasis{\sphinxupquote{Array-like}}\sphinxstyleliteralemphasis{\sphinxupquote{ or }}\sphinxstyleliteralemphasis{\sphinxupquote{scalar float}}) \textendash{} Times to perform the conversion at.
If \sphinxtitleref{t} is a single value, it is used for all of the elements of
\sphinxtitleref{R}, \sphinxtitleref{Z}. If the \sphinxtitleref{each\_t} keyword is True, then \sphinxtitleref{t} must be
scalar or have exactly one dimension. If the \sphinxtitleref{each\_t} keyword is
False, \sphinxtitleref{t} must have the same shape as \sphinxtitleref{R} and \sphinxtitleref{Z} (or their
meshgrid if \sphinxtitleref{make\_grid} is True).

\end{itemize}

\item[{Keyword Arguments}] \leavevmode\begin{itemize}
\item {} 
\sphinxstyleliteralstrong{\sphinxupquote{each\_t}} (\sphinxstyleliteralemphasis{\sphinxupquote{Boolean}}) \textendash{} When True, the elements in \sphinxtitleref{R}, \sphinxtitleref{Z} are evaluated
at each value in \sphinxtitleref{t}. If True, \sphinxtitleref{t} must have only one dimension
(or be a scalar). If False, \sphinxtitleref{t} must match the shape of \sphinxtitleref{R} and
\sphinxtitleref{Z} or be a scalar. Default is True (evaluate ALL \sphinxtitleref{R}, \sphinxtitleref{Z} at
EACH element in \sphinxtitleref{t}).

\item {} 
\sphinxstyleliteralstrong{\sphinxupquote{make\_grid}} (\sphinxstyleliteralemphasis{\sphinxupquote{Boolean}}) \textendash{} Set to True to pass \sphinxtitleref{R} and \sphinxtitleref{Z} through
\sphinxcode{\sphinxupquote{scipy.meshgrid()}} before evaluating. If this is set to
True, \sphinxtitleref{R} and \sphinxtitleref{Z} must each only have a single dimension, but
can have different lengths. Default is False (do not form
meshgrid).

\item {} 
\sphinxstyleliteralstrong{\sphinxupquote{length\_unit}} (\sphinxstyleliteralemphasis{\sphinxupquote{String}}\sphinxstyleliteralemphasis{\sphinxupquote{ or }}\sphinxstyleliteralemphasis{\sphinxupquote{1}}) \textendash{} 
Length unit that \sphinxtitleref{R}, \sphinxtitleref{Z} are given in.
If a string is given, it must be a valid unit specifier:
\begin{quote}


\begin{savenotes}\sphinxattablestart
\centering
\begin{tabulary}{\linewidth}[t]{|T|T|}
\hline

’m’
&
meters
\\
\hline
’cm’
&
centimeters
\\
\hline
’mm’
&
millimeters
\\
\hline
’in’
&
inches
\\
\hline
’ft’
&
feet
\\
\hline
’yd’
&
yards
\\
\hline
’smoot’
&
smoots
\\
\hline
’cubit’
&
cubits
\\
\hline
’hand’
&
hands
\\
\hline
’default’
&
meters
\\
\hline
\end{tabulary}
\par
\sphinxattableend\end{savenotes}
\end{quote}

If length\_unit is 1 or None, meters are assumed. The default
value is 1 (use meters).


\item {} 
\sphinxstyleliteralstrong{\sphinxupquote{return\_t}} (\sphinxstyleliteralemphasis{\sphinxupquote{Boolean}}) \textendash{} Set to True to return a tuple of (\sphinxtitleref{j},
\sphinxtitleref{time\_idxs}), where \sphinxtitleref{time\_idxs} is the array of time indices
actually used in evaluating \sphinxtitleref{j} with nearest-neighbor
interpolation. (This is mostly present as an internal helper.)
Default is False (only return \sphinxtitleref{j}).

\end{itemize}

\item[{Returns}] \leavevmode

\sphinxtitleref{j} or (\sphinxtitleref{j}, \sphinxtitleref{time\_idxs})
\begin{itemize}
\item {} 
\sphinxstylestrong{j} (\sphinxtitleref{Array or scalar float}) - The magnitude of the current
density. If all of the input arguments are scalar, then a scalar
is returned. Otherwise, a scipy Array is returned. If \sphinxtitleref{R} and \sphinxtitleref{Z}
both have the same shape then \sphinxtitleref{j} has this shape as well, unless
the \sphinxtitleref{make\_grid} keyword was True, in which case \sphinxtitleref{j} has shape
(len(\sphinxtitleref{Z}), len(\sphinxtitleref{R})).

\item {} 
\sphinxstylestrong{time\_idxs} (Array with same shape as \sphinxtitleref{j}) - The indices
(in \sphinxcode{\sphinxupquote{self.getTimeBase()}}) that were used for
nearest-neighbor interpolation. Only returned if \sphinxtitleref{return\_t} is
True.

\end{itemize}


\end{description}\end{quote}
\subsubsection*{Examples}

All assume that \sphinxtitleref{Eq\_instance} is a valid instance of the appropriate
extension of the {\hyperref[\detokenize{eqtools:eqtools.core.Equilibrium}]{\sphinxcrossref{\sphinxcode{\sphinxupquote{Equilibrium}}}}} abstract class.

Find single j value at R=0.6m, Z=0.0m, t=0.26s:

\begin{sphinxVerbatim}[commandchars=\\\{\}]
\PYG{n}{j\PYGZus{}val} \PYG{o}{=} \PYG{n}{Eq\PYGZus{}instance}\PYG{o}{.}\PYG{n}{rz2j}\PYG{p}{(}\PYG{l+m+mf}{0.6}\PYG{p}{,} \PYG{l+m+mi}{0}\PYG{p}{,} \PYG{l+m+mf}{0.26}\PYG{p}{)}
\end{sphinxVerbatim}

Find j values at (R, Z) points (0.6m, 0m) and (0.8m, 0m) at the
single time t=0.26s. Note that the \sphinxtitleref{Z} vector must be fully
specified, even if the values are all the same:

\begin{sphinxVerbatim}[commandchars=\\\{\}]
\PYG{n}{j\PYGZus{}arr} \PYG{o}{=} \PYG{n}{Eq\PYGZus{}instance}\PYG{o}{.}\PYG{n}{rz2j}\PYG{p}{(}\PYG{p}{[}\PYG{l+m+mf}{0.6}\PYG{p}{,} \PYG{l+m+mf}{0.8}\PYG{p}{]}\PYG{p}{,} \PYG{p}{[}\PYG{l+m+mi}{0}\PYG{p}{,} \PYG{l+m+mi}{0}\PYG{p}{]}\PYG{p}{,} \PYG{l+m+mf}{0.26}\PYG{p}{)}
\end{sphinxVerbatim}

Find j values at (R, Z) points (0.6m, 0m) at times t={[}0.2s, 0.3s{]}:

\begin{sphinxVerbatim}[commandchars=\\\{\}]
\PYG{n}{j\PYGZus{}arr} \PYG{o}{=} \PYG{n}{Eq\PYGZus{}instance}\PYG{o}{.}\PYG{n}{rz2j}\PYG{p}{(}\PYG{l+m+mf}{0.6}\PYG{p}{,} \PYG{l+m+mi}{0}\PYG{p}{,} \PYG{p}{[}\PYG{l+m+mf}{0.2}\PYG{p}{,} \PYG{l+m+mf}{0.3}\PYG{p}{]}\PYG{p}{)}
\end{sphinxVerbatim}

Find j values at (R, Z, t) points (0.6m, 0m, 0.2s) and
(0.5m, 0.2m, 0.3s):

\begin{sphinxVerbatim}[commandchars=\\\{\}]
\PYG{n}{j\PYGZus{}arr} \PYG{o}{=} \PYG{n}{Eq\PYGZus{}instance}\PYG{o}{.}\PYG{n}{rz2j}\PYG{p}{(}\PYG{p}{[}\PYG{l+m+mf}{0.6}\PYG{p}{,} \PYG{l+m+mf}{0.5}\PYG{p}{]}\PYG{p}{,} \PYG{p}{[}\PYG{l+m+mi}{0}\PYG{p}{,} \PYG{l+m+mf}{0.2}\PYG{p}{]}\PYG{p}{,} \PYG{p}{[}\PYG{l+m+mf}{0.2}\PYG{p}{,} \PYG{l+m+mf}{0.3}\PYG{p}{]}\PYG{p}{,} \PYG{n}{each\PYGZus{}t}\PYG{o}{=}\PYG{k+kc}{False}\PYG{p}{)}
\end{sphinxVerbatim}

Find j values on grid defined by 1D vector of radial positions \sphinxtitleref{R}
and 1D vector of vertical positions \sphinxtitleref{Z} at time t=0.2s:

\begin{sphinxVerbatim}[commandchars=\\\{\}]
\PYG{n}{j\PYGZus{}mat} \PYG{o}{=} \PYG{n}{Eq\PYGZus{}instance}\PYG{o}{.}\PYG{n}{rz2j}\PYG{p}{(}\PYG{n}{R}\PYG{p}{,} \PYG{n}{Z}\PYG{p}{,} \PYG{l+m+mf}{0.2}\PYG{p}{,} \PYG{n}{make\PYGZus{}grid}\PYG{o}{=}\PYG{k+kc}{True}\PYG{p}{)}
\end{sphinxVerbatim}

\end{fulllineitems}

\index{rz2FieldLineTrace() (eqtools.core.Equilibrium method)@\spxentry{rz2FieldLineTrace()}\spxextra{eqtools.core.Equilibrium method}}

\begin{fulllineitems}
\phantomsection\label{\detokenize{eqtools:eqtools.core.Equilibrium.rz2FieldLineTrace}}\pysiglinewithargsret{\sphinxbfcode{\sphinxupquote{rz2FieldLineTrace}}}{\emph{R0}, \emph{Z0}, \emph{t}, \emph{phi0=0.0}, \emph{field='B'}, \emph{num\_rev=1.0}, \emph{rev\_method='toroidal'}, \emph{dphi=0.06283185307179587}, \emph{integrator='dopri5'}}{}
Trace a field line starting from a given (R, phi, Z) point.
\begin{quote}\begin{description}
\item[{Parameters}] \leavevmode\begin{itemize}
\item {} 
\sphinxstyleliteralstrong{\sphinxupquote{R0}} (\sphinxstyleliteralemphasis{\sphinxupquote{float}}) \textendash{} Major radial coordinate of starting point.

\item {} 
\sphinxstyleliteralstrong{\sphinxupquote{Z0}} (\sphinxstyleliteralemphasis{\sphinxupquote{float}}) \textendash{} Vertical coordinate of starting point.

\item {} 
\sphinxstyleliteralstrong{\sphinxupquote{t}} (\sphinxstyleliteralemphasis{\sphinxupquote{float}}) \textendash{} Time to trace field line at.

\end{itemize}

\item[{Keyword Arguments}] \leavevmode\begin{itemize}
\item {} 
\sphinxstyleliteralstrong{\sphinxupquote{phi0}} (\sphinxstyleliteralemphasis{\sphinxupquote{float}}) \textendash{} Toroidal angle of starting point in radians. Default
is 0.0.

\item {} 
\sphinxstyleliteralstrong{\sphinxupquote{field}} (\sphinxstyleliteralemphasis{\sphinxupquote{\{'B'}}\sphinxstyleliteralemphasis{\sphinxupquote{, }}\sphinxstyleliteralemphasis{\sphinxupquote{'j'\}}}) \textendash{} The field to use. Can be magnetic field (‘B’) or
current density (‘j’). Default is ‘B’ (magnetic field).

\item {} 
\sphinxstyleliteralstrong{\sphinxupquote{num\_rev}} (\sphinxstyleliteralemphasis{\sphinxupquote{float}}) \textendash{} The number of revolutions to trace the field line
through. Whether this refers to toroidal or poloidal revolutions
is determined by the \sphinxtitleref{rev\_method} keyword. Default is 1.0.

\item {} 
\sphinxstyleliteralstrong{\sphinxupquote{rev\_method}} (\sphinxstyleliteralemphasis{\sphinxupquote{'toroidal'}}\sphinxstyleliteralemphasis{\sphinxupquote{, }}\sphinxstyleliteralemphasis{\sphinxupquote{'poloidal'}}) \textendash{} Whether \sphinxtitleref{num\_rev} refers to the
number of toroidal or poloidal revolutions the field line should
make. Note that ‘poloidal’ only makes sense for close field
lines. Default is ‘toroidal’.

\item {} 
\sphinxstyleliteralstrong{\sphinxupquote{dphi}} (\sphinxstyleliteralemphasis{\sphinxupquote{float}}) \textendash{} Toroidal step size, in radians. Default is 0.02*pi.
The number of steps taken is then 2*pi times the number of
toroidal rotations divided by dphi. This can be negative to
trace a field line clockwise instead of counterclockwise.

\item {} 
\sphinxstyleliteralstrong{\sphinxupquote{integrator}} (\sphinxstyleliteralemphasis{\sphinxupquote{str}}) \textendash{} The integrator to use with
\sphinxcode{\sphinxupquote{scipy.integrate.ode}}. Default is ‘dopri5’ (explicit
Dormand-Prince of order (4)5). Can also be an instance of
\sphinxcode{\sphinxupquote{scipy.integrate.ode}} for which the integrator and its
options has been set.

\end{itemize}

\item[{Returns}] \leavevmode
Containing the (R, Z, phi) coordinates.

\item[{Return type}] \leavevmode
array, (\sphinxtitleref{nsteps} + 1, 3)

\end{description}\end{quote}

\end{fulllineitems}

\index{rho2FieldLineTrace() (eqtools.core.Equilibrium method)@\spxentry{rho2FieldLineTrace()}\spxextra{eqtools.core.Equilibrium method}}

\begin{fulllineitems}
\phantomsection\label{\detokenize{eqtools:eqtools.core.Equilibrium.rho2FieldLineTrace}}\pysiglinewithargsret{\sphinxbfcode{\sphinxupquote{rho2FieldLineTrace}}}{\emph{rho}, \emph{t}, \emph{origin='psinorm'}, \emph{**kwargs}}{}
Trace a field line starting from a given normalized coordinate point.

The field line is started at the outboard midplane.
\begin{quote}\begin{description}
\item[{Parameters}] \leavevmode\begin{itemize}
\item {} 
\sphinxstyleliteralstrong{\sphinxupquote{rho}} (\sphinxstyleliteralemphasis{\sphinxupquote{float}}) \textendash{} Flux surface label of starting point.

\item {} 
\sphinxstyleliteralstrong{\sphinxupquote{t}} (\sphinxstyleliteralemphasis{\sphinxupquote{float}}) \textendash{} Time to trace field line at.

\end{itemize}

\item[{Keyword Arguments}] \leavevmode\begin{itemize}
\item {} 
\sphinxstyleliteralstrong{\sphinxupquote{origin}} (\sphinxstyleliteralemphasis{\sphinxupquote{\{'psinorm'}}\sphinxstyleliteralemphasis{\sphinxupquote{, }}\sphinxstyleliteralemphasis{\sphinxupquote{'phinorm'}}\sphinxstyleliteralemphasis{\sphinxupquote{, }}\sphinxstyleliteralemphasis{\sphinxupquote{'volnorm'}}\sphinxstyleliteralemphasis{\sphinxupquote{, }}\sphinxstyleliteralemphasis{\sphinxupquote{'r/a'}}\sphinxstyleliteralemphasis{\sphinxupquote{, }}\sphinxstyleliteralemphasis{\sphinxupquote{'Rmid'}}\sphinxstyleliteralemphasis{\sphinxupquote{, }}\sphinxstyleliteralemphasis{\sphinxupquote{'Fnorm'\}}}) \textendash{} The flux surface coordinates which \sphinxtitleref{rhovals} is given in.
Default is ‘psinorm’.

\item {} 
\sphinxstyleliteralstrong{\sphinxupquote{phi0}} (\sphinxstyleliteralemphasis{\sphinxupquote{float}}) \textendash{} Toroidal angle of starting point in radians. Default
is 0.0.

\item {} 
\sphinxstyleliteralstrong{\sphinxupquote{field}} (\sphinxstyleliteralemphasis{\sphinxupquote{\{'B'}}\sphinxstyleliteralemphasis{\sphinxupquote{, }}\sphinxstyleliteralemphasis{\sphinxupquote{'j'\}}}) \textendash{} The field to use. Can be magnetic field (‘B’) or
current density (‘j’). Default is ‘B’ (magnetic field).

\item {} 
\sphinxstyleliteralstrong{\sphinxupquote{num\_rev}} (\sphinxstyleliteralemphasis{\sphinxupquote{float}}) \textendash{} The number of revolutions to trace the field line
through. Whether this refers to toroidal or poloidal revolutions
is determined by the \sphinxtitleref{rev\_method} keyword. Default is 1.0.

\item {} 
\sphinxstyleliteralstrong{\sphinxupquote{rev\_method}} (\sphinxstyleliteralemphasis{\sphinxupquote{'toroidal'}}\sphinxstyleliteralemphasis{\sphinxupquote{, }}\sphinxstyleliteralemphasis{\sphinxupquote{'poloidal'}}) \textendash{} Whether \sphinxtitleref{num\_rev} refers to the
number of toroidal or poloidal revolutions the field line should
make. Note that ‘poloidal’ only makes sense for close field
lines. Default is ‘toroidal’.

\item {} 
\sphinxstyleliteralstrong{\sphinxupquote{dphi}} (\sphinxstyleliteralemphasis{\sphinxupquote{float}}) \textendash{} Toroidal step size, in radians. Default is 0.02*pi.
The number of steps taken is then 2*pi times the number of
toroidal rotations divided by dphi. This can be negative to
trace a field line clockwise instead of counterclockwise.

\item {} 
\sphinxstyleliteralstrong{\sphinxupquote{integrator}} (\sphinxstyleliteralemphasis{\sphinxupquote{str}}) \textendash{} The integrator to use with
\sphinxcode{\sphinxupquote{scipy.integrate.ode}}. Default is ‘dopri5’ (explicit
Dormand-Prince of order (4)5). Can also be an instance of
\sphinxcode{\sphinxupquote{scipy.integrate.ode}} for which the integrator and its
options has been set.

\end{itemize}

\item[{Returns}] \leavevmode
Containing the (R, Z, phi) coordinates.

\item[{Return type}] \leavevmode
array, (\sphinxtitleref{nsteps} + 1, 3)

\end{description}\end{quote}

\end{fulllineitems}

\index{plotField() (eqtools.core.Equilibrium method)@\spxentry{plotField()}\spxextra{eqtools.core.Equilibrium method}}

\begin{fulllineitems}
\phantomsection\label{\detokenize{eqtools:eqtools.core.Equilibrium.plotField}}\pysiglinewithargsret{\sphinxbfcode{\sphinxupquote{plotField}}}{\emph{t}, \emph{rhovals=6}, \emph{rhomin=0.05}, \emph{rhomax=0.95}, \emph{color='b'}, \emph{cmap='plasma'}, \emph{alpha=0.5}, \emph{arrows=True}, \emph{linewidth=1.0}, \emph{arrowlinewidth=3.0}, \emph{a=None}, \emph{**kwargs}}{}
Plot the field lines starting from a number of points.

The field lines are started at the outboard midplane.

If uniformly-spaced psinorm points are used, the spacing of the magnetic
field lines will be directly proportional to the field strength,
assuming a sufficient number of revolutions is traced.
\begin{quote}\begin{description}
\item[{Parameters}] \leavevmode
\sphinxstyleliteralstrong{\sphinxupquote{t}} (\sphinxstyleliteralemphasis{\sphinxupquote{float}}) \textendash{} Time to trace field line at.

\item[{Keyword Arguments}] \leavevmode\begin{itemize}
\item {} 
\sphinxstyleliteralstrong{\sphinxupquote{rhovals}} (\sphinxstyleliteralemphasis{\sphinxupquote{int}}\sphinxstyleliteralemphasis{\sphinxupquote{ or }}\sphinxstyleliteralemphasis{\sphinxupquote{array of int}}) \textendash{} The number of uniformly-spaced rho
points between \sphinxtitleref{rhomin} and \sphinxtitleref{rhomax} to use, or an explicit grid of rho
points to use. Default is 6.

\item {} 
\sphinxstyleliteralstrong{\sphinxupquote{rhomin}} (\sphinxstyleliteralemphasis{\sphinxupquote{float}}) \textendash{} The minimum value of rho to use when using a
uniformly-spaced grid. Default is 0.05.

\item {} 
\sphinxstyleliteralstrong{\sphinxupquote{rhomax}} (\sphinxstyleliteralemphasis{\sphinxupquote{float}}) \textendash{} The maximum value of rho to use when using a
uniformly-spaced grid. Default is 0.95.

\item {} 
\sphinxstyleliteralstrong{\sphinxupquote{color}} (\sphinxstyleliteralemphasis{\sphinxupquote{str}}) \textendash{} The color to plot the field lines in. Default is ‘b’.
If set to ‘sequential’, each field line will be a different
color, in the sequence matplotlib assigns them. If set to
‘magnitude’, the coloring will be proportional to the magnitude
of the field. Note that this is very time-consuming, as the
limitations of matplotlib mean that each line segment must be
plotted individually.

\item {} 
\sphinxstyleliteralstrong{\sphinxupquote{cmap}} (\sphinxstyleliteralemphasis{\sphinxupquote{str}}) \textendash{} The colormap to use when \sphinxtitleref{color} is ‘magnitude’. Default
is ‘plasma’, a perceptually uniform sequential colormap.

\item {} 
\sphinxstyleliteralstrong{\sphinxupquote{alpha}} (\sphinxstyleliteralemphasis{\sphinxupquote{float}}) \textendash{} The transparency to plot the field lines with.
Default is 0.5.

\item {} 
\sphinxstyleliteralstrong{\sphinxupquote{arrows}} (\sphinxstyleliteralemphasis{\sphinxupquote{bool}}) \textendash{} If True, an arrowhead indicating the field direction
will be drawn at the start of each field line. Default is True.

\item {} 
\sphinxstyleliteralstrong{\sphinxupquote{linewidth}} (\sphinxstyleliteralemphasis{\sphinxupquote{float}}) \textendash{} The line width to use when plotting the field
lines. Default is 1.0.

\item {} 
\sphinxstyleliteralstrong{\sphinxupquote{arrowlinewidth}} (\sphinxstyleliteralemphasis{\sphinxupquote{float}}) \textendash{} The line width to use when plotting the
arrows. Default is 3.0

\item {} 
\sphinxstyleliteralstrong{\sphinxupquote{a}} (\sphinxcode{\sphinxupquote{matplotlib.axes.\_subplots.Axes3DSubplot}}) \textendash{} The axes to
plot the field lines on. Default is to make a new figure. Note
that a colorbar will be drawn when \sphinxtitleref{color} is magnitude, but
only if \sphinxtitleref{a} is not provided.

\item {} 
\sphinxstyleliteralstrong{\sphinxupquote{origin}} (\sphinxstyleliteralemphasis{\sphinxupquote{\{'psinorm'}}\sphinxstyleliteralemphasis{\sphinxupquote{, }}\sphinxstyleliteralemphasis{\sphinxupquote{'phinorm'}}\sphinxstyleliteralemphasis{\sphinxupquote{, }}\sphinxstyleliteralemphasis{\sphinxupquote{'volnorm'}}\sphinxstyleliteralemphasis{\sphinxupquote{, }}\sphinxstyleliteralemphasis{\sphinxupquote{'r/a'}}\sphinxstyleliteralemphasis{\sphinxupquote{, }}\sphinxstyleliteralemphasis{\sphinxupquote{'Rmid'}}\sphinxstyleliteralemphasis{\sphinxupquote{, }}\sphinxstyleliteralemphasis{\sphinxupquote{'Fnorm'\}}}) \textendash{} The flux surface coordinates which \sphinxtitleref{rhovals} is given in.
Default is ‘psinorm’.

\item {} 
\sphinxstyleliteralstrong{\sphinxupquote{phi0}} (\sphinxstyleliteralemphasis{\sphinxupquote{float}}) \textendash{} Toroidal angle of starting point in radians. Default
is 0.0.

\item {} 
\sphinxstyleliteralstrong{\sphinxupquote{field}} (\sphinxstyleliteralemphasis{\sphinxupquote{\{'B'}}\sphinxstyleliteralemphasis{\sphinxupquote{, }}\sphinxstyleliteralemphasis{\sphinxupquote{'j'\}}}) \textendash{} The field to use. Can be magnetic field (‘B’) or
current density (‘j’). Default is ‘B’ (magnetic field).

\item {} 
\sphinxstyleliteralstrong{\sphinxupquote{num\_rev}} (\sphinxstyleliteralemphasis{\sphinxupquote{float}}) \textendash{} The number of revolutions to trace the field line
through. Whether this refers to toroidal or poloidal revolutions
is determined by the \sphinxtitleref{rev\_method} keyword. Default is 1.0.

\item {} 
\sphinxstyleliteralstrong{\sphinxupquote{rev\_method}} (\sphinxstyleliteralemphasis{\sphinxupquote{'toroidal'}}\sphinxstyleliteralemphasis{\sphinxupquote{, }}\sphinxstyleliteralemphasis{\sphinxupquote{'poloidal'}}) \textendash{} Whether \sphinxtitleref{num\_rev} refers to the
number of toroidal or poloidal revolutions the field line should
make. Note that ‘poloidal’ only makes sense for close field
lines. Default is ‘toroidal’.

\item {} 
\sphinxstyleliteralstrong{\sphinxupquote{dphi}} (\sphinxstyleliteralemphasis{\sphinxupquote{float}}) \textendash{} Toroidal step size, in radians. Default is 0.02*pi.
The number of steps taken is then 2*pi times the number of
toroidal rotations divided by dphi. This can be negative to
trace a field line clockwise instead of counterclockwise.

\item {} 
\sphinxstyleliteralstrong{\sphinxupquote{integrator}} (\sphinxstyleliteralemphasis{\sphinxupquote{str}}) \textendash{} The integrator to use with
\sphinxcode{\sphinxupquote{scipy.integrate.ode}}. Default is ‘dopri5’ (explicit
Dormand-Prince of order (4)5). Can also be an instance of
\sphinxcode{\sphinxupquote{scipy.integrate.ode}} for which the integrator and its
options has been set.

\end{itemize}

\item[{Returns}] \leavevmode
The figure and axis which the field lines were plotted in.

\item[{Return type}] \leavevmode
(figure, axis)

\end{description}\end{quote}

\end{fulllineitems}

\index{getMagRSpline() (eqtools.core.Equilibrium method)@\spxentry{getMagRSpline()}\spxextra{eqtools.core.Equilibrium method}}

\begin{fulllineitems}
\phantomsection\label{\detokenize{eqtools:eqtools.core.Equilibrium.getMagRSpline}}\pysiglinewithargsret{\sphinxbfcode{\sphinxupquote{getMagRSpline}}}{\emph{length\_unit=1}, \emph{kind='nearest'}}{}
Gets the univariate spline to interpolate R\_mag as a function of time.

Only used if the instance was created with keyword tspline=True.
\begin{quote}\begin{description}
\item[{Keyword Arguments}] \leavevmode\begin{itemize}
\item {} 
\sphinxstyleliteralstrong{\sphinxupquote{length\_unit}} (\sphinxstyleliteralemphasis{\sphinxupquote{String}}\sphinxstyleliteralemphasis{\sphinxupquote{ or }}\sphinxstyleliteralemphasis{\sphinxupquote{1}}) \textendash{} 
Length unit that R\_mag is returned in. If
a string is given, it must be a valid unit specifier:
\begin{quote}


\begin{savenotes}\sphinxattablestart
\centering
\begin{tabulary}{\linewidth}[t]{|T|T|}
\hline

’m’
&
meters
\\
\hline
’cm’
&
centimeters
\\
\hline
’mm’
&
millimeters
\\
\hline
’in’
&
inches
\\
\hline
’ft’
&
feet
\\
\hline
’yd’
&
yards
\\
\hline
’smoot’
&
smoots
\\
\hline
’cubit’
&
cubits
\\
\hline
’hand’
&
hands
\\
\hline
’default’
&
meters
\\
\hline
\end{tabulary}
\par
\sphinxattableend\end{savenotes}
\end{quote}

If length\_unit is 1 or None, meters are assumed. The default
value is 1 (R\_out returned in meters).


\item {} 
\sphinxstyleliteralstrong{\sphinxupquote{kind}} (\sphinxstyleliteralemphasis{\sphinxupquote{String}}\sphinxstyleliteralemphasis{\sphinxupquote{ or }}\sphinxstyleliteralemphasis{\sphinxupquote{non-negative int}}) \textendash{} Specifies the type of interpolation to be performed in getting
from t to R\_mag. This is passed to
\sphinxcode{\sphinxupquote{scipy.interpolate.interp1d}}. Valid options are:
‘linear’, ‘nearest’, ‘zero’, ‘slinear’, ‘quadratic’, ‘cubic’
If this keyword is an integer, it specifies the order of spline
to use. See the documentation for interp1d for more details.
Default value is ‘cubic’ (3rd order spline interpolation) when
\sphinxtitleref{trispline} is True, ‘nearest’ otherwise.

\end{itemize}

\item[{Returns}] \leavevmode
\begin{description}
\item[{\sphinxcode{\sphinxupquote{trispline.UnivariateInterpolator}} or}] \leavevmode
\sphinxcode{\sphinxupquote{scipy.interpolate.interp1d}} to convert from t to MagR.

\end{description}


\end{description}\end{quote}

\end{fulllineitems}

\index{getMagZSpline() (eqtools.core.Equilibrium method)@\spxentry{getMagZSpline()}\spxextra{eqtools.core.Equilibrium method}}

\begin{fulllineitems}
\phantomsection\label{\detokenize{eqtools:eqtools.core.Equilibrium.getMagZSpline}}\pysiglinewithargsret{\sphinxbfcode{\sphinxupquote{getMagZSpline}}}{\emph{length\_unit=1}, \emph{kind='nearest'}}{}
Gets the univariate spline to interpolate Z\_mag as a function of time.

Generated for completeness of the core position calculation when using
tspline = True
\begin{quote}\begin{description}
\item[{Keyword Arguments}] \leavevmode\begin{itemize}
\item {} 
\sphinxstyleliteralstrong{\sphinxupquote{length\_unit}} (\sphinxstyleliteralemphasis{\sphinxupquote{String}}\sphinxstyleliteralemphasis{\sphinxupquote{ or }}\sphinxstyleliteralemphasis{\sphinxupquote{1}}) \textendash{} 
Length unit that R\_mag is returned in. If
a string is given, it must be a valid unit specifier:
\begin{quote}


\begin{savenotes}\sphinxattablestart
\centering
\begin{tabulary}{\linewidth}[t]{|T|T|}
\hline

’m’
&
meters
\\
\hline
’cm’
&
centimeters
\\
\hline
’mm’
&
millimeters
\\
\hline
’in’
&
inches
\\
\hline
’ft’
&
feet
\\
\hline
’yd’
&
yards
\\
\hline
’smoot’
&
smoots
\\
\hline
’cubit’
&
cubits
\\
\hline
’hand’
&
hands
\\
\hline
’default’
&
meters
\\
\hline
\end{tabulary}
\par
\sphinxattableend\end{savenotes}
\end{quote}

If length\_unit is 1 or None, meters are assumed. The default
value is 1 (R\_out returned in meters).


\item {} 
\sphinxstyleliteralstrong{\sphinxupquote{kind}} (\sphinxstyleliteralemphasis{\sphinxupquote{String}}\sphinxstyleliteralemphasis{\sphinxupquote{ or }}\sphinxstyleliteralemphasis{\sphinxupquote{non-negative int}}) \textendash{} Specifies the type of interpolation to be performed in getting
from t to Z\_mag. This is passed to
\sphinxcode{\sphinxupquote{scipy.interpolate.interp1d}}. Valid options are:
‘linear’, ‘nearest’, ‘zero’, ‘slinear’, ‘quadratic’, ‘cubic’
If this keyword is an integer, it specifies the order of spline
to use. See the documentation for interp1d for more details.
Default value is ‘cubic’ (3rd order spline interpolation) when
\sphinxtitleref{trispline} is True, ‘nearest’ otherwise.

\end{itemize}

\item[{Returns}] \leavevmode
\begin{description}
\item[{\sphinxcode{\sphinxupquote{trispline.UnivariateInterpolator}} or}] \leavevmode
\sphinxcode{\sphinxupquote{scipy.interpolate.interp1d}} to convert from t to MagZ.

\end{description}


\end{description}\end{quote}

\end{fulllineitems}

\index{getRmidOutSpline() (eqtools.core.Equilibrium method)@\spxentry{getRmidOutSpline()}\spxextra{eqtools.core.Equilibrium method}}

\begin{fulllineitems}
\phantomsection\label{\detokenize{eqtools:eqtools.core.Equilibrium.getRmidOutSpline}}\pysiglinewithargsret{\sphinxbfcode{\sphinxupquote{getRmidOutSpline}}}{\emph{length\_unit=1}, \emph{kind='nearest'}}{}
Gets the univariate spline to interpolate R\_mid\_out as a function of time.

Generated for completeness of the core position calculation when using
tspline = True
\begin{quote}\begin{description}
\item[{Keyword Arguments}] \leavevmode\begin{itemize}
\item {} 
\sphinxstyleliteralstrong{\sphinxupquote{length\_unit}} (\sphinxstyleliteralemphasis{\sphinxupquote{String}}\sphinxstyleliteralemphasis{\sphinxupquote{ or }}\sphinxstyleliteralemphasis{\sphinxupquote{1}}) \textendash{} 
Length unit that R\_mag is returned in. If
a string is given, it must be a valid unit specifier:
\begin{quote}


\begin{savenotes}\sphinxattablestart
\centering
\begin{tabulary}{\linewidth}[t]{|T|T|}
\hline

’m’
&
meters
\\
\hline
’cm’
&
centimeters
\\
\hline
’mm’
&
millimeters
\\
\hline
’in’
&
inches
\\
\hline
’ft’
&
feet
\\
\hline
’yd’
&
yards
\\
\hline
’smoot’
&
smoots
\\
\hline
’cubit’
&
cubits
\\
\hline
’hand’
&
hands
\\
\hline
’default’
&
meters
\\
\hline
\end{tabulary}
\par
\sphinxattableend\end{savenotes}
\end{quote}

If length\_unit is 1 or None, meters are assumed. The default
value is 1 (R\_out returned in meters).


\item {} 
\sphinxstyleliteralstrong{\sphinxupquote{kind}} (\sphinxstyleliteralemphasis{\sphinxupquote{String}}\sphinxstyleliteralemphasis{\sphinxupquote{ or }}\sphinxstyleliteralemphasis{\sphinxupquote{non-negative int}}) \textendash{} Specifies the type of interpolation to be performed in getting
from t to R\_mid\_out. This is passed to
\sphinxcode{\sphinxupquote{scipy.interpolate.interp1d}}. Valid options are:
‘linear’, ‘nearest’, ‘zero’, ‘slinear’, ‘quadratic’, ‘cubic’
If this keyword is an integer, it specifies the order of spline
to use. See the documentation for interp1d for more details.
Default value is ‘cubic’ (3rd order spline interpolation) when
\sphinxtitleref{trispline} is True, ‘nearest’ otherwise.

\end{itemize}

\item[{Returns}] \leavevmode
\begin{description}
\item[{\sphinxcode{\sphinxupquote{trispline.UnivariateInterpolator}} or}] \leavevmode
\sphinxcode{\sphinxupquote{scipy.interpolate.interp1d}} to convert from t to R\_mid.

\end{description}


\end{description}\end{quote}

\end{fulllineitems}

\index{getAOutSpline() (eqtools.core.Equilibrium method)@\spxentry{getAOutSpline()}\spxextra{eqtools.core.Equilibrium method}}

\begin{fulllineitems}
\phantomsection\label{\detokenize{eqtools:eqtools.core.Equilibrium.getAOutSpline}}\pysiglinewithargsret{\sphinxbfcode{\sphinxupquote{getAOutSpline}}}{\emph{length\_unit=1}, \emph{kind='nearest'}}{}
Gets the univariate spline to interpolate a\_out as a function of time.
\begin{quote}\begin{description}
\item[{Keyword Arguments}] \leavevmode\begin{itemize}
\item {} 
\sphinxstyleliteralstrong{\sphinxupquote{length\_unit}} (\sphinxstyleliteralemphasis{\sphinxupquote{String}}\sphinxstyleliteralemphasis{\sphinxupquote{ or }}\sphinxstyleliteralemphasis{\sphinxupquote{1}}) \textendash{} 
Length unit that a\_out is returned in. If
a string is given, it must be a valid unit specifier:
\begin{quote}


\begin{savenotes}\sphinxattablestart
\centering
\begin{tabulary}{\linewidth}[t]{|T|T|}
\hline

’m’
&
meters
\\
\hline
’cm’
&
centimeters
\\
\hline
’mm’
&
millimeters
\\
\hline
’in’
&
inches
\\
\hline
’ft’
&
feet
\\
\hline
’yd’
&
yards
\\
\hline
’smoot’
&
smoots
\\
\hline
’cubit’
&
cubits
\\
\hline
’hand’
&
hands
\\
\hline
’default’
&
meters
\\
\hline
\end{tabulary}
\par
\sphinxattableend\end{savenotes}
\end{quote}

If \sphinxtitleref{length\_unit} is 1 or None, meters are assumed. The default
value is 1 (a\_out returned in meters).


\item {} 
\sphinxstyleliteralstrong{\sphinxupquote{kind}} (\sphinxstyleliteralemphasis{\sphinxupquote{String}}\sphinxstyleliteralemphasis{\sphinxupquote{ or }}\sphinxstyleliteralemphasis{\sphinxupquote{non-negative int}}) \textendash{} Specifies the type of interpolation to be performed in getting
from t to a\_out. This is passed to
\sphinxcode{\sphinxupquote{scipy.interpolate.interp1d}}. Valid options are:
‘linear’, ‘nearest’, ‘zero’, ‘slinear’, ‘quadratic’, ‘cubic’
If this keyword is an integer, it specifies the order of spline
to use. See the documentation for interp1d for more details.
Default value is ‘cubic’ (3rd order spline interpolation) when
\sphinxtitleref{trispline} is True, ‘nearest’ otherwise.

\end{itemize}

\item[{Returns}] \leavevmode
\begin{description}
\item[{\sphinxcode{\sphinxupquote{trispline.UnivariateInterpolator}} or}] \leavevmode
\sphinxcode{\sphinxupquote{scipy.interpolate.interp1d}} to convert from t to a\_out.

\end{description}


\end{description}\end{quote}

\end{fulllineitems}

\index{getBtVacSpline() (eqtools.core.Equilibrium method)@\spxentry{getBtVacSpline()}\spxextra{eqtools.core.Equilibrium method}}

\begin{fulllineitems}
\phantomsection\label{\detokenize{eqtools:eqtools.core.Equilibrium.getBtVacSpline}}\pysiglinewithargsret{\sphinxbfcode{\sphinxupquote{getBtVacSpline}}}{\emph{kind='nearest'}}{}
Gets the univariate spline to interpolate BtVac as a function of time.

Only used if the instance was created with keyword tspline=True.
\begin{quote}\begin{description}
\item[{Keyword Arguments}] \leavevmode
\sphinxstyleliteralstrong{\sphinxupquote{kind}} (\sphinxstyleliteralemphasis{\sphinxupquote{String}}\sphinxstyleliteralemphasis{\sphinxupquote{ or }}\sphinxstyleliteralemphasis{\sphinxupquote{non-negative int}}) \textendash{} Specifies the type of interpolation to be performed in getting
from t to BtVac. This is passed to
\sphinxcode{\sphinxupquote{scipy.interpolate.interp1d}}. Valid options are:
‘linear’, ‘nearest’, ‘zero’, ‘slinear’, ‘quadratic’, ‘cubic’
If this keyword is an integer, it specifies the order of spline
to use. See the documentation for interp1d for more details.
Default value is ‘cubic’ (3rd order spline interpolation) when
\sphinxtitleref{trispline} is True, ‘nearest’ otherwise.

\item[{Returns}] \leavevmode
\begin{description}
\item[{\sphinxcode{\sphinxupquote{trispline.UnivariateInterpolator}} or}] \leavevmode
\sphinxcode{\sphinxupquote{scipy.interpolate.interp1d}} to convert from t to BtVac.

\end{description}


\end{description}\end{quote}

\end{fulllineitems}

\index{getInfo() (eqtools.core.Equilibrium method)@\spxentry{getInfo()}\spxextra{eqtools.core.Equilibrium method}}

\begin{fulllineitems}
\phantomsection\label{\detokenize{eqtools:eqtools.core.Equilibrium.getInfo}}\pysiglinewithargsret{\sphinxbfcode{\sphinxupquote{getInfo}}}{}{}
Abstract method.  See child classes for implementation.

Returns namedtuple of instance parameters (shot, equilibrium type, size, timebase, etc.)

\end{fulllineitems}

\index{getTimeBase() (eqtools.core.Equilibrium method)@\spxentry{getTimeBase()}\spxextra{eqtools.core.Equilibrium method}}

\begin{fulllineitems}
\phantomsection\label{\detokenize{eqtools:eqtools.core.Equilibrium.getTimeBase}}\pysiglinewithargsret{\sphinxbfcode{\sphinxupquote{getTimeBase}}}{}{}
Abstract method.  See child classes for implementation.

Returns timebase array {[}t{]}

\end{fulllineitems}

\index{getFluxGrid() (eqtools.core.Equilibrium method)@\spxentry{getFluxGrid()}\spxextra{eqtools.core.Equilibrium method}}

\begin{fulllineitems}
\phantomsection\label{\detokenize{eqtools:eqtools.core.Equilibrium.getFluxGrid}}\pysiglinewithargsret{\sphinxbfcode{\sphinxupquote{getFluxGrid}}}{}{}
Abstract method.  See child classes for implementation.
\begin{description}
\item[{returns 3D grid of psi(r,z,t)}] \leavevmode\begin{description}
\item[{The array returned should have the following dimensions:}] \leavevmode
First dimension: time
Second dimension: Z
Third dimension: R

\end{description}

\end{description}

\end{fulllineitems}

\index{getRGrid() (eqtools.core.Equilibrium method)@\spxentry{getRGrid()}\spxextra{eqtools.core.Equilibrium method}}

\begin{fulllineitems}
\phantomsection\label{\detokenize{eqtools:eqtools.core.Equilibrium.getRGrid}}\pysiglinewithargsret{\sphinxbfcode{\sphinxupquote{getRGrid}}}{}{}
Abstract method.  See child classes for implementation.

Returns vector of R-values for psiRZ grid {[}r{]}

\end{fulllineitems}

\index{getZGrid() (eqtools.core.Equilibrium method)@\spxentry{getZGrid()}\spxextra{eqtools.core.Equilibrium method}}

\begin{fulllineitems}
\phantomsection\label{\detokenize{eqtools:eqtools.core.Equilibrium.getZGrid}}\pysiglinewithargsret{\sphinxbfcode{\sphinxupquote{getZGrid}}}{}{}
Abstract method.  See child classes for implementation.

Returns vector of Z-values for psiRZ grid {[}z{]}

\end{fulllineitems}

\index{getFluxAxis() (eqtools.core.Equilibrium method)@\spxentry{getFluxAxis()}\spxextra{eqtools.core.Equilibrium method}}

\begin{fulllineitems}
\phantomsection\label{\detokenize{eqtools:eqtools.core.Equilibrium.getFluxAxis}}\pysiglinewithargsret{\sphinxbfcode{\sphinxupquote{getFluxAxis}}}{}{}
Abstract method.  See child classes for implementation.

Returns psi at magnetic axis {[}t{]}

\end{fulllineitems}

\index{getFluxLCFS() (eqtools.core.Equilibrium method)@\spxentry{getFluxLCFS()}\spxextra{eqtools.core.Equilibrium method}}

\begin{fulllineitems}
\phantomsection\label{\detokenize{eqtools:eqtools.core.Equilibrium.getFluxLCFS}}\pysiglinewithargsret{\sphinxbfcode{\sphinxupquote{getFluxLCFS}}}{}{}
Abstract method.  See child classes for implementation.

Returns psi a separatrix {[}t{]}

\end{fulllineitems}

\index{getRLCFS() (eqtools.core.Equilibrium method)@\spxentry{getRLCFS()}\spxextra{eqtools.core.Equilibrium method}}

\begin{fulllineitems}
\phantomsection\label{\detokenize{eqtools:eqtools.core.Equilibrium.getRLCFS}}\pysiglinewithargsret{\sphinxbfcode{\sphinxupquote{getRLCFS}}}{}{}
Abstract method.  See child classes for implementation.

Returns R-positions (n points) mapping LCFS {[}t,n{]}

\end{fulllineitems}

\index{getZLCFS() (eqtools.core.Equilibrium method)@\spxentry{getZLCFS()}\spxextra{eqtools.core.Equilibrium method}}

\begin{fulllineitems}
\phantomsection\label{\detokenize{eqtools:eqtools.core.Equilibrium.getZLCFS}}\pysiglinewithargsret{\sphinxbfcode{\sphinxupquote{getZLCFS}}}{}{}
Abstract method.  See child classes for implementation.

Returns Z-positions (n points) mapping LCFS {[}t,n{]}

\end{fulllineitems}

\index{remapLCFS() (eqtools.core.Equilibrium method)@\spxentry{remapLCFS()}\spxextra{eqtools.core.Equilibrium method}}

\begin{fulllineitems}
\phantomsection\label{\detokenize{eqtools:eqtools.core.Equilibrium.remapLCFS}}\pysiglinewithargsret{\sphinxbfcode{\sphinxupquote{remapLCFS}}}{}{}
Abstract method.  See child classes for implementation.

Overwrites stored R,Z positions of LCFS with explicitly calculated psinorm=1
surface.  This surface is then masked using core.inPolygon() to only draw within
vacuum vessel, the end result replacing RLCFS, ZLCFS with an R,Z array showing
the divertor legs of the flux surface in addition to the core-enclosing closed
flux surface.

\end{fulllineitems}

\index{getFluxVol() (eqtools.core.Equilibrium method)@\spxentry{getFluxVol()}\spxextra{eqtools.core.Equilibrium method}}

\begin{fulllineitems}
\phantomsection\label{\detokenize{eqtools:eqtools.core.Equilibrium.getFluxVol}}\pysiglinewithargsret{\sphinxbfcode{\sphinxupquote{getFluxVol}}}{}{}
Abstract method.  See child classes for implementation.

Returns volume contained within flux surface as function of psi {[}psi,t{]}.
Psi assumed to be evenly-spaced grid on {[}0,1{]}

\end{fulllineitems}

\index{getVolLCFS() (eqtools.core.Equilibrium method)@\spxentry{getVolLCFS()}\spxextra{eqtools.core.Equilibrium method}}

\begin{fulllineitems}
\phantomsection\label{\detokenize{eqtools:eqtools.core.Equilibrium.getVolLCFS}}\pysiglinewithargsret{\sphinxbfcode{\sphinxupquote{getVolLCFS}}}{}{}
Abstract method.  See child classes for implementation.

Returns plasma volume within LCFS {[}t{]}

\end{fulllineitems}

\index{getRmidPsi() (eqtools.core.Equilibrium method)@\spxentry{getRmidPsi()}\spxextra{eqtools.core.Equilibrium method}}

\begin{fulllineitems}
\phantomsection\label{\detokenize{eqtools:eqtools.core.Equilibrium.getRmidPsi}}\pysiglinewithargsret{\sphinxbfcode{\sphinxupquote{getRmidPsi}}}{}{}
Abstract method.  See child classes for implementation.

Returns outboard-midplane major radius of flux surface {[}t,psi{]}

\end{fulllineitems}

\index{getF() (eqtools.core.Equilibrium method)@\spxentry{getF()}\spxextra{eqtools.core.Equilibrium method}}

\begin{fulllineitems}
\phantomsection\label{\detokenize{eqtools:eqtools.core.Equilibrium.getF}}\pysiglinewithargsret{\sphinxbfcode{\sphinxupquote{getF}}}{}{}
Abstract method.  See child classes for implementation.

Returns F=RB\_\{Phi\}(Psi), often calculated for grad-shafranov solutions  {[}psi,t{]}

\end{fulllineitems}

\index{getFluxPres() (eqtools.core.Equilibrium method)@\spxentry{getFluxPres()}\spxextra{eqtools.core.Equilibrium method}}

\begin{fulllineitems}
\phantomsection\label{\detokenize{eqtools:eqtools.core.Equilibrium.getFluxPres}}\pysiglinewithargsret{\sphinxbfcode{\sphinxupquote{getFluxPres}}}{}{}
Abstract method.  See child classes for implementation.

Returns calculated pressure profile {[}psi,t{]}.
Psi assumed to be evenly-spaced grid on {[}0,1{]}

\end{fulllineitems}

\index{getFFPrime() (eqtools.core.Equilibrium method)@\spxentry{getFFPrime()}\spxextra{eqtools.core.Equilibrium method}}

\begin{fulllineitems}
\phantomsection\label{\detokenize{eqtools:eqtools.core.Equilibrium.getFFPrime}}\pysiglinewithargsret{\sphinxbfcode{\sphinxupquote{getFFPrime}}}{}{}
Abstract method.  See child classes for implementation.

Returns FF’ function used for grad-shafranov solutions {[}psi,t{]}

\end{fulllineitems}

\index{getPPrime() (eqtools.core.Equilibrium method)@\spxentry{getPPrime()}\spxextra{eqtools.core.Equilibrium method}}

\begin{fulllineitems}
\phantomsection\label{\detokenize{eqtools:eqtools.core.Equilibrium.getPPrime}}\pysiglinewithargsret{\sphinxbfcode{\sphinxupquote{getPPrime}}}{}{}
Abstract method.  See child classes for implementation.

Returns plasma pressure gradient as a function of psi {[}psi,t{]}

\end{fulllineitems}

\index{getElongation() (eqtools.core.Equilibrium method)@\spxentry{getElongation()}\spxextra{eqtools.core.Equilibrium method}}

\begin{fulllineitems}
\phantomsection\label{\detokenize{eqtools:eqtools.core.Equilibrium.getElongation}}\pysiglinewithargsret{\sphinxbfcode{\sphinxupquote{getElongation}}}{}{}
Abstract method.  See child classes for implementation.

Returns LCFS elongation {[}t{]}

\end{fulllineitems}

\index{getUpperTriangularity() (eqtools.core.Equilibrium method)@\spxentry{getUpperTriangularity()}\spxextra{eqtools.core.Equilibrium method}}

\begin{fulllineitems}
\phantomsection\label{\detokenize{eqtools:eqtools.core.Equilibrium.getUpperTriangularity}}\pysiglinewithargsret{\sphinxbfcode{\sphinxupquote{getUpperTriangularity}}}{}{}
Abstract method.  See child classes for implementation.

Returns LCFS upper triangularity {[}t{]}

\end{fulllineitems}

\index{getLowerTriangularity() (eqtools.core.Equilibrium method)@\spxentry{getLowerTriangularity()}\spxextra{eqtools.core.Equilibrium method}}

\begin{fulllineitems}
\phantomsection\label{\detokenize{eqtools:eqtools.core.Equilibrium.getLowerTriangularity}}\pysiglinewithargsret{\sphinxbfcode{\sphinxupquote{getLowerTriangularity}}}{}{}
Abstract method.  See child classes for implementation.

Returns LCFS lower triangularity {[}t{]}

\end{fulllineitems}

\index{getShaping() (eqtools.core.Equilibrium method)@\spxentry{getShaping()}\spxextra{eqtools.core.Equilibrium method}}

\begin{fulllineitems}
\phantomsection\label{\detokenize{eqtools:eqtools.core.Equilibrium.getShaping}}\pysiglinewithargsret{\sphinxbfcode{\sphinxupquote{getShaping}}}{}{}
Abstract method.  See child classes for implementation.

Returns dimensionless shaping parameters for plasma.
Namedtuple containing \{LCFS elongation, LCFS upper/lower triangularity\}

\end{fulllineitems}

\index{getMagR() (eqtools.core.Equilibrium method)@\spxentry{getMagR()}\spxextra{eqtools.core.Equilibrium method}}

\begin{fulllineitems}
\phantomsection\label{\detokenize{eqtools:eqtools.core.Equilibrium.getMagR}}\pysiglinewithargsret{\sphinxbfcode{\sphinxupquote{getMagR}}}{}{}
Abstract method.  See child classes for implementation.

Returns magnetic-axis major radius {[}t{]}

\end{fulllineitems}

\index{getMagZ() (eqtools.core.Equilibrium method)@\spxentry{getMagZ()}\spxextra{eqtools.core.Equilibrium method}}

\begin{fulllineitems}
\phantomsection\label{\detokenize{eqtools:eqtools.core.Equilibrium.getMagZ}}\pysiglinewithargsret{\sphinxbfcode{\sphinxupquote{getMagZ}}}{}{}
Abstract method.  See child classes for implementation.

Returns magnetic-axis Z {[}t{]}

\end{fulllineitems}

\index{getAreaLCFS() (eqtools.core.Equilibrium method)@\spxentry{getAreaLCFS()}\spxextra{eqtools.core.Equilibrium method}}

\begin{fulllineitems}
\phantomsection\label{\detokenize{eqtools:eqtools.core.Equilibrium.getAreaLCFS}}\pysiglinewithargsret{\sphinxbfcode{\sphinxupquote{getAreaLCFS}}}{}{}
Abstract method.  See child classes for implementation.

Returns LCFS surface area {[}t{]}

\end{fulllineitems}

\index{getAOut() (eqtools.core.Equilibrium method)@\spxentry{getAOut()}\spxextra{eqtools.core.Equilibrium method}}

\begin{fulllineitems}
\phantomsection\label{\detokenize{eqtools:eqtools.core.Equilibrium.getAOut}}\pysiglinewithargsret{\sphinxbfcode{\sphinxupquote{getAOut}}}{}{}
Abstract method.  See child classes for implementation.

Returns outboard-midplane minor radius {[}t{]}

\end{fulllineitems}

\index{getRmidOut() (eqtools.core.Equilibrium method)@\spxentry{getRmidOut()}\spxextra{eqtools.core.Equilibrium method}}

\begin{fulllineitems}
\phantomsection\label{\detokenize{eqtools:eqtools.core.Equilibrium.getRmidOut}}\pysiglinewithargsret{\sphinxbfcode{\sphinxupquote{getRmidOut}}}{}{}
Abstract method.  See child classes for implementation.

Returns outboard-midplane major radius {[}t{]}

\end{fulllineitems}

\index{getGeometry() (eqtools.core.Equilibrium method)@\spxentry{getGeometry()}\spxextra{eqtools.core.Equilibrium method}}

\begin{fulllineitems}
\phantomsection\label{\detokenize{eqtools:eqtools.core.Equilibrium.getGeometry}}\pysiglinewithargsret{\sphinxbfcode{\sphinxupquote{getGeometry}}}{}{}
Abstract method.  See child classes for implementation.

Returns dimensional geometry parameters
Namedtuple containing \{mag axis R,Z, LCFS area, volume, outboard-midplane major radius\}

\end{fulllineitems}

\index{getQProfile() (eqtools.core.Equilibrium method)@\spxentry{getQProfile()}\spxextra{eqtools.core.Equilibrium method}}

\begin{fulllineitems}
\phantomsection\label{\detokenize{eqtools:eqtools.core.Equilibrium.getQProfile}}\pysiglinewithargsret{\sphinxbfcode{\sphinxupquote{getQProfile}}}{}{}
Abstract method.  See child classes for implementation.

Returns safety factor q profile {[}psi,t{]}
Psi assumed to be evenly-spaced grid on {[}0,1{]}

\end{fulllineitems}

\index{getQ0() (eqtools.core.Equilibrium method)@\spxentry{getQ0()}\spxextra{eqtools.core.Equilibrium method}}

\begin{fulllineitems}
\phantomsection\label{\detokenize{eqtools:eqtools.core.Equilibrium.getQ0}}\pysiglinewithargsret{\sphinxbfcode{\sphinxupquote{getQ0}}}{}{}
Abstract method.  See child classes for implementation.

Returns q on magnetic axis {[}t{]}

\end{fulllineitems}

\index{getQ95() (eqtools.core.Equilibrium method)@\spxentry{getQ95()}\spxextra{eqtools.core.Equilibrium method}}

\begin{fulllineitems}
\phantomsection\label{\detokenize{eqtools:eqtools.core.Equilibrium.getQ95}}\pysiglinewithargsret{\sphinxbfcode{\sphinxupquote{getQ95}}}{}{}
Abstract method.  See child classes for implementation.

Returns q on 95\% flux surface {[}t{]}

\end{fulllineitems}

\index{getQLCFS() (eqtools.core.Equilibrium method)@\spxentry{getQLCFS()}\spxextra{eqtools.core.Equilibrium method}}

\begin{fulllineitems}
\phantomsection\label{\detokenize{eqtools:eqtools.core.Equilibrium.getQLCFS}}\pysiglinewithargsret{\sphinxbfcode{\sphinxupquote{getQLCFS}}}{}{}
Abstract method.  See child classes for implementation.

Returns q on LCFS {[}t{]}

\end{fulllineitems}

\index{getQ1Surf() (eqtools.core.Equilibrium method)@\spxentry{getQ1Surf()}\spxextra{eqtools.core.Equilibrium method}}

\begin{fulllineitems}
\phantomsection\label{\detokenize{eqtools:eqtools.core.Equilibrium.getQ1Surf}}\pysiglinewithargsret{\sphinxbfcode{\sphinxupquote{getQ1Surf}}}{}{}
Abstract method.  See child classes for implementation.

Returns outboard-midplane minor radius of q=1 surface {[}t{]}

\end{fulllineitems}

\index{getQ2Surf() (eqtools.core.Equilibrium method)@\spxentry{getQ2Surf()}\spxextra{eqtools.core.Equilibrium method}}

\begin{fulllineitems}
\phantomsection\label{\detokenize{eqtools:eqtools.core.Equilibrium.getQ2Surf}}\pysiglinewithargsret{\sphinxbfcode{\sphinxupquote{getQ2Surf}}}{}{}
Abstract method.  See child classes for implementation.

Returns outboard-midplane minor radius of q=2 surface {[}t{]}

\end{fulllineitems}

\index{getQ3Surf() (eqtools.core.Equilibrium method)@\spxentry{getQ3Surf()}\spxextra{eqtools.core.Equilibrium method}}

\begin{fulllineitems}
\phantomsection\label{\detokenize{eqtools:eqtools.core.Equilibrium.getQ3Surf}}\pysiglinewithargsret{\sphinxbfcode{\sphinxupquote{getQ3Surf}}}{}{}
Abstract method.  See child classes for implementation.

Returns outboard-midplane minor radius of q=3 surface {[}t{]}

\end{fulllineitems}

\index{getQs() (eqtools.core.Equilibrium method)@\spxentry{getQs()}\spxextra{eqtools.core.Equilibrium method}}

\begin{fulllineitems}
\phantomsection\label{\detokenize{eqtools:eqtools.core.Equilibrium.getQs}}\pysiglinewithargsret{\sphinxbfcode{\sphinxupquote{getQs}}}{}{}
Abstract method.  See child classes for implementation.

Returns specific q-profile values.
Namedtuple containing \{q0, q95, qLCFS, minor radius of q=1,2,3 surfaces\}

\end{fulllineitems}

\index{getBtVac() (eqtools.core.Equilibrium method)@\spxentry{getBtVac()}\spxextra{eqtools.core.Equilibrium method}}

\begin{fulllineitems}
\phantomsection\label{\detokenize{eqtools:eqtools.core.Equilibrium.getBtVac}}\pysiglinewithargsret{\sphinxbfcode{\sphinxupquote{getBtVac}}}{}{}
Abstract method.  See child classes for implementation.

Returns vacuum on-axis toroidal field {[}t{]}

\end{fulllineitems}

\index{getBtPla() (eqtools.core.Equilibrium method)@\spxentry{getBtPla()}\spxextra{eqtools.core.Equilibrium method}}

\begin{fulllineitems}
\phantomsection\label{\detokenize{eqtools:eqtools.core.Equilibrium.getBtPla}}\pysiglinewithargsret{\sphinxbfcode{\sphinxupquote{getBtPla}}}{}{}
Abstract method.  See child classes for implementation.

Returns plasma on-axis toroidal field {[}t{]}

\end{fulllineitems}

\index{getBpAvg() (eqtools.core.Equilibrium method)@\spxentry{getBpAvg()}\spxextra{eqtools.core.Equilibrium method}}

\begin{fulllineitems}
\phantomsection\label{\detokenize{eqtools:eqtools.core.Equilibrium.getBpAvg}}\pysiglinewithargsret{\sphinxbfcode{\sphinxupquote{getBpAvg}}}{}{}
Abstract method.  See child classes for implementation.

Returns average poloidal field {[}t{]}

\end{fulllineitems}

\index{getFields() (eqtools.core.Equilibrium method)@\spxentry{getFields()}\spxextra{eqtools.core.Equilibrium method}}

\begin{fulllineitems}
\phantomsection\label{\detokenize{eqtools:eqtools.core.Equilibrium.getFields}}\pysiglinewithargsret{\sphinxbfcode{\sphinxupquote{getFields}}}{}{}
Abstract method.  See child classes for implementation.

Returns magnetic-field values.
Namedtuple containing \{Btor on magnetic axis (plasma and vacuum), avg Bpol\}

\end{fulllineitems}

\index{getIpCalc() (eqtools.core.Equilibrium method)@\spxentry{getIpCalc()}\spxextra{eqtools.core.Equilibrium method}}

\begin{fulllineitems}
\phantomsection\label{\detokenize{eqtools:eqtools.core.Equilibrium.getIpCalc}}\pysiglinewithargsret{\sphinxbfcode{\sphinxupquote{getIpCalc}}}{}{}
Abstract method.  See child classes for implementation.

Returns calculated plasma current {[}t{]}

\end{fulllineitems}

\index{getIpMeas() (eqtools.core.Equilibrium method)@\spxentry{getIpMeas()}\spxextra{eqtools.core.Equilibrium method}}

\begin{fulllineitems}
\phantomsection\label{\detokenize{eqtools:eqtools.core.Equilibrium.getIpMeas}}\pysiglinewithargsret{\sphinxbfcode{\sphinxupquote{getIpMeas}}}{}{}
Abstract method.  See child classes for implementation.

Returns measured plasma current {[}t{]}

\end{fulllineitems}

\index{getJp() (eqtools.core.Equilibrium method)@\spxentry{getJp()}\spxextra{eqtools.core.Equilibrium method}}

\begin{fulllineitems}
\phantomsection\label{\detokenize{eqtools:eqtools.core.Equilibrium.getJp}}\pysiglinewithargsret{\sphinxbfcode{\sphinxupquote{getJp}}}{}{}
Abstract method.  See child classes for implementation.

Returns grid of calculated toroidal current density {[}t,z,r{]}

\end{fulllineitems}

\index{getBetaT() (eqtools.core.Equilibrium method)@\spxentry{getBetaT()}\spxextra{eqtools.core.Equilibrium method}}

\begin{fulllineitems}
\phantomsection\label{\detokenize{eqtools:eqtools.core.Equilibrium.getBetaT}}\pysiglinewithargsret{\sphinxbfcode{\sphinxupquote{getBetaT}}}{}{}
Abstract method.  See child classes for implementation.

Returns calculated global toroidal beta {[}t{]}

\end{fulllineitems}

\index{getBetaP() (eqtools.core.Equilibrium method)@\spxentry{getBetaP()}\spxextra{eqtools.core.Equilibrium method}}

\begin{fulllineitems}
\phantomsection\label{\detokenize{eqtools:eqtools.core.Equilibrium.getBetaP}}\pysiglinewithargsret{\sphinxbfcode{\sphinxupquote{getBetaP}}}{}{}
Abstract method.  See child classes for implementation.

Returns calculated global poloidal beta {[}t{]}

\end{fulllineitems}

\index{getLi() (eqtools.core.Equilibrium method)@\spxentry{getLi()}\spxextra{eqtools.core.Equilibrium method}}

\begin{fulllineitems}
\phantomsection\label{\detokenize{eqtools:eqtools.core.Equilibrium.getLi}}\pysiglinewithargsret{\sphinxbfcode{\sphinxupquote{getLi}}}{}{}
Abstract method.  See child classes for implementation.

Returns calculated internal inductance of plasma {[}t{]}

\end{fulllineitems}

\index{getBetas() (eqtools.core.Equilibrium method)@\spxentry{getBetas()}\spxextra{eqtools.core.Equilibrium method}}

\begin{fulllineitems}
\phantomsection\label{\detokenize{eqtools:eqtools.core.Equilibrium.getBetas}}\pysiglinewithargsret{\sphinxbfcode{\sphinxupquote{getBetas}}}{}{}
Abstract method.  See child classes for implementation.

Returns calculated betas and inductance.
Namedtuple of \{betat,betap,Li\}

\end{fulllineitems}

\index{getDiamagFlux() (eqtools.core.Equilibrium method)@\spxentry{getDiamagFlux()}\spxextra{eqtools.core.Equilibrium method}}

\begin{fulllineitems}
\phantomsection\label{\detokenize{eqtools:eqtools.core.Equilibrium.getDiamagFlux}}\pysiglinewithargsret{\sphinxbfcode{\sphinxupquote{getDiamagFlux}}}{}{}
Abstract method.  See child classes for implementation.

Returns diamagnetic flux {[}t{]}

\end{fulllineitems}

\index{getDiamagBetaT() (eqtools.core.Equilibrium method)@\spxentry{getDiamagBetaT()}\spxextra{eqtools.core.Equilibrium method}}

\begin{fulllineitems}
\phantomsection\label{\detokenize{eqtools:eqtools.core.Equilibrium.getDiamagBetaT}}\pysiglinewithargsret{\sphinxbfcode{\sphinxupquote{getDiamagBetaT}}}{}{}
Abstract method.  See child classes for implementation.

Returns diamagnetic-loop toroidal beta {[}t{]}

\end{fulllineitems}

\index{getDiamagBetaP() (eqtools.core.Equilibrium method)@\spxentry{getDiamagBetaP()}\spxextra{eqtools.core.Equilibrium method}}

\begin{fulllineitems}
\phantomsection\label{\detokenize{eqtools:eqtools.core.Equilibrium.getDiamagBetaP}}\pysiglinewithargsret{\sphinxbfcode{\sphinxupquote{getDiamagBetaP}}}{}{}
Abstract method.  See child classes for implementation.

Returns diamagnetic-loop poloidal beta {[}t{]}

\end{fulllineitems}

\index{getDiamagTauE() (eqtools.core.Equilibrium method)@\spxentry{getDiamagTauE()}\spxextra{eqtools.core.Equilibrium method}}

\begin{fulllineitems}
\phantomsection\label{\detokenize{eqtools:eqtools.core.Equilibrium.getDiamagTauE}}\pysiglinewithargsret{\sphinxbfcode{\sphinxupquote{getDiamagTauE}}}{}{}
Abstract method.  See child classes for implementation.

Returns diamagnetic-loop energy confinement time {[}t{]}

\end{fulllineitems}

\index{getDiamagWp() (eqtools.core.Equilibrium method)@\spxentry{getDiamagWp()}\spxextra{eqtools.core.Equilibrium method}}

\begin{fulllineitems}
\phantomsection\label{\detokenize{eqtools:eqtools.core.Equilibrium.getDiamagWp}}\pysiglinewithargsret{\sphinxbfcode{\sphinxupquote{getDiamagWp}}}{}{}
Abstract method.  See child classes for implementation.

Returns diamagnetic-loop plasma stored energy {[}t{]}

\end{fulllineitems}

\index{getDiamag() (eqtools.core.Equilibrium method)@\spxentry{getDiamag()}\spxextra{eqtools.core.Equilibrium method}}

\begin{fulllineitems}
\phantomsection\label{\detokenize{eqtools:eqtools.core.Equilibrium.getDiamag}}\pysiglinewithargsret{\sphinxbfcode{\sphinxupquote{getDiamag}}}{}{}
Abstract method.  See child classes for implementation.

Returns diamagnetic measurements of plasma parameters.
Namedtuple of \{diamag. flux, betat, betap from coils, tau\_E from diamag., diamag. stored energy\}

\end{fulllineitems}

\index{getWMHD() (eqtools.core.Equilibrium method)@\spxentry{getWMHD()}\spxextra{eqtools.core.Equilibrium method}}

\begin{fulllineitems}
\phantomsection\label{\detokenize{eqtools:eqtools.core.Equilibrium.getWMHD}}\pysiglinewithargsret{\sphinxbfcode{\sphinxupquote{getWMHD}}}{}{}
Abstract method.  See child classes for implementation.

Returns calculated MHD stored energy {[}t{]}

\end{fulllineitems}

\index{getTauMHD() (eqtools.core.Equilibrium method)@\spxentry{getTauMHD()}\spxextra{eqtools.core.Equilibrium method}}

\begin{fulllineitems}
\phantomsection\label{\detokenize{eqtools:eqtools.core.Equilibrium.getTauMHD}}\pysiglinewithargsret{\sphinxbfcode{\sphinxupquote{getTauMHD}}}{}{}
Abstract method.  See child classes for implementation.

Returns calculated MHD energy confinement time {[}t{]}

\end{fulllineitems}

\index{getPinj() (eqtools.core.Equilibrium method)@\spxentry{getPinj()}\spxextra{eqtools.core.Equilibrium method}}

\begin{fulllineitems}
\phantomsection\label{\detokenize{eqtools:eqtools.core.Equilibrium.getPinj}}\pysiglinewithargsret{\sphinxbfcode{\sphinxupquote{getPinj}}}{}{}
Abstract method.  See child classes for implementation.

Returns calculated injected power {[}t{]}

\end{fulllineitems}

\index{getCurrentSign() (eqtools.core.Equilibrium method)@\spxentry{getCurrentSign()}\spxextra{eqtools.core.Equilibrium method}}

\begin{fulllineitems}
\phantomsection\label{\detokenize{eqtools:eqtools.core.Equilibrium.getCurrentSign}}\pysiglinewithargsret{\sphinxbfcode{\sphinxupquote{getCurrentSign}}}{}{}
Abstract method.  See child classes for implementation.

Returns calculated current direction, where CCW = +

\end{fulllineitems}

\index{getWbdot() (eqtools.core.Equilibrium method)@\spxentry{getWbdot()}\spxextra{eqtools.core.Equilibrium method}}

\begin{fulllineitems}
\phantomsection\label{\detokenize{eqtools:eqtools.core.Equilibrium.getWbdot}}\pysiglinewithargsret{\sphinxbfcode{\sphinxupquote{getWbdot}}}{}{}
Abstract method.  See child classes for implementation.

Returns calculated d/dt of magnetic stored energy {[}t{]}

\end{fulllineitems}

\index{getWpdot() (eqtools.core.Equilibrium method)@\spxentry{getWpdot()}\spxextra{eqtools.core.Equilibrium method}}

\begin{fulllineitems}
\phantomsection\label{\detokenize{eqtools:eqtools.core.Equilibrium.getWpdot}}\pysiglinewithargsret{\sphinxbfcode{\sphinxupquote{getWpdot}}}{}{}
Abstract method.  See child classes for implementation.

Returns calculated d/dt of plasma stored energy {[}t{]}

\end{fulllineitems}

\index{getBCentr() (eqtools.core.Equilibrium method)@\spxentry{getBCentr()}\spxextra{eqtools.core.Equilibrium method}}

\begin{fulllineitems}
\phantomsection\label{\detokenize{eqtools:eqtools.core.Equilibrium.getBCentr}}\pysiglinewithargsret{\sphinxbfcode{\sphinxupquote{getBCentr}}}{}{}
Abstract method.  See child classes for implementation.

Returns Vacuum Toroidal magnetic field at Rcent point {[}t{]}

\end{fulllineitems}

\index{getRCentr() (eqtools.core.Equilibrium method)@\spxentry{getRCentr()}\spxextra{eqtools.core.Equilibrium method}}

\begin{fulllineitems}
\phantomsection\label{\detokenize{eqtools:eqtools.core.Equilibrium.getRCentr}}\pysiglinewithargsret{\sphinxbfcode{\sphinxupquote{getRCentr}}}{}{}
Abstract method.  See child classes for implementation.

Radial position for Vacuum Toroidal magnetic field calculation

\end{fulllineitems}

\index{getEnergy() (eqtools.core.Equilibrium method)@\spxentry{getEnergy()}\spxextra{eqtools.core.Equilibrium method}}

\begin{fulllineitems}
\phantomsection\label{\detokenize{eqtools:eqtools.core.Equilibrium.getEnergy}}\pysiglinewithargsret{\sphinxbfcode{\sphinxupquote{getEnergy}}}{}{}
Abstract method.  See child classes for implementation.

Returns stored-energy parameters.
Namedtuple of \{stored energy, confinement time, injected power, d/dt of magnetic, plasma stored energy\}

\end{fulllineitems}

\index{getParam() (eqtools.core.Equilibrium method)@\spxentry{getParam()}\spxextra{eqtools.core.Equilibrium method}}

\begin{fulllineitems}
\phantomsection\label{\detokenize{eqtools:eqtools.core.Equilibrium.getParam}}\pysiglinewithargsret{\sphinxbfcode{\sphinxupquote{getParam}}}{\emph{path}}{}
Abstract method.  See child classes for implementation.

Backup function: takes parameter name for variable, returns variable directly.
Acts as wrapper to direct data-access routines from within object.

\end{fulllineitems}

\index{getMachineCrossSection() (eqtools.core.Equilibrium method)@\spxentry{getMachineCrossSection()}\spxextra{eqtools.core.Equilibrium method}}

\begin{fulllineitems}
\phantomsection\label{\detokenize{eqtools:eqtools.core.Equilibrium.getMachineCrossSection}}\pysiglinewithargsret{\sphinxbfcode{\sphinxupquote{getMachineCrossSection}}}{}{}
Abstract method.  See child classes for implementation.

Returns (R,Z) coordinates of vacuum wall cross-section for plotting/masking routines.

\end{fulllineitems}

\index{getMachineCrossSectionFull() (eqtools.core.Equilibrium method)@\spxentry{getMachineCrossSectionFull()}\spxextra{eqtools.core.Equilibrium method}}

\begin{fulllineitems}
\phantomsection\label{\detokenize{eqtools:eqtools.core.Equilibrium.getMachineCrossSectionFull}}\pysiglinewithargsret{\sphinxbfcode{\sphinxupquote{getMachineCrossSectionFull}}}{}{}
Abstract method.  See child classes for implementation.

Returns (R,Z) coordinates of machine wall cross-section for plotting routines.
Returns a more detailed cross-section than getLimiter(), generally a vector map
displaying non-critical cross-section information.  If this is unavailable, this
should point to self.getMachineCrossSection(), which pulls the limiter outline
stored by default in data files e.g. g-eqdsk files.

\end{fulllineitems}

\index{gfile() (eqtools.core.Equilibrium method)@\spxentry{gfile()}\spxextra{eqtools.core.Equilibrium method}}

\begin{fulllineitems}
\phantomsection\label{\detokenize{eqtools:eqtools.core.Equilibrium.gfile}}\pysiglinewithargsret{\sphinxbfcode{\sphinxupquote{gfile}}}{\emph{time=None}, \emph{nw=None}, \emph{nh=None}, \emph{shot=None}, \emph{name=None}, \emph{tunit='ms'}, \emph{title='EQTOOLS'}, \emph{nbbbs=100}}{}
Generates an EFIT gfile with gfile naming convention
\begin{quote}\begin{description}
\item[{Keyword Arguments}] \leavevmode\begin{itemize}
\item {} 
\sphinxstyleliteralstrong{\sphinxupquote{time}} (\sphinxstyleliteralemphasis{\sphinxupquote{scalar float}}) \textendash{} Time of equilibrium to
generate the gfile from. This will use the specified
spline functionality to do so. Allows for it to be
unspecified for single-time-frame equilibria.

\item {} 
\sphinxstyleliteralstrong{\sphinxupquote{nw}} (\sphinxstyleliteralemphasis{\sphinxupquote{scalar integer}}) \textendash{} Number of points in R.
R is the major radius, and describes the ‘width’ of the
gfile.

\item {} 
\sphinxstyleliteralstrong{\sphinxupquote{nh}} (\sphinxstyleliteralemphasis{\sphinxupquote{scalar integer}}) \textendash{} Number of points in Z. In cylindrical
coordinates Z is the height, and nh describes the ‘height’
of the gfile.

\item {} 
\sphinxstyleliteralstrong{\sphinxupquote{shot}} (\sphinxstyleliteralemphasis{\sphinxupquote{scalar integer}}) \textendash{} The shot numer of the equilibrium.
Used to help generate the gfile name if unspecified.

\item {} 
\sphinxstyleliteralstrong{\sphinxupquote{name}} (\sphinxstyleliteralemphasis{\sphinxupquote{String}}) \textendash{} Name of the gfile.  If unspecified, will follow
standard gfile naming convention (g+shot.time) under current
python operating directory.  This allows for it to be saved
in other directories, etc.

\item {} 
\sphinxstyleliteralstrong{\sphinxupquote{tunit}} (\sphinxstyleliteralemphasis{\sphinxupquote{String}}) \textendash{} Specified unit for tin. It can only be ‘ms’ for
milliseconds or ‘s’ for seconds.

\item {} 
\sphinxstyleliteralstrong{\sphinxupquote{title}} (\sphinxstyleliteralemphasis{\sphinxupquote{String}}) \textendash{} Title of the gfile on the first line. Name cannot
exceed 10 digits. This is so that the style of the first line
is preserved.

\item {} 
\sphinxstyleliteralstrong{\sphinxupquote{nbbbs}} (\sphinxstyleliteralemphasis{\sphinxupquote{scalar integer}}) \textendash{} Number of points to define the plasma
seperatrix within the gfile.  The points are defined equally
spaced in angle about the plasma center.  This will cause the
x-point to be poorly defined.

\end{itemize}

\item[{Raises}] \leavevmode
\sphinxstyleliteralstrong{\sphinxupquote{ValueError}} \textendash{} If title is longer than 10 characters.

\end{description}\end{quote}
\subsubsection*{Examples}

All assume that \sphinxtitleref{Eq\_instance} is a valid instance of the appropriate
extension of the {\hyperref[\detokenize{eqtools:eqtools.core.Equilibrium}]{\sphinxcrossref{\sphinxcode{\sphinxupquote{Equilibrium}}}}} abstract class (example
shot number of 1001).

Generate a gfile at t=0.26s, output of g1001.26:

\begin{sphinxVerbatim}[commandchars=\\\{\}]
\PYG{n}{Eq\PYGZus{}instance}\PYG{o}{.}\PYG{n}{gfile}\PYG{p}{(}\PYG{o}{.}\PYG{l+m+mi}{26}\PYG{p}{)}
\end{sphinxVerbatim}

\end{fulllineitems}

\index{plotFlux() (eqtools.core.Equilibrium method)@\spxentry{plotFlux()}\spxextra{eqtools.core.Equilibrium method}}

\begin{fulllineitems}
\phantomsection\label{\detokenize{eqtools:eqtools.core.Equilibrium.plotFlux}}\pysiglinewithargsret{\sphinxbfcode{\sphinxupquote{plotFlux}}}{\emph{fill=True}, \emph{mask=True}, \emph{lw=3.0}, \emph{add\_title=True}}{}
Plots flux contours directly from psi grid.

Returns the Figure instance created and the time slider widget (in case
you need to modify the callback). \sphinxtitleref{f.axes} contains the contour plot as
the first element and the time slice slider as the second element.
\begin{quote}\begin{description}
\item[{Keyword Arguments}] \leavevmode\begin{itemize}
\item {} 
\sphinxstyleliteralstrong{\sphinxupquote{fill}} (\sphinxstyleliteralemphasis{\sphinxupquote{Boolean}}) \textendash{} Set True to plot filled contours.  Set False (default) to plot white-background
color contours.

\item {} 
\sphinxstyleliteralstrong{\sphinxupquote{mask}} (\sphinxstyleliteralemphasis{\sphinxupquote{Boolean}}) \textendash{} Set True (default) to mask the contours according to the vacuum
vessel outline.

\item {} 
\sphinxstyleliteralstrong{\sphinxupquote{lw}} (\sphinxstyleliteralemphasis{\sphinxupquote{float}}) \textendash{} Linewidth when plotting LCFS. Default is 3.0.

\item {} 
\sphinxstyleliteralstrong{\sphinxupquote{add\_title}} (\sphinxstyleliteralemphasis{\sphinxupquote{Boolean}}) \textendash{} Set True (default) to add a figure title with the time indicated.

\end{itemize}

\end{description}\end{quote}

\end{fulllineitems}


\end{fulllineitems}



\subsection{eqtools.eqdskreader module}
\label{\detokenize{eqtools:module-eqtools.eqdskreader}}\label{\detokenize{eqtools:eqtools-eqdskreader-module}}\index{eqtools.eqdskreader (module)@\spxentry{eqtools.eqdskreader}\spxextra{module}}
This module contains the EqdskReader class, which creates Equilibrium class
functionality for equilibria stored in eqdsk files from EFIT(a- and g-files).
\begin{description}
\item[{Classes:}] \leavevmode\begin{description}
\item[{EqdskReader:}] \leavevmode
Class inheriting Equilibrium reading g- and a-files for
equilibrium data.

\end{description}

\end{description}
\index{EqdskReader (class in eqtools.eqdskreader)@\spxentry{EqdskReader}\spxextra{class in eqtools.eqdskreader}}

\begin{fulllineitems}
\phantomsection\label{\detokenize{eqtools:eqtools.eqdskreader.EqdskReader}}\pysiglinewithargsret{\sphinxbfcode{\sphinxupquote{class }}\sphinxcode{\sphinxupquote{eqtools.eqdskreader.}}\sphinxbfcode{\sphinxupquote{EqdskReader}}}{\emph{shot=None}, \emph{time=None}, \emph{gfile=None}, \emph{afile=None}, \emph{length\_unit='m'}, \emph{verbose=True}}{}
Bases: {\hyperref[\detokenize{eqtools:eqtools.core.Equilibrium}]{\sphinxcrossref{\sphinxcode{\sphinxupquote{eqtools.core.Equilibrium}}}}}

Equilibrium subclass working from eqdsk ASCII-file equilibria.

Inherits mapping and structural data from Equilibrium, populates equilibrium
and profile data from g- and a-files for a selected shot and time window.

Create instance of EqdskReader.

Generates object and reads data from selected g-file (either manually set or
autodetected based on user shot and time selection), storing as object
attributes for usage in Equilibrium mapping methods.

Calling structure - user may call class with shot and time (ms) values, set
by keywords (or positional placement allows calling without explicit keyword
syntax).  EqdskReader then attempts to construct filenames from the
shot/time, of the form ‘g{[}shot{]}.{[}time{]}’ and ‘a{[}shot{]}.{[}time{]}’.  Alternately,
the user may skip this input and explicitly set paths to the g- and/or
a-files, using the gfile and afile keyword arguments.  If both types of
calls are set, the explicit g-file and a-file paths override the
auto-generated filenames from the shot and time.
\begin{quote}\begin{description}
\item[{Keyword Arguments}] \leavevmode\begin{itemize}
\item {} 
\sphinxstyleliteralstrong{\sphinxupquote{shot}} (\sphinxstyleliteralemphasis{\sphinxupquote{Integer}}) \textendash{} Shot index.

\item {} 
\sphinxstyleliteralstrong{\sphinxupquote{time}} (\sphinxstyleliteralemphasis{\sphinxupquote{Integer}}) \textendash{} Time index (typically ms).  Shot and Time used to
autogenerate filenames.

\item {} 
\sphinxstyleliteralstrong{\sphinxupquote{gfile}} (\sphinxstyleliteralemphasis{\sphinxupquote{String}}) \textendash{} Manually selects ASCII file for equilibrium read.

\item {} 
\sphinxstyleliteralstrong{\sphinxupquote{afile}} (\sphinxstyleliteralemphasis{\sphinxupquote{String}}) \textendash{} Manually selects ASCII file for time-history read.

\item {} 
\sphinxstyleliteralstrong{\sphinxupquote{length\_unit}} (\sphinxstyleliteralemphasis{\sphinxupquote{String}}) \textendash{} Flag setting length unit for equilibrium scales.
Defaults to ‘m’ for lengths in meters.

\item {} 
\sphinxstyleliteralstrong{\sphinxupquote{verbose}} (\sphinxstyleliteralemphasis{\sphinxupquote{Boolean}}) \textendash{} When set to False, suppresses terminal outputs during
CSV read.  Defaults to True (prints terminal output).

\end{itemize}

\item[{Raises}] \leavevmode\begin{itemize}
\item {} 
\sphinxstyleliteralstrong{\sphinxupquote{IOError}} \textendash{} if both name/shot and explicit filenames are not set.

\item {} 
\sphinxstyleliteralstrong{\sphinxupquote{ValueError}} \textendash{} if the g-file cannot be found, or if multiple valid
    g/a-files are found.

\end{itemize}

\end{description}\end{quote}
\subsubsection*{Examples}

Instantiate EqdskReader for a given \sphinxtitleref{shot} and \sphinxtitleref{time} \textendash{} will search current
working directory for files of the form g{[}shot{]}.{[}time{]} and
a{[}shot{]}.{[}time{]}, suppressing terminal outputs:

\begin{sphinxVerbatim}[commandchars=\\\{\}]
\PYG{n}{edr} \PYG{o}{=} \PYG{n}{eqtools}\PYG{o}{.}\PYG{n}{EqdskReader}\PYG{p}{(}\PYG{n}{shot}\PYG{p}{,}\PYG{n}{time}\PYG{p}{,}\PYG{n}{verbose}\PYG{o}{=}\PYG{k+kc}{False}\PYG{p}{)}
\end{sphinxVerbatim}

or:

\begin{sphinxVerbatim}[commandchars=\\\{\}]
\PYG{n}{edr} \PYG{o}{=} \PYG{n}{eqtools}\PYG{o}{.}\PYG{n}{EqdskReader}\PYG{p}{(}\PYG{n}{shot}\PYG{o}{=}\PYG{n}{shot}\PYG{p}{,}\PYG{n}{time}\PYG{o}{=}\PYG{n}{time}\PYG{p}{,}\PYG{n}{verbose}\PYG{o}{=}\PYG{k+kc}{False}\PYG{p}{)}
\end{sphinxVerbatim}

Instantiate EqdskReader with explicit file paths \sphinxtitleref{gfile\_path} and
\sphinxtitleref{afile\_path}:

\begin{sphinxVerbatim}[commandchars=\\\{\}]
\PYG{n}{edr} \PYG{o}{=} \PYG{n}{eqtools}\PYG{o}{.}\PYG{n}{EqdskReader}\PYG{p}{(}\PYG{n}{gfile}\PYG{o}{=}\PYG{n}{gfile\PYGZus{}path}\PYG{p}{,}\PYG{n}{afile}\PYG{o}{=}\PYG{n}{afile\PYGZus{}path}\PYG{p}{)}
\end{sphinxVerbatim}
\index{getInfo() (eqtools.eqdskreader.EqdskReader method)@\spxentry{getInfo()}\spxextra{eqtools.eqdskreader.EqdskReader method}}

\begin{fulllineitems}
\phantomsection\label{\detokenize{eqtools:eqtools.eqdskreader.EqdskReader.getInfo}}\pysiglinewithargsret{\sphinxbfcode{\sphinxupquote{getInfo}}}{}{}
returns namedtuple of equilibrium information
\begin{quote}\begin{description}
\item[{Returns}] \leavevmode

namedtuple containing
\begin{quote}


\begin{savenotes}\sphinxattablestart
\centering
\begin{tabulary}{\linewidth}[t]{|T|T|}
\hline

shot
&
shot index
\\
\hline
time
&
time point of g-file
\\
\hline
nr
&
size of R-axis of spatial grid
\\
\hline
nz
&
size of Z-axis of spatial grid
\\
\hline
efittype
&
EFIT calculation type (magnetic, kinetic, MSE)
\\
\hline
\end{tabulary}
\par
\sphinxattableend\end{savenotes}
\end{quote}


\end{description}\end{quote}

\end{fulllineitems}

\index{readAFile() (eqtools.eqdskreader.EqdskReader method)@\spxentry{readAFile()}\spxextra{eqtools.eqdskreader.EqdskReader method}}

\begin{fulllineitems}
\phantomsection\label{\detokenize{eqtools:eqtools.eqdskreader.EqdskReader.readAFile}}\pysiglinewithargsret{\sphinxbfcode{\sphinxupquote{readAFile}}}{\emph{afile}}{}
Reads a-file (scalar time-history data) to pull additional
equilibrium data not found in g-file, populates remaining data
(initialized as None) in object.
\begin{quote}\begin{description}
\item[{Parameters}] \leavevmode
\sphinxstyleliteralstrong{\sphinxupquote{afile}} (\sphinxstyleliteralemphasis{\sphinxupquote{String}}) \textendash{} Path to ASCII a-file.

\item[{Raises}] \leavevmode
\sphinxstyleliteralstrong{\sphinxupquote{IOError}} \textendash{} If afile is not found.

\end{description}\end{quote}

\end{fulllineitems}

\index{rz2psi() (eqtools.eqdskreader.EqdskReader method)@\spxentry{rz2psi()}\spxextra{eqtools.eqdskreader.EqdskReader method}}

\begin{fulllineitems}
\phantomsection\label{\detokenize{eqtools:eqtools.eqdskreader.EqdskReader.rz2psi}}\pysiglinewithargsret{\sphinxbfcode{\sphinxupquote{rz2psi}}}{\emph{R}, \emph{Z}, \emph{*args}, \emph{**kwargs}}{}
Calculates the non-normalized poloidal flux at the given (\sphinxtitleref{R}, \sphinxtitleref{Z}).
Wrapper for
{\hyperref[\detokenize{eqtools:eqtools.core.Equilibrium.rz2psi}]{\sphinxcrossref{\sphinxcode{\sphinxupquote{Equilibrium.rz2psi}}}}} masking
out timebase dependence.
\begin{quote}\begin{description}
\item[{Parameters}] \leavevmode\begin{itemize}
\item {} 
\sphinxstyleliteralstrong{\sphinxupquote{R}} (\sphinxstyleliteralemphasis{\sphinxupquote{Array-like}}\sphinxstyleliteralemphasis{\sphinxupquote{ or }}\sphinxstyleliteralemphasis{\sphinxupquote{scalar float}}) \textendash{} Values of the radial coordinate to
map to poloidal flux.  If \sphinxtitleref{R} and \sphinxtitleref{Z} are both scalar, then a
scalar \sphinxtitleref{psi} is returned.  \sphinxtitleref{R} and \sphinxtitleref{Z} must have the same shape
unless the \sphinxtitleref{make\_grid} keyword is set.  If \sphinxtitleref{make\_grid} is True,
\sphinxtitleref{R} must have shape (\sphinxtitleref{len\_R},).

\item {} 
\sphinxstyleliteralstrong{\sphinxupquote{Z}} (\sphinxstyleliteralemphasis{\sphinxupquote{Array-like}}\sphinxstyleliteralemphasis{\sphinxupquote{ or }}\sphinxstyleliteralemphasis{\sphinxupquote{scalar float}}) \textendash{} Values of the vertical coordinate to
map to poloidal flux.  If \sphinxtitleref{R} and \sphinxtitleref{Z} are both scalar, then a
scalar \sphinxtitleref{psi} is returned.  \sphinxtitleref{R} and \sphinxtitleref{Z} must have the same shape
unless the \sphinxtitleref{make\_grid} keyword is set.  If \sphinxtitleref{make\_grid} is True,
\sphinxtitleref{Z} must have shape (\sphinxtitleref{len\_Z},).

\end{itemize}

\end{description}\end{quote}

All keyword arguments are passed to the parent
{\hyperref[\detokenize{eqtools:eqtools.core.Equilibrium.rz2psi}]{\sphinxcrossref{\sphinxcode{\sphinxupquote{Equilibrium.rz2psi}}}}}.
Remaining arguments in *args are ignored.
\begin{quote}\begin{description}
\item[{Returns}] \leavevmode
non-normalized poloidal flux.  If
all input arguments are scalar, then \sphinxtitleref{psi} is scalar.  IF \sphinxtitleref{R} and \sphinxtitleref{Z}
have the same shape, then \sphinxtitleref{psi} has this shape as well.  If \sphinxtitleref{make\_grid}
is True, then \sphinxtitleref{psi} has the shape (\sphinxtitleref{len\_R}, \sphinxtitleref{len\_Z}).

\item[{Return type}] \leavevmode
psi (Array-like or scalar float)

\end{description}\end{quote}
\subsubsection*{Examples}

All assume that Eq\_instance is a valid instance EqdskReader:

Find single psi value at R=0.6m, Z=0.0m:

\begin{sphinxVerbatim}[commandchars=\\\{\}]
\PYG{n}{psi\PYGZus{}val} \PYG{o}{=} \PYG{n}{Eq\PYGZus{}instance}\PYG{o}{.}\PYG{n}{rz2psi}\PYG{p}{(}\PYG{l+m+mf}{0.6}\PYG{p}{,} \PYG{l+m+mi}{0}\PYG{p}{)}
\end{sphinxVerbatim}

Find psi values at (R, Z) points (0.6m, 0m) and (0.8m, 0m).
Note that the Z vector must be fully specified,
even if the values are all the same:

\begin{sphinxVerbatim}[commandchars=\\\{\}]
\PYG{n}{psi\PYGZus{}arr} \PYG{o}{=} \PYG{n}{Eq\PYGZus{}instance}\PYG{o}{.}\PYG{n}{rz2psi}\PYG{p}{(}\PYG{p}{[}\PYG{l+m+mf}{0.6}\PYG{p}{,} \PYG{l+m+mf}{0.8}\PYG{p}{]}\PYG{p}{,} \PYG{p}{[}\PYG{l+m+mi}{0}\PYG{p}{,} \PYG{l+m+mi}{0}\PYG{p}{]}\PYG{p}{)}
\end{sphinxVerbatim}

Find psi values on grid defined by 1D vector of radial positions
R and 1D vector of vertical positions Z:

\begin{sphinxVerbatim}[commandchars=\\\{\}]
\PYG{n}{psi\PYGZus{}mat} \PYG{o}{=} \PYG{n}{Eq\PYGZus{}instance}\PYG{o}{.}\PYG{n}{rz2psi}\PYG{p}{(}\PYG{n}{R}\PYG{p}{,} \PYG{n}{Z}\PYG{p}{,} \PYG{n}{make\PYGZus{}grid}\PYG{o}{=}\PYG{k+kc}{True}\PYG{p}{)}
\end{sphinxVerbatim}

\end{fulllineitems}

\index{rz2psinorm() (eqtools.eqdskreader.EqdskReader method)@\spxentry{rz2psinorm()}\spxextra{eqtools.eqdskreader.EqdskReader method}}

\begin{fulllineitems}
\phantomsection\label{\detokenize{eqtools:eqtools.eqdskreader.EqdskReader.rz2psinorm}}\pysiglinewithargsret{\sphinxbfcode{\sphinxupquote{rz2psinorm}}}{\emph{R}, \emph{Z}, \emph{*args}, \emph{**kwargs}}{}
Calculates the normalized poloidal flux at the given (R,Z).
Wrapper for
{\hyperref[\detokenize{eqtools:eqtools.core.Equilibrium.rz2psinorm}]{\sphinxcrossref{\sphinxcode{\sphinxupquote{Equilibrium.rz2psinorm}}}}}
masking out timebase dependence.

Uses the definition:
\begin{equation*}
\begin{split}\texttt{psi\_norm} = \frac{\psi - \psi(0)}{\psi(a) - \psi(0)}\end{split}
\end{equation*}\begin{quote}\begin{description}
\item[{Parameters}] \leavevmode\begin{itemize}
\item {} 
\sphinxstyleliteralstrong{\sphinxupquote{R}} (\sphinxstyleliteralemphasis{\sphinxupquote{Array-like}}\sphinxstyleliteralemphasis{\sphinxupquote{ or }}\sphinxstyleliteralemphasis{\sphinxupquote{scalar float}}) \textendash{} Values of the radial coordinate to
map to normalized poloidal flux.  Must have the same shape as
\sphinxtitleref{Z} unless the \sphinxtitleref{make\_grid} keyword is set. If the \sphinxtitleref{make\_grid}
keyword is True, \sphinxtitleref{R} must have shape (\sphinxtitleref{len\_R},).

\item {} 
\sphinxstyleliteralstrong{\sphinxupquote{Z}} (\sphinxstyleliteralemphasis{\sphinxupquote{Array-like}}\sphinxstyleliteralemphasis{\sphinxupquote{ or }}\sphinxstyleliteralemphasis{\sphinxupquote{scalar float}}) \textendash{} Values of the vertical coordinate to
map to normalized poloidal flux.  Must have the same shape as
\sphinxtitleref{R} unless the \sphinxtitleref{make\_grid} keyword is set. If the \sphinxtitleref{make\_grid}
keyword is True, \sphinxtitleref{Z} must have shape (\sphinxtitleref{len\_Z},).

\end{itemize}

\end{description}\end{quote}

All keyword arguments are passed to the parent
{\hyperref[\detokenize{eqtools:eqtools.core.Equilibrium.rz2psinorm}]{\sphinxcrossref{\sphinxcode{\sphinxupquote{Equilibrium.rz2psinorm}}}}}.
Remaining arguments in *args are ignored.
\begin{quote}\begin{description}
\item[{Returns}] \leavevmode
non-normalized poloidal flux.  If
all input arguments are scalar, then \sphinxtitleref{psinorm} is scalar.  IF \sphinxtitleref{R} and \sphinxtitleref{Z}
have the same shape, then \sphinxtitleref{psinorm} has this shape as well.  If \sphinxtitleref{make\_grid}
is True, then \sphinxtitleref{psinorm} has the shape (\sphinxtitleref{len\_R}, \sphinxtitleref{len\_Z}).

\item[{Return type}] \leavevmode
psinorm (Array-like or scalar float)

\end{description}\end{quote}
\subsubsection*{Examples}

All assume that Eq\_instance is a valid instance of EqdskReader:

Find single psinorm value at R=0.6m, Z=0.0m:

\begin{sphinxVerbatim}[commandchars=\\\{\}]
\PYG{n}{psi\PYGZus{}val} \PYG{o}{=} \PYG{n}{Eq\PYGZus{}instance}\PYG{o}{.}\PYG{n}{rz2psinorm}\PYG{p}{(}\PYG{l+m+mf}{0.6}\PYG{p}{,} \PYG{l+m+mi}{0}\PYG{p}{)}
\end{sphinxVerbatim}

Find psinorm values at (R, Z) points (0.6m, 0m) and (0.8m, 0m).
Note that the Z vector must be fully specified,
even if the values are all the same:

\begin{sphinxVerbatim}[commandchars=\\\{\}]
\PYG{n}{psi\PYGZus{}arr} \PYG{o}{=} \PYG{n}{Eq\PYGZus{}instance}\PYG{o}{.}\PYG{n}{rz2psinorm}\PYG{p}{(}\PYG{p}{[}\PYG{l+m+mf}{0.6}\PYG{p}{,} \PYG{l+m+mf}{0.8}\PYG{p}{]}\PYG{p}{,} \PYG{p}{[}\PYG{l+m+mi}{0}\PYG{p}{,} \PYG{l+m+mi}{0}\PYG{p}{]}\PYG{p}{)}
\end{sphinxVerbatim}

Find psinorm values on grid defined by 1D vector of radial positions
R and 1D vector of vertical positions Z:

\begin{sphinxVerbatim}[commandchars=\\\{\}]
\PYG{n}{psi\PYGZus{}mat} \PYG{o}{=} \PYG{n}{Eq\PYGZus{}instance}\PYG{o}{.}\PYG{n}{rz2psinorm}\PYG{p}{(}\PYG{n}{R}\PYG{p}{,} \PYG{n}{Z}\PYG{p}{,} \PYG{n}{make\PYGZus{}grid}\PYG{o}{=}\PYG{k+kc}{True}\PYG{p}{)}
\end{sphinxVerbatim}

\end{fulllineitems}

\index{rz2phinorm() (eqtools.eqdskreader.EqdskReader method)@\spxentry{rz2phinorm()}\spxextra{eqtools.eqdskreader.EqdskReader method}}

\begin{fulllineitems}
\phantomsection\label{\detokenize{eqtools:eqtools.eqdskreader.EqdskReader.rz2phinorm}}\pysiglinewithargsret{\sphinxbfcode{\sphinxupquote{rz2phinorm}}}{\emph{R}, \emph{Z}, \emph{*args}, \emph{**kwargs}}{}
Calculates normalized toroidal flux at a given (R,Z), using
\begin{equation*}
\begin{split}\texttt{phi} &= \int q(\psi)\,d\psi\\
\texttt{phi\_norm} &= \frac{\phi}{\phi(a)}\end{split}
\end{equation*}
Wrapper for
{\hyperref[\detokenize{eqtools:eqtools.core.Equilibrium.rz2phinorm}]{\sphinxcrossref{\sphinxcode{\sphinxupquote{Equilibrium.rz2phinorm}}}}}
masking out timebase dependence.
\begin{quote}\begin{description}
\item[{Parameters}] \leavevmode\begin{itemize}
\item {} 
\sphinxstyleliteralstrong{\sphinxupquote{R}} (\sphinxstyleliteralemphasis{\sphinxupquote{Array-like}}\sphinxstyleliteralemphasis{\sphinxupquote{ or }}\sphinxstyleliteralemphasis{\sphinxupquote{scalar float}}) \textendash{} Values of the radial coordinate to
map to normalized toroidal flux. Must have the same shape as \sphinxtitleref{Z}
unless the \sphinxtitleref{make\_grid} keyword is set. If the \sphinxtitleref{make\_grid}
keyword is True, R must have shape (\sphinxtitleref{len\_R},).

\item {} 
\sphinxstyleliteralstrong{\sphinxupquote{Z}} (\sphinxstyleliteralemphasis{\sphinxupquote{Array-like}}\sphinxstyleliteralemphasis{\sphinxupquote{ or }}\sphinxstyleliteralemphasis{\sphinxupquote{scalar float}}) \textendash{} Values of the vertical coordinate to
map to normalized toroidal flux. Must have the same shape as \sphinxtitleref{R}
unless the \sphinxtitleref{make\_grid} keyword is set. If the \sphinxtitleref{make\_grid}
keyword is True, Z must have shape (\sphinxtitleref{len\_Z},).

\end{itemize}

\end{description}\end{quote}

All keyword arguments are passed to the parent
{\hyperref[\detokenize{eqtools:eqtools.core.Equilibrium.rz2phinorm}]{\sphinxcrossref{\sphinxcode{\sphinxupquote{Equilibrium.rz2phinorm}}}}}.
Remaining arguments in *args are ignored.
\begin{quote}\begin{description}
\item[{Returns}] \leavevmode
non-normalized poloidal flux.  If
all input arguments are scalar, then \sphinxtitleref{phinorm} is scalar.  IF \sphinxtitleref{R} and \sphinxtitleref{Z}
have the same shape, then \sphinxtitleref{phinorm} has this shape as well.  If \sphinxtitleref{make\_grid}
is True, then \sphinxtitleref{phinorm} has the shape (\sphinxtitleref{len\_R}, \sphinxtitleref{len\_Z}).

\item[{Return type}] \leavevmode
phinorm (Array-like or scalar float)

\end{description}\end{quote}
\subsubsection*{Examples}

All assume that Eq\_instance is a valid instance of EqdskReader.

Find single phinorm value at R=0.6m, Z=0.0m:

\begin{sphinxVerbatim}[commandchars=\\\{\}]
\PYG{n}{phi\PYGZus{}val} \PYG{o}{=} \PYG{n}{Eq\PYGZus{}instance}\PYG{o}{.}\PYG{n}{rz2phinorm}\PYG{p}{(}\PYG{l+m+mf}{0.6}\PYG{p}{,} \PYG{l+m+mi}{0}\PYG{p}{)}
\end{sphinxVerbatim}

Find phinorm values at (R, Z) points (0.6m, 0m) and (0.8m, 0m).
Note that the Z vector must be fully specified,
even if the values are all the same:

\begin{sphinxVerbatim}[commandchars=\\\{\}]
\PYG{n}{phi\PYGZus{}arr} \PYG{o}{=} \PYG{n}{Eq\PYGZus{}instance}\PYG{o}{.}\PYG{n}{rz2phinorm}\PYG{p}{(}\PYG{p}{[}\PYG{l+m+mf}{0.6}\PYG{p}{,} \PYG{l+m+mf}{0.8}\PYG{p}{]}\PYG{p}{,} \PYG{p}{[}\PYG{l+m+mi}{0}\PYG{p}{,} \PYG{l+m+mi}{0}\PYG{p}{]}\PYG{p}{)}
\end{sphinxVerbatim}

Find phinorm values on grid defined by 1D vector of radial positions
R and 1D vector of vertical positions Z:

\begin{sphinxVerbatim}[commandchars=\\\{\}]
\PYG{n}{phi\PYGZus{}mat} \PYG{o}{=} \PYG{n}{Eq\PYGZus{}instance}\PYG{o}{.}\PYG{n}{rz2phinorm}\PYG{p}{(}\PYG{n}{R}\PYG{p}{,} \PYG{n}{Z}\PYG{p}{,} \PYG{n}{make\PYGZus{}grid}\PYG{o}{=}\PYG{k+kc}{True}\PYG{p}{)}
\end{sphinxVerbatim}

\end{fulllineitems}

\index{rz2volnorm() (eqtools.eqdskreader.EqdskReader method)@\spxentry{rz2volnorm()}\spxextra{eqtools.eqdskreader.EqdskReader method}}

\begin{fulllineitems}
\phantomsection\label{\detokenize{eqtools:eqtools.eqdskreader.EqdskReader.rz2volnorm}}\pysiglinewithargsret{\sphinxbfcode{\sphinxupquote{rz2volnorm}}}{\emph{*args}, \emph{**kwargs}}{}
Calculates the normalized flux surface volume.

Not implemented for EqdskReader, as necessary parameter
is not read from a/g-files.
\begin{quote}\begin{description}
\item[{Raises}] \leavevmode
\sphinxstyleliteralstrong{\sphinxupquote{NotImplementedError}} \textendash{} in all cases.

\end{description}\end{quote}

\end{fulllineitems}

\index{rz2rho() (eqtools.eqdskreader.EqdskReader method)@\spxentry{rz2rho()}\spxextra{eqtools.eqdskreader.EqdskReader method}}

\begin{fulllineitems}
\phantomsection\label{\detokenize{eqtools:eqtools.eqdskreader.EqdskReader.rz2rho}}\pysiglinewithargsret{\sphinxbfcode{\sphinxupquote{rz2rho}}}{\emph{method}, \emph{R}, \emph{Z}, \emph{t=False}, \emph{sqrt=False}, \emph{make\_grid=False}, \emph{k=3}, \emph{length\_unit=1}}{}
Convert the passed (R, Z) coordinates into one of several
normalized coordinates.  Wrapper for
{\hyperref[\detokenize{eqtools:eqtools.core.Equilibrium.rz2rho}]{\sphinxcrossref{\sphinxcode{\sphinxupquote{Equilibrium.rz2rho}}}}} masking
timebase dependence.
\begin{quote}\begin{description}
\item[{Parameters}] \leavevmode\begin{itemize}
\item {} 
\sphinxstyleliteralstrong{\sphinxupquote{method}} (\sphinxstyleliteralemphasis{\sphinxupquote{String}}) \textendash{} 
Indicates which normalized coordinates to use.
Valid options are:
\begin{quote}


\begin{savenotes}\sphinxattablestart
\centering
\begin{tabulary}{\linewidth}[t]{|T|T|}
\hline

psinorm
&
Normalized poloidal flux
\\
\hline
phinorm
&
Normalized toroidal flux
\\
\hline
volnorm
&
Normalized volume
\\
\hline
\end{tabulary}
\par
\sphinxattableend\end{savenotes}
\end{quote}


\item {} 
\sphinxstyleliteralstrong{\sphinxupquote{R}} (\sphinxstyleliteralemphasis{\sphinxupquote{Array-like}}\sphinxstyleliteralemphasis{\sphinxupquote{ or }}\sphinxstyleliteralemphasis{\sphinxupquote{scalar float}}) \textendash{} Values of the radial coordinate to
map to normalized coordinate. Must have the same shape as \sphinxtitleref{Z}
unless the make\_grid keyword is set. If the make\_grid keyword
is True, \sphinxtitleref{R} must have shape (\sphinxtitleref{len\_R},).

\item {} 
\sphinxstyleliteralstrong{\sphinxupquote{Z}} (\sphinxstyleliteralemphasis{\sphinxupquote{Array-like}}\sphinxstyleliteralemphasis{\sphinxupquote{ or }}\sphinxstyleliteralemphasis{\sphinxupquote{scalar float}}) \textendash{} Values of the vertical coordinate to
map to normalized coordinate. Must have the same shape as \sphinxtitleref{R}
unless the make\_grid keyword is set. If the make\_grid keyword
is True, \sphinxtitleref{Z} must have shape (\sphinxtitleref{len\_Z},).

\end{itemize}

\item[{Keyword Arguments}] \leavevmode\begin{itemize}
\item {} 
\sphinxstyleliteralstrong{\sphinxupquote{t}} (\sphinxstyleliteralemphasis{\sphinxupquote{indeterminant}}) \textendash{} Provides duck typing for inclusion of t values.
Passed t values either as an Arg or Kwarg are neglected.

\item {} 
\sphinxstyleliteralstrong{\sphinxupquote{sqrt}} (\sphinxstyleliteralemphasis{\sphinxupquote{Boolean}}) \textendash{} Set to True to return the square root of normalized
coordinate. Only the square root of positive values is taken.
Negative values are replaced with zeros, consistent with Steve
Wolfe’s IDL implementation efit\_rz2rho.pro. Default is False
(return normalized coordinate itself).

\item {} 
\sphinxstyleliteralstrong{\sphinxupquote{make\_grid}} (\sphinxstyleliteralemphasis{\sphinxupquote{Boolean}}) \textendash{} Set to True to pass R and Z through meshgrid
before evaluating. If this is set to True, R and Z must each
only have a single dimension, but can have different lengths.
Default is False (do not form meshgrid).

\item {} 
\sphinxstyleliteralstrong{\sphinxupquote{k}} (\sphinxstyleliteralemphasis{\sphinxupquote{positive int}}) \textendash{} The degree of polynomial spline interpolation to
use in converting coordinates.

\item {} 
\sphinxstyleliteralstrong{\sphinxupquote{length\_unit}} (\sphinxstyleliteralemphasis{\sphinxupquote{String}}\sphinxstyleliteralemphasis{\sphinxupquote{ or }}\sphinxstyleliteralemphasis{\sphinxupquote{1}}) \textendash{} 
Length unit that R and Z are being given
in. If a string is given, it must be a valid unit specifier:


\begin{savenotes}\sphinxattablestart
\centering
\begin{tabulary}{\linewidth}[t]{|T|T|}
\hline

’m’
&
meters
\\
\hline
’cm’
&
centimeters
\\
\hline
’mm’
&
millimeters
\\
\hline
’in’
&
inches
\\
\hline
’ft’
&
feet
\\
\hline
’yd’
&
yards
\\
\hline
’smoot’
&
smoots
\\
\hline
’cubit’
&
cubits
\\
\hline
’hand’
&
hands
\\
\hline
’default’
&
meters
\\
\hline
\end{tabulary}
\par
\sphinxattableend\end{savenotes}

If length\_unit is 1 or None, meters are assumed. The default
value is 1 (R and Z given in meters).


\end{itemize}

\item[{Returns}] \leavevmode
If all of the input arguments are
scalar, then a scalar is returned. Otherwise, a scipy Array
instance is returned. If R and Z both have the same shape then
rho has this shape as well. If the make\_grid keyword was True
then rho has shape (len(Z), len(R)).

\item[{Return type}] \leavevmode
rho (Array-like or scalar float)

\item[{Raises}] \leavevmode
\sphinxstyleliteralstrong{\sphinxupquote{ValueError}} \textendash{} If method is not one of the supported values.

\end{description}\end{quote}
\subsubsection*{Examples}

All assume that Eq\_instance is a valid instance of the appropriate
extension of the Equilibrium abstract class.

Find single psinorm value at R=0.6m, Z=0.0m:

\begin{sphinxVerbatim}[commandchars=\\\{\}]
\PYG{n}{psi\PYGZus{}val} \PYG{o}{=} \PYG{n}{Eq\PYGZus{}instance}\PYG{o}{.}\PYG{n}{rz2rho}\PYG{p}{(}\PYG{l+s+s1}{\PYGZsq{}}\PYG{l+s+s1}{psinorm}\PYG{l+s+s1}{\PYGZsq{}}\PYG{p}{,} \PYG{l+m+mf}{0.6}\PYG{p}{,} \PYG{l+m+mi}{0}\PYG{p}{)}
\end{sphinxVerbatim}

Find psinorm values at (R, Z) points (0.6m, 0m) and (0.8m, 0m).
Note that the Z vector must be fully specified,
even if the values are all the same:

\begin{sphinxVerbatim}[commandchars=\\\{\}]
\PYG{n}{psi\PYGZus{}arr} \PYG{o}{=} \PYG{n}{Eq\PYGZus{}instance}\PYG{o}{.}\PYG{n}{rz2rho}\PYG{p}{(}\PYG{l+s+s1}{\PYGZsq{}}\PYG{l+s+s1}{psinorm}\PYG{l+s+s1}{\PYGZsq{}}\PYG{p}{,} \PYG{p}{[}\PYG{l+m+mf}{0.6}\PYG{p}{,} \PYG{l+m+mf}{0.8}\PYG{p}{]}\PYG{p}{,} \PYG{p}{[}\PYG{l+m+mi}{0}\PYG{p}{,} \PYG{l+m+mi}{0}\PYG{p}{]}\PYG{p}{)}
\end{sphinxVerbatim}

Find psinorm values on grid defined by 1D vector of radial positions R
and 1D vector of vertical positions Z:

\begin{sphinxVerbatim}[commandchars=\\\{\}]
\PYG{n}{psi\PYGZus{}mat} \PYG{o}{=} \PYG{n}{Eq\PYGZus{}instance}\PYG{o}{.}\PYG{n}{rz2rho}\PYG{p}{(}\PYG{l+s+s1}{\PYGZsq{}}\PYG{l+s+s1}{psinorm}\PYG{l+s+s1}{\PYGZsq{}}\PYG{p}{,} \PYG{n}{R}\PYG{p}{,} \PYG{n}{Z}\PYG{p}{,} \PYG{n}{make\PYGZus{}grid}\PYG{o}{=}\PYG{k+kc}{True}\PYG{p}{)}
\end{sphinxVerbatim}

\end{fulllineitems}

\index{rz2rmid() (eqtools.eqdskreader.EqdskReader method)@\spxentry{rz2rmid()}\spxextra{eqtools.eqdskreader.EqdskReader method}}

\begin{fulllineitems}
\phantomsection\label{\detokenize{eqtools:eqtools.eqdskreader.EqdskReader.rz2rmid}}\pysiglinewithargsret{\sphinxbfcode{\sphinxupquote{rz2rmid}}}{\emph{R}, \emph{Z}, \emph{t=False}, \emph{sqrt=False}, \emph{make\_grid=False}, \emph{rho=False}, \emph{k=3}, \emph{length\_unit=1}}{}
Maps the given points to the outboard midplane major radius, R\_mid.
Wrapper for
{\hyperref[\detokenize{eqtools:eqtools.core.Equilibrium.rz2rmid}]{\sphinxcrossref{\sphinxcode{\sphinxupquote{Equilibrium.rz2rmid}}}}}
masking timebase dependence.

Based on the IDL version efit\_rz2rmid.pro by Steve Wolfe.
\begin{quote}\begin{description}
\item[{Parameters}] \leavevmode\begin{itemize}
\item {} 
\sphinxstyleliteralstrong{\sphinxupquote{R}} (\sphinxstyleliteralemphasis{\sphinxupquote{Array-like}}\sphinxstyleliteralemphasis{\sphinxupquote{ or }}\sphinxstyleliteralemphasis{\sphinxupquote{scalar float}}) \textendash{} Values of the radial coordinate to
map to midplane radius. Must have the same shape as \sphinxtitleref{Z} unless
the make\_grid keyword is set. If the make\_grid keyword is True,
\sphinxtitleref{R} must have shape (\sphinxtitleref{len\_R},).

\item {} 
\sphinxstyleliteralstrong{\sphinxupquote{Z}} (\sphinxstyleliteralemphasis{\sphinxupquote{Array-like}}\sphinxstyleliteralemphasis{\sphinxupquote{ or }}\sphinxstyleliteralemphasis{\sphinxupquote{scalar float}}) \textendash{} Values of the vertical coordinate to
map to midplane radius. Must have the same shape as \sphinxtitleref{R} unless
the make\_grid keyword is set. If the make\_grid keyword is True,
\sphinxtitleref{Z} must have shape (\sphinxtitleref{len\_Z},).

\end{itemize}

\item[{Keyword Arguments}] \leavevmode\begin{itemize}
\item {} 
\sphinxstyleliteralstrong{\sphinxupquote{t}} (\sphinxstyleliteralemphasis{\sphinxupquote{indeterminant}}) \textendash{} Provides duck typing for inclusion of t values.
Passed t values either as an Arg or Kwarg are neglected.

\item {} 
\sphinxstyleliteralstrong{\sphinxupquote{sqrt}} (\sphinxstyleliteralemphasis{\sphinxupquote{Boolean}}) \textendash{} Set to True to return the square root of midplane
radius. Only the square root of positive values is taken.
Negative values are replaced with zeros, consistent with Steve
Wolfe’s IDL implementation efit\_rz2rho.pro. Default is False
(return R\_mid itself).

\item {} 
\sphinxstyleliteralstrong{\sphinxupquote{make\_grid}} (\sphinxstyleliteralemphasis{\sphinxupquote{Boolean}}) \textendash{} Set to True to pass \sphinxtitleref{R} and \sphinxtitleref{Z} through
meshgrid before evaluating. If this is set to True, \sphinxtitleref{R} and \sphinxtitleref{Z}
must each only have a single dimension, but can have different
lengths.  Default is False (do not form meshgrid).

\item {} 
\sphinxstyleliteralstrong{\sphinxupquote{rho}} (\sphinxstyleliteralemphasis{\sphinxupquote{Boolean}}) \textendash{} Set to True to return r/a (normalized minor radius)
instead of \sphinxtitleref{R\_mid}. Default is False (return major radius,
R\_mid).

\item {} 
\sphinxstyleliteralstrong{\sphinxupquote{k}} (\sphinxstyleliteralemphasis{\sphinxupquote{positive int}}) \textendash{} The degree of polynomial spline interpolation to
use in converting coordinates.

\item {} 
\sphinxstyleliteralstrong{\sphinxupquote{length\_unit}} (\sphinxstyleliteralemphasis{\sphinxupquote{String}}\sphinxstyleliteralemphasis{\sphinxupquote{ or }}\sphinxstyleliteralemphasis{\sphinxupquote{1}}) \textendash{} 
Length unit that R and Z are being given
in AND that R\_mid is returned in. If a string is given, it
must be a valid unit specifier:


\begin{savenotes}\sphinxattablestart
\centering
\begin{tabulary}{\linewidth}[t]{|T|T|}
\hline

’m’
&
meters
\\
\hline
’cm’
&
centimeters
\\
\hline
’mm’
&
millimeters
\\
\hline
’in’
&
inches
\\
\hline
’ft’
&
feet
\\
\hline
’yd’
&
yards
\\
\hline
’smoot’
&
smoots
\\
\hline
’cubit’
&
cubits
\\
\hline
’hand’
&
hands
\\
\hline
’default’
&
meters
\\
\hline
\end{tabulary}
\par
\sphinxattableend\end{savenotes}

If length\_unit is 1 or None, meters are assumed. The default
value is 1 (R and Z given in meters, R\_mid returned in meters).


\end{itemize}

\item[{Returns}] \leavevmode
If all of the input arguments are
scalar, then a scalar is returned. Otherwise, a scipy Array
instance is returned. If \sphinxtitleref{R} and \sphinxtitleref{Z} both have the same shape
then \sphinxtitleref{R\_mid} has this shape as well. If the make\_grid keyword
was True then \sphinxtitleref{R\_mid} has shape (\sphinxtitleref{len(Z)}, \sphinxtitleref{len(R)}).

\item[{Return type}] \leavevmode
R\_mid (Array or scalar float)

\end{description}\end{quote}
\subsubsection*{Examples}

All assume that Eq\_instance is a valid instance of the appropriate
extension of the Equilibrium abstract class.

Find single R\_mid value at R=0.6m, Z=0.0m:

\begin{sphinxVerbatim}[commandchars=\\\{\}]
\PYG{n}{R\PYGZus{}mid\PYGZus{}val} \PYG{o}{=} \PYG{n}{Eq\PYGZus{}instance}\PYG{o}{.}\PYG{n}{rz2rmid}\PYG{p}{(}\PYG{l+m+mf}{0.6}\PYG{p}{,} \PYG{l+m+mi}{0}\PYG{p}{)}
\end{sphinxVerbatim}

Find R\_mid values at (R, Z) points (0.6m, 0m) and (0.8m, 0m).
Note that the Z vector must be fully specified,
even if the values are all the same:

\begin{sphinxVerbatim}[commandchars=\\\{\}]
\PYG{n}{R\PYGZus{}mid\PYGZus{}arr} \PYG{o}{=} \PYG{n}{Eq\PYGZus{}instance}\PYG{o}{.}\PYG{n}{rz2rmid}\PYG{p}{(}\PYG{p}{[}\PYG{l+m+mf}{0.6}\PYG{p}{,} \PYG{l+m+mf}{0.8}\PYG{p}{]}\PYG{p}{,} \PYG{p}{[}\PYG{l+m+mi}{0}\PYG{p}{,} \PYG{l+m+mi}{0}\PYG{p}{]}\PYG{p}{)}
\end{sphinxVerbatim}

Find R\_mid values on grid defined by 1D vector of radial positions R
and 1D vector of vertical positions Z:

\begin{sphinxVerbatim}[commandchars=\\\{\}]
\PYG{n}{R\PYGZus{}mid\PYGZus{}mat} \PYG{o}{=} \PYG{n}{Eq\PYGZus{}instance}\PYG{o}{.}\PYG{n}{rz2rmid}\PYG{p}{(}\PYG{n}{R}\PYG{p}{,} \PYG{n}{Z}\PYG{p}{,} \PYG{n}{make\PYGZus{}grid}\PYG{o}{=}\PYG{k+kc}{True}\PYG{p}{)}
\end{sphinxVerbatim}

\end{fulllineitems}

\index{psinorm2rmid() (eqtools.eqdskreader.EqdskReader method)@\spxentry{psinorm2rmid()}\spxextra{eqtools.eqdskreader.EqdskReader method}}

\begin{fulllineitems}
\phantomsection\label{\detokenize{eqtools:eqtools.eqdskreader.EqdskReader.psinorm2rmid}}\pysiglinewithargsret{\sphinxbfcode{\sphinxupquote{psinorm2rmid}}}{\emph{psi\_norm}, \emph{t=False}, \emph{rho=False}, \emph{k=3}, \emph{length\_unit=1}}{}
Calculates the outboard R\_mid location corresponding to the passed
psi\_norm (normalized poloidal flux) values.
\begin{quote}\begin{description}
\item[{Parameters}] \leavevmode
\sphinxstyleliteralstrong{\sphinxupquote{psi\_norm}} (\sphinxstyleliteralemphasis{\sphinxupquote{Array-like}}\sphinxstyleliteralemphasis{\sphinxupquote{ or }}\sphinxstyleliteralemphasis{\sphinxupquote{scalar float}}) \textendash{} Values of the normalized
poloidal flux to map to midplane radius.

\item[{Keyword Arguments}] \leavevmode\begin{itemize}
\item {} 
\sphinxstyleliteralstrong{\sphinxupquote{t}} (\sphinxstyleliteralemphasis{\sphinxupquote{indeterminant}}) \textendash{} Provides duck typing for inclusion of t values.
Passed \sphinxtitleref{t} values either as an Arg or Kwarg are neglected.

\item {} 
\sphinxstyleliteralstrong{\sphinxupquote{rho}} (\sphinxstyleliteralemphasis{\sphinxupquote{Boolean}}) \textendash{} Set to True to return r/a (normalized minor radius)
instead of \sphinxtitleref{R\_mid}. Default is False (return major radius,
\sphinxtitleref{R\_mid}).

\item {} 
\sphinxstyleliteralstrong{\sphinxupquote{k}} (\sphinxstyleliteralemphasis{\sphinxupquote{positive int}}) \textendash{} The degree of polynomial spline interpolation to
use in converting coordinates.

\item {} 
\sphinxstyleliteralstrong{\sphinxupquote{length\_unit}} (\sphinxstyleliteralemphasis{\sphinxupquote{String}}\sphinxstyleliteralemphasis{\sphinxupquote{ or }}\sphinxstyleliteralemphasis{\sphinxupquote{1}}) \textendash{} 
Length unit that \sphinxtitleref{R\_mid} is returned in.
If a string is given, it must be a valid unit specifier:


\begin{savenotes}\sphinxattablestart
\centering
\begin{tabulary}{\linewidth}[t]{|T|T|}
\hline

’m’
&
meters
\\
\hline
’cm’
&
centimeters
\\
\hline
’mm’
&
millimeters
\\
\hline
’in’
&
inches
\\
\hline
’ft’
&
feet
\\
\hline
’yd’
&
yards
\\
\hline
’smoot’
&
smoots
\\
\hline
’cubit’
&
cubits
\\
\hline
’hand’
&
hands
\\
\hline
’default’
&
meters
\\
\hline
\end{tabulary}
\par
\sphinxattableend\end{savenotes}

If \sphinxtitleref{length\_unit} is 1 or None, meters are assumed. The default
value is 1 (\sphinxtitleref{R\_mid} returned in meters).


\end{itemize}

\item[{Returns}] \leavevmode
If all of the input arguments
are scalar, then a scalar is returned. Otherwise, a scipy Array
instance is returned.

\item[{Return type}] \leavevmode
R\_mid (Array-like or scalar float)

\end{description}\end{quote}
\subsubsection*{Examples}

All assume that Eq\_instance is a valid instance of the appropriate
extension of the Equilibrium abstract class.

Find single R\_mid value for psinorm=0.7:

\begin{sphinxVerbatim}[commandchars=\\\{\}]
\PYG{n}{R\PYGZus{}mid\PYGZus{}val} \PYG{o}{=} \PYG{n}{Eq\PYGZus{}instance}\PYG{o}{.}\PYG{n}{psinorm2rmid}\PYG{p}{(}\PYG{l+m+mf}{0.7}\PYG{p}{)}
\end{sphinxVerbatim}

Find R\_mid values at psi\_norm values of 0.5 and 0.7.
Note that the Z vector must be fully specified, even if the
values are all the same:

\begin{sphinxVerbatim}[commandchars=\\\{\}]
\PYG{n}{R\PYGZus{}mid\PYGZus{}arr} \PYG{o}{=} \PYG{n}{Eq\PYGZus{}instance}\PYG{o}{.}\PYG{n}{psinorm2rmid}\PYG{p}{(}\PYG{p}{[}\PYG{l+m+mf}{0.5}\PYG{p}{,} \PYG{l+m+mf}{0.7}\PYG{p}{]}\PYG{p}{)}
\end{sphinxVerbatim}

\end{fulllineitems}

\index{psinorm2volnorm() (eqtools.eqdskreader.EqdskReader method)@\spxentry{psinorm2volnorm()}\spxextra{eqtools.eqdskreader.EqdskReader method}}

\begin{fulllineitems}
\phantomsection\label{\detokenize{eqtools:eqtools.eqdskreader.EqdskReader.psinorm2volnorm}}\pysiglinewithargsret{\sphinxbfcode{\sphinxupquote{psinorm2volnorm}}}{\emph{*args}, \emph{**kwargs}}{}
Calculates the outboard R\_mid location corresponding to psi\_norm
(normalized poloidal flux) values.

Not implemented for EqdskReader, as necessary parameter is not read
from a/g-files.
\begin{quote}\begin{description}
\item[{Raises}] \leavevmode
\sphinxstyleliteralstrong{\sphinxupquote{NotImplementedError}} \textendash{} in all cases.

\end{description}\end{quote}

\end{fulllineitems}

\index{psinorm2phinorm() (eqtools.eqdskreader.EqdskReader method)@\spxentry{psinorm2phinorm()}\spxextra{eqtools.eqdskreader.EqdskReader method}}

\begin{fulllineitems}
\phantomsection\label{\detokenize{eqtools:eqtools.eqdskreader.EqdskReader.psinorm2phinorm}}\pysiglinewithargsret{\sphinxbfcode{\sphinxupquote{psinorm2phinorm}}}{\emph{psi\_norm}, \emph{t=False}, \emph{k=3}}{}
Calculates the normalized toroidal flux corresponding to the passed
psi\_norm (normalized poloidal flux) values.
\begin{quote}\begin{description}
\item[{Parameters}] \leavevmode
\sphinxstyleliteralstrong{\sphinxupquote{psi\_norm}} (\sphinxstyleliteralemphasis{\sphinxupquote{Array-like}}\sphinxstyleliteralemphasis{\sphinxupquote{ or }}\sphinxstyleliteralemphasis{\sphinxupquote{scalar float}}) \textendash{} Values of the normalized
poloidal flux to map to normalized toroidal flux.

\item[{Keyword Arguments}] \leavevmode\begin{itemize}
\item {} 
\sphinxstyleliteralstrong{\sphinxupquote{t}} (\sphinxstyleliteralemphasis{\sphinxupquote{indeterminant}}) \textendash{} Provides duck typing for inclusion of t values.
Passed \sphinxtitleref{t} values either as an Arg or Kwarg are neglected.

\item {} 
\sphinxstyleliteralstrong{\sphinxupquote{k}} (\sphinxstyleliteralemphasis{\sphinxupquote{positive int}}) \textendash{} The degree of polynomial spline interpolation to
use in converting coordinates.

\end{itemize}

\item[{Returns}] \leavevmode
If all of the input arguments
are scalar, then a scalar is returned. Otherwise, a scipy Array
instance is returned.

\item[{Return type}] \leavevmode
phinorm (Array-like or scalar float)

\end{description}\end{quote}
\subsubsection*{Examples}

All assume that Eq\_instance is a valid instance of the appropriate
extension of the Equilibrium abstract class.

Find single phinorm value for psinorm=0.7:

\begin{sphinxVerbatim}[commandchars=\\\{\}]
\PYG{n}{phinorm\PYGZus{}val} \PYG{o}{=} \PYG{n}{Eq\PYGZus{}instance}\PYG{o}{.}\PYG{n}{psinorm2phinorm}\PYG{p}{(}\PYG{l+m+mf}{0.7}\PYG{p}{)}
\end{sphinxVerbatim}

Find phinorm values at psi\_norm values of 0.5 and 0.7.
Note that the Z vector must be fully specified, even if the
values are all the same:

\begin{sphinxVerbatim}[commandchars=\\\{\}]
\PYG{n}{phinorm\PYGZus{}arr} \PYG{o}{=} \PYG{n}{Eq\PYGZus{}instance}\PYG{o}{.}\PYG{n}{psinorm2phinorm}\PYG{p}{(}\PYG{p}{[}\PYG{l+m+mf}{0.5}\PYG{p}{,} \PYG{l+m+mf}{0.7}\PYG{p}{]}\PYG{p}{)}
\end{sphinxVerbatim}

\end{fulllineitems}

\index{getTimeBase() (eqtools.eqdskreader.EqdskReader method)@\spxentry{getTimeBase()}\spxextra{eqtools.eqdskreader.EqdskReader method}}

\begin{fulllineitems}
\phantomsection\label{\detokenize{eqtools:eqtools.eqdskreader.EqdskReader.getTimeBase}}\pysiglinewithargsret{\sphinxbfcode{\sphinxupquote{getTimeBase}}}{}{}
Returns EFIT time point.
\begin{quote}\begin{description}
\item[{Returns}] \leavevmode
1-element, 1D array of time in s.  Returns array for
consistency with {\hyperref[\detokenize{eqtools:eqtools.core.Equilibrium}]{\sphinxcrossref{\sphinxcode{\sphinxupquote{Equilibrium}}}}}
implementations with time variation.

\item[{Return type}] \leavevmode
time (Array)

\end{description}\end{quote}

\end{fulllineitems}

\index{getCurrentSign() (eqtools.eqdskreader.EqdskReader method)@\spxentry{getCurrentSign()}\spxextra{eqtools.eqdskreader.EqdskReader method}}

\begin{fulllineitems}
\phantomsection\label{\detokenize{eqtools:eqtools.eqdskreader.EqdskReader.getCurrentSign}}\pysiglinewithargsret{\sphinxbfcode{\sphinxupquote{getCurrentSign}}}{}{}
Returns the sign of the current, based on the check in Steve Wolfe’s
IDL implementation efit\_rz2psi.pro.
\begin{quote}\begin{description}
\item[{Returns}] \leavevmode
1 for positive current, -1 for reversed.

\item[{Return type}] \leavevmode
currentSign (Int)

\end{description}\end{quote}

\end{fulllineitems}

\index{getFluxGrid() (eqtools.eqdskreader.EqdskReader method)@\spxentry{getFluxGrid()}\spxextra{eqtools.eqdskreader.EqdskReader method}}

\begin{fulllineitems}
\phantomsection\label{\detokenize{eqtools:eqtools.eqdskreader.EqdskReader.getFluxGrid}}\pysiglinewithargsret{\sphinxbfcode{\sphinxupquote{getFluxGrid}}}{}{}
Returns EFIT flux grid.
\begin{quote}\begin{description}
\item[{Returns}] \leavevmode
{[}1,r,z{]} Array of flux values.  Includes 1-element
time axis for consistency with
{\hyperref[\detokenize{eqtools:eqtools.core.Equilibrium}]{\sphinxcrossref{\sphinxcode{\sphinxupquote{Equilibrium}}}}} implementations
with time variation.

\item[{Return type}] \leavevmode
psiRZ (Array)

\end{description}\end{quote}

\end{fulllineitems}

\index{getRGrid() (eqtools.eqdskreader.EqdskReader method)@\spxentry{getRGrid()}\spxextra{eqtools.eqdskreader.EqdskReader method}}

\begin{fulllineitems}
\phantomsection\label{\detokenize{eqtools:eqtools.eqdskreader.EqdskReader.getRGrid}}\pysiglinewithargsret{\sphinxbfcode{\sphinxupquote{getRGrid}}}{\emph{length\_unit=1}}{}
Returns EFIT R-axis.
\begin{quote}\begin{description}
\item[{Returns}] \leavevmode
{[}r{]} array of R-axis values for RZ grid.

\item[{Return type}] \leavevmode
R (Array)

\end{description}\end{quote}

\end{fulllineitems}

\index{getZGrid() (eqtools.eqdskreader.EqdskReader method)@\spxentry{getZGrid()}\spxextra{eqtools.eqdskreader.EqdskReader method}}

\begin{fulllineitems}
\phantomsection\label{\detokenize{eqtools:eqtools.eqdskreader.EqdskReader.getZGrid}}\pysiglinewithargsret{\sphinxbfcode{\sphinxupquote{getZGrid}}}{\emph{length\_unit=1}}{}
Returns EFIT Z-axis.
\begin{quote}\begin{description}
\item[{Returns}] \leavevmode
{[}z{]} array of Z-axis values for RZ grid.

\item[{Return type}] \leavevmode
Z (Array)

\end{description}\end{quote}

\end{fulllineitems}

\index{getFluxAxis() (eqtools.eqdskreader.EqdskReader method)@\spxentry{getFluxAxis()}\spxextra{eqtools.eqdskreader.EqdskReader method}}

\begin{fulllineitems}
\phantomsection\label{\detokenize{eqtools:eqtools.eqdskreader.EqdskReader.getFluxAxis}}\pysiglinewithargsret{\sphinxbfcode{\sphinxupquote{getFluxAxis}}}{}{}
Returns psi on magnetic axis.
\begin{quote}\begin{description}
\item[{Returns}] \leavevmode
{[}1{]} array of psi on magnetic axis.  Returns array for
consistency with {\hyperref[\detokenize{eqtools:eqtools.core.Equilibrium}]{\sphinxcrossref{\sphinxcode{\sphinxupquote{Equilibrium}}}}}
implementations with time variation.

\item[{Return type}] \leavevmode
psi0 (Array)

\end{description}\end{quote}

\end{fulllineitems}

\index{getFluxLCFS() (eqtools.eqdskreader.EqdskReader method)@\spxentry{getFluxLCFS()}\spxextra{eqtools.eqdskreader.EqdskReader method}}

\begin{fulllineitems}
\phantomsection\label{\detokenize{eqtools:eqtools.eqdskreader.EqdskReader.getFluxLCFS}}\pysiglinewithargsret{\sphinxbfcode{\sphinxupquote{getFluxLCFS}}}{}{}
Returns psi at separatrix.
\begin{quote}\begin{description}
\item[{Returns}] \leavevmode
{[}1{]} array of psi at separatrix.  Returns array for
consistency with {\hyperref[\detokenize{eqtools:eqtools.core.Equilibrium}]{\sphinxcrossref{\sphinxcode{\sphinxupquote{Equilibrium}}}}}
implementations with time variation.

\item[{Return type}] \leavevmode
psia (Array)

\end{description}\end{quote}

\end{fulllineitems}

\index{getRLCFS() (eqtools.eqdskreader.EqdskReader method)@\spxentry{getRLCFS()}\spxextra{eqtools.eqdskreader.EqdskReader method}}

\begin{fulllineitems}
\phantomsection\label{\detokenize{eqtools:eqtools.eqdskreader.EqdskReader.getRLCFS}}\pysiglinewithargsret{\sphinxbfcode{\sphinxupquote{getRLCFS}}}{\emph{length\_unit=1}}{}
Returns array of R-values of LCFS.
\begin{quote}\begin{description}
\item[{Returns}] \leavevmode
{[}1,n{]} array of R values describing LCFS.  Returns
array for consistency with
{\hyperref[\detokenize{eqtools:eqtools.core.Equilibrium}]{\sphinxcrossref{\sphinxcode{\sphinxupquote{Equilibrium}}}}} implementations
with time variation.

\item[{Return type}] \leavevmode
RLCFS (Array)

\end{description}\end{quote}

\end{fulllineitems}

\index{getZLCFS() (eqtools.eqdskreader.EqdskReader method)@\spxentry{getZLCFS()}\spxextra{eqtools.eqdskreader.EqdskReader method}}

\begin{fulllineitems}
\phantomsection\label{\detokenize{eqtools:eqtools.eqdskreader.EqdskReader.getZLCFS}}\pysiglinewithargsret{\sphinxbfcode{\sphinxupquote{getZLCFS}}}{\emph{length\_unit=1}}{}
Returns array of Z-values of LCFS.
\begin{quote}\begin{description}
\item[{Returns}] \leavevmode
{[}1,n{]} array of Z values describing LCFS.  Returns
array for consistency with
{\hyperref[\detokenize{eqtools:eqtools.core.Equilibrium}]{\sphinxcrossref{\sphinxcode{\sphinxupquote{Equilibrium}}}}} implementations
with time variation.

\item[{Return type}] \leavevmode
ZLCFS (Array)

\end{description}\end{quote}

\end{fulllineitems}

\index{remapLCFS() (eqtools.eqdskreader.EqdskReader method)@\spxentry{remapLCFS()}\spxextra{eqtools.eqdskreader.EqdskReader method}}

\begin{fulllineitems}
\phantomsection\label{\detokenize{eqtools:eqtools.eqdskreader.EqdskReader.remapLCFS}}\pysiglinewithargsret{\sphinxbfcode{\sphinxupquote{remapLCFS}}}{\emph{mask=False}}{}
Overwrites RLCFS, ZLCFS values pulled from EFIT with
explicitly-calculated contour of psinorm=1 surface.
\begin{quote}\begin{description}
\item[{Keyword Arguments}] \leavevmode
\sphinxstyleliteralstrong{\sphinxupquote{mask}} (\sphinxstyleliteralemphasis{\sphinxupquote{Boolean}}) \textendash{} Set True to mask LCFS path to limiter outline
(using inPolygon).  Set False to draw full contour of
psi = psiLCFS.  Defaults to False.

\end{description}\end{quote}

\end{fulllineitems}

\index{getFluxVol() (eqtools.eqdskreader.EqdskReader method)@\spxentry{getFluxVol()}\spxextra{eqtools.eqdskreader.EqdskReader method}}

\begin{fulllineitems}
\phantomsection\label{\detokenize{eqtools:eqtools.eqdskreader.EqdskReader.getFluxVol}}\pysiglinewithargsret{\sphinxbfcode{\sphinxupquote{getFluxVol}}}{}{}
Returns volume contained within a flux surface as a function of psi.

Not implemented in {\hyperref[\detokenize{eqtools:eqtools.eqdskreader.EqdskReader}]{\sphinxcrossref{\sphinxcode{\sphinxupquote{EqdskReader}}}}}, as required data is not
stored in g/a-files.
\begin{quote}\begin{description}
\item[{Raises}] \leavevmode
\sphinxstyleliteralstrong{\sphinxupquote{NotImplementedError}} \textendash{} in all cases.

\end{description}\end{quote}

\end{fulllineitems}

\index{getVolLCFS() (eqtools.eqdskreader.EqdskReader method)@\spxentry{getVolLCFS()}\spxextra{eqtools.eqdskreader.EqdskReader method}}

\begin{fulllineitems}
\phantomsection\label{\detokenize{eqtools:eqtools.eqdskreader.EqdskReader.getVolLCFS}}\pysiglinewithargsret{\sphinxbfcode{\sphinxupquote{getVolLCFS}}}{\emph{length\_unit=3}}{}
Returns volume with LCFS.
\begin{quote}\begin{description}
\item[{Returns}] \leavevmode
{[}1{]} array of plasma volume.  Returns array for
consistency with
{\hyperref[\detokenize{eqtools:eqtools.core.Equilibrium}]{\sphinxcrossref{\sphinxcode{\sphinxupquote{Equilibrium}}}}}
implementations with time variation.

\item[{Return type}] \leavevmode
Vol (Array)

\item[{Raises}] \leavevmode
\sphinxstyleliteralstrong{\sphinxupquote{ValueError}} \textendash{} if a-file data is not read.

\end{description}\end{quote}

\end{fulllineitems}

\index{getRmidPsi() (eqtools.eqdskreader.EqdskReader method)@\spxentry{getRmidPsi()}\spxextra{eqtools.eqdskreader.EqdskReader method}}

\begin{fulllineitems}
\phantomsection\label{\detokenize{eqtools:eqtools.eqdskreader.EqdskReader.getRmidPsi}}\pysiglinewithargsret{\sphinxbfcode{\sphinxupquote{getRmidPsi}}}{}{}
Returns outboard-midplane major radius of flux surfaces.

Data not read from a/g-files, not implemented for {\hyperref[\detokenize{eqtools:eqtools.eqdskreader.EqdskReader}]{\sphinxcrossref{\sphinxcode{\sphinxupquote{EqdskReader}}}}}.
\begin{quote}\begin{description}
\item[{Raises}] \leavevmode
\sphinxstyleliteralstrong{\sphinxupquote{NotImplementedError}} \textendash{} in all cases.

\end{description}\end{quote}

\end{fulllineitems}

\index{getF() (eqtools.eqdskreader.EqdskReader method)@\spxentry{getF()}\spxextra{eqtools.eqdskreader.EqdskReader method}}

\begin{fulllineitems}
\phantomsection\label{\detokenize{eqtools:eqtools.eqdskreader.EqdskReader.getF}}\pysiglinewithargsret{\sphinxbfcode{\sphinxupquote{getF}}}{}{}
returns F=RB\_\{Phi\}(Psi), calculated for grad-shafranov solutions
{[}psi,t{]}
\begin{quote}\begin{description}
\item[{Returns}] \leavevmode
{[}1,n{]} array of F(psi).  Returns array for
consistency with
{\hyperref[\detokenize{eqtools:eqtools.core.Equilibrium}]{\sphinxcrossref{\sphinxcode{\sphinxupquote{Equilibrium}}}}}
implementations with time variation.

\item[{Return type}] \leavevmode
F (Array)

\end{description}\end{quote}

\end{fulllineitems}

\index{getFluxPres() (eqtools.eqdskreader.EqdskReader method)@\spxentry{getFluxPres()}\spxextra{eqtools.eqdskreader.EqdskReader method}}

\begin{fulllineitems}
\phantomsection\label{\detokenize{eqtools:eqtools.eqdskreader.EqdskReader.getFluxPres}}\pysiglinewithargsret{\sphinxbfcode{\sphinxupquote{getFluxPres}}}{}{}
Returns pressure on flux surface p(psi).
\begin{quote}\begin{description}
\item[{Returns}] \leavevmode
{[}1,n{]} array of pressure.  Returns array for
consistency with
{\hyperref[\detokenize{eqtools:eqtools.core.Equilibrium}]{\sphinxcrossref{\sphinxcode{\sphinxupquote{Equilibrium}}}}}
implementations with time variation.

\item[{Return type}] \leavevmode
p (Array)

\end{description}\end{quote}

\end{fulllineitems}

\index{getFFPrime() (eqtools.eqdskreader.EqdskReader method)@\spxentry{getFFPrime()}\spxextra{eqtools.eqdskreader.EqdskReader method}}

\begin{fulllineitems}
\phantomsection\label{\detokenize{eqtools:eqtools.eqdskreader.EqdskReader.getFFPrime}}\pysiglinewithargsret{\sphinxbfcode{\sphinxupquote{getFFPrime}}}{}{}
returns FF’ function used for grad-shafranov solutions.
\begin{quote}\begin{description}
\item[{Returns}] \leavevmode
{[}1,n{]} array of FF’(psi).  Returns array for
consistency with
{\hyperref[\detokenize{eqtools:eqtools.core.Equilibrium}]{\sphinxcrossref{\sphinxcode{\sphinxupquote{Equilibrium}}}}}
implementations with time variation.

\item[{Return type}] \leavevmode
FF (Array)

\end{description}\end{quote}

\end{fulllineitems}

\index{getPPrime() (eqtools.eqdskreader.EqdskReader method)@\spxentry{getPPrime()}\spxextra{eqtools.eqdskreader.EqdskReader method}}

\begin{fulllineitems}
\phantomsection\label{\detokenize{eqtools:eqtools.eqdskreader.EqdskReader.getPPrime}}\pysiglinewithargsret{\sphinxbfcode{\sphinxupquote{getPPrime}}}{}{}
returns plasma pressure gradient as a function of psi.
\begin{quote}\begin{description}
\item[{Returns}] \leavevmode
{[}1,n{]} array of pp’(psi).  Returns array for
consistency with
{\hyperref[\detokenize{eqtools:eqtools.core.Equilibrium}]{\sphinxcrossref{\sphinxcode{\sphinxupquote{Equilibrium}}}}}
implementations with time variation.

\item[{Return type}] \leavevmode
pp (Array)

\end{description}\end{quote}

\end{fulllineitems}

\index{getElongation() (eqtools.eqdskreader.EqdskReader method)@\spxentry{getElongation()}\spxextra{eqtools.eqdskreader.EqdskReader method}}

\begin{fulllineitems}
\phantomsection\label{\detokenize{eqtools:eqtools.eqdskreader.EqdskReader.getElongation}}\pysiglinewithargsret{\sphinxbfcode{\sphinxupquote{getElongation}}}{}{}
Returns elongation of LCFS.
\begin{quote}\begin{description}
\item[{Returns}] \leavevmode
{[}1{]} array of plasma elongation.  Returns array for
consistency with
{\hyperref[\detokenize{eqtools:eqtools.core.Equilibrium}]{\sphinxcrossref{\sphinxcode{\sphinxupquote{Equilibrium}}}}}
implementations with time variation.

\item[{Return type}] \leavevmode
kappa (Array)

\item[{Raises}] \leavevmode
\sphinxstyleliteralstrong{\sphinxupquote{ValueError}} \textendash{} if a-file data is not read.

\end{description}\end{quote}

\end{fulllineitems}

\index{getUpperTriangularity() (eqtools.eqdskreader.EqdskReader method)@\spxentry{getUpperTriangularity()}\spxextra{eqtools.eqdskreader.EqdskReader method}}

\begin{fulllineitems}
\phantomsection\label{\detokenize{eqtools:eqtools.eqdskreader.EqdskReader.getUpperTriangularity}}\pysiglinewithargsret{\sphinxbfcode{\sphinxupquote{getUpperTriangularity}}}{}{}
Returns upper triangularity of LCFS.
\begin{quote}\begin{description}
\item[{Returns}] \leavevmode
{[}1{]} array of plasma upper triangularity.  Returns
array for consistency with
{\hyperref[\detokenize{eqtools:eqtools.core.Equilibrium}]{\sphinxcrossref{\sphinxcode{\sphinxupquote{Equilibrium}}}}}
implementations with time variation.

\item[{Return type}] \leavevmode
delta (Array)

\item[{Raises}] \leavevmode
\sphinxstyleliteralstrong{\sphinxupquote{ValueError}} \textendash{} if a-file data is not read.

\end{description}\end{quote}

\end{fulllineitems}

\index{getLowerTriangularity() (eqtools.eqdskreader.EqdskReader method)@\spxentry{getLowerTriangularity()}\spxextra{eqtools.eqdskreader.EqdskReader method}}

\begin{fulllineitems}
\phantomsection\label{\detokenize{eqtools:eqtools.eqdskreader.EqdskReader.getLowerTriangularity}}\pysiglinewithargsret{\sphinxbfcode{\sphinxupquote{getLowerTriangularity}}}{}{}
Returns lower triangularity of LCFS.
\begin{quote}\begin{description}
\item[{Returns}] \leavevmode
{[}1{]} array of plasma lower triangularity.  Returns
array for consistency with
{\hyperref[\detokenize{eqtools:eqtools.core.Equilibrium}]{\sphinxcrossref{\sphinxcode{\sphinxupquote{Equilibrium}}}}}
implementations with time variation.

\item[{Return type}] \leavevmode
delta (Array)

\item[{Raises}] \leavevmode
\sphinxstyleliteralstrong{\sphinxupquote{ValueError}} \textendash{} if a-file data is not read.

\end{description}\end{quote}

\end{fulllineitems}

\index{getShaping() (eqtools.eqdskreader.EqdskReader method)@\spxentry{getShaping()}\spxextra{eqtools.eqdskreader.EqdskReader method}}

\begin{fulllineitems}
\phantomsection\label{\detokenize{eqtools:eqtools.eqdskreader.EqdskReader.getShaping}}\pysiglinewithargsret{\sphinxbfcode{\sphinxupquote{getShaping}}}{}{}
Pulls LCFS elongation, upper/lower triangularity.
\begin{quote}\begin{description}
\item[{Returns}] \leavevmode
namedtuple containing {[}kappa,delta\_u,delta\_l{]}.

\item[{Raises}] \leavevmode
\sphinxstyleliteralstrong{\sphinxupquote{ValueError}} \textendash{} if a-file data is not read.

\end{description}\end{quote}

\end{fulllineitems}

\index{getMagR() (eqtools.eqdskreader.EqdskReader method)@\spxentry{getMagR()}\spxextra{eqtools.eqdskreader.EqdskReader method}}

\begin{fulllineitems}
\phantomsection\label{\detokenize{eqtools:eqtools.eqdskreader.EqdskReader.getMagR}}\pysiglinewithargsret{\sphinxbfcode{\sphinxupquote{getMagR}}}{\emph{length\_unit=1}}{}
Returns major radius of magnetic axis.
\begin{quote}\begin{description}
\item[{Keyword Arguments}] \leavevmode
\sphinxstyleliteralstrong{\sphinxupquote{length\_unit}} (\sphinxstyleliteralemphasis{\sphinxupquote{String}}\sphinxstyleliteralemphasis{\sphinxupquote{ or }}\sphinxstyleliteralemphasis{\sphinxupquote{1}}) \textendash{} length unit R is specified in.  Defaults
to 1 (default unit of rmagx, typically m).

\item[{Returns}] \leavevmode
{[}1{]} array of major radius of magnetic axis.  Returns
array for consistency with
{\hyperref[\detokenize{eqtools:eqtools.core.Equilibrium}]{\sphinxcrossref{\sphinxcode{\sphinxupquote{Equilibrium}}}}}
implementations with time variation.

\item[{Return type}] \leavevmode
magR (Array)

\item[{Raises}] \leavevmode
\sphinxstyleliteralstrong{\sphinxupquote{ValueError}} \textendash{} if a-file data is not read.

\end{description}\end{quote}

\end{fulllineitems}

\index{getMagZ() (eqtools.eqdskreader.EqdskReader method)@\spxentry{getMagZ()}\spxextra{eqtools.eqdskreader.EqdskReader method}}

\begin{fulllineitems}
\phantomsection\label{\detokenize{eqtools:eqtools.eqdskreader.EqdskReader.getMagZ}}\pysiglinewithargsret{\sphinxbfcode{\sphinxupquote{getMagZ}}}{\emph{length\_unit=1}}{}
Returns Z of magnetic axis.
\begin{quote}\begin{description}
\item[{Keyword Arguments}] \leavevmode
\sphinxstyleliteralstrong{\sphinxupquote{length\_unit}} (\sphinxstyleliteralemphasis{\sphinxupquote{String}}\sphinxstyleliteralemphasis{\sphinxupquote{ or }}\sphinxstyleliteralemphasis{\sphinxupquote{1}}) \textendash{} length unit Z is specified in.  Defaults
to 1 (default unit of zmagx, typically m).

\item[{Returns}] \leavevmode
{[}1{]} array of Z of magnetic axis.  Returns array for
consistency with
{\hyperref[\detokenize{eqtools:eqtools.core.Equilibrium}]{\sphinxcrossref{\sphinxcode{\sphinxupquote{Equilibrium}}}}}
implementations with time variation.

\item[{Return type}] \leavevmode
magZ (Array)

\item[{Raises}] \leavevmode
\sphinxstyleliteralstrong{\sphinxupquote{ValueError}} \textendash{} if a-file data is not read.

\end{description}\end{quote}

\end{fulllineitems}

\index{getAreaLCFS() (eqtools.eqdskreader.EqdskReader method)@\spxentry{getAreaLCFS()}\spxextra{eqtools.eqdskreader.EqdskReader method}}

\begin{fulllineitems}
\phantomsection\label{\detokenize{eqtools:eqtools.eqdskreader.EqdskReader.getAreaLCFS}}\pysiglinewithargsret{\sphinxbfcode{\sphinxupquote{getAreaLCFS}}}{\emph{length\_unit=2}}{}
Returns surface area of LCFS.
\begin{quote}\begin{description}
\item[{Keyword Arguments}] \leavevmode
\sphinxstyleliteralstrong{\sphinxupquote{length\_unit}} (\sphinxstyleliteralemphasis{\sphinxupquote{String}}\sphinxstyleliteralemphasis{\sphinxupquote{ or }}\sphinxstyleliteralemphasis{\sphinxupquote{2}}) \textendash{} unit area is specified in.  Defaults to 2
(default unit, typically m\textasciicircum{}2).

\item[{Returns}] \leavevmode
{[}1{]} array of surface area of LCFS.  Returns array
for consistency with
{\hyperref[\detokenize{eqtools:eqtools.core.Equilibrium}]{\sphinxcrossref{\sphinxcode{\sphinxupquote{Equilibrium}}}}}
implementations with time variation.

\item[{Return type}] \leavevmode
AreaLCFS (Array)

\item[{Raises}] \leavevmode
\sphinxstyleliteralstrong{\sphinxupquote{ValueError}} \textendash{} if a-file data is not read.

\end{description}\end{quote}

\end{fulllineitems}

\index{getAOut() (eqtools.eqdskreader.EqdskReader method)@\spxentry{getAOut()}\spxextra{eqtools.eqdskreader.EqdskReader method}}

\begin{fulllineitems}
\phantomsection\label{\detokenize{eqtools:eqtools.eqdskreader.EqdskReader.getAOut}}\pysiglinewithargsret{\sphinxbfcode{\sphinxupquote{getAOut}}}{\emph{length\_unit=1}}{}
Returns outboard-midplane minor radius of LCFS.
\begin{quote}\begin{description}
\item[{Keyword Arguments}] \leavevmode
\sphinxstyleliteralstrong{\sphinxupquote{length\_unit}} (\sphinxstyleliteralemphasis{\sphinxupquote{String}}\sphinxstyleliteralemphasis{\sphinxupquote{ or }}\sphinxstyleliteralemphasis{\sphinxupquote{1}}) \textendash{} unit radius is specified in.  Defaults
to 1 (default unit, typically m).

\item[{Returns}] \leavevmode
{[}1{]} array of outboard-midplane minor radius at LCFS.

\item[{Return type}] \leavevmode
AOut (Array)

\item[{Raises}] \leavevmode
\sphinxstyleliteralstrong{\sphinxupquote{ValueError}} \textendash{} if a-file data is not read.

\end{description}\end{quote}

\end{fulllineitems}

\index{getRmidOut() (eqtools.eqdskreader.EqdskReader method)@\spxentry{getRmidOut()}\spxextra{eqtools.eqdskreader.EqdskReader method}}

\begin{fulllineitems}
\phantomsection\label{\detokenize{eqtools:eqtools.eqdskreader.EqdskReader.getRmidOut}}\pysiglinewithargsret{\sphinxbfcode{\sphinxupquote{getRmidOut}}}{\emph{length\_unit=1}}{}
Returns outboard-midplane major radius of LCFS.
\begin{quote}\begin{description}
\item[{Keyword Arguments}] \leavevmode
\sphinxstyleliteralstrong{\sphinxupquote{length\_unit}} (\sphinxstyleliteralemphasis{\sphinxupquote{String}}\sphinxstyleliteralemphasis{\sphinxupquote{ or }}\sphinxstyleliteralemphasis{\sphinxupquote{1}}) \textendash{} unit radius is specified in.  Defaults to
1 (default unit, typically m).

\item[{Returns}] \leavevmode
{[}1{]} array of outboard-midplane major radius at LCFS.
Returns array for consistency with
{\hyperref[\detokenize{eqtools:eqtools.core.Equilibrium}]{\sphinxcrossref{\sphinxcode{\sphinxupquote{Equilibrium}}}}}
implementations with time variation.

\item[{Return type}] \leavevmode
Rmid (Array)

\item[{Raises}] \leavevmode
\sphinxstyleliteralstrong{\sphinxupquote{ValueError}} \textendash{} if a-file data is not read.

\end{description}\end{quote}

\end{fulllineitems}

\index{getGeometry() (eqtools.eqdskreader.EqdskReader method)@\spxentry{getGeometry()}\spxextra{eqtools.eqdskreader.EqdskReader method}}

\begin{fulllineitems}
\phantomsection\label{\detokenize{eqtools:eqtools.eqdskreader.EqdskReader.getGeometry}}\pysiglinewithargsret{\sphinxbfcode{\sphinxupquote{getGeometry}}}{\emph{length\_unit=None}}{}
Pulls dimensional geometry parameters.
\begin{quote}\begin{description}
\item[{Keyword Arguments}] \leavevmode
\sphinxstyleliteralstrong{\sphinxupquote{length\_unit}} (\sphinxstyleliteralemphasis{\sphinxupquote{String}}) \textendash{} length unit parameters are specified in.
Defaults to None, using default units for individual getter
methods for constituent parameters.

\item[{Returns}] \leavevmode
namedtuple containing {[}Rmag,Zmag,AreaLCFS,aOut,RmidOut{]}

\item[{Raises}] \leavevmode
\sphinxstyleliteralstrong{\sphinxupquote{ValueError}} \textendash{} if a-file data is not read.

\end{description}\end{quote}

\end{fulllineitems}

\index{getQProfile() (eqtools.eqdskreader.EqdskReader method)@\spxentry{getQProfile()}\spxextra{eqtools.eqdskreader.EqdskReader method}}

\begin{fulllineitems}
\phantomsection\label{\detokenize{eqtools:eqtools.eqdskreader.EqdskReader.getQProfile}}\pysiglinewithargsret{\sphinxbfcode{\sphinxupquote{getQProfile}}}{}{}
Returns safety factor q(psi).
\begin{quote}\begin{description}
\item[{Returns}] \leavevmode
{[}1,n{]} array of q(psi).

\item[{Return type}] \leavevmode
qpsi (Array)

\end{description}\end{quote}

\end{fulllineitems}

\index{getQ0() (eqtools.eqdskreader.EqdskReader method)@\spxentry{getQ0()}\spxextra{eqtools.eqdskreader.EqdskReader method}}

\begin{fulllineitems}
\phantomsection\label{\detokenize{eqtools:eqtools.eqdskreader.EqdskReader.getQ0}}\pysiglinewithargsret{\sphinxbfcode{\sphinxupquote{getQ0}}}{}{}
Returns safety factor q on-axis, q0.
\begin{quote}\begin{description}
\item[{Returns}] \leavevmode
{[}1{]} array of q(psi=0).  Returns array for consistency
with {\hyperref[\detokenize{eqtools:eqtools.core.Equilibrium}]{\sphinxcrossref{\sphinxcode{\sphinxupquote{Equilibrium}}}}}
implementations with time variation.

\item[{Return type}] \leavevmode
q0 (Array)

\item[{Raises}] \leavevmode
\sphinxstyleliteralstrong{\sphinxupquote{ValueError}} \textendash{} if a-file data is not read.

\end{description}\end{quote}

\end{fulllineitems}

\index{getQ95() (eqtools.eqdskreader.EqdskReader method)@\spxentry{getQ95()}\spxextra{eqtools.eqdskreader.EqdskReader method}}

\begin{fulllineitems}
\phantomsection\label{\detokenize{eqtools:eqtools.eqdskreader.EqdskReader.getQ95}}\pysiglinewithargsret{\sphinxbfcode{\sphinxupquote{getQ95}}}{}{}
Returns safety factor q at 95\% flux surface.
\begin{quote}\begin{description}
\item[{Returns}] \leavevmode
{[}1{]} array of q(psi=0.95).  Returns array for consistency
with {\hyperref[\detokenize{eqtools:eqtools.core.Equilibrium}]{\sphinxcrossref{\sphinxcode{\sphinxupquote{Equilibrium}}}}}
implementations with time variation.

\item[{Return type}] \leavevmode
q95 (Array)

\item[{Raises}] \leavevmode
\sphinxstyleliteralstrong{\sphinxupquote{ValueError}} \textendash{} if a-file data is not read.

\end{description}\end{quote}

\end{fulllineitems}

\index{getQLCFS() (eqtools.eqdskreader.EqdskReader method)@\spxentry{getQLCFS()}\spxextra{eqtools.eqdskreader.EqdskReader method}}

\begin{fulllineitems}
\phantomsection\label{\detokenize{eqtools:eqtools.eqdskreader.EqdskReader.getQLCFS}}\pysiglinewithargsret{\sphinxbfcode{\sphinxupquote{getQLCFS}}}{}{}
Returns safety factor q at LCFS (interpolated).
\begin{quote}\begin{description}
\item[{Returns}] \leavevmode
{[}1{]} array of q* (interpolated).  Returns array for
consistency with {\hyperref[\detokenize{eqtools:eqtools.core.Equilibrium}]{\sphinxcrossref{\sphinxcode{\sphinxupquote{Equilibrium}}}}}
implementations with time variation.

\item[{Return type}] \leavevmode
qLCFS (Array)

\item[{Raises}] \leavevmode
\sphinxstyleliteralstrong{\sphinxupquote{ValueError}} \textendash{} if a-file data is not loaded.

\end{description}\end{quote}

\end{fulllineitems}

\index{getQ1Surf() (eqtools.eqdskreader.EqdskReader method)@\spxentry{getQ1Surf()}\spxextra{eqtools.eqdskreader.EqdskReader method}}

\begin{fulllineitems}
\phantomsection\label{\detokenize{eqtools:eqtools.eqdskreader.EqdskReader.getQ1Surf}}\pysiglinewithargsret{\sphinxbfcode{\sphinxupquote{getQ1Surf}}}{\emph{length\_unit=1}}{}
Returns outboard-midplane minor radius of q=1 surface.
\begin{quote}\begin{description}
\item[{Keyword Arguments}] \leavevmode
\sphinxstyleliteralstrong{\sphinxupquote{length\_unit}} (\sphinxstyleliteralemphasis{\sphinxupquote{String}}\sphinxstyleliteralemphasis{\sphinxupquote{ or }}\sphinxstyleliteralemphasis{\sphinxupquote{1}}) \textendash{} unit of minor radius.  Defaults to 1
(default unit, typically m)

\item[{Returns}] \leavevmode
{[}1{]} array of minor radius of q=1 surface.  Returns
array for consistency with
{\hyperref[\detokenize{eqtools:eqtools.core.Equilibrium}]{\sphinxcrossref{\sphinxcode{\sphinxupquote{Equilibrium}}}}}
implementations with time variation.

\item[{Return type}] \leavevmode
qr1 (Array)

\item[{Raises}] \leavevmode
\sphinxstyleliteralstrong{\sphinxupquote{ValueError}} \textendash{} if a-file data is not read.

\end{description}\end{quote}

\end{fulllineitems}

\index{getQ2Surf() (eqtools.eqdskreader.EqdskReader method)@\spxentry{getQ2Surf()}\spxextra{eqtools.eqdskreader.EqdskReader method}}

\begin{fulllineitems}
\phantomsection\label{\detokenize{eqtools:eqtools.eqdskreader.EqdskReader.getQ2Surf}}\pysiglinewithargsret{\sphinxbfcode{\sphinxupquote{getQ2Surf}}}{\emph{length\_unit=1}}{}
Returns outboard-midplane minor radius of q=2 surface.
\begin{quote}\begin{description}
\item[{Keyword Arguments}] \leavevmode
\sphinxstyleliteralstrong{\sphinxupquote{length\_unit}} (\sphinxstyleliteralemphasis{\sphinxupquote{String}}\sphinxstyleliteralemphasis{\sphinxupquote{ or }}\sphinxstyleliteralemphasis{\sphinxupquote{1}}) \textendash{} unit of minor radius.  Defaults to 1
(default unit, typically m)

\item[{Returns}] \leavevmode
{[}1{]} array of minor radius of q=2 surface.  Returns
array for consistency with
{\hyperref[\detokenize{eqtools:eqtools.core.Equilibrium}]{\sphinxcrossref{\sphinxcode{\sphinxupquote{Equilibrium}}}}}
implementations with time variation.

\item[{Return type}] \leavevmode
qr2 (Array)

\item[{Raises}] \leavevmode
\sphinxstyleliteralstrong{\sphinxupquote{ValueError}} \textendash{} if a-file data is not read.

\end{description}\end{quote}

\end{fulllineitems}

\index{getQ3Surf() (eqtools.eqdskreader.EqdskReader method)@\spxentry{getQ3Surf()}\spxextra{eqtools.eqdskreader.EqdskReader method}}

\begin{fulllineitems}
\phantomsection\label{\detokenize{eqtools:eqtools.eqdskreader.EqdskReader.getQ3Surf}}\pysiglinewithargsret{\sphinxbfcode{\sphinxupquote{getQ3Surf}}}{\emph{length\_unit=1}}{}
Returns outboard-midplane minor radius of q=3 surface.
\begin{quote}\begin{description}
\item[{Keyword Arguments}] \leavevmode
\sphinxstyleliteralstrong{\sphinxupquote{length\_unit}} (\sphinxstyleliteralemphasis{\sphinxupquote{String}}\sphinxstyleliteralemphasis{\sphinxupquote{ or }}\sphinxstyleliteralemphasis{\sphinxupquote{1}}) \textendash{} unit of minor radius.  Defaults to 1
(default unit, typically m)

\item[{Returns}] \leavevmode
{[}1{]} array of minor radius of q=3 surface.  Returns
array for consistency with
{\hyperref[\detokenize{eqtools:eqtools.core.Equilibrium}]{\sphinxcrossref{\sphinxcode{\sphinxupquote{Equilibrium}}}}}
implementations with time variation.

\item[{Return type}] \leavevmode
qr3 (Array)

\item[{Raises}] \leavevmode
\sphinxstyleliteralstrong{\sphinxupquote{ValueError}} \textendash{} if a-file data is not read.

\end{description}\end{quote}

\end{fulllineitems}

\index{getQs() (eqtools.eqdskreader.EqdskReader method)@\spxentry{getQs()}\spxextra{eqtools.eqdskreader.EqdskReader method}}

\begin{fulllineitems}
\phantomsection\label{\detokenize{eqtools:eqtools.eqdskreader.EqdskReader.getQs}}\pysiglinewithargsret{\sphinxbfcode{\sphinxupquote{getQs}}}{\emph{length\_unit=1}}{}
Pulls q-profile data.
\begin{quote}\begin{description}
\item[{Keyword Arguments}] \leavevmode
\sphinxstyleliteralstrong{\sphinxupquote{length\_unit}} (\sphinxstyleliteralemphasis{\sphinxupquote{String}}\sphinxstyleliteralemphasis{\sphinxupquote{ or }}\sphinxstyleliteralemphasis{\sphinxupquote{1}}) \textendash{} unit of minor radius.  Defaults to 1
(default unit, typically m)

\item[{Returns}] \leavevmode
namedtuple containing {[}q0,q95,qLCFS,rq1,rq2,rq3{]}

\item[{Raises}] \leavevmode
\sphinxstyleliteralstrong{\sphinxupquote{ValueError}} \textendash{} if a-file data is not read.

\end{description}\end{quote}

\end{fulllineitems}

\index{getBtVac() (eqtools.eqdskreader.EqdskReader method)@\spxentry{getBtVac()}\spxextra{eqtools.eqdskreader.EqdskReader method}}

\begin{fulllineitems}
\phantomsection\label{\detokenize{eqtools:eqtools.eqdskreader.EqdskReader.getBtVac}}\pysiglinewithargsret{\sphinxbfcode{\sphinxupquote{getBtVac}}}{}{}
Returns vacuum toroidal field on-axis.
\begin{quote}\begin{description}
\item[{Returns}] \leavevmode
{[}1{]} array of vacuum toroidal field.  Returns array
for consistency with
{\hyperref[\detokenize{eqtools:eqtools.core.Equilibrium}]{\sphinxcrossref{\sphinxcode{\sphinxupquote{Equilibrium}}}}}
implementations with time variation.

\item[{Return type}] \leavevmode
BtVac (Array)

\item[{Raises}] \leavevmode
\sphinxstyleliteralstrong{\sphinxupquote{ValueError}} \textendash{} if a-file data is not read.

\end{description}\end{quote}

\end{fulllineitems}

\index{getBtPla() (eqtools.eqdskreader.EqdskReader method)@\spxentry{getBtPla()}\spxextra{eqtools.eqdskreader.EqdskReader method}}

\begin{fulllineitems}
\phantomsection\label{\detokenize{eqtools:eqtools.eqdskreader.EqdskReader.getBtPla}}\pysiglinewithargsret{\sphinxbfcode{\sphinxupquote{getBtPla}}}{}{}
Returns plasma toroidal field on-axis.
\begin{quote}\begin{description}
\item[{Returns}] \leavevmode
{[}1{]} array of toroidal field including plasma effects.
Returns array for consistency with
{\hyperref[\detokenize{eqtools:eqtools.core.Equilibrium}]{\sphinxcrossref{\sphinxcode{\sphinxupquote{Equilibrium}}}}}
implementations with time variation.

\item[{Return type}] \leavevmode
BtPla (Array)

\item[{Raises}] \leavevmode
\sphinxstyleliteralstrong{\sphinxupquote{ValueError}} \textendash{} if a-file data is not read.

\end{description}\end{quote}

\end{fulllineitems}

\index{getBpAvg() (eqtools.eqdskreader.EqdskReader method)@\spxentry{getBpAvg()}\spxextra{eqtools.eqdskreader.EqdskReader method}}

\begin{fulllineitems}
\phantomsection\label{\detokenize{eqtools:eqtools.eqdskreader.EqdskReader.getBpAvg}}\pysiglinewithargsret{\sphinxbfcode{\sphinxupquote{getBpAvg}}}{}{}
Returns average poloidal field.
\begin{quote}\begin{description}
\item[{Returns}] \leavevmode
{[}1{]} array of average poloidal field.  Returns array
for consistency with
{\hyperref[\detokenize{eqtools:eqtools.core.Equilibrium}]{\sphinxcrossref{\sphinxcode{\sphinxupquote{Equilibrium}}}}}
implementations with time variation.

\item[{Return type}] \leavevmode
BpAvg (Array)

\item[{Raises}] \leavevmode
\sphinxstyleliteralstrong{\sphinxupquote{ValueError}} \textendash{} if a-file data is not read.

\end{description}\end{quote}

\end{fulllineitems}

\index{getFields() (eqtools.eqdskreader.EqdskReader method)@\spxentry{getFields()}\spxextra{eqtools.eqdskreader.EqdskReader method}}

\begin{fulllineitems}
\phantomsection\label{\detokenize{eqtools:eqtools.eqdskreader.EqdskReader.getFields}}\pysiglinewithargsret{\sphinxbfcode{\sphinxupquote{getFields}}}{}{}
Pulls vacuum and plasma toroidal field, poloidal field data.
\begin{quote}\begin{description}
\item[{Returns}] \leavevmode
namedtuple containing {[}BtVac,BtPla,BpAvg{]}

\item[{Raises}] \leavevmode
\sphinxstyleliteralstrong{\sphinxupquote{ValueError}} \textendash{} if a-file data is not read.

\end{description}\end{quote}

\end{fulllineitems}

\index{getIpCalc() (eqtools.eqdskreader.EqdskReader method)@\spxentry{getIpCalc()}\spxextra{eqtools.eqdskreader.EqdskReader method}}

\begin{fulllineitems}
\phantomsection\label{\detokenize{eqtools:eqtools.eqdskreader.EqdskReader.getIpCalc}}\pysiglinewithargsret{\sphinxbfcode{\sphinxupquote{getIpCalc}}}{}{}
Returns EFIT-calculated plasma current.
\begin{quote}\begin{description}
\item[{Returns}] \leavevmode
{[}1{]} array of EFIT-reconstructed plasma current.
Returns array for consistency with
{\hyperref[\detokenize{eqtools:eqtools.core.Equilibrium}]{\sphinxcrossref{\sphinxcode{\sphinxupquote{Equilibrium}}}}}
implementations with time variation.

\item[{Return type}] \leavevmode
IpCalc (Array)

\end{description}\end{quote}

\end{fulllineitems}

\index{getIpMeas() (eqtools.eqdskreader.EqdskReader method)@\spxentry{getIpMeas()}\spxextra{eqtools.eqdskreader.EqdskReader method}}

\begin{fulllineitems}
\phantomsection\label{\detokenize{eqtools:eqtools.eqdskreader.EqdskReader.getIpMeas}}\pysiglinewithargsret{\sphinxbfcode{\sphinxupquote{getIpMeas}}}{}{}
Returns measured plasma current.
\begin{quote}\begin{description}
\item[{Returns}] \leavevmode
{[}1{]} array of measured plasma current.  Returns
array for consistency with
{\hyperref[\detokenize{eqtools:eqtools.core.Equilibrium}]{\sphinxcrossref{\sphinxcode{\sphinxupquote{Equilibrium}}}}}
implementations with time variation.

\item[{Return type}] \leavevmode
IpMeas (Array)

\item[{Raises}] \leavevmode
\sphinxstyleliteralstrong{\sphinxupquote{ValueError}} \textendash{} if a-file data is not read.

\end{description}\end{quote}

\end{fulllineitems}

\index{getJp() (eqtools.eqdskreader.EqdskReader method)@\spxentry{getJp()}\spxextra{eqtools.eqdskreader.EqdskReader method}}

\begin{fulllineitems}
\phantomsection\label{\detokenize{eqtools:eqtools.eqdskreader.EqdskReader.getJp}}\pysiglinewithargsret{\sphinxbfcode{\sphinxupquote{getJp}}}{}{}
Returns (r,z) grid of toroidal plasma current density.

Data not read from g-file, not implemented for {\hyperref[\detokenize{eqtools:eqtools.eqdskreader.EqdskReader}]{\sphinxcrossref{\sphinxcode{\sphinxupquote{EqdskReader}}}}}.
\begin{quote}\begin{description}
\item[{Raises}] \leavevmode
\sphinxstyleliteralstrong{\sphinxupquote{NotImplementedError}} \textendash{} In all cases.

\end{description}\end{quote}

\end{fulllineitems}

\index{getBetaT() (eqtools.eqdskreader.EqdskReader method)@\spxentry{getBetaT()}\spxextra{eqtools.eqdskreader.EqdskReader method}}

\begin{fulllineitems}
\phantomsection\label{\detokenize{eqtools:eqtools.eqdskreader.EqdskReader.getBetaT}}\pysiglinewithargsret{\sphinxbfcode{\sphinxupquote{getBetaT}}}{}{}
Returns EFIT-calculated toroidal beta.
\begin{quote}\begin{description}
\item[{Returns}] \leavevmode
{[}1{]} array of average toroidal beta.  Returns array
for consistency with
{\hyperref[\detokenize{eqtools:eqtools.core.Equilibrium}]{\sphinxcrossref{\sphinxcode{\sphinxupquote{Equilibrium}}}}}
implementations with time variation.

\item[{Return type}] \leavevmode
BetaT (Array)

\item[{Raises}] \leavevmode
\sphinxstyleliteralstrong{\sphinxupquote{ValueError}} \textendash{} if a-file data is not read.

\end{description}\end{quote}

\end{fulllineitems}

\index{getBetaP() (eqtools.eqdskreader.EqdskReader method)@\spxentry{getBetaP()}\spxextra{eqtools.eqdskreader.EqdskReader method}}

\begin{fulllineitems}
\phantomsection\label{\detokenize{eqtools:eqtools.eqdskreader.EqdskReader.getBetaP}}\pysiglinewithargsret{\sphinxbfcode{\sphinxupquote{getBetaP}}}{}{}
Returns EFIT-calculated poloidal beta.
\begin{quote}\begin{description}
\item[{Returns}] \leavevmode
{[}1{]} array of average poloidal beta.  Returns array
for consistency with
{\hyperref[\detokenize{eqtools:eqtools.core.Equilibrium}]{\sphinxcrossref{\sphinxcode{\sphinxupquote{Equilibrium}}}}}
implementations with time variation.

\item[{Return type}] \leavevmode
BetaP (Array)

\item[{Raises}] \leavevmode
\sphinxstyleliteralstrong{\sphinxupquote{ValueError}} \textendash{} if a-file data is not read

\end{description}\end{quote}

\end{fulllineitems}

\index{getLi() (eqtools.eqdskreader.EqdskReader method)@\spxentry{getLi()}\spxextra{eqtools.eqdskreader.EqdskReader method}}

\begin{fulllineitems}
\phantomsection\label{\detokenize{eqtools:eqtools.eqdskreader.EqdskReader.getLi}}\pysiglinewithargsret{\sphinxbfcode{\sphinxupquote{getLi}}}{}{}
Returns internal inductance of plasma.
\begin{quote}\begin{description}
\item[{Returns}] \leavevmode
{[}1{]} array of internal inductance.  Returns array for
consistency with
{\hyperref[\detokenize{eqtools:eqtools.core.Equilibrium}]{\sphinxcrossref{\sphinxcode{\sphinxupquote{Equilibrium}}}}}
implementations with time variation.

\item[{Return type}] \leavevmode
Li (Array)

\item[{Raises}] \leavevmode
\sphinxstyleliteralstrong{\sphinxupquote{ValueError}} \textendash{} if a-file data is not read.

\end{description}\end{quote}

\end{fulllineitems}

\index{getBetas() (eqtools.eqdskreader.EqdskReader method)@\spxentry{getBetas()}\spxextra{eqtools.eqdskreader.EqdskReader method}}

\begin{fulllineitems}
\phantomsection\label{\detokenize{eqtools:eqtools.eqdskreader.EqdskReader.getBetas}}\pysiglinewithargsret{\sphinxbfcode{\sphinxupquote{getBetas}}}{}{}
Pulls EFIT-calculated betas and internal inductance.
\begin{quote}\begin{description}
\item[{Returns}] \leavevmode
namedtuple containing {[}betat,betap,Li{]}

\item[{Raises}] \leavevmode
\sphinxstyleliteralstrong{\sphinxupquote{ValueError}} \textendash{} if a-file data is not read.

\end{description}\end{quote}

\end{fulllineitems}

\index{getDiamagFlux() (eqtools.eqdskreader.EqdskReader method)@\spxentry{getDiamagFlux()}\spxextra{eqtools.eqdskreader.EqdskReader method}}

\begin{fulllineitems}
\phantomsection\label{\detokenize{eqtools:eqtools.eqdskreader.EqdskReader.getDiamagFlux}}\pysiglinewithargsret{\sphinxbfcode{\sphinxupquote{getDiamagFlux}}}{}{}
Returns diamagnetic flux.
\begin{quote}\begin{description}
\item[{Returns}] \leavevmode
{[}1{]} array of measured diamagnetic flux.  Returns array
for consistency with
{\hyperref[\detokenize{eqtools:eqtools.core.Equilibrium}]{\sphinxcrossref{\sphinxcode{\sphinxupquote{Equilibrium}}}}}
implementations with time variation.

\item[{Return type}] \leavevmode
Flux (Array)

\item[{Raises}] \leavevmode
\sphinxstyleliteralstrong{\sphinxupquote{ValueError}} \textendash{} if a-file data is not read.

\end{description}\end{quote}

\end{fulllineitems}

\index{getDiamagBetaT() (eqtools.eqdskreader.EqdskReader method)@\spxentry{getDiamagBetaT()}\spxextra{eqtools.eqdskreader.EqdskReader method}}

\begin{fulllineitems}
\phantomsection\label{\detokenize{eqtools:eqtools.eqdskreader.EqdskReader.getDiamagBetaT}}\pysiglinewithargsret{\sphinxbfcode{\sphinxupquote{getDiamagBetaT}}}{}{}
Returns diamagnetic-loop measured toroidal beta.
\begin{quote}\begin{description}
\item[{Returns}] \leavevmode
{[}1{]} array of measured diamagnetic toroidal beta.
Returns array for consistency with
{\hyperref[\detokenize{eqtools:eqtools.core.Equilibrium}]{\sphinxcrossref{\sphinxcode{\sphinxupquote{Equilibrium}}}}}
implementations with time variation.

\item[{Return type}] \leavevmode
BetaT (Array)

\item[{Raises}] \leavevmode
\sphinxstyleliteralstrong{\sphinxupquote{ValueError}} \textendash{} if a-file data is not read.

\end{description}\end{quote}

\end{fulllineitems}

\index{getDiamagBetaP() (eqtools.eqdskreader.EqdskReader method)@\spxentry{getDiamagBetaP()}\spxextra{eqtools.eqdskreader.EqdskReader method}}

\begin{fulllineitems}
\phantomsection\label{\detokenize{eqtools:eqtools.eqdskreader.EqdskReader.getDiamagBetaP}}\pysiglinewithargsret{\sphinxbfcode{\sphinxupquote{getDiamagBetaP}}}{}{}
Returns diamagnetic-loop measured poloidal beta.
\begin{quote}\begin{description}
\item[{Returns}] \leavevmode
{[}1{]} array of measured diamagnetic poloidal beta.
Returns array for consistency with

\item[{Return type}] \leavevmode

BetaP (Array)
\begin{description}
\item[{{\hyperref[\detokenize{eqtools:eqtools.core.Equilibrium}]{\sphinxcrossref{\sphinxcode{\sphinxupquote{Equilibrium}}}}}}] \leavevmode
implementations with time variation.

\end{description}


\item[{Raises}] \leavevmode
\sphinxstyleliteralstrong{\sphinxupquote{ValueError}} \textendash{} if a-file data is not read.

\end{description}\end{quote}

\end{fulllineitems}

\index{getDiamagTauE() (eqtools.eqdskreader.EqdskReader method)@\spxentry{getDiamagTauE()}\spxextra{eqtools.eqdskreader.EqdskReader method}}

\begin{fulllineitems}
\phantomsection\label{\detokenize{eqtools:eqtools.eqdskreader.EqdskReader.getDiamagTauE}}\pysiglinewithargsret{\sphinxbfcode{\sphinxupquote{getDiamagTauE}}}{}{}
Returns diamagnetic-loop energy confinement time.
\begin{quote}\begin{description}
\item[{Returns}] \leavevmode
{[}1{]} array of measured energy confinement time.
Returns array for consistency with
{\hyperref[\detokenize{eqtools:eqtools.core.Equilibrium}]{\sphinxcrossref{\sphinxcode{\sphinxupquote{Equilibrium}}}}}
implementations with time variation.

\item[{Return type}] \leavevmode
TauE (Array)

\item[{Raises}] \leavevmode
\sphinxstyleliteralstrong{\sphinxupquote{ValueError}} \textendash{} if a-file data is not read.

\end{description}\end{quote}

\end{fulllineitems}

\index{getDiamagWp() (eqtools.eqdskreader.EqdskReader method)@\spxentry{getDiamagWp()}\spxextra{eqtools.eqdskreader.EqdskReader method}}

\begin{fulllineitems}
\phantomsection\label{\detokenize{eqtools:eqtools.eqdskreader.EqdskReader.getDiamagWp}}\pysiglinewithargsret{\sphinxbfcode{\sphinxupquote{getDiamagWp}}}{}{}
Returns diamagnetic-loop measured stored energy.
\begin{quote}\begin{description}
\item[{Returns}] \leavevmode
{[}1{]} array of diamagnetic stored energy.
Returns array for consistency with
{\hyperref[\detokenize{eqtools:eqtools.core.Equilibrium}]{\sphinxcrossref{\sphinxcode{\sphinxupquote{Equilibrium}}}}}
implementations with time variation.

\item[{Return type}] \leavevmode
Wp (Array)

\item[{Raises}] \leavevmode
\sphinxstyleliteralstrong{\sphinxupquote{ValueError}} \textendash{} if a-file data is not read.

\end{description}\end{quote}

\end{fulllineitems}

\index{getDiamag() (eqtools.eqdskreader.EqdskReader method)@\spxentry{getDiamag()}\spxextra{eqtools.eqdskreader.EqdskReader method}}

\begin{fulllineitems}
\phantomsection\label{\detokenize{eqtools:eqtools.eqdskreader.EqdskReader.getDiamag}}\pysiglinewithargsret{\sphinxbfcode{\sphinxupquote{getDiamag}}}{}{}
Pulls diamagnetic flux, diamag. measured toroidal and poloidal beta,
stored energy, and energy confinement time.
\begin{quote}\begin{description}
\item[{Returns}] \leavevmode
namedtuple containing {[}diaFlux,diaBetat,diaBetap,diaTauE,diaWp{]}

\item[{Raises}] \leavevmode
\sphinxstyleliteralstrong{\sphinxupquote{ValueError}} \textendash{} if a-file data is not read

\end{description}\end{quote}

\end{fulllineitems}

\index{getWMHD() (eqtools.eqdskreader.EqdskReader method)@\spxentry{getWMHD()}\spxextra{eqtools.eqdskreader.EqdskReader method}}

\begin{fulllineitems}
\phantomsection\label{\detokenize{eqtools:eqtools.eqdskreader.EqdskReader.getWMHD}}\pysiglinewithargsret{\sphinxbfcode{\sphinxupquote{getWMHD}}}{}{}
Returns EFIT-calculated stored energy.
\begin{quote}\begin{description}
\item[{Returns}] \leavevmode
{[}1{]} array of EFIT-reconstructed stored energy.
Returns array for consistency with
{\hyperref[\detokenize{eqtools:eqtools.core.Equilibrium}]{\sphinxcrossref{\sphinxcode{\sphinxupquote{Equilibrium}}}}}
implementations with time variation.

\item[{Return type}] \leavevmode
WMHD (Array)

\item[{Raises}] \leavevmode
\sphinxstyleliteralstrong{\sphinxupquote{ValueError}} \textendash{} if a-file data is not read.

\end{description}\end{quote}

\end{fulllineitems}

\index{getTauMHD() (eqtools.eqdskreader.EqdskReader method)@\spxentry{getTauMHD()}\spxextra{eqtools.eqdskreader.EqdskReader method}}

\begin{fulllineitems}
\phantomsection\label{\detokenize{eqtools:eqtools.eqdskreader.EqdskReader.getTauMHD}}\pysiglinewithargsret{\sphinxbfcode{\sphinxupquote{getTauMHD}}}{}{}
Returns EFIT-calculated energy confinement time.
\begin{quote}\begin{description}
\item[{Returns}] \leavevmode
{[}1{]} array of EFIT-reconstructed energy confinement
time.  Returns array for consistency with
{\hyperref[\detokenize{eqtools:eqtools.core.Equilibrium}]{\sphinxcrossref{\sphinxcode{\sphinxupquote{Equilibrium}}}}}
implementations with time variation.

\item[{Return type}] \leavevmode
tauMHD (Array)

\item[{Raises}] \leavevmode
\sphinxstyleliteralstrong{\sphinxupquote{ValueError}} \textendash{} if a-file data is not read.

\end{description}\end{quote}

\end{fulllineitems}

\index{getPinj() (eqtools.eqdskreader.EqdskReader method)@\spxentry{getPinj()}\spxextra{eqtools.eqdskreader.EqdskReader method}}

\begin{fulllineitems}
\phantomsection\label{\detokenize{eqtools:eqtools.eqdskreader.EqdskReader.getPinj}}\pysiglinewithargsret{\sphinxbfcode{\sphinxupquote{getPinj}}}{}{}
Returns EFIT injected power.
\begin{quote}\begin{description}
\item[{Returns}] \leavevmode
{[}1{]} array of EFIT-reconstructed injected power.
Returns array for consistency with
{\hyperref[\detokenize{eqtools:eqtools.core.Equilibrium}]{\sphinxcrossref{\sphinxcode{\sphinxupquote{Equilibrium}}}}}
implementations with time variation.

\item[{Return type}] \leavevmode
Pinj (Array)

\item[{Raises}] \leavevmode
\sphinxstyleliteralstrong{\sphinxupquote{ValueError}} \textendash{} if a-file data is not read.

\end{description}\end{quote}

\end{fulllineitems}

\index{getWbdot() (eqtools.eqdskreader.EqdskReader method)@\spxentry{getWbdot()}\spxextra{eqtools.eqdskreader.EqdskReader method}}

\begin{fulllineitems}
\phantomsection\label{\detokenize{eqtools:eqtools.eqdskreader.EqdskReader.getWbdot}}\pysiglinewithargsret{\sphinxbfcode{\sphinxupquote{getWbdot}}}{}{}
Returns EFIT d/dt of magnetic stored energy
\begin{quote}\begin{description}
\item[{Returns}] \leavevmode
{[}1{]} array of d(Wb)/dt.  Returns array for consistency
with {\hyperref[\detokenize{eqtools:eqtools.core.Equilibrium}]{\sphinxcrossref{\sphinxcode{\sphinxupquote{Equilibrium}}}}}
implementations with time variation.

\item[{Return type}] \leavevmode
dWdt (Array)

\item[{Raises}] \leavevmode
\sphinxstyleliteralstrong{\sphinxupquote{ValueError}} \textendash{} if a-file data is not read.

\end{description}\end{quote}

\end{fulllineitems}

\index{getWpdot() (eqtools.eqdskreader.EqdskReader method)@\spxentry{getWpdot()}\spxextra{eqtools.eqdskreader.EqdskReader method}}

\begin{fulllineitems}
\phantomsection\label{\detokenize{eqtools:eqtools.eqdskreader.EqdskReader.getWpdot}}\pysiglinewithargsret{\sphinxbfcode{\sphinxupquote{getWpdot}}}{}{}
Returns EFIT d/dt of plasma stored energy.
\begin{quote}\begin{description}
\item[{Returns}] \leavevmode
{[}1{]} array of d(Wp)/dt.  Returns array for consistency
with {\hyperref[\detokenize{eqtools:eqtools.core.Equilibrium}]{\sphinxcrossref{\sphinxcode{\sphinxupquote{Equilibrium}}}}}
implementations with time variation.

\item[{Return type}] \leavevmode
dWdt (Array)

\item[{Raises}] \leavevmode
\sphinxstyleliteralstrong{\sphinxupquote{ValueError}} \textendash{} if a-file data is not read.

\end{description}\end{quote}

\end{fulllineitems}

\index{getBCentr() (eqtools.eqdskreader.EqdskReader method)@\spxentry{getBCentr()}\spxextra{eqtools.eqdskreader.EqdskReader method}}

\begin{fulllineitems}
\phantomsection\label{\detokenize{eqtools:eqtools.eqdskreader.EqdskReader.getBCentr}}\pysiglinewithargsret{\sphinxbfcode{\sphinxupquote{getBCentr}}}{}{}
returns Vacuum toroidal magnetic field in Tesla at Rcentr
\begin{quote}\begin{description}
\item[{Returns}] \leavevmode
{[}nt{]} array of B\_t at center {[}T{]}

\item[{Return type}] \leavevmode
B\_cent (Array)

\item[{Raises}] \leavevmode
\sphinxstyleliteralstrong{\sphinxupquote{ValueError}} \textendash{} if module cannot retrieve data from MDS tree.

\end{description}\end{quote}

\end{fulllineitems}

\index{getRCentr() (eqtools.eqdskreader.EqdskReader method)@\spxentry{getRCentr()}\spxextra{eqtools.eqdskreader.EqdskReader method}}

\begin{fulllineitems}
\phantomsection\label{\detokenize{eqtools:eqtools.eqdskreader.EqdskReader.getRCentr}}\pysiglinewithargsret{\sphinxbfcode{\sphinxupquote{getRCentr}}}{\emph{length\_unit=1}}{}
returns radius where Bcentr evaluated
\begin{quote}\begin{description}
\item[{Returns}] \leavevmode
Radial position where Bcent calculated {[}m{]}

\item[{Return type}] \leavevmode
R

\item[{Raises}] \leavevmode
\sphinxstyleliteralstrong{\sphinxupquote{ValueError}} \textendash{} if module cannot retrieve data from MDS tree.

\end{description}\end{quote}

\end{fulllineitems}

\index{getEnergy() (eqtools.eqdskreader.EqdskReader method)@\spxentry{getEnergy()}\spxextra{eqtools.eqdskreader.EqdskReader method}}

\begin{fulllineitems}
\phantomsection\label{\detokenize{eqtools:eqtools.eqdskreader.EqdskReader.getEnergy}}\pysiglinewithargsret{\sphinxbfcode{\sphinxupquote{getEnergy}}}{}{}
Pulls EFIT stored energy, energy confinement time, injected power,
and d/dt of magnetic and plasma stored energy.
\begin{quote}\begin{description}
\item[{Returns}] \leavevmode
namedtuple containing {[}WMHD,tauMHD,Pinj,Wbdot,Wpdot{]}

\item[{Raises}] \leavevmode
\sphinxstyleliteralstrong{\sphinxupquote{ValueError}} \textendash{} if a-file data is not read.

\end{description}\end{quote}

\end{fulllineitems}

\index{getParam() (eqtools.eqdskreader.EqdskReader method)@\spxentry{getParam()}\spxextra{eqtools.eqdskreader.EqdskReader method}}

\begin{fulllineitems}
\phantomsection\label{\detokenize{eqtools:eqtools.eqdskreader.EqdskReader.getParam}}\pysiglinewithargsret{\sphinxbfcode{\sphinxupquote{getParam}}}{\emph{name}}{}
Backup function, applying a direct path input for tree-like data
storage access for parameters not typically found in Equilbrium object.
Directly calls attributes read from g/a-files in copy-safe manner.
\begin{quote}\begin{description}
\item[{Parameters}] \leavevmode
\sphinxstyleliteralstrong{\sphinxupquote{name}} (\sphinxstyleliteralemphasis{\sphinxupquote{String}}) \textendash{} Parameter name for value stored in EqdskReader
instance.

\item[{Returns}] \leavevmode
value stored as attribute in
{\hyperref[\detokenize{eqtools:eqtools.eqdskreader.EqdskReader}]{\sphinxcrossref{\sphinxcode{\sphinxupquote{EqdskReader}}}}}.

\item[{Return type}] \leavevmode
param (Array-like or scalar float)

\item[{Raises}] \leavevmode
\sphinxstyleliteralstrong{\sphinxupquote{AttributeError}} \textendash{} raised if no attribute is found.

\end{description}\end{quote}

\end{fulllineitems}

\index{getMachineCrossSection() (eqtools.eqdskreader.EqdskReader method)@\spxentry{getMachineCrossSection()}\spxextra{eqtools.eqdskreader.EqdskReader method}}

\begin{fulllineitems}
\phantomsection\label{\detokenize{eqtools:eqtools.eqdskreader.EqdskReader.getMachineCrossSection}}\pysiglinewithargsret{\sphinxbfcode{\sphinxupquote{getMachineCrossSection}}}{}{}
Method to pull machine cross-section from data storage, convert to
standard format for plotting routine.
\begin{quote}\begin{description}
\item[{Returns}] \leavevmode

(\sphinxtitleref{R\_limiter}, \sphinxtitleref{Z\_limiter})
\begin{itemize}
\item {} 
\sphinxstylestrong{R\_limiter} (\sphinxtitleref{Array}) - {[}n{]} array of x-values for machine cross-section.

\item {} 
\sphinxstylestrong{Z\_limiter} (\sphinxtitleref{Array}) - {[}n{]} array of y-values for machine cross-section.

\end{itemize}


\end{description}\end{quote}

\end{fulllineitems}

\index{getMachineCrossSectionFull() (eqtools.eqdskreader.EqdskReader method)@\spxentry{getMachineCrossSectionFull()}\spxextra{eqtools.eqdskreader.EqdskReader method}}

\begin{fulllineitems}
\phantomsection\label{\detokenize{eqtools:eqtools.eqdskreader.EqdskReader.getMachineCrossSectionFull}}\pysiglinewithargsret{\sphinxbfcode{\sphinxupquote{getMachineCrossSectionFull}}}{}{}
Returns vectorization of machine cross-section.

Absent additional data (not found in eqdsks) simply returns
self.getMachineCrossSection().

\end{fulllineitems}

\index{gfile() (eqtools.eqdskreader.EqdskReader method)@\spxentry{gfile()}\spxextra{eqtools.eqdskreader.EqdskReader method}}

\begin{fulllineitems}
\phantomsection\label{\detokenize{eqtools:eqtools.eqdskreader.EqdskReader.gfile}}\pysiglinewithargsret{\sphinxbfcode{\sphinxupquote{gfile}}}{\emph{time=None}, \emph{nw=None}, \emph{nh=None}, \emph{shot=None}, \emph{name=None}, \emph{tunit='ms'}, \emph{title='EQTOOLS'}, \emph{nbbbs=100}}{}
Generates an EFIT gfile with gfile naming convention
\begin{quote}\begin{description}
\item[{Keyword Arguments}] \leavevmode\begin{itemize}
\item {} 
\sphinxstyleliteralstrong{\sphinxupquote{time}} (\sphinxstyleliteralemphasis{\sphinxupquote{scalar float}}) \textendash{} Time of equilibrium to
generate the gfile from. This will use the specified
spline functionality to do so. Allows for it to be
unspecified for single-time-frame equilibria.

\item {} 
\sphinxstyleliteralstrong{\sphinxupquote{nw}} (\sphinxstyleliteralemphasis{\sphinxupquote{scalar integer}}) \textendash{} Number of points in R.
R is the major radius, and describes the ‘width’ of the
gfile.

\item {} 
\sphinxstyleliteralstrong{\sphinxupquote{nh}} (\sphinxstyleliteralemphasis{\sphinxupquote{scalar integer}}) \textendash{} Number of points in Z. In cylindrical
coordinates Z is the height, and nh describes the ‘height’
of the gfile.

\item {} 
\sphinxstyleliteralstrong{\sphinxupquote{shot}} (\sphinxstyleliteralemphasis{\sphinxupquote{scalar integer}}) \textendash{} The shot numer of the equilibrium.
Used to help generate the gfile name if unspecified.

\item {} 
\sphinxstyleliteralstrong{\sphinxupquote{name}} (\sphinxstyleliteralemphasis{\sphinxupquote{String}}) \textendash{} Name of the gfile.  If unspecified, will follow
standard gfile naming convention (g+shot.time) under current
python operating directory.  This allows for it to be saved
in other directories, etc.

\item {} 
\sphinxstyleliteralstrong{\sphinxupquote{tunit}} (\sphinxstyleliteralemphasis{\sphinxupquote{String}}) \textendash{} Specified unit for tin. It can only be ‘ms’ for
milliseconds or ‘s’ for seconds.

\item {} 
\sphinxstyleliteralstrong{\sphinxupquote{title}} (\sphinxstyleliteralemphasis{\sphinxupquote{String}}) \textendash{} Title of the gfile on the first line. Name cannot
exceed 10 digits. This is so that the style of the first line
is preserved.

\item {} 
\sphinxstyleliteralstrong{\sphinxupquote{nbbbs}} (\sphinxstyleliteralemphasis{\sphinxupquote{scalar integer}}) \textendash{} Number of points to define the plasma
seperatrix within the gfile.  The points are defined equally
spaced in angle about the plasma center.  This will cause the
x-point to be poorly defined.

\end{itemize}

\item[{Raises}] \leavevmode
\sphinxstyleliteralstrong{\sphinxupquote{ValueError}} \textendash{} If title is longer than 10 characters.

\end{description}\end{quote}
\subsubsection*{Examples}

All assume that \sphinxtitleref{Eq\_instance} is a valid instance of the appropriate
extension of the \sphinxcode{\sphinxupquote{Equilibrium}} abstract class (example
shot number of 1001).

Generate a gfile (time at t=.26s) output of g1001.26:

\begin{sphinxVerbatim}[commandchars=\\\{\}]
\PYG{n}{Eq\PYGZus{}instance}\PYG{o}{.}\PYG{n}{gfile}\PYG{p}{(}\PYG{p}{)}
\end{sphinxVerbatim}

\end{fulllineitems}

\index{plotFlux() (eqtools.eqdskreader.EqdskReader method)@\spxentry{plotFlux()}\spxextra{eqtools.eqdskreader.EqdskReader method}}

\begin{fulllineitems}
\phantomsection\label{\detokenize{eqtools:eqtools.eqdskreader.EqdskReader.plotFlux}}\pysiglinewithargsret{\sphinxbfcode{\sphinxupquote{plotFlux}}}{\emph{fill=True}, \emph{mask=True}}{}
streamlined plotting of flux contours directly from psi grid
\begin{quote}\begin{description}
\item[{Keyword Arguments}] \leavevmode\begin{itemize}
\item {} 
\sphinxstyleliteralstrong{\sphinxupquote{fill}} (\sphinxstyleliteralemphasis{\sphinxupquote{Boolean}}) \textendash{} Default True.  Set True to plot filled contours of
flux delineated by black outlines.  Set False to instead plot
color-coded line contours on a blank background.

\item {} 
\sphinxstyleliteralstrong{\sphinxupquote{mask}} (\sphinxstyleliteralemphasis{\sphinxupquote{Boolean}}) \textendash{} Default True.  Set True to draw a clipping mask
based on the limiter outline for the flux contours.  Set False
to draw the full RZ grid.

\end{itemize}

\end{description}\end{quote}

\end{fulllineitems}


\end{fulllineitems}



\subsection{eqtools.filewriter module}
\label{\detokenize{eqtools:eqtools-filewriter-module}}

\subsection{eqtools.pfilereader module}
\label{\detokenize{eqtools:module-eqtools.pfilereader}}\label{\detokenize{eqtools:eqtools-pfilereader-module}}\index{eqtools.pfilereader (module)@\spxentry{eqtools.pfilereader}\spxextra{module}}
This module contains the {\hyperref[\detokenize{eqtools:eqtools.pfilereader.PFileReader}]{\sphinxcrossref{\sphinxcode{\sphinxupquote{PFileReader}}}}} class, a lightweight data
handler for p-file (radial profile) datasets.
\begin{description}
\item[{Classes:}] \leavevmode\begin{description}
\item[{PFileReader:}] \leavevmode
Data-storage class for p-file data.  Reads
data from ASCII p-file, storing as copy-safe object
attributes.

\end{description}

\end{description}
\index{PFileReader (class in eqtools.pfilereader)@\spxentry{PFileReader}\spxextra{class in eqtools.pfilereader}}

\begin{fulllineitems}
\phantomsection\label{\detokenize{eqtools:eqtools.pfilereader.PFileReader}}\pysiglinewithargsret{\sphinxbfcode{\sphinxupquote{class }}\sphinxcode{\sphinxupquote{eqtools.pfilereader.}}\sphinxbfcode{\sphinxupquote{PFileReader}}}{\emph{pfile}, \emph{verbose=True}}{}
Bases: \sphinxcode{\sphinxupquote{object}}

Class to read ASCII p-file (profile data storage) into lightweight,
user-friendly data structure.

P-files store data blocks containing the following: a header with parameter
name, parameter units, x-axis units, and number of data points, followed by
values of axis x, parameter y, and derivative dy/dx.  Each parameter block
is read into a namedtuple storing


\begin{savenotes}\sphinxattablestart
\centering
\begin{tabulary}{\linewidth}[t]{|T|T|}
\hline

‘name’
&
parameter name
\\
\hline
‘npts’
&
array size
\\
\hline
‘x’
&
abscissa array
\\
\hline
‘y’
&
data array
\\
\hline
‘dydx’
&
data gradient
\\
\hline
‘xunits’
&
abscissa units
\\
\hline
‘units’
&
data units
\\
\hline
\end{tabulary}
\par
\sphinxattableend\end{savenotes}

with each namedtuple stored as an attribute of the PFileReader  instance.
This gracefully handles variable formats of p-files (differing versions of
p-files will have different parameters stored).  Data blocks are accessed
as attributes in a copy-safe manner.

Creates instance of PFileReader.
\begin{quote}\begin{description}
\item[{Parameters}] \leavevmode
\sphinxstyleliteralstrong{\sphinxupquote{pfile}} (\sphinxstyleliteralemphasis{\sphinxupquote{String}}) \textendash{} Path to ASCII p-file to be loaded.

\item[{Keyword Arguments}] \leavevmode
\sphinxstyleliteralstrong{\sphinxupquote{verbose}} (\sphinxstyleliteralemphasis{\sphinxupquote{Boolean}}) \textendash{} Option to print message on object creation
listing available data parameters. Defaults to True.

\end{description}\end{quote}
\subsubsection*{Examples}

Load p-file data located at \sphinxtitleref{file\_path}, while suppressing terminal
output of stored parameters:

\begin{sphinxVerbatim}[commandchars=\\\{\}]
\PYG{n}{pfr} \PYG{o}{=} \PYG{n}{eqtools}\PYG{o}{.}\PYG{n}{PFileReader}\PYG{p}{(}\PYG{n}{file\PYGZus{}path}\PYG{p}{,}\PYG{n}{verbose}\PYG{o}{=}\PYG{k+kc}{False}\PYG{p}{)}
\end{sphinxVerbatim}

Recover electron density data (for example):

\begin{sphinxVerbatim}[commandchars=\\\{\}]
\PYG{n}{ne\PYGZus{}data} \PYG{o}{=} \PYG{n}{pfr}\PYG{o}{.}\PYG{n}{ne}
\end{sphinxVerbatim}

Recover abscissa and electron density data (for example):

\begin{sphinxVerbatim}[commandchars=\\\{\}]
\PYG{n}{ne} \PYG{o}{=} \PYG{n}{pfr}\PYG{o}{.}\PYG{n}{ne}\PYG{o}{.}\PYG{n}{y}
\PYG{n}{abscis} \PYG{o}{=} \PYG{n}{pfr}\PYG{o}{.}\PYG{n}{ne}\PYG{o}{.}\PYG{n}{x}
\end{sphinxVerbatim}

Available parameters in pfr may be listed via the overridden \_\_str\_\_
command.

\end{fulllineitems}



\subsection{eqtools.trispline module}
\label{\detokenize{eqtools:module-eqtools.trispline}}\label{\detokenize{eqtools:eqtools-trispline-module}}\index{eqtools.trispline (module)@\spxentry{eqtools.trispline}\spxextra{module}}
This module provides interface to the tricubic spline interpolator. It also
contains an enhanced bivariate spline which generates bounds errors.
\index{Spline (class in eqtools.trispline)@\spxentry{Spline}\spxextra{class in eqtools.trispline}}

\begin{fulllineitems}
\phantomsection\label{\detokenize{eqtools:eqtools.trispline.Spline}}\pysiglinewithargsret{\sphinxbfcode{\sphinxupquote{class }}\sphinxcode{\sphinxupquote{eqtools.trispline.}}\sphinxbfcode{\sphinxupquote{Spline}}}{\emph{x}, \emph{y}, \emph{z}, \emph{f}, \emph{boundary='natural'}, \emph{dx=0}, \emph{dy=0}, \emph{dz=0}, \emph{bounds\_error=True}, \emph{fill\_value=nan}}{}
Bases: \sphinxcode{\sphinxupquote{object}}

Tricubic interpolating spline with forced edge derivative equal zero
conditions.  It assumes a cartesian grid.  The ordering of f{[}z,y,x{]} is
extremely important for the proper evaluation of the spline.  It assumes
that f is in C order.

Create a new Spline instance.
\begin{quote}\begin{description}
\item[{Parameters}] \leavevmode\begin{itemize}
\item {} 
\sphinxstyleliteralstrong{\sphinxupquote{x}} (\sphinxstyleliteralemphasis{\sphinxupquote{1-dimensional float array}}) \textendash{} Values of the positions of the 1st
Dimension of f. Must be monotonic without duplicates.

\item {} 
\sphinxstyleliteralstrong{\sphinxupquote{y}} (\sphinxstyleliteralemphasis{\sphinxupquote{1-dimensional float array}}) \textendash{} Values of the positions of the 2nd
dimension of f. Must be monotonic without duplicates.

\item {} 
\sphinxstyleliteralstrong{\sphinxupquote{z}} (\sphinxstyleliteralemphasis{\sphinxupquote{1-dimensional float array}}) \textendash{} Values of the positions of the 3rd
dimension of f. Must be monotonic without duplicates.

\item {} 
\sphinxstyleliteralstrong{\sphinxupquote{f}} (\sphinxstyleliteralemphasis{\sphinxupquote{3-dimensional float array}}) \textendash{} f{[}x,y,z{]}. NaN and Inf will hamper
performance and affect interpolation in 4x4x4 space about its value.

\end{itemize}

\item[{Keyword Arguments}] \leavevmode\begin{itemize}
\item {} 
\sphinxstyleliteralstrong{\sphinxupquote{regular}} (\sphinxstyleliteralemphasis{\sphinxupquote{Boolean}}) \textendash{} If the grid is known to be regular, forces
matrix-based fast evaluation of interpolation.

\item {} 
\sphinxstyleliteralstrong{\sphinxupquote{fast}} (\sphinxstyleliteralemphasis{\sphinxupquote{Boolean}}) \textendash{} Outdated input to test the indexing performance of the
c code vs internal python handling.

\end{itemize}

\item[{Raises}] \leavevmode\begin{itemize}
\item {} 
\sphinxstyleliteralstrong{\sphinxupquote{ValueError}} \textendash{} If any of the dimensions do not match specified f dim

\item {} 
\sphinxstyleliteralstrong{\sphinxupquote{ValueError}} \textendash{} If x,y, or z are not monotonic

\end{itemize}

\end{description}\end{quote}
\subsubsection*{Examples}

All assume that \sphinxtitleref{x}, \sphinxtitleref{y}, \sphinxtitleref{z}, and \sphinxtitleref{f} are valid instances of the appropriate
numpy arrays which take independent variables x,y,z and create numpy array
f. \sphinxtitleref{x1}, \sphinxtitleref{y1}, and \sphinxtitleref{z1} are numpy arrays which data f is to be interpolated.

Generate a Trispline instance map with data x, y, z and f:

\begin{sphinxVerbatim}[commandchars=\\\{\}]
\PYG{n+nb}{map} \PYG{o}{=} \PYG{n}{Spline}\PYG{p}{(}\PYG{n}{x}\PYG{p}{,} \PYG{n}{y}\PYG{p}{,} \PYG{n}{z}\PYG{p}{,} \PYG{n}{f}\PYG{p}{)}
\end{sphinxVerbatim}

Evaluate Trispline instance map at x1, y1, z1:

\begin{sphinxVerbatim}[commandchars=\\\{\}]
\PYG{n}{output} \PYG{o}{=} \PYG{n+nb}{map}\PYG{o}{.}\PYG{n}{ev}\PYG{p}{(}\PYG{n}{x1}\PYG{p}{,} \PYG{n}{y1}\PYG{p}{,} \PYG{n}{z1}\PYG{p}{)}
\end{sphinxVerbatim}
\index{ev() (eqtools.trispline.Spline method)@\spxentry{ev()}\spxextra{eqtools.trispline.Spline method}}

\begin{fulllineitems}
\phantomsection\label{\detokenize{eqtools:eqtools.trispline.Spline.ev}}\pysiglinewithargsret{\sphinxbfcode{\sphinxupquote{ev}}}{\emph{xi}, \emph{yi}, \emph{zi}, \emph{dx=0}, \emph{dy=0}, \emph{dz=0}}{}
evaluates tricubic spline at point (xi,yi,zi) which is f{[}xi,yi,zi{]}.
\begin{quote}\begin{description}
\item[{Parameters}] \leavevmode\begin{itemize}
\item {} 
\sphinxstyleliteralstrong{\sphinxupquote{xi}} (\sphinxstyleliteralemphasis{\sphinxupquote{scalar float}}\sphinxstyleliteralemphasis{\sphinxupquote{ or }}\sphinxstyleliteralemphasis{\sphinxupquote{1-dimensional float}}) \textendash{} Position in x dimension.
This is the first dimension of 3d-valued grid.

\item {} 
\sphinxstyleliteralstrong{\sphinxupquote{yi}} (\sphinxstyleliteralemphasis{\sphinxupquote{scalar float}}\sphinxstyleliteralemphasis{\sphinxupquote{ or }}\sphinxstyleliteralemphasis{\sphinxupquote{1-dimensional float}}) \textendash{} Position in y dimension.
This is the second dimension of 3d-valued grid.

\item {} 
\sphinxstyleliteralstrong{\sphinxupquote{zi}} (\sphinxstyleliteralemphasis{\sphinxupquote{scalar float}}\sphinxstyleliteralemphasis{\sphinxupquote{ or }}\sphinxstyleliteralemphasis{\sphinxupquote{1-dimensional float}}) \textendash{} Position in z dimension.
This is the third dimension of 3d-valued grid.

\end{itemize}

\item[{Returns}] \leavevmode
The interpolated value at (xi,yi,zi).

\item[{Return type}] \leavevmode
val (array or scalar float)

\item[{Raises}] \leavevmode
\sphinxstyleliteralstrong{\sphinxupquote{ValueError}} \textendash{} If any of the dimensions exceed the evaluation boundary
    of the grid

\end{description}\end{quote}

\end{fulllineitems}


\end{fulllineitems}

\index{RectBivariateSpline (class in eqtools.trispline)@\spxentry{RectBivariateSpline}\spxextra{class in eqtools.trispline}}

\begin{fulllineitems}
\phantomsection\label{\detokenize{eqtools:eqtools.trispline.RectBivariateSpline}}\pysiglinewithargsret{\sphinxbfcode{\sphinxupquote{class }}\sphinxcode{\sphinxupquote{eqtools.trispline.}}\sphinxbfcode{\sphinxupquote{RectBivariateSpline}}}{\emph{x, y, z, bbox={[}None, None, None, None{]}, kx=3, ky=3, s=0, bounds\_error=True, fill\_value=nan}}{}
Bases: \sphinxcode{\sphinxupquote{scipy.interpolate.fitpack2.RectBivariateSpline}}

the lack of a graceful bounds error causes the fortran to fail hard.
This masks scipy.interpolate.RectBivariateSpline with a proper bound
checker and value filler such that it will not fail in use for EqTools

Can be used for both smoothing and interpolating data.
\begin{quote}\begin{description}
\item[{Parameters}] \leavevmode\begin{itemize}
\item {} 
\sphinxstyleliteralstrong{\sphinxupquote{x}} (\sphinxstyleliteralemphasis{\sphinxupquote{1-dimensional float array}}) \textendash{} 1-D array of coordinates in monotonically increasing order.

\item {} 
\sphinxstyleliteralstrong{\sphinxupquote{y}} (\sphinxstyleliteralemphasis{\sphinxupquote{1-dimensional float array}}) \textendash{} 1-D array of coordinates in monotonically increasing order.

\item {} 
\sphinxstyleliteralstrong{\sphinxupquote{z}} (\sphinxstyleliteralemphasis{\sphinxupquote{2-dimensional float array}}) \textendash{} 2-D array of data with shape (x.size,y.size).

\end{itemize}

\item[{Keyword Arguments}] \leavevmode\begin{itemize}
\item {} 
\sphinxstyleliteralstrong{\sphinxupquote{bbox}} (\sphinxstyleliteralemphasis{\sphinxupquote{1-dimensional float}}) \textendash{} Sequence of length 4 specifying the
boundary of the rectangular approximation domain.  By default,
\sphinxcode{\sphinxupquote{bbox={[}min(x,tx),max(x,tx), min(y,ty),max(y,ty){]}}}.

\item {} 
\sphinxstyleliteralstrong{\sphinxupquote{kx}} (\sphinxstyleliteralemphasis{\sphinxupquote{integer}}) \textendash{} Degrees of the bivariate spline. Default is 3.

\item {} 
\sphinxstyleliteralstrong{\sphinxupquote{ky}} (\sphinxstyleliteralemphasis{\sphinxupquote{integer}}) \textendash{} Degrees of the bivariate spline. Default is 3.

\item {} 
\sphinxstyleliteralstrong{\sphinxupquote{s}} (\sphinxstyleliteralemphasis{\sphinxupquote{float}}) \textendash{} Positive smoothing factor defined for estimation condition,
\sphinxcode{\sphinxupquote{sum((w{[}i{]}*(z{[}i{]}-s(x{[}i{]}, y{[}i{]})))**2, axis=0) \textless{}= s}}
Default is \sphinxcode{\sphinxupquote{s=0}}, which is for interpolation.

\end{itemize}

\end{description}\end{quote}
\index{ev() (eqtools.trispline.RectBivariateSpline method)@\spxentry{ev()}\spxextra{eqtools.trispline.RectBivariateSpline method}}

\begin{fulllineitems}
\phantomsection\label{\detokenize{eqtools:eqtools.trispline.RectBivariateSpline.ev}}\pysiglinewithargsret{\sphinxbfcode{\sphinxupquote{ev}}}{\emph{xi}, \emph{yi}}{}~\begin{description}
\item[{Evaluate the rectBiVariateSpline at (xi,yi).  (x,y)values are}] \leavevmode
checked for being in the bounds of the interpolated data.

\end{description}
\begin{quote}\begin{description}
\item[{Parameters}] \leavevmode\begin{itemize}
\item {} 
\sphinxstyleliteralstrong{\sphinxupquote{xi}} (\sphinxstyleliteralemphasis{\sphinxupquote{float array}}) \textendash{} input x dimensional values

\item {} 
\sphinxstyleliteralstrong{\sphinxupquote{yi}} (\sphinxstyleliteralemphasis{\sphinxupquote{float array}}) \textendash{} input x dimensional values

\end{itemize}

\item[{Returns}] \leavevmode
evaluated spline at points (x{[}i{]}, y{[}i{]}), i=0,…,len(x)-1

\item[{Return type}] \leavevmode
val (float array)

\end{description}\end{quote}

\end{fulllineitems}


\end{fulllineitems}

\index{BivariateInterpolator (class in eqtools.trispline)@\spxentry{BivariateInterpolator}\spxextra{class in eqtools.trispline}}

\begin{fulllineitems}
\phantomsection\label{\detokenize{eqtools:eqtools.trispline.BivariateInterpolator}}\pysiglinewithargsret{\sphinxbfcode{\sphinxupquote{class }}\sphinxcode{\sphinxupquote{eqtools.trispline.}}\sphinxbfcode{\sphinxupquote{BivariateInterpolator}}}{\emph{x}, \emph{y}, \emph{z}}{}
Bases: \sphinxcode{\sphinxupquote{object}}

This class provides a wrapper for \sphinxtitleref{scipy.interpolate.CloughTocher2DInterpolator}.

This is necessary because \sphinxtitleref{scipy.interpolate.SmoothBivariateSpline} cannot
be made to interpolate, and gives inaccurate answers near the boundaries.
\index{ev() (eqtools.trispline.BivariateInterpolator method)@\spxentry{ev()}\spxextra{eqtools.trispline.BivariateInterpolator method}}

\begin{fulllineitems}
\phantomsection\label{\detokenize{eqtools:eqtools.trispline.BivariateInterpolator.ev}}\pysiglinewithargsret{\sphinxbfcode{\sphinxupquote{ev}}}{\emph{xi}, \emph{yi}}{}
\end{fulllineitems}


\end{fulllineitems}

\index{UnivariateInterpolator (class in eqtools.trispline)@\spxentry{UnivariateInterpolator}\spxextra{class in eqtools.trispline}}

\begin{fulllineitems}
\phantomsection\label{\detokenize{eqtools:eqtools.trispline.UnivariateInterpolator}}\pysiglinewithargsret{\sphinxbfcode{\sphinxupquote{class }}\sphinxcode{\sphinxupquote{eqtools.trispline.}}\sphinxbfcode{\sphinxupquote{UnivariateInterpolator}}}{\emph{*args}, \emph{**kwargs}}{}
Bases: \sphinxcode{\sphinxupquote{scipy.interpolate.fitpack2.InterpolatedUnivariateSpline}}

Interpolated spline class which overcomes the shortcomings of interp1d
(inaccurate near edges) and InterpolatedUnivariateSpline (can’t set NaN
where it extrapolates).

\end{fulllineitems}



\subsection{Module contents}
\label{\detokenize{eqtools:module-eqtools}}\label{\detokenize{eqtools:module-contents}}\index{eqtools (module)@\spxentry{eqtools}\spxextra{module}}
Provides classes for interacting with magnetic equilibrium data in a variety of formats.


\chapter{Indices and tables}
\label{\detokenize{index:indices-and-tables}}\begin{itemize}
\item {} 
\DUrole{xref,std,std-ref}{genindex}

\item {} 
\DUrole{xref,std,std-ref}{modindex}

\item {} 
\DUrole{xref,std,std-ref}{search}

\end{itemize}


\renewcommand{\indexname}{Python Module Index}
\begin{sphinxtheindex}
\let\bigletter\sphinxstyleindexlettergroup
\bigletter{e}
\item\relax\sphinxstyleindexentry{eqtools}\sphinxstyleindexpageref{eqtools:\detokenize{module-eqtools}}
\item\relax\sphinxstyleindexentry{eqtools.afilereader}\sphinxstyleindexpageref{eqtools:\detokenize{module-eqtools.afilereader}}
\item\relax\sphinxstyleindexentry{eqtools.AUGData}\sphinxstyleindexpageref{eqtools:\detokenize{module-eqtools.AUGData}}
\item\relax\sphinxstyleindexentry{eqtools.CModEFIT}\sphinxstyleindexpageref{eqtools:\detokenize{module-eqtools.CModEFIT}}
\item\relax\sphinxstyleindexentry{eqtools.core}\sphinxstyleindexpageref{eqtools:\detokenize{module-eqtools.core}}
\item\relax\sphinxstyleindexentry{eqtools.D3DEFIT}\sphinxstyleindexpageref{eqtools:\detokenize{module-eqtools.D3DEFIT}}
\item\relax\sphinxstyleindexentry{eqtools.EFIT}\sphinxstyleindexpageref{eqtools:\detokenize{module-eqtools.EFIT}}
\item\relax\sphinxstyleindexentry{eqtools.eqdskreader}\sphinxstyleindexpageref{eqtools:\detokenize{module-eqtools.eqdskreader}}
\item\relax\sphinxstyleindexentry{eqtools.FromArrays}\sphinxstyleindexpageref{eqtools:\detokenize{module-eqtools.FromArrays}}
\item\relax\sphinxstyleindexentry{eqtools.NSTXEFIT}\sphinxstyleindexpageref{eqtools:\detokenize{module-eqtools.NSTXEFIT}}
\item\relax\sphinxstyleindexentry{eqtools.pfilereader}\sphinxstyleindexpageref{eqtools:\detokenize{module-eqtools.pfilereader}}
\item\relax\sphinxstyleindexentry{eqtools.TCVLIUQE}\sphinxstyleindexpageref{eqtools:\detokenize{module-eqtools.TCVLIUQE}}
\item\relax\sphinxstyleindexentry{eqtools.trispline}\sphinxstyleindexpageref{eqtools:\detokenize{module-eqtools.trispline}}
\end{sphinxtheindex}

\renewcommand{\indexname}{Index}
\printindex
\end{document}