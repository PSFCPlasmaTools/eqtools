% Generated by Sphinx.
\def\sphinxdocclass{report}
\documentclass[letterpaper,10pt,english]{sphinxmanual}
\usepackage[utf8]{inputenc}
\DeclareUnicodeCharacter{00A0}{\nobreakspace}
\usepackage[T1]{fontenc}
\usepackage{babel}
\usepackage{times}
\usepackage[Bjarne]{fncychap}
\usepackage{longtable}
\usepackage{sphinx}
\usepackage{multirow}
\hypersetup{bookmarksdepth=3}

\title{eqtools Documentation}
\date{September 05, 2013}
\release{0.0}
\author{Mark Chilenski, Ian Faust and John Walk}
\newcommand{\sphinxlogo}{}
\renewcommand{\releasename}{Release}
\makeindex

\makeatletter
\def\PYG@reset{\let\PYG@it=\relax \let\PYG@bf=\relax%
    \let\PYG@ul=\relax \let\PYG@tc=\relax%
    \let\PYG@bc=\relax \let\PYG@ff=\relax}
\def\PYG@tok#1{\csname PYG@tok@#1\endcsname}
\def\PYG@toks#1+{\ifx\relax#1\empty\else%
    \PYG@tok{#1}\expandafter\PYG@toks\fi}
\def\PYG@do#1{\PYG@bc{\PYG@tc{\PYG@ul{%
    \PYG@it{\PYG@bf{\PYG@ff{#1}}}}}}}
\def\PYG#1#2{\PYG@reset\PYG@toks#1+\relax+\PYG@do{#2}}

\expandafter\def\csname PYG@tok@gd\endcsname{\def\PYG@tc##1{\textcolor[rgb]{0.63,0.00,0.00}{##1}}}
\expandafter\def\csname PYG@tok@gu\endcsname{\let\PYG@bf=\textbf\def\PYG@tc##1{\textcolor[rgb]{0.50,0.00,0.50}{##1}}}
\expandafter\def\csname PYG@tok@gt\endcsname{\def\PYG@tc##1{\textcolor[rgb]{0.00,0.27,0.87}{##1}}}
\expandafter\def\csname PYG@tok@gs\endcsname{\let\PYG@bf=\textbf}
\expandafter\def\csname PYG@tok@gr\endcsname{\def\PYG@tc##1{\textcolor[rgb]{1.00,0.00,0.00}{##1}}}
\expandafter\def\csname PYG@tok@cm\endcsname{\let\PYG@it=\textit\def\PYG@tc##1{\textcolor[rgb]{0.25,0.50,0.56}{##1}}}
\expandafter\def\csname PYG@tok@vg\endcsname{\def\PYG@tc##1{\textcolor[rgb]{0.73,0.38,0.84}{##1}}}
\expandafter\def\csname PYG@tok@m\endcsname{\def\PYG@tc##1{\textcolor[rgb]{0.13,0.50,0.31}{##1}}}
\expandafter\def\csname PYG@tok@mh\endcsname{\def\PYG@tc##1{\textcolor[rgb]{0.13,0.50,0.31}{##1}}}
\expandafter\def\csname PYG@tok@cs\endcsname{\def\PYG@tc##1{\textcolor[rgb]{0.25,0.50,0.56}{##1}}\def\PYG@bc##1{\setlength{\fboxsep}{0pt}\colorbox[rgb]{1.00,0.94,0.94}{\strut ##1}}}
\expandafter\def\csname PYG@tok@ge\endcsname{\let\PYG@it=\textit}
\expandafter\def\csname PYG@tok@vc\endcsname{\def\PYG@tc##1{\textcolor[rgb]{0.73,0.38,0.84}{##1}}}
\expandafter\def\csname PYG@tok@il\endcsname{\def\PYG@tc##1{\textcolor[rgb]{0.13,0.50,0.31}{##1}}}
\expandafter\def\csname PYG@tok@go\endcsname{\def\PYG@tc##1{\textcolor[rgb]{0.20,0.20,0.20}{##1}}}
\expandafter\def\csname PYG@tok@cp\endcsname{\def\PYG@tc##1{\textcolor[rgb]{0.00,0.44,0.13}{##1}}}
\expandafter\def\csname PYG@tok@gi\endcsname{\def\PYG@tc##1{\textcolor[rgb]{0.00,0.63,0.00}{##1}}}
\expandafter\def\csname PYG@tok@gh\endcsname{\let\PYG@bf=\textbf\def\PYG@tc##1{\textcolor[rgb]{0.00,0.00,0.50}{##1}}}
\expandafter\def\csname PYG@tok@ni\endcsname{\let\PYG@bf=\textbf\def\PYG@tc##1{\textcolor[rgb]{0.84,0.33,0.22}{##1}}}
\expandafter\def\csname PYG@tok@nl\endcsname{\let\PYG@bf=\textbf\def\PYG@tc##1{\textcolor[rgb]{0.00,0.13,0.44}{##1}}}
\expandafter\def\csname PYG@tok@nn\endcsname{\let\PYG@bf=\textbf\def\PYG@tc##1{\textcolor[rgb]{0.05,0.52,0.71}{##1}}}
\expandafter\def\csname PYG@tok@no\endcsname{\def\PYG@tc##1{\textcolor[rgb]{0.38,0.68,0.84}{##1}}}
\expandafter\def\csname PYG@tok@na\endcsname{\def\PYG@tc##1{\textcolor[rgb]{0.25,0.44,0.63}{##1}}}
\expandafter\def\csname PYG@tok@nb\endcsname{\def\PYG@tc##1{\textcolor[rgb]{0.00,0.44,0.13}{##1}}}
\expandafter\def\csname PYG@tok@nc\endcsname{\let\PYG@bf=\textbf\def\PYG@tc##1{\textcolor[rgb]{0.05,0.52,0.71}{##1}}}
\expandafter\def\csname PYG@tok@nd\endcsname{\let\PYG@bf=\textbf\def\PYG@tc##1{\textcolor[rgb]{0.33,0.33,0.33}{##1}}}
\expandafter\def\csname PYG@tok@ne\endcsname{\def\PYG@tc##1{\textcolor[rgb]{0.00,0.44,0.13}{##1}}}
\expandafter\def\csname PYG@tok@nf\endcsname{\def\PYG@tc##1{\textcolor[rgb]{0.02,0.16,0.49}{##1}}}
\expandafter\def\csname PYG@tok@si\endcsname{\let\PYG@it=\textit\def\PYG@tc##1{\textcolor[rgb]{0.44,0.63,0.82}{##1}}}
\expandafter\def\csname PYG@tok@s2\endcsname{\def\PYG@tc##1{\textcolor[rgb]{0.25,0.44,0.63}{##1}}}
\expandafter\def\csname PYG@tok@vi\endcsname{\def\PYG@tc##1{\textcolor[rgb]{0.73,0.38,0.84}{##1}}}
\expandafter\def\csname PYG@tok@nt\endcsname{\let\PYG@bf=\textbf\def\PYG@tc##1{\textcolor[rgb]{0.02,0.16,0.45}{##1}}}
\expandafter\def\csname PYG@tok@nv\endcsname{\def\PYG@tc##1{\textcolor[rgb]{0.73,0.38,0.84}{##1}}}
\expandafter\def\csname PYG@tok@s1\endcsname{\def\PYG@tc##1{\textcolor[rgb]{0.25,0.44,0.63}{##1}}}
\expandafter\def\csname PYG@tok@gp\endcsname{\let\PYG@bf=\textbf\def\PYG@tc##1{\textcolor[rgb]{0.78,0.36,0.04}{##1}}}
\expandafter\def\csname PYG@tok@sh\endcsname{\def\PYG@tc##1{\textcolor[rgb]{0.25,0.44,0.63}{##1}}}
\expandafter\def\csname PYG@tok@ow\endcsname{\let\PYG@bf=\textbf\def\PYG@tc##1{\textcolor[rgb]{0.00,0.44,0.13}{##1}}}
\expandafter\def\csname PYG@tok@sx\endcsname{\def\PYG@tc##1{\textcolor[rgb]{0.78,0.36,0.04}{##1}}}
\expandafter\def\csname PYG@tok@bp\endcsname{\def\PYG@tc##1{\textcolor[rgb]{0.00,0.44,0.13}{##1}}}
\expandafter\def\csname PYG@tok@c1\endcsname{\let\PYG@it=\textit\def\PYG@tc##1{\textcolor[rgb]{0.25,0.50,0.56}{##1}}}
\expandafter\def\csname PYG@tok@kc\endcsname{\let\PYG@bf=\textbf\def\PYG@tc##1{\textcolor[rgb]{0.00,0.44,0.13}{##1}}}
\expandafter\def\csname PYG@tok@c\endcsname{\let\PYG@it=\textit\def\PYG@tc##1{\textcolor[rgb]{0.25,0.50,0.56}{##1}}}
\expandafter\def\csname PYG@tok@mf\endcsname{\def\PYG@tc##1{\textcolor[rgb]{0.13,0.50,0.31}{##1}}}
\expandafter\def\csname PYG@tok@err\endcsname{\def\PYG@bc##1{\setlength{\fboxsep}{0pt}\fcolorbox[rgb]{1.00,0.00,0.00}{1,1,1}{\strut ##1}}}
\expandafter\def\csname PYG@tok@kd\endcsname{\let\PYG@bf=\textbf\def\PYG@tc##1{\textcolor[rgb]{0.00,0.44,0.13}{##1}}}
\expandafter\def\csname PYG@tok@ss\endcsname{\def\PYG@tc##1{\textcolor[rgb]{0.32,0.47,0.09}{##1}}}
\expandafter\def\csname PYG@tok@sr\endcsname{\def\PYG@tc##1{\textcolor[rgb]{0.14,0.33,0.53}{##1}}}
\expandafter\def\csname PYG@tok@mo\endcsname{\def\PYG@tc##1{\textcolor[rgb]{0.13,0.50,0.31}{##1}}}
\expandafter\def\csname PYG@tok@mi\endcsname{\def\PYG@tc##1{\textcolor[rgb]{0.13,0.50,0.31}{##1}}}
\expandafter\def\csname PYG@tok@kn\endcsname{\let\PYG@bf=\textbf\def\PYG@tc##1{\textcolor[rgb]{0.00,0.44,0.13}{##1}}}
\expandafter\def\csname PYG@tok@o\endcsname{\def\PYG@tc##1{\textcolor[rgb]{0.40,0.40,0.40}{##1}}}
\expandafter\def\csname PYG@tok@kr\endcsname{\let\PYG@bf=\textbf\def\PYG@tc##1{\textcolor[rgb]{0.00,0.44,0.13}{##1}}}
\expandafter\def\csname PYG@tok@s\endcsname{\def\PYG@tc##1{\textcolor[rgb]{0.25,0.44,0.63}{##1}}}
\expandafter\def\csname PYG@tok@kp\endcsname{\def\PYG@tc##1{\textcolor[rgb]{0.00,0.44,0.13}{##1}}}
\expandafter\def\csname PYG@tok@w\endcsname{\def\PYG@tc##1{\textcolor[rgb]{0.73,0.73,0.73}{##1}}}
\expandafter\def\csname PYG@tok@kt\endcsname{\def\PYG@tc##1{\textcolor[rgb]{0.56,0.13,0.00}{##1}}}
\expandafter\def\csname PYG@tok@sc\endcsname{\def\PYG@tc##1{\textcolor[rgb]{0.25,0.44,0.63}{##1}}}
\expandafter\def\csname PYG@tok@sb\endcsname{\def\PYG@tc##1{\textcolor[rgb]{0.25,0.44,0.63}{##1}}}
\expandafter\def\csname PYG@tok@k\endcsname{\let\PYG@bf=\textbf\def\PYG@tc##1{\textcolor[rgb]{0.00,0.44,0.13}{##1}}}
\expandafter\def\csname PYG@tok@se\endcsname{\let\PYG@bf=\textbf\def\PYG@tc##1{\textcolor[rgb]{0.25,0.44,0.63}{##1}}}
\expandafter\def\csname PYG@tok@sd\endcsname{\let\PYG@it=\textit\def\PYG@tc##1{\textcolor[rgb]{0.25,0.44,0.63}{##1}}}

\def\PYGZbs{\char`\\}
\def\PYGZus{\char`\_}
\def\PYGZob{\char`\{}
\def\PYGZcb{\char`\}}
\def\PYGZca{\char`\^}
\def\PYGZam{\char`\&}
\def\PYGZlt{\char`\<}
\def\PYGZgt{\char`\>}
\def\PYGZsh{\char`\#}
\def\PYGZpc{\char`\%}
\def\PYGZdl{\char`\$}
\def\PYGZhy{\char`\-}
\def\PYGZsq{\char`\'}
\def\PYGZdq{\char`\"}
\def\PYGZti{\char`\~}
% for compatibility with earlier versions
\def\PYGZat{@}
\def\PYGZlb{[}
\def\PYGZrb{]}
\makeatother

\begin{document}

\maketitle
\tableofcontents
\phantomsection\label{index::doc}



\chapter{Overview}
\label{index:overview}\label{index:eqtools-tools-for-interacting-with-magnetic-equilibria}
Useful description and details go here.


\chapter{Package Reference}
\label{index:package-reference}

\section{eqtools Package}
\label{eqtools::doc}\label{eqtools:eqtools-package}

\subsection{\texttt{eqtools} Package}
\label{eqtools:id1}\phantomsection\label{eqtools:module-eqtools.__init__}\index{eqtools.\_\_init\_\_ (module)}
Provides classes for interacting with magnetic equilibrium data in a variety of formats.


\subsection{\texttt{core} Module}
\label{eqtools:module-eqtools.core}\label{eqtools:core-module}\index{eqtools.core (module)}
This module provides the core classes for eqtools, including the base Equilibrium class.
\index{ModuleWarning}

\begin{fulllineitems}
\phantomsection\label{eqtools:eqtools.core.ModuleWarning}\pysigline{\strong{exception }\code{eqtools.core.}\bfcode{ModuleWarning}}
Bases: \code{exceptions.Warning}

Warning class to notify the user of unavailable modules.

\end{fulllineitems}

\index{PropertyAccessMixin (class in eqtools.core)}

\begin{fulllineitems}
\phantomsection\label{eqtools:eqtools.core.PropertyAccessMixin}\pysigline{\strong{class }\code{eqtools.core.}\bfcode{PropertyAccessMixin}}
Bases: \code{object}

Mixin to implement access of getter methods through a property-type
interface without the need to apply a decorator to every property.

For any getter obj.getSomething(), the call obj.Something will work.

This is accomplished by overriding \_\_getattribute\_\_ such that if an
attribute ATTR does not exist it then attempts to call self.getATTR(). If
self.getATTR() does not exist, an AttributeError will be raised as usual.

Also overrides \_\_setattr\_\_ such that it will raise an AttributeError when
attempting to write an attribute ATTR for which there is already a method
getATTR.
\index{\_\_getattribute\_\_() (eqtools.core.PropertyAccessMixin method)}

\begin{fulllineitems}
\phantomsection\label{eqtools:eqtools.core.PropertyAccessMixin.__getattribute__}\pysiglinewithargsret{\bfcode{\_\_getattribute\_\_}}{\emph{name}}{}
Get an attribute.

Tries to get attribute as-written. If this fails, tries to call the
method get\textless{}name\textgreater{} with no arguments. If this fails, raises
AttributeError. This effectively generates a Python `property' for
each getter method.
\begin{quote}\begin{description}
\item[{Parameters }] \leavevmode
\textbf{name} --
String.
Name of the attribute to retrieve. If the instance
has an attribute with this name, the attribute is returned. If
the instance does not have an attribute with this name but does
have a method called `get'+name, this method is called and the
result is returned.

\item[{Returns}] \leavevmode
The value of the attribute requested.

\item[{Raises }] \leavevmode
\code{AttributeError} -- 
If neither attribute name or method `get'+name exist.

\end{description}\end{quote}

\end{fulllineitems}

\index{\_\_setattr\_\_() (eqtools.core.PropertyAccessMixin method)}

\begin{fulllineitems}
\phantomsection\label{eqtools:eqtools.core.PropertyAccessMixin.__setattr__}\pysiglinewithargsret{\bfcode{\_\_setattr\_\_}}{\emph{name}, \emph{value}}{}
Set an attribute.

Raises AttributeError if the object already has a method get{[}name{]}, as
creation of such an attribute would interfere with the automatic
property generation in \_\_getattribute\_\_.
\begin{quote}\begin{description}
\item[{Parameters }] \leavevmode\begin{itemize}
\item {} 
\textbf{name} --
String.
Name of the attribute to set.

\item {} 
\textbf{value} --
Object.
Value to set the attribute to.

\end{itemize}

\item[{Raises }] \leavevmode
\code{AttributeError} -- 
If a method called `get'+name already exists.

\end{description}\end{quote}

\end{fulllineitems}


\end{fulllineitems}

\index{Equilibrium (class in eqtools.core)}

\begin{fulllineitems}
\phantomsection\label{eqtools:eqtools.core.Equilibrium}\pysiglinewithargsret{\strong{class }\code{eqtools.core.}\bfcode{Equilibrium}}{\emph{length\_unit='m'}, \emph{tspline=False}, \emph{monotonic=False}}{}
Bases: \code{object}

Abstract class of data handling object for magnetic reconstruction outputs.

Defines the mapping routines and method fingerprints necessary.
Each variable or set of variables is recovered with a corresponding
getter method. Essential data for mapping are pulled on initialization
(psirz grid, for example) to frontload timing overhead. Additional data
are pulled at the first request and stored for subsequent usage.

\begin{notice}{note}{Note:}
This abstract class should not be used directly. Device- and code-
specific subclasses are set up to account for inter-device/-code differences
in data storage.
\end{notice}

Create a new Equilibrium instance.
\begin{quote}\begin{description}
\item[{Keyword Arguments}] \leavevmode\begin{itemize}
\item {} 
\textbf{length\_unit} --
String. Sets the base unit used for any quantity whose
dimensions are length to any power. Valid options are:
\begin{quote}

\begin{tabulary}{\linewidth}{|L|L|}
\hline

`m'
 & 
meters
\\\hline

`cm'
 & 
centimeters
\\\hline

`mm'
 & 
millimeters
\\\hline

`in'
 & 
inches
\\\hline

`ft'
 & 
feet
\\\hline

`yd'
 & 
yards
\\\hline

`smoot'
 & 
smoots
\\\hline

`cubit'
 & 
cubits
\\\hline

`hand'
 & 
hands
\\\hline

`default'
 & 
whatever the default in the tree is (no conversion is performed, units may be inconsistent)
\\\hline
\end{tabulary}

\end{quote}

Default is `m' (all units taken and returned in meters).

\item {} 
\textbf{tspline} --
Boolean.
Sets whether or not interpolation in time is
performed using a tricubic spline or nearest-neighbor
interpolation. Tricubic spline interpolation requires at least
four complete equilibria at different times. It is also assumed
that they are functionally correlated, and that parameters do
not vary out of their boundaries (derivative = 0 boundary
condition). Default is False (use nearest neighbor interpolation).

\item {} 
\textbf{monotonic} --
Boolean.
Sets whether or not the ``monotonic'' form of time window
finding is used. If True, the timebase must be monotonically
increasing. Default is False (use slower, safer method).

\end{itemize}

\item[{Raises }] \leavevmode\begin{itemize}
\item {} 
\code{ValueError} -- 
If length\_unit is not a valid unit specifier.

\item {} 
\code{ValueError} -- 
If tspline is True by module trispline did not load
successfully.

\end{itemize}

\end{description}\end{quote}
\index{\_\_str\_\_() (eqtools.core.Equilibrium method)}

\begin{fulllineitems}
\phantomsection\label{eqtools:eqtools.core.Equilibrium.__str__}\pysiglinewithargsret{\bfcode{\_\_str\_\_}}{}{}
String representation of this instance.
\begin{quote}\begin{description}
\item[{Returns}] \leavevmode
String describing this object.

\end{description}\end{quote}

\end{fulllineitems}

\index{rz2psi() (eqtools.core.Equilibrium method)}

\begin{fulllineitems}
\phantomsection\label{eqtools:eqtools.core.Equilibrium.rz2psi}\pysiglinewithargsret{\bfcode{rz2psi}}{\emph{R}, \emph{Z}, \emph{t}, \emph{return\_t=False}, \emph{make\_grid=False}, \emph{length\_unit=1}}{}
Converts the passed R, Z, t arrays to psi values.

If tspline is False for this Equilibrium instance, uses
scipy.interpolate.RectBivariateSpline to interpolate in terms of R and
Z. Finds the nearest time slices to those given: nearest-neighbor
interpolation in time. Otherwise, uses the tricubic package to perform
a trivariate interpolation in space and time.
\begin{quote}\begin{description}
\item[{Parameters }] \leavevmode\begin{itemize}
\item {} 
\textbf{R} --
Array-like or scalar float.
Values of the radial coordinate to
map to poloidal flux. If R and Z are both scalar values, they
are used as the coordinate pair for all of the values in t.
Must have the same shape as Z unless the make\_grid keyword is
set. If the make\_grid keyword is True, R must have shape (len\_R,).

\item {} 
\textbf{Z} --
Array-like or scalar float.
Values of the vertical coordinate to
map to poloidal flux. If R and Z are both scalar values, they
are used as the coordinate pair for all of the values in t.
Must have the same shape as R unless the make\_grid keyword is
set. If the make\_grid keyword is True, Z must have shape (len\_Z,).

\item {} 
\textbf{t} --
Array-like or single value.
If t is a single value, it is used
for all of the elements of R, Z. If t is array-like and the
make\_grid keyword is False, t must have the same dimensions as
R and Z. If t is array-like and the make\_grid keyword is True,
t must have shape (len(Z), len(R)).

\end{itemize}

\item[{Keyword Arguments}] \leavevmode\begin{itemize}
\item {} 
\textbf{return\_t} --
Boolean.
Set to True to return a tuple of (psi, time\_idxs),
where time\_idxs is the array of time indices actually used in
evaluating psi with nearest-neighbor interpolation. (This is
mostly present as an internal helper.) Default is False (only
return psi).

\item {} 
\textbf{make\_grid} --
Boolean.
Set to True to pass R and Z through meshgrid
before evaluating. If this is set to True, R and Z must each
only have a single dimension, but can have different lengths.
When using this option, it is highly recommended to only pass
a scalar value for t (such that each point in the flux grid is
evaluated at this same value t). Otherwise, t must have the
same shape as the resulting meshgrid, and each element in the
returned psi array will be at the corresponding time in the t
array. Default is False (do not form meshgrid).

\item {} 
\textbf{length\_unit} --
String or 1.
Length unit that R and Z are being given
in. If a string is given, it must be a valid unit specifier:
\begin{quote}

\begin{tabulary}{\linewidth}{|L|L|}
\hline

`m'
 & 
meters
\\\hline

`cm'
 & 
centimeters
\\\hline

`mm'
 & 
millimeters
\\\hline

`in'
 & 
inches
\\\hline

`ft'
 & 
feet
\\\hline

`yd'
 & 
yards
\\\hline

`smoot'
 & 
smoots
\\\hline

`cubit'
 & 
cubits
\\\hline

`hand'
 & 
hands
\\\hline

`default'
 & 
meters
\\\hline
\end{tabulary}

\end{quote}

If length\_unit is 1 or None, meters are assumed. The default
value is 1 (R and Z given in meters).

\end{itemize}

\item[{Returns}] \leavevmode

psi or (psi, time\_idxs)
\begin{itemize}
\item {} 
\textbf{psi} - Array or scalar float. If all of the input arguments are scalar,
then a scalar is returned. Otherwise, a scipy Array instance is
returned. If R and Z both have the same shape then psi has this
shape as well. If the make\_grid keyword was True then psi has
shape (len(Z), len(R)).

\item {} 
\textbf{time\_idxs} - Array with same shape as psi. The indices (in the
timebase as returned by {\hyperref[eqtools:eqtools.core.Equilibrium.getTimeBase]{\code{getTimeBase()}}}) that were used
for nearest-neighbor interpolation. Only returned if return\_t
is True.

\end{itemize}


\end{description}\end{quote}
\paragraph{Examples}

All assume that Eq\_instance is a valid instance of the appropriate
extension of the Equilibrium abstract class.

Find single psi value at R=0.6m, Z=0.0m, t=0.26s:

\begin{Verbatim}[commandchars=\\\{\}]
\PYG{n}{psi\PYGZus{}val} \PYG{o}{=} \PYG{n}{Eq\PYGZus{}instance}\PYG{o}{.}\PYG{n}{rz2psi}\PYG{p}{(}\PYG{l+m+mf}{0.6}\PYG{p}{,} \PYG{l+m+mi}{0}\PYG{p}{,} \PYG{l+m+mf}{0.26}\PYG{p}{)}
\end{Verbatim}

Find psi values at (R, Z) points (0.6m, 0m) and (0.8m, 0m) at the single
time t=0.26s. Note that the Z vector must be fully specified, even if
the values are all the same:

\begin{Verbatim}[commandchars=\\\{\}]
\PYG{n}{psi\PYGZus{}arr} \PYG{o}{=} \PYG{n}{Eq\PYGZus{}instance}\PYG{o}{.}\PYG{n}{rz2psi}\PYG{p}{(}\PYG{p}{[}\PYG{l+m+mf}{0.6}\PYG{p}{,} \PYG{l+m+mf}{0.8}\PYG{p}{]}\PYG{p}{,} \PYG{p}{[}\PYG{l+m+mi}{0}\PYG{p}{,} \PYG{l+m+mi}{0}\PYG{p}{]}\PYG{p}{,} \PYG{l+m+mf}{0.26}\PYG{p}{)}
\end{Verbatim}

Find psi values at (R, Z) points (0.6m, 0m) at times t={[}0.2s, 0.3s{]}:

\begin{Verbatim}[commandchars=\\\{\}]
\PYG{n}{psi\PYGZus{}arr} \PYG{o}{=} \PYG{n}{Eq\PYGZus{}instance}\PYG{o}{.}\PYG{n}{rz2psi}\PYG{p}{(}\PYG{l+m+mf}{0.6}\PYG{p}{,} \PYG{l+m+mi}{0}\PYG{p}{,} \PYG{p}{[}\PYG{l+m+mf}{0.2}\PYG{p}{,} \PYG{l+m+mf}{0.3}\PYG{p}{]}\PYG{p}{)}
\end{Verbatim}

Find psi values at (R, Z, t) points (0.6m, 0m, 0.2s) and (0.5m, 0.2m, 0.3s):

\begin{Verbatim}[commandchars=\\\{\}]
\PYG{n}{psi\PYGZus{}arr} \PYG{o}{=} \PYG{n}{Eq\PYGZus{}instance}\PYG{o}{.}\PYG{n}{rz2psi}\PYG{p}{(}\PYG{p}{[}\PYG{l+m+mf}{0.6}\PYG{p}{,} \PYG{l+m+mf}{0.5}\PYG{p}{]}\PYG{p}{,} \PYG{p}{[}\PYG{l+m+mi}{0}\PYG{p}{,} \PYG{l+m+mf}{0.2}\PYG{p}{]}\PYG{p}{,} \PYG{p}{[}\PYG{l+m+mf}{0.2}\PYG{p}{,} \PYG{l+m+mf}{0.3}\PYG{p}{]}\PYG{p}{)}
\end{Verbatim}

Find psi values on grid defined by 1D vector of radial positions R and
1D vector of vertical positions Z at time t=0.2s:

\begin{Verbatim}[commandchars=\\\{\}]
\PYG{n}{psi\PYGZus{}mat} \PYG{o}{=} \PYG{n}{Eq\PYGZus{}instance}\PYG{o}{.}\PYG{n}{rz2psi}\PYG{p}{(}\PYG{n}{R}\PYG{p}{,} \PYG{n}{Z}\PYG{p}{,} \PYG{l+m+mf}{0.2}\PYG{p}{,} \PYG{n}{make\PYGZus{}grid}\PYG{o}{=}\PYG{n+nb+bp}{True}\PYG{p}{)}
\end{Verbatim}

\end{fulllineitems}

\index{rz2psinorm() (eqtools.core.Equilibrium method)}

\begin{fulllineitems}
\phantomsection\label{eqtools:eqtools.core.Equilibrium.rz2psinorm}\pysiglinewithargsret{\bfcode{rz2psinorm}}{\emph{R}, \emph{Z}, \emph{t}, \emph{return\_t=False}, \emph{sqrt=False}, \emph{make\_grid=False}, \emph{length\_unit=1}}{}
Calculates the normalized poloidal flux at the given (R, Z, t).

Uses the definition:
\begin{gather}
\begin{split}\texttt{psi\_norm} = \frac{\psi - \psi(0)}{\psi(a) - \psi(0)}\end{split}\notag
\end{gather}
If tspline is False for this Equilibrium instance, uses
scipy.interpolate.RectBivariateSpline to interpolate in terms of R and
Z. Finds the nearest time slices to those given: nearest-neighbor
interpolation in time. Otherwise, uses the tricubic package to perform
a trivariate interpolation in space and time.
\begin{quote}\begin{description}
\item[{Parameters }] \leavevmode\begin{itemize}
\item {} 
\textbf{R} --
Array-like or scalar float.
Values of the radial coordinate to
map to normalized poloidal flux. If R and Z are both scalar
values, they are used as the coordinate pair for all of the
values in t. Must have the same shape as Z unless the make\_grid
keyword is set. If the make\_grid keyword is True, R must have
shape (len\_R,).

\item {} 
\textbf{Z} --
Array-like or scalar float.
Values of the vertical coordinate to
map to normalized poloidal flux. If R and Z are both scalar
values, they are used as the coordinate pair for all of the
values in t. Must have the same shape as R unless the make\_grid
keyword is set. If the make\_grid keyword is True, Z must have
shape (len\_Z,).

\item {} 
\textbf{t} --
Array-like or single value.
If t is a single value, it is used
for all of the elements of R, Z. If t is array-like and the
make\_grid keyword is False, t must have the same dimensions as
R and Z. If t is array-like and the make\_grid keyword is True,
t must have shape (len(Z), len(R)).

\end{itemize}

\item[{Keyword Arguments}] \leavevmode\begin{itemize}
\item {} 
\textbf{return\_t} --
Boolean.
Set to True to return a tuple of (psinorm,
time\_idxs), where time\_idxs is the array of time indices
actually used in evaluating psi with nearest-neighbor
interpolation. (This is mostly present as an internal helper.)
Default is False (only return psinorm).

\item {} 
\textbf{sqrt} --
Boolean.
Set to True to return the square root of normalized
flux. Only the square root of positive psi\_norm values is taken.
Negative values are replaced with zeros, consistent with Steve
Wolfe's IDL implementation efit\_rz2rho.pro. Default is False
(return psinorm).

\item {} 
\textbf{make\_grid} --
Boolean.
Set to True to pass R and Z through meshgrid
before evaluating. If this is set to True, R and Z must each
only have a single dimension, but can have different lengths.
When using this option, it is highly recommended to only pass
a scalar value for t (such that each point in the flux grid is
evaluated at this same value t). Otherwise, t must have the
same shape as the resulting meshgrid, and each element in the
returned psi array will be at the corresponding time in the t
array. Default is False (do not form meshgrid).

\item {} 
\textbf{length\_unit} --
String or 1.
Length unit that R and Z are being given
in. If a string is given, it must be a valid unit specifier:
\begin{quote}

\begin{tabulary}{\linewidth}{|L|L|}
\hline

`m'
 & 
meters
\\\hline

`cm'
 & 
centimeters
\\\hline

`mm'
 & 
millimeters
\\\hline

`in'
 & 
inches
\\\hline

`ft'
 & 
feet
\\\hline

`yd'
 & 
yards
\\\hline

`smoot'
 & 
smoots
\\\hline

`cubit'
 & 
cubits
\\\hline

`hand'
 & 
hands
\\\hline

`default'
 & 
meters
\\\hline
\end{tabulary}

\end{quote}

If length\_unit is 1 or None, meters are assumed. The default
value is 1 (R and Z given in meters).

\end{itemize}

\item[{Returns}] \leavevmode

psinorm or (psinorm, time\_idxs)
\begin{itemize}
\item {} 
\textbf{psinorm} - Array or scalar float. If all of the input arguments are
scalar, then a scalar is returned. Otherwise, a scipy Array
instance is returned. If R and Z both have the same shape then
psinorm has this shape as well. If the make\_grid keyword was
True then psinorm has shape (len(Z), len(R)).

\item {} 
\textbf{time\_idxs} - Array with same shape as psinorm. The indices (in the
timebase returned by {\hyperref[eqtools:eqtools.core.Equilibrium.getTimeBase]{\code{getTimeBase()}}}) that were used for
nearest-neighbor interpolation. Only returned if return\_t is True.

\end{itemize}


\end{description}\end{quote}
\paragraph{Examples}

All assume that Eq\_instance is a valid instance of the appropriate
extension of the Equilibrium abstract class.

Find single psinorm value at R=0.6m, Z=0.0m, t=0.26s:

\begin{Verbatim}[commandchars=\\\{\}]
\PYG{n}{psi\PYGZus{}val} \PYG{o}{=} \PYG{n}{Eq\PYGZus{}instance}\PYG{o}{.}\PYG{n}{rz2psinorm}\PYG{p}{(}\PYG{l+m+mf}{0.6}\PYG{p}{,} \PYG{l+m+mi}{0}\PYG{p}{,} \PYG{l+m+mf}{0.26}\PYG{p}{)}
\end{Verbatim}

Find psinorm values at (R, Z) points (0.6m, 0m) and (0.8m, 0m) at the
single time t=0.26s. Note that the Z vector must be fully specified,
even if the values are all the same:

\begin{Verbatim}[commandchars=\\\{\}]
\PYG{n}{psi\PYGZus{}arr} \PYG{o}{=} \PYG{n}{Eq\PYGZus{}instance}\PYG{o}{.}\PYG{n}{rz2psinorm}\PYG{p}{(}\PYG{p}{[}\PYG{l+m+mf}{0.6}\PYG{p}{,} \PYG{l+m+mf}{0.8}\PYG{p}{]}\PYG{p}{,} \PYG{p}{[}\PYG{l+m+mi}{0}\PYG{p}{,} \PYG{l+m+mi}{0}\PYG{p}{]}\PYG{p}{,} \PYG{l+m+mf}{0.26}\PYG{p}{)}
\end{Verbatim}

Find psinorm values at (R, Z) points (0.6m, 0m) at times t={[}0.2s, 0.3s{]}:

\begin{Verbatim}[commandchars=\\\{\}]
\PYG{n}{psi\PYGZus{}arr} \PYG{o}{=} \PYG{n}{Eq\PYGZus{}instance}\PYG{o}{.}\PYG{n}{rz2psinorm}\PYG{p}{(}\PYG{l+m+mf}{0.6}\PYG{p}{,} \PYG{l+m+mi}{0}\PYG{p}{,} \PYG{p}{[}\PYG{l+m+mf}{0.2}\PYG{p}{,} \PYG{l+m+mf}{0.3}\PYG{p}{]}\PYG{p}{)}
\end{Verbatim}

Find psinorm values at (R, Z, t) points (0.6m, 0m, 0.2s) and (0.5m, 0.2m, 0.3s):

\begin{Verbatim}[commandchars=\\\{\}]
\PYG{n}{psi\PYGZus{}arr} \PYG{o}{=} \PYG{n}{Eq\PYGZus{}instance}\PYG{o}{.}\PYG{n}{rz2psinorm}\PYG{p}{(}\PYG{p}{[}\PYG{l+m+mf}{0.6}\PYG{p}{,} \PYG{l+m+mf}{0.5}\PYG{p}{]}\PYG{p}{,} \PYG{p}{[}\PYG{l+m+mi}{0}\PYG{p}{,} \PYG{l+m+mf}{0.2}\PYG{p}{]}\PYG{p}{,} \PYG{p}{[}\PYG{l+m+mf}{0.2}\PYG{p}{,} \PYG{l+m+mf}{0.3}\PYG{p}{]}\PYG{p}{)}
\end{Verbatim}

Find psinorm values on grid defined by 1D vector of radial positions R
and 1D vector of vertical positions Z at time t=0.2s:

\begin{Verbatim}[commandchars=\\\{\}]
\PYG{n}{psi\PYGZus{}mat} \PYG{o}{=} \PYG{n}{Eq\PYGZus{}instance}\PYG{o}{.}\PYG{n}{rz2psinorm}\PYG{p}{(}\PYG{n}{R}\PYG{p}{,} \PYG{n}{Z}\PYG{p}{,} \PYG{l+m+mf}{0.2}\PYG{p}{,} \PYG{n}{make\PYGZus{}grid}\PYG{o}{=}\PYG{n+nb+bp}{True}\PYG{p}{)}
\end{Verbatim}

\end{fulllineitems}

\index{rz2phinorm() (eqtools.core.Equilibrium method)}

\begin{fulllineitems}
\phantomsection\label{eqtools:eqtools.core.Equilibrium.rz2phinorm}\pysiglinewithargsret{\bfcode{rz2phinorm}}{\emph{*args}, \emph{**kwargs}}{}
Calculates the normalized toroidal flux.

Uses the definitions:
\begin{gather}
\begin{split}\texttt{phi} &= \int q(\psi)\,d\psi\end{split}\notag\\\begin{split}\texttt{phi\_norm} &= \frac{\phi}{\phi(a)}\end{split}\notag
\end{gather}
This is based on the IDL version efit\_rz2rho.pro by Steve Wolfe.

If tspline is False for this Equilibrium instance, uses
scipy.interpolate.RectBivariateSpline to interpolate in terms of R and
Z. Finds the nearest time slices to those given: nearest-neighbor
interpolation in time. Otherwise, uses the tricubic package to perform
a trivariate interpolation in space and time.
\begin{quote}\begin{description}
\item[{Parameters }] \leavevmode\begin{itemize}
\item {} 
\textbf{R} --
Array-like or scalar float.
Values of the radial coordinate to
map to normalized toroidal flux. If R and Z are both scalar
values, they are used as the coordinate pair for all of the
values in t. Must have the same shape as Z unless the make\_grid
keyword is set. If the make\_grid keyword is True, R must have
shape (len\_R,).

\item {} 
\textbf{Z} --
Array-like or scalar float.
Values of the vertical coordinate to
map to normalized toroidal flux. If R and Z are both scalar
values, they are used as the coordinate pair for all of the
values in t. Must have the same shape as R unless the make\_grid
keyword is set. If the make\_grid keyword is True, Z must have
shape (len\_Z,).

\item {} 
\textbf{t} --
Array-like or single value.
If t is a single value, it is used
for all of the elements of R, Z. If t is array-like and the
make\_grid keyword is False, t must have the same dimensions as
R and Z. If t is array-like and the make\_grid keyword is True,
t must have shape (len(Z), len(R)).

\end{itemize}

\item[{Keyword Arguments}] \leavevmode\begin{itemize}
\item {} 
\textbf{return\_t} --
Boolean.
Set to True to return a tuple of (phinorm,
time\_idxs), where time\_idxs is the array of time indices
actually used in evaluating phinorm with nearest-neighbor
interpolation. (This is mostly present as an internal helper.)
Default is False (only return phinorm).

\item {} 
\textbf{sqrt} --
Boolean.
Set to True to return the square root of normalized
flux. Only the square root of positive phi\_norm values is taken.
Negative values are replaced with zeros, consistent with Steve
Wolfe's IDL implementation efit\_rz2rho.pro. Default is False
(return phinorm).

\item {} 
\textbf{make\_grid} --
Boolean.
Set to True to pass R and Z through meshgrid
before evaluating. If this is set to True, R and Z must each
only have a single dimension, but can have different lengths.
When using this option, it is highly recommended to only pass
a scalar value for t (such that each point in the flux grid is
evaluated at this same value t). Otherwise, t must have the
same shape as the resulting meshgrid, and each element in the
returned psi array will be at the corresponding time in the t
array. Default is False (do not form meshgrid).

\item {} 
\textbf{rho} --
Boolean.
For phinorm, this should always be set to False, the
default value.

\item {} 
\textbf{kind} --
String or non-negative int.
Specifies the type of interpolation
to be performed in getting from psinorm to phinorm. This is
passed to scipy.interpolate.interp1d. Valid options are:
`linear', `nearest', `zero', `slinear', `quadratic', `cubic'
If this keyword is an integer, it specifies the order of spline
to use. See the documentation for interp1d for more details.
Default value is `cubic' (3rd order spline interpolation). On
some builds of scipy, this can cause problems, in which case
you should try `linear' until you can rebuild your scipy install.

\item {} 
\textbf{length\_unit} --
String or 1.
Length unit that R and Z are being given
in. If a string is given, it must be a valid unit specifier:
\begin{quote}

\begin{tabulary}{\linewidth}{|L|L|}
\hline

`m'
 & 
meters
\\\hline

`cm'
 & 
centimeters
\\\hline

`mm'
 & 
millimeters
\\\hline

`in'
 & 
inches
\\\hline

`ft'
 & 
feet
\\\hline

`yd'
 & 
yards
\\\hline

`smoot'
 & 
smoots
\\\hline

`cubit'
 & 
cubits
\\\hline

`hand'
 & 
hands
\\\hline

`default'
 & 
meters
\\\hline
\end{tabulary}

\end{quote}

If length\_unit is 1 or None, meters are assumed. The default
value is 1 (R and Z given in meters).

\end{itemize}

\item[{Returns}] \leavevmode

phinorm or (phinorm, time\_idxs)
\begin{itemize}
\item {} 
\textbf{phinorm} - Array or scalar float. If all of the input arguments are
scalar, then a scalar is returned. Otherwise, a scipy Array
instance is returned. If R and Z both have the same shape then
phinorm has this shape as well. If the make\_grid keyword was
True then phinorm has shape (len(Z), len(R)).

\item {} 
\textbf{time\_idxs} - Array with same shape as phinorm. The indices (in
the timebase returned by {\hyperref[eqtools:eqtools.core.Equilibrium.getTimeBase]{\code{getTimeBase()}}}) that were used
for nearest-neighbor interpolation. Only returned if return\_t is
True.

\end{itemize}


\end{description}\end{quote}
\paragraph{Examples}

All assume that Eq\_instance is a valid instance of the appropriate
extension of the Equilibrium abstract class.

Find single phinorm value at R=0.6m, Z=0.0m, t=0.26s:

\begin{Verbatim}[commandchars=\\\{\}]
\PYG{n}{phi\PYGZus{}val} \PYG{o}{=} \PYG{n}{Eq\PYGZus{}instance}\PYG{o}{.}\PYG{n}{rz2phinorm}\PYG{p}{(}\PYG{l+m+mf}{0.6}\PYG{p}{,} \PYG{l+m+mi}{0}\PYG{p}{,} \PYG{l+m+mf}{0.26}\PYG{p}{)}
\end{Verbatim}

Find phinorm values at (R, Z) points (0.6m, 0m) and (0.8m, 0m) at the
single time t=0.26s. Note that the Z vector must be fully specified,
even if the values are all the same:

\begin{Verbatim}[commandchars=\\\{\}]
\PYG{n}{phi\PYGZus{}arr} \PYG{o}{=} \PYG{n}{Eq\PYGZus{}instance}\PYG{o}{.}\PYG{n}{rz2phinorm}\PYG{p}{(}\PYG{p}{[}\PYG{l+m+mf}{0.6}\PYG{p}{,} \PYG{l+m+mf}{0.8}\PYG{p}{]}\PYG{p}{,} \PYG{p}{[}\PYG{l+m+mi}{0}\PYG{p}{,} \PYG{l+m+mi}{0}\PYG{p}{]}\PYG{p}{,} \PYG{l+m+mf}{0.26}\PYG{p}{)}
\end{Verbatim}

Find phinorm values at (R, Z) points (0.6m, 0m) at times t={[}0.2s, 0.3s{]}:

\begin{Verbatim}[commandchars=\\\{\}]
\PYG{n}{phi\PYGZus{}arr} \PYG{o}{=} \PYG{n}{Eq\PYGZus{}instance}\PYG{o}{.}\PYG{n}{rz2phinorm}\PYG{p}{(}\PYG{l+m+mf}{0.6}\PYG{p}{,} \PYG{l+m+mi}{0}\PYG{p}{,} \PYG{p}{[}\PYG{l+m+mf}{0.2}\PYG{p}{,} \PYG{l+m+mf}{0.3}\PYG{p}{]}\PYG{p}{)}
\end{Verbatim}

Find phinorm values at (R, Z, t) points (0.6m, 0m, 0.2s) and (0.5m, 0.2m, 0.3s):

\begin{Verbatim}[commandchars=\\\{\}]
\PYG{n}{phi\PYGZus{}arr} \PYG{o}{=} \PYG{n}{Eq\PYGZus{}instance}\PYG{o}{.}\PYG{n}{rz2phinorm}\PYG{p}{(}\PYG{p}{[}\PYG{l+m+mf}{0.6}\PYG{p}{,} \PYG{l+m+mf}{0.5}\PYG{p}{]}\PYG{p}{,} \PYG{p}{[}\PYG{l+m+mi}{0}\PYG{p}{,} \PYG{l+m+mf}{0.2}\PYG{p}{]}\PYG{p}{,} \PYG{p}{[}\PYG{l+m+mf}{0.2}\PYG{p}{,} \PYG{l+m+mf}{0.3}\PYG{p}{]}\PYG{p}{)}
\end{Verbatim}

Find phinorm values on grid defined by 1D vector of radial positions R
and 1D vector of vertical positions Z at time t=0.2s:

\begin{Verbatim}[commandchars=\\\{\}]
\PYG{n}{phi\PYGZus{}mat} \PYG{o}{=} \PYG{n}{Eq\PYGZus{}instance}\PYG{o}{.}\PYG{n}{rz2phinorm}\PYG{p}{(}\PYG{n}{R}\PYG{p}{,} \PYG{n}{Z}\PYG{p}{,} \PYG{l+m+mf}{0.2}\PYG{p}{,} \PYG{n}{make\PYGZus{}grid}\PYG{o}{=}\PYG{n+nb+bp}{True}\PYG{p}{)}
\end{Verbatim}

\end{fulllineitems}

\index{rz2volnorm() (eqtools.core.Equilibrium method)}

\begin{fulllineitems}
\phantomsection\label{eqtools:eqtools.core.Equilibrium.rz2volnorm}\pysiglinewithargsret{\bfcode{rz2volnorm}}{\emph{*args}, \emph{**kwargs}}{}
Calculates the normalized flux surface volume.

Based on the IDL version efit\_rz2rho.pro by Steve Wolfe.

If tspline is False for this Equilibrium instance, uses
scipy.interpolate.RectBivariateSpline to interpolate in terms of R and
Z. Finds the nearest time slices to those given: nearest-neighbor
interpolation in time. Otherwise, uses the tricubic package to perform
a trivariate interpolation in space and time.
\begin{quote}\begin{description}
\item[{Parameters }] \leavevmode\begin{itemize}
\item {} 
\textbf{R} --
Array-like or scalar float.
Values of the radial coordinate to
map to normalized volume. If R and Z are both scalar values,
they are used as the coordinate pair for all of the values in t.
Must have the same shape as Z unless the make\_grid keyword is
set. If the make\_grid keyword is True, R must have shape (len\_R,).

\item {} 
\textbf{Z} --
Array-like or scalar float.
Values of the vertical coordinate to
map to normalized volume. If R and Z are both scalar values,
they are used as the coordinate pair for all of the values in t.
Must have the same shape as R unless the make\_grid keyword is
set. If the make\_grid keyword is True, Z must have shape (len\_Z,).

\item {} 
\textbf{t} --
Array-like or single value.
If t is a single value, it is used
for all of the elements of R, Z. If t is array-like and the
make\_grid keyword is False, t must have the same dimensions as
R and Z. If t is array-like and the make\_grid keyword is True,
t must have shape (len(Z), len(R)).

\end{itemize}

\item[{Keyword Arguments}] \leavevmode\begin{itemize}
\item {} 
\textbf{return\_t} --
Boolean.
Set to True to return a tuple of (volnorm,
time\_idxs), where time\_idxs is the array of time indices
actually used in evaluating volnorm with nearest-neighbor
interpolation. (This is mostly present as an internal helper.)
Default is False (only return volnorm).

\item {} 
\textbf{sqrt} --
Boolean.
Set to True to return the square root of normalized
volume. Only the square root of positive volnorm values is
taken. Negative values are replaced with zeros, consistent with
Steve Wolfe's IDL implementation efit\_rz2rho.pro. Default is
False (return volnorm).

\item {} 
\textbf{make\_grid} --
Boolean.
Set to True to pass R and Z through meshgrid
before evaluating. If this is set to True, R and Z must each
only have a single dimension, but can have different lengths.
When using this option, it is highly recommended to only pass
a scalar value for t (such that each point in the flux grid is
evaluated at this same value t). Otherwise, t must have the
same shape as the resulting meshgrid, and each element in the
returned psi array will be at the corresponding time in the t
array. Default is False (do not form meshgrid).

\item {} 
\textbf{rho} --
Boolean.
For volnorm, this should always be set to False, the
default value.

\item {} 
\textbf{kind} --
String or non-negative int.
Specifies the type of interpolation
to be performed in getting from psinorm to volnorm. This is
passed to scipy.interpolate.interp1d. Valid options are:
`linear', `nearest', `zero', `slinear', `quadratic', `cubic'
If this keyword is an integer, it specifies the order of spline
to use. See the documentation for interp1d for more details.
Default value is `cubic' (3rd order spline interpolation). On
some builds of scipy, this can cause problems, in which case
you should try `linear' until you can rebuild your scipy install.

\item {} 
\textbf{length\_unit} --
String or 1.
Length unit that R and Z are being given
in. If a string is given, it must be a valid unit specifier:
\begin{quote}

\begin{tabulary}{\linewidth}{|L|L|}
\hline

`m'
 & 
meters
\\\hline

`cm'
 & 
centimeters
\\\hline

`mm'
 & 
millimeters
\\\hline

`in'
 & 
inches
\\\hline

`ft'
 & 
feet
\\\hline

`yd'
 & 
yards
\\\hline

`smoot'
 & 
smoots
\\\hline

`cubit'
 & 
cubits
\\\hline

`hand'
 & 
hands
\\\hline

`default'
 & 
meters
\\\hline
\end{tabulary}

\end{quote}

If length\_unit is 1 or None, meters are assumed. The default
value is 1 (R and Z given in meters).

\end{itemize}

\item[{Returns}] \leavevmode

volnorm or (volnorm, time\_idxs)
\begin{itemize}
\item {} 
\textbf{volnorm} - Array or scalar float. If all of the input arguments are
scalar, then a scalar is returned. Otherwise, a scipy Array
instance is returned. If R and Z both have the same shape then
volnorm has this shape as well. If the make\_grid keyword was
True then volnorm has shape (len(Z), len(R)).

\item {} 
\textbf{time\_idxs} - Array with same shape as volnorm. The indices (in
self.getTimeBase()) that were used for nearest-neighbor
interpolation. Only returned if return\_t is True.

\end{itemize}


\end{description}\end{quote}
\paragraph{Examples}

All assume that Eq\_instance is a valid instance of the appropriate
extension of the Equilibrium abstract class.

Find single volnorm value at R=0.6m, Z=0.0m, t=0.26s:

\begin{Verbatim}[commandchars=\\\{\}]
\PYG{n}{psi\PYGZus{}val} \PYG{o}{=} \PYG{n}{Eq\PYGZus{}instance}\PYG{o}{.}\PYG{n}{rz2volnorm}\PYG{p}{(}\PYG{l+m+mf}{0.6}\PYG{p}{,} \PYG{l+m+mi}{0}\PYG{p}{,} \PYG{l+m+mf}{0.26}\PYG{p}{)}
\end{Verbatim}

Find volnorm values at (R, Z) points (0.6m, 0m) and (0.8m, 0m) at the
single time t=0.26s. Note that the Z vector must be fully specified,
even if the values are all the same:

\begin{Verbatim}[commandchars=\\\{\}]
\PYG{n}{vol\PYGZus{}arr} \PYG{o}{=} \PYG{n}{Eq\PYGZus{}instance}\PYG{o}{.}\PYG{n}{rz2volnorm}\PYG{p}{(}\PYG{p}{[}\PYG{l+m+mf}{0.6}\PYG{p}{,} \PYG{l+m+mf}{0.8}\PYG{p}{]}\PYG{p}{,} \PYG{p}{[}\PYG{l+m+mi}{0}\PYG{p}{,} \PYG{l+m+mi}{0}\PYG{p}{]}\PYG{p}{,} \PYG{l+m+mf}{0.26}\PYG{p}{)}
\end{Verbatim}

Find volnorm values at (R, Z) points (0.6m, 0m) at times t={[}0.2s, 0.3s{]}:

\begin{Verbatim}[commandchars=\\\{\}]
\PYG{n}{vol\PYGZus{}arr} \PYG{o}{=} \PYG{n}{Eq\PYGZus{}instance}\PYG{o}{.}\PYG{n}{rz2volnorm}\PYG{p}{(}\PYG{l+m+mf}{0.6}\PYG{p}{,} \PYG{l+m+mi}{0}\PYG{p}{,} \PYG{p}{[}\PYG{l+m+mf}{0.2}\PYG{p}{,} \PYG{l+m+mf}{0.3}\PYG{p}{]}\PYG{p}{)}
\end{Verbatim}

Find volnorm values at (R, Z, t) points (0.6m, 0m, 0.2s) and (0.5m, 0.2m, 0.3s):

\begin{Verbatim}[commandchars=\\\{\}]
\PYG{n}{vol\PYGZus{}arr} \PYG{o}{=} \PYG{n}{Eq\PYGZus{}instance}\PYG{o}{.}\PYG{n}{rz2volnorm}\PYG{p}{(}\PYG{p}{[}\PYG{l+m+mf}{0.6}\PYG{p}{,} \PYG{l+m+mf}{0.5}\PYG{p}{]}\PYG{p}{,} \PYG{p}{[}\PYG{l+m+mi}{0}\PYG{p}{,} \PYG{l+m+mf}{0.2}\PYG{p}{]}\PYG{p}{,} \PYG{p}{[}\PYG{l+m+mf}{0.2}\PYG{p}{,} \PYG{l+m+mf}{0.3}\PYG{p}{]}\PYG{p}{)}
\end{Verbatim}

Find volnorm values on grid defined by 1D vector of radial positions R
and 1D vector of vertical positions Z at time t=0.2s:

\begin{Verbatim}[commandchars=\\\{\}]
\PYG{n}{vol\PYGZus{}mat} \PYG{o}{=} \PYG{n}{Eq\PYGZus{}instance}\PYG{o}{.}\PYG{n}{rz2volnorm}\PYG{p}{(}\PYG{n}{R}\PYG{p}{,} \PYG{n}{Z}\PYG{p}{,} \PYG{l+m+mf}{0.2}\PYG{p}{,} \PYG{n}{make\PYGZus{}grid}\PYG{o}{=}\PYG{n+nb+bp}{True}\PYG{p}{)}
\end{Verbatim}

\end{fulllineitems}

\index{rz2rho() (eqtools.core.Equilibrium method)}

\begin{fulllineitems}
\phantomsection\label{eqtools:eqtools.core.Equilibrium.rz2rho}\pysiglinewithargsret{\bfcode{rz2rho}}{\emph{method}, \emph{*args}, \emph{**kwargs}}{}
Convert the passed (R, Z, t) coordinates into one of several normalized coordinates.

If tspline is False for this Equilibrium instance, uses
scipy.interpolate.RectBivariateSpline to interpolate in terms of R and
Z. Finds the nearest time slices to those given: nearest-neighbor
interpolation in time. Otherwise, uses the tricubic package to perform
a trivariate interpolation in space and time.
\begin{quote}\begin{description}
\item[{Parameters }] \leavevmode\begin{itemize}
\item {} 
\textbf{method} --
String.
Indicates which normalized coordinates to use.
Valid options are:
\begin{quote}

\begin{tabulary}{\linewidth}{|L|L|}
\hline

psinorm
 & 
Normalized poloidal flux
\\\hline

phinorm
 & 
Normalized toroidal flux
\\\hline

volnorm
 & 
Normalized volume
\\\hline
\end{tabulary}

\end{quote}

\item {} 
\textbf{R} --
Array-like or scalar float.
Values of the radial coordinate to
map to normalized coordinate. If R and Z are both scalar values,
they are used as the coordinate pair for all of the values in t.
Must have the same shape as Z unless the make\_grid keyword is
set. If the make\_grid keyword is True, R must have shape (len\_R,).

\item {} 
\textbf{Z} --
Array-like or scalar float.
Values of the vertical coordinate to
map to normalized coordinate. If R and Z are both scalar values,
they are used as the coordinate pair for all of the values in t.
Must have the same shape as R unless the make\_grid keyword is
set. If the make\_grid keyword is True, Z must have shape (len\_Z,).

\item {} 
\textbf{t} --
Array-like or single value.
If t is a single value, it is used
for all of the elements of R, Z. If t is array-like and the
make\_grid keyword is False, t must have the same dimensions as
R and Z. If t is array-like and the make\_grid keyword is True,
t must have shape (len(Z), len(R)).

\end{itemize}

\item[{Keyword Arguments}] \leavevmode\begin{itemize}
\item {} 
\textbf{return\_t} --
Boolean.
Set to True to return a tuple of (volnorm,
time\_idxs), where time\_idxs is the array of time indices
actually used in evaluating volnorm with nearest-neighbor
interpolation. (This is mostly present as an internal helper.)
Default is False (only return volnorm).

\item {} 
\textbf{sqrt} --
Boolean.
Set to True to return the square root of normalized
coordinate. Only the square root of positive values is taken.
Negative values are replaced with zeros, consistent with Steve
Wolfe's IDL implementation efit\_rz2rho.pro. Default is False
(return normalized coordinate itself).

\item {} 
\textbf{make\_grid} --
Boolean.
Set to True to pass R and Z through meshgrid
before evaluating. If this is set to True, R and Z must each
only have a single dimension, but can have different lengths.
When using this option, it is highly recommended to only pass
a scalar value for t (such that each point in the flux grid is
evaluated at this same value t). Otherwise, t must have the
same shape as the resulting meshgrid, and each element in the
returned psi array will be at the corresponding time in the t
array. Default is False (do not form meshgrid).

\item {} 
\textbf{rho} (\emph{phinorm and volnorm only}) --
Boolean.
For phinorm and volnorm,
this should always be set to False, the default value.

\item {} 
\textbf{kind} (\emph{phinorm and volnorm only}) --
String or non-negative int.
Specifies the type of interpolation to be performed in getting
from psinorm to phinorm or volnorm. This is passed to
scipy.interpolate.interp1d. Valid options are:
`linear', `nearest', `zero', `slinear', `quadratic', `cubic'
If this keyword is an integer, it specifies the order of spline
to use. See the documentation for interp1d for more details.
Default value is `cubic' (3rd order spline interpolation). On
some builds of scipy, this can cause problems, in which case
you should try `linear' until you can rebuild your scipy install.

\item {} 
\textbf{length\_unit} --
String or 1.
Length unit that R and Z are being given
in. If a string is given, it must be a valid unit specifier:
\begin{quote}

\begin{tabulary}{\linewidth}{|L|L|}
\hline

`m'
 & 
meters
\\\hline

`cm'
 & 
centimeters
\\\hline

`mm'
 & 
millimeters
\\\hline

`in'
 & 
inches
\\\hline

`ft'
 & 
feet
\\\hline

`yd'
 & 
yards
\\\hline

`smoot'
 & 
smoots
\\\hline

`cubit'
 & 
cubits
\\\hline

`hand'
 & 
hands
\\\hline

`default'
 & 
meters
\\\hline
\end{tabulary}

\end{quote}

If length\_unit is 1 or None, meters are assumed. The default
value is 1 (R and Z given in meters).

\end{itemize}

\item[{Returns}] \leavevmode

rho or (rho, time\_idxs)
\begin{itemize}
\item {} 
\textbf{rho} - Array or scalar float. If all of the input arguments are
scalar, then a scalar is returned. Otherwise, a scipy Array
instance is returned. If R and Z both have the same shape then
rho has this shape as well. If the make\_grid keyword was True
then rho has shape (len(Z), len(R)).

\item {} 
\textbf{time\_idxs} - Array with same shape as rho. The indices (in
self.getTimeBase()) that were used for nearest-neighbor
interpolation. Only returned if return\_t is True.

\end{itemize}


\item[{Raises }] \leavevmode
\code{ValueError} -- 
If method is not one of the supported values.

\end{description}\end{quote}
\paragraph{Examples}

All assume that Eq\_instance is a valid instance of the appropriate
extension of the Equilibrium abstract class.

Find single psinorm value at R=0.6m, Z=0.0m, t=0.26s:

\begin{Verbatim}[commandchars=\\\{\}]
\PYG{n}{psi\PYGZus{}val} \PYG{o}{=} \PYG{n}{Eq\PYGZus{}instance}\PYG{o}{.}\PYG{n}{rz2rho}\PYG{p}{(}\PYG{l+s}{\PYGZsq{}}\PYG{l+s}{psinorm}\PYG{l+s}{\PYGZsq{}}\PYG{p}{,} \PYG{l+m+mf}{0.6}\PYG{p}{,} \PYG{l+m+mi}{0}\PYG{p}{,} \PYG{l+m+mf}{0.26}\PYG{p}{)}
\end{Verbatim}

Find psinorm values at (R, Z) points (0.6m, 0m) and (0.8m, 0m) at the
single time t=0.26s. Note that the Z vector must be fully specified,
even if the values are all the same:

\begin{Verbatim}[commandchars=\\\{\}]
\PYG{n}{psi\PYGZus{}arr} \PYG{o}{=} \PYG{n}{Eq\PYGZus{}instance}\PYG{o}{.}\PYG{n}{rz2rho}\PYG{p}{(}\PYG{l+s}{\PYGZsq{}}\PYG{l+s}{psinorm}\PYG{l+s}{\PYGZsq{}}\PYG{p}{,} \PYG{p}{[}\PYG{l+m+mf}{0.6}\PYG{p}{,} \PYG{l+m+mf}{0.8}\PYG{p}{]}\PYG{p}{,} \PYG{p}{[}\PYG{l+m+mi}{0}\PYG{p}{,} \PYG{l+m+mi}{0}\PYG{p}{]}\PYG{p}{,} \PYG{l+m+mf}{0.26}\PYG{p}{)}
\end{Verbatim}

Find psinorm values at (R, Z) points (0.6m, 0m) at times t={[}0.2s, 0.3s{]}:

\begin{Verbatim}[commandchars=\\\{\}]
\PYG{n}{psi\PYGZus{}arr} \PYG{o}{=} \PYG{n}{Eq\PYGZus{}instance}\PYG{o}{.}\PYG{n}{rz2rho}\PYG{p}{(}\PYG{l+s}{\PYGZsq{}}\PYG{l+s}{psinorm}\PYG{l+s}{\PYGZsq{}}\PYG{p}{,} \PYG{l+m+mf}{0.6}\PYG{p}{,} \PYG{l+m+mi}{0}\PYG{p}{,} \PYG{p}{[}\PYG{l+m+mf}{0.2}\PYG{p}{,} \PYG{l+m+mf}{0.3}\PYG{p}{]}\PYG{p}{)}
\end{Verbatim}

Find psinorm values at (R, Z, t) points (0.6m, 0m, 0.2s) and (0.5m, 0.2m, 0.3s):

\begin{Verbatim}[commandchars=\\\{\}]
\PYG{n}{psi\PYGZus{}arr} \PYG{o}{=} \PYG{n}{Eq\PYGZus{}instance}\PYG{o}{.}\PYG{n}{rz2rho}\PYG{p}{(}\PYG{l+s}{\PYGZsq{}}\PYG{l+s}{psinorm}\PYG{l+s}{\PYGZsq{}}\PYG{p}{,} \PYG{p}{[}\PYG{l+m+mf}{0.6}\PYG{p}{,} \PYG{l+m+mf}{0.5}\PYG{p}{]}\PYG{p}{,} \PYG{p}{[}\PYG{l+m+mi}{0}\PYG{p}{,} \PYG{l+m+mf}{0.2}\PYG{p}{]}\PYG{p}{,} \PYG{p}{[}\PYG{l+m+mf}{0.2}\PYG{p}{,} \PYG{l+m+mf}{0.3}\PYG{p}{]}\PYG{p}{)}
\end{Verbatim}

Find psinorm values on grid defined by 1D vector of radial positions R
and 1D vector of vertical positions Z at time t=0.2s:

\begin{Verbatim}[commandchars=\\\{\}]
\PYG{n}{psi\PYGZus{}mat} \PYG{o}{=} \PYG{n}{Eq\PYGZus{}instance}\PYG{o}{.}\PYG{n}{rz2rho}\PYG{p}{(}\PYG{l+s}{\PYGZsq{}}\PYG{l+s}{psinorm}\PYG{l+s}{\PYGZsq{}}\PYG{p}{,} \PYG{n}{R}\PYG{p}{,} \PYG{n}{Z}\PYG{p}{,} \PYG{l+m+mf}{0.2}\PYG{p}{,} \PYG{n}{make\PYGZus{}grid}\PYG{o}{=}\PYG{n+nb+bp}{True}\PYG{p}{)}
\end{Verbatim}

\end{fulllineitems}

\index{rz2rmid() (eqtools.core.Equilibrium method)}

\begin{fulllineitems}
\phantomsection\label{eqtools:eqtools.core.Equilibrium.rz2rmid}\pysiglinewithargsret{\bfcode{rz2rmid}}{\emph{*args}, \emph{**kwargs}}{}
Maps the given points to the outboard midplane major radius, R\_mid.

Based on the IDL version efit\_rz2rmid.pro by Steve Wolfe.

If tspline is False for this Equilibrium instance, uses
scipy.interpolate.RectBivariateSpline to interpolate in terms of R and
Z. Finds the nearest time slices to those given: nearest-neighbor
interpolation in time. Otherwise, uses the tricubic package to perform
a trivariate interpolation in space and time.
\begin{quote}\begin{description}
\item[{Parameters }] \leavevmode\begin{itemize}
\item {} 
\textbf{R} --
Array-like or scalar float.
Values of the radial coordinate to
map to midplane radius. If R and Z are both scalar values,
they are used as the coordinate pair for all of the values in t.
Must have the same shape as Z unless the make\_grid keyword is
set. If the make\_grid keyword is True, R must have shape (len\_R,).

\item {} 
\textbf{Z} --
Array-like or scalar float.
Values of the vertical coordinate to
map to midplane radius. If R and Z are both scalar values,
they are used as the coordinate pair for all of the values in t.
Must have the same shape as R unless the make\_grid keyword is
set. If the make\_grid keyword is True, Z must have shape (len\_Z,).

\item {} 
\textbf{t} --
Array-like or single value.
If t is a single value, it is used
for all of the elements of R, Z. If t is array-like and the
make\_grid keyword is False, t must have the same dimensions as
R and Z. If t is array-like and the make\_grid keyword is True,
t must have shape (len(Z), len(R)).

\end{itemize}

\item[{Keyword Arguments}] \leavevmode\begin{itemize}
\item {} 
\textbf{return\_t} --
Boolean.
Set to True to return a tuple of (R\_mid,
time\_idxs), where time\_idxs is the array of time indices
actually used in evaluating R\_mid with nearest-neighbor
interpolation. (This is mostly present as an internal helper.)
Default is False (only return R\_mid).

\item {} 
\textbf{sqrt} --
Boolean.
Set to True to return the square root of midplane
radius. Only the square root of positive values is taken.
Negative values are replaced with zeros, consistent with Steve
Wolfe's IDL implementation efit\_rz2rho.pro. Default is False
(return R\_mid itself).

\item {} 
\textbf{make\_grid} --
Boolean.
Set to True to pass R and Z through meshgrid
before evaluating. If this is set to True, R and Z must each
only have a single dimension, but can have different lengths.
When using this option, it is highly recommended to only pass
a scalar value for t (such that each point in the flux grid is
evaluated at this same value t). Otherwise, t must have the
same shape as the resulting meshgrid, and each element in the
returned psi array will be at the corresponding time in the t
array. Default is False (do not form meshgrid).

\item {} 
\textbf{rho} --
Boolean.
Set to True to return r/a (normalized minor radius)
instead of R\_mid. Default is False (return major radius, R\_mid).

\item {} 
\textbf{kind} --
String or non-negative int.
Specifies the type of interpolation
to be performed in getting from psinorm to R\_mid. This is
passed to scipy.interpolate.interp1d. Valid options are:
`linear', `nearest', `zero', `slinear', `quadratic', `cubic'
If this keyword is an integer, it specifies the order of spline
to use. See the documentation for interp1d for more details.
Default value is `cubic' (3rd order spline interpolation). On
some builds of scipy, this can cause problems, in which case
you should try `linear' until you can rebuild your scipy install.

\item {} 
\textbf{length\_unit} --
String or 1.
Length unit that R and Z are being given
in AND that R\_mid is returned in. If a string is given, it
must be a valid unit specifier:
\begin{quote}

\begin{tabulary}{\linewidth}{|L|L|}
\hline

`m'
 & 
meters
\\\hline

`cm'
 & 
centimeters
\\\hline

`mm'
 & 
millimeters
\\\hline

`in'
 & 
inches
\\\hline

`ft'
 & 
feet
\\\hline

`yd'
 & 
yards
\\\hline

`smoot'
 & 
smoots
\\\hline

`cubit'
 & 
cubits
\\\hline

`hand'
 & 
hands
\\\hline

`default'
 & 
meters
\\\hline
\end{tabulary}

\end{quote}

If length\_unit is 1 or None, meters are assumed. The default
value is 1 (R and Z given in meters, R\_mid returned in meters).

\end{itemize}

\item[{Returns}] \leavevmode

R\_mid or (R\_mid, time\_idxs)
\begin{itemize}
\item {} 
\textbf{R\_mid} - Array or scalar float. If all of the input arguments are
scalar, then a scalar is returned. Otherwise, a scipy Array
instance is returned. If R and Z both have the same shape then
R\_mid has this shape as well. If the make\_grid keyword was True
then R\_mid has shape (len(Z), len(R)).

\item {} 
\textbf{time\_idxs} - Array with same shape as R\_mid. The indices (in
the timebase returned by {\hyperref[eqtools:eqtools.core.Equilibrium.getTimeBase]{\code{getTimeBase()}}}) that were used
for nearest-neighbor interpolation. Only returned if return\_t is
True.

\end{itemize}


\end{description}\end{quote}
\paragraph{Examples}

All assume that Eq\_instance is a valid instance of the appropriate
extension of the Equilibrium abstract class.

Find single R\_mid value at R=0.6m, Z=0.0m, t=0.26s:

\begin{Verbatim}[commandchars=\\\{\}]
\PYG{n}{R\PYGZus{}mid\PYGZus{}val} \PYG{o}{=} \PYG{n}{Eq\PYGZus{}instance}\PYG{o}{.}\PYG{n}{rz2rmid}\PYG{p}{(}\PYG{l+m+mf}{0.6}\PYG{p}{,} \PYG{l+m+mi}{0}\PYG{p}{,} \PYG{l+m+mf}{0.26}\PYG{p}{)}
\end{Verbatim}

Find R\_mid values at (R, Z) points (0.6m, 0m) and (0.8m, 0m) at the
single time t=0.26s. Note that the Z vector must be fully specified,
even if the values are all the same:

\begin{Verbatim}[commandchars=\\\{\}]
\PYG{n}{R\PYGZus{}mid\PYGZus{}arr} \PYG{o}{=} \PYG{n}{Eq\PYGZus{}instance}\PYG{o}{.}\PYG{n}{rz2rmid}\PYG{p}{(}\PYG{p}{[}\PYG{l+m+mf}{0.6}\PYG{p}{,} \PYG{l+m+mf}{0.8}\PYG{p}{]}\PYG{p}{,} \PYG{p}{[}\PYG{l+m+mi}{0}\PYG{p}{,} \PYG{l+m+mi}{0}\PYG{p}{]}\PYG{p}{,} \PYG{l+m+mf}{0.26}\PYG{p}{)}
\end{Verbatim}

Find R\_mid values at (R, Z) points (0.6m, 0m) at times t={[}0.2s, 0.3s{]}:

\begin{Verbatim}[commandchars=\\\{\}]
\PYG{n}{R\PYGZus{}mid\PYGZus{}arr} \PYG{o}{=} \PYG{n}{Eq\PYGZus{}instance}\PYG{o}{.}\PYG{n}{rz2rmid}\PYG{p}{(}\PYG{l+m+mf}{0.6}\PYG{p}{,} \PYG{l+m+mi}{0}\PYG{p}{,} \PYG{p}{[}\PYG{l+m+mf}{0.2}\PYG{p}{,} \PYG{l+m+mf}{0.3}\PYG{p}{]}\PYG{p}{)}
\end{Verbatim}

Find R\_mid values at (R, Z, t) points (0.6m, 0m, 0.2s) and (0.5m, 0.2m, 0.3s):

\begin{Verbatim}[commandchars=\\\{\}]
\PYG{n}{R\PYGZus{}mid\PYGZus{}arr} \PYG{o}{=} \PYG{n}{Eq\PYGZus{}instance}\PYG{o}{.}\PYG{n}{rz2rmid}\PYG{p}{(}\PYG{p}{[}\PYG{l+m+mf}{0.6}\PYG{p}{,} \PYG{l+m+mf}{0.5}\PYG{p}{]}\PYG{p}{,} \PYG{p}{[}\PYG{l+m+mi}{0}\PYG{p}{,} \PYG{l+m+mf}{0.2}\PYG{p}{]}\PYG{p}{,} \PYG{p}{[}\PYG{l+m+mf}{0.2}\PYG{p}{,} \PYG{l+m+mf}{0.3}\PYG{p}{]}\PYG{p}{)}
\end{Verbatim}

Find R\_mid values on grid defined by 1D vector of radial positions R
and 1D vector of vertical positions Z at time t=0.2s:

\begin{Verbatim}[commandchars=\\\{\}]
\PYG{n}{R\PYGZus{}mid\PYGZus{}mat} \PYG{o}{=} \PYG{n}{Eq\PYGZus{}instance}\PYG{o}{.}\PYG{n}{rz2rmid}\PYG{p}{(}\PYG{n}{R}\PYG{p}{,} \PYG{n}{Z}\PYG{p}{,} \PYG{l+m+mf}{0.2}\PYG{p}{,} \PYG{n}{make\PYGZus{}grid}\PYG{o}{=}\PYG{n+nb+bp}{True}\PYG{p}{)}
\end{Verbatim}

\end{fulllineitems}

\index{psinorm2rmid() (eqtools.core.Equilibrium method)}

\begin{fulllineitems}
\phantomsection\label{eqtools:eqtools.core.Equilibrium.psinorm2rmid}\pysiglinewithargsret{\bfcode{psinorm2rmid}}{\emph{psi\_norm}, \emph{t}, \emph{return\_t=False}, \emph{rho=False}, \emph{kind='cubic'}, \emph{length\_unit=1}}{}
Calculates the outboard R\_mid location corresponding to the passed psi\_norm (normalized poloidal flux) values.

If tspline is False for this Equilibrium instance, uses
scipy.interpolate.RectBivariateSpline to interpolate in terms of R and
Z. Finds the nearest time slices to those given: nearest-neighbor
interpolation in time. Otherwise, uses the tricubic package to perform
a trivariate interpolation in space and time.
\begin{quote}\begin{description}
\item[{Parameters }] \leavevmode\begin{itemize}
\item {} 
\textbf{psi\_norm} --
Array-like or scalar float.
Values of the normalized
poloidal flux to map to midplane radius. If psi\_norm is a scalar,
it is used as the value for all of the values in t.

\item {} 
\textbf{t} --
Array-like or single value.
If t is a single value, it is used
for all of the elements of psi\_norm. If neither t nor psi\_norm
are scalars, t must have the same shape as psi\_norm.

\end{itemize}

\item[{Keyword Arguments}] \leavevmode\begin{itemize}
\item {} 
\textbf{return\_t} --
Boolean.
Set to True to return a tuple of (R\_mid,
time\_idxs), where time\_idxs is the array of time indices
actually used in evaluating R\_mid with nearest-neighbor
interpolation. (This is mostly present as an internal helper.)
Default is False (only return R\_mid).

\item {} 
\textbf{rho} --
Boolean.
Set to True to return r/a (normalized minor radius)
instead of R\_mid. Default is False (return major radius, R\_mid).

\item {} 
\textbf{kind} --
String or non-negative int.
Specifies the type of interpolation
to be performed in getting from psinorm to R\_mid. This is
passed to scipy.interpolate.interp1d. Valid options are:
`linear', `nearest', `zero', `slinear', `quadratic', `cubic'
If this keyword is an integer, it specifies the order of spline
to use. See the documentation for interp1d for more details.
Default value is `cubic' (3rd order spline interpolation). On
some builds of scipy, this can cause problems, in which case
you should try `linear' until you can rebuild your scipy install.

\item {} 
\textbf{length\_unit} --
String or 1.
Length unit that R\_mid is returned in. If
a string is given, it must be a valid unit specifier:
\begin{quote}

\begin{tabulary}{\linewidth}{|L|L|}
\hline

`m'
 & 
meters
\\\hline

`cm'
 & 
centimeters
\\\hline

`mm'
 & 
millimeters
\\\hline

`in'
 & 
inches
\\\hline

`ft'
 & 
feet
\\\hline

`yd'
 & 
yards
\\\hline

`smoot'
 & 
smoots
\\\hline

`cubit'
 & 
cubits
\\\hline

`hand'
 & 
hands
\\\hline

`default'
 & 
meters
\\\hline
\end{tabulary}

\end{quote}

If length\_unit is 1 or None, meters are assumed. The default
value is 1 (R\_mid returned in meters).

\end{itemize}

\item[{Returns}] \leavevmode

R\_mid or (R\_mid, time\_idxs)
\begin{itemize}
\item {} 
\textbf{R\_mid} - Array or scalar float. If all of the input arguments are
scalar, then a scalar is returned. Otherwise, a scipy Array
instance is returned. R\_mid will have the same shape as t and
psi\_norm (or whichever one is Array-like).

\item {} 
\textbf{time\_idxs} - Array with same shape as R\_mid. The indices (in
the timebase returned by {\hyperref[eqtools:eqtools.core.Equilibrium.getTimeBase]{\code{getTimeBase()}}}) that were used
for nearest-neighbor interpolation. Only returned if return\_t is
True.

\end{itemize}


\end{description}\end{quote}
\paragraph{Examples}

All assume that Eq\_instance is a valid instance of the appropriate
extension of the Equilibrium abstract class.

Find single R\_mid value for psinorm=0.7, t=0.26s:

\begin{Verbatim}[commandchars=\\\{\}]
\PYG{n}{R\PYGZus{}mid\PYGZus{}val} \PYG{o}{=} \PYG{n}{Eq\PYGZus{}instance}\PYG{o}{.}\PYG{n}{psinorm2rmid}\PYG{p}{(}\PYG{l+m+mf}{0.7}\PYG{p}{,} \PYG{l+m+mf}{0.26}\PYG{p}{)}
\end{Verbatim}

Find R\_mid values at psi\_norm values of 0.5 and 0.7 at the single time
t=0.26s. Note that the Z vector must be fully specified, even if the
values are all the same:

\begin{Verbatim}[commandchars=\\\{\}]
\PYG{n}{R\PYGZus{}mid\PYGZus{}arr} \PYG{o}{=} \PYG{n}{Eq\PYGZus{}instance}\PYG{o}{.}\PYG{n}{psinorm2rmid}\PYG{p}{(}\PYG{p}{[}\PYG{l+m+mf}{0.5}\PYG{p}{,} \PYG{l+m+mf}{0.7}\PYG{p}{]}\PYG{p}{,} \PYG{l+m+mf}{0.26}\PYG{p}{)}
\end{Verbatim}

Find R\_mid values at psi\_norm=0.5 at times t={[}0.2s, 0.3s{]}:

\begin{Verbatim}[commandchars=\\\{\}]
\PYG{n}{R\PYGZus{}mid\PYGZus{}arr} \PYG{o}{=} \PYG{n}{Eq\PYGZus{}instance}\PYG{o}{.}\PYG{n}{psinorm2rmid}\PYG{p}{(}\PYG{l+m+mf}{0.5}\PYG{p}{,} \PYG{p}{[}\PYG{l+m+mf}{0.2}\PYG{p}{,} \PYG{l+m+mf}{0.3}\PYG{p}{]}\PYG{p}{)}
\end{Verbatim}

Find R\_mid values at (psinorm, t) points (0.6, 0.2s) and (0.5, 0.3s):

\begin{Verbatim}[commandchars=\\\{\}]
\PYG{n}{R\PYGZus{}mid\PYGZus{}arr} \PYG{o}{=} \PYG{n}{Eq\PYGZus{}instance}\PYG{o}{.}\PYG{n}{psinorm2rmid}\PYG{p}{(}\PYG{p}{[}\PYG{l+m+mf}{0.6}\PYG{p}{,} \PYG{l+m+mf}{0.5}\PYG{p}{]}\PYG{p}{,} \PYG{p}{[}\PYG{l+m+mf}{0.2}\PYG{p}{,} \PYG{l+m+mf}{0.3}\PYG{p}{]}\PYG{p}{)}
\end{Verbatim}

\end{fulllineitems}

\index{psinorm2volnorm() (eqtools.core.Equilibrium method)}

\begin{fulllineitems}
\phantomsection\label{eqtools:eqtools.core.Equilibrium.psinorm2volnorm}\pysiglinewithargsret{\bfcode{psinorm2volnorm}}{\emph{psi\_norm}, \emph{t}, \emph{return\_t=False}, \emph{kind='cubic'}}{}
Calculates the normalized volume corresponding to the passed psi\_norm (normalized poloidal flux) values.

If tspline is False for this Equilibrium instance, uses
scipy.interpolate.RectBivariateSpline to interpolate in terms of R and
Z. Finds the nearest time slices to those given: nearest-neighbor
interpolation in time. Otherwise, uses the tricubic package to perform
a trivariate interpolation in space and time.
\begin{quote}\begin{description}
\item[{Parameters }] \leavevmode\begin{itemize}
\item {} 
\textbf{psi\_norm} --
Array-like or scalar float.
Values of the normalized
poloidal flux to map to normalized volume. If psi\_norm is a
scalar, it is used as the value for all of the values in t.

\item {} 
\textbf{t} --
Array-like or single value.
If t is a single value, it is used
for all of the elements of psi\_norm. If neither t nor psi\_norm
are scalars, t must have the same shape as psi\_norm.

\end{itemize}

\item[{Keyword Arguments}] \leavevmode\begin{itemize}
\item {} 
\textbf{return\_t} --
Boolean.
Set to True to return a tuple of (volnorm,
time\_idxs), where time\_idxs is the array of time indices
actually used in evaluating volnorm with nearest-neighbor
interpolation. (This is mostly present as an internal helper.)
Default is False (only return volnorm).

\item {} 
\textbf{kind} --
String or non-negative int.
Specifies the type of interpolation
to be performed in getting from psinorm to volnorm. This is
passed to scipy.interpolate.interp1d. Valid options are:
`linear', `nearest', `zero', `slinear', `quadratic', `cubic'
If this keyword is an integer, it specifies the order of spline
to use. See the documentation for interp1d for more details.
Default value is `cubic' (3rd order spline interpolation). On
some builds of scipy, this can cause problems, in which case
you should try `linear' until you can rebuild your scipy install.

\end{itemize}

\item[{Returns}] \leavevmode

volnorm or (volnorm, time\_idxs)
\begin{itemize}
\item {} 
\textbf{volnorm} - Array or scalar float. If all of the input arguments are
scalar, then a scalar is returned. Otherwise, a scipy Array
instance is returned. volnorm will have the same shape as t and
psi\_norm (or whichever one is Array-like).

\item {} 
\textbf{time\_idxs} - Array with same shape as volnorm. The indices (in
the timebase returned by {\hyperref[eqtools:eqtools.core.Equilibrium.getTimeBase]{\code{getTimeBase()}}}) that were used
for nearest-neighbor interpolation. Only returned if return\_t is
True.

\end{itemize}


\end{description}\end{quote}
\paragraph{Examples}

All assume that Eq\_instance is a valid instance of the appropriate
extension of the Equilibrium abstract class.

Find single volnorm value for psinorm=0.7, t=0.26s:

\begin{Verbatim}[commandchars=\\\{\}]
\PYG{n}{volnorm\PYGZus{}val} \PYG{o}{=} \PYG{n}{Eq\PYGZus{}instance}\PYG{o}{.}\PYG{n}{psinorm2volnorm}\PYG{p}{(}\PYG{l+m+mf}{0.7}\PYG{p}{,} \PYG{l+m+mf}{0.26}\PYG{p}{)}
\end{Verbatim}

Find volnorm values at psi\_norm values of 0.5 and 0.7 at the single time
t=0.26s. Note that the Z vector must be fully specified, even if the
values are all the same:

\begin{Verbatim}[commandchars=\\\{\}]
\PYG{n}{volnorm\PYGZus{}arr} \PYG{o}{=} \PYG{n}{Eq\PYGZus{}instance}\PYG{o}{.}\PYG{n}{psinorm2volnorm}\PYG{p}{(}\PYG{p}{[}\PYG{l+m+mf}{0.5}\PYG{p}{,} \PYG{l+m+mf}{0.7}\PYG{p}{]}\PYG{p}{,} \PYG{l+m+mf}{0.26}\PYG{p}{)}
\end{Verbatim}

Find volnorm values at psi\_norm=0.5 at times t={[}0.2s, 0.3s{]}:

\begin{Verbatim}[commandchars=\\\{\}]
\PYG{n}{volnorm\PYGZus{}arr} \PYG{o}{=} \PYG{n}{Eq\PYGZus{}instance}\PYG{o}{.}\PYG{n}{psinorm2volnorm}\PYG{p}{(}\PYG{l+m+mf}{0.5}\PYG{p}{,} \PYG{p}{[}\PYG{l+m+mf}{0.2}\PYG{p}{,} \PYG{l+m+mf}{0.3}\PYG{p}{]}\PYG{p}{)}
\end{Verbatim}

Find volnorm values at (psinorm, t) points (0.6, 0.2s) and (0.5, 0.3s):

\begin{Verbatim}[commandchars=\\\{\}]
\PYG{n}{volnorm\PYGZus{}arr} \PYG{o}{=} \PYG{n}{Eq\PYGZus{}instance}\PYG{o}{.}\PYG{n}{psinorm2volnorm}\PYG{p}{(}\PYG{p}{[}\PYG{l+m+mf}{0.6}\PYG{p}{,} \PYG{l+m+mf}{0.5}\PYG{p}{]}\PYG{p}{,} \PYG{p}{[}\PYG{l+m+mf}{0.2}\PYG{p}{,} \PYG{l+m+mf}{0.3}\PYG{p}{]}\PYG{p}{)}
\end{Verbatim}

\end{fulllineitems}

\index{psinorm2phinorm() (eqtools.core.Equilibrium method)}

\begin{fulllineitems}
\phantomsection\label{eqtools:eqtools.core.Equilibrium.psinorm2phinorm}\pysiglinewithargsret{\bfcode{psinorm2phinorm}}{\emph{psi\_norm}, \emph{t}, \emph{return\_t=False}, \emph{kind='cubic'}}{}
Calculates the normalized toroidal flux corresponding to the passed psi\_norm (normalized poloidal flux) values.

If tspline is False for this Equilibrium instance, uses
scipy.interpolate.RectBivariateSpline to interpolate in terms of R and
Z. Finds the nearest time slices to those given: nearest-neighbor
interpolation in time. Otherwise, uses the tricubic package to perform
a trivariate interpolation in space and time.
\begin{quote}\begin{description}
\item[{Parameters }] \leavevmode\begin{itemize}
\item {} 
\textbf{psi\_norm} --
Array-like or scalar float.
Values of the normalized
poloidal flux to map to normalized toroidal flux. If psi\_norm
is a scalar, it is used as the value for all of the values in t.

\item {} 
\textbf{t} --
Array-like or single value.
If t is a single value, it is used
for all of the elements of psi\_norm. If neither t nor psi\_norm
are scalars, t must have the same shape as psi\_norm.

\end{itemize}

\item[{Keyword Arguments}] \leavevmode\begin{itemize}
\item {} 
\textbf{return\_t} --
Boolean.
Set to True to return a tuple of (phinorm,
time\_idxs), where time\_idxs is the array of time indices
actually used in evaluating phinorm with nearest-neighbor
interpolation. (This is mostly present as an internal helper.)
Default is False (only return phinorm).

\item {} 
\textbf{kind} --
String or non-negative int.
Specifies the type of interpolation
to be performed in getting from psinorm to phinorm. This is
passed to scipy.interpolate.interp1d. Valid options are:
`linear', `nearest', `zero', `slinear', `quadratic', `cubic'
If this keyword is an integer, it specifies the order of spline
to use. See the documentation for interp1d for more details.
Default value is `cubic' (3rd order spline interpolation). On
some builds of scipy, this can cause problems, in which case
you should try `linear' until you can rebuild your scipy install.

\end{itemize}

\item[{Returns}] \leavevmode

phinorm or (phinorm, time\_idxs)
\begin{itemize}
\item {} 
\textbf{phinorm} - Array or scalar float. If all of the input arguments are
scalar, then a scalar is returned. Otherwise, a scipy Array
instance is returned. phinorm will have the same shape as t and
psi\_norm (or whichever one is Array-like).

\item {} 
\textbf{time\_idxs} - Array with same shape as phinorm. The indices (in
the timebase returned by {\hyperref[eqtools:eqtools.core.Equilibrium.getTimeBase]{\code{getTimeBase()}}}) that were used
for nearest-neighbor interpolation. Only returned if return\_t is
True.

\end{itemize}


\end{description}\end{quote}
\paragraph{Examples}

All assume that Eq\_instance is a valid instance of the appropriate
extension of the Equilibrium abstract class.

Find single phinorm value for psinorm=0.7, t=0.26s:

\begin{Verbatim}[commandchars=\\\{\}]
\PYG{n}{phinorm\PYGZus{}val} \PYG{o}{=} \PYG{n}{Eq\PYGZus{}instance}\PYG{o}{.}\PYG{n}{psinorm2phinorm}\PYG{p}{(}\PYG{l+m+mf}{0.7}\PYG{p}{,} \PYG{l+m+mf}{0.26}\PYG{p}{)}
\end{Verbatim}

Find phinorm values at psi\_norm values of 0.5 and 0.7 at the single time
t=0.26s. Note that the Z vector must be fully specified, even if the
values are all the same:

\begin{Verbatim}[commandchars=\\\{\}]
\PYG{n}{phinorm\PYGZus{}arr} \PYG{o}{=} \PYG{n}{Eq\PYGZus{}instance}\PYG{o}{.}\PYG{n}{psinorm2phinorm}\PYG{p}{(}\PYG{p}{[}\PYG{l+m+mf}{0.5}\PYG{p}{,} \PYG{l+m+mf}{0.7}\PYG{p}{]}\PYG{p}{,} \PYG{l+m+mf}{0.26}\PYG{p}{)}
\end{Verbatim}

Find phinorm values at psi\_norm=0.5 at times t={[}0.2s, 0.3s{]}:

\begin{Verbatim}[commandchars=\\\{\}]
\PYG{n}{phinorm\PYGZus{}arr} \PYG{o}{=} \PYG{n}{Eq\PYGZus{}instance}\PYG{o}{.}\PYG{n}{psinorm2phinorm}\PYG{p}{(}\PYG{l+m+mf}{0.5}\PYG{p}{,} \PYG{p}{[}\PYG{l+m+mf}{0.2}\PYG{p}{,} \PYG{l+m+mf}{0.3}\PYG{p}{]}\PYG{p}{)}
\end{Verbatim}

Find phinorm values at (psinorm, t) points (0.6, 0.2s) and (0.5, 0.3s):

\begin{Verbatim}[commandchars=\\\{\}]
\PYG{n}{phinorm\PYGZus{}arr} \PYG{o}{=} \PYG{n}{Eq\PYGZus{}instance}\PYG{o}{.}\PYG{n}{psinorm2phinorm}\PYG{p}{(}\PYG{p}{[}\PYG{l+m+mf}{0.6}\PYG{p}{,} \PYG{l+m+mf}{0.5}\PYG{p}{]}\PYG{p}{,} \PYG{p}{[}\PYG{l+m+mf}{0.2}\PYG{p}{,} \PYG{l+m+mf}{0.3}\PYG{p}{]}\PYG{p}{)}
\end{Verbatim}

\end{fulllineitems}

\index{getInfo() (eqtools.core.Equilibrium method)}

\begin{fulllineitems}
\phantomsection\label{eqtools:eqtools.core.Equilibrium.getInfo}\pysiglinewithargsret{\bfcode{getInfo}}{}{}
Abstract method.  See child classes for implementation.

Returns namedtuple of instance parameters (shot, equilibrium type, size, timebase, etc.)

\end{fulllineitems}

\index{getTimeBase() (eqtools.core.Equilibrium method)}

\begin{fulllineitems}
\phantomsection\label{eqtools:eqtools.core.Equilibrium.getTimeBase}\pysiglinewithargsret{\bfcode{getTimeBase}}{}{}
Abstract method.  See child classes for implementation.

Returns timebase array {[}t{]}

\end{fulllineitems}

\index{getFluxGrid() (eqtools.core.Equilibrium method)}

\begin{fulllineitems}
\phantomsection\label{eqtools:eqtools.core.Equilibrium.getFluxGrid}\pysiglinewithargsret{\bfcode{getFluxGrid}}{}{}
Abstract method.  See child classes for implementation.
\begin{description}
\item[{returns 3D grid of psi(r,z,t)}] \leavevmode\begin{description}
\item[{The array returned should have the following dimensions:}] \leavevmode
First dimension: time
Second dimension: Z
Third dimension: R

\end{description}

\end{description}

\end{fulllineitems}

\index{getRGrid() (eqtools.core.Equilibrium method)}

\begin{fulllineitems}
\phantomsection\label{eqtools:eqtools.core.Equilibrium.getRGrid}\pysiglinewithargsret{\bfcode{getRGrid}}{}{}
Abstract method.  See child classes for implementation.

Returns vector of R-values for psiRZ grid {[}r{]}

\end{fulllineitems}

\index{getZGrid() (eqtools.core.Equilibrium method)}

\begin{fulllineitems}
\phantomsection\label{eqtools:eqtools.core.Equilibrium.getZGrid}\pysiglinewithargsret{\bfcode{getZGrid}}{}{}
Abstract method.  See child classes for implementation.

Returns vector of Z-values for psiRZ grid {[}z{]}

\end{fulllineitems}

\index{getFluxAxis() (eqtools.core.Equilibrium method)}

\begin{fulllineitems}
\phantomsection\label{eqtools:eqtools.core.Equilibrium.getFluxAxis}\pysiglinewithargsret{\bfcode{getFluxAxis}}{}{}
Abstract method.  See child classes for implementation.

Returns psi at magnetic axis {[}t{]}

\end{fulllineitems}

\index{getFluxLCFS() (eqtools.core.Equilibrium method)}

\begin{fulllineitems}
\phantomsection\label{eqtools:eqtools.core.Equilibrium.getFluxLCFS}\pysiglinewithargsret{\bfcode{getFluxLCFS}}{}{}
Abstract method.  See child classes for implementation.

Returns psi a separatrix {[}t{]}

\end{fulllineitems}

\index{getRLCFS() (eqtools.core.Equilibrium method)}

\begin{fulllineitems}
\phantomsection\label{eqtools:eqtools.core.Equilibrium.getRLCFS}\pysiglinewithargsret{\bfcode{getRLCFS}}{}{}
Abstract method.  See child classes for implementation.

Returns R-positions (n points) mapping LCFS {[}t,n{]}

\end{fulllineitems}

\index{getZLCFS() (eqtools.core.Equilibrium method)}

\begin{fulllineitems}
\phantomsection\label{eqtools:eqtools.core.Equilibrium.getZLCFS}\pysiglinewithargsret{\bfcode{getZLCFS}}{}{}
Abstract method.  See child classes for implementation.

Returns Z-positions (n points) mapping LCFS {[}t,n{]}

\end{fulllineitems}

\index{getFluxVol() (eqtools.core.Equilibrium method)}

\begin{fulllineitems}
\phantomsection\label{eqtools:eqtools.core.Equilibrium.getFluxVol}\pysiglinewithargsret{\bfcode{getFluxVol}}{}{}
Abstract method.  See child classes for implementation.

Returns volume contained within flux surface as function of psi {[}psi,t{]}.
Psi assumed to be evenly-spaced grid on {[}0,1{]}

\end{fulllineitems}

\index{getVolLCFS() (eqtools.core.Equilibrium method)}

\begin{fulllineitems}
\phantomsection\label{eqtools:eqtools.core.Equilibrium.getVolLCFS}\pysiglinewithargsret{\bfcode{getVolLCFS}}{}{}
Abstract method.  See child classes for implementation.

Returns plasma volume within LCFS {[}t{]}

\end{fulllineitems}

\index{getRmidPsi() (eqtools.core.Equilibrium method)}

\begin{fulllineitems}
\phantomsection\label{eqtools:eqtools.core.Equilibrium.getRmidPsi}\pysiglinewithargsret{\bfcode{getRmidPsi}}{}{}
Abstract method.  See child classes for implementation.

Returns outboard-midplane major radius of flux surface {[}t,psi{]}

\end{fulllineitems}

\index{getFluxPres() (eqtools.core.Equilibrium method)}

\begin{fulllineitems}
\phantomsection\label{eqtools:eqtools.core.Equilibrium.getFluxPres}\pysiglinewithargsret{\bfcode{getFluxPres}}{}{}
Abstract method.  See child classes for implementation.

Returns calculated pressure profile {[}psi,t{]}.
Psi assumed to be evenly-spaced grid on {[}0,1{]}

\end{fulllineitems}

\index{getElongation() (eqtools.core.Equilibrium method)}

\begin{fulllineitems}
\phantomsection\label{eqtools:eqtools.core.Equilibrium.getElongation}\pysiglinewithargsret{\bfcode{getElongation}}{}{}
Abstract method.  See child classes for implementation.

Returns LCFS elongation {[}t{]}

\end{fulllineitems}

\index{getUpperTriangularity() (eqtools.core.Equilibrium method)}

\begin{fulllineitems}
\phantomsection\label{eqtools:eqtools.core.Equilibrium.getUpperTriangularity}\pysiglinewithargsret{\bfcode{getUpperTriangularity}}{}{}
Abstract method.  See child classes for implementation.

Returns LCFS upper triangularity {[}t{]}

\end{fulllineitems}

\index{getLowerTriangularity() (eqtools.core.Equilibrium method)}

\begin{fulllineitems}
\phantomsection\label{eqtools:eqtools.core.Equilibrium.getLowerTriangularity}\pysiglinewithargsret{\bfcode{getLowerTriangularity}}{}{}
Abstract method.  See child classes for implementation.

Returns LCFS lower triangularity {[}t{]}

\end{fulllineitems}

\index{getShaping() (eqtools.core.Equilibrium method)}

\begin{fulllineitems}
\phantomsection\label{eqtools:eqtools.core.Equilibrium.getShaping}\pysiglinewithargsret{\bfcode{getShaping}}{}{}
Abstract method.  See child classes for implementation.

Returns dimensionless shaping parameters for plasma.
Namedtuple containing \{LCFS elongation, LCFS upper/lower triangularity\}

\end{fulllineitems}

\index{getMagR() (eqtools.core.Equilibrium method)}

\begin{fulllineitems}
\phantomsection\label{eqtools:eqtools.core.Equilibrium.getMagR}\pysiglinewithargsret{\bfcode{getMagR}}{}{}
Abstract method.  See child classes for implementation.

Returns magnetic-axis major radius {[}t{]}

\end{fulllineitems}

\index{getMagZ() (eqtools.core.Equilibrium method)}

\begin{fulllineitems}
\phantomsection\label{eqtools:eqtools.core.Equilibrium.getMagZ}\pysiglinewithargsret{\bfcode{getMagZ}}{}{}
Abstract method.  See child classes for implementation.

Returns magnetic-axis Z {[}t{]}

\end{fulllineitems}

\index{getAreaLCFS() (eqtools.core.Equilibrium method)}

\begin{fulllineitems}
\phantomsection\label{eqtools:eqtools.core.Equilibrium.getAreaLCFS}\pysiglinewithargsret{\bfcode{getAreaLCFS}}{}{}
Abstract method.  See child classes for implementation.

Returns LCFS surface area {[}t{]}

\end{fulllineitems}

\index{getAOut() (eqtools.core.Equilibrium method)}

\begin{fulllineitems}
\phantomsection\label{eqtools:eqtools.core.Equilibrium.getAOut}\pysiglinewithargsret{\bfcode{getAOut}}{}{}
Abstract method.  See child classes for implementation.

Returns outboard-midplane minor radius {[}t{]}

\end{fulllineitems}

\index{getRmidOut() (eqtools.core.Equilibrium method)}

\begin{fulllineitems}
\phantomsection\label{eqtools:eqtools.core.Equilibrium.getRmidOut}\pysiglinewithargsret{\bfcode{getRmidOut}}{}{}
Abstract method.  See child classes for implementation.

Returns outboard-midplane major radius {[}t{]}

\end{fulllineitems}

\index{getGeometry() (eqtools.core.Equilibrium method)}

\begin{fulllineitems}
\phantomsection\label{eqtools:eqtools.core.Equilibrium.getGeometry}\pysiglinewithargsret{\bfcode{getGeometry}}{}{}
Abstract method.  See child classes for implementation.

Returns dimensional geometry parameters
Namedtuple containing \{mag axis R,Z, LCFS area, volume, outboard-midplane major radius\}

\end{fulllineitems}

\index{getQProfile() (eqtools.core.Equilibrium method)}

\begin{fulllineitems}
\phantomsection\label{eqtools:eqtools.core.Equilibrium.getQProfile}\pysiglinewithargsret{\bfcode{getQProfile}}{}{}
Abstract method.  See child classes for implementation.

Returns safety factor q profile {[}psi,t{]}
Psi assumed to be evenly-spaced grid on {[}0,1{]}

\end{fulllineitems}

\index{getQ0() (eqtools.core.Equilibrium method)}

\begin{fulllineitems}
\phantomsection\label{eqtools:eqtools.core.Equilibrium.getQ0}\pysiglinewithargsret{\bfcode{getQ0}}{}{}
Abstract method.  See child classes for implementation.

Returns q on magnetic axis {[}t{]}

\end{fulllineitems}

\index{getQ95() (eqtools.core.Equilibrium method)}

\begin{fulllineitems}
\phantomsection\label{eqtools:eqtools.core.Equilibrium.getQ95}\pysiglinewithargsret{\bfcode{getQ95}}{}{}
Abstract method.  See child classes for implementation.

Returns q on 95\% flux surface {[}t{]}

\end{fulllineitems}

\index{getQLCFS() (eqtools.core.Equilibrium method)}

\begin{fulllineitems}
\phantomsection\label{eqtools:eqtools.core.Equilibrium.getQLCFS}\pysiglinewithargsret{\bfcode{getQLCFS}}{}{}
Abstract method.  See child classes for implementation.

Returns q on LCFS {[}t{]}

\end{fulllineitems}

\index{getQ1Surf() (eqtools.core.Equilibrium method)}

\begin{fulllineitems}
\phantomsection\label{eqtools:eqtools.core.Equilibrium.getQ1Surf}\pysiglinewithargsret{\bfcode{getQ1Surf}}{}{}
Abstract method.  See child classes for implementation.

Returns outboard-midplane minor radius of q=1 surface {[}t{]}

\end{fulllineitems}

\index{getQ2Surf() (eqtools.core.Equilibrium method)}

\begin{fulllineitems}
\phantomsection\label{eqtools:eqtools.core.Equilibrium.getQ2Surf}\pysiglinewithargsret{\bfcode{getQ2Surf}}{}{}
Abstract method.  See child classes for implementation.

Returns outboard-midplane minor radius of q=2 surface {[}t{]}

\end{fulllineitems}

\index{getQ3Surf() (eqtools.core.Equilibrium method)}

\begin{fulllineitems}
\phantomsection\label{eqtools:eqtools.core.Equilibrium.getQ3Surf}\pysiglinewithargsret{\bfcode{getQ3Surf}}{}{}
Abstract method.  See child classes for implementation.

Returns outboard-midplane minor radius of q=3 surface {[}t{]}

\end{fulllineitems}

\index{getQs() (eqtools.core.Equilibrium method)}

\begin{fulllineitems}
\phantomsection\label{eqtools:eqtools.core.Equilibrium.getQs}\pysiglinewithargsret{\bfcode{getQs}}{}{}
Abstract method.  See child classes for implementation.

Returns specific q-profile values.
Namedtuple containing \{q0, q95, qLCFS, minor radius of q=1,2,3 surfaces\}

\end{fulllineitems}

\index{getBtVac() (eqtools.core.Equilibrium method)}

\begin{fulllineitems}
\phantomsection\label{eqtools:eqtools.core.Equilibrium.getBtVac}\pysiglinewithargsret{\bfcode{getBtVac}}{}{}
Abstract method.  See child classes for implementation.

Returns vacuum on-axis toroidal field {[}t{]}

\end{fulllineitems}

\index{getBtPla() (eqtools.core.Equilibrium method)}

\begin{fulllineitems}
\phantomsection\label{eqtools:eqtools.core.Equilibrium.getBtPla}\pysiglinewithargsret{\bfcode{getBtPla}}{}{}
Abstract method.  See child classes for implementation.

Returns plasma on-axis toroidal field {[}t{]}

\end{fulllineitems}

\index{getBpAvg() (eqtools.core.Equilibrium method)}

\begin{fulllineitems}
\phantomsection\label{eqtools:eqtools.core.Equilibrium.getBpAvg}\pysiglinewithargsret{\bfcode{getBpAvg}}{}{}
Abstract method.  See child classes for implementation.

Returns average poloidal field {[}t{]}

\end{fulllineitems}

\index{getFields() (eqtools.core.Equilibrium method)}

\begin{fulllineitems}
\phantomsection\label{eqtools:eqtools.core.Equilibrium.getFields}\pysiglinewithargsret{\bfcode{getFields}}{}{}
Abstract method.  See child classes for implementation.

Returns magnetic-field values.
Namedtuple containing \{Btor on magnetic axis (plasma and vacuum), avg Bpol\}

\end{fulllineitems}

\index{getIpCalc() (eqtools.core.Equilibrium method)}

\begin{fulllineitems}
\phantomsection\label{eqtools:eqtools.core.Equilibrium.getIpCalc}\pysiglinewithargsret{\bfcode{getIpCalc}}{}{}
Abstract method.  See child classes for implementation.

Returns calculated plasma current {[}t{]}

\end{fulllineitems}

\index{getIpMeas() (eqtools.core.Equilibrium method)}

\begin{fulllineitems}
\phantomsection\label{eqtools:eqtools.core.Equilibrium.getIpMeas}\pysiglinewithargsret{\bfcode{getIpMeas}}{}{}
Abstract method.  See child classes for implementation.

Returns measured plasma current {[}t{]}

\end{fulllineitems}

\index{getJp() (eqtools.core.Equilibrium method)}

\begin{fulllineitems}
\phantomsection\label{eqtools:eqtools.core.Equilibrium.getJp}\pysiglinewithargsret{\bfcode{getJp}}{}{}
Abstract method.  See child classes for implementation.

Returns grid of calculated toroidal current density {[}t,z,r{]}

\end{fulllineitems}

\index{getBetaT() (eqtools.core.Equilibrium method)}

\begin{fulllineitems}
\phantomsection\label{eqtools:eqtools.core.Equilibrium.getBetaT}\pysiglinewithargsret{\bfcode{getBetaT}}{}{}
Abstract method.  See child classes for implementation.

Returns calculated global toroidal beta {[}t{]}

\end{fulllineitems}

\index{getBetaP() (eqtools.core.Equilibrium method)}

\begin{fulllineitems}
\phantomsection\label{eqtools:eqtools.core.Equilibrium.getBetaP}\pysiglinewithargsret{\bfcode{getBetaP}}{}{}
Abstract method.  See child classes for implementation.

Returns calculated global poloidal beta {[}t{]}

\end{fulllineitems}

\index{getLi() (eqtools.core.Equilibrium method)}

\begin{fulllineitems}
\phantomsection\label{eqtools:eqtools.core.Equilibrium.getLi}\pysiglinewithargsret{\bfcode{getLi}}{}{}
Abstract method.  See child classes for implementation.

Returns calculated internal inductance of plasma {[}t{]}

\end{fulllineitems}

\index{getBetas() (eqtools.core.Equilibrium method)}

\begin{fulllineitems}
\phantomsection\label{eqtools:eqtools.core.Equilibrium.getBetas}\pysiglinewithargsret{\bfcode{getBetas}}{}{}
Abstract method.  See child classes for implementation.

Returns calculated betas and inductance.
Namedtuple of \{betat,betap,Li\}

\end{fulllineitems}

\index{getDiamagFlux() (eqtools.core.Equilibrium method)}

\begin{fulllineitems}
\phantomsection\label{eqtools:eqtools.core.Equilibrium.getDiamagFlux}\pysiglinewithargsret{\bfcode{getDiamagFlux}}{}{}
Abstract method.  See child classes for implementation.

Returns diamagnetic flux {[}t{]}

\end{fulllineitems}

\index{getDiamagBetaT() (eqtools.core.Equilibrium method)}

\begin{fulllineitems}
\phantomsection\label{eqtools:eqtools.core.Equilibrium.getDiamagBetaT}\pysiglinewithargsret{\bfcode{getDiamagBetaT}}{}{}
Abstract method.  See child classes for implementation.

Returns diamagnetic-loop toroidal beta {[}t{]}

\end{fulllineitems}

\index{getDiamagBetaP() (eqtools.core.Equilibrium method)}

\begin{fulllineitems}
\phantomsection\label{eqtools:eqtools.core.Equilibrium.getDiamagBetaP}\pysiglinewithargsret{\bfcode{getDiamagBetaP}}{}{}
Abstract method.  See child classes for implementation.

Returns diamagnetic-loop poloidal beta {[}t{]}

\end{fulllineitems}

\index{getDiamagTauE() (eqtools.core.Equilibrium method)}

\begin{fulllineitems}
\phantomsection\label{eqtools:eqtools.core.Equilibrium.getDiamagTauE}\pysiglinewithargsret{\bfcode{getDiamagTauE}}{}{}
Abstract method.  See child classes for implementation.

Returns diamagnetic-loop energy confinement time {[}t{]}

\end{fulllineitems}

\index{getDiamagWp() (eqtools.core.Equilibrium method)}

\begin{fulllineitems}
\phantomsection\label{eqtools:eqtools.core.Equilibrium.getDiamagWp}\pysiglinewithargsret{\bfcode{getDiamagWp}}{}{}
Abstract method.  See child classes for implementation.

Returns diamagnetic-loop plasma stored energy {[}t{]}

\end{fulllineitems}

\index{getDiamag() (eqtools.core.Equilibrium method)}

\begin{fulllineitems}
\phantomsection\label{eqtools:eqtools.core.Equilibrium.getDiamag}\pysiglinewithargsret{\bfcode{getDiamag}}{}{}
Abstract method.  See child classes for implementation.

Returns diamagnetic measurements of plasma parameters.
Namedtuple of \{diamag. flux, betat, betap from coils, tau\_E from diamag., diamag. stored energy\}

\end{fulllineitems}

\index{getWMHD() (eqtools.core.Equilibrium method)}

\begin{fulllineitems}
\phantomsection\label{eqtools:eqtools.core.Equilibrium.getWMHD}\pysiglinewithargsret{\bfcode{getWMHD}}{}{}
Abstract method.  See child classes for implementation.

Returns calculated MHD stored energy {[}t{]}

\end{fulllineitems}

\index{getTauMHD() (eqtools.core.Equilibrium method)}

\begin{fulllineitems}
\phantomsection\label{eqtools:eqtools.core.Equilibrium.getTauMHD}\pysiglinewithargsret{\bfcode{getTauMHD}}{}{}
Abstract method.  See child classes for implementation.

Returns calculated MHD energy confinement time {[}t{]}

\end{fulllineitems}

\index{getPinj() (eqtools.core.Equilibrium method)}

\begin{fulllineitems}
\phantomsection\label{eqtools:eqtools.core.Equilibrium.getPinj}\pysiglinewithargsret{\bfcode{getPinj}}{}{}
Abstract method.  See child classes for implementation.

Returns calculated injected power {[}t{]}

\end{fulllineitems}

\index{getCurrentSign() (eqtools.core.Equilibrium method)}

\begin{fulllineitems}
\phantomsection\label{eqtools:eqtools.core.Equilibrium.getCurrentSign}\pysiglinewithargsret{\bfcode{getCurrentSign}}{}{}
Abstract method.  See child classes for implementation.

Returns calculated current direction, where CCW = +

\end{fulllineitems}

\index{getWbdot() (eqtools.core.Equilibrium method)}

\begin{fulllineitems}
\phantomsection\label{eqtools:eqtools.core.Equilibrium.getWbdot}\pysiglinewithargsret{\bfcode{getWbdot}}{}{}
Abstract method.  See child classes for implementation.

Returns calculated d/dt of magnetic stored energy {[}t{]}

\end{fulllineitems}

\index{getWpdot() (eqtools.core.Equilibrium method)}

\begin{fulllineitems}
\phantomsection\label{eqtools:eqtools.core.Equilibrium.getWpdot}\pysiglinewithargsret{\bfcode{getWpdot}}{}{}
Abstract method.  See child classes for implementation.

Returns calculated d/dt of plasma stored energy {[}t{]}

\end{fulllineitems}

\index{getEnergy() (eqtools.core.Equilibrium method)}

\begin{fulllineitems}
\phantomsection\label{eqtools:eqtools.core.Equilibrium.getEnergy}\pysiglinewithargsret{\bfcode{getEnergy}}{}{}
Abstract method.  See child classes for implementation.

Returns stored-energy parameters.
Namedtuple of \{stored energy, confinement time, injected power, d/dt of magnetic, plasma stored energy\}

\end{fulllineitems}

\index{getParam() (eqtools.core.Equilibrium method)}

\begin{fulllineitems}
\phantomsection\label{eqtools:eqtools.core.Equilibrium.getParam}\pysiglinewithargsret{\bfcode{getParam}}{\emph{path}}{}
Abstract method.  See child classes for implementation.

Backup function: takes parameter name for variable, returns variable directly.
Acts as wrapper to direct data-access routines from within object.

\end{fulllineitems}

\index{getMachineCrossSection() (eqtools.core.Equilibrium method)}

\begin{fulllineitems}
\phantomsection\label{eqtools:eqtools.core.Equilibrium.getMachineCrossSection}\pysiglinewithargsret{\bfcode{getMachineCrossSection}}{}{}
Abstract method.  See child classes for implementation.

Returns (R,Z) coordinates of machine wall cross-section for plotting routines.

\end{fulllineitems}

\index{plotFlux() (eqtools.core.Equilibrium method)}

\begin{fulllineitems}
\phantomsection\label{eqtools:eqtools.core.Equilibrium.plotFlux}\pysiglinewithargsret{\bfcode{plotFlux}}{}{}
Plots flux contours directly from psi grid.

\end{fulllineitems}


\end{fulllineitems}



\subsection{\texttt{EFIT} Module}
\label{eqtools:efit-module}\label{eqtools:module-eqtools.EFIT}\index{eqtools.EFIT (module)}\index{EFITTree (class in eqtools.EFIT)}

\begin{fulllineitems}
\phantomsection\label{eqtools:eqtools.EFIT.EFITTree}\pysiglinewithargsret{\strong{class }\code{eqtools.EFIT.}\bfcode{EFITTree}}{\emph{shot}, \emph{tree}, \emph{root}, \emph{length\_unit='m'}, \emph{tspline=False}, \emph{fast=False}}{}
Bases: {\hyperref[eqtools:eqtools.core.Equilibrium]{\code{eqtools.core.Equilibrium}}}

Inherits Equilibrium class. EFIT-specific data handling class for machines using
standard EFIT tag names/tree structure with MDSplus. Constructor and/or data loading may
need overriding in a machine-specific implementation.
Pulls EFIT data from selected MDS tree and shot, stores as object attributes.
Each EFIT variable or set of variables is recovered with a corresponding getter method.
Essential data for EFIT mapping are pulled on initialization (e.g. psirz grid).
Additional data are pulled at the first request and stored for subsequent usage.

Intializes EFITTree object. Pulls data from MDS tree for storage in
instance attributes. Core attributes are populated from the MDS tree
on initialization. Additional attributes are initialized as None,
filled on the first request to the object.

INPUTS:
shot:   shot index
tree:   MDSplus tree to open to fetch EFIT data.
root:   Root path for EFIT data in MDSplus tree.
\index{\_\_str\_\_() (eqtools.EFIT.EFITTree method)}

\begin{fulllineitems}
\phantomsection\label{eqtools:eqtools.EFIT.EFITTree.__str__}\pysiglinewithargsret{\bfcode{\_\_str\_\_}}{}{}
string formatting for EFITTree class.

\end{fulllineitems}

\index{getInfo() (eqtools.EFIT.EFITTree method)}

\begin{fulllineitems}
\phantomsection\label{eqtools:eqtools.EFIT.EFITTree.getInfo}\pysiglinewithargsret{\bfcode{getInfo}}{}{}
returns namedtuple of shot information
outputs:
namedtuple containing
\begin{quote}

shot:   C-Mod shot index (long)
tree:   EFIT tree (string)
nr:     size of R-axis for spatial grid
nz:     size of Z-axis for spatial grid
nt:     size of timebase for flux grid
\end{quote}

\end{fulllineitems}

\index{getTimeBase() (eqtools.EFIT.EFITTree method)}

\begin{fulllineitems}
\phantomsection\label{eqtools:eqtools.EFIT.EFITTree.getTimeBase}\pysiglinewithargsret{\bfcode{getTimeBase}}{}{}
returns EFIT time base vector

\end{fulllineitems}

\index{getFluxGrid() (eqtools.EFIT.EFITTree method)}

\begin{fulllineitems}
\phantomsection\label{eqtools:eqtools.EFIT.EFITTree.getFluxGrid}\pysiglinewithargsret{\bfcode{getFluxGrid}}{}{}
returns EFIT flux grid, {[}t,z,r{]}

\end{fulllineitems}

\index{getRGrid() (eqtools.EFIT.EFITTree method)}

\begin{fulllineitems}
\phantomsection\label{eqtools:eqtools.EFIT.EFITTree.getRGrid}\pysiglinewithargsret{\bfcode{getRGrid}}{\emph{length\_unit=1}}{}
returns EFIT R-axis {[}r{]}

\end{fulllineitems}

\index{getZGrid() (eqtools.EFIT.EFITTree method)}

\begin{fulllineitems}
\phantomsection\label{eqtools:eqtools.EFIT.EFITTree.getZGrid}\pysiglinewithargsret{\bfcode{getZGrid}}{\emph{length\_unit=1}}{}
returns EFIT Z-axis {[}z{]}

\end{fulllineitems}

\index{getFluxAxis() (eqtools.EFIT.EFITTree method)}

\begin{fulllineitems}
\phantomsection\label{eqtools:eqtools.EFIT.EFITTree.getFluxAxis}\pysiglinewithargsret{\bfcode{getFluxAxis}}{}{}
returns psi on magnetic axis {[}t{]}

\end{fulllineitems}

\index{getFluxLCFS() (eqtools.EFIT.EFITTree method)}

\begin{fulllineitems}
\phantomsection\label{eqtools:eqtools.EFIT.EFITTree.getFluxLCFS}\pysiglinewithargsret{\bfcode{getFluxLCFS}}{}{}
returns psi at separatrix {[}t{]}

\end{fulllineitems}

\index{getFluxVol() (eqtools.EFIT.EFITTree method)}

\begin{fulllineitems}
\phantomsection\label{eqtools:eqtools.EFIT.EFITTree.getFluxVol}\pysiglinewithargsret{\bfcode{getFluxVol}}{\emph{length\_unit=3}}{}
returns volume within flux surface {[}psi,t{]}

\end{fulllineitems}

\index{getVolLCFS() (eqtools.EFIT.EFITTree method)}

\begin{fulllineitems}
\phantomsection\label{eqtools:eqtools.EFIT.EFITTree.getVolLCFS}\pysiglinewithargsret{\bfcode{getVolLCFS}}{\emph{length\_unit=3}}{}
returns volume within LCFS {[}t{]}

\end{fulllineitems}

\index{getRmidPsi() (eqtools.EFIT.EFITTree method)}

\begin{fulllineitems}
\phantomsection\label{eqtools:eqtools.EFIT.EFITTree.getRmidPsi}\pysiglinewithargsret{\bfcode{getRmidPsi}}{\emph{length\_unit=1}}{}
returns maximum major radius of each flux surface {[}t,psi{]}

\end{fulllineitems}

\index{getRLCFS() (eqtools.EFIT.EFITTree method)}

\begin{fulllineitems}
\phantomsection\label{eqtools:eqtools.EFIT.EFITTree.getRLCFS}\pysiglinewithargsret{\bfcode{getRLCFS}}{\emph{length\_unit=1}}{}
returns R-values of LCFS position {[}t,n{]}

\end{fulllineitems}

\index{getZLCFS() (eqtools.EFIT.EFITTree method)}

\begin{fulllineitems}
\phantomsection\label{eqtools:eqtools.EFIT.EFITTree.getZLCFS}\pysiglinewithargsret{\bfcode{getZLCFS}}{\emph{length\_unit=1}}{}
returns Z-values of LCFS position {[}t,n{]}

\end{fulllineitems}

\index{getFluxPres() (eqtools.EFIT.EFITTree method)}

\begin{fulllineitems}
\phantomsection\label{eqtools:eqtools.EFIT.EFITTree.getFluxPres}\pysiglinewithargsret{\bfcode{getFluxPres}}{}{}
returns pressure at flux surface {[}psi,t{]}

\end{fulllineitems}

\index{getElongation() (eqtools.EFIT.EFITTree method)}

\begin{fulllineitems}
\phantomsection\label{eqtools:eqtools.EFIT.EFITTree.getElongation}\pysiglinewithargsret{\bfcode{getElongation}}{}{}
returns LCFS elongation {[}t{]}

\end{fulllineitems}

\index{getUpperTriangularity() (eqtools.EFIT.EFITTree method)}

\begin{fulllineitems}
\phantomsection\label{eqtools:eqtools.EFIT.EFITTree.getUpperTriangularity}\pysiglinewithargsret{\bfcode{getUpperTriangularity}}{}{}
returns LCFS upper triangularity {[}t{]}

\end{fulllineitems}

\index{getLowerTriangularity() (eqtools.EFIT.EFITTree method)}

\begin{fulllineitems}
\phantomsection\label{eqtools:eqtools.EFIT.EFITTree.getLowerTriangularity}\pysiglinewithargsret{\bfcode{getLowerTriangularity}}{}{}
returns LCFS lower triangularity {[}t{]}

\end{fulllineitems}

\index{getShaping() (eqtools.EFIT.EFITTree method)}

\begin{fulllineitems}
\phantomsection\label{eqtools:eqtools.EFIT.EFITTree.getShaping}\pysiglinewithargsret{\bfcode{getShaping}}{}{}
pulls LCFS elongation and upper/lower triangularity
returns namedtuple containing \{kappa, delta\_u, delta\_l\}

\end{fulllineitems}

\index{getMagR() (eqtools.EFIT.EFITTree method)}

\begin{fulllineitems}
\phantomsection\label{eqtools:eqtools.EFIT.EFITTree.getMagR}\pysiglinewithargsret{\bfcode{getMagR}}{\emph{length\_unit=1}}{}
returns magnetic-axis major radius {[}t{]}

\end{fulllineitems}

\index{getMagZ() (eqtools.EFIT.EFITTree method)}

\begin{fulllineitems}
\phantomsection\label{eqtools:eqtools.EFIT.EFITTree.getMagZ}\pysiglinewithargsret{\bfcode{getMagZ}}{\emph{length\_unit=1}}{}
returns magnetic-axis Z {[}t{]}

\end{fulllineitems}

\index{getAreaLCFS() (eqtools.EFIT.EFITTree method)}

\begin{fulllineitems}
\phantomsection\label{eqtools:eqtools.EFIT.EFITTree.getAreaLCFS}\pysiglinewithargsret{\bfcode{getAreaLCFS}}{\emph{length\_unit=2}}{}
returns LCFS cross-sectional area {[}t{]}

\end{fulllineitems}

\index{getAOut() (eqtools.EFIT.EFITTree method)}

\begin{fulllineitems}
\phantomsection\label{eqtools:eqtools.EFIT.EFITTree.getAOut}\pysiglinewithargsret{\bfcode{getAOut}}{\emph{length\_unit=1}}{}
returns outboard-midplane minor radius at LCFS {[}t{]}

\end{fulllineitems}

\index{getRmidOut() (eqtools.EFIT.EFITTree method)}

\begin{fulllineitems}
\phantomsection\label{eqtools:eqtools.EFIT.EFITTree.getRmidOut}\pysiglinewithargsret{\bfcode{getRmidOut}}{\emph{length\_unit=1}}{}
returns outboard-midplane major radius {[}t{]}

\end{fulllineitems}

\index{getGeometry() (eqtools.EFIT.EFITTree method)}

\begin{fulllineitems}
\phantomsection\label{eqtools:eqtools.EFIT.EFITTree.getGeometry}\pysiglinewithargsret{\bfcode{getGeometry}}{\emph{length\_unit=None}}{}
pulls dimensional geometry parameters
returns namedtuple containing \{magnetic-axis R,Z, LCFS area, outboard-midplane LCFS a,R\}

\end{fulllineitems}

\index{getQProfile() (eqtools.EFIT.EFITTree method)}

\begin{fulllineitems}
\phantomsection\label{eqtools:eqtools.EFIT.EFITTree.getQProfile}\pysiglinewithargsret{\bfcode{getQProfile}}{}{}
returns safety factor q {[}psi,t{]}

\end{fulllineitems}

\index{getQ0() (eqtools.EFIT.EFITTree method)}

\begin{fulllineitems}
\phantomsection\label{eqtools:eqtools.EFIT.EFITTree.getQ0}\pysiglinewithargsret{\bfcode{getQ0}}{}{}
returns q on magnetic axis {[}t{]}

\end{fulllineitems}

\index{getQ95() (eqtools.EFIT.EFITTree method)}

\begin{fulllineitems}
\phantomsection\label{eqtools:eqtools.EFIT.EFITTree.getQ95}\pysiglinewithargsret{\bfcode{getQ95}}{}{}
returns q at 95\% flux surface {[}t{]}

\end{fulllineitems}

\index{getQLCFS() (eqtools.EFIT.EFITTree method)}

\begin{fulllineitems}
\phantomsection\label{eqtools:eqtools.EFIT.EFITTree.getQLCFS}\pysiglinewithargsret{\bfcode{getQLCFS}}{}{}
returns q on LCFS {[}t{]}

\end{fulllineitems}

\index{getQ1Surf() (eqtools.EFIT.EFITTree method)}

\begin{fulllineitems}
\phantomsection\label{eqtools:eqtools.EFIT.EFITTree.getQ1Surf}\pysiglinewithargsret{\bfcode{getQ1Surf}}{\emph{length\_unit=1}}{}
returns outboard-midplane minor radius of q=1 surface {[}t{]}

\end{fulllineitems}

\index{getQ2Surf() (eqtools.EFIT.EFITTree method)}

\begin{fulllineitems}
\phantomsection\label{eqtools:eqtools.EFIT.EFITTree.getQ2Surf}\pysiglinewithargsret{\bfcode{getQ2Surf}}{\emph{length\_unit=1}}{}
returns outboard-midplane minor radius of q=2 surface {[}t{]}

\end{fulllineitems}

\index{getQ3Surf() (eqtools.EFIT.EFITTree method)}

\begin{fulllineitems}
\phantomsection\label{eqtools:eqtools.EFIT.EFITTree.getQ3Surf}\pysiglinewithargsret{\bfcode{getQ3Surf}}{\emph{length\_unit=1}}{}
returns outboard-midplane minor radius of q=3 surface {[}t{]}

\end{fulllineitems}

\index{getQs() (eqtools.EFIT.EFITTree method)}

\begin{fulllineitems}
\phantomsection\label{eqtools:eqtools.EFIT.EFITTree.getQs}\pysiglinewithargsret{\bfcode{getQs}}{\emph{length\_unit=1}}{}
pulls q values
returns namedtuple containing \{q0,q95,qLCFS,rq1,rq2,rq3\}

\end{fulllineitems}

\index{getBtVac() (eqtools.EFIT.EFITTree method)}

\begin{fulllineitems}
\phantomsection\label{eqtools:eqtools.EFIT.EFITTree.getBtVac}\pysiglinewithargsret{\bfcode{getBtVac}}{}{}
returns on-axis vacuum toroidal field {[}t{]}

\end{fulllineitems}

\index{getBtPla() (eqtools.EFIT.EFITTree method)}

\begin{fulllineitems}
\phantomsection\label{eqtools:eqtools.EFIT.EFITTree.getBtPla}\pysiglinewithargsret{\bfcode{getBtPla}}{}{}
returns on-axis plasma toroidal field {[}t{]}

\end{fulllineitems}

\index{getBpAvg() (eqtools.EFIT.EFITTree method)}

\begin{fulllineitems}
\phantomsection\label{eqtools:eqtools.EFIT.EFITTree.getBpAvg}\pysiglinewithargsret{\bfcode{getBpAvg}}{}{}
returns average poloidal field {[}t{]}

\end{fulllineitems}

\index{getFields() (eqtools.EFIT.EFITTree method)}

\begin{fulllineitems}
\phantomsection\label{eqtools:eqtools.EFIT.EFITTree.getFields}\pysiglinewithargsret{\bfcode{getFields}}{}{}
pulls vacuum and plasma toroidal field, avg poloidal field
returns namedtuple containing \{btaxv,btaxp,bpolav\}

\end{fulllineitems}

\index{getIpCalc() (eqtools.EFIT.EFITTree method)}

\begin{fulllineitems}
\phantomsection\label{eqtools:eqtools.EFIT.EFITTree.getIpCalc}\pysiglinewithargsret{\bfcode{getIpCalc}}{}{}
returns EFIT-calculated plasma current {[}t{]}

\end{fulllineitems}

\index{getIpMeas() (eqtools.EFIT.EFITTree method)}

\begin{fulllineitems}
\phantomsection\label{eqtools:eqtools.EFIT.EFITTree.getIpMeas}\pysiglinewithargsret{\bfcode{getIpMeas}}{}{}
returns magnetics-measured plasma current {[}t{]}

\end{fulllineitems}

\index{getJp() (eqtools.EFIT.EFITTree method)}

\begin{fulllineitems}
\phantomsection\label{eqtools:eqtools.EFIT.EFITTree.getJp}\pysiglinewithargsret{\bfcode{getJp}}{}{}
returns EFIT-calculated plasma current density Jp on flux grid {[}t,r,z{]}

\end{fulllineitems}

\index{getBetaT() (eqtools.EFIT.EFITTree method)}

\begin{fulllineitems}
\phantomsection\label{eqtools:eqtools.EFIT.EFITTree.getBetaT}\pysiglinewithargsret{\bfcode{getBetaT}}{}{}
returns EFIT-calculated toroidal beta {[}t{]}

\end{fulllineitems}

\index{getBetaP() (eqtools.EFIT.EFITTree method)}

\begin{fulllineitems}
\phantomsection\label{eqtools:eqtools.EFIT.EFITTree.getBetaP}\pysiglinewithargsret{\bfcode{getBetaP}}{}{}
returns EFIT-calculated poloidal beta {[}t{]}

\end{fulllineitems}

\index{getLi() (eqtools.EFIT.EFITTree method)}

\begin{fulllineitems}
\phantomsection\label{eqtools:eqtools.EFIT.EFITTree.getLi}\pysiglinewithargsret{\bfcode{getLi}}{}{}
returns EFIT-calculated internal inductance {[}t{]}

\end{fulllineitems}

\index{getBetas() (eqtools.EFIT.EFITTree method)}

\begin{fulllineitems}
\phantomsection\label{eqtools:eqtools.EFIT.EFITTree.getBetas}\pysiglinewithargsret{\bfcode{getBetas}}{}{}
pulls calculated betap, betat, internal inductance
returns namedtuple containing \{betat,betap,Li\}

\end{fulllineitems}

\index{getDiamagFlux() (eqtools.EFIT.EFITTree method)}

\begin{fulllineitems}
\phantomsection\label{eqtools:eqtools.EFIT.EFITTree.getDiamagFlux}\pysiglinewithargsret{\bfcode{getDiamagFlux}}{}{}
returns measured diamagnetic-loop flux {[}t{]}

\end{fulllineitems}

\index{getDiamagBetaT() (eqtools.EFIT.EFITTree method)}

\begin{fulllineitems}
\phantomsection\label{eqtools:eqtools.EFIT.EFITTree.getDiamagBetaT}\pysiglinewithargsret{\bfcode{getDiamagBetaT}}{}{}
returns diamagnetic-loop toroidal beta {[}t{]}

\end{fulllineitems}

\index{getDiamagBetaP() (eqtools.EFIT.EFITTree method)}

\begin{fulllineitems}
\phantomsection\label{eqtools:eqtools.EFIT.EFITTree.getDiamagBetaP}\pysiglinewithargsret{\bfcode{getDiamagBetaP}}{}{}
returns diamagnetic-loop avg poloidal beta {[}t{]}

\end{fulllineitems}

\index{getDiamagTauE() (eqtools.EFIT.EFITTree method)}

\begin{fulllineitems}
\phantomsection\label{eqtools:eqtools.EFIT.EFITTree.getDiamagTauE}\pysiglinewithargsret{\bfcode{getDiamagTauE}}{}{}
returns diamagnetic-loop energy confinement time {[}t{]}

\end{fulllineitems}

\index{getDiamagWp() (eqtools.EFIT.EFITTree method)}

\begin{fulllineitems}
\phantomsection\label{eqtools:eqtools.EFIT.EFITTree.getDiamagWp}\pysiglinewithargsret{\bfcode{getDiamagWp}}{}{}
returns diamagnetic-loop plasma stored energy {[}t{]}

\end{fulllineitems}

\index{getDiamag() (eqtools.EFIT.EFITTree method)}

\begin{fulllineitems}
\phantomsection\label{eqtools:eqtools.EFIT.EFITTree.getDiamag}\pysiglinewithargsret{\bfcode{getDiamag}}{}{}
pulls diamagnetic flux measurements, toroidal and poloidal beta, energy confinement time and stored energy
returns namedtuple containing \{diamag. flux, betatd, betapd, tauDiamag, WDiamag\}

\end{fulllineitems}

\index{getWMHD() (eqtools.EFIT.EFITTree method)}

\begin{fulllineitems}
\phantomsection\label{eqtools:eqtools.EFIT.EFITTree.getWMHD}\pysiglinewithargsret{\bfcode{getWMHD}}{}{}
returns EFIT-calculated MHD stored energy {[}t{]}

\end{fulllineitems}

\index{getTauMHD() (eqtools.EFIT.EFITTree method)}

\begin{fulllineitems}
\phantomsection\label{eqtools:eqtools.EFIT.EFITTree.getTauMHD}\pysiglinewithargsret{\bfcode{getTauMHD}}{}{}
returns EFIT-calculated MHD energy confinement time {[}t{]}

\end{fulllineitems}

\index{getPinj() (eqtools.EFIT.EFITTree method)}

\begin{fulllineitems}
\phantomsection\label{eqtools:eqtools.EFIT.EFITTree.getPinj}\pysiglinewithargsret{\bfcode{getPinj}}{}{}
returns EFIT-calculated injected power {[}t{]}

\end{fulllineitems}

\index{getWbdot() (eqtools.EFIT.EFITTree method)}

\begin{fulllineitems}
\phantomsection\label{eqtools:eqtools.EFIT.EFITTree.getWbdot}\pysiglinewithargsret{\bfcode{getWbdot}}{}{}
returns EFIT-calculated d/dt of magnetic stored energy {[}t{]}

\end{fulllineitems}

\index{getWpdot() (eqtools.EFIT.EFITTree method)}

\begin{fulllineitems}
\phantomsection\label{eqtools:eqtools.EFIT.EFITTree.getWpdot}\pysiglinewithargsret{\bfcode{getWpdot}}{}{}
returns EFIT-calculated d/dt of plasma stored energy {[}t{]}

\end{fulllineitems}

\index{getEnergy() (eqtools.EFIT.EFITTree method)}

\begin{fulllineitems}
\phantomsection\label{eqtools:eqtools.EFIT.EFITTree.getEnergy}\pysiglinewithargsret{\bfcode{getEnergy}}{}{}
pulls EFIT-calculated energy parameters - stored energy, tau\_E, injected power, d/dt of magnetic and plasma stored energy
returns namedtuple containing \{WMHD,tauMHD,Pinj,Wbdot,Wpdot\}

\end{fulllineitems}

\index{getCurrentSign() (eqtools.EFIT.EFITTree method)}

\begin{fulllineitems}
\phantomsection\label{eqtools:eqtools.EFIT.EFITTree.getCurrentSign}\pysiglinewithargsret{\bfcode{getCurrentSign}}{}{}
Returns the sign of the current, based on the check in Steve Wolfe's
IDL implementation efit\_rz2psi.pro.

\end{fulllineitems}

\index{getParam() (eqtools.EFIT.EFITTree method)}

\begin{fulllineitems}
\phantomsection\label{eqtools:eqtools.EFIT.EFITTree.getParam}\pysiglinewithargsret{\bfcode{getParam}}{\emph{path}}{}
backup function - path to parameter as input, returns desired variable
acts as wrapper for MDS call

\end{fulllineitems}


\end{fulllineitems}



\subsection{\texttt{CModEFIT} Module}
\label{eqtools:module-eqtools.CModEFIT}\label{eqtools:cmodefit-module}\index{eqtools.CModEFIT (module)}
This module provides classes for working with C-Mod EFIT data.
\index{CModEFITTree (class in eqtools.CModEFIT)}

\begin{fulllineitems}
\phantomsection\label{eqtools:eqtools.CModEFIT.CModEFITTree}\pysiglinewithargsret{\strong{class }\code{eqtools.CModEFIT.}\bfcode{CModEFITTree}}{\emph{shot}, \emph{tree='ANALYSIS'}, \emph{length\_unit='m'}, \emph{tspline=False}, \emph{fast=False}}{}
Bases: {\hyperref[eqtools:eqtools.EFIT.EFITTree]{\code{eqtools.EFIT.EFITTree}}}

Inherits \code{EFITTree} class. Machine-specific data
handling class for Alcator C-Mod. Pulls EFIT data from selected MDS tree
and shot, stores as object attributes. Each EFIT variable or set of
variables is recovered with a corresponding getter method. Essential data
for EFIT mapping are pulled on initialization (e.g. psirz grid). Additional
data are pulled at the first request and stored for subsequent usage.

Intializes C-Mod version of EFITTree object.  Pulls data from MDS tree for storage
in instance attributes.  Core attributes are populated from the MDS tree on initialization.
Additional attributes are initialized as None, filled on the first request to the object.
\begin{quote}\begin{description}
\item[{Parameters }] \leavevmode
\textbf{shot} --
(long) int. C-Mod shot index (long)

\item[{Keyword Arguments}] \leavevmode\begin{itemize}
\item {} 
\textbf{tree} --
str. optional input for EFIT tree, defaults to `ANALYSIS'
(i.e., EFIT data are under analysis::top.efit.results).
For any string TREE (such as `EFIT20') other than `ANALYSIS',
data are taken from TREE::top.results.

\item {} 
\textbf{length\_unit} --
str. Sets the base unit used for any quantity whose
dimensions are length to any power. Valid options are:
\begin{quote}

\begin{tabulary}{\linewidth}{|L|L|}
\hline

`m'
 & 
meters
\\\hline

`cm'
 & 
centimeters
\\\hline

`mm'
 & 
millimeters
\\\hline

`in'
 & 
inches
\\\hline

`ft'
 & 
feet
\\\hline

`yd'
 & 
yards
\\\hline

`smoot'
 & 
smoots
\\\hline

`cubit'
 & 
cubits
\\\hline

`hand'
 & 
hands
\\\hline

`default'
 & 
whatever the default in the tree is (no conversion is performed, units may be inconsistent)
\\\hline
\end{tabulary}

\end{quote}

Default is `m' (all units taken and returned in meters).

\item {} 
\textbf{tspline} --
Boolean. Sets whether or not interpolation in time is
performed using a tricubic spline or nearest-neighbor
interpolation. Tricubic spline interpolation requires at least
four complete equilibria at different times. It is also assumed
that they are functionally correlated, and that parameters do
not vary out of their boundaries (derivative = 0 boundary
condition). Default is False (use nearest neighbor interpolation).

\item {} 
\textbf{fast} --
Boolean. Sets whether or not the ``fast'' form of time window
finding is used. If True, the timebase must be monotonically
increasing. Default is False (use slower, safer method).

\end{itemize}

\end{description}\end{quote}
\index{getMachineCrossSection() (eqtools.CModEFIT.CModEFITTree method)}

\begin{fulllineitems}
\phantomsection\label{eqtools:eqtools.CModEFIT.CModEFITTree.getMachineCrossSection}\pysiglinewithargsret{\bfcode{getMachineCrossSection}}{}{}
Pulls C-Mod cross-section data from tree, converts to plottable
vector format for use in other plotting routines
\begin{quote}\begin{description}
\item[{Parameters }] \leavevmode
\textbf{shot} --
(long) int. C-Mod shot index (used for tree access)

\item[{Returns}] \leavevmode
\emph{(x, y)} --
The coordinates of the machine cross-section.

\end{description}\end{quote}

\end{fulllineitems}


\end{fulllineitems}

\index{CModEFITTreeProp (class in eqtools.CModEFIT)}

\begin{fulllineitems}
\phantomsection\label{eqtools:eqtools.CModEFIT.CModEFITTreeProp}\pysiglinewithargsret{\strong{class }\code{eqtools.CModEFIT.}\bfcode{CModEFITTreeProp}}{\emph{shot}, \emph{tree='ANALYSIS'}, \emph{length\_unit='m'}, \emph{tspline=False}, \emph{fast=False}}{}
Bases: {\hyperref[eqtools:eqtools.CModEFIT.CModEFITTree]{\code{eqtools.CModEFIT.CModEFITTree}}}, {\hyperref[eqtools:eqtools.core.PropertyAccessMixin]{\code{eqtools.core.PropertyAccessMixin}}}

CModEFITTree with the PropertyAccessMixin added to enable property-style
access. This is good for interactive use, but may drag the performance down.

Intializes C-Mod version of EFITTree object.  Pulls data from MDS tree for storage
in instance attributes.  Core attributes are populated from the MDS tree on initialization.
Additional attributes are initialized as None, filled on the first request to the object.
\begin{quote}\begin{description}
\item[{Parameters }] \leavevmode
\textbf{shot} --
(long) int. C-Mod shot index (long)

\item[{Keyword Arguments}] \leavevmode\begin{itemize}
\item {} 
\textbf{tree} --
str. optional input for EFIT tree, defaults to `ANALYSIS'
(i.e., EFIT data are under analysis::top.efit.results).
For any string TREE (such as `EFIT20') other than `ANALYSIS',
data are taken from TREE::top.results.

\item {} 
\textbf{length\_unit} --
str. Sets the base unit used for any quantity whose
dimensions are length to any power. Valid options are:
\begin{quote}

\begin{tabulary}{\linewidth}{|L|L|}
\hline

`m'
 & 
meters
\\\hline

`cm'
 & 
centimeters
\\\hline

`mm'
 & 
millimeters
\\\hline

`in'
 & 
inches
\\\hline

`ft'
 & 
feet
\\\hline

`yd'
 & 
yards
\\\hline

`smoot'
 & 
smoots
\\\hline

`cubit'
 & 
cubits
\\\hline

`hand'
 & 
hands
\\\hline

`default'
 & 
whatever the default in the tree is (no conversion is performed, units may be inconsistent)
\\\hline
\end{tabulary}

\end{quote}

Default is `m' (all units taken and returned in meters).

\item {} 
\textbf{tspline} --
Boolean. Sets whether or not interpolation in time is
performed using a tricubic spline or nearest-neighbor
interpolation. Tricubic spline interpolation requires at least
four complete equilibria at different times. It is also assumed
that they are functionally correlated, and that parameters do
not vary out of their boundaries (derivative = 0 boundary
condition). Default is False (use nearest neighbor interpolation).

\item {} 
\textbf{fast} --
Boolean. Sets whether or not the ``fast'' form of time window
finding is used. If True, the timebase must be monotonically
increasing. Default is False (use slower, safer method).

\end{itemize}

\end{description}\end{quote}

\end{fulllineitems}



\subsection{\texttt{eqdskreader} Module}
\label{eqtools:eqdskreader-module}\label{eqtools:module-eqtools.eqdskreader}\index{eqtools.eqdskreader (module)}
This module contains the EqdskReader class, which creates Equilibrium class
functionality for equilibria stored in eqdsk files from EFIT(a- and g-files).
\begin{description}
\item[{Classes:}] \leavevmode\begin{description}
\item[{EqdskReader: class inheriting Equilibrium reading g- and a-files for}] \leavevmode
equilibrium data.

\end{description}

\end{description}
\index{EqdskReader (class in eqtools.eqdskreader)}

\begin{fulllineitems}
\phantomsection\label{eqtools:eqtools.eqdskreader.EqdskReader}\pysiglinewithargsret{\strong{class }\code{eqtools.eqdskreader.}\bfcode{EqdskReader}}{\emph{shot=None}, \emph{time=None}, \emph{gfile=None}, \emph{afile=None}, \emph{length\_unit='m'}, \emph{verbose=True}}{}
Bases: {\hyperref[eqtools:eqtools.core.Equilibrium]{\code{eqtools.core.Equilibrium}}}

Equilibrium subclass working from eqdsk ASCII-file equilibria.

Inherits mapping and structural data from Equilibrium, populates equilibrium
and profile data from g- and a-files for a selected shot and time window.

Create instance of EqdskReader.

Generates object and reads data from selected g-file (either manually set or
autodetected based on user shot and time selection), storing as object
attributes for usage in Equilibrium mapping methods.

Calling structure - user may call class with shot and time (ms) values, set by keywords
(or positional placement allows calling without explicit keyword syntax).  EqdskReader
then attempts to construct filenames from the shot/time, of the form `g{[}shot{]}.{[}time{]}' and
`a{[}shot{]}.{[}time{]}'.  Alternately, the user may skip this input and explicitly set paths to
the g- and/or a-files, using the gfile and afile keyword arguments.  If both types of calls
are set, the explicit g-file and a-file paths override the auto-generated filenames from
the shot and time.
\begin{quote}\begin{description}
\item[{Keyword Arguments}] \leavevmode\begin{itemize}
\item {} 
\textbf{shot} --
Int.  Shot index.

\item {} 
\textbf{time} --
Int.  Time index (typically ms).  Shot and Time used to autogenerate filenames.

\item {} 
\textbf{gfile} --
String.  Manually selects ASCII file for equilibrium read.

\item {} 
\textbf{afile} --
String.  Manually selects ASCII file for time-history read.

\item {} 
\textbf{length\_unit} --
String.  Flag setting length unit for equilibrium scales.
Defaults to `m' for lengths in meters.

\item {} 
\textbf{verbose} --
Boolean.  When set to false, suppresses terminal outputs during CSV read.
Defaults to True (prints terminal output).

\end{itemize}

\item[{Raises }] \leavevmode\begin{itemize}
\item {} 
\code{IOError} -- 
if both name/shot and explicit filenames are not set.

\item {} 
\code{ValueError} -- 
if the g-file cannot be found, or if multiple valid g/a-files are found.

\end{itemize}

\end{description}\end{quote}
\index{getInfo() (eqtools.eqdskreader.EqdskReader method)}

\begin{fulllineitems}
\phantomsection\label{eqtools:eqtools.eqdskreader.EqdskReader.getInfo}\pysiglinewithargsret{\bfcode{getInfo}}{}{}
returns namedtuple of equilibrium information
outputs:
namedtuple containing
\begin{quote}

shot:       shot index
time:       time point of g-file
nr:         size of R-axis of spatial grid
nz:         size of Z-axis of spatial grid
efittype:   EFIT calculation type (magnetic, kinetic, MSE)
\end{quote}

\end{fulllineitems}

\index{readAFile() (eqtools.eqdskreader.EqdskReader method)}

\begin{fulllineitems}
\phantomsection\label{eqtools:eqtools.eqdskreader.EqdskReader.readAFile}\pysiglinewithargsret{\bfcode{readAFile}}{\emph{afile}}{}
Reads a-file (scalar time-history data) to pull additional equilibrium data
not found in g-file, populates remaining data (initialized as None) in object.
\begin{quote}\begin{description}
\item[{Parameters }] \leavevmode
\textbf{afile} --
String.  Path to ASCII a-file.

\item[{Raises }] \leavevmode
\code{IOError} -- 
If afile is not found.

\end{description}\end{quote}

\end{fulllineitems}

\index{rz2psi() (eqtools.eqdskreader.EqdskReader method)}

\begin{fulllineitems}
\phantomsection\label{eqtools:eqtools.eqdskreader.EqdskReader.rz2psi}\pysiglinewithargsret{\bfcode{rz2psi}}{\emph{R}, \emph{Z}, \emph{*args}, \emph{**kwargs}}{}
Converts passed, R,Z arrays to psi values.
Wrapper for Equilibrium.rz2psi masking out timebase dependence.
\begin{quote}\begin{description}
\item[{Parameters }] \leavevmode\begin{itemize}
\item {} 
\textbf{R} --
Array-like or scalar float. Values of the radial coordinate to
map to poloidal flux. If the make\_grid keyword is True, R must
have shape (len\_R,).

\item {} 
\textbf{Z} --
Array-like or scalar float. Values of the vertical coordinate to
map to poloidal flux. Must have the same shape as R unless the
make\_grid keyword is set. If the make\_grid keyword is True, Z
must have shape (len\_Z,).

\item {} 
\textbf{*args} --
slot for time input for consistent syntax with Equilibrium.rz2psi.
will return dummy value for time if input in EqdskReader.

\end{itemize}

\item[{Keyword Arguments}] \leavevmode\begin{itemize}
\item {} 
\textbf{make\_grid} --
Boolean. Set to True to pass R and Z through meshgrid
before evaluating. If this is set to True, R and Z must each
only have a single dimension, but can have different lengths.
Default is False (do not form meshgrid).

\item {} 
\textbf{length\_unit} --
String or 1. Length unit that R and Z are being given
in. If a string is given, it must be a valid unit specifier:
\begin{quote}

`m'         meters
`cm'        centimeters
`mm'        millimeters
`in'        inches
`ft'        feet
`yd'        yards
`smoot'     smoots
`cubit'     cubits
`hand'      hands
`default'   meters
\end{quote}

If length\_unit is 1 or None, meters are assumed. The default
value is 1 (R and Z given in meters).

\item {} 
\textbf{**kwargs} --
other keywords (i.e., return\_t) to rz2psi are valid
(necessary for proper inheritance and usage in other mapping routines)
but will return dummy values.

\end{itemize}

\item[{Returns}] \leavevmode

\emph{psi} --
Array or scalar float. If all of the input arguments are scalar,
\begin{quote}

then a scalar is returned. Otherwise, a scipy Array instance is
returned. If R and Z both have the same shape then psi has this
shape as well. If the make\_grid keyword was True then psi has
shape (len(Z), len(R)).
\end{quote}


\end{description}\end{quote}

\end{fulllineitems}

\index{rz2psinorm() (eqtools.eqdskreader.EqdskReader method)}

\begin{fulllineitems}
\phantomsection\label{eqtools:eqtools.eqdskreader.EqdskReader.rz2psinorm}\pysiglinewithargsret{\bfcode{rz2psinorm}}{\emph{R}, \emph{Z}, \emph{*args}, \emph{**kwargs}}{}
Calculates the normalized poloidal flux at the given (R,Z).
Wrapper for Equilibrium.rz2psinorm masking out timebase dependence.

Uses the definition:
psi\_norm = (psi - psi(0)) / (psi(a) - psi(0))
\begin{quote}\begin{description}
\item[{Parameters }] \leavevmode\begin{itemize}
\item {} 
\textbf{R} --
Array-like or scalar float. Values of the radial coordinate to
map to normalized poloidal flux. If R and Z are both scalar
values, they are used as the coordinate pair for all of the
values in t. Must have the same shape as Z unless the make\_grid
keyword is set. If the make\_grid keyword is True, R must have
shape (len\_R,).

\item {} 
\textbf{Z} --
Array-like or scalar float. Values of the vertical coordinate to
map to normalized poloidal flux. If R and Z are both scalar
values, they are used as the coordinate pair for all of the
values in t. Must have the same shape as R unless the make\_grid
keyword is set. If the make\_grid keyword is True, Z must have
shape (len\_Z,).

\item {} 
\textbf{*args} --
slot for time input for consistent syntax with Equilibrium.rz2psinorm.
will return dummy value for time if input in EqdskReader.

\end{itemize}

\item[{Keyword Arguments}] \leavevmode\begin{itemize}
\item {} 
\textbf{sqrt} --
Boolean. Set to True to return the square root of normalized
flux. Only the square root of positive psi\_norm values is taken.
Negative values are replaced with zeros, consistent with Steve
Wolfe's IDL implementation efit\_rz2rho.pro. Default is False
(return psinorm).

\item {} 
\textbf{make\_grid} --
Boolean. Set to True to pass R and Z through meshgrid
before evaluating. If this is set to True, R and Z must each
only have a single dimension, but can have different lengths.
Default is False (do not form meshgrid).

\item {} 
\textbf{length\_unit} --
String or 1. Length unit that R and Z are being given
in. If a string is given, it must be a valid unit specifier:
\begin{quote}

`m'         meters
`cm'        centimeters
`mm'        millimeters
`in'        inches
`ft'        feet
`yd'        yards
`smoot'     smoots
`cubit'     cubits
`hand'      hands
`default'   meters
\end{quote}

If length\_unit is 1 or None, meters are assumed. The default
value is 1 (R and Z given in meters).

\item {} 
\textbf{**kwargs} --
other keywords passed to Equilibrium.rz2psinorm are valid,
but will return dummy values (i.e. for timebase keywords)

\end{itemize}

\item[{Returns}] \leavevmode

\emph{psinorm} --
Array or scalar float. If all of the input arguments are
\begin{quote}

scalar, then a scalar is returned. Otherwise, a scipy Array
instance is returned. If R and Z both have the same shape then
psinorm has this shape as well. If the make\_grid keyword was
True then psinorm has shape (len(Z), len(R)).
\end{quote}


\end{description}\end{quote}

Examples:
All assume that Eq\_instance is a valid instance EqdskReader:

Find single psinorm value at R=0.6m, Z=0.0m:
psi\_val = Eq\_instance.rz2psinorm(0.6, 0)
\begin{description}
\item[{Find psinorm values at (R, Z) points (0.6m, 0m) and (0.8m, 0m).}] \leavevmode
Note that the Z vector must be fully specified,

\end{description}

even if the values are all the same:
psi\_arr = Eq\_instance.rz2psinorm({[}0.6, 0.8{]}, {[}0, 0{]})

Find psinorm values on grid defined by 1D vector of radial positions R
and 1D vector of vertical positions Z:
psi\_mat = Eq\_instance.rz2psinorm(R, Z, make\_grid=True)

\end{fulllineitems}

\index{rz2phinorm() (eqtools.eqdskreader.EqdskReader method)}

\begin{fulllineitems}
\phantomsection\label{eqtools:eqtools.eqdskreader.EqdskReader.rz2phinorm}\pysiglinewithargsret{\bfcode{rz2phinorm}}{\emph{R}, \emph{Z}, \emph{*args}, \emph{**kwargs}}{}
Calculates normalized toroidal flux at a given (R,Z).
Wrapper for Equilibrium.rz2phinorm masking out timebase dependence.
\begin{quote}\begin{description}
\item[{Parameters }] \leavevmode\begin{itemize}
\item {} 
\textbf{R} --
Array-like or scalar float. Values of the radial coordinate to
map to normalized toroidal flux. Must have the same shape as Z
unless the make\_grid keyword is set. If the make\_grid keyword
is True, R must have shape (len\_R,).

\item {} 
\textbf{Z} --
Array-like or scalar float. Values of the vertical coordinate to
map to normalized toroidal flux. Must have the same shape as R
unless the make\_grid keyword is set. If the make\_grid keyword
is True, Z must have shape (len\_Z,).

\item {} 
\textbf{*args} --
slot for time input for consistent syntax with Equilibrium.rz2phinorm.
will return dummy value for time if input in EqdskReader.

\end{itemize}

\item[{Keyword Arguments}] \leavevmode\begin{itemize}
\item {} 
\textbf{sqrt} --
Boolean. Set to True to return the square root of normalized
flux. Only the square root of positive phi\_norm values is taken.
Negative values are replaced with zeros, consistent with Steve
Wolfe's IDL implementation efit\_rz2rho.pro. Default is False
(return phinorm).

\item {} 
\textbf{make\_grid} --
Boolean. Set to True to pass R and Z through meshgrid
before evaluating. If this is set to True, R and Z must each
only have a single dimension, but can have different lengths.
Default is False (do not form meshgrid).

\item {} 
\textbf{kind} --
String or non-negative int. Specifies the type of interpolation
to be performed in getting from psinorm to phinorm. This is
passed to scipy.interpolate.interp1d. Valid options are:
`linear', `nearest', `zero', `slinear', `quadratic', `cubic'
If this keyword is an integer, it specifies the order of spline
to use. See the documentation for interp1d for more details.
Default value is `cubic' (3rd order spline interpolation). On
some builds of scipy, this can cause problems, in which case
you should try `linear' until you can rebuild your scipy install.

\item {} 
\textbf{length\_unit} --
String or 1. Length unit that R and Z are being given
in. If a string is given, it must be a valid unit specifier:
\begin{quote}

`m'         meters
`cm'        centimeters
`mm'        millimeters
`in'        inches
`ft'        feet
`yd'        yards
`smoot'     smoots
`cubit'     cubits
`hand'      hands
`default'   meters
\end{quote}

If length\_unit is 1 or None, meters are assumed. The default
value is 1 (R and Z given in meters).

\item {} 
\textbf{**kwargs} --
other keywords passed to Equilibrium.rz2phinorm are valid,
but will return dummy values (i.e. for timebase keywords)

\end{itemize}

\item[{Returns}] \leavevmode

\emph{phinorm} --
Array or scalar float. If all of the input arguments are
\begin{quote}

scalar, then a scalar is returned. Otherwise, a scipy Array
instance is returned. If R and Z both have the same shape then
phinorm has this shape as well. If the make\_grid keyword was
True then phinorm has shape (len(Z), len(R)).
\end{quote}


\end{description}\end{quote}

Examples:
All assume that Eq\_instance is a valid instance of EqdskReader.

Find single phinorm value at R=0.6m, Z=0.0m:
phi\_val = Eq\_instance.rz2phinorm(0.6, 0)

Find phinorm values at (R, Z) points (0.6m, 0m) and (0.8m, 0m).
Note that the Z vector must be fully specified,
even if the values are all the same:
phi\_arr = Eq\_instance.rz2phinorm({[}0.6, 0.8{]}, {[}0, 0{]})

Find phinorm values on grid defined by 1D vector of radial positions R
and 1D vector of vertical positions Z:
phi\_mat = Eq\_instance.rz2phinorm(R, Z, make\_grid=True)

\end{fulllineitems}

\index{rz2volnorm() (eqtools.eqdskreader.EqdskReader method)}

\begin{fulllineitems}
\phantomsection\label{eqtools:eqtools.eqdskreader.EqdskReader.rz2volnorm}\pysiglinewithargsret{\bfcode{rz2volnorm}}{\emph{*args}, \emph{**kwargs}}{}
Calculates the normalized flux surface volume.
Not implemented for EqdskReader, as necessary parameter
is not read from a/g-files.
\begin{quote}\begin{description}
\item[{Raises }] \leavevmode
\code{NotImplementedError} -- 
in all cases.

\end{description}\end{quote}

\end{fulllineitems}

\index{rz2rho() (eqtools.eqdskreader.EqdskReader method)}

\begin{fulllineitems}
\phantomsection\label{eqtools:eqtools.eqdskreader.EqdskReader.rz2rho}\pysiglinewithargsret{\bfcode{rz2rho}}{\emph{method}, \emph{R}, \emph{Z}, \emph{sqrt=False}, \emph{make\_grid=False}, \emph{kind='cubic'}, \emph{length\_unit=1}}{}
Convert the passed (R, Z) coordinates into one of several normalized coordinates.
Wrapper for Equilibrium.rz2rho masking timebase dependence.
\begin{quote}\begin{description}
\item[{Parameters }] \leavevmode\begin{itemize}
\item {} 
\textbf{method} --
String. Indicates which normalized coordinates to use.
Valid options are:
\begin{quote}

psinorm     Normalized poloidal flux
phinorm     Normalized toroidal flux
volnorm     Normalized volume
\end{quote}

\item {} 
\textbf{R} --
Array-like or scalar float. Values of the radial coordinate to
map to normalized coordinate. Must have the same shape as Z
unless the make\_grid keyword is set. If the make\_grid keyword
is True, R must have shape (len\_R,).

\item {} 
\textbf{Z} --
Array-like or scalar float. Values of the vertical coordinate to
map to normalized coordinate. Must have the same shape as R
unless the make\_grid keyword is set. If the make\_grid keyword
is True, Z must have shape (len\_Z,).

\end{itemize}

\item[{Keyword Arguments}] \leavevmode\begin{itemize}
\item {} 
\textbf{sqrt} --
Boolean. Set to True to return the square root of normalized
coordinate. Only the square root of positive values is taken.
Negative values are replaced with zeros, consistent with Steve
Wolfe's IDL implementation efit\_rz2rho.pro. Default is False
(return normalized coordinate itself).

\item {} 
\textbf{make\_grid} --
Boolean. Set to True to pass R and Z through meshgrid
before evaluating. If this is set to True, R and Z must each
only have a single dimension, but can have different lengths.
Default is False (do not form meshgrid).

\item {} 
\textbf{kind} (\emph{phinorm and volnorm only}) --
String or non-negative int.
Specifies the type of interpolation to be performed in getting
from psinorm to phinorm or volnorm. This is passed to
scipy.interpolate.interp1d. Valid options are:
`linear', `nearest', `zero', `slinear', `quadratic', `cubic'
If this keyword is an integer, it specifies the order of spline
to use. See the documentation for interp1d for more details.
Default value is `cubic' (3rd order spline interpolation). On
some builds of scipy, this can cause problems, in which case
you should try `linear' until you can rebuild your scipy install.

\item {} 
\textbf{length\_unit} --
String or 1. Length unit that R and Z are being given
in. If a string is given, it must be a valid unit specifier:
\begin{quote}

`m'         meters
`cm'        centimeters
`mm'        millimeters
`in'        inches
`ft'        feet
`yd'        yards
`smoot'     smoots
`cubit'     cubits
`hand'      hands
`default'   meters
\end{quote}

If length\_unit is 1 or None, meters are assumed. The default
value is 1 (R and Z given in meters).

\end{itemize}

\item[{Returns}] \leavevmode

\emph{rho} --
Array or scalar float. If all of the input arguments are
\begin{quote}

scalar, then a scalar is returned. Otherwise, a scipy Array
instance is returned. If R and Z both have the same shape then
rho has this shape as well. If the make\_grid keyword was True
then rho has shape (len(Z), len(R)).
\end{quote}


\item[{Raises }] \leavevmode
\code{ValueError} -- 
If method is not one of the supported values.

\end{description}\end{quote}

Examples:
All assume that Eq\_instance is a valid instance of the appropriate
extension of the Equilibrium abstract class.

Find single psinorm value at R=0.6m, Z=0.0m:
psi\_val = Eq\_instance.rz2rho(`psinorm', 0.6, 0)

Find psinorm values at (R, Z) points (0.6m, 0m) and (0.8m, 0m).
Note that the Z vector must be fully specified,
even if the values are all the same:
psi\_arr = Eq\_instance.rz2rho(`psinorm', {[}0.6, 0.8{]}, {[}0, 0{]})

Find psinorm values on grid defined by 1D vector of radial positions R
and 1D vector of vertical positions Z:
psi\_mat = Eq\_instance.rz2rho(`psinorm', R, Z, make\_grid=True)

\end{fulllineitems}

\index{rz2rmid() (eqtools.eqdskreader.EqdskReader method)}

\begin{fulllineitems}
\phantomsection\label{eqtools:eqtools.eqdskreader.EqdskReader.rz2rmid}\pysiglinewithargsret{\bfcode{rz2rmid}}{\emph{R}, \emph{Z}, \emph{sqrt=False}, \emph{make\_grid=False}, \emph{rho=False}, \emph{kind='cubic'}, \emph{length\_unit=1}}{}
Maps the given points to the outboard midplane major radius, R\_mid.
Wrapper for Equilibrium.rz2rmid masking timebase dependence.

Based on the IDL version efit\_rz2rmid.pro by Steve Wolfe.
\begin{quote}\begin{description}
\item[{Parameters }] \leavevmode\begin{itemize}
\item {} 
\textbf{R} --
Array-like or scalar float. Values of the radial coordinate to
map to midplane radius. Must have the same shape as Z unless
the make\_grid keyword is set. If the make\_grid keyword is True,
R must have shape (len\_R,).

\item {} 
\textbf{Z} --
Array-like or scalar float. Values of the vertical coordinate to
map to midplane radius. Must have the same shape as R unless the
make\_grid keyword is set. If the make\_grid keyword is True, Z
must have shape (len\_Z,).

\end{itemize}

\item[{Keyword Arguments}] \leavevmode\begin{itemize}
\item {} 
\textbf{sqrt} --
Boolean. Set to True to return the square root of midplane
radius. Only the square root of positive values is taken.
Negative values are replaced with zeros, consistent with Steve
Wolfe's IDL implementation efit\_rz2rho.pro. Default is False
(return R\_mid itself).

\item {} 
\textbf{make\_grid} --
Boolean. Set to True to pass R and Z through meshgrid
before evaluating. If this is set to True, R and Z must each
only have a single dimension, but can have different lengths.
Default is False (do not form meshgrid).

\item {} 
\textbf{rho} --
Boolean. Set to True to return r/a (normalized minor radius)
instead of R\_mid. Default is False (return major radius, R\_mid).

\item {} 
\textbf{kind} --
String or non-negative int. Specifies the type of interpolation
to be performed in getting from psinorm to R\_mid. This is
passed to scipy.interpolate.interp1d. Valid options are:
`linear', `nearest', `zero', `slinear', `quadratic', `cubic'
If this keyword is an integer, it specifies the order of spline
to use. See the documentation for interp1d for more details.
Default value is `cubic' (3rd order spline interpolation). On
some builds of scipy, this can cause problems, in which case
you should try `linear' until you can rebuild your scipy install.

\item {} 
\textbf{length\_unit} --
String or 1. Length unit that R and Z are being given
in AND that R\_mid is returned in. If a string is given, it
must be a valid unit specifier:
\begin{quote}

`m'         meters
`cm'        centimeters
`mm'        millimeters
`in'        inches
`ft'        feet
`yd'        yards
`smoot'     smoots
`cubit'     cubits
`hand'      hands
`default'   meters
\end{quote}

If length\_unit is 1 or None, meters are assumed. The default
value is 1 (R and Z given in meters, R\_mid returned in meters).

\end{itemize}

\item[{Returns}] \leavevmode

\emph{R\_mid} --
Array or scalar float. If all of the input arguments are
\begin{quote}

scalar, then a scalar is returned. Otherwise, a scipy Array
instance is returned. If R and Z both have the same shape then
R\_mid has this shape as well. If the make\_grid keyword was True
then R\_mid has shape (len(Z), len(R)).
\end{quote}


\end{description}\end{quote}

Examples:
All assume that Eq\_instance is a valid instance of the appropriate
extension of the Equilibrium abstract class.

Find single R\_mid value at R=0.6m, Z=0.0m:
R\_mid\_val = Eq\_instance.rz2rmid(0.6, 0)

Find R\_mid values at (R, Z) points (0.6m, 0m) and (0.8m, 0m).
Note that the Z vector must be fully specified,
even if the values are all the same:
R\_mid\_arr = Eq\_instance.rz2rmid({[}0.6, 0.8{]}, {[}0, 0{]})

Find R\_mid values on grid defined by 1D vector of radial positions R
and 1D vector of vertical positions Z:
R\_mid\_mat = Eq\_instance.rz2rmid(R, Z, make\_grid=True)

\end{fulllineitems}

\index{psinorm2rmid() (eqtools.eqdskreader.EqdskReader method)}

\begin{fulllineitems}
\phantomsection\label{eqtools:eqtools.eqdskreader.EqdskReader.psinorm2rmid}\pysiglinewithargsret{\bfcode{psinorm2rmid}}{\emph{psi\_norm}, \emph{rho=False}, \emph{kind='cubic'}, \emph{length\_unit=1}}{}
Calculates the outboard R\_mid location corresponding to the passed psi\_norm (normalized poloidal flux) values.
\begin{quote}\begin{description}
\item[{Parameters }] \leavevmode
\textbf{psi\_norm} --
Array-like or scalar float. Values of the normalized
poloidal flux to map to midplane radius.

\item[{Keyword Arguments}] \leavevmode\begin{itemize}
\item {} 
\textbf{rho} --
Boolean. Set to True to return r/a (normalized minor radius)
instead of R\_mid. Default is False (return major radius, R\_mid).

\item {} 
\textbf{kind} --
String or non-negative int. Specifies the type of interpolation
to be performed in getting from psinorm to R\_mid. This is
passed to scipy.interpolate.interp1d. Valid options are:
`linear', `nearest', `zero', `slinear', `quadratic', `cubic'
If this keyword is an integer, it specifies the order of spline
to use. See the documentation for interp1d for more details.
Default value is `cubic' (3rd order spline interpolation). On
some builds of scipy, this can cause problems, in which case
you should try `linear' until you can rebuild your scipy install.

\item {} 
\textbf{length\_unit} --
String or 1. Length unit that R\_mid is returned in. If
a string is given, it must be a valid unit specifier:
\begin{quote}

`m'         meters
`cm'        centimeters
`mm'        millimeters
`in'        inches
`ft'        feet
`yd'        yards
`smoot'     smoots
`cubit'     cubits
`hand'      hands
`default'   meters
\end{quote}

If length\_unit is 1 or None, meters are assumed. The default
value is 1 (R\_mid returned in meters).

\end{itemize}

\item[{Returns}] \leavevmode

\emph{R\_mid} --
Array or scalar float. If all of the input arguments are
\begin{quote}

scalar, then a scalar is returned. Otherwise, a scipy Array
instance is returned.
\end{quote}


\end{description}\end{quote}

Examples:
All assume that Eq\_instance is a valid instance of the appropriate
extension of the Equilibrium abstract class.

Find single R\_mid value for psinorm=0.7:
R\_mid\_val = Eq\_instance.psinorm2rmid(0.7)

Find R\_mid values at psi\_norm values of 0.5 and 0.7.
Note that the Z vector must be fully specified, even if the
values are all the same:
R\_mid\_arr = Eq\_instance.psinorm2rmid({[}0.5, 0.7{]})

\end{fulllineitems}

\index{psinorm2volnorm() (eqtools.eqdskreader.EqdskReader method)}

\begin{fulllineitems}
\phantomsection\label{eqtools:eqtools.eqdskreader.EqdskReader.psinorm2volnorm}\pysiglinewithargsret{\bfcode{psinorm2volnorm}}{\emph{*args}, \emph{**kwargs}}{}
Calculates the outboard R\_mid location corresponding to psi\_norm (normalized poloidal flux) values.
Not implemented for EqdskReader, as necessary parameter
is not read from a/g-files.

\end{fulllineitems}

\index{psinorm2phinorm() (eqtools.eqdskreader.EqdskReader method)}

\begin{fulllineitems}
\phantomsection\label{eqtools:eqtools.eqdskreader.EqdskReader.psinorm2phinorm}\pysiglinewithargsret{\bfcode{psinorm2phinorm}}{\emph{psi\_norm}, \emph{kind='cubic'}}{}
Calculates the normalized toroidal flux corresponding to the passed psi\_norm (normalized poloidal flux) values.
\begin{quote}\begin{description}
\item[{Parameters }] \leavevmode
\textbf{psi\_norm} --
Array-like or scalar float. Values of the normalized
poloidal flux to map to normalized toroidal flux.

\item[{Keyword Arguments}] \leavevmode
\textbf{kind} --
String or non-negative int. Specifies the type of interpolation
to be performed in getting from psinorm to phinorm. This is
`linear', `nearest', `zero', `slinear', `quadratic', `cubic'
passed to scipy.interpolate.interp1d. Valid options are:
If this keyword is an integer, it specifies the order of spline
to use. See the documentation for interp1d for more details.
Default value is `cubic' (3rd order spline interpolation). On
some builds of scipy, this can cause problems, in which case
you should try `linear' until you can rebuild your scipy install.

\item[{Returns}] \leavevmode

\emph{phinorm} --
Array or scalar float. If all of the input arguments are
\begin{quote}

scalar, then a scalar is returned. Otherwise, a scipy Array
instance is returned.
\end{quote}


\end{description}\end{quote}

Examples:
All assume that Eq\_instance is a valid instance of the appropriate
extension of the Equilibrium abstract class.

Find single phinorm value for psinorm=0.7:
phinorm\_val = Eq\_instance.psinorm2phinorm(0.7)

Find phinorm values at psi\_norm values of 0.5 and 0.7.
Note that the Z vector must be fully specified, even if the
values are all the same:
phinorm\_arr = Eq\_instance.psinorm2phinorm({[}0.5, 0.7{]})

\end{fulllineitems}

\index{getTimeBase() (eqtools.eqdskreader.EqdskReader method)}

\begin{fulllineitems}
\phantomsection\label{eqtools:eqtools.eqdskreader.EqdskReader.getTimeBase}\pysiglinewithargsret{\bfcode{getTimeBase}}{}{}
Returns EFIT time point

\end{fulllineitems}

\index{getCurrentSign() (eqtools.eqdskreader.EqdskReader method)}

\begin{fulllineitems}
\phantomsection\label{eqtools:eqtools.eqdskreader.EqdskReader.getCurrentSign}\pysiglinewithargsret{\bfcode{getCurrentSign}}{}{}
Returns the sign of the current, based on the check in Steve Wolfe's
IDL implementation efit\_rz2psi.pro.

\end{fulllineitems}

\index{getFluxGrid() (eqtools.eqdskreader.EqdskReader method)}

\begin{fulllineitems}
\phantomsection\label{eqtools:eqtools.eqdskreader.EqdskReader.getFluxGrid}\pysiglinewithargsret{\bfcode{getFluxGrid}}{}{}
Returns EFIT flux grid, {[}r,z{]}

\end{fulllineitems}

\index{getRGrid() (eqtools.eqdskreader.EqdskReader method)}

\begin{fulllineitems}
\phantomsection\label{eqtools:eqtools.eqdskreader.EqdskReader.getRGrid}\pysiglinewithargsret{\bfcode{getRGrid}}{\emph{length\_unit=1}}{}
Returns EFIT R-axis {[}r{]}

\end{fulllineitems}

\index{getZGrid() (eqtools.eqdskreader.EqdskReader method)}

\begin{fulllineitems}
\phantomsection\label{eqtools:eqtools.eqdskreader.EqdskReader.getZGrid}\pysiglinewithargsret{\bfcode{getZGrid}}{\emph{length\_unit=1}}{}
Returns EFIT Z-axis {[}z{]}

\end{fulllineitems}

\index{getFluxAxis() (eqtools.eqdskreader.EqdskReader method)}

\begin{fulllineitems}
\phantomsection\label{eqtools:eqtools.eqdskreader.EqdskReader.getFluxAxis}\pysiglinewithargsret{\bfcode{getFluxAxis}}{}{}
Returns psi on magnetic axis

\end{fulllineitems}

\index{getFluxLCFS() (eqtools.eqdskreader.EqdskReader method)}

\begin{fulllineitems}
\phantomsection\label{eqtools:eqtools.eqdskreader.EqdskReader.getFluxLCFS}\pysiglinewithargsret{\bfcode{getFluxLCFS}}{}{}
Returns psi at separatrix

\end{fulllineitems}

\index{getRLCFS() (eqtools.eqdskreader.EqdskReader method)}

\begin{fulllineitems}
\phantomsection\label{eqtools:eqtools.eqdskreader.EqdskReader.getRLCFS}\pysiglinewithargsret{\bfcode{getRLCFS}}{\emph{length\_unit=1}}{}
Returns array of R-values of LCFS

\end{fulllineitems}

\index{getZLCFS() (eqtools.eqdskreader.EqdskReader method)}

\begin{fulllineitems}
\phantomsection\label{eqtools:eqtools.eqdskreader.EqdskReader.getZLCFS}\pysiglinewithargsret{\bfcode{getZLCFS}}{\emph{length\_unit=1}}{}
Returns array of Z-values of LCFS

\end{fulllineitems}

\index{getFluxVol() (eqtools.eqdskreader.EqdskReader method)}

\begin{fulllineitems}
\phantomsection\label{eqtools:eqtools.eqdskreader.EqdskReader.getFluxVol}\pysiglinewithargsret{\bfcode{getFluxVol}}{}{}
\end{fulllineitems}

\index{getVolLCFS() (eqtools.eqdskreader.EqdskReader method)}

\begin{fulllineitems}
\phantomsection\label{eqtools:eqtools.eqdskreader.EqdskReader.getVolLCFS}\pysiglinewithargsret{\bfcode{getVolLCFS}}{\emph{length\_unit=3}}{}
Returns volume with LCFS.
\begin{quote}\begin{description}
\item[{Raises }] \leavevmode
\code{ValueError} -- 
if a-file data is not read.

\end{description}\end{quote}

\end{fulllineitems}

\index{getRmidPsi() (eqtools.eqdskreader.EqdskReader method)}

\begin{fulllineitems}
\phantomsection\label{eqtools:eqtools.eqdskreader.EqdskReader.getRmidPsi}\pysiglinewithargsret{\bfcode{getRmidPsi}}{}{}
Returns outboard-midplane major radius of flux surfaces.
Data not read from a/g-files, not implemented for EqdskReader.
\begin{quote}\begin{description}
\item[{Raises }] \leavevmode
\code{NotImplementedError} -- 
RmidPsi not read from a/g-files.

\end{description}\end{quote}

\end{fulllineitems}

\index{getFluxPres() (eqtools.eqdskreader.EqdskReader method)}

\begin{fulllineitems}
\phantomsection\label{eqtools:eqtools.eqdskreader.EqdskReader.getFluxPres}\pysiglinewithargsret{\bfcode{getFluxPres}}{}{}
Returns pressure on flux surface p(psi)

\end{fulllineitems}

\index{getElongation() (eqtools.eqdskreader.EqdskReader method)}

\begin{fulllineitems}
\phantomsection\label{eqtools:eqtools.eqdskreader.EqdskReader.getElongation}\pysiglinewithargsret{\bfcode{getElongation}}{}{}
Returns elongation of LCFS.
\begin{quote}\begin{description}
\item[{Raises }] \leavevmode
\code{ValueError} -- 
if a-file data is not read.

\end{description}\end{quote}

\end{fulllineitems}

\index{getUpperTriangularity() (eqtools.eqdskreader.EqdskReader method)}

\begin{fulllineitems}
\phantomsection\label{eqtools:eqtools.eqdskreader.EqdskReader.getUpperTriangularity}\pysiglinewithargsret{\bfcode{getUpperTriangularity}}{}{}
Returns upper triangularity of LCFS.
\begin{quote}\begin{description}
\item[{Raises }] \leavevmode
\code{ValueError} -- 
if a-file data is not read.

\end{description}\end{quote}

\end{fulllineitems}

\index{getLowerTriangularity() (eqtools.eqdskreader.EqdskReader method)}

\begin{fulllineitems}
\phantomsection\label{eqtools:eqtools.eqdskreader.EqdskReader.getLowerTriangularity}\pysiglinewithargsret{\bfcode{getLowerTriangularity}}{}{}
Returns lower triangularity of LCFS.
\begin{quote}\begin{description}
\item[{Raises }] \leavevmode
\code{ValueError} -- 
if a-file data is not read.

\end{description}\end{quote}

\end{fulllineitems}

\index{getShaping() (eqtools.eqdskreader.EqdskReader method)}

\begin{fulllineitems}
\phantomsection\label{eqtools:eqtools.eqdskreader.EqdskReader.getShaping}\pysiglinewithargsret{\bfcode{getShaping}}{}{}
Pulls LCFS elongation, upper/lower triangularity.
Returns namedtuple containing {[}kappa,delta\_u,delta\_l{]}.
\begin{quote}\begin{description}
\item[{Raises }] \leavevmode
\code{ValueError} -- 
if a-file data is not read.

\end{description}\end{quote}

\end{fulllineitems}

\index{getMagR() (eqtools.eqdskreader.EqdskReader method)}

\begin{fulllineitems}
\phantomsection\label{eqtools:eqtools.eqdskreader.EqdskReader.getMagR}\pysiglinewithargsret{\bfcode{getMagR}}{\emph{length\_unit=1}}{}
Returns major radius of magnetic axis.
\begin{quote}\begin{description}
\item[{Raises }] \leavevmode
\code{ValueError} -- 
if a-file data is not read.

\end{description}\end{quote}

\end{fulllineitems}

\index{getMagZ() (eqtools.eqdskreader.EqdskReader method)}

\begin{fulllineitems}
\phantomsection\label{eqtools:eqtools.eqdskreader.EqdskReader.getMagZ}\pysiglinewithargsret{\bfcode{getMagZ}}{\emph{length\_unit=1}}{}
Returns Z of magnetic axis.
\begin{quote}\begin{description}
\item[{Raises }] \leavevmode
\code{ValueError} -- 
if a-file data is not read.

\end{description}\end{quote}

\end{fulllineitems}

\index{getAreaLCFS() (eqtools.eqdskreader.EqdskReader method)}

\begin{fulllineitems}
\phantomsection\label{eqtools:eqtools.eqdskreader.EqdskReader.getAreaLCFS}\pysiglinewithargsret{\bfcode{getAreaLCFS}}{\emph{length\_unit=2}}{}
Returns surface area of LCFS.
\begin{quote}\begin{description}
\item[{Raises }] \leavevmode
\code{ValueError} -- 
if a-file data is not read.

\end{description}\end{quote}

\end{fulllineitems}

\index{getAOut() (eqtools.eqdskreader.EqdskReader method)}

\begin{fulllineitems}
\phantomsection\label{eqtools:eqtools.eqdskreader.EqdskReader.getAOut}\pysiglinewithargsret{\bfcode{getAOut}}{\emph{length\_unit=1}}{}
Returns outboard-midplane minor radius of LCFS.
\begin{quote}\begin{description}
\item[{Raises }] \leavevmode
\code{ValueError} -- 
if a-file data is not read.

\end{description}\end{quote}

\end{fulllineitems}

\index{getRmidOut() (eqtools.eqdskreader.EqdskReader method)}

\begin{fulllineitems}
\phantomsection\label{eqtools:eqtools.eqdskreader.EqdskReader.getRmidOut}\pysiglinewithargsret{\bfcode{getRmidOut}}{\emph{length\_unit=1}}{}
Returns outboard-midplane major radius of LCFS.
\begin{quote}\begin{description}
\item[{Raises }] \leavevmode
\code{ValueError} -- 
if a-file data is not read.

\end{description}\end{quote}

\end{fulllineitems}

\index{getGeometry() (eqtools.eqdskreader.EqdskReader method)}

\begin{fulllineitems}
\phantomsection\label{eqtools:eqtools.eqdskreader.EqdskReader.getGeometry}\pysiglinewithargsret{\bfcode{getGeometry}}{\emph{length\_unit=None}}{}
Pulls dimensional geometry parameters.
Returns namedtuple containing {[}Rmag,Zmag,AreaLCFS,aOut,RmidOut{]}
\begin{quote}\begin{description}
\item[{Keyword Arguments}] \leavevmode
\textbf{length\_unit} --
TODO

\item[{Raises }] \leavevmode
\code{ValueError} -- 
if a-file data is not read.

\end{description}\end{quote}

\end{fulllineitems}

\index{getQProfile() (eqtools.eqdskreader.EqdskReader method)}

\begin{fulllineitems}
\phantomsection\label{eqtools:eqtools.eqdskreader.EqdskReader.getQProfile}\pysiglinewithargsret{\bfcode{getQProfile}}{}{}
Returns safety factor q(psi).

\end{fulllineitems}

\index{getQ0() (eqtools.eqdskreader.EqdskReader method)}

\begin{fulllineitems}
\phantomsection\label{eqtools:eqtools.eqdskreader.EqdskReader.getQ0}\pysiglinewithargsret{\bfcode{getQ0}}{}{}
Returns safety factor q on-axis, q0.
\begin{quote}\begin{description}
\item[{Raises }] \leavevmode
\code{ValueError} -- 
if a-file data is not read.

\end{description}\end{quote}

\end{fulllineitems}

\index{getQ95() (eqtools.eqdskreader.EqdskReader method)}

\begin{fulllineitems}
\phantomsection\label{eqtools:eqtools.eqdskreader.EqdskReader.getQ95}\pysiglinewithargsret{\bfcode{getQ95}}{}{}
Returns safety factor q at 95\% flux surface.
\begin{quote}\begin{description}
\item[{Raises }] \leavevmode
\code{ValueError} -- 
if a-file data is not read.

\end{description}\end{quote}

\end{fulllineitems}

\index{getQLCFS() (eqtools.eqdskreader.EqdskReader method)}

\begin{fulllineitems}
\phantomsection\label{eqtools:eqtools.eqdskreader.EqdskReader.getQLCFS}\pysiglinewithargsret{\bfcode{getQLCFS}}{}{}
Returns safety factor q at LCFS (interpolated).
\begin{quote}\begin{description}
\item[{Raises }] \leavevmode
\code{ValueError} -- 
if a-file data is not loaded.

\end{description}\end{quote}

\end{fulllineitems}

\index{getQ1Surf() (eqtools.eqdskreader.EqdskReader method)}

\begin{fulllineitems}
\phantomsection\label{eqtools:eqtools.eqdskreader.EqdskReader.getQ1Surf}\pysiglinewithargsret{\bfcode{getQ1Surf}}{\emph{length\_unit=1}}{}
Returns outboard-midplane minor radius of q=1 surface.
\begin{quote}\begin{description}
\item[{Raises }] \leavevmode
\code{ValueError} -- 
if a-file data is not read.

\end{description}\end{quote}

\end{fulllineitems}

\index{getQ2Surf() (eqtools.eqdskreader.EqdskReader method)}

\begin{fulllineitems}
\phantomsection\label{eqtools:eqtools.eqdskreader.EqdskReader.getQ2Surf}\pysiglinewithargsret{\bfcode{getQ2Surf}}{\emph{length\_unit=1}}{}
Returns outboard-midplane minor radius of q=2 surface.
\begin{quote}\begin{description}
\item[{Raises }] \leavevmode
\code{ValueError} -- 
if a-file data is not read.

\end{description}\end{quote}

\end{fulllineitems}

\index{getQ3Surf() (eqtools.eqdskreader.EqdskReader method)}

\begin{fulllineitems}
\phantomsection\label{eqtools:eqtools.eqdskreader.EqdskReader.getQ3Surf}\pysiglinewithargsret{\bfcode{getQ3Surf}}{\emph{length\_unit=1}}{}
Returns outboard-midplane minor radius of q=3 surface.
\begin{quote}\begin{description}
\item[{Raises }] \leavevmode
\code{ValueError} -- 
if a-file data is not read.

\end{description}\end{quote}

\end{fulllineitems}

\index{getQs() (eqtools.eqdskreader.EqdskReader method)}

\begin{fulllineitems}
\phantomsection\label{eqtools:eqtools.eqdskreader.EqdskReader.getQs}\pysiglinewithargsret{\bfcode{getQs}}{\emph{length\_unit=1}}{}
Pulls q-profile data.
Returns namedtuple containing {[}q0,q95,qLCFS,rq1,rq2,rq3{]}
\begin{quote}\begin{description}
\item[{Raises }] \leavevmode
\code{ValueError} -- 
if a-file data is not read.

\end{description}\end{quote}

\end{fulllineitems}

\index{getBtVac() (eqtools.eqdskreader.EqdskReader method)}

\begin{fulllineitems}
\phantomsection\label{eqtools:eqtools.eqdskreader.EqdskReader.getBtVac}\pysiglinewithargsret{\bfcode{getBtVac}}{}{}
Returns vacuum toroidal field on-axis.
\begin{quote}\begin{description}
\item[{Raises }] \leavevmode
\code{ValueError} -- 
if a-file data is not read.

\end{description}\end{quote}

\end{fulllineitems}

\index{getBtPla() (eqtools.eqdskreader.EqdskReader method)}

\begin{fulllineitems}
\phantomsection\label{eqtools:eqtools.eqdskreader.EqdskReader.getBtPla}\pysiglinewithargsret{\bfcode{getBtPla}}{}{}
Returns plasma toroidal field on-axis.
\begin{quote}\begin{description}
\item[{Raises }] \leavevmode
\code{ValueError} -- 
if a-file data is not read.

\end{description}\end{quote}

\end{fulllineitems}

\index{getBpAvg() (eqtools.eqdskreader.EqdskReader method)}

\begin{fulllineitems}
\phantomsection\label{eqtools:eqtools.eqdskreader.EqdskReader.getBpAvg}\pysiglinewithargsret{\bfcode{getBpAvg}}{}{}
Returns average poloidal field.
\begin{quote}\begin{description}
\item[{Raises }] \leavevmode
\code{ValueError} -- 
if a-file data is not read.

\end{description}\end{quote}

\end{fulllineitems}

\index{getFields() (eqtools.eqdskreader.EqdskReader method)}

\begin{fulllineitems}
\phantomsection\label{eqtools:eqtools.eqdskreader.EqdskReader.getFields}\pysiglinewithargsret{\bfcode{getFields}}{}{}
Pulls vacuum and plasma toroidal field, poloidal field data.
Returns namedtuple containing {[}BtVac,BtPla,BpAvg{]}
\begin{quote}\begin{description}
\item[{Raises }] \leavevmode
\code{ValueError} -- 
if a-file data is not read.

\end{description}\end{quote}

\end{fulllineitems}

\index{getIpCalc() (eqtools.eqdskreader.EqdskReader method)}

\begin{fulllineitems}
\phantomsection\label{eqtools:eqtools.eqdskreader.EqdskReader.getIpCalc}\pysiglinewithargsret{\bfcode{getIpCalc}}{}{}
Returns EFIT-calculated plasma current.

\end{fulllineitems}

\index{getIpMeas() (eqtools.eqdskreader.EqdskReader method)}

\begin{fulllineitems}
\phantomsection\label{eqtools:eqtools.eqdskreader.EqdskReader.getIpMeas}\pysiglinewithargsret{\bfcode{getIpMeas}}{}{}
Returns measured plasma current.
\begin{quote}\begin{description}
\item[{Raises }] \leavevmode
\code{ValueError} -- 
if a-file data is not read.

\end{description}\end{quote}

\end{fulllineitems}

\index{getJp() (eqtools.eqdskreader.EqdskReader method)}

\begin{fulllineitems}
\phantomsection\label{eqtools:eqtools.eqdskreader.EqdskReader.getJp}\pysiglinewithargsret{\bfcode{getJp}}{}{}
Returns (r,z) grid of toroidal plasma current density.
Data not read from g-file, not implemented for EqdskReader.
\begin{quote}\begin{description}
\item[{Raises }] \leavevmode
\code{NotImplementedError} -- 
Jp not read from g-file.

\end{description}\end{quote}

\end{fulllineitems}

\index{getBetaT() (eqtools.eqdskreader.EqdskReader method)}

\begin{fulllineitems}
\phantomsection\label{eqtools:eqtools.eqdskreader.EqdskReader.getBetaT}\pysiglinewithargsret{\bfcode{getBetaT}}{}{}
Returns EFIT-calculated toroidal beta.
\begin{quote}\begin{description}
\item[{Raises }] \leavevmode
\code{ValueError} -- 
if a-file data is not read.

\end{description}\end{quote}

\end{fulllineitems}

\index{getBetaP() (eqtools.eqdskreader.EqdskReader method)}

\begin{fulllineitems}
\phantomsection\label{eqtools:eqtools.eqdskreader.EqdskReader.getBetaP}\pysiglinewithargsret{\bfcode{getBetaP}}{}{}
Returns EFIT-calculated poloidal beta.
\begin{quote}\begin{description}
\item[{Raises }] \leavevmode
\code{ValueError} -- 
if a-file data is not read

\end{description}\end{quote}

\end{fulllineitems}

\index{getLi() (eqtools.eqdskreader.EqdskReader method)}

\begin{fulllineitems}
\phantomsection\label{eqtools:eqtools.eqdskreader.EqdskReader.getLi}\pysiglinewithargsret{\bfcode{getLi}}{}{}
Returns internal inductance of plasma.
\begin{quote}\begin{description}
\item[{Raises }] \leavevmode
\code{ValueError} -- 
if a-file data is not read.

\end{description}\end{quote}

\end{fulllineitems}

\index{getBetas() (eqtools.eqdskreader.EqdskReader method)}

\begin{fulllineitems}
\phantomsection\label{eqtools:eqtools.eqdskreader.EqdskReader.getBetas}\pysiglinewithargsret{\bfcode{getBetas}}{}{}
Pulls EFIT-calculated betas and internal inductance.
Returns a namedtuple containing {[}betat,betap,Li{]}
\begin{quote}\begin{description}
\item[{Raises }] \leavevmode
\code{ValueError} -- 
if a-file data is not read.

\end{description}\end{quote}

\end{fulllineitems}

\index{getDiamagFlux() (eqtools.eqdskreader.EqdskReader method)}

\begin{fulllineitems}
\phantomsection\label{eqtools:eqtools.eqdskreader.EqdskReader.getDiamagFlux}\pysiglinewithargsret{\bfcode{getDiamagFlux}}{}{}
Returns diamagnetic flux.
\begin{quote}\begin{description}
\item[{Raises }] \leavevmode
\code{ValueError} -- 
if a-file data is not read.

\end{description}\end{quote}

\end{fulllineitems}

\index{getDiamagBetaT() (eqtools.eqdskreader.EqdskReader method)}

\begin{fulllineitems}
\phantomsection\label{eqtools:eqtools.eqdskreader.EqdskReader.getDiamagBetaT}\pysiglinewithargsret{\bfcode{getDiamagBetaT}}{}{}
Returns diamagnetic-loop measured toroidal beta.
\begin{quote}\begin{description}
\item[{Raises }] \leavevmode
\code{ValueError} -- 
if a-file data is not read.

\end{description}\end{quote}

\end{fulllineitems}

\index{getDiamagBetaP() (eqtools.eqdskreader.EqdskReader method)}

\begin{fulllineitems}
\phantomsection\label{eqtools:eqtools.eqdskreader.EqdskReader.getDiamagBetaP}\pysiglinewithargsret{\bfcode{getDiamagBetaP}}{}{}
Returns diamagnetic-loop measured poloidal beta.
\begin{quote}\begin{description}
\item[{Raises }] \leavevmode
\code{ValueError} -- 
if a-file data is not read.

\end{description}\end{quote}

\end{fulllineitems}

\index{getDiamagTauE() (eqtools.eqdskreader.EqdskReader method)}

\begin{fulllineitems}
\phantomsection\label{eqtools:eqtools.eqdskreader.EqdskReader.getDiamagTauE}\pysiglinewithargsret{\bfcode{getDiamagTauE}}{}{}
Returns diamagnetic-loop energy confinement time.
\begin{quote}\begin{description}
\item[{Raises }] \leavevmode
\code{ValueError} -- 
if a-file data is not read.

\end{description}\end{quote}

\end{fulllineitems}

\index{getDiamagWp() (eqtools.eqdskreader.EqdskReader method)}

\begin{fulllineitems}
\phantomsection\label{eqtools:eqtools.eqdskreader.EqdskReader.getDiamagWp}\pysiglinewithargsret{\bfcode{getDiamagWp}}{}{}
Returns diamagnetic-loop measured stored energy.
\begin{quote}\begin{description}
\item[{Raises }] \leavevmode
\code{ValueError} -- 
if a-file data is not read.

\end{description}\end{quote}

\end{fulllineitems}

\index{getDiamag() (eqtools.eqdskreader.EqdskReader method)}

\begin{fulllineitems}
\phantomsection\label{eqtools:eqtools.eqdskreader.EqdskReader.getDiamag}\pysiglinewithargsret{\bfcode{getDiamag}}{}{}
Pulls diamagnetic flux, diamag. measured toroidal and poloidal beta, stored energy, and energy confinement time.
Returns a namedtuple containing {[}diaFlux,diaBetat,diaBetap,diaTauE,diaWp{]}
\begin{quote}\begin{description}
\item[{Raises }] \leavevmode
\code{ValueError} -- 
if a-file data is not read

\end{description}\end{quote}

\end{fulllineitems}

\index{getWMHD() (eqtools.eqdskreader.EqdskReader method)}

\begin{fulllineitems}
\phantomsection\label{eqtools:eqtools.eqdskreader.EqdskReader.getWMHD}\pysiglinewithargsret{\bfcode{getWMHD}}{}{}
Returns EFIT-calculated stored energy.
\begin{quote}\begin{description}
\item[{Raises }] \leavevmode
\code{ValueError} -- 
if a-file data is not read.

\end{description}\end{quote}

\end{fulllineitems}

\index{getTauMHD() (eqtools.eqdskreader.EqdskReader method)}

\begin{fulllineitems}
\phantomsection\label{eqtools:eqtools.eqdskreader.EqdskReader.getTauMHD}\pysiglinewithargsret{\bfcode{getTauMHD}}{}{}
Returns EFIT-calculated energy confinement time.
\begin{quote}\begin{description}
\item[{Raises }] \leavevmode
\code{ValueError} -- 
if a-file data is not read.

\end{description}\end{quote}

\end{fulllineitems}

\index{getPinj() (eqtools.eqdskreader.EqdskReader method)}

\begin{fulllineitems}
\phantomsection\label{eqtools:eqtools.eqdskreader.EqdskReader.getPinj}\pysiglinewithargsret{\bfcode{getPinj}}{}{}
Returns EFIT injected power.
\begin{quote}\begin{description}
\item[{Raises }] \leavevmode
\code{ValueError} -- 
if a-file data is not read.

\end{description}\end{quote}

\end{fulllineitems}

\index{getWbdot() (eqtools.eqdskreader.EqdskReader method)}

\begin{fulllineitems}
\phantomsection\label{eqtools:eqtools.eqdskreader.EqdskReader.getWbdot}\pysiglinewithargsret{\bfcode{getWbdot}}{}{}
Returns EFIT d/dt of magnetic stored energy
\begin{quote}\begin{description}
\item[{Raises }] \leavevmode
\code{ValueError} -- 
if a-file data is not read.

\end{description}\end{quote}

\end{fulllineitems}

\index{getWpdot() (eqtools.eqdskreader.EqdskReader method)}

\begin{fulllineitems}
\phantomsection\label{eqtools:eqtools.eqdskreader.EqdskReader.getWpdot}\pysiglinewithargsret{\bfcode{getWpdot}}{}{}
Returns EFIT d/dt of plasma stored energy.
\begin{quote}\begin{description}
\item[{Raises }] \leavevmode
\code{ValueError} -- 
if a-file data is not read.

\end{description}\end{quote}

\end{fulllineitems}

\index{getEnergy() (eqtools.eqdskreader.EqdskReader method)}

\begin{fulllineitems}
\phantomsection\label{eqtools:eqtools.eqdskreader.EqdskReader.getEnergy}\pysiglinewithargsret{\bfcode{getEnergy}}{}{}
Pulls EFIT stored energy, energy confinement time, injected power, and d/dt of magnetic and plasma stored energy.
Returns namedtuple containing {[}WMHD,tauMHD,Pinj,Wbdot,Wpdot{]}
\begin{quote}\begin{description}
\item[{Raises }] \leavevmode
\code{ValueError} -- 
if a-file data is not read.

\end{description}\end{quote}

\end{fulllineitems}

\index{getParam() (eqtools.eqdskreader.EqdskReader method)}

\begin{fulllineitems}
\phantomsection\label{eqtools:eqtools.eqdskreader.EqdskReader.getParam}\pysiglinewithargsret{\bfcode{getParam}}{\emph{name}}{}
Backup function, applying a direct path input for tree-like data storage access
for parameters not typically found in Equilbrium object.  Directly calls attributes
read from g/a-files in copy-safe manner.
\begin{quote}\begin{description}
\item[{Parameters }] \leavevmode
\textbf{name} --
String.  Parameter name for value stored in EqdskReader instance.

\item[{Raises }] \leavevmode
\code{AttributeError} -- 
raised if no attribute is found.

\end{description}\end{quote}

\end{fulllineitems}

\index{getMachineCrossSection() (eqtools.eqdskreader.EqdskReader method)}

\begin{fulllineitems}
\phantomsection\label{eqtools:eqtools.eqdskreader.EqdskReader.getMachineCrossSection}\pysiglinewithargsret{\bfcode{getMachineCrossSection}}{}{}
Method to pull machine cross-section from data storage, convert to standard format for plotting routine.
Not implemented for eqdsk class.

\end{fulllineitems}

\index{plotFlux() (eqtools.eqdskreader.EqdskReader method)}

\begin{fulllineitems}
\phantomsection\label{eqtools:eqtools.eqdskreader.EqdskReader.plotFlux}\pysiglinewithargsret{\bfcode{plotFlux}}{}{}
streamlined plotting of flux contours directly from psi grid

\end{fulllineitems}


\end{fulllineitems}



\subsection{\texttt{afilereader} Module}
\label{eqtools:afilereader-module}\label{eqtools:module-eqtools.afilereader}\index{eqtools.afilereader (module)}
This module contains the AFileReader class, a lightweight data
handler for a-file (time-history) datasets.
\begin{description}
\item[{Classes:}] \leavevmode\begin{description}
\item[{AFileReader: Data-storage class for a-file data.  Reads}] \leavevmode
data from ASCII a-file, storing as copy-safe object
attributes.

\end{description}

\end{description}
\index{AFileReader (class in eqtools.afilereader)}

\begin{fulllineitems}
\phantomsection\label{eqtools:eqtools.afilereader.AFileReader}\pysiglinewithargsret{\strong{class }\code{eqtools.afilereader.}\bfcode{AFileReader}}{\emph{afile}}{}
Bases: \code{object}

Class to read ASCII a-file (time-history data storage) into lightweight, user-friendly data structure.

A-files store data blocks of scalar time-history data for EFIT plasma equilibrium.  Each parameter is
read into a pseudo-private object attribute (marked by a leading underscore), followed by the standard
EFIT variable names.

initialize object, reading from file.

INPUTS:
afile: (str) path to a-file
\index{\_\_str\_\_() (eqtools.afilereader.AFileReader method)}

\begin{fulllineitems}
\phantomsection\label{eqtools:eqtools.afilereader.AFileReader.__str__}\pysiglinewithargsret{\bfcode{\_\_str\_\_}}{}{}
overrides default \_\_str\_\_method with more useful output.

\end{fulllineitems}

\index{\_\_getattribute\_\_() (eqtools.afilereader.AFileReader method)}

\begin{fulllineitems}
\phantomsection\label{eqtools:eqtools.afilereader.AFileReader.__getattribute__}\pysiglinewithargsret{\bfcode{\_\_getattribute\_\_}}{\emph{name}}{}
Copy-safe attribute retrieval method overriding default object.\_\_getattribute\_\_.

Tries to retrieve attribute as-written (first check for default object attributes).
If that fails, looks for pseudo-private attributes, marked by preceding underscore,
to retrieve data values.  If this fails, raise AttributeError.
\begin{quote}\begin{description}
\item[{Parameters }] \leavevmode
\textbf{name} --
String.  Name (without leading underscore for data variables) of attribute.

\item[{Raises }] \leavevmode
\code{AttributeError} -- 
if no attribute can be found.

\end{description}\end{quote}

\end{fulllineitems}

\index{\_\_setattr\_\_() (eqtools.afilereader.AFileReader method)}

\begin{fulllineitems}
\phantomsection\label{eqtools:eqtools.afilereader.AFileReader.__setattr__}\pysiglinewithargsret{\bfcode{\_\_setattr\_\_}}{\emph{name}, \emph{value}}{}
Copy-safe attribute setting method overriding default object.\_\_setattr\_\_.

Raises error if object already has attribute \_\{name\} for input name,
as such an attribute would interfere with automatic property generation in
\_\_getattribute\_\_.
\begin{quote}\begin{description}
\item[{Parameters }] \leavevmode
\textbf{name} --
String.  Attribute name.

\item[{Raises }] \leavevmode
\code{AttributeError} -- 
if attempting to create attribute with protected
pseudo-private name.

\end{description}\end{quote}

\end{fulllineitems}


\end{fulllineitems}



\subsection{\texttt{pfilereader} Module}
\label{eqtools:pfilereader-module}\label{eqtools:module-eqtools.pfilereader}\index{eqtools.pfilereader (module)}
This module contains the PFileReader class, a lightweight data
handler for p-file (radial profile) datasets.
\begin{description}
\item[{Classes:}] \leavevmode\begin{description}
\item[{PFileReader: Data-storage class for p-file data.  Reads}] \leavevmode
data from ASCII p-file, storing as copy-safe object
attributes.

\end{description}

\end{description}
\index{PFileReader (class in eqtools.pfilereader)}

\begin{fulllineitems}
\phantomsection\label{eqtools:eqtools.pfilereader.PFileReader}\pysiglinewithargsret{\strong{class }\code{eqtools.pfilereader.}\bfcode{PFileReader}}{\emph{pfile}, \emph{verbose=True}}{}
Bases: \code{object}

Class to read ASCII p-file (profile data storage) into lightweight, user-friendly data structure.

P-files store data blocks containing the following: a header with parameter
name, parameter units, x-axis units, and number of data points, followed by values of
axis x, parameter y, and derivative dy/dx.  Each parameter block is read into a
namedtuple storing {[}'name','npts','units','xunits','x','y','dydx'{]}, with each namedtuple
stored as an attribute of the PFileReader instance.  This gracefully handles variable
formats of p-files (differing versions of p-files will have different parameters stored).
Data blocks are accessed as attributes in a copy-safe manner.

Creates instance of PFileReader.
\begin{quote}\begin{description}
\item[{Parameters }] \leavevmode
\textbf{pfile} --
String.  Path to ASCII p-file to be loaded.

\item[{Keyword Arguments}] \leavevmode
\textbf{verbose} --
Boolean.  Option to print message on object creation
listing available data parameters.  Defaults to True.

\end{description}\end{quote}
\index{\_\_str\_\_() (eqtools.pfilereader.PFileReader method)}

\begin{fulllineitems}
\phantomsection\label{eqtools:eqtools.pfilereader.PFileReader.__str__}\pysiglinewithargsret{\bfcode{\_\_str\_\_}}{}{}
overrides default string method for useful output.

\end{fulllineitems}

\index{\_\_getattribute\_\_() (eqtools.pfilereader.PFileReader method)}

\begin{fulllineitems}
\phantomsection\label{eqtools:eqtools.pfilereader.PFileReader.__getattribute__}\pysiglinewithargsret{\bfcode{\_\_getattribute\_\_}}{\emph{name}}{}
Copy-safe attribute retrieval method overriding default object.\_\_getattribute\_\_.

Tries to retrieve attribute as-written (first check for default object attributes).
If that fails, looks for pseudo-private attributes, marked by preceding underscore,
to retrieve data blocks.  If this fails, raise AttributeError.
\begin{quote}\begin{description}
\item[{Parameters }] \leavevmode
\textbf{name} --
String.  Name (without leading underscore for data variables) of attribute.

\item[{Raises }] \leavevmode
\code{AttributeError} -- 
if no attribute can be found.

\end{description}\end{quote}

\end{fulllineitems}

\index{\_\_setattr\_\_() (eqtools.pfilereader.PFileReader method)}

\begin{fulllineitems}
\phantomsection\label{eqtools:eqtools.pfilereader.PFileReader.__setattr__}\pysiglinewithargsret{\bfcode{\_\_setattr\_\_}}{\emph{name}, \emph{value}}{}
Copy-safe attribute setting method overriding default object.\_\_setattr\_\_.

Raises error if object already has attribute \_\{name\} for input name,
as such an attribute would interfere with automatic property generation in
\_\_getattribute\_\_.
\begin{quote}\begin{description}
\item[{Parameters }] \leavevmode
\textbf{name} --
String.  Attribute name.

\item[{Raises }] \leavevmode
\code{AttributeError} -- 
if attempting to create attribute with protected
pseudo-private name.

\end{description}\end{quote}

\end{fulllineitems}


\end{fulllineitems}



\subsection{\texttt{FromArrays} Module}
\label{eqtools:module-eqtools.FromArrays}\label{eqtools:fromarrays-module}\index{eqtools.FromArrays (module)}\index{ArrayEquilibrium (class in eqtools.FromArrays)}

\begin{fulllineitems}
\phantomsection\label{eqtools:eqtools.FromArrays.ArrayEquilibrium}\pysiglinewithargsret{\strong{class }\code{eqtools.FromArrays.}\bfcode{ArrayEquilibrium}}{\emph{psiRZ}, \emph{rGrid}, \emph{zGrid}, \emph{time}, \emph{q}, \emph{fluxVol}, \emph{length\_unit='m'}, \emph{tspline=False}, \emph{fast=False}}{}
Bases: {\hyperref[eqtools:eqtools.core.Equilibrium]{\code{eqtools.core.Equilibrium}}}

Class to represent an equilibrium specified as arrays of data.

Create ArrayEquilibrium instance from arrays of data.
\begin{quote}\begin{description}
\item[{Parameters }] \leavevmode\begin{itemize}
\item {} 
\textbf{psiRZ} --
Array-like, (M, N, P). Flux values at M times, N Z locations
and P R locations.

\item {} 
\textbf{rGrid} --
Array-like, (P,). R coordinates that psiRZ is given at.

\item {} 
\textbf{zGrid} --
Array-like, (N,). Z coordinates that psiRZ is given at.

\item {} 
\textbf{time} --
Array-like, (M,). Times that psiRZ is given at.

\item {} 
\textbf{q} --
Array-like, (Q, M). q profile evaluated at Q values of psinorm
from 0 to 1, given at M times.

\item {} 
\textbf{fluxVol} --
Array-like, (S, M). Flux surface volumes evaluated at S
values of psinorm from 0 to 1, given at M times.

\end{itemize}

\item[{Keyword Arguments}] \leavevmode
\textbf{length\_unit} --
String. Base unit for any quantity whose dimensions
are length to any power. Default is `m'. Valid options are:
\begin{quote}

`m'         meters
`cm'        centimeters
`mm'        millimeters
`in'        inches
`ft'        feet
`yd'        yards
`smoot'     smoots
`default'   whatever the default in the tree is (no
\begin{quote}

conversion is performed, units may be
inconsistent)
\end{quote}
\end{quote}

\item[{Returns}] \leavevmode
The ArrayEquilibrium instance created.

\end{description}\end{quote}
\index{getTimeBase() (eqtools.FromArrays.ArrayEquilibrium method)}

\begin{fulllineitems}
\phantomsection\label{eqtools:eqtools.FromArrays.ArrayEquilibrium.getTimeBase}\pysiglinewithargsret{\bfcode{getTimeBase}}{}{}
Returns a copy of the time base vector, array dimensions are (M,).

\end{fulllineitems}

\index{getFluxGrid() (eqtools.FromArrays.ArrayEquilibrium method)}

\begin{fulllineitems}
\phantomsection\label{eqtools:eqtools.FromArrays.ArrayEquilibrium.getFluxGrid}\pysiglinewithargsret{\bfcode{getFluxGrid}}{}{}
Returns a copy of the flux array, dimensions are (M, N, P), corresponding to (time, Z, R).

\end{fulllineitems}

\index{getRGrid() (eqtools.FromArrays.ArrayEquilibrium method)}

\begin{fulllineitems}
\phantomsection\label{eqtools:eqtools.FromArrays.ArrayEquilibrium.getRGrid}\pysiglinewithargsret{\bfcode{getRGrid}}{\emph{length\_unit=1}}{}
Returns a copy of the radial grid, dimensions are (P,).

\end{fulllineitems}

\index{getZGrid() (eqtools.FromArrays.ArrayEquilibrium method)}

\begin{fulllineitems}
\phantomsection\label{eqtools:eqtools.FromArrays.ArrayEquilibrium.getZGrid}\pysiglinewithargsret{\bfcode{getZGrid}}{\emph{length\_unit=1}}{}
Returns a copy of the vertical grid, dimensions are (N,).

\end{fulllineitems}


\end{fulllineitems}



\subsection{\texttt{trispline} Module}
\label{eqtools:trispline-module}

\chapter{Indices and tables}
\label{index:indices-and-tables}\begin{itemize}
\item {} 
\emph{genindex}

\item {} 
\emph{modindex}

\item {} 
\emph{search}

\end{itemize}


\renewcommand{\indexname}{Python Module Index}
\begin{theindex}
\def\bigletter#1{{\Large\sffamily#1}\nopagebreak\vspace{1mm}}
\bigletter{e}
\item {\texttt{eqtools.\_\_init\_\_}}, \pageref{eqtools:module-eqtools.__init__}
\item {\texttt{eqtools.afilereader}}, \pageref{eqtools:module-eqtools.afilereader}
\item {\texttt{eqtools.CModEFIT}}, \pageref{eqtools:module-eqtools.CModEFIT}
\item {\texttt{eqtools.core}}, \pageref{eqtools:module-eqtools.core}
\item {\texttt{eqtools.EFIT}}, \pageref{eqtools:module-eqtools.EFIT}
\item {\texttt{eqtools.eqdskreader}}, \pageref{eqtools:module-eqtools.eqdskreader}
\item {\texttt{eqtools.FromArrays}}, \pageref{eqtools:module-eqtools.FromArrays}
\item {\texttt{eqtools.pfilereader}}, \pageref{eqtools:module-eqtools.pfilereader}
\end{theindex}

\renewcommand{\indexname}{Index}
\printindex
\end{document}
